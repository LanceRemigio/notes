\section{Square Roots of a Positive Operator}

\begin{theorem}[Positive Square Root]
    Let \( T: H \to H  \) be a positive bounded self-adjoint linear operator on a complex Hilbert space \( H  \). Then a bounded self-adjoint linear operator \( A  \) is called a \textbf{square root} of \( T  \) if  
    \[  A^{2} = T.  \]
    If, in addition, \( A \geq 0  \), then \( A  \) is called a \textbf{positive square root} of \( T  \) and is denoted by
    \[  A = T^{1/2}. \]
    That is, \( T^{1/2}  \) exists and unique.
\end{theorem}

\begin{theorem}[Positive Square Root]\label{9.4-2}
Every positive bounded self-adjoint linear operator \( T: H \to H  \) on a complex Hilbert Space \( H  \) has a positive square root \( A  \), which is unique. This operator \( A  \) commutes with every bounded linear operator on \( H  \) which commutes with \( T  \).    
\end{theorem}
