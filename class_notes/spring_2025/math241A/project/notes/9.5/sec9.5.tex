\section{Projection Operators}

\begin{definition}[Projections]
    The map \( P : H \to Y  \) defined by \( x \mapsto y  = Px  \) is called the projection of \( H  \) onto \(  Y \).
\end{definition}

\begin{prop}
    A linear operator \( P : H \to H  \) is a projection on \( H  \) is there is a closed subspace \( Y  \) of \( H  \) such that \( Y  \) is the range of \( P  \) and \( Y^{\perp} \) is the null space of \( P \) and \( P |_{Y} \) is the identity operator on \( Y  \).
\end{prop}
The above implies that we can write any \( x \in H  \) in the following way
\[  x = y + z  \tag{1} \]
where \( y \in Y  \) and \( z \in Y^{\perp} \). Furthermore, we can write \( H  \) in terms of \( Y  \) and \( Y^{\perp} \) as a direct sum; that is,
\[  H = Y \oplus Y^{\perp}. \]
We can also rewrite (1) in the following way
\[  x = y + z = Px + (I-P)x. \]
Another way to characterize the projection on \( H  \) is the presented in the following theorem below:

\begin{theorem}[Projection]\label{9.5-1}
    A bounded linear operator \( P: H \to H  \) on a Hilbert Space \( H  \) is a projection if and only if \( P  \) is self-adjoint and idempotent (That is, \( P^{2} = P  \)).
\end{theorem}

\begin{proof}
    (\( \Longrightarrow \)) Suppose that \( {P}_{1} {P}_{2} = {P}_{2} {P}_{1} \). Then by Theorem {\hyperref[3.10-4]{3.10-4}}, \( P \) is self-adjoint. Also, we see that \( P \) is idempotent because
    \[  P^{2} = ({P}_{1} {P}_{2})({P}_{1} {P}_{2}) = {P}_{1}^{2} {P}_{2}^{2} = {P}_{1} {P}_{2} = P ] \]
    where \( {P}_{1}  \) and \( {P}_{2}  \) are projections on \( H  \).
    By {\hyperref[9.5-1]{9.5-1}} , \( P  \) is a projection, and for every \( x \in H  \), we have 
    \[  Px = {P}_{1}({P}_{2}x) = {P}_{2}({P}_{1}x). \]
    Since \( {P}_{1}  \) is a projects \( H  \) onto \( {Y}_{1} \), we must have \( {P}_{1}({P}_{2}x) \in {Y}_{1} \). Similarly, we see that \( {P}_{2}({P}_{1}x \in {Y}_{2}) \). Hence, we see that \( Px \in {Y}_{1} \cap {Y}_{2} \). Note that this projection is onto since   
    \[  Py = {P}_{1} {P}_{2} y = {P}_{1} y = y.  \]

    (\( \Longleftarrow \)) If \( P = {P}_{1} {P}_{2}  \) is a projection defined on \( H  \), then \( P \) is self-adjoint by {\hyperref[9.5-1]{9.5-1}}. Also, we see that \( {P}_{1} \) and \( {P}_{2} \) commute by Theorem {\hyperref[3.10-4]{3.10-4}}.
\end{proof}

\begin{theorem}[Positivity, norm]\label{9.5-2}
   For any projection \( P  \) on a Hilbert space \( H  \), 
   \begin{enumerate}
       \item[(i)] \( \langle Px , x  \rangle = \|Px\|^{2} \)
        \item[(ii)] \( P \geq 0  \) 
        \item[(iii)] \( \|P\| \leq   1  \) and \( \|P\| = 1  \) if \( P(H) \neq \{ 0 \}  \).
   \end{enumerate}
\end{theorem}
\begin{proof}

\end{proof}

\begin{theorem}[Sum of Projections]
    Let \( {P}_{1} \) and \( {P}_{2} \) be projections on a Hilbert space \( H  \). Then:
    \begin{enumerate}
    \item[(a)] The sum \( P = {P}_{1} + {P}_{2} \) is a projection on \( H  \) if and only if \( {Y}_{1} = {P}_{1}(H) \) and \( {Y}_{2} = {P}_{2}(H) \) are orthogonal.
    \item[(b)] If \( P = {P}_{1} + {P}_{2}  \) is a projection, \( P  \) projects \( H \) onto \( Y = {Y}_{1} \oplus {Y}_{2} \). 
    \end{enumerate}
\end{theorem}
\begin{enumerate}
    \item[(a)] \( (\Longrightarrow) \) Suppose that \( P =  {P}_{1} + {P}_{2}  \) is a projection on \( H  \). Our goal is to show that \( {Y}_{1} \) and \( {Y}_{2}  \) are orthogonal; that is, it suffices to show that \( {P}_{1} {P}_{2} = 0  \) by Theorem {\hyperref[9.5-3]{9.5-3}}(b). Observe that \( P  \) is idempotent. Hence, we have  
        \begin{align*}
            {P}_{1} + {P}_{2} &= ({P}_{1} + {P}_{2})^{2} \\
                              &= {P}_{1}^{2} + {P}_{1} {P}_{2} + {P}_{2} {P}_{1} + {P}_{2}^{2} \\
                              &= {P}_{1} + {P}_{1} {P}_{2} + {P}_{2} {P}_{1}  +{P}_{2}.
        \end{align*}
        Hence, we are left with 
        \[  {P}_{1}{P}_{2} + {P}_{2} {P}_{1} = 0.  \]
        Multiplying the quantity above by \( {P}_{2}  \) on the left, we get 
        \[  {P}_{2} {P}_{1} {P}_{2} + {P}_{2}^{2} {P}_{1} = 0 \iff {P}_{2} {P}_{1} {P}_{2} + {P}_{2} {P}_{1} = 0. \]
        Multiplying the quantity above by \( {P}_{2} \) on the right side, we obtain
        \[  {P}_{2} {P}_{1} {P}_{2}^{2} + {P}_{2} {P}_{1} {P}_{2} = 0 \implies 2{P}_{2} {P}_{1} {P}_{2} = 0.    \]
        Hence, it follows that 
        \[  {P}_{1}{P}_{2} = 0  \]
        which is our desired result.
        (\( \Longleftarrow \)) Suppose \( {Y}_{1} \perp {Y}_{2} \). To show that \( P = {P}_{1} + {P}_{2} \) is a projection on \( H  \), it suffices to show that \( P^{2} = P  \) and \( P  \) is self-adjoint. Starting with the first result, since \( {P}_{1} \) and \( {P}_{2} \) are both idempotent, we have that 
        \begin{align*}
            P^{2} &= ({P}_{1} + {P}_{2})^{2} \\
                  &= {P}_{1}^{2} + {P}_{1} {P}_{2}  +{P}_{2} {P}_{1} + {P}_{2}^{2} \\
                  &= {P}_{1}^{2} + {P}_{2}^{2} \\
                  &= {P}_{1} + {P}_{2} \\
                  &= P.
        \end{align*}
        Hence, we see that \( P^{2} = P  \) and so \( P  \) is idempotent. Also, since \( {P}_{1} \) and \( {P}_{2}  \) are also self-adjoint, we can see immediately that \( P  = {P}_{1} + {P}_{2} \) is also idempotent. Thus, we can conclude based on Theorem {\hyperref[9.5-3]{9.5-3}} that \( P  \) is indeed a projection on \( H  \).
    \item[(b)] Suppose \( P = {P}_{1} + {P}_{2} \) is a projection. Note that \( Y  \) is the closed subspace onto which \( P  \) projects to. By definition of \( P  \), it follows that for all \( x \in  H \), we have 
        \begin{align*}
            Px &= ({P}_{1} + {P}_{2})x  \\
               &= {P}_{1}x + {P}_{2}x.
        \end{align*}
        Since \( P  \) is an onto projection, we have 
        \[  y = Px = {P}_{1}x + {P}_{2}x \tag{*} \]
        where \( {P}_{1}x \in {Y}_{1}  \) and \( {P}_{2}x \in {Y}_{2} \). We will show now that \( Y = {Y}_{1} \oplus {Y}_{2} \). Let \( y \in Y  \). From (*), it follows that \( y \in {Y}_{1} \oplus {Y}_{2} \) (clearly, \( P \) being a projection on \( Y  \) means that \( {Y}_{1} \perp {Y}_{2}  \) and so \( {Y}_{1} \cap {Y}_{2} = \{ 0  \}  \)). Now, let \( v \in {Y}_{1} \oplus {Y}_{2} \). Then it follows that 
        \begin{align*}
            P(v) &= {P}_{1}v + {P}_{2}v  \\
                 &= {P}_{1}({y}_{1}) + {P}_{2}({y}_{2}) \\
                 &= {Y}_{1} + {Y}_{2} \\
                 &= v.
        \end{align*}
        Hence, \( v \in Y  \) and so we have \( {Y}_{1} \oplus {Y}_{2} \subseteq  Y  \). Thus, we conclude that \(  Y = {Y}_{1} \oplus {Y}_{2} \).
\end{enumerate}

