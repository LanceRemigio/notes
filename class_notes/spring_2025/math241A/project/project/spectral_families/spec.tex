\subsection{Properties of Spectral Family}

Our goal for this section is to prove that our notion of a spectral family \( \mathcal{G} \) introduced in the introduction our paper does, indeed, contain all the properties that we desire for an integral representation of \( T  \).

To define \( \mathcal{G} \), we will use the following operator
\[  {T}_{\lambda} = T - \lambda I  \]
and the positive square root of \( {T}_{\lambda}^{2}  \), which we denote by \( {B}_{\lambda} \); that is,
\[  {B}_{\lambda} = ({T}_{\lambda}^{2})^{1/2} \]
as well as 
\[  {T}_{\lambda}^{+} = \frac{ 1 }{ 2 }  ({B}_{\lambda} - {T}_{\lambda}) \]
which is denoted as the \textbf{positive part} of \( {T}_{\lambda} \).

\begin{definition}[Spectral Family \( \mathcal{G} \) of \( T  \) Kreyszig (9.8)]
    We denote the \textbf{Spectral Family \( \mathcal{G} \) of \( T  \)} by \( \mathcal{G} = ({E}_{\lambda})_{\lambda \in |R } \) where \( {E}_{\lambda}  \) is the projection of \( H  \) onto the null space \( N({T}_{\lambda}^{+}) \) of \( {T}_{\lambda}^{+} \).
\end{definition}

Our goal for the remaining of this section is to prove that indeed \( \mathcal{G} \) contains all the properties outlined in definition {\hyperref[9.7-1]{9.7-1}}. Consider the following operators 
\begin{align*}
    B &= (T^{2})^{1/2} \tag{Postive Square Root of \( T^{2} \)} \\
    T^{+} &= \frac{ 1 }{ 2 } ( B +T ) \tag{Positive part of \( T  \)} \\
    T^{-}&= \frac{ 1 }{ 2 }  (B - T ) \tag{Negative part of \( T  \)}
\end{align*}
and the projection of \( H  \) onto the null space \( T^{+ } \) which we denote by \( E  \) i.e  
\[  E: H \to Y = N(T^{+}). \]
Subtracting and adding the positive and negative part of \( T  \), respectively, we see that 
\begin{align*}
    T &= T^{+ } - T^{-} \\
    B &= T^{+} + T^{-}.
\end{align*}

A natural question to ask at this point is: why do we need to decompose an operator into its positive and negative parts in the first place? Since we've been leveraging the many useful properties of positive operators—such as the existence of square roots, monotonic sequences of projections, and strong operator convergence—it is advantageous to break a more complicated operator into simpler, more manageable components. This decomposition allows us to better understand the operator’s structure and behavior. Without it, we would lose access to the essential tools that make the spectral theorem work.

The next lemma says that \( B , T^{+}, T^{-1}  \), and \( E \) all contain the necessary properties that we desire in order for our spectral integral representation to work. 

\begin{lemma}[Operators Related to \( T  \) Kreyszig (9.8-1)]\label{9.8-1} 
    The operators just defined have the following properties 
    \begin{enumerate}
        \item[(a)] \( B , T^{+},  \) and \( T^{-} \) are bounded and self-adjoint.
        \item[(b)] \( B, T^{+},  \) and \( T^{-} \) commute with every bounded linear operator that \( T  \) commutes with; in particular,
            \[  BT = TB  \ \ T^{+}T = T T^{+} \ \ T^{-} T = T T^{-} \ \ T^{+} T^{-} = T^{-} T^{+}.  \]
        \item[(c)] \( E  \) commutes with every bounded self-adjoint linear operator that \( T  \) commutes with; in particular, 
            \[  E T = TE  \ \ \ EB = BE.  \]
        \item[(d)] Furthermore,
            \begin{align*}
                T^{+} T^{-} &= 0   &T^{-} T^{+} = 0   \\
                T^{+} E &= E T^{+ } = 0   &T^{-}E = E T^{-} = T^{-} \\
                TE  &= - T^{-}   &T(I  - E ) = T^{+} \\
                T^{+} &\geq 0   &T^{-1} \geq 0. 
            \end{align*}
    \end{enumerate}
\end{lemma}
\begin{proof}
\begin{enumerate}
    \item[(a)] By definition of \( B  \) and \( T  \), it follows immediately that \( B , T^{+} \) and \( T^{-} \) are bounded and self-adjoint. 
    \item[(b)] Suppose \( T S  = ST  \). Then we have 
        \[  T^{2} S = T (T S) = T (ST) = (TS)T = (ST)T = S T^{2}  \]
        Hence, \( T^{2} S = S T^{2} \). From {\hyperref[9.4-2]{9.4-2}}, it also follows that \( BS = SB \) when applied \( T^{2} \). Hence, we have 
        \[  T^{+}S = \frac{ 1 }{ 2 }  (BS  + TS) = \frac{ 1 }{ 2 }  (SB + ST) = S T^{+}.  \]
        Similarly, we have 
        \[  T^{-} S = \frac{ 1 }{ 2 } (BS - TS) = \frac{ 1 }{ 2 }  (SB - ST) = S T^{-}.  \]
        Hence, we have \( T^{+}S = S T^{+} \) and \( T^{-} S = S T^{-} \).
    \item[(c)] For every \( x \in H  \), we have \( y  = Ex \in Y = N(T^{+}) \) since \( E  \) is a projection of \( H  \) onto \( Y  \). Also, we have \( E S E x = SE x   \) for all \( x \in H  \); that is, \( ESE = SE \). Since projections are self-adjoint by {\hyperref[9.5-1]{9.5-1}}, we see that \( S  \) is also self-adjoint. Using {\hyperref[(6g)-3.9]{(6g)-3..9}}, we obtain   
        \[  ES = E^{*} S^{*} = (SE)^{*} = (ESE)^{*} = E^{*} S^{*} E^{*} = ESE = SE. \]
    \item[(d)] We will prove the following   
            \begin{align*}
                T^{+} T^{-} &= 0    &T^{-} T^{+} &= 0 \tag{i}   \\
                T^{+} E &= E T^{+ } = 0    &T^{-}E &= E T^{-} = T^{-} \tag{ii} \\
                TE  &= - T^{-}      &T(I  - E ) &= T^{+} \tag{iii} \\
                T^{+} &\geq 0    &T^{-1} &\geq 0. \tag{iv} 
            \end{align*}
            From \( B = (T^{2})^{1/2} \), we have \( B^{2} = T^{2} \). Note that \( BT = TB \) by (a). Hence, we see that 
            \[  T^{+}T = T^{-}T^{+} = \frac{ 1 }{ 2 } (B - T ) \frac{ 1 }{ 2 }  (B+T) = \frac{ 1 }{ 4 }  (B^{2} + BT - TB - T^{2}) = 0.  \]
            This gives us (i).

            Next, we prove (ii). By definition of \( E  \), we can see that \( Ex \in N (T^{+}) \) and so \( T^{+ } E x = 0  \) for all \( x \in  H \). Because \( T^{+ }  \) is self-adjoint, we have \( ET^{+} x = T^{+} E x = 0  \) by (a) and (c); that is, we have 
            \[  E T^{+} = T^{+}E = 0.  \]
            Moreover, we have \( T^{+} T^{-}x  = 0  \) by (i). Indeed, we have 
            \[  T^{+}T^{-}x = T^{-}T^{+}x = T^{-}0 = 0. \]

            Next, we prove (iii). Since \( T = T^{+ } - T^{-} \) and (ii), we can see that   
            \[  TE = (T^{+} - T^{-}) E = T^{+}E - T^{-}E = 0 - T^{-} E = -T^{-}.   \]
            Using \( T=  T^{+ } + T^{-} \) again, we get
            \[  T(I - E ) = T - T E = T + T^{-} = T^{+}. \]
            Finally, we prove (iv). Using \( B = T^{+ } + T^{-} \), part (ii), and {\hyperref[9.3-1]{Theorem 9.3-1}}, we have  
            \[  T^{+} = B - T^{-} = B - EB = (I - E) B \geq 0  \]
            where \( I - E \geq 0  \) by {\hyperref[9.5-2]{9.5-2}}.
\end{enumerate}
\end{proof}

In addition to decomposing \( T \) into more manageable components, the lemma above establishes key properties such as the commutativity and orthogonality of these parts. It also clarifies the role of the projection operator \( E  \), which, when applied to \( T \), isolates its positive and negative components for deeper insight into the behavior of \( T \).

Roughly speaking, each projection \( {E}_{\lambda} \) corresponds to the portion of the Hilbert space associated with the spectrum of \( T \) up to the value \( \lambda \). This structure allows us to approximate \( T \) by partitioning the spectrum and summing over the action of these projections on narrower spectral intervals. The orthogonality of projections on disjoint subintervals of the spectrum ensures that their contributions are non-overlapping, mirroring the disjoint behavior seen in the decomposition into positive and negative parts.

As our next step in developing the associated spectral family of \( T  \) with \( \lambda  \) as the parameter, we shall replace our notion of \( T, B , T^{+}, T^{-} ,   \) and \( E  \) by \( {T}_{\lambda} = T - \lambda I  \), \( {B}_{\lambda} = ({T}_{\lambda}^{2})^{1/2}  \), and the positive part and negative part of \( T  \) given by
\begin{align*}
    {T}_{\lambda}^{+} &= \frac{ 1 }{ 2 }  ({B}_{\lambda} + {T}_{\lambda}) \\
    {T}_{\lambda}^{-} &= \frac{ 1 }{ 2 }  ({B}_{\lambda} - {T}_{\lambda})
\end{align*}
and the projection
\[  {E}_{\lambda} : H \to {Y}_{\lambda} = N({T}_{\lambda}^{+}) \]
of \( H \) onto the null space \( {Y}_{\lambda} = N({T}_{\lambda}^{+}) \) of \( {T}_{\lambda}^{+} \). Now, we will restate the lemma we just proved in the following way: 

\begin{lemma}[Operators Related to \( T_{\lambda} \) Kreyszig (9.8-2)]\label{9.8-2}
   The previous lemma remains true if we replace \( T, B , T^{+}, T^{-}, E  \) by \( {T}_{\lambda}, {B}_{\lambda}, {T}_{\lambda}^{+}, {T}_{\lambda}^{-}, {E}_{\lambda} \) where \( {T}_{\lambda} = T - \lambda I  \) and \( \lambda \in \R  \). Moreover, for any real \( \kappa, \lambda , \mu, \nu, \tau  \), the following operators all commute:
   \begin{align*}  {T}_{\kappa} \ \ \   {B}_{\lambda} \ \ \  {T}_{\mu}^{+} \ \ \  {T}_{\nu}^{-} \ \ \   {E}_{\tau}   \end{align*}
\end{lemma}
\begin{proof}
The first statement follows from the previous lemma. To show the second statement, we observe that \( IS = SI  \) and 
\[  {T}_{\lambda} = T - \lambda I  = T - \mu I + \mu I - \lambda I  = T - \mu I + (\mu - \lambda)I  = {T}_{\mu} + (\mu - \lambda) I. \]
Hence, we have 
\begin{align*}
    ST = TS &\implies S {T}_{\mu } = {T}_{\mu } S   \\
            &\implies S  {T}_{\lambda} = {T}_{\lambda} S  \\
            &\implies S {B}_{\lambda} = {B}_{\lambda} S, S {B}_{\mu } = {B}_{\mu } S  
\end{align*}
and etc. For \( S = {T}_{\kappa} \) and so we have \( {T}_{\kappa} {B}_{\lambda} = {B}_{\lambda} {T}_{\kappa}   \) etc.
\end{proof}


In fact, it can be proven from the lemma above that we can uniquely express \( T  \) in terms of a spectral family \( \mathcal{G} = ({E}_{\lambda}) \). 

In addition to the properties introduced earlier in this section, we require our spectral family to satisfy the following additional conditions:
\begin{enumerate}
    \item[(i)] If \( \lambda < \mu  \), then \( {E}_{\lambda} \leq {E}_{\mu } \).
    \item[(ii)] If \( \lambda < m  \), then \( {E}_{\lambda} = 0  \).
    \item[(iii)] If \( \lambda \geq M   \), then \( {E}_{\lambda} = I  \) 
    \item[(iv)] If \( \mu \to \lambda + 0  \), then \( {E}_{\mu } x \to {E}_{\lambda} x  \)
\end{enumerate}
where \( m = \inf_{\|x\| = 1 } \langle Tx  ,  x  \rangle  \) and \( M = \sup_{\|x\| = 1 } \langle Tx  , x  \rangle \).


\begin{theorem}[Spectal Family Associated with an Operator Kreyszig (9.8-3)]\label{9.8-3}
    Let \( T : H \to H  \) be a bounded self-adjoint linear operator on a complex Hilbert space \( H  \). Furthermore, let \( {E}_{\lambda}  \) (\( \lambda \in \R  \)) be the projection of \( H  \) onto the null space \( {Y}_{\lambda } = N({T}_{\lambda}^{+}) \) of the positive part \( {T}_{\lambda}^{+}  \) of \( {T}_{\lambda} = T - \lambda I  \). Then \( G = ({E}_{\lambda})_{\lambda \in \R }\) is a spectral family on the interval \( [m,M ] \subseteq \R  \) where \( m  \) and \( M  \) are given by (1) in Section 9.2.
\end{theorem}
\begin{proof}
Our goal is to show that properties (i) through (iv) listed above.

To prove (i) through (iv), we will use part of {\hyperref[9.8-1]{9.8-1}} formulated for \( {T}_{\lambda}, {T}_{\mu} , {T}_{\lambda}^{+} \) instead of \( T  \), \( T^{+} \); namely,  
\begin{enumerate}
    \item[(I)] \( {T}_{\mu }^{+} {T}_{\mu }^{-} = 0  \)
    \item[(II)] \( {T}_{\lambda} {E}_{\lambda} = - {T}_{\lambda}^{-}  \), \( {T}_{\lambda} (I - {E}_{\lambda}) = {T}_{\lambda}^{+} \), \( {T}_{\mu } {E}_{\mu} \)
    \item[(III)] \( {T}_{\lambda}^{+} \geq 0  \), \( {T}_{\lambda}^{-} \geq 0  \), \( {T}_{\mu }^{+} \geq 0  \) and \( {T}_{\mu }^{k-} \geq 0  \).
\end{enumerate}

\begin{enumerate}
    \item[(i)] Let \( \lambda < \mu  \). Then 
        \[  {T}_{\lambda} = {T}_{\lambda}^{+} - {T}_{\lambda}^{-} \leq {T}_{\lambda}^{+}  \] because \( -T^{-} \leq 0  \) by (III). Hence, we have  
        \[  {T}_{\lambda}^{+} - {T}_{\mu } \geq {T}_{\lambda } - {T}_{\mu  } = (\mu  - \lambda ) I \geq 0.  \] 
    Note that \( {T}_{\lambda}^{+} - {T}_{\mu  } \) is self-adjoint and commutes with \( {T}_{\mu }^{+} \) by {\hyperref[9.8-2]{9.8-2}}, and \( {T}_{\mu }^{+} \geq 0  \) by (III). Using Theorem {\hyperref[9.3-1]{9.3-1}}, we have  
    \[  {T}_{\mu }^{+} ({T}_{\lambda}^{+} - {T}_{\mu }) = {T}_{\mu }^{+} ({T}_{\lambda}^{+} - {T}_{\mu  }^{+} + {T}_{\mu }^{+}) \geq 0.  \]
    Here, we have \( {T}_{\mu }^{+} {T}_{\mu}^{-}  = 0 \) by (I). Hence, we have \( {T}_{\mu  }^{+} {T}_{\lambda}^{+} \geq {T}_{\mu }^{+2}  \); that is, for all \( x \in H  \), 
    \[  \langle {T}_{\mu  }^{+} {T}_{\lambda}^{+} x  , x  \rangle \geq \langle {T}_{\mu }^{+} x  , x  \rangle = \|{T}_{\mu }^{+} x \|^{2} \geq 0.  \]
    This tell us that \( {T}_{\lambda}^{+} x = 0  \) implies \( {T}_{\mu  }^{+} x = 0  \) for all \( x \in H  \). Thus, \( N({T}_{\lambda}^{+}) \subseteq  N({T}_{\mu }^{+}) \), and so we have \( {E}_{\lambda } \leq {E}_{\mu }  \) by {\hyperref[9.6-1]{9.6-1}}. Here, we have \( \lambda < \mu  \). 
\item[(ii)] Let \( \lambda < m  \). Suppose for contradiction that \( {E}_{\lambda} \neq 0  \). Then \( {E}_{\lambda}z \neq 0   \) for some \( z \in H  \). We set \( x = {E}_{\lambda}z  \). Then we have
    \[  {E}_{\lambda} x  = {E}_{\lambda} ({E}_{\lambda} z ) = {E}_{\lambda}^{2} z = x.    \]
    Without loss of generality, suppose \( \|x\| = 1  \). Hence, we have 
    \begin{align*}
        \langle {T}_{\lambda}{E}_{\lambda} x  ,  x  \rangle &= \langle {T}_{\lambda} x  ,  x  \rangle \\
                                                            &= \langle (T - \lambda I ) x  , x  \rangle \\
                                                            &= \langle Tx , x  \rangle - \langle \lambda x  ,  x  \rangle \\
                                                            &= \langle Tx , x  \rangle - \lambda \langle x , x  \rangle \\
                                                            &= \langle Tx , x  \rangle - \lambda  \tag{\( \|x\| = 1  \)} \\
                                                            &\geq \inf_{\|\tilde{x}\| = 1 } \langle T \tilde{x} ,  \tilde{x} \rangle - \lambda \\
                                                            &= m - \lambda > 0. 
    \end{align*}
    But note that this contradicts the fact that \( {T}_{\lambda} {E}_{\lambda} = - {T}_{\lambda}^{-} \leq 0  \) from (II) and (III).
    \item[(iii)] Suppose for contradiction that \( \lambda > M  \) but \( {E}_{\lambda} \neq I  \), so that \( I - {E}_{\lambda} \neq 0  \). Then \( (I - {E}_{\lambda})x = x  \) for some \( x  \) of norm \( \|x\| = 1  \). Hence, we have 
        \begin{align*}
            \langle {T}_{\lambda}(I - {E}_{\lambda}) x  ,  x  \rangle &= \langle {T}_{\lambda} x  ,  x  \rangle  \\
                                                                      &=  \langle {T}_{\lambda} x - {T}_{\lambda} {E}_{\lambda} x    ,  x  \rangle \\
                                                                      &= \langle {T}_{\lambda} x  ,  x  \rangle - \langle {T}_{\lambda} ({E}_{\lambda} x ) , x  \rangle \\
                                                                      &= \langle {T}_{\lambda}x  ,  x  \rangle - \langle {T}_{\lambda}(0)  ,  x  \rangle \\
                                                                      &= \langle {T}_{\lambda} x  ,  x  \rangle \\
                                                                      &= \langle (T - \lambda I ) x  ,  x  \rangle \\
                                                                      &= \langle T x  ,  x  \rangle - \lambda \langle x  ,  x  \rangle \\
                                                                      &= \langle T x  ,  x  \rangle - \lambda \tag{\( \|x\| = 1  \)} \\
                                                                      &\leq \sup_{\|\tilde{x}\| = 1 } \langle T \tilde{x}  ,  \tilde{x} \rangle - \lambda \\
                                                                      &= M - \lambda < 0. 
        \end{align*}
        But this contradicts \( {T}_{\lambda}(I - {E}_{\lambda} ) = {T}_{\lambda}^{+} \geq 0  \) which is obtained from (II) and (III). Moreover, \( {E}_{M} = I  \) by the continuity from the right.
    \item[(iv)] With an interval \( \Delta = (\lambda, \mu ] \), we associate the operator  
        \[  \Delta E = {E}_{\mu } - {E}_{\lambda}. \]
        Since \( \lambda < \mu   \), we have \( {E}_{\lambda} \leq {E}_{\mu } \) by (i) and so \( {E}_{\lambda}(H) \subseteq  {E}_{\mu  }(H)  \) by {\hyperref[9.6-1]{9.6-1}}. Moreover, \( \Delta E  \) is a projection by {\hyperref[9.6-2]{9.6-2}} and \( \Delta E \geq 0  \) by {\hyperref[9.5-2]{9.5-2}}. Using {\hyperref[9.6-1]{9.6-1}},     
        \[  {E}_{\mu } \Delta E = {E}_{\mu }^{2} - {E}_{\mu } {E}_{\lambda} = {E}_{\mu } - {E}_{\lambda} = \Delta E \tag{*} \]
        and
        \[  (I - {E}_{\lambda}) \Delta E = \Delta E - {E}_{\lambda} ({E}_{\mu  } - {E}_{\lambda}) = \Delta E. \tag{**} \]
        Since \( \Delta E  \), \( {T}_{\mu  }^{-}  \) and \( {T}_{\lambda}^{+} \) are positive and commute by {\hyperref[9.8-2]{9.8-2}}, we have that \( {T}_{\mu }^{-} \Delta E  \) and \( {T}_{\lambda}^{+} \Delta E  \) are positive by {\hyperref[9.3-1]{9.3-1}}. From (*) and (**), we can see that 
        \begin{align*}
            {T}_{\mu } \Delta E  &= {T}_{\mu } {E}_{\mu  } \Delta E = - {T}_{\mu }^{-} \Delta E \leq 0  \\
            {T}_{\lambda} \Delta E &= {T}_{\lambda} (I - {E}_{\lambda}) \Delta E = {T}_{\lambda}^{+} \Delta E  \geq 0.  
        \end{align*}
        This tells us that \( T\Delta E \leq \mu \Delta E  \) and \( T \Delta E \geq \lambda \Delta E  \), respectively. Together, we have 
        \[ \lambda \Delta E \leq T \Delta E \leq \mu \Delta E \tag{\( \dagger \)}   \]
        where \( \Delta E = {E}_{\mu  } - {E}_{\lambda} \).
\end{enumerate}

Now, fix \( \lambda  \) and let \( \mu  \to \lambda  \) form the right in a monotone fashion. Then \( \Delta E x \to P(\lambda) \) by the analogue of Theorem {\hyperref[9.3-3]{9.3-3}} for a decreasing sequence. Here \( P(\lambda) \) is bounded and self-adjoint. Now, since \( \Delta E  \) is idempotent (because it is a projection), we can see that \( P(\lambda) \) is also idempotent. Thus, \( P(\lambda) \) is a projection. Also, \( \lambda P(\lambda) = TP(\lambda) \) by (\( \dagger \)); that is, \( {T}_{\lambda} P(\lambda) = 0  \). Thus, by using (II) and {\hyperref[9.8-2]{[9.8-2]}}, we have 
\[  {T}_{\lambda}^{+} P(\lambda) = {T}_{\lambda}(I - {E}_{\lambda}) P(\lambda) = (I - {E}_{\lambda}) {T}_{\lambda} P(\lambda) = 0.  \]
Hence, we have \( {T}_{\lambda}^{+} P(\lambda) x = 0  \) for all \( x \in H  \).  This tell us that \( P(\lambda) x \in N ({T}_{\lambda}^{+}) \). Also, \( {E}_{\lambda} \) is a projection from \( H  \) onto \( N({T}_{\lambda}^{+}) \). As a consequence, we have 
\[  {E}_{\lambda}P(\lambda) x = P(\lambda) x,  \]
that is, \( {E}_{\lambda} P(\lambda) = P(\lambda) \). However, if we let \( \mu \to \lambda + 0  \) in (*), then 
\[  (I - {E}_{\lambda}) P(\lambda) = P(\lambda). \]
All together, we have \( P(\lambda) = 0  \). Since \( \Delta E x \to P(\lambda) x \), we see that (16) holds and so our Spectral family \( \mathcal{G} \) is continuous from the right.
\end{proof}  
\begin{remark}
   Note that (\( \dagger \)) is a key inequality that we will use in our proof of the spectral representation of \( T  \) in the next section.
\end{remark}

