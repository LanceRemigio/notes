\documentclass[a4paper]{article}
\usepackage[utf8]{inputenc}
\usepackage[T1]{fontenc}
% \usepackage{fourier}
\usepackage{textcomp}
\usepackage{hyperref}
\usepackage[english]{babel}
\usepackage{url}
% \usepackage{hyperref}
% \hypersetup{
%     colorlinks,
%     linkcolor={black},
%     citecolor={black},
%     urlcolor={blue!80!black}
% }
\usepackage{graphicx} \usepackage{float}
\usepackage{booktabs}
\usepackage{enumitem}
% \usepackage{parskip}
% \usepackage{parskip}
\usepackage{emptypage}
\usepackage{subcaption}
\usepackage{multicol}
\usepackage[usenames,dvipsnames]{xcolor}
\usepackage{ocgx}
% \usepackage{cmbright}


\usepackage[margin=1in]{geometry}
\usepackage{amsmath, amsfonts, mathtools, amsthm, amssymb}
\usepackage{thmtools}
\usepackage{mathrsfs}
\usepackage{cancel}
\usepackage{bm}
\newcommand\N{\ensuremath{\mathbb{N}}}
\newcommand\R{\ensuremath{\mathbb{R}}}
\newcommand\Z{\ensuremath{\mathbb{Z}}}
\renewcommand\O{\ensuremath{\emptyset}}
\newcommand\Q{\ensuremath{\mathbb{Q}}}
\newcommand\C{\ensuremath{\mathbb{C}}}
\newcommand\F{\ensuremath{\mathbb{F}}}
\DeclareMathOperator{\sgn}{sgn}
\DeclareMathOperator{\diam}{diam}
\DeclareMathOperator{\LO}{LO}
\DeclareMathOperator{\UP}{UP}
\DeclareMathOperator{\card}{card}
\DeclareMathOperator{\Arg}{Arg}
\DeclareMathOperator{\Dom}{Dom}
\DeclareMathOperator{\Log}{Log}
\DeclareMathOperator{\dist}{dist}
% \DeclareMathOperator{\span}{span}
\usepackage{systeme}
\let\svlim\lim\def\lim{\svlim\limits}
\renewcommand\implies\Longrightarrow
\let\impliedby\Longleftarrow
\let\iff\Longleftrightarrow
\let\epsilon\varepsilon
\usepackage{stmaryrd} % for \lightning
\newcommand\contra{\scalebox{1.1}{$\lightning$}}
% \let\phi\varphi
\renewcommand\qedsymbol{$\blacksquare$}

% correct
\definecolor{correct}{HTML}{009900}
\newcommand\correct[2]{\ensuremath{\:}{\color{red}{#1}}\ensuremath{\to }{\color{correct}{#2}}\ensuremath{\:}}
\newcommand\green[1]{{\color{correct}{#1}}}

% horizontal rule
\newcommand\hr{
    \noindent\rule[0.5ex]{\linewidth}{0.5pt}
}

% hide parts
\newcommand\hide[1]{}

% si unitx
\usepackage{siunitx}
\sisetup{locale = FR}
% \renewcommand\vec[1]{\mathbf{#1}}
\newcommand\mat[1]{\mathbf{#1}}

% tikz
\usepackage{tikz}
\usepackage{tikz-cd}
\usetikzlibrary{intersections, angles, quotes, calc, positioning}
\usetikzlibrary{arrows.meta}
\usepackage{pgfplots}
\pgfplotsset{compat=1.13}

\tikzset{
    force/.style={thick, {Circle[length=2pt]}-stealth, shorten <=-1pt}
}

% theorems
\makeatother
\usepackage{thmtools}
\usepackage[framemethod=TikZ]{mdframed}
\mdfsetup{skipabove=1em,skipbelow=1em}

\theoremstyle{definition}

\declaretheoremstyle[
    headfont=\bfseries\sffamily\color{ForestGreen!70!black}, bodyfont=\normalfont,
    mdframed={
        linewidth=1pt,
        rightline=false, topline=false, bottomline=false,
        linecolor=ForestGreen, backgroundcolor=ForestGreen!5,
    }
]{thmgreenbox}

\declaretheoremstyle[
    headfont=\bfseries\sffamily\color{NavyBlue!70!black}, bodyfont=\normalfont,
    mdframed={
        linewidth=1pt,
        rightline=false, topline=false, bottomline=false,
        linecolor=NavyBlue, backgroundcolor=NavyBlue!5,
    }
]{thmbluebox}

\declaretheoremstyle[
    headfont=\bfseries\sffamily\color{NavyBlue!70!black}, bodyfont=\normalfont,
    mdframed={
        linewidth=1pt,
        rightline=false, topline=false, bottomline=false,
        linecolor=NavyBlue
    }
]{thmblueline}

\declaretheoremstyle[
    headfont=\bfseries\sffamily, bodyfont=\normalfont,
    numbered = no,
    mdframed={
        rightline=true, topline=true, bottomline=true,
    }
]{thmbox}

\declaretheoremstyle[
    headfont=\bfseries\sffamily, bodyfont=\normalfont,
    numbered=no,
    % mdframed={
    %     rightline=true, topline=false, bottomline=true,
    % },
    qed=\qedsymbol
]{thmproofbox}

\declaretheoremstyle[
    headfont=\bfseries\sffamily\color{NavyBlue!70!black}, bodyfont=\normalfont,
    numbered=no,
    mdframed={
        rightline=false, topline=false, bottomline=false,
        linecolor=NavyBlue, backgroundcolor=NavyBlue!1,
    },
]{thmexplanationbox}

\declaretheorem[
    style=thmbox, 
    % numberwithin = section,
    numbered = no,
    name=Definition
    ]{definition}

\declaretheorem[
    style=thmbox, 
    name=Example,
    ]{eg}

\declaretheorem[
    style=thmbox, 
    % numberwithin = section,
    name=Proposition]{prop}

\declaretheorem[
    style = thmbox,
    numbered=yes,
    name =Problem
    ]{problem}

\declaretheorem[style=thmbox, name=Theorem]{theorem}
\declaretheorem[style=thmbox, name=Lemma]{lemma}
\declaretheorem[style=thmbox, name=Corollary]{corollary}

\declaretheorem[style=thmproofbox, name=Proof]{replacementproof}

\declaretheorem[style=thmproofbox, 
                name = Solution
                ]{replacementsolution}

\renewenvironment{proof}[1][\proofname]{\vspace{-1pt}\begin{replacementproof}}{\end{replacementproof}}

\newenvironment{solution}
    {
        \vspace{-1pt}\begin{replacementsolution}
    }
    { 
            \end{replacementsolution}
    }

\declaretheorem[style=thmexplanationbox, name=Proof]{tmpexplanation}
\newenvironment{explanation}[1][]{\vspace{-10pt}\begin{tmpexplanation}}{\end{tmpexplanation}}

\declaretheorem[style=thmbox, numbered=no, name=Remark]{remark}
\declaretheorem[style=thmbox, numbered=no, name=Note]{note}

\newtheorem*{uovt}{UOVT}
\newtheorem*{notation}{Notation}
\newtheorem*{previouslyseen}{As previously seen}
% \newtheorem*{problem}{Problem}
\newtheorem*{observe}{Observe}
\newtheorem*{property}{Property}
\newtheorem*{intuition}{Intuition}

\usepackage{etoolbox}
\AtEndEnvironment{vb}{\null\hfill$\diamond$}%
\AtEndEnvironment{intermezzo}{\null\hfill$\diamond$}%
% \AtEndEnvironment{opmerking}{\null\hfill$\diamond$}%

% http://tex.stackexchange.com/questions/22119/how-can-i-change-the-spacing-before-theorems-with-amsthm
\makeatletter
% \def\thm@space@setup{%
%   \thm@preskip=\parskip \thm@postskip=0pt
% }
\newcommand{\oefening}[1]{%
    \def\@oefening{#1}%
    \subsection*{Oefening #1}
}

\newcommand{\suboefening}[1]{%
    \subsubsection*{Oefening \@oefening.#1}
}

\newcommand{\exercise}[1]{%
    \def\@exercise{#1}%
    \subsection*{Exercise #1}
}

\newcommand{\subexercise}[1]{%
    \subsubsection*{Exercise \@exercise.#1}
}


\usepackage{xifthen}

\def\testdateparts#1{\dateparts#1\relax}
\def\dateparts#1 #2 #3 #4 #5\relax{
    \marginpar{\small\textsf{\mbox{#1 #2 #3 #5}}}
}

\def\@lesson{}%
\newcommand{\lesson}[3]{
    \ifthenelse{\isempty{#3}}{%
        \def\@lesson{Lecture #1}%
    }{%
        \def\@lesson{Lecture #1: #3}%
    }%
    \subsection*{\@lesson}
    \testdateparts{#2}
}

% \renewcommand\date[1]{\marginpar{#1}}


% fancy headers
\usepackage{fancyhdr}
\pagestyle{fancy}

\makeatother

% notes
\usepackage{todonotes}
\usepackage{tcolorbox}

\tcbuselibrary{breakable}
\newenvironment{verbetering}{\begin{tcolorbox}[
    arc=0mm,
    colback=white,
    colframe=green!60!black,
    title=Opmerking,
    fonttitle=\sffamily,
    breakable
]}{\end{tcolorbox}}

\newenvironment{noot}[1]{\begin{tcolorbox}[
    arc=0mm,
    colback=white,
    colframe=white!60!black,
    title=#1,
    fonttitle=\sffamily,
    breakable
]}{\end{tcolorbox}}

% figure support
\usepackage{import}
\usepackage{xifthen}
\pdfminorversion=7
\usepackage{pdfpages}
\usepackage{transparent}
\newcommand{\incfig}[1]{%
    \def\svgwidth{\columnwidth}
    \import{./figures/}{#1.pdf_tex}
}

% %http://tex.stackexchange.com/questions/76273/multiple-pdfs-with-page-group-included-in-a-single-page-warning
\pdfsuppresswarningpagegroup=1


\title{Math 241 Homework 2}
\author{Lance Remigio}
\begin{document}
\maketitle

\begin{remark}
    In the first two problems of this homework, whenever I state \( {x}_{i}^{(k)} \to {x}_{i} \), I mean whenever we let \( k \to \infty  \).
\end{remark}
\begin{problem}
    Prove that \( (\R^{n}, {d}_{\infty }) \) is complete.
\end{problem}
\begin{proof}
    Let \( (\vec{ {x}_{k} } ) \) be a Cauchy sequence in \( \R^{n} \). Note that \( 1 \leq i \leq n  \) denotes the \( i \)th component of elements in \( \R^{n} \) and \( k \in \N \) is the index for each sequence in \( \R^{n} \). By a result found in quiz 1, it follows that \( ({x}_{i}^{(k)}) \) for \( 1 \leq i \leq n  \) is also Cauchy. Since \( \R  \) is a complete metric space with respect to the standard metric on \( \R  \), we find that each \( {x}_{i}^{(k)} \) is also a convergent sequence. By another result in quiz 1, it follows that \( (\vec{ {x}_{k} } ) \) is a convergent sequence; that is, for each \( 1 \leq i \leq n  \), \( {x}_{i}^{(k)} \to {x}_{i} \) where \( {x}_{i} \in \R  \). Clearly, we have 
    \[  \vec{ x } =  ({x}_{1}, {x}_{2}, \dots, {x}_{n} ) \in \R^{n}. \]
    Hence, \( \R^{n} \) with respect to the \( {d}_{\infty } \) metric is complete.
\end{proof}

\begin{problem}
    \begin{enumerate}
        \item[(i)] Let \( \vec{x} = \displaystyle \begin{pmatrix} {x}_{1} \\ \vdots \\ {x}_{n}  \end{pmatrix}  \). Prove that \( {\pi}_{i}: \R^{n} \to \R  \) defined by
            \(  {\pi}_{i}(\vec{ x } ) = {x}_{i} \)
            are continuous maps with respect to \( {d}_{\infty }  \) on \( \R^{n}  \) and the standard metric on \( \R  \).
        \item[(ii)] Prove that \( {\pi}_{i} \) in (i) are continuous maps with respect to \( {d}_{\text{euclid}} \) on \( \R^{n} \) and the standard metric on \( \R  \).
    \end{enumerate}
\end{problem}
\begin{proof}
\begin{enumerate}
    \item[(i)] Our goal is to show that \( {\pi}_{i} \) is a continuous map with respect to \( {d}_{\infty } \) on \( \R^{n} \); we will do this via the sequential criterion of continuity. Suppose \( \vec{ {x}_{k} }  \to \vec{ x }  \) for some \( \vec{ x }  \) in \( \R^{n} \). By a result found in quiz 1, we can see that \( {x}_{i}^{(k)} \to {x}_{i} \) for \( 1 \leq i \leq n  \). By definition of \( {\pi}_{i} \), we find that as \( k \to \infty  \), we get
        \[ {\pi}_{i}(\vec{ x_{k} } ) = {x}_{i}^{(k)} \to {x}_{i} = {\pi}_{i}(\vec{ x } ).  \]
        Hence, we have that \( {\pi}_{i} \) is a continuous map with respect to \( {d}_{\infty } \) and the standard metric on \( \R  \).
    \item[(ii)] Our goal is to show that \( {\pi}_{i}: \R^{n} \to \R  \) is continuous with respect to \( {d}_{\text{euclid}}  \). Let \( (\vec{ {x}_{k} } )\) be a sequence in \( \R^{n} \) such that \( \vec{ {x}_{k} } \to \vec{ x }    \). Since \( {\pi}_{i} \) is continuous on \( \R^{n} \) with respect to the \( {d}_{\infty }  \) metric, we have \( {d}_{\infty }(\vec{ {x}_{k} }, \vec{ x } ) \to 0 \). Notice that 
        \[  0 \leq {d}_{\text{euclid}}(\vec{ {x}_{k} } , \vec{ x } ) \leq (n)^{1/2} {d}_{\infty }(\vec{ {x}_{k} } , \vec{ x } ). \tag{1} \]
        Now, \( {d}_{\infty }(\vec{ {x}_{k}}, \vec{ x }  ) \to 0  \) implies that \( {d}_{\text{euclid}}(\vec{ {x}_{k} } , \vec{ x } ) \to 0  \) as \( k \to \infty  \) by applying the squeeze theorem to (1). Thus, we can see that  
        \begin{align*}
            | {\pi}_{i}(\vec{ {x}_{k} } ) - {\pi }_{i}(\vec{ x } )  | &= \Big( | {\pi}_{i}(\vec{ {x}_{k} } ) - {\pi}_{i}( \vec{ x }  ) |^{2} \Big)^{1/2}  \\
                                                                      &= \Big( | {x}_{i}^{(k)} - {x}_{i} |^{2}  \Big)^{1/2} \\
                                                                      &\leq \Big( \sum_{ i=1 }^{ n} | {x}_{i}^{(k)} - {x}_{i} |^{2} \Big)^{1/2} \\
                                                                      &= {d}_{\infty }(\vec{ {x}_{k} } , \vec{ x } ) \to 0. 
        \end{align*}
        Hence, we conclude that 
        \[   | {\pi}_{i}(\vec{ {x}_{k} } ) - {\pi }_{i}(\vec{ x } )  | \to 0 \]
        and so \( {\pi}_{i} \) is continuous map with respect to \( {d}_{\infty } \) and the standard metric on \( \R  \).
\end{enumerate}
\end{proof}

\begin{problem}
    \begin{enumerate}
        \item[(i)] Define \( d: \R \times \R \to \R  \) by \( d(x,y) = | e^{x} - e^{y} |  \). Prove that \( d \) is a metric on \( \R  \).
        \item[(ii)] Prove or disprove: \( (\R , d) \) is complete.
    \end{enumerate}
\end{problem}
\begin{proof}
    \begin{enumerate}
        \item[(i)]  \begin{enumerate}
                
    \item[(I)] It follows immediately that \( d(x,y) > 0  \) by the way \( d  \) is defined. Let \( x,y \in \R  \). Then
        \begin{align*}
            d(x,y) = 0 &\iff | e^{x} - e^{y} | = 0  \\
                       &\iff e^{x}= e^{y} \tag{standard metric on \( \R  \)} \\
                       &\iff \ln(e^{x}) = \ln(e^{y}) \\
                       &\iff x = y.
        \end{align*}
        Hence, property (i) is satisfied.
    \item[(II)] We have 
        \[  d(x,y) = | e^{x} - e^{y} |  = | e^{y} - e^{x} |  = d(y,x). \]
        Hence, property (ii) is satisfied.
    \item[(III)] Let \( x,y,z \in \R  \). Then we have
        \begin{align*}
            | e^{x} - e^{y} | &= | e^{x} - e^{z} + e^{z} - e^{y} |  \\
                              &\leq | e^{x} - e^{z} |  + | e^{z} - e^{y} |  \\
                              &= d(x,z) + d(z,y).
        \end{align*}
        Hence, we have \( d(x,y) \leq d(x,z) + d(z,y) \).
\end{enumerate}
From the properties above, we conclude that \( d(x,y) = | e^{x} - e^{y} |  \) does indeed define a metric on \( \R \).
        \item[(ii)] We claim that the metric defined above does NOT make \( \R \) complete. Define the sequence \( {x}_{n} = \ln \Big(  \frac{ 1 }{ n }  \Big) \). It follows immediately with respect to \( d  \) that \( ({x}_{n}) \) is a Cauchy sequence in \( \R  \). Indeed, let \( \epsilon > 0  \). Since \( 1/n \) is a Cauchy sequence in \( \R  \) with respect to the standard metric, there exists an \( N \in \N \) such that for any \( n,m \geq N  \), we have 
            \[  d({x}_{n}, {x}_{m}) = | e^{\ln(1/n)} - e^{\ln(1/m)}  | = \Big| \frac{ 1 }{ n }  - \frac{ 1 }{ m }  \Big| < \epsilon.   \]
            Since \( \R  \) is complete with respect to the standard metric, we can see that \( \frac{ 1 }{ n } \to 0  \), but there does not exists an \( x \in \R  \) such that \( x = \ln(0) \). Hence, \( \R  \) cannot be complete with the metric defined above.
    \end{enumerate}
\end{proof}

\begin{problem}
    Let \( X = \N \) be the set of positive integers.
\begin{enumerate}
    \item[(i)] Let \( d(m,n) = | m - n |  \). Prove that \( (X,d) \) is complete.
    \item[(ii)] Let \( d(m,n) = \Big| \frac{ 1 }{ m }  - \frac{ 1 }{ n }  \Big|  \). Prove that \( (X,d) \) is not complete.
\end{enumerate}
\end{problem}
\begin{proof}
\begin{enumerate}
    \item[(i)] Let \( ({x}_{n}) \) be a Cauchy sequence in \( \N \). Let \( \epsilon > 0  \). Our goal is to find an \( N \in \N  \) such that for any \( n \geq N  \), we have  
        \[  | {x}_{n} - x | < \epsilon \]
        where \( x \in \N \). Let \( \epsilon > 0 \). Since \( ({x}_{n}) \) is a Cauchy sequence in \( \N \), there exists an \( \hat{N} \in \N \) such that for any \( n,m \geq \hat{N} \)
        \[  | {x}_{n} - {x}_{m} | < \epsilon. \]
        Note that since \( {x}_{n}, {x}_{m} \in \N \), we can use \( \hat{N}  \) as the same \( N  \) we were looking for. Indeed, if we fix \( m > \hat{N} \) and suppose that for any \( n \geq \hat{N}  \), we have 
        \[  | {x}_{n} - {x}_{m} | <  \epsilon. \]
        Then clearly, \( {x}_{n} \to x  \) (where \( x = {x}_{m} \in \N \), in this case). Hence, we conclude that \( ({x}_{n}) \to x  \).
\end{enumerate}
\item[(ii)] We claim that \( (X,d) \) is not complete with respect to \( d(m,n) = | \frac{ 1 }{ m }  - \frac{ 1 }{ n }  |  \). Consider the sequence \( {x}_{n} = n \). With respect to the metric above, we can see that \( ({x}_{n}) \) is a Cauchy sequence in \( \N \). Indeed, let \( \epsilon > 0  \). Using the Archimedean Property, we can find an \( \hat{N} \) such that  
    \[  \frac{ 1 }{ \hat{N} } < \frac{ \epsilon }{ 2 }. \]
    If we let \( n,m \geq \hat{N} \), we have 
    \begin{align*}
        \frac{ 1 }{ n } &\leq \frac{ 1 }{ \hat{N} } < \frac{ \epsilon }{ 2 } \tag{1}  \\
        \frac{ 1 }{ m }  &\leq \frac{ 1 }{ \hat{N} }  < \frac{ \epsilon }{ 2 }. \tag{2}
    \end{align*}
    Using (1) and (2) along with the triangle inequality, we can see that 
    \begin{align*}
        \Big| \frac{ 1 }{ n }  - \frac{ 1 }{ m }  \Big|  &\leq \Big| \frac{ 1 }{ n }  \Big| + \Big| \frac{ 1 }{ m }  \Big|   
        < \frac{ 2 }{ \hat{N} } < \epsilon.
    \end{align*}
    Hence, \( ({x}_{n}) \) is a Cauchy sequence, but \( {x}_{n} \to 0  \) (with respect to the metric above) where \( 0  \) is clearly not in \( \N \). Hence, \( \N \) cannot be complete with the above metric.
\end{proof}

\begin{problem}
    Let \( X = \{ f: [0,1] \to \R : f \ \text{is continuous} \}  \).
       \begin{enumerate}
           \item[(i)] Define \( d(f,g) = \displaystyle \int_{ 0 }^{ 1 }  | f(t) - g(t) |  \ dt \). Prove that \( d \) is  a metric on \( X  \). Prove that \( d \) is a metric on \( X  \).
            \item[(ii)] Prove that \( (X,d) \) is not complete.
       \end{enumerate} 
\end{problem}
\begin{proof}
\begin{enumerate}
    \item[(i)] Our goal is to show that \( d  \) is a metric on \( X  \). 
        \begin{enumerate}
            \item[(I)] Let \( f,g \in X  \). Using the fact that the standard metric \( | \cdot |   \) is a nonnegative continuous function on \( [0,1] \) along with the fact given to us, it follows that
                \[  \int_{ 0 }^{ 1 } | f(t) - g(t) |  \ dt = 0 \iff | f(t) - g(t) |  = 0.  \]
                Thus, we have that
                \begin{align*}
                    d(f,g) = 0 &\iff \int_{ 0 }^{ 1 }  | f(t) - g(t) |  \ dt \\
                               &\iff | f(t) - g(t) | = 0 \\
                               &\iff f(t)  = g(t) \tag{\( | \cdot |  \) is a metric on \( \R  \)} \\ 
                \end{align*}
            \item[(II)] Observe that for any \( f,g \in X  \), we see that
                \[  d(f,g) = \int_{ 0 }^{ 1 }  | f(t) - g(t) |  \ dt  = \int_{ 0 }^{ 1 }  | g(t) - f(t) |  \ dt = d(g,f). \]
            \item[(III)] Let \( f,g,h \in X  \). Then by the triangle inequality of the standard metric of \( \R  \) and the linearity of integral, we have
                \begin{align*}
                    d(f,g) &= \int_{ 0 }^{ 1 } | f(t) - g(t) |  \ dt  \\
                           &\leq \int_{ 0 }^{ 1 }  \Big( | f(t) - h(t) |  + | h(t) - g(t) | \Big)  \ dt \\
                           &= \int_{ 0 }^{ 1 }  | f(t) - h(t) |  \ dt + \int_{ 0 }^{ 1 }  | h(t) - g(t) |  \ dt \\
                           &= d(f,h) + d(h,g).
                \end{align*}
                Hence, property (III) is satisfied.
        \end{enumerate}
        Thus, we conclude that \( d  \) does indeed define a metric on \( X  \).
    \item[(ii)] Our goal is to construct a sequence \( ({x}_{n}) \) that is Cauchy in \( X  \), but it does not converge in \( X  \). Based on the area of the triangle, which is represented by \( d({x}_{m}, {x}_{n}) \), found in figure 10 of the book, we can define \( N = \frac{ 1 }{ \epsilon }  \) such that for any \( m,n \geq N \), we have   
        \[  d({x}_{n}, {x}_{m}) < \epsilon. \]
        As a consequence, we can see that \( ({x}_{n}) \) is a Cauchy sequence.
        Now, we want to show that \( ({x}_{n}) \) does not converge in \( X  \). Suppose for sake of contradiction that \( ({x}_{n}) \) does converge in \( X  \). Suppose \( x \in X  \). Indeed, if we define 
        \[  {x}_{m}(t) = 0 \ \text{if} \ t \in [0,1/2] \]
        and 
        \[  {x}_{m}(t) = 1  \ \text{if} \ t \in [{a}_{m},1] \]
        where \( {a}_{m} =\frac{ 1 }{ 2 }  + \frac{ 1 }{ m }  \), we see that  
        \begin{align*}
            d({x}_{m},x) &= \int_{ 0 }^{ 1 }  | {x}_{m}(t) - x(t) |  \ dt \\
                         &= \int_{ 0 }^{ 1/2 }  | x(t) |  \ dt + \int_{ 1/2 }^{ {a}_{m} } | {x}_{m}(t) - x(t) |  \ dt + \int_{ {a}_{m} }^{ 1 }  | 1 - x(t) |  \ dt.
        \end{align*}
        By our integration properties, we can see that  
        \[  d({x}_{m},x) = \int_{ 0 }^{ 1/2 }  | x(t) |  \ dt + \int_{ 1/2 }^{ {a}_{m} } | {x}_{m}(t) - x(t) |  \ dt + \int_{ {a}_{m} }^{ 1 } | 1 - x(t) |  \ dt. \]
        Since each corresponding integrand above is nonnegative, we can see that each integral on the right-hand side is nonnegative. Since \( d({x}_{m},x) \to 0  \) (by assumption), we can see that each integral on the right-hand side above approaches zero. Since \( x(t) \) is a continuous function for all \( t \in [0,1] \), we have that 
        \[ x(t) = 
        \begin{cases}
            0 &\text{if} \ t \in [0,1/2) \\ 
            1 &\text{if} \ t  \in (1/2,1].
        \end{cases} \]
        But note that \( x(t) \) cannot be continuous; that is, \( x \notin X  \). The reason is as follows: if we take a sequence \( ({t}_{n}) \) in the interval \( [0,1/2) \) (that is, take a sequence from the left side), then we see that \( x({t}_{n}) \to 0   \). However, if we take a sequence \( ({r}_{n})  \) in the interval \( (1/2,1] \) (that is, take the right-handed limit), then \( x({r}_{n}) \to 1 \). By the sequential criterion of continuity, we see immediately that \( x(t)  \) cannot be continuous. Hence, we have \( {x}_{m}(t)  \) converges to a limit that does not belong to \( X \). Hence, we conclude that \( X  \) cannot be a complete metric space with the metric \( d  \) defined above.
\end{enumerate}
\end{proof}



\end{document}

