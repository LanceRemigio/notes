\documentclass[a4paper]{article}
\usepackage[utf8]{inputenc}
\usepackage[T1]{fontenc}
\usepackage{textcomp}
\usepackage{hyperref}
% \usepackage{fourier}
% \usepackage[dutch]{babel}
\usepackage{url}
% \usepackage{hyperref}
% \hypersetup{
%     colorlinks,
%     linkcolor={black},
%     citecolor={black},
%     urlcolor={blue!80!black}
% }
\usepackage{graphicx}
\usepackage{float}
\usepackage{booktabs}
\usepackage{enumitem}
% \usepackage{parskip}
\usepackage{emptypage}
\usepackage{subcaption}
\usepackage{multicol}
\usepackage[usenames,dvipsnames]{xcolor}

% \usepackage{cmbright}


\usepackage[margin=1in]{geometry}
\usepackage{amsmath, amsfonts, mathtools, amsthm, amssymb}
\usepackage{mathrsfs}
\usepackage{cancel}
\usepackage{bm}
\newcommand\N{\ensuremath{\mathbb{N}}}
\newcommand\R{\ensuremath{\mathbb{R}}}
\newcommand\Z{\ensuremath{\mathbb{Z}}}
\renewcommand\O{\ensuremath{\emptyset}}
\newcommand\Q{\ensuremath{\mathbb{Q}}}
\newcommand\C{\ensuremath{\mathbb{C}}}
\DeclareMathOperator{\sgn}{sgn}
\usepackage{systeme}
\let\svlim\lim\def\lim{\svlim\limits}
\let\implies\Rightarrow
\let\impliedby\Leftarrow
\let\iff\Leftrightarrow
\let\epsilon\varepsilon
\usepackage{stmaryrd} % for \lightning
\newcommand\contra{\scalebox{1.1}{$\lightning$}}
% \let\phi\varphi
\renewcommand\qedsymbol{$\blacksquare$}




% correct
\definecolor{correct}{HTML}{009900}
\newcommand\correct[2]{\ensuremath{\:}{\color{red}{#1}}\ensuremath{\to }{\color{correct}{#2}}\ensuremath{\:}}
\newcommand\green[1]{{\color{correct}{#1}}}



% horizontal rule
\newcommand\hr{
    \noindent\rule[0.5ex]{\linewidth}{0.5pt}
}


% hide parts
\newcommand\hide[1]{}



% si unitx
\usepackage{siunitx}
\sisetup{locale = FR}
% \renewcommand\vec[1]{\mathbf{#1}}
\newcommand\mat[1]{\mathbf{#1}}


% tikz
\usepackage{tikz}
\usepackage{tikz-cd}
\usetikzlibrary{intersections, angles, quotes, calc, positioning}
\usetikzlibrary{arrows.meta}
\usepackage{pgfplots}
\pgfplotsset{compat=1.13}


\tikzset{
    force/.style={thick, {Circle[length=2pt]}-stealth, shorten <=-1pt}
}

% theorems
\makeatother
\usepackage{thmtools}
\usepackage[framemethod=TikZ]{mdframed}
\mdfsetup{skipabove=1em,skipbelow=0em}


\theoremstyle{definition}

\declaretheoremstyle[
    headfont=\bfseries\sffamily\color{ForestGreen!70!black}, bodyfont=\normalfont,
    mdframed={
        linewidth=2pt,
        rightline=false, topline=false, bottomline=false,
        linecolor=ForestGreen, backgroundcolor=ForestGreen!5,
    }
]{thmgreenbox}

\declaretheoremstyle[
    headfont=\bfseries\sffamily\color{NavyBlue!70!black}, bodyfont=\normalfont,
    mdframed={
        linewidth=2pt,
        rightline=false, topline=false, bottomline=false,
        linecolor=NavyBlue, backgroundcolor=NavyBlue!5,
    }
]{thmbluebox}

\declaretheoremstyle[
    headfont=\bfseries\sffamily\color{NavyBlue!70!black}, bodyfont=\normalfont,
    mdframed={
        linewidth=2pt,
        rightline=false, topline=false, bottomline=false,
        linecolor=NavyBlue
    }
]{thmblueline}

\declaretheoremstyle[
    headfont=\bfseries\sffamily\color{RawSienna!70!black}, bodyfont=\normalfont,
    mdframed={
        linewidth=2pt,
        rightline=false, topline=false, bottomline=false,
        linecolor=RawSienna, backgroundcolor=RawSienna!5,
    }
]{thmredbox}

\declaretheoremstyle[
    headfont=\bfseries\sffamily\color{RawSienna!70!black}, bodyfont=\normalfont,
    numbered=no,
    mdframed={
        linewidth=2pt,
        rightline=false, topline=false, bottomline=false,
        linecolor=RawSienna, backgroundcolor=RawSienna!1,
    },
    qed=\qedsymbol
]{thmproofbox}

\declaretheoremstyle[
    headfont=\bfseries\sffamily\color{NavyBlue!70!black}, bodyfont=\normalfont,
    numbered=no,
    mdframed={
        linewidth=2pt,
        rightline=false, topline=false, bottomline=false,
        linecolor=NavyBlue, backgroundcolor=NavyBlue!1,
    },
]{thmexplanationbox}

\declaretheorem[style=thmgreenbox, numberwithin = section, name=Definition]{definition}
\declaretheorem[style=thmbluebox, name=Example]{eg}
\declaretheorem[style=thmredbox, numberwithin = section, name=Proposition]{prop}
\declaretheorem[style=thmredbox, numberwithin = section, name=Theorem]{theorem}
\declaretheorem[style=thmredbox, numberwithin = section,  name=Lemma]{lemma}
\declaretheorem[style=thmredbox, numberwithin = section,  numbered=no, name=Corollary]{corollary}


\declaretheorem[style=thmproofbox, name=Proof]{replacementproof}
\renewenvironment{proof}[1][\proofname]{\vspace{-10pt}\begin{replacementproof}}{\end{replacementproof}}


\declaretheorem[style=thmexplanationbox, name=Proof]{tmpexplanation}
\newenvironment{explanation}[1][]{\vspace{-10pt}\begin{tmpexplanation}}{\end{tmpexplanation}}


\declaretheorem[style=thmblueline, numbered=no, name=Remark]{remark}
\declaretheorem[style=thmblueline, numbered=no, name=Note]{note}

\newtheorem*{uovt}{UOVT}
\newtheorem*{notation}{Notation}
\newtheorem*{previouslyseen}{As previously seen}
\newtheorem*{problem}{Problem}
\newtheorem*{observe}{Observe}
\newtheorem*{property}{Property}
\newtheorem*{intuition}{Intuition}


\usepackage{etoolbox}
\AtEndEnvironment{vb}{\null\hfill$\diamond$}%
\AtEndEnvironment{intermezzo}{\null\hfill$\diamond$}%
% \AtEndEnvironment{opmerking}{\null\hfill$\diamond$}%

% http://tex.stackexchange.com/questions/22119/how-can-i-change-the-spacing-before-theorems-with-amsthm
\makeatletter
% \def\thm@space@setup{%
%   \thm@preskip=\parskip \thm@postskip=0pt
% }
\newcommand{\oefening}[1]{%
    \def\@oefening{#1}%
    \subsection*{Oefening #1}
}

\newcommand{\suboefening}[1]{%
    \subsubsection*{Oefening \@oefening.#1}
}

\newcommand{\exercise}[1]{%
    \def\@exercise{#1}%
    \subsection*{Exercise #1}
}

\newcommand{\subexercise}[1]{%
    \subsubsection*{Exercise \@exercise.#1}
}


\usepackage{xifthen}

\def\testdateparts#1{\dateparts#1\relax}
\def\dateparts#1 #2 #3 #4 #5\relax{
    \marginpar{\small\textsf{\mbox{#1 #2 #3 #5}}}
}

\def\@lesson{}%
\newcommand{\lesson}[3]{
    \ifthenelse{\isempty{#3}}{%
        \def\@lesson{Lecture #1}%
    }{%
        \def\@lesson{Lecture #1: #3}%
    }%
    \subsection*{\@lesson}
    \testdateparts{#2}
}

% \renewcommand\date[1]{\marginpar{#1}}


% fancy headers
\usepackage{fancyhdr}
\pagestyle{fancy}

\fancyhead[LE,RO]{Lance Remigio}
\fancyhead[RO,LE]{\@lesson}
\fancyhead[RE,LO]{}
\fancyfoot[LE,RO]{\thepage}
\fancyfoot[C]{\leftmark}

\makeatother




% notes
\usepackage{todonotes}
\usepackage{tcolorbox}

\tcbuselibrary{breakable}
\newenvironment{verbetering}{\begin{tcolorbox}[
    arc=0mm,
    colback=white,
    colframe=green!60!black,
    title=Opmerking,
    fonttitle=\sffamily,
    breakable
]}{\end{tcolorbox}}

\newenvironment{noot}[1]{\begin{tcolorbox}[
    arc=0mm,
    colback=white,
    colframe=white!60!black,
    title=#1,
    fonttitle=\sffamily,
    breakable
]}{\end{tcolorbox}}




% figure support
\usepackage{import}
\usepackage{xifthen}
\pdfminorversion=7
\usepackage{pdfpages}
\usepackage{transparent}
\newcommand{\incfig}[1]{%
    \def\svgwidth{\columnwidth}
    \import{./figures/}{#1.pdf_tex}
}

% %http://tex.stackexchange.com/questions/76273/multiple-pdfs-with-page-group-included-in-a-single-page-warning
\pdfsuppresswarningpagegroup=1



\title{Homework 5}
\author{Lance Remigio}
\begin{document}
\maketitle

\begin{problem}
    Let \( (V, \|\cdot\|) \) be a normed space and \( Y  \) be a vector subspace of \( V  \). Last time, we saw that \( V / Y  = \{ v + Y  : v \in V  \}  \) is also a vector space. Now, assume that \( Y  \) is closed in \( (V, \|\cdot\| ) \).
\end{problem}

\begin{enumerate}
    \item[(i)] Let \( v  \) and \( v' \) such that \( v - v' \in Y  \). Show that \( \inf_{y \in Y} \|v + Y\| = \inf_{y \in Y} \|v' + y\|  \).
        \begin{proof}
        From problem 2(i) of Homework 4, \( v - v' \in Y \) implies that \( v + Y = v' + Y  \). Hence, we have 
        \begin{align*}
            v + Y = v' + Y  &\implies  \|v + y \| = \|v' + y\| \ \forall y \in Y  \\
                            &\implies \inf_{y \in Y} \|v + y\| = \inf_{y \in Y} \|v' + y\|.
    \end{align*}
        \end{proof}
    \item[(ii)] For \( [v] = v + Y  \in V/ Y \), define 
        \[  \|[v]\|_0 = \inf_{y \in Y } \|v + y\|. \]
        Show that \( \|\cdot\|_{0} \) defines a norm on \( V / W  \).
        \begin{proof}
            Clearly, we have \( \|[v]\|_{0} \geq 0  \) since \( \| \cdot \|  \) satisfies property (I).
        \begin{enumerate}
            \item[(I)] Suppose \( v + Y = {0}_{V/W}  \) where \( [0] =  {0}_{V/W} = {0}_{V} + Y  \). Then by definition of \( \|\cdot\|_{0} \), we have \( \|[0]\|_{0} = 0  \). From part (a), we have
                \begin{align*}
                    \|[v]\|_{0} = \|[0]\|_{0}  &\iff \inf_{y \in Y} \|  v + y \| = 0   \\
                                               &\iff \|[v]\|_{0} = 0. 
                \end{align*}
                Hence, the property (I) is satisfied.
            \item[(II)] Let \( \alpha \in F \) where \( F  \) is a field. Then we have 
                \begin{align*}
                    \|[\alpha v] \|_{0} &= \|\alpha v + Y\|_{0} \\
                                        &= \inf_{y \in Y} \|\alpha v + Y \| \\
                                        &= \inf_{y \in Y} \| \alpha (v + Y )\| \\
                                        &= \inf_{y \in Y} | \alpha | \| v + Y \| \\
                                        &= | \alpha |  \inf_{y \in Y} \|v + Y\| \tag{\( \|\cdot\| \) is a norm} \\
                                        &=  | \alpha |  \|[v]\|_{0}.
                \end{align*}
            \item[(III)] Let \( {v}_{1}, {v}_{2} \in V / W  \). Then since \( \|\cdot\| \) is a norm, we have that  
                \begin{align*}
                    \|[{v}_{1} + {v}_{2}]\|_{0} &= \|({v}_{1} + {v}_{2}) + Y \|_{0} \\
                                                &=  \| ({v}_{1} + Y) + ({v}_{2} + Y) \|_0 \\
                                                &= \inf_{y \in Y} \| ({v}_{1} + {y}_{1})  + ({v}_{2} + {y}_{2}) \| \\
                                                &\leq \inf_{y \in Y} \Big[ \| {v}_{1} + {y}_{1} \| + \| {v}_{2} + {y}_{2} \| \Big] \\
                                                &= \inf_{y \in Y} \|{v}_{1} + {y}_{1}\| + \inf_{y \in Y} \|{v}_{2} + {y}_{2}\| \\
                                                &= \|[{v}_{1}]\|_{0} + \|[{v}_{2}]\|_{0}.
                \end{align*}
        \end{enumerate}
        \end{proof}
    \item[(iii)] For any \( v \in V  \), show that \( \|[v]\|_{0} \leq \|u\| \).
        \begin{proof}
            By the triangle inequality, we have 
            \[ \|v\| =   \|v \| + \|{0}_{Y}\| \geq \| v + {0}_{Y} \| \geq \inf_{y \in Y} \|v + Y \| = \|[v]\|_{0}.  \]
        \end{proof}
    \item[(iv)] We have a Canonical map \( \pi : V \to V \setminus  Y  \), \( \pi(u) = [u] \). Show that \( \pi  \) is linear and continuous. Here continuity means that if \( \|v_n - v \| \to 0  \) in \( V  \), then \( \|[{v}_{n}] - [v]\|_{0} \to 0  \) in \( V  / W  \).
        \begin{proof}
        First, we show that \( \pi  \) is linear. For any \( {u}_{1}, {u}_{2} \in V  \), we have 
        \begin{align*}
            \pi({u}_{1} + {u}_{1}) &= [{u}_{1} + {u}_{2}] \\
                                   &= ({u}_{1} + {u}_{2}) + Y  \\
                                   &= ({u}_{1} + Y) + ({u}_{2} + Y) \\
                                   &= [{u}_{1}] + [{u}_{2}] \\
                                   &= \pi({u}_{1}) + \pi({u}_{2}).
        \end{align*} 
        
        Let \( \alpha \in F  \) where \( F  \) is a field and let \( u \in V  \). Then we have
        \[  \pi( \alpha u ) = [\alpha u] = (\alpha u) + Y = \alpha (u + Y ) = \alpha [u] = \alpha \pi(u). \]
        Hence, we conclude that \( \pi  \) is a linear map from \( V \to V / Y \).
        Now, we want to show that \( \pi  \) is, indeed, continuous for any \( v \in V  \). Let \( {v}_{n}  \) be a sequence in \(  V   \) such that \( {v}_{n} \to v  \); that is, \( \| {v}_{n} - v  \| \to 0  \). Our goal is to show that \( \|[{v}_{n}] - [v] \| \to 0 \). By part (b), we can see that 
        \[  0 \leq \|[{v}_{n}] - [v] \|_{0} \leq \|{v}_{n} - v \| \to 0.  \]
        Hence, we have \( \|[{v}_{n}] \to [v]\|_{0} \to 0  \) by Squeeze Theorem and so we conclude that \( \pi: V \to V / Y  \) is a continuous function on \( V  \).
        \end{proof}
\end{enumerate}

\begin{problem}
    Consider the normed space \( (\ell^{\infty }, \|\cdot\|_{\infty }) \). Define a sequence \( e^{(n)} \) in \( \ell^{\infty } \) by  
    \[  x^{(n)} = ({\delta}_{j}^{(n)}), \ \ {\delta}_{j}^{(n)} = 
    \begin{cases}
        1 &\text{if} \ j = n \\
        0 &\text{otherwise}. 
    \end{cases} \]
\end{problem}

\begin{enumerate}
    \item[(i)] Compute \( \|x^{(n)} - x^{(n')}\|_{\infty} \) for \( n \neq n' \).
        \begin{solution}
        Let \( n > m  \). Then we see that 
        \[  {\delta}_{j}^{(m)} = (0,0, \dots, \underbrace{1}_{j=m}, 0, \dots) \]
        and 
        \[  {\delta}_{j}^{(n)} =  (0,0, \dots, \underbrace{1}_{j=n}, 0, \dots)\]
        Then we have 
        \[ {\delta}_{j}^{(n)} - {\delta}_{j}^{(m)} = 
        \begin{cases}
            1 &\text{if} \ j = n \\
            -1 &\text{if} \ j = m \\
            0 &\text{if} \ j \neq n,m
        \end{cases}  \]
        Clearly, we can see that \( \|x^{(n)} - x^{(m)} \|_{\infty} = 1  \).
        \end{solution}
    \item[(ii)] Does \( (x^{(n)}) \) have a convergent subsequence?
        \begin{proof}
        We claim that \( x^{(n)} \) does not have a convergent subsequence. Indeed, from part (i), we can see that 
        \[  \|x^{(n)} - x^{(m)}\|_{\infty } \geq \frac{ 1 }{ 2 } \ \forall n \neq m  \]
        and 
        \[  \|x^{(n)}\|_{\infty } = 1 \] 
        for all \( n \in \N \). By a theorem proven in class, we can see that \( x^{(n)} \) does not have a convergent subsequence.
        \end{proof}
    \item[(iii)] Prove that \( \mathcal{S}(\ell^{\infty }) = \{ x = ({x}_{j}) \in \ell^{\infty } : \|x\|_{\infty } = 1  \}  \) is closed and bounded but not compact.
        \begin{proof}
            It follows immediately that \( S(\ell^{\infty }) \) is bounded by construction. We will show that \( S(\ell^{\infty }) \) is closed. Let \( x \in \overline{S(\ell^{\infty })} \). Then there exists a sequence \( x^{(n)} \) in \( S(\ell^{\infty }) \) such that \( x^{(n)} \to x  \) for some \( x \). This implies that \( x^{(n)} \) is a Cauchy sequence in \( \ell^{\infty } \). Since \( \ell^{\infty } \) is a Banach space, it follows that \( x^{(n)} \to y  \) for some \( y \in \ell^{\infty } \). Our goal is to show that \( y \in s(\ell^{\infty }) \); that is, we want to show that \( \|y\|_{\infty } = 1 \). Using the triangle inequality , we can see that 
            \begin{align*}
                \|y\|_{\infty } &\leq \|y - x^{(n)}\|_{\infty} + \|x^{(n)}\|_{\infty }  \\
                                &= \|y - x^{(n)}\|_{\infty } + 1.
        \end{align*} 
        Hence, we have 
        \[  \|y\|_{\infty } - 1 \leq \|y - x^{(n)}\|_{\infty }. \tag{1}\]
        Similarly, we have 
        \begin{align*}
            \|x^{(n)}\|_{\infty } &\leq \| x^{(n)} - y \|_{\infty } + \|y\|_{\infty }
        \end{align*}
        and so, 
        \[  1 - \|y\|_{\infty } \leq \|x^{(n)} - y \|_{\infty }. \tag{2} \]
        Now, (1) and (2) imply that 
        \[  0 \leq | \|y\|_{\infty } - 1 | \leq \|x^{(n)} - y\|_{\infty } \to 0.  \]
        Since \( | \cdot |  \) and \( \|\cdot\|_{\infty } \) are continuous functions, we have 
        \[  \lim_{ n \to \infty  }  | \|y\|_{\infty } - 1 | = \Big| \lim_{ n \to \infty  } (\|y\|_{\infty } - 1)   \Big| =   0   \]
        and so we conclude that \( \|y\|_{\infty} = 1 \) which proves that \( S(\ell^{\infty }) \) closed. But note that by part (ii), \( x^{(n)} \) does not have convergent subsequence. By Sequential Compactness, it follows that \( S(\ell^{\infty}) \) is not a compact set.  
        \end{proof}
\end{enumerate}

\begin{problem}
    Let \( (V,\|\cdot\|)  \) be a normed space and \( Y  \) be a subspace of \( V  \) such that \( Y \neq V  \). Let \( v \in V \setminus  Y  \). Define \( d(v,Y) = \inf_{y \in Y } \|v-y\| \).
\end{problem}

\begin{enumerate}
    \item[(i)] Show that if \( d(v,Y) = 0  \), then \( v \in \overline{Y} \).
        \begin{proof}
            Suppose that \( d(v,Y) = 0  \). Our goal is to show that \( v \in \overline{Y} \); that is, we want to show that for any \( \epsilon > 0  \), \( B(y,\epsilon) \cap Y  \neq \emptyset \). Let \( \epsilon > 0  \) be given. By a characterization of the infimum, we know there exists \( \hat{y} \in Y  \) such that  
            \[  \|v - \hat{y} \| < \inf_{y \in Y} \|v - y\| + \epsilon. \]
            By assumption, \( d(v,Y) = \inf_{y \in Y} \|v - Y\| = 0  \) and so we have, from the above inequality that
            \[  \|v - \hat{y} \| < \epsilon. \]
            Hence, \( \hat{y} \in B(v,\epsilon)  \). Since \( \hat{y} \in Y  \), we can conclude that 
            \[  B(v,\epsilon) \cap Y \neq \emptyset  \]
            and so \( \hat{y} \in \overline{Y} \).
        \end{proof}
    \item[(ii)] Assume that \( Y  \) is closed. Prove that \( d > 0  \). 
        \begin{proof}
            Suppose that \( Y  \) is closed. Our goal is to show that \( d(v,Y) > 0  \) for all \( v \in V \setminus  Y  \). To this end, let \( v \in V \setminus  Y  \). Suppose for sake of contradiction that \( d(v,Y) \leq 0  \). If \( d(v,Y) < 0  \), then we have \( \|v - y \| < 0  \) which is absurd. If \( f(v,Y) = 0  \), then from part (a) we have that \( v \in \overline{Y} \). But \( Y  \) is closed and so \( Y = \overline{Y} \). This tell us that \( v \in  Y  \) which contradicts our assumption that \( v \in V \setminus  Y  \).
        \end{proof}
\end{enumerate}

\begin{problem}
    Read section 2.6 of Kryszig and write down statements of key theorems, lemmas, and propositions.
\end{problem}

\begin{theorem}[Range and Null Space]
    Let \( T  \) be a linear operator. Then:
    \begin{enumerate}
        \item[(a)] The range \( R(T) \) is a vector space.
        \item[(b)] If \( \text{dim}(T) = n < \infty  \), then \( \text{dim}(R(T)) \leq n  \).
        \item[(c)] The null space \( N(T) \) is a vector space.
    \end{enumerate}
\end{theorem}

\begin{theorem}[Inverse Operator]
    Let \( X , Y  \) be vector spaces, both real or both complex. Let \( T : D(T) \to Y  \) be  a linear operator with domain \( D(T) \subseteq  X  \) and range \( R(T) \subseteq  Y  \). Then: 
    \begin{enumerate}
        \item[(a)] The inverse \( T^{-1}: R(T) \to D(T) \) exists if and only if
            \[  T(x) = 0 \implies x = 0.  \]
        \item[(b)] If \( T^{-1} \) exists, it is a linear operator.
        \item[(c)] If \( \text{dim}(D(T)) = n < \infty  \) and \( T^{-1} \) exists, then \( R(T) = \text{dim}(D(T)) \).
    \end{enumerate}
\end{theorem}

\begin{lemma}[Inverse of Product]
   Let \( T: X \to Y  \) and \( S: Y \to Z  \) be bijective linear operators, where \( X,Y,Z \) are vector spaces. Then the inverse \( (ST)^{-1} : Z \to X  \) of the product (the comoposite) \( ST \) exists, and  
   \[  (ST)^{-1} = T^{-1} S^{-1}. \]
\end{lemma}


\begin{problem}[i]
         Let \( T : D(T) \to W  \) be a linear operator. Assume that \( T^{-1} : R(T) \to D(T) \) exists. Show that if \( \{ {v}_{1}, \dots, {v}_{n} \}  \) is linearly dependent on \( D(T) \), then \( \{ T ({v}_{1}), \dots, T ({v}_{n}) \}  \) is linearly dependent on \( W  \).
\end{problem}
\begin{proof}
We will show the claim through contrapositive. Suppose \( \{ T({v}_{1}), T({v}_{2}), \dots, T({v}_{n}) \}  \) is linearly independent; that is, the equation
\[  \sum_{ i=1  }^{ n } {c}_{i} T({v}_{i}) = 0  \tag{*} \]
has the trivial solution \( {c}_{i} = 0   \) for all \( 1 \leq i \leq n  \). Since \( T  \) is linear, (*) implies that  
\[  T \Big(  \sum_{ i=1  }^{ n } {c}_{i} {v}_{i} \Big) = 0.  \]
Note that \( T^{-1} \) exists and so \( T  \) must be injective (and surjective). Hence, we have \( N(T) = \{ 0  \}  \). Thus, we have 
\[  \sum_{ i=1  }^{ n } {c}_{i} {v}_{i} \in N(T) \implies \sum_{ i=1  }^{ n } {c}_{i} {v}_{i} = 0.   \]
But then \( {c}_{i} = 0  \) for all \( 1 \leq i \leq n  \). Hence, we see that \( \{ {v}_{1} , \dots, {v}_{n} \}  \) is a linearly independent set on \( V  \).
\end{proof}

\begin{problem}[ii]
        Let \( V  \) and \( W  \) be two vector spaces and \( T : V \to W  \) be a linear operator. Assume that \( T: V \to W  \) be a linear operator. Assume that \( V  \) and \( W  \) are finite dimensional and \( \text{dim}(V) = \text{dim}(W) \). Prove that \( R(T) = W  \) if and only if \( T^{-1} \) exists.

\end{problem}
\begin{proof}
    Assume that \( V  \) and \( W  \) are finite dimensional and \( \text{dim}(V) \) and \( \text{dim}(W) \).

\( (\Longrightarrow) \) Suppose \( R(T) = W  \). Let \( \text{dim}(V) = \text{dim}(W) = n \). Our goal is to show that \( T^{-1} \) exists. It suffices to show that \( T  \) is both surjective and injective. Note that, by assumption, \( T  \) is immediately surjective. So, it suffices to show that \( T  \) is injective. Let \( \beta = \{ {v}_{1}, {v}_{2}, \dots, {v}_{n} \}  \) and \( \omega = \{ {w}_{1}, {w}_{2}, \dots, {w}_{n} \} \). Since \( T  \) is surjective, we get \( T({v}_{i}) = {w}_{i} \) for \( 1 \leq i \leq n  \). Let \( x,y \in V \). Then since \( \beta \) is a basis, we have   
\begin{align*}
    x &= \sum_{ i=1  }^{ n } {c}_{i} {v}_{i}, \\
    y &= \sum_{ i=1  }^{ n } {b}_{i} {v}_{i}
\end{align*}
Suppose \( T(x) = T(y) \). Then by the linearity and surjective of \( T  \), we see that
\begin{align*}
    T(x) = T(y) &\implies T \Big(  \sum_{ i=1  }^{ n } {c}_{i} {v}_{i}  \Big) = T \Big(  \sum_{ i=1  }^{ n } {b}_{i} {v}_{i} \Big) \\
                &\implies \sum_{ i=1  }^{ n } {c}_{i} T({v}_{i}) = \sum_{ i=1  }^{ n } {b}_{i} T({v}_{i}) \\
                &\implies \sum_{ i=1  }^{ n } ({c}_{i} - {b}_{i}) T({v}_{i}) = 0  \\
                &\implies \sum_{ i=1  }^{ n } ({c}_{i} - {b}_{i}) {w}_{i} = 0.
\end{align*}
Since \( \omega \) is basis for \( W  \), \( {w}_{i}  \) for all \( 1 \leq i \leq n  \) are linearly independent. Hence, \( {c}_{i}  - {b}_{i} = 0  \) for all \( 1 \leq i \leq n  \). Hence, \( {c}_{i} = {b}_{i} \) for all \( 1 \leq i \leq n  \). This tells us that \( x = y \). Thus, \( T  \) must be injective. Thus, \( T^{-1}  \) must exists.

\( (\Longleftarrow) \) If \( T^{-1} \) exists, then \( T  \) must be a bijective map between \( V  \) and \( W  \). Hence, we immediately have that \( R(T) = W  \). 


\end{proof}




\end{document}

