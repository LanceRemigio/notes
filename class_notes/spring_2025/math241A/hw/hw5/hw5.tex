\documentclass[a4paper]{article}
\usepackage[utf8]{inputenc}
\usepackage[T1]{fontenc}
% \usepackage{fourier}
\usepackage{textcomp}
\usepackage{hyperref}
\usepackage[english]{babel}
\usepackage{url}
% \usepackage{hyperref}
% \hypersetup{
%     colorlinks,
%     linkcolor={black},
%     citecolor={black},
%     urlcolor={blue!80!black}
% }
\usepackage{graphicx} \usepackage{float}
\usepackage{booktabs}
\usepackage{enumitem}
% \usepackage{parskip}
% \usepackage{parskip}
\usepackage{emptypage}
\usepackage{subcaption}
\usepackage{multicol}
\usepackage[usenames,dvipsnames]{xcolor}
\usepackage{ocgx}
% \usepackage{cmbright}


\usepackage[margin=1in]{geometry}
\usepackage{amsmath, amsfonts, mathtools, amsthm, amssymb}
\usepackage{thmtools}
\usepackage{mathrsfs}
\usepackage{cancel}
\usepackage{bm}
\newcommand\N{\ensuremath{\mathbb{N}}}
\newcommand\R{\ensuremath{\mathbb{R}}}
\newcommand\Z{\ensuremath{\mathbb{Z}}}
\renewcommand\O{\ensuremath{\emptyset}}
\newcommand\Q{\ensuremath{\mathbb{Q}}}
\newcommand\C{\ensuremath{\mathbb{C}}}
\newcommand\F{\ensuremath{\mathbb{F}}}
\DeclareMathOperator{\sgn}{sgn}
\DeclareMathOperator{\diam}{diam}
\DeclareMathOperator{\LO}{LO}
\DeclareMathOperator{\UP}{UP}
\DeclareMathOperator{\card}{card}
\DeclareMathOperator{\Arg}{Arg}
\DeclareMathOperator{\Dom}{Dom}
\DeclareMathOperator{\Log}{Log}
\DeclareMathOperator{\dist}{dist}
% \DeclareMathOperator{\span}{span}
\usepackage{systeme}
\let\svlim\lim\def\lim{\svlim\limits}
\renewcommand\implies\Longrightarrow
\let\impliedby\Longleftarrow
\let\iff\Longleftrightarrow
\let\epsilon\varepsilon
\usepackage{stmaryrd} % for \lightning
\newcommand\contra{\scalebox{1.1}{$\lightning$}}
% \let\phi\varphi
\renewcommand\qedsymbol{$\blacksquare$}

% correct
\definecolor{correct}{HTML}{009900}
\newcommand\correct[2]{\ensuremath{\:}{\color{red}{#1}}\ensuremath{\to }{\color{correct}{#2}}\ensuremath{\:}}
\newcommand\green[1]{{\color{correct}{#1}}}

% horizontal rule
\newcommand\hr{
    \noindent\rule[0.5ex]{\linewidth}{0.5pt}
}

% hide parts
\newcommand\hide[1]{}

% si unitx
\usepackage{siunitx}
\sisetup{locale = FR}
% \renewcommand\vec[1]{\mathbf{#1}}
\newcommand\mat[1]{\mathbf{#1}}

% tikz
\usepackage{tikz}
\usepackage{tikz-cd}
\usetikzlibrary{intersections, angles, quotes, calc, positioning}
\usetikzlibrary{arrows.meta}
\usepackage{pgfplots}
\pgfplotsset{compat=1.13}

\tikzset{
    force/.style={thick, {Circle[length=2pt]}-stealth, shorten <=-1pt}
}

% theorems
\makeatother
\usepackage{thmtools}
\usepackage[framemethod=TikZ]{mdframed}
\mdfsetup{skipabove=1em,skipbelow=1em}

\theoremstyle{definition}

\declaretheoremstyle[
    headfont=\bfseries\sffamily\color{ForestGreen!70!black}, bodyfont=\normalfont,
    mdframed={
        linewidth=1pt,
        rightline=false, topline=false, bottomline=false,
        linecolor=ForestGreen, backgroundcolor=ForestGreen!5,
    }
]{thmgreenbox}

\declaretheoremstyle[
    headfont=\bfseries\sffamily\color{NavyBlue!70!black}, bodyfont=\normalfont,
    mdframed={
        linewidth=1pt,
        rightline=false, topline=false, bottomline=false,
        linecolor=NavyBlue, backgroundcolor=NavyBlue!5,
    }
]{thmbluebox}

\declaretheoremstyle[
    headfont=\bfseries\sffamily\color{NavyBlue!70!black}, bodyfont=\normalfont,
    mdframed={
        linewidth=1pt,
        rightline=false, topline=false, bottomline=false,
        linecolor=NavyBlue
    }
]{thmblueline}

\declaretheoremstyle[
    headfont=\bfseries\sffamily, bodyfont=\normalfont,
    numbered = no,
    mdframed={
        rightline=true, topline=true, bottomline=true,
    }
]{thmbox}

\declaretheoremstyle[
    headfont=\bfseries\sffamily, bodyfont=\normalfont,
    numbered=no,
    % mdframed={
    %     rightline=true, topline=false, bottomline=true,
    % },
    qed=\qedsymbol
]{thmproofbox}

\declaretheoremstyle[
    headfont=\bfseries\sffamily\color{NavyBlue!70!black}, bodyfont=\normalfont,
    numbered=no,
    mdframed={
        rightline=false, topline=false, bottomline=false,
        linecolor=NavyBlue, backgroundcolor=NavyBlue!1,
    },
]{thmexplanationbox}

\declaretheorem[
    style=thmbox, 
    % numberwithin = section,
    numbered = no,
    name=Definition
    ]{definition}

\declaretheorem[
    style=thmbox, 
    name=Example,
    ]{eg}

\declaretheorem[
    style=thmbox, 
    % numberwithin = section,
    name=Proposition]{prop}

\declaretheorem[
    style = thmbox,
    numbered=yes,
    name =Problem
    ]{problem}

\declaretheorem[style=thmbox, name=Theorem]{theorem}
\declaretheorem[style=thmbox, name=Lemma]{lemma}
\declaretheorem[style=thmbox, name=Corollary]{corollary}

\declaretheorem[style=thmproofbox, name=Proof]{replacementproof}

\declaretheorem[style=thmproofbox, 
                name = Solution
                ]{replacementsolution}

\renewenvironment{proof}[1][\proofname]{\vspace{-1pt}\begin{replacementproof}}{\end{replacementproof}}

\newenvironment{solution}
    {
        \vspace{-1pt}\begin{replacementsolution}
    }
    { 
            \end{replacementsolution}
    }

\declaretheorem[style=thmexplanationbox, name=Proof]{tmpexplanation}
\newenvironment{explanation}[1][]{\vspace{-10pt}\begin{tmpexplanation}}{\end{tmpexplanation}}

\declaretheorem[style=thmbox, numbered=no, name=Remark]{remark}
\declaretheorem[style=thmbox, numbered=no, name=Note]{note}

\newtheorem*{uovt}{UOVT}
\newtheorem*{notation}{Notation}
\newtheorem*{previouslyseen}{As previously seen}
% \newtheorem*{problem}{Problem}
\newtheorem*{observe}{Observe}
\newtheorem*{property}{Property}
\newtheorem*{intuition}{Intuition}

\usepackage{etoolbox}
\AtEndEnvironment{vb}{\null\hfill$\diamond$}%
\AtEndEnvironment{intermezzo}{\null\hfill$\diamond$}%
% \AtEndEnvironment{opmerking}{\null\hfill$\diamond$}%

% http://tex.stackexchange.com/questions/22119/how-can-i-change-the-spacing-before-theorems-with-amsthm
\makeatletter
% \def\thm@space@setup{%
%   \thm@preskip=\parskip \thm@postskip=0pt
% }
\newcommand{\oefening}[1]{%
    \def\@oefening{#1}%
    \subsection*{Oefening #1}
}

\newcommand{\suboefening}[1]{%
    \subsubsection*{Oefening \@oefening.#1}
}

\newcommand{\exercise}[1]{%
    \def\@exercise{#1}%
    \subsection*{Exercise #1}
}

\newcommand{\subexercise}[1]{%
    \subsubsection*{Exercise \@exercise.#1}
}


\usepackage{xifthen}

\def\testdateparts#1{\dateparts#1\relax}
\def\dateparts#1 #2 #3 #4 #5\relax{
    \marginpar{\small\textsf{\mbox{#1 #2 #3 #5}}}
}

\def\@lesson{}%
\newcommand{\lesson}[3]{
    \ifthenelse{\isempty{#3}}{%
        \def\@lesson{Lecture #1}%
    }{%
        \def\@lesson{Lecture #1: #3}%
    }%
    \subsection*{\@lesson}
    \testdateparts{#2}
}

% \renewcommand\date[1]{\marginpar{#1}}


% fancy headers
\usepackage{fancyhdr}
\pagestyle{fancy}

\makeatother

% notes
\usepackage{todonotes}
\usepackage{tcolorbox}

\tcbuselibrary{breakable}
\newenvironment{verbetering}{\begin{tcolorbox}[
    arc=0mm,
    colback=white,
    colframe=green!60!black,
    title=Opmerking,
    fonttitle=\sffamily,
    breakable
]}{\end{tcolorbox}}

\newenvironment{noot}[1]{\begin{tcolorbox}[
    arc=0mm,
    colback=white,
    colframe=white!60!black,
    title=#1,
    fonttitle=\sffamily,
    breakable
]}{\end{tcolorbox}}

% figure support
\usepackage{import}
\usepackage{xifthen}
\pdfminorversion=7
\usepackage{pdfpages}
\usepackage{transparent}
\newcommand{\incfig}[1]{%
    \def\svgwidth{\columnwidth}
    \import{./figures/}{#1.pdf_tex}
}

% %http://tex.stackexchange.com/questions/76273/multiple-pdfs-with-page-group-included-in-a-single-page-warning
\pdfsuppresswarningpagegroup=1


\title{Homework 5}
\author{Lance Remigio}
\begin{document}
\maketitle

\begin{problem}
    Let \( (V, \|\cdot\|) \) be a normed space and \( Y  \) be a vector subspace of \( V  \). Last time, we saw that \( V / Y  = \{ v + Y  : v \in V  \}  \) is also a vector space. Now, assume that \( Y  \) is closed in \( (V, \|\cdot\| ) \).
\end{problem}

\begin{enumerate}
    \item[(i)] Let \( v  \) and \( v' \) such that \( v - v' \in Y  \). Show that \( \inf_{y \in Y} \|v + Y\| = \inf_{y \in Y} \|v' + y\|  \).
        \begin{proof}
        From problem 2(i) of Homework 4, \( v - v' \in Y \) implies that \( v + Y = v' + Y  \). Hence, we have 
        \begin{align*}
            v + Y = v' + Y  &\implies  \|v + y \| = \|v' + y\| \ \forall y \in Y  \\
                            &\implies \inf_{y \in Y} \|v + y\| = \inf_{y \in Y} \|v' + y\|.
    \end{align*}
        \end{proof}
    \item[(ii)] For \( [v] = v + Y  \in V/ Y \), define 
        \[  \|[v]\|_0 = \inf_{y \in Y } \|v + y\|. \]
        Show that \( \|\cdot\|_{0} \) defines a norm on \( V / W  \).
        \begin{proof}
            Clearly, we have \( \|[v]\|_{0} \geq 0  \) since \( \| \cdot \|  \) satisfies property (I).
        \begin{enumerate}
            \item[(I)] Suppose \( v + Y = {0}_{V/W}  \) where \( [0] =  {0}_{V/W} = {0}_{V} + Y  \). Then by definition of \( \|\cdot\|_{0} \), we have \( \|[0]\|_{0} = 0  \). From part (a), we have
                \begin{align*}
                    \|[v]\|_{0} = \|[0]\|_{0}  &\iff \inf_{y \in Y} \|  v + y \| = 0   \\
                                               &\iff \|[v]\|_{0} = 0. 
                \end{align*}
                Hence, the property (I) is satisfied.
            \item[(II)] Let \( \alpha \in F \) where \( F  \) is a field. Then we have 
                \begin{align*}
                    \|[\alpha v] \|_{0} &= \|\alpha v + Y\|_{0} \\
                                        &= \inf_{y \in Y} \|\alpha v + Y \| \\
                                        &= \inf_{y \in Y} \| \alpha (v + Y )\| \\
                                        &= \inf_{y \in Y} | \alpha | \| v + Y \| \\
                                        &= | \alpha |  \inf_{y \in Y} \|v + Y\| \tag{\( \|\cdot\| \) is a norm} \\
                                        &=  | \alpha |  \|[v]\|_{0}.
                \end{align*}
            \item[(III)] Let \( {v}_{1}, {v}_{2} \in V / W  \). Then since \( \|\cdot\| \) is a norm, we have that  
                \begin{align*}
                    \|[{v}_{1} + {v}_{2}]\|_{0} &= \|({v}_{1} + {v}_{2}) + Y \|_{0} \\
                                                &=  \| ({v}_{1} + Y) + ({v}_{2} + Y) \|_0 \\
                                                &= \inf_{y \in Y} \| ({v}_{1} + {y}_{1})  + ({v}_{2} + {y}_{2}) \| \\
                                                &\leq \inf_{y \in Y} \Big[ \| {v}_{1} + {y}_{1} \| + \| {v}_{2} + {y}_{2} \| \Big] \\
                                                &= \inf_{y \in Y} \|{v}_{1} + {y}_{1}\| + \inf_{y \in Y} \|{v}_{2} + {y}_{2}\| \\
                                                &= \|[{v}_{1}]\|_{0} + \|[{v}_{2}]\|_{0}.
                \end{align*}
        \end{enumerate}
        \end{proof}
    \item[(iii)] For any \( v \in V  \), show that \( \|[v]\|_{0} \leq \|u\| \).
        \begin{proof}
            By the triangle inequality, we have 
            \[ \|v\| =   \|v \| + \|{0}_{Y}\| \geq \| v + {0}_{Y} \| \geq \inf_{y \in Y} \|v + Y \| = \|[v]\|_{0}.  \]
        \end{proof}
    \item[(iv)] We have a Canonical map \( \pi : V \to V \setminus  Y  \), \( \pi(u) = [u] \). Show that \( \pi  \) is linear and continuous. Here continuity means that if \( \|v_n - v \| \to 0  \) in \( V  \), then \( \|[{v}_{n}] - [v]\|_{0} \to 0  \) in \( V  / W  \).
        \begin{proof}
        First, we show that \( \pi  \) is linear. For any \( {u}_{1}, {u}_{2} \in V  \), we have 
        \begin{align*}
            \pi({u}_{1} + {u}_{1}) &= [{u}_{1} + {u}_{2}] \\
                                   &= ({u}_{1} + {u}_{2}) + Y  \\
                                   &= ({u}_{1} + Y) + ({u}_{2} + Y) \\
                                   &= [{u}_{1}] + [{u}_{2}] \\
                                   &= \pi({u}_{1}) + \pi({u}_{2}).
        \end{align*} 
        \end{proof}
\end{enumerate}

\end{document}

