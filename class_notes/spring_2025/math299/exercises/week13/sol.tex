\documentclass[a4paper]{article}
\usepackage[utf8]{inputenc}
\usepackage[T1]{fontenc}
% \usepackage{fourier}
\usepackage{textcomp}
\usepackage{hyperref}
\usepackage[english]{babel}
\usepackage{url}
% \usepackage{hyperref}
% \hypersetup{
%     colorlinks,
%     linkcolor={black},
%     citecolor={black},
%     urlcolor={blue!80!black}
% }
\usepackage{graphicx} \usepackage{float}
\usepackage{booktabs}
\usepackage{enumitem}
% \usepackage{parskip}
% \usepackage{parskip}
\usepackage{emptypage}
\usepackage{subcaption}
\usepackage{multicol}
\usepackage[usenames,dvipsnames]{xcolor}
\usepackage{ocgx}
% \usepackage{cmbright}


\usepackage[margin=1in]{geometry}
\usepackage{amsmath, amsfonts, mathtools, amsthm, amssymb}
\usepackage{thmtools}
\usepackage{mathrsfs}
\usepackage{cancel}
\usepackage{bm}
\newcommand\N{\ensuremath{\mathbb{N}}}
\newcommand\R{\ensuremath{\mathbb{R}}}
\newcommand\Z{\ensuremath{\mathbb{Z}}}
\renewcommand\O{\ensuremath{\emptyset}}
\newcommand\Q{\ensuremath{\mathbb{Q}}}
\newcommand\C{\ensuremath{\mathbb{C}}}
\newcommand\F{\ensuremath{\mathbb{F}}}
\DeclareMathOperator{\sgn}{sgn}
\DeclareMathOperator{\diam}{diam}
\DeclareMathOperator{\LO}{LO}
\DeclareMathOperator{\UP}{UP}
\DeclareMathOperator{\card}{card}
\DeclareMathOperator{\Arg}{Arg}
\DeclareMathOperator{\Dom}{Dom}
\DeclareMathOperator{\Log}{Log}
\DeclareMathOperator{\dist}{dist}
% \DeclareMathOperator{\span}{span}
\usepackage{systeme}
\let\svlim\lim\def\lim{\svlim\limits}
\renewcommand\implies\Longrightarrow
\let\impliedby\Longleftarrow
\let\iff\Longleftrightarrow
\let\epsilon\varepsilon
\usepackage{stmaryrd} % for \lightning
\newcommand\contra{\scalebox{1.1}{$\lightning$}}
% \let\phi\varphi
\renewcommand\qedsymbol{$\blacksquare$}

% correct
\definecolor{correct}{HTML}{009900}
\newcommand\correct[2]{\ensuremath{\:}{\color{red}{#1}}\ensuremath{\to }{\color{correct}{#2}}\ensuremath{\:}}
\newcommand\green[1]{{\color{correct}{#1}}}

% horizontal rule
\newcommand\hr{
    \noindent\rule[0.5ex]{\linewidth}{0.5pt}
}

% hide parts
\newcommand\hide[1]{}

% si unitx
\usepackage{siunitx}
\sisetup{locale = FR}
% \renewcommand\vec[1]{\mathbf{#1}}
\newcommand\mat[1]{\mathbf{#1}}

% tikz
\usepackage{tikz}
\usepackage{tikz-cd}
\usetikzlibrary{intersections, angles, quotes, calc, positioning}
\usetikzlibrary{arrows.meta}
\usepackage{pgfplots}
\pgfplotsset{compat=1.13}

\tikzset{
    force/.style={thick, {Circle[length=2pt]}-stealth, shorten <=-1pt}
}

% theorems
\makeatother
\usepackage{thmtools}
\usepackage[framemethod=TikZ]{mdframed}
\mdfsetup{skipabove=1em,skipbelow=1em}

\theoremstyle{definition}

\declaretheoremstyle[
    headfont=\bfseries\sffamily\color{ForestGreen!70!black}, bodyfont=\normalfont,
    mdframed={
        linewidth=1pt,
        rightline=false, topline=false, bottomline=false,
        linecolor=ForestGreen, backgroundcolor=ForestGreen!5,
    }
]{thmgreenbox}

\declaretheoremstyle[
    headfont=\bfseries\sffamily\color{NavyBlue!70!black}, bodyfont=\normalfont,
    mdframed={
        linewidth=1pt,
        rightline=false, topline=false, bottomline=false,
        linecolor=NavyBlue, backgroundcolor=NavyBlue!5,
    }
]{thmbluebox}

\declaretheoremstyle[
    headfont=\bfseries\sffamily\color{NavyBlue!70!black}, bodyfont=\normalfont,
    mdframed={
        linewidth=1pt,
        rightline=false, topline=false, bottomline=false,
        linecolor=NavyBlue
    }
]{thmblueline}

\declaretheoremstyle[
    headfont=\bfseries\sffamily, bodyfont=\normalfont,
    numbered = no,
    mdframed={
        rightline=true, topline=true, bottomline=true,
    }
]{thmbox}

\declaretheoremstyle[
    headfont=\bfseries\sffamily, bodyfont=\normalfont,
    numbered=no,
    % mdframed={
    %     rightline=true, topline=false, bottomline=true,
    % },
    qed=\qedsymbol
]{thmproofbox}

\declaretheoremstyle[
    headfont=\bfseries\sffamily\color{NavyBlue!70!black}, bodyfont=\normalfont,
    numbered=no,
    mdframed={
        rightline=false, topline=false, bottomline=false,
        linecolor=NavyBlue, backgroundcolor=NavyBlue!1,
    },
]{thmexplanationbox}

\declaretheorem[
    style=thmbox, 
    % numberwithin = section,
    numbered = no,
    name=Definition
    ]{definition}

\declaretheorem[
    style=thmbox, 
    name=Example,
    ]{eg}

\declaretheorem[
    style=thmbox, 
    % numberwithin = section,
    name=Proposition]{prop}

\declaretheorem[
    style = thmbox,
    numbered=yes,
    name =Problem
    ]{problem}

\declaretheorem[style=thmbox, name=Theorem]{theorem}
\declaretheorem[style=thmbox, name=Lemma]{lemma}
\declaretheorem[style=thmbox, name=Corollary]{corollary}

\declaretheorem[style=thmproofbox, name=Proof]{replacementproof}

\declaretheorem[style=thmproofbox, 
                name = Solution
                ]{replacementsolution}

\renewenvironment{proof}[1][\proofname]{\vspace{-1pt}\begin{replacementproof}}{\end{replacementproof}}

\newenvironment{solution}
    {
        \vspace{-1pt}\begin{replacementsolution}
    }
    { 
            \end{replacementsolution}
    }

\declaretheorem[style=thmexplanationbox, name=Proof]{tmpexplanation}
\newenvironment{explanation}[1][]{\vspace{-10pt}\begin{tmpexplanation}}{\end{tmpexplanation}}

\declaretheorem[style=thmbox, numbered=no, name=Remark]{remark}
\declaretheorem[style=thmbox, numbered=no, name=Note]{note}

\newtheorem*{uovt}{UOVT}
\newtheorem*{notation}{Notation}
\newtheorem*{previouslyseen}{As previously seen}
% \newtheorem*{problem}{Problem}
\newtheorem*{observe}{Observe}
\newtheorem*{property}{Property}
\newtheorem*{intuition}{Intuition}

\usepackage{etoolbox}
\AtEndEnvironment{vb}{\null\hfill$\diamond$}%
\AtEndEnvironment{intermezzo}{\null\hfill$\diamond$}%
% \AtEndEnvironment{opmerking}{\null\hfill$\diamond$}%

% http://tex.stackexchange.com/questions/22119/how-can-i-change-the-spacing-before-theorems-with-amsthm
\makeatletter
% \def\thm@space@setup{%
%   \thm@preskip=\parskip \thm@postskip=0pt
% }
\newcommand{\oefening}[1]{%
    \def\@oefening{#1}%
    \subsection*{Oefening #1}
}

\newcommand{\suboefening}[1]{%
    \subsubsection*{Oefening \@oefening.#1}
}

\newcommand{\exercise}[1]{%
    \def\@exercise{#1}%
    \subsection*{Exercise #1}
}

\newcommand{\subexercise}[1]{%
    \subsubsection*{Exercise \@exercise.#1}
}


\usepackage{xifthen}

\def\testdateparts#1{\dateparts#1\relax}
\def\dateparts#1 #2 #3 #4 #5\relax{
    \marginpar{\small\textsf{\mbox{#1 #2 #3 #5}}}
}

\def\@lesson{}%
\newcommand{\lesson}[3]{
    \ifthenelse{\isempty{#3}}{%
        \def\@lesson{Lecture #1}%
    }{%
        \def\@lesson{Lecture #1: #3}%
    }%
    \subsection*{\@lesson}
    \testdateparts{#2}
}

% \renewcommand\date[1]{\marginpar{#1}}


% fancy headers
\usepackage{fancyhdr}
\pagestyle{fancy}

\makeatother

% notes
\usepackage{todonotes}
\usepackage{tcolorbox}

\tcbuselibrary{breakable}
\newenvironment{verbetering}{\begin{tcolorbox}[
    arc=0mm,
    colback=white,
    colframe=green!60!black,
    title=Opmerking,
    fonttitle=\sffamily,
    breakable
]}{\end{tcolorbox}}

\newenvironment{noot}[1]{\begin{tcolorbox}[
    arc=0mm,
    colback=white,
    colframe=white!60!black,
    title=#1,
    fonttitle=\sffamily,
    breakable
]}{\end{tcolorbox}}

% figure support
\usepackage{import}
\usepackage{xifthen}
\pdfminorversion=7
\usepackage{pdfpages}
\usepackage{transparent}
\newcommand{\incfig}[1]{%
    \def\svgwidth{\columnwidth}
    \import{./figures/}{#1.pdf_tex}
}

% %http://tex.stackexchange.com/questions/76273/multiple-pdfs-with-page-group-included-in-a-single-page-warning
\pdfsuppresswarningpagegroup=1


\title{Exercises}
\author{Lance Remigio}
\begin{document}
\maketitle

\section{Regular Curves}

\begin{problem}
    Show that it is not possible to parametrize the cissoid of Diocles 
    \[  x (x^{2} + y^{2}) = 2 R y^{2} \]
    so that it is regular at the origin.
\end{problem}
\begin{proof}
Let \( x = r \cos \theta   \) and \( y = r \sin \theta  \). Then the cissoid of Diocles can be parametrized via polar coordinates:
\begin{align*}
    &x(x^{2} + y^{2}) = 2 R y^{2} \\
    &\implies r \cos \theta \cdot r^{2} = 2 R \cdot r^{2} \sin^{2} \theta \\
    &\implies r^{3} \cos \theta = 2 R \cdot r^{2} \sin^{2} \theta \\
    &\implies r = 2 R \sin \theta \tan \theta. 
\end{align*}
Now, for \( - \frac{ \pi }{ 2 }  < \theta < \frac{ \pi  }{ 2  }  \), we obtain
\begin{align*}
    x &= 2 R \sin \theta \tan \theta = 2 R \sin^{2} \theta \\
    y &= 2 R \sin \theta \tan \theta \cdot \sin \theta \\
      &= 2 R \frac{ \sin^{3} \theta  }{  \cos \theta  }.
\end{align*}
Note that if we let \( t = \sin \theta  \), we get 
\[  q(t) = 2 R (x(t), y(y)) = 2 R \Big(  t^{2}, \frac{ t^{3} }{ \sqrt{ 1 - t^{2} }  }  \Big). \]
Differentiating, we get
\begin{align*}
    \dot{x}(t) &= 4 R t  \\
    \dot{y}(t) &= 2 R \frac{ 3 t^{2} - 2 t^{4} }{ (1 - t^{3})^{3/2} }.
\end{align*}
and so we have 
\[  | \dot{q}(t) |  = 2 R \sqrt{  4t^{2} + \frac{ 3t^{2} - 2t^{4} }{ (1 - t^{2})^{3/2} }  }. \]
But note that if \( {t}_{0} = 0  \), then \( | \dot{q}(t) |  = 0  \) which tells us that the cissoid of Diocles is not regular at the origin.
\end{proof}




\section{Curvature}

\begin{problem}
    Let \( q(t) = r(t) (\cos t , \sin t ) \). Show that the speed is given by
    \[  \Big(  \frac{ ds  }{ dt }  \Big)^{2} = \Big(  \frac{ dr }{ dt }  \Big)^{2} + r^{2} \]
    and the curvature 
    \[  \kappa = \frac{ \Big| 2 (\frac{ dr }{ dt })^{2} + r^{2} - r \frac{ d^{2} r  }{ d t^{2} }   \Big|  }{  ((\frac{ dr }{ dt } )^{2} + r^{2})^{3/2} }. \]
\end{problem}
\begin{proof}
    To find the curvature of \( q  \), it suffices to use the following formula: 
    \[  \kappa = \frac{ | \dot{x} \ddot{y} - \ddot{x} \dot{y} |  }{ | \dot{q} |^{3} }. \]
    Differentiating \( q(t) \) once, we obtain
    \begin{align*}
        \dot{q}(t) &= \dot{r} (\cos t , \sin t ) + r(- \sin t , \cos t)  \\
                   &= (\dot{r} \cos t - r \sin t , \dot{r} \sin t + r \cos t ) 
    \end{align*}
    and so the magnitude of \( \dot{q}(t)  \) is given by:
    \begin{align*}
        \dot{s}(t) =  |\dot{q}(t)| &= \sqrt{ (\dot{r} \cos t - r \sin t)^{2} + (\dot{r} \sin t + r \cos t )^{2} }  \\
                       &= \sqrt{ \dot{r}^{2} \cos^{2}t - 2 r \dot{r} \sin t \cos t + 2 r \dot{r} \sin t \cos t + r^{2} \sin^{2}t  + \dot{r}^{2} \sin^{2} t + r^{2} \cos^{2}t  } \\
                       &= \sqrt{ \dot{r}^{2} + r^{2} }.
    \end{align*}
    But this implies that 
    \[  |\dot{q}(t)|^{2} = \dot{r}^{2} + r^{2}. \]
    Now, notice that 
    \begin{align*}
        \dot{x} \ddot{y} &= (\dot{r} \cos t - r \sin t )(2 \dot{r} \cos t + (\ddot{r} - r)\sin t) \\
                         &= 2 \dot{r}^{2} + \dot{r} \ddot{r} \sin t \cos t - \dot{r} r \sin t \cos t - 2 \dot{r} r \sin t \cos t - r \ddot{r} \sin^{2}t + r^{2} \sin^{2} t
    \end{align*}
    and
    \begin{align*}
        \ddot{x} \dot{y} &= ((\ddot{r} - r) \cos t - 2 \dot{r} \sin t) (\dot{r} \sin t + r \cos t) \\
                         &= \ddot{r} \dot{r} \sin t \cos t - \dot{R} r \sin t \cos t - 2 \dot{r}^{2} \sin^{2} t + r \ddot{r} \cos^{2} t - r^{2} \cos^{2}t - 2 \dot{r} r \sin t \cos t. 
    \end{align*}
    Subtracting \( \dot{x} \ddot{y}   \) and \( \ddot{x} \dot{y} \), we get that 
    \[  2 \dot{r}^{2} + r^{2} - r \ddot{r} \]
    and so the curvature of \( q  \) is just
    \begin{align*}
        \kappa &= \frac{ | \dot{x} \ddot{y} - \dot{x} \dot{y} | }{ | \dot{q} |^{3} }  \\
               &= \frac{ | 2 \dot{r}^{2} + r^{2} - r \ddot{r} |  }{ (\dot{r}^{2} + r^{2})^{3/2} }
    \end{align*}
    which is our desired result.

\end{proof}



\begin{problem}
    Compute the curvature of the logarithmic spiral
    \[ a e^{b t} (\cos t , \sin t). \]
\end{problem}
\begin{solution}
Let \( q(t) = a e^{bt} (\cos t, \sin t) \). To compute the curvature of the logarithmic spiral, we first need to differentiate \( q(t)  \) twice. Indeed, we have 
\begin{align*}
    \dot{q}(t) &= ab e^{bt } (\cos t, \sin t) + a e^{bt } (-\sin t, \cos t) \\
               &= ae^{bt } (b \cos t - \sin t, b \sin t + \cos t )
\end{align*}
and so 
\begin{align*}
    \| \dot{q}(t) \| &= a e^{bt } \sqrt{ (b \cos t  - \sin t )^{2} + (b \sin t + \cos t)^{2} }   \\
        &= a e^{bt } \sqrt{ b^{2} + 1  }.
\end{align*}
Now, differentiating one more time, we get 
\begin{align*}  \ddot{q}(t) &= ab e^{bt } (-b \sin t - \cos t , b \cos t - \sin t ) + a e^{bt } (-n \sin t - \cos t , b \cos t - \sin t )  \\
    &= a e^{bt } ((b^{2} - 1) \cos t - 2b \sin t , (b^{2} - 1 ) \sin t + 2b \cos t).
\end{align*}
Since curvature is given by the following formula
\[  \kappa = \frac{ | \dot{x} \ddot{y} - \ddot{x} \dot{y} |  }{ | \dot{q} |^{3}  },  \]
we can obtain the following 
\begin{align*}
    \dot{x} \ddot{y} &= a e^{bt } (b(b^{2} -1)\sin t \cos t + 2 b^{2} \cos^{2} t - (b^{2} - 1) \sin^{2} t - 2b \sin t \cos t  )  \\
    \dot{y} \ddot{x} &= ae^{bt } (b (b^{2} -1) \sin t \cos t - 2 b^{2} \sin^{2} t + (b^{2} - 1) \cos^{2}t - 2b \sin t \cos t ).
\end{align*}
Hence, we have the following
\[  \dot{x} \ddot{y} - \dot{y} \ddot{x} = a^{2} e^{2bt} (b^{2} + 1) \]
and so our curvature is just 
\[  \kappa = \frac{ | \dot{x} \ddot{y} - \dot{y} \ddot{x} |  }{ | \dot{q} |^{3}  }  = \frac{ e^{-bt} }{ a \sqrt{ b^{2} + 1 }  }. \]
\end{solution}

\section{Regular Surfaces}

\begin{definition}[Regular Surfaces]
    A subset \( S \subseteq \R^{3} \) is a regular surface if, for each \( p \in S  \), there exists a neighborhood \( V  \) in \( \R^{3} \) and a map \( x: U \to V \cap S  \) of an open set \( U \subseteq  \R^{2} \) onto \( V \cap S \subseteq \R^{3}  \) such that 
    \begin{enumerate}
        \item[(1)] \( \textbf{x}   \) is differentiable; that is, the functions \( x(u,v)  \), \( y(u,v) \), and \( z(u,v) \) have continuous partial derivatives of all orders in \( U  \).
        \item[(2)] \( \textbf{x} \) is a homeomorphism; that is, \( \textbf{x} \) contains an inverse \( \textbf{x}^{-1}: V \cap S \to U  \) which is continuous.
        \item[(3)] For each \( q \in U  \), the differential \( d \textbf{x}_q : \R^{2} \to \R^{3} \) is injective.
    \end{enumerate}
\end{definition}


\begin{problem}
    Show that the cylinder \( C = \{ (x,y,z) : x^{2} + y^{2} = 1  \}  \) is a regular surface.
\end{problem}
\begin{proof}
Define the following parametrization for \( C  \) by \( x = \cos (u) \), \( y = \sin (u) \), and \( z = v  \) for every \( (u,v) \in U  \) where \( U = (a,b) \times (c,d) \subseteq \R^{2} \) is an open set.
We will show that this parametrization of \( C  \) satisfies properties (1) through (3) of the definition above. 
Indeed, we have 
\begin{enumerate}
    \item[(1)] Clearly, we see that \( \cos(u) \), \( \sin(u) \), and \( v  \) are functions that have derivatives of all order. Hence, \( \textbf{x}(u,v) = (\cos(u), \sin(u), v)  \) contains derivatives of all order on \( C \cap U  \). 
    \item[(2)] We can see that the component functions contain their respective inverses on \( C \cap U  \); that is, \( \cos^{-1}(x) = u  \), \( \sin^{-1}(y) = v   \), and \( z = v  \). Thus, \(  \textbf{x}^{-1} \) exists and is continuous (since their respective components are also continuous).
    \item[(3)] Note that  
        \[  \frac{ \partial \textbf{x} }{ \partial u  }  = (\cos(u), \sin(u), v) \]
        and
            \[  \frac{ \partial \textbf{x} }{ \partial v  }  = (0,0,1). \]
            Hence, we have 
            \[  \text{d} \textbf{X}_{(u,v)} = \begin{pmatrix} - \sin u & 0 \\ \cos u & 0 \\ 0 & 1  \end{pmatrix}. \]
            Notice that the matrix above has rank \( 2  \) and thus \( \text{d} \textbf{x}_{(u,v)} \) must be an injective linear map.
\end{enumerate}
Since properties (1)-(3) are satisfied, it follows that \( C  \) is a regular surface.
\end{proof}

\section{Gauss Map and Fundamental Properties}

\begin{problem}
    Show that the mean curvature \( H  \) at \( p \in S  \) is given by
    \[  H = \frac{ 1 }{ \pi }  \int_{ 0 }^{ \pi  }  {k}_{n}(\theta) \ d\theta \tag{*} \] where \( {k}_{n}(\theta) \) is the normal curvature at \( p  \) along a direction making an angle \( \theta  \) with a fixed direction.
\end{problem}
\begin{proof}
Let \( {k}_{1} \) and \( {k}_{2} \) be the maximum normal curvature and minimum normal curvature, respectively. To show (*), it suffices to show that 
\[  \frac{ 1 }{ \pi  }  \int_{ 0  }^{ \pi  }  {k }_{n}(\theta) \ d\theta = \frac{ {k}_{1} + {k}_{2}  }{ 2  }. \]
Choose \( \{ {e}_{1}, {e}_{2} \}   \) as our basis for \( {T}_{p}(S) \). Then for any \( v \in {T}_{p}(S) \), we have 
\[  v = {e}_{1} \cos \theta + {e}_{2} \sin \theta  \]
where \( \theta  \) is a fixed direction. Using the Second Fundamental Form, we have
\[  {k}_{n} = {k}_{1} \cos^{2}\theta + {k}_{2} \sin^{2} \theta.  \]
Hence, we can see that 
\begin{align*}
    \frac{ 1 }{ \pi  }  \int_{ 0  }^{ \pi  }  {k }_{n}(\theta) \ d\theta &= \frac{ 1 }{ \pi  }  \int_{ 0 }^{ \pi  } [{k}_{1} \cos^{2} \theta + {k}_{2} \sin^{2} \theta ]  \ d\theta  \\
                                                                         &= \frac{ 1 }{ \pi  }  \int_{ 0 }^{ \pi  }  {k }_{1} \cos^{2} \theta  \ d \theta + \frac{ 1 }{ \pi  }  \int_{ 0 }^{  \pi  } {k}_{2} \sin^{2} \theta  \ d \theta \\
                                                                         &= \frac{ {k}_{1}  }{ 2 \pi  }  \int_{ 0 }^{ \pi  }  [1 + \cos 2 \theta] \ d\theta + \frac{ {k}_{2}  }{ 2 \pi  }  \int_{ 0  }^{  \pi  }  [1 - \cos 2 \theta ] \ d \theta \\ 
                                                                         &=\frac{ {k}_{1} }{ 2 \pi  }  \Big[ \pi + \frac{ 1 }{ 2 } \sin 4 \pi \Big] + \frac{ {k }_{2}  }{  2 \pi  } \Big[\pi - \frac{ 1 }{ 2 } \sin 4 \pi \Big] \\
                                                                         &= \frac{ {k}_{1} + {k}_{2}  }{  2  } \\
                                                                         &= H.
\end{align*}
Thus, we have 
\[  H = \frac{ 1 }{ \pi  }  \int_{ 0  }^{  \pi  }  {k}_{n}(\theta) \ d \theta. \]
\end{proof}

\begin{problem}
    Show that if \( H \equiv 0  \) on \( S  \) and \( S  \) has no planar points, then the Gauss map \( N : S \to \S^{2} \) has the following property:
    \[  \langle d {N}_{p} ({w}_{1}) , d {N}_{p} ({w}_{2}) \rangle = - K(p) \langle {w}_{1}  ,  {w}_{2} \rangle \]
    for all \( p \in S  \) and all \( {w}_{1}, {w}_{2} \in {T}_{p}(S) \).
\end{problem}
\begin{proof}
Note that since \( H \equiv  0  \), then \( {k}_{1} = - {k}_{2} \). Also, since \( S  \) has no planar points \( d {N}_{p} \neq 0  \) for all \( p \in S  \). Choose an orthonormal eigenbasis \( \{ {e}_{1}, {e}_{2} \}  \). Then for \( {w}_{1}, {w}_{2} \in {T}_{p}(S) \), we have 
\[  {w}_{1} = {e}_{1} \cos \theta + {e}_{2} \sin \theta \tag{1}  \]
and
\[  {w}_{2} = {e}_{1} \cos \phi + {e}_{2} \sin \phi. \tag{2} \]
where \( \phi  \) and \( \theta  \) are angles formed with \( {e}_{1} \). Hence, we have 
\begin{align*}
    d {N}_{p} ({w}_{1}) &= {e}_{1} {k}_{1} \cos \theta + {e}_{2} {k}_{2} \sin \theta   \\
    d {N}_{p} ({w}_{2}) &= {e}_{1} {k}_{1} \cos \phi + {e}_{2} {k}_{2} \sin \phi.
\end{align*}
Then we have 
\begin{align*}
    - K(p) \langle {w}_{1}  , {w}_{2} \rangle &= - {k}_{1} {k}_{2} \langle {w}_{1} ,  {w}_{2} \rangle  \\
                                              &=  \langle - {k}_{1} {w}_{1}  , {k}_{2} {w }_{2} \rangle \\
                                              &= \langle - {k}_{1} [{e}_{1} \cos \theta + {e}_{2} \sin \theta ]  ,  {k}_{2} [ {e}_{1} \cos \phi + {e}_{2} \sin \phi ] \rangle \\
                                              &= \langle {k}_{2} {e}_{1} \cos \theta + {k}_{2} {e}_{2} \sin \theta ,  {k}_{2} {e}_{1} \cos \phi + {k}_{2} {e}_{2} \sin \phi \rangle \\
                                              &= \langle d {N}_{p}({w}_{1}) , d {N}_{p}({w}_{2}) \rangle.
\end{align*}
Hence, we conclude that 
\[  \langle d {N}_{p}({w}_{1}) ,  d {N}_{p} ({w}_{2}) \rangle = - K(p) \langle {w}_{1}   ,  {w}_{2} \rangle. \]
\end{proof}


\end{document}

