\documentclass[a4paper]{article}
\usepackage[utf8]{inputenc}
\usepackage[T1]{fontenc}
% \usepackage{fourier}
\usepackage{textcomp}
\usepackage{hyperref}
\usepackage[english]{babel}
\usepackage{url}
% \usepackage{hyperref}
% \hypersetup{
%     colorlinks,
%     linkcolor={black},
%     citecolor={black},
%     urlcolor={blue!80!black}
% }
\usepackage{graphicx} \usepackage{float}
\usepackage{booktabs}
\usepackage{enumitem}
% \usepackage{parskip}
% \usepackage{parskip}
\usepackage{emptypage}
\usepackage{subcaption}
\usepackage{multicol}
\usepackage[usenames,dvipsnames]{xcolor}
\usepackage{ocgx}
% \usepackage{cmbright}


\usepackage[margin=1in]{geometry}
\usepackage{amsmath, amsfonts, mathtools, amsthm, amssymb}
\usepackage{thmtools}
\usepackage{mathrsfs}
\usepackage{cancel}
\usepackage{bm}
\newcommand\N{\ensuremath{\mathbb{N}}}
\newcommand\R{\ensuremath{\mathbb{R}}}
\newcommand\Z{\ensuremath{\mathbb{Z}}}
\renewcommand\O{\ensuremath{\emptyset}}
\newcommand\Q{\ensuremath{\mathbb{Q}}}
\newcommand\C{\ensuremath{\mathbb{C}}}
\newcommand\F{\ensuremath{\mathbb{F}}}
\DeclareMathOperator{\sgn}{sgn}
\DeclareMathOperator{\diam}{diam}
\DeclareMathOperator{\LO}{LO}
\DeclareMathOperator{\UP}{UP}
\DeclareMathOperator{\card}{card}
\DeclareMathOperator{\Arg}{Arg}
\DeclareMathOperator{\Dom}{Dom}
\DeclareMathOperator{\Log}{Log}
\DeclareMathOperator{\dist}{dist}
% \DeclareMathOperator{\span}{span}
\usepackage{systeme}
\let\svlim\lim\def\lim{\svlim\limits}
\renewcommand\implies\Longrightarrow
\let\impliedby\Longleftarrow
\let\iff\Longleftrightarrow
\let\epsilon\varepsilon
\usepackage{stmaryrd} % for \lightning
\newcommand\contra{\scalebox{1.1}{$\lightning$}}
% \let\phi\varphi
\renewcommand\qedsymbol{$\blacksquare$}

% correct
\definecolor{correct}{HTML}{009900}
\newcommand\correct[2]{\ensuremath{\:}{\color{red}{#1}}\ensuremath{\to }{\color{correct}{#2}}\ensuremath{\:}}
\newcommand\green[1]{{\color{correct}{#1}}}

% horizontal rule
\newcommand\hr{
    \noindent\rule[0.5ex]{\linewidth}{0.5pt}
}

% hide parts
\newcommand\hide[1]{}

% si unitx
\usepackage{siunitx}
\sisetup{locale = FR}
% \renewcommand\vec[1]{\mathbf{#1}}
\newcommand\mat[1]{\mathbf{#1}}

% tikz
\usepackage{tikz}
\usepackage{tikz-cd}
\usetikzlibrary{intersections, angles, quotes, calc, positioning}
\usetikzlibrary{arrows.meta}
\usepackage{pgfplots}
\pgfplotsset{compat=1.13}

\tikzset{
    force/.style={thick, {Circle[length=2pt]}-stealth, shorten <=-1pt}
}

% theorems
\makeatother
\usepackage{thmtools}
\usepackage[framemethod=TikZ]{mdframed}
\mdfsetup{skipabove=1em,skipbelow=1em}

\theoremstyle{definition}

\declaretheoremstyle[
    headfont=\bfseries\sffamily\color{ForestGreen!70!black}, bodyfont=\normalfont,
    mdframed={
        linewidth=1pt,
        rightline=false, topline=false, bottomline=false,
        linecolor=ForestGreen, backgroundcolor=ForestGreen!5,
    }
]{thmgreenbox}

\declaretheoremstyle[
    headfont=\bfseries\sffamily\color{NavyBlue!70!black}, bodyfont=\normalfont,
    mdframed={
        linewidth=1pt,
        rightline=false, topline=false, bottomline=false,
        linecolor=NavyBlue, backgroundcolor=NavyBlue!5,
    }
]{thmbluebox}

\declaretheoremstyle[
    headfont=\bfseries\sffamily\color{NavyBlue!70!black}, bodyfont=\normalfont,
    mdframed={
        linewidth=1pt,
        rightline=false, topline=false, bottomline=false,
        linecolor=NavyBlue
    }
]{thmblueline}

\declaretheoremstyle[
    headfont=\bfseries\sffamily, bodyfont=\normalfont,
    numbered = no,
    mdframed={
        rightline=true, topline=true, bottomline=true,
    }
]{thmbox}

\declaretheoremstyle[
    headfont=\bfseries\sffamily, bodyfont=\normalfont,
    numbered=no,
    % mdframed={
    %     rightline=true, topline=false, bottomline=true,
    % },
    qed=\qedsymbol
]{thmproofbox}

\declaretheoremstyle[
    headfont=\bfseries\sffamily\color{NavyBlue!70!black}, bodyfont=\normalfont,
    numbered=no,
    mdframed={
        rightline=false, topline=false, bottomline=false,
        linecolor=NavyBlue, backgroundcolor=NavyBlue!1,
    },
]{thmexplanationbox}

\declaretheorem[
    style=thmbox, 
    % numberwithin = section,
    numbered = no,
    name=Definition
    ]{definition}

\declaretheorem[
    style=thmbox, 
    name=Example,
    ]{eg}

\declaretheorem[
    style=thmbox, 
    % numberwithin = section,
    name=Proposition]{prop}

\declaretheorem[
    style = thmbox,
    numbered=yes,
    name =Problem
    ]{problem}

\declaretheorem[style=thmbox, name=Theorem]{theorem}
\declaretheorem[style=thmbox, name=Lemma]{lemma}
\declaretheorem[style=thmbox, name=Corollary]{corollary}

\declaretheorem[style=thmproofbox, name=Proof]{replacementproof}

\declaretheorem[style=thmproofbox, 
                name = Solution
                ]{replacementsolution}

\renewenvironment{proof}[1][\proofname]{\vspace{-1pt}\begin{replacementproof}}{\end{replacementproof}}

\newenvironment{solution}
    {
        \vspace{-1pt}\begin{replacementsolution}
    }
    { 
            \end{replacementsolution}
    }

\declaretheorem[style=thmexplanationbox, name=Proof]{tmpexplanation}
\newenvironment{explanation}[1][]{\vspace{-10pt}\begin{tmpexplanation}}{\end{tmpexplanation}}

\declaretheorem[style=thmbox, numbered=no, name=Remark]{remark}
\declaretheorem[style=thmbox, numbered=no, name=Note]{note}

\newtheorem*{uovt}{UOVT}
\newtheorem*{notation}{Notation}
\newtheorem*{previouslyseen}{As previously seen}
% \newtheorem*{problem}{Problem}
\newtheorem*{observe}{Observe}
\newtheorem*{property}{Property}
\newtheorem*{intuition}{Intuition}

\usepackage{etoolbox}
\AtEndEnvironment{vb}{\null\hfill$\diamond$}%
\AtEndEnvironment{intermezzo}{\null\hfill$\diamond$}%
% \AtEndEnvironment{opmerking}{\null\hfill$\diamond$}%

% http://tex.stackexchange.com/questions/22119/how-can-i-change-the-spacing-before-theorems-with-amsthm
\makeatletter
% \def\thm@space@setup{%
%   \thm@preskip=\parskip \thm@postskip=0pt
% }
\newcommand{\oefening}[1]{%
    \def\@oefening{#1}%
    \subsection*{Oefening #1}
}

\newcommand{\suboefening}[1]{%
    \subsubsection*{Oefening \@oefening.#1}
}

\newcommand{\exercise}[1]{%
    \def\@exercise{#1}%
    \subsection*{Exercise #1}
}

\newcommand{\subexercise}[1]{%
    \subsubsection*{Exercise \@exercise.#1}
}


\usepackage{xifthen}

\def\testdateparts#1{\dateparts#1\relax}
\def\dateparts#1 #2 #3 #4 #5\relax{
    \marginpar{\small\textsf{\mbox{#1 #2 #3 #5}}}
}

\def\@lesson{}%
\newcommand{\lesson}[3]{
    \ifthenelse{\isempty{#3}}{%
        \def\@lesson{Lecture #1}%
    }{%
        \def\@lesson{Lecture #1: #3}%
    }%
    \subsection*{\@lesson}
    \testdateparts{#2}
}

% \renewcommand\date[1]{\marginpar{#1}}


% fancy headers
\usepackage{fancyhdr}
\pagestyle{fancy}

\makeatother

% notes
\usepackage{todonotes}
\usepackage{tcolorbox}

\tcbuselibrary{breakable}
\newenvironment{verbetering}{\begin{tcolorbox}[
    arc=0mm,
    colback=white,
    colframe=green!60!black,
    title=Opmerking,
    fonttitle=\sffamily,
    breakable
]}{\end{tcolorbox}}

\newenvironment{noot}[1]{\begin{tcolorbox}[
    arc=0mm,
    colback=white,
    colframe=white!60!black,
    title=#1,
    fonttitle=\sffamily,
    breakable
]}{\end{tcolorbox}}

% figure support
\usepackage{import}
\usepackage{xifthen}
\pdfminorversion=7
\usepackage{pdfpages}
\usepackage{transparent}
\newcommand{\incfig}[1]{%
    \def\svgwidth{\columnwidth}
    \import{./figures/}{#1.pdf_tex}
}

% %http://tex.stackexchange.com/questions/76273/multiple-pdfs-with-page-group-included-in-a-single-page-warning
\pdfsuppresswarningpagegroup=1


\title{Homework 7}
\author{Lance Remigio}
\begin{document}
\maketitle

\begin{problem}
    Let \( (V, \|\cdot\|) \) be an infinite dimensional normed space.
\end{problem}

\begin{enumerate}
    \item[(i)] Assume that \( (V, \|\cdot\|) \) is Banach.
\end{enumerate}


\begin{problem}[Extra Credit]
   Let \( (V, \|\cdot\|) \) be a normed space in which for any sequence \( ({v}_{n}) \) in \( V  \)  
   \[  \sum_{ n=1  }^{ \infty  } \|{v}_{n}\| < \infty \implies \sum_{ n=1  }^{ \infty  } {v}_{n} \ \text{converges in} \ V.  \]
   Prove that \( (V , \|\cdot\|) \) is Banach.
\end{problem}
\begin{proof}
Suppose that every absolutely convergent series is convergent. Our goal is to show that \( (V, \|\cdot\|) \) is a Banach space. To do this, we will show that every Cauchy sequence in \( V  \) converges. Let \( ({v}_{n}) \) be a Cauchy sequence in \( V  \). From here, our strategy is to find a subsequence \( ({v}_{{n}_{k}}) \) of \( ({v}_{n}) \) such that \( ({v}_{{n}_{k}}) \) converges in \( V  \) (by the {\hyperref[lemma]{lemma}}). By definition, \( ({v}_{n}) \) being Cauchy implies that for all \( \epsilon > 0  \), there exists an \( N \in \N \) such that for any \( n > m > N  \), we have 
\[  \|{v}_{n} - {v}_{m}\| < \epsilon. \]
For \( \epsilon = 1 \), there exists an \( {n}_{1} \in \N \) such that for any \( n > m > {n}_{1} \), we have 
\[  \|{v}_{n} - {v}_{m}\| < 1. \]
Furthermore, if \( \epsilon = \frac{ 1 }{ 2 }   \). So, there exists an \( {n}_{2} > {n}_{1} \) by the Archimedean Property such that for any \( n > m > {n}_{2} \), we have 
\[  \|{v}_{n} - {v}_{m}\| < \frac{ 1 }{ 2 }. \] 
In particular, if \( \epsilon = \frac{ 1 }{ 2^{k-1} }   \) for all \( k \in \N \), then we can find an \( {n}_{k} \in \N \) 
such that for any \( n > m > {n}_{k}   \), we have 
\[  \|{v}_{n} - {v}_{m}\| < \frac{ 1 }{ 2^{k-1} }. \]
Moreover, by the Archimedean Property we can find an \( {n}_{k+1} \in \N    \) such that \( {n}_{k+1} > {n}_{k} > {n}_{k-1} \). Hence, it follows that \( ({v}_{{n}_{k}} ) \) is a subsequence in \( V  \) such that
\[   0 \leq  \|{v}_{{n}_{k+1}} - {v}_{{n}_{k}} \| < \frac{ 1 }{ 2^{k-1} }. \tag{*} \]
Note that since \( \sum_{ k=1  }^{ \infty  } \frac{ 1 }{ 2^{k-1} }  \) is a geometric series it follows from the Comparison Test that  
\[  \sum_{ k=1  }^{ \infty  } \|{v}_{{n}_{k+1}} - {v}_{{n}_{k}} \|   \]
converges to some \( v \in V  \). By assumption, this tells us that 
\[  \sum_{ k=1  }^{ \infty  } ({v}_{{n}_{k+1}} - {v}_{{n}_{k}})  \]
converges to some \( v  \) in \( V  \). Now, observe that 
\begin{align*}
    {v}_{{n}_{1}} + \sum_{ j=1  }^{ k - 1  } ({v}_{{n}_{j+1}} - {v}_{{n}_{j}}) &= {v}_{{n}_{1}} + ({v}_{{n}_{2}} - {v}_{{n}_{1}}) + ({v}_{{n}_{3}} - {v}_{{n}_{2}}) + \cdots + ({v}_{{n}_{k}} - {v}_{{n}_{k-1}}) \\ &= {v}_{{n}_{k}}.  
\end{align*}
Taking the limit on both sides of the above equality, we see that 
\begin{align*}
    \lim_{ k  \to \infty  }  {v}_{{n}_{k}} &= \lim_{ k  \to \infty  }  \Big[ {v}_{{n}_{k }} + \sum_{ j=1  }^{ k - 1  } ({v}_{{n}_{j-1}} - {v}_{{n}_{j}})\Big]  \\
                                           &= {v}_{{n}_{1}} + \lim_{ k  \to \infty  }  \sum_{ j=1  }^{ k-1  } ({v}_{{n}_{j+1}} - {v}_{{n}_{j}}) \\
                                           &= {v}_{{n}_{1}} + v.
\end{align*}
Thus, we now see that \( ({v}_{{n}_{k}}) \) converges in \( V  \) which tells us that \( ({v}_{n}) \) is a converges in \( V  \). Hence, 
\end{proof}

\begin{lemma}\label{lemma}
   Let \( (V, \|\cdot\|) \) be a normed space. Suppose \( ({v}_{n}) \) is a Cauchy sequence, and some subsequence \( ({v}_{{n}_{k}}) \) converges to a point \( v  \) in \( V  \). Then \( ({v}_{n}) \) converges to \( v  \) in \( V  \).   
\end{lemma}
\begin{proof}
Let \( n > m  \). Since \( ({v}_{n}) \) is a Cauchy sequence in \( V  \), it follows that 
\[  \|{v}_{n} - {v}_{m} \| \to 0  \]
as \( n,m \to \infty  \). Also, \( ({v}_{{n}_{k}}) \) converges to some \( v \in V  \). So, for \( k \to \infty  \), we have 
\[  \|{v}_{{n}_{k}} - v \| \to 0. \]
Using the triangle inequality, it follows that 
\[  0 \leq \|{v}_{n} - v \| \leq \|{v}_{n} - {v}_{{n}_{k}} \| + \|{v}_{{n}_{k }} - v \| \to 0.  \]
Using the Squeeze Theorem, we have 
\[  \|{v}_{n} - v \| \to 0  \]
as \( n \to \infty  \) and we are done.
\end{proof}


\end{document}

