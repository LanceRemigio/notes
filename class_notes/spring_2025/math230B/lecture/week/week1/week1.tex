\documentclass[a4paper]{article}
\usepackage[utf8]{inputenc}
\usepackage[T1]{fontenc}
\usepackage{textcomp}
\usepackage{hyperref}
% \usepackage{fourier}
% \usepackage[dutch]{babel}
\usepackage{url}
% \usepackage{hyperref}
% \hypersetup{
%     colorlinks,
%     linkcolor={black},
%     citecolor={black},
%     urlcolor={blue!80!black}
% }
\usepackage{graphicx}
\usepackage{float}
\usepackage{booktabs}
\usepackage{enumitem}
% \usepackage{parskip}
\usepackage{emptypage}
\usepackage{subcaption}
\usepackage{multicol}
\usepackage[usenames,dvipsnames]{xcolor}

% \usepackage{cmbright}


\usepackage[margin=1in]{geometry}
\usepackage{amsmath, amsfonts, mathtools, amsthm, amssymb}
\usepackage{mathrsfs}
\usepackage{cancel}
\usepackage{bm}
\newcommand\N{\ensuremath{\mathbb{N}}}
\newcommand\R{\ensuremath{\mathbb{R}}}
\newcommand\Z{\ensuremath{\mathbb{Z}}}
\renewcommand\O{\ensuremath{\emptyset}}
\newcommand\Q{\ensuremath{\mathbb{Q}}}
\newcommand\C{\ensuremath{\mathbb{C}}}
\DeclareMathOperator{\sgn}{sgn}
\usepackage{systeme}
\let\svlim\lim\def\lim{\svlim\limits}
\let\implies\Rightarrow
\let\impliedby\Leftarrow
\let\iff\Leftrightarrow
\let\epsilon\varepsilon
\usepackage{stmaryrd} % for \lightning
\newcommand\contra{\scalebox{1.1}{$\lightning$}}
% \let\phi\varphi
\renewcommand\qedsymbol{$\blacksquare$}




% correct
\definecolor{correct}{HTML}{009900}
\newcommand\correct[2]{\ensuremath{\:}{\color{red}{#1}}\ensuremath{\to }{\color{correct}{#2}}\ensuremath{\:}}
\newcommand\green[1]{{\color{correct}{#1}}}



% horizontal rule
\newcommand\hr{
    \noindent\rule[0.5ex]{\linewidth}{0.5pt}
}


% hide parts
\newcommand\hide[1]{}



% si unitx
\usepackage{siunitx}
\sisetup{locale = FR}
% \renewcommand\vec[1]{\mathbf{#1}}
\newcommand\mat[1]{\mathbf{#1}}


% tikz
\usepackage{tikz}
\usepackage{tikz-cd}
\usetikzlibrary{intersections, angles, quotes, calc, positioning}
\usetikzlibrary{arrows.meta}
\usepackage{pgfplots}
\pgfplotsset{compat=1.13}


\tikzset{
    force/.style={thick, {Circle[length=2pt]}-stealth, shorten <=-1pt}
}

% theorems
\makeatother
\usepackage{thmtools}
\usepackage[framemethod=TikZ]{mdframed}
\mdfsetup{skipabove=1em,skipbelow=0em}


\theoremstyle{definition}

\declaretheoremstyle[
    headfont=\bfseries\sffamily\color{ForestGreen!70!black}, bodyfont=\normalfont,
    mdframed={
        linewidth=2pt,
        rightline=false, topline=false, bottomline=false,
        linecolor=ForestGreen, backgroundcolor=ForestGreen!5,
    }
]{thmgreenbox}

\declaretheoremstyle[
    headfont=\bfseries\sffamily\color{NavyBlue!70!black}, bodyfont=\normalfont,
    mdframed={
        linewidth=2pt,
        rightline=false, topline=false, bottomline=false,
        linecolor=NavyBlue, backgroundcolor=NavyBlue!5,
    }
]{thmbluebox}

\declaretheoremstyle[
    headfont=\bfseries\sffamily\color{NavyBlue!70!black}, bodyfont=\normalfont,
    mdframed={
        linewidth=2pt,
        rightline=false, topline=false, bottomline=false,
        linecolor=NavyBlue
    }
]{thmblueline}

\declaretheoremstyle[
    headfont=\bfseries\sffamily\color{RawSienna!70!black}, bodyfont=\normalfont,
    mdframed={
        linewidth=2pt,
        rightline=false, topline=false, bottomline=false,
        linecolor=RawSienna, backgroundcolor=RawSienna!5,
    }
]{thmredbox}

\declaretheoremstyle[
    headfont=\bfseries\sffamily\color{RawSienna!70!black}, bodyfont=\normalfont,
    numbered=no,
    mdframed={
        linewidth=2pt,
        rightline=false, topline=false, bottomline=false,
        linecolor=RawSienna, backgroundcolor=RawSienna!1,
    },
    qed=\qedsymbol
]{thmproofbox}

\declaretheoremstyle[
    headfont=\bfseries\sffamily\color{NavyBlue!70!black}, bodyfont=\normalfont,
    numbered=no,
    mdframed={
        linewidth=2pt,
        rightline=false, topline=false, bottomline=false,
        linecolor=NavyBlue, backgroundcolor=NavyBlue!1,
    },
]{thmexplanationbox}

\declaretheorem[style=thmgreenbox, numberwithin = section, name=Definition]{definition}
\declaretheorem[style=thmbluebox, name=Example]{eg}
\declaretheorem[style=thmredbox, numberwithin = section, name=Proposition]{prop}
\declaretheorem[style=thmredbox, numberwithin = section, name=Theorem]{theorem}
\declaretheorem[style=thmredbox, numberwithin = section,  name=Lemma]{lemma}
\declaretheorem[style=thmredbox, numberwithin = section,  numbered=no, name=Corollary]{corollary}


\declaretheorem[style=thmproofbox, name=Proof]{replacementproof}
\renewenvironment{proof}[1][\proofname]{\vspace{-10pt}\begin{replacementproof}}{\end{replacementproof}}


\declaretheorem[style=thmexplanationbox, name=Proof]{tmpexplanation}
\newenvironment{explanation}[1][]{\vspace{-10pt}\begin{tmpexplanation}}{\end{tmpexplanation}}


\declaretheorem[style=thmblueline, numbered=no, name=Remark]{remark}
\declaretheorem[style=thmblueline, numbered=no, name=Note]{note}

\newtheorem*{uovt}{UOVT}
\newtheorem*{notation}{Notation}
\newtheorem*{previouslyseen}{As previously seen}
\newtheorem*{problem}{Problem}
\newtheorem*{observe}{Observe}
\newtheorem*{property}{Property}
\newtheorem*{intuition}{Intuition}


\usepackage{etoolbox}
\AtEndEnvironment{vb}{\null\hfill$\diamond$}%
\AtEndEnvironment{intermezzo}{\null\hfill$\diamond$}%
% \AtEndEnvironment{opmerking}{\null\hfill$\diamond$}%

% http://tex.stackexchange.com/questions/22119/how-can-i-change-the-spacing-before-theorems-with-amsthm
\makeatletter
% \def\thm@space@setup{%
%   \thm@preskip=\parskip \thm@postskip=0pt
% }
\newcommand{\oefening}[1]{%
    \def\@oefening{#1}%
    \subsection*{Oefening #1}
}

\newcommand{\suboefening}[1]{%
    \subsubsection*{Oefening \@oefening.#1}
}

\newcommand{\exercise}[1]{%
    \def\@exercise{#1}%
    \subsection*{Exercise #1}
}

\newcommand{\subexercise}[1]{%
    \subsubsection*{Exercise \@exercise.#1}
}


\usepackage{xifthen}

\def\testdateparts#1{\dateparts#1\relax}
\def\dateparts#1 #2 #3 #4 #5\relax{
    \marginpar{\small\textsf{\mbox{#1 #2 #3 #5}}}
}

\def\@lesson{}%
\newcommand{\lesson}[3]{
    \ifthenelse{\isempty{#3}}{%
        \def\@lesson{Lecture #1}%
    }{%
        \def\@lesson{Lecture #1: #3}%
    }%
    \subsection*{\@lesson}
    \testdateparts{#2}
}

% \renewcommand\date[1]{\marginpar{#1}}


% fancy headers
\usepackage{fancyhdr}
\pagestyle{fancy}

\fancyhead[LE,RO]{Lance Remigio}
\fancyhead[RO,LE]{\@lesson}
\fancyhead[RE,LO]{}
\fancyfoot[LE,RO]{\thepage}
\fancyfoot[C]{\leftmark}

\makeatother




% notes
\usepackage{todonotes}
\usepackage{tcolorbox}

\tcbuselibrary{breakable}
\newenvironment{verbetering}{\begin{tcolorbox}[
    arc=0mm,
    colback=white,
    colframe=green!60!black,
    title=Opmerking,
    fonttitle=\sffamily,
    breakable
]}{\end{tcolorbox}}

\newenvironment{noot}[1]{\begin{tcolorbox}[
    arc=0mm,
    colback=white,
    colframe=white!60!black,
    title=#1,
    fonttitle=\sffamily,
    breakable
]}{\end{tcolorbox}}




% figure support
\usepackage{import}
\usepackage{xifthen}
\pdfminorversion=7
\usepackage{pdfpages}
\usepackage{transparent}
\newcommand{\incfig}[1]{%
    \def\svgwidth{\columnwidth}
    \import{./figures/}{#1.pdf_tex}
}

% %http://tex.stackexchange.com/questions/76273/multiple-pdfs-with-page-group-included-in-a-single-page-warning
\pdfsuppresswarningpagegroup=1



\begin{document}

\section{Lecture 1}

\subsection{Topics}

\begin{itemize}
    \item The derivative
    \item Continuity and Differentiability
    \item Differentiability Rules
\end{itemize}

\begin{definition}[Differentiability]
    \begin{enumerate}
        \item[(*)] Let \( I \subseteq  \R  \) be an interval, \( f: I \to \R  \), \( c \in I  \). We say \( f  \) is \textbf{differentiable} at \( c  \) if 
    \[  \lim_{ x \to c }  \frac{ f(x) -f(c) }{ x - c  } \]
    exists (that is, it equals a real number).
    \item[(*)] In this case, the quantity \( \lim_{ x \to c }  \frac{ f(x) - f(c) }{ x - c  }  \) is called the derivative of \( f  \) at \( c  \) and is denoted by
        \[  f'(c), \frac{ df }{ dx }(c), \frac{ df }{ dx } \Bigg|_{x = c} \]
    \item[(*)] If \( f: I \to \R  \) is differentiable at every point \( c \in I  \), we say \( f \) is differentiable (on \( I  \)).
    \end{enumerate}
\end{definition}

\begin{remark}
    The following are equivalent characterizations of the differentiability:
    \begin{align*}
        f'(c) = L &\iff \lim_{ x \to c }  \frac{ f(x) - f(c) }{ x - c  }  = L  \\
                  &\iff \forall \epsilon > 0 \ \exists \delta > 0 \ \text{such that} \ \text{if} \ 0 < | x - c  | < \delta \ \text{then} \ \Big| \frac{ f(x) - f(c) }{ x - c  }  - L  \Big|  < \epsilon \\
                  &\iff \forall \epsilon > 0 \ \exists \delta > 0 \ \text{such that if} \ 0 < | h  |  < \delta \ \text{then} \ \Big| \frac{ f(c+h) - f(c) }{  h  }  - L  \Big|  < \epsilon \\
                  &\iff \lim_{ h \to 0 }  \frac{ f(c+h) - f(c) }{  h  }  = L 
    \end{align*}
\end{remark}

\begin{theorem}[Differentiability Implies Continuous]
    Let \( I \subseteq  \R  \), \( c \in I  \), and \( f: I \to \R  \) is differentiable at \( c  \). Then \( f \) is continuous at \( c  \).
\end{theorem}

\begin{proof}
It suffices to show that \( \lim_{ x \to c } f(x) = f(c) \). Note that 
\begin{align*}
    \lim_{ x \to c } (f(x) - f(c)) &= \lim_{ x \to c }  \Big[ \frac{ f(x) - f(c) }{  x - c  }  \Big] (x - c) \\
                                   &= \Big[ \lim_{ x \to c }  \frac{ f(x) - f(c) }{  x - c  }  \Big] \Big[ \lim_{ x \to c }  (x - c) \Big] \\
                                   &= (f'(c)) (0) \\
                                   &= 0.
\end{align*}
So, we have 
\begin{align*}
    \lim_{ x \to c } f(x) &= \lim_{ x \to c }  [f(x)  - f(c) + f(c)] \\ 
                          &= \lim_{ x \to c } (f(x) - f(c)) + \lim_{ x \to c } f(c) \\   
                          &= 0 + \lim_{ x \to c } f(c) \\
                          &= 0 + f(c) \\
                          &= f(c).
\end{align*}
\end{proof}

\begin{corollary}
    If \( f: I \to \R  \) is NOT continuous at \( c \in I  \), then \( f \) is NOT differentiable at \( c  \).
\end{corollary}

\begin{eg}
    Let \( f: \R \to \R  \) be defined by 
    \[  f(x) = 
    \begin{cases}
        x^{2} &\text{if} \  x \in \Q \\
        0 &\text{if} \ x \notin \Q 
    \end{cases}. \]
\end{eg}

\begin{enumerate}
    \item[(i)] Prove that \( f  \) is continuous at \( 0  \).
        \begin{proof}
        Our goal is to show that 
        \[  \forall \epsilon > 0 \ \exists \delta > 0 \ \text{such that if} \ | x |  < \delta \ \text{then} \ | f(x) - f(c) | < \epsilon.  \]
        Let \( \epsilon > 0  \) be given. Note that if \( x \notin \Q  \),   
       \[  | f(x) |  = | 0  |  < \epsilon. \] 
       Otherwise, we have \( | f(x)  |  = | x^{2} |  = | x |^{2} \). IN this case, we claim that \( \delta = \sqrt{ \epsilon }  \) will work. Indeed, if \( | x  |  < \delta \), then we have 
       \[  | f(x) |  = | x |^{2} < (\sqrt{ \epsilon } )^{2} = \epsilon. \]
        \end{proof}
    \item[(ii)] Prove \( f \) is discontinuous at all \( x \neq 0  \).
        \begin{proof}
        Let \( c \neq 0  \). Our goal is to show that \( f  \) is discontinuous at \( c  \). By the sequential criterion for continuity, it suffices to find a sequence \( ({a}_{n}) \) such that \( {a}_{n} \to c  \) but \( f({a}_{n}) \not\to f(c) \). We will consider two cases; that is, we could either have \( c \notin \Q  \) or \( c \in \Q  \). 

        Suppose \( c \notin \Q  \). Since \( \Q  \) is dense in \( \R  \), there exists a sequence of rational numbers \( ({r}_{n}) \) such that \( {r}_{n} \to c  \). Note that \( f({r}_{n}) = {r}_{n}^{2} \to c^{2} \neq 0   \), but \( f(c) = 0  \). Clearly, \( f({r}_{n}) \not\to f(c) \) and so \( f  \) must be discontinuous at \( c  \).

        Suppose \( c \in \Q  \). Since the set of irrational numbers is also dense in \( \R  \), we can find a sequence \( ({s}_{n}) \) such that \( {s}_{n} \to c  \). Note that \( f({s}_{n})  =  0  \), but \( f(c) = c^{2} \neq 0  \). Thus, \( f({s}_{n}) \not\to f(c) \). Therefore, \( f  \) must be discontinuous at \( c  \). 
        \end{proof}
    \item[(iii)] Prove that \( f  \) is nondifferentiable at all \( x \neq 0  \).
        \begin{proof}
        Let \( c \neq 0  \). Since \( f  \) is discontinuous at \( c  \), we can conclude that \( f  \) is not differentiable at \( c  \).
        \end{proof}
    \item[(iv)] Prove that \( f'(0) = 0   \).
        \begin{proof}
        We need to show 
        \[  \lim_{ x \to c } \frac{ f(x) - f(0) }{ x - 0  }  = \frac{ f(x) }{ x  } = 0. \]
        \end{proof}
\end{enumerate}


\end{document}

