\documentclass[11pt,a4paper]{article} 
\usepackage{standalone}
\usepackage{import}
\usepackage[utf8]{inputenc}
\usepackage[T1]{fontenc}
\usepackage{textcomp}
\usepackage{hyperref}
% \usepackage{fourier}
% \usepackage[dutch]{babel}
\usepackage{url}
% \usepackage{hyperref}
% \hypersetup{
%     colorlinks,
%     linkcolor={black},
%     citecolor={black},
%     urlcolor={blue!80!black}
% }
\usepackage{graphicx}
\usepackage{float}
\usepackage{booktabs}
\usepackage{enumitem}
% \usepackage{parskip}
\usepackage{emptypage}
\usepackage{subcaption}
\usepackage{multicol}
\usepackage[usenames,dvipsnames]{xcolor}

% \usepackage{cmbright}


\usepackage[margin=1in]{geometry}
\usepackage{amsmath, amsfonts, mathtools, amsthm, amssymb}
\usepackage{mathrsfs}
\usepackage{cancel}
\usepackage{bm}
\newcommand\N{\ensuremath{\mathbb{N}}}
\newcommand\R{\ensuremath{\mathbb{R}}}
\newcommand\Z{\ensuremath{\mathbb{Z}}}
\renewcommand\O{\ensuremath{\emptyset}}
\newcommand\Q{\ensuremath{\mathbb{Q}}}
\newcommand\C{\ensuremath{\mathbb{C}}}
\DeclareMathOperator{\sgn}{sgn}
\usepackage{systeme}
\let\svlim\lim\def\lim{\svlim\limits}
\let\implies\Rightarrow
\let\impliedby\Leftarrow
\let\iff\Leftrightarrow
\let\epsilon\varepsilon
\usepackage{stmaryrd} % for \lightning
\newcommand\contra{\scalebox{1.1}{$\lightning$}}
% \let\phi\varphi
\renewcommand\qedsymbol{$\blacksquare$}




% correct
\definecolor{correct}{HTML}{009900}
\newcommand\correct[2]{\ensuremath{\:}{\color{red}{#1}}\ensuremath{\to }{\color{correct}{#2}}\ensuremath{\:}}
\newcommand\green[1]{{\color{correct}{#1}}}



% horizontal rule
\newcommand\hr{
    \noindent\rule[0.5ex]{\linewidth}{0.5pt}
}


% hide parts
\newcommand\hide[1]{}



% si unitx
\usepackage{siunitx}
\sisetup{locale = FR}
% \renewcommand\vec[1]{\mathbf{#1}}
\newcommand\mat[1]{\mathbf{#1}}


% tikz
\usepackage{tikz}
\usepackage{tikz-cd}
\usetikzlibrary{intersections, angles, quotes, calc, positioning}
\usetikzlibrary{arrows.meta}
\usepackage{pgfplots}
\pgfplotsset{compat=1.13}


\tikzset{
    force/.style={thick, {Circle[length=2pt]}-stealth, shorten <=-1pt}
}

% theorems
\makeatother
\usepackage{thmtools}
\usepackage[framemethod=TikZ]{mdframed}
\mdfsetup{skipabove=1em,skipbelow=0em}


\theoremstyle{definition}

\declaretheoremstyle[
    headfont=\bfseries\sffamily\color{ForestGreen!70!black}, bodyfont=\normalfont,
    mdframed={
        linewidth=2pt,
        rightline=false, topline=false, bottomline=false,
        linecolor=ForestGreen, backgroundcolor=ForestGreen!5,
    }
]{thmgreenbox}

\declaretheoremstyle[
    headfont=\bfseries\sffamily\color{NavyBlue!70!black}, bodyfont=\normalfont,
    mdframed={
        linewidth=2pt,
        rightline=false, topline=false, bottomline=false,
        linecolor=NavyBlue, backgroundcolor=NavyBlue!5,
    }
]{thmbluebox}

\declaretheoremstyle[
    headfont=\bfseries\sffamily\color{NavyBlue!70!black}, bodyfont=\normalfont,
    mdframed={
        linewidth=2pt,
        rightline=false, topline=false, bottomline=false,
        linecolor=NavyBlue
    }
]{thmblueline}

\declaretheoremstyle[
    headfont=\bfseries\sffamily\color{RawSienna!70!black}, bodyfont=\normalfont,
    mdframed={
        linewidth=2pt,
        rightline=false, topline=false, bottomline=false,
        linecolor=RawSienna, backgroundcolor=RawSienna!5,
    }
]{thmredbox}

\declaretheoremstyle[
    headfont=\bfseries\sffamily\color{RawSienna!70!black}, bodyfont=\normalfont,
    numbered=no,
    mdframed={
        linewidth=2pt,
        rightline=false, topline=false, bottomline=false,
        linecolor=RawSienna, backgroundcolor=RawSienna!1,
    },
    qed=\qedsymbol
]{thmproofbox}

\declaretheoremstyle[
    headfont=\bfseries\sffamily\color{NavyBlue!70!black}, bodyfont=\normalfont,
    numbered=no,
    mdframed={
        linewidth=2pt,
        rightline=false, topline=false, bottomline=false,
        linecolor=NavyBlue, backgroundcolor=NavyBlue!1,
    },
]{thmexplanationbox}

\declaretheorem[style=thmgreenbox, numberwithin = section, name=Definition]{definition}
\declaretheorem[style=thmbluebox, name=Example]{eg}
\declaretheorem[style=thmredbox, numberwithin = section, name=Proposition]{prop}
\declaretheorem[style=thmredbox, numberwithin = section, name=Theorem]{theorem}
\declaretheorem[style=thmredbox, numberwithin = section,  name=Lemma]{lemma}
\declaretheorem[style=thmredbox, numberwithin = section,  numbered=no, name=Corollary]{corollary}


\declaretheorem[style=thmproofbox, name=Proof]{replacementproof}
\renewenvironment{proof}[1][\proofname]{\vspace{-10pt}\begin{replacementproof}}{\end{replacementproof}}


\declaretheorem[style=thmexplanationbox, name=Proof]{tmpexplanation}
\newenvironment{explanation}[1][]{\vspace{-10pt}\begin{tmpexplanation}}{\end{tmpexplanation}}


\declaretheorem[style=thmblueline, numbered=no, name=Remark]{remark}
\declaretheorem[style=thmblueline, numbered=no, name=Note]{note}

\newtheorem*{uovt}{UOVT}
\newtheorem*{notation}{Notation}
\newtheorem*{previouslyseen}{As previously seen}
\newtheorem*{problem}{Problem}
\newtheorem*{observe}{Observe}
\newtheorem*{property}{Property}
\newtheorem*{intuition}{Intuition}


\usepackage{etoolbox}
\AtEndEnvironment{vb}{\null\hfill$\diamond$}%
\AtEndEnvironment{intermezzo}{\null\hfill$\diamond$}%
% \AtEndEnvironment{opmerking}{\null\hfill$\diamond$}%

% http://tex.stackexchange.com/questions/22119/how-can-i-change-the-spacing-before-theorems-with-amsthm
\makeatletter
% \def\thm@space@setup{%
%   \thm@preskip=\parskip \thm@postskip=0pt
% }
\newcommand{\oefening}[1]{%
    \def\@oefening{#1}%
    \subsection*{Oefening #1}
}

\newcommand{\suboefening}[1]{%
    \subsubsection*{Oefening \@oefening.#1}
}

\newcommand{\exercise}[1]{%
    \def\@exercise{#1}%
    \subsection*{Exercise #1}
}

\newcommand{\subexercise}[1]{%
    \subsubsection*{Exercise \@exercise.#1}
}


\usepackage{xifthen}

\def\testdateparts#1{\dateparts#1\relax}
\def\dateparts#1 #2 #3 #4 #5\relax{
    \marginpar{\small\textsf{\mbox{#1 #2 #3 #5}}}
}

\def\@lesson{}%
\newcommand{\lesson}[3]{
    \ifthenelse{\isempty{#3}}{%
        \def\@lesson{Lecture #1}%
    }{%
        \def\@lesson{Lecture #1: #3}%
    }%
    \subsection*{\@lesson}
    \testdateparts{#2}
}

% \renewcommand\date[1]{\marginpar{#1}}


% fancy headers
\usepackage{fancyhdr}
\pagestyle{fancy}

\fancyhead[LE,RO]{Lance Remigio}
\fancyhead[RO,LE]{\@lesson}
\fancyhead[RE,LO]{}
\fancyfoot[LE,RO]{\thepage}
\fancyfoot[C]{\leftmark}

\makeatother




% notes
\usepackage{todonotes}
\usepackage{tcolorbox}

\tcbuselibrary{breakable}
\newenvironment{verbetering}{\begin{tcolorbox}[
    arc=0mm,
    colback=white,
    colframe=green!60!black,
    title=Opmerking,
    fonttitle=\sffamily,
    breakable
]}{\end{tcolorbox}}

\newenvironment{noot}[1]{\begin{tcolorbox}[
    arc=0mm,
    colback=white,
    colframe=white!60!black,
    title=#1,
    fonttitle=\sffamily,
    breakable
]}{\end{tcolorbox}}




% figure support
\usepackage{import}
\usepackage{xifthen}
\pdfminorversion=7
\usepackage{pdfpages}
\usepackage{transparent}
\newcommand{\incfig}[1]{%
    \def\svgwidth{\columnwidth}
    \import{./figures/}{#1.pdf_tex}
}

% %http://tex.stackexchange.com/questions/76273/multiple-pdfs-with-page-group-included-in-a-single-page-warning
\pdfsuppresswarningpagegroup=1




\title{Math 230A: Homework 1}
\author{Lance Remigio}

\begin{document}
\maketitle
\date
\begin{enumerate}
    \item Mark each statement True or False.  
        \begin{enumerate}
            \item[1-1)] If \( x  \) and \( y  \) are elements of an ordered field, then either \( x \leq y  \) or \( y < x  \). \textbf{True.}
            \item[1-2)] Every ordered field has the least upper bound property. \textbf{False.}
            \item[1-3)] If \( E \subseteq \R  \) is bounded above and \( \alpha = \sup E  \), then \( \alpha \in E  \). \textbf{False.}
            \item[1-4)] If \( E \subseteq \R  \) and \( \alpha \geq x  \), for all \( x \in E  \), then \( \alpha = \sup E  \). \textbf{False.}
            \item[1-5)] If \( E \subseteq \R  \) and \( \alpha \geq x  \), for all \( x \in E \), and \( \alpha \in E  \), then \( \alpha = \sup E   \). \textbf{True.}
            \item[1-6)] If \( E \subseteq \R  \) and \( \alpha \geq x  \), for all \( x \in E  \), and \( \alpha \notin E  \), then \( \alpha \neq \sup(E ) \). \textbf{False.}  
\end{enumerate}

\item Prove the following: Suppose \( \alpha  \) is an upper bound for \( E \subset \R   \). Then \( \alpha = \sup E   \) if and only if for all \( \epsilon > 0  \), there exists \( {x}_{0} \in E  \) such that \( {x}_{0} > \alpha - \epsilon  \). \label{Problem 2}  
    \begin{proof}
    Suppose \( \alpha  \) is an upper bound for \( E \subseteq \R  \). For the forwards direction, suppose \( \alpha = \sup E  \) and let \( \epsilon > 0 \). Note that \( \alpha - \epsilon < \alpha  \) implies that \( \alpha - \epsilon  \) is NOT an upper bound of \( E  \). By definition, there must exist an element \( {x}_{0} \in E  \) such that \( {x}_{0} > \alpha - \epsilon  \). Thus, \( \alpha < {x}_{0} + \epsilon  \).

    For the backwards direction, let \( \epsilon > 0  \) and let \( \alpha  \) be an upper bound of \( E  \). We need to show \( \alpha = \sup E  \); that is, we need to show that \( \alpha  \) is an upper bound of \( E  \) and that any \( \gamma < \alpha  \) implies \( \gamma  \) is NOT an upper bound of \( E  \). Notice that the first property is satisfied by our assumption. Thus, all that is left to show is the second property.  

    Suppose \( \gamma < \alpha  \). Thus, we have \( \alpha - \gamma > 0  \), so pick \( \epsilon = \alpha - \gamma  \).  Our goal is to find some element \( x \) in \( E  \) such that \( x > \alpha - \epsilon \). Choose \( \epsilon = \alpha - \gamma  \). By assumption, there exists an element \( {x}_{0} \in E  \) such that     
    \[  {x}_{0} > \alpha - \epsilon = \alpha - (\alpha - \gamma) = \gamma. \]
    Thus, \( {x}_{0} > \gamma  \) for some \( {x}_{0} \in E  \). Therefore, \( \gamma  \) is NOT an upper bound of \( E  \) and so we conclude that 
    \[  \alpha = \sup E. \]
    \end{proof}

\item Let \( E  = \{ n / n + 1 : n \in \N  \}  \). Prove that \( \sup E  = 1  \). 
    \begin{proof}
    To show that \( 1  \) is the supremum of \( E  \), we need to show that \( 1  \) is an upper bound for \( E  \) and that it satisfies the lemma found in {\hyperref[Problem 2]{Problem 2}}. First, we will show that \( 1  \) is an upper bound of \( E  \). Observe that for any \( n \in \N  \), we have  
    \[  \frac{ n  }{  n + 1  }  < \frac{  n }{ n }  = 1.  \]
    Thus, \( E  \) is bounded above by \( 1 \). 

    Let \( \epsilon > 0  \). By the Archimedean Property, choose \( n \in \N  \) such that   
    \[ \frac{ 1 }{ n + 1  } < \epsilon.  \]
    Then observe that 
    \[  \frac{ 1 }{ n + 1  }  = \frac{ (n+1) - 1  }{ n + 1  } = 1 - \frac{ n  }{ n + 1  }.   \]
    Thus, 
    \[  1 - \frac{ n }{ n + 1  } < \epsilon \iff \frac{ n }{ n + 1 }  > 1 - \epsilon. \]
    But note that \( \frac{  n  }{  n+ 1 } \in E   \). By {\hyperref[Problem 2]{Problem 2}}, we conclude that \( \sup E = 1    \). 
    
    

     \end{proof}

 \item State the analogue of the Very Useful Theorem for infimums. \label{Problem 3}
     \begin{solution}
        Suppose \( \beta  \) is a lower bound for \( E \subseteq \R    \). Then \( \beta = \inf E  \) if and only if for all \( \epsilon > 0  \), there exists \( {y}_{0} \in E  \) such that \( {y}_{0} < \beta + \epsilon  \).
        \end{solution}
    \item Suppose \( a,b \in \R  \) and \( a \leq b + \epsilon  \) for every \( \epsilon > 0  \). Prove \( a \leq b  \). \label{Problem 5}
        \begin{proof}
        Suppose \( a, b \in \R  \) and \( a \leq b + \epsilon  \) for all \( \epsilon > 0  \). Suppose for sake of contradiction that \(  a > b  \). Note that \( a - b > 0   \). Pick \( \epsilon  = \frac{  a - b  }{  2  }  \). Then 
        \begin{align*}
            b  + \epsilon = b + \frac{ a - b  }{ 2 } = \frac{  2b + a - b }{ 2  }     
                                                     &= \frac{ a  +b  }{ 2  } \\ 
                                                     &< \frac{ a + a  }{ 2  } \\
                                                     &= \frac{ 2 a  }{  2  } \\
                                                     &= a.
        \end{align*}
        But this means that \( a < b + \epsilon  \) which is a contradiction of our assumption that \( a \leq b + \epsilon  \). Thus, it must be the case that \( a \leq b  \).
    \end{proof}
    \item Complete the following proof.   
\begin{theorem}[Greatest-lower-bound property of \( \R  \)]
   Every nonempty subset of \( A  \) of \( \R  \) that is bounded below has a greatest lower bound. In other words, \( \inf A  \) exists and is a real number.   
\end{theorem}
\begin{proof}
Continuation of the proof presented in homework. Set \( \beta = - \alpha  \) where \(  \alpha  = \sup (-A )   \). Thus, there exists \( a \in A  \) such that \( - \gamma < - a  \). Multiplying by a negative on both sides of this inequality, we get \(  \gamma > a  \) for some \( a \in A  \). If \( \gamma  \) was a lower bound of \( A  \), then \( \gamma \leq a   \) for all \( a \in A  \). However, we have the negation of this, so we must have that \( \gamma  \) is NOT a lower bound of \( A  \), which satisfies part (ii). Since \( \beta \leq \alpha  \) for all \( a \in A  \) and \( \beta < \gamma  \) implies \( \gamma  \) is not a lower bound for \( A  \), we conclude that \( \beta  \) must be the infimum of \( A  \).  
\end{proof}
\item \begin{enumerate} \label{Problem 7}
    \item[(i)] Let \( A \subseteq \R   \) be a nonempty set, which is bounded from above. Show that if \( \sup A \notin A  \), then for all \( \epsilon > 0  \) the open interval \( (\sup A - \epsilon, \sup A ) \) contains infinitely many elements of \( A  \).
        \begin{proof}
            Let \( \epsilon > 0  \). Our goal is to show that \( (\sup A - \epsilon, \sup A ) \) has infinitely many points. Note that as a consequence of the result found in {\hyperref[Problem 2]{Problem 2}}, there exists \( {x}_{0} \in A  \) such that  
            \[  \sup A - \epsilon < {x}_{0} \leq \sup A. \]
            By assumption, we have that \( \sup A \notin A  \). Thus, we have
            \[  \sup A - \epsilon  < {x}_{0} < \sup A  \]
            for some \( {x}_{0} \in A  \). So, \( (\sup A - \epsilon, \sup A ) \) is nonempty.

            Now, we will show, through contradiction that there exists an \( \epsilon > 0  \) such that \( (\sup A - \epsilon, \sup A ) \) contains infinitely many points. Thus, suppose that \( (\sup A - \epsilon , \sup A ) \cap A  \) contains FINITELY many elements. Let us denote these elements by \( {a}_{i} \) for  \( 1 \leq i \leq n  \) in 
        \[  (\sup A - \epsilon, \sup A ) \cap A = \{ {a}_{1}, \dots, {a}_{n}:  1 \leq i \leq n   \}. \]
        For all \( 1 \leq k \leq n  \), let the distance between each \( {a}_{k } \) and \( \sup A  \) by  
            \[  {d}_{k } = | \sup A - {a}_{k} |.  \]
            Since \( {a}_{k } \in A  \) and \( \sup A \notin A  \), we must have \(  | \sup A - {a}_{k } | > 0  \). Let 
            \[  d = \frac{ \min \{ {d}_{1}, \dots, {d}_{n} \}   }{ 2  }.   \]
            But this implies that none of the \( {a}_{1}, \dots, {a}_{n} \) lies inside the interval \( (\sup A - d , \sup A ) \). That is, 
            \[  A \cap (\sup A - d , \sup A ) = \emptyset, \]
            which is a contradiction. Thus, \( (\sup A - \epsilon , \sup A ) \) must contain infinitely many points.


        \end{proof}
    \item[(ii)] Let \( B \subseteq \R  \) be a nonempty set, which is bounded from below. Show that if \( \inf B \notin B  \), then for all \( \epsilon > 0  \) the open interval \( (\inf B , \inf B + \epsilon ) \) contains infinitely many elements of \( B  \).
    \begin{proof}
    Let \( \epsilon > 0  \). Our goal is to show that \( (\inf B, \inf B + \epsilon ) \) contain infinitely many points of \( B  \). Note that as a consequence of {\hyperref[Problem 3]{Problem 3}}, there exists an \( \beta \in B  \) such that  
    \[ \inf B   \leq  \beta < \inf B + \epsilon. \]
    Since \( \inf B \notin B  \), we must have
    \[  \inf B < \beta < \inf B + \epsilon. \]
    Thus, the open interval \( (\inf B, \inf B + \epsilon )  \) is nonempty.

    Now, we will show, through contradiction, that \( (\inf B, \inf B + \epsilon) \cap B \) contains infinitely many points. Suppose there exists an \( \epsilon > 0  \) such that \( (\inf B, \inf B + \epsilon) \) FINITELY many points. Denote these points by \( {b}_{i}  \) for \( 1 \leq i \leq n  \) in
    \[  (\inf B, \inf B + \epsilon) \cap B = \{ {b}_{i} : 1 \leq i \leq n   \}  \]
    For all \( 1 \leq k \leq n  \), let \( {d}_{k } = | \inf B - {b}_{k } |  \). Since \( {b}_{k } \in B  \) and \( \inf B \notin  B  \), we must have \( | \inf B - {b}_{k } | > 0  \). Let 
    \[  d = \frac{ \min \{ {d}_{1}, {d}_{2}, \dots, {d}_{n} \}  }{ 2 }. \]
    But this implies that none of the \( {b}_{i} \) for \( 1 \leq i \leq n   \) lies within the open interval \( (\inf B , \inf B + \epsilon) \). That is, the set 
    \[  (\inf B , \inf B + \epsilon) \cap B = \emptyset \]
    which is a contradiction. Thus, the open interval \( (\inf B, \inf B + \epsilon) \) must have infinitely many points.
    \end{proof}
\end{enumerate}

\item Suppose the construction of \( \R  \) and the proof of the least-upper-bound property for \( \R  \) can be completed without directly using the well-ordering principle for \( \N  \). (Of course, statements equivalent to the well-ordering principle might have been used, but we are assuming that the nontrivial fact of their equivalence to the well-ordering principle was not utilized in the construction of \( \R  \) or in proving its Dedekind completeness.) Use the greatest-lower-bound property of \( \R  \) and the result of the previous exercises to prove the well-ordering principle for \( \N  \); that is, prove that every nonempty subset of \( \N  \) has a minimum.
    \begin{proof}
    Let \( E  \) be a nonempty subset of \( \N  \). First, we will show that \( \inf E  \) exists. Observe that \( 0  \) is a lower bound of \( E  \). Thus, \( E  \) is bounded below. Thus, \( \inf E  \) must exists by the greatest-lower-bound property of \( \R  \).    

    Let \( \epsilon = \frac{ 1 }{ 5 }   \). To show that \( E  \) contains a minimum, we suppose for contradiction that \( \inf E \notin E  \). By {\hyperref[Problem 2]{Part 2 of Problem 7}}, the open interval \( (\inf E, \inf E + \frac{ 1 }{ 5 } ) \) intersects \( E  \) at infinitely many points (these points being natural numbers). Suppose we choose an \( m \in \N \) such that \( m  \in (\inf E , \inf E + \frac{ 1 }{ 5 }  ) \).  Note that \( m  \) also lies interval \( (m - 1, m + 1) \). Thus, \( (\inf E, \inf E + \epsilon) \subseteq (m-1,m+1) \). However, note that \( (\inf E, \inf E + \frac{ 1 }{ 5 }) \) contains infinitely many natural numbers which cannot happen within the open interval \( (m-1,m+1) \) since this interval contains at most one element. Thus, this is a contradiction. Thus, it must be the case that \( \inf E \in E  \).
\end{proof}

\item Let \( A, B \subseteq \R  \) be nonempty, bounded sets and let \( c \in \R  \). Define the following sets:
    \begin{align*}
        A + B &= \{ a + b : a \in A, b \in B \}  \\
        A - B &= \{ a - b : a \in A , b \in B  \}  \\
        A \cdot B &= \{ ab : a \in A, b \in B \} \\
        cA &= \{ ca: a \in A \}.
    \end{align*}
    Prove that 
    \begin{enumerate}
        \item[9-1)] \( \inf(A + B) = \inf(A) + \inf(B) \).
            \begin{proof}
                In order to show that \( \inf(A+B) = \inf A + \inf B  \), we need to show that 
                \begin{enumerate}
                    \item[(i)] The set \( A + B   \) is bounded below by \( \inf(A) + \inf(B) \).
                    \item[(ii)] For every \( \epsilon > 0  \), there exists \( \varphi \in  A + B  \) such that  
                        \[  \varphi < \inf(A) + \inf(B) + \epsilon. \]
                \end{enumerate} 
                Note that \( A + B \neq \emptyset \) since \(  a + b \in A + B  \) and \( A  \) and \( B  \) are nonempty sets. Since \( A  \) and \( B  \) are also bounded below, we can see that \( a \geq \inf A  \) \( a \in A  \) and \( b \geq \inf B  \) for all \( b \in B  \) implies  
                \[  a + b \geq \inf(A) + \inf (B) \ \ \text{for all} \ a + b \in A + B,    \]
                which shows (i).
    Now, let \( \epsilon > 0 \). Using the lemma found in {\hyperref[Problem 3]{Problem 3}}, there exists an \( \alpha \in A  \) and \( \beta \in \beta  \) such that 
                \[  \alpha < \inf(A) + \frac{ \epsilon }{ 2 } \]
                and
                \[  \beta < \inf(B) + \frac{ \epsilon }{ 2 },  \]
                respectively. Adding these two inequalities, we get
                \[ \varphi =  \alpha + \beta < \inf(A) + \inf(B) + \epsilon \ \ \text{for some} \ \varphi \in A + B.  \]
                Thus, we conclude that \( \inf(A+B) = \inf(A) + \inf(B) \).  
            \end{proof}
        \item[9-2)] \( \sup (A + B) = \sup (A) + \sup (B) \) 

            \begin{proof}
            In order to show that \( \sup(A+B) = \sup(A) + \sup(B) \), we need to show that   
            \begin{enumerate}
                \item[(i)] \( A + B  \) is bounded above and
                \item[(ii)] For every \( \epsilon > 0 \) ,there exists a \( \lambda > \sup(A) + \sup(B) - \epsilon \).
            \end{enumerate}
            Note that \( A + B \neq \emptyset  \) since \( A  \) and \( B  \) is nonempty.Since \( A  \) and \( B  \) are bounded above, we have that \( a \leq \sup (A)  \) for all \( a \in A   \) and \(  b \leq \sup (B) \) for all \( b \in B  \) implies that 
            \[  a + b \leq \sup(A) + \sup(B)  \]
            for all \( a + b \in A + B  \), which proves (i).
            Let \( \epsilon > 0  \). Since \( \sup(A) \) and \( \sup(B) \) exists, we see that there exists \( \alpha \in A  \) and \( \beta \in B  \) such that 
        \[  \alpha > \sup(A) - \frac{ \epsilon }{ 2 }   \]
        and 
        \[  \beta > \sup(B) - \frac{ \epsilon  }{ 2 }  \]
        by the lemma found in {\hyperref[Problem 2]{Problem 2}}. Adding these two inequalities results in 
        \[ \lambda =  \alpha + \beta > \sup(A) + \sup(B) - \epsilon. \]
        Thus, \( \sup(A+B) = \sup(A) + \sup(B) \) by the lemma found in {\hyperref[Problem 2]{Problem 2}}.
            \end{proof}
        \item[9-3)] \( \sup(-A) = - \inf(A) \) 
            \begin{proof}
            To show that \( \sup(-A) = - \inf(A) \), we need to show that  
            \begin{enumerate}
                \item[(i)] For all \( - a \in -A  \), we have \( -a \leq - \inf(A)  \) and
                \item[(ii)] For all \( \epsilon > 0  \), there exists \( - \alpha \in -A  \) such that 
                    \[  - \alpha < - \inf(A) - \epsilon. \]
            \end{enumerate}
            Since \( A  \) is bounded below and nonempty, we know that \( a \geq \inf(A) \) for all \( a \in A  \). Multiplying this inequality by a negative, we get that
            \[  - a \leq - \inf(A) \ \ \text{for all} \ -a \in - A. \]
            Thus, \( -A  \) is bounded above by \( - \inf(A) \).

            Let \( \epsilon > 0 \). Since \( \inf(A)  \) exists, there exists \( \gamma \in A  \) such that   
            \[  \gamma < \inf(A) + \epsilon. \]
            By multiplying by a negative on this inequality, we must have that 
            \[  - \gamma > - \inf(A) - \epsilon \]
            for some \( - \gamma \in -A  \). Thus, we must have that \( \sup(-A) = -\inf(A)  \).
            \end{proof}
        \item[9-4)] \( \inf(-A) = - \sup(A) \).
            \begin{proof}
                To show that \( \inf(-A) = - \sup(A) \), we must show that 
                \begin{enumerate}
                    \item[(i)] For all \( - a \in - A  \), we have \( - a \geq - \sup(A) \). 
                    \item[(ii)] For all \( \epsilon > 0  \), there exists \( - \beta \in - A  \) such that 
                        \[  - \beta < - \sup(A) + \epsilon. \]
                \end{enumerate}
                To show (i), let \( a \in A  \). Since \( A  \) is nonempty and bounded above, we have that  
                \[ a \leq   \sup(A).  \]
                Multiplying by a negative on this inequality, we get
                \[  - a \geq - \sup(A) \]
                for all \( - a \in -A  \), which shows (i). 

                Now, let \( \epsilon > 0 \). Since \( \sup (A)  \) exists, there exists \( \gamma \in A  \) such that  
                \[  \gamma > \sup A - \epsilon. \]
                Multiplying by a negative, we get
                \[  - \gamma < - \sup A + \epsilon  \]
                for some \( -\gamma \in -A  \), which shows (ii). By the lemma found in {\hyperref[Problem 3]{Problem 3}}, we conclude that \( \inf(-A) = - \sup(A) \).
            \end{proof}
        \item[9-5)] \( \sup (A -B) = \sup(A) - \inf(B) \).
            \begin{proof}
            Observe that 
            \begin{align*}
                \sup(A - B) = \sup(A + (-B)) &= \sup(A) + \sup(-B) \tag{9-2)} \\
                                             &= \sup(A) - \inf(B) \tag{9-3)}
            \end{align*}
            \end{proof}
        \item[9-6)] \( \inf(A - B) = \inf(A) - \sup(B) \).
            \begin{proof}
            Observe that
            \begin{align*}
                \inf(A -B) = \inf(A + (-B)) &= \inf(A) + \inf(-B) \tag{9-1)} \\
                                            &= \inf(A) - \sup(A). \tag{9-4)}
            \end{align*}
            \end{proof}
        \item[9-7)] \( \sup(cA) = c \sup (A)  \) if \( c > 0  \).
               \begin{proof}
                   Let \( c > 0  \). We will show that \( \sup(cA) = c \sup(A) \). Note that \( cA  \) is nonempty because \( ca \in cA  \). First, we show that \( cA  \) is bounded above by \( c \sup(A) \). Since \( \sup(A) \) exists, we know that  
                   \[  a \leq \sup(A) \ \ \text{for all} \ a \in A.   \]
                   Multiplying by \( c  \), we must have
                   \[  ca \leq c \sup(A) \ \ \text{for all} \  ca \in cA. \]
                   Now, let \( \epsilon > 0 \). Since \( \sup(A)  \) exists, there exists \( \alpha \in A  \) such that   
                   \[  \alpha > \sup(A) - \frac{ \epsilon }{ c }.  \]
                   Multiplying by \( c > 0  \), we must have
                   \[  c \alpha > c \sup(A) - \epsilon \ \ \text{for some } c \alpha \in cA.   \]
                   Thus, \( c \sup(A) = \sup(cA) \) by the lemma found in {\hyperref[Problem 2]{Problem 2}}. 

               \end{proof} 
        \item[9-8)] \( \inf(cA) = c \inf (A)  \) if \( c > 0  \).
            \begin{proof}
    Let \( c > 0  \). We will show that \( \inf(cA) = c \inf(A) \). Note that \( cA  \) is nonempty because \( ca \in cA  \). First, we show that \( cA  \) is bounded below by \( c \inf(A) \). Since \( \inf(A) \) exists, we know that  
                   \[  a \geq \inf(A) \ \ \text{for all} \ a \in A.   \]
                   Multiplying by \( c  \), we must have
                   \[  ca \geq c \inf(A) \ \ \text{for all} \  ca \in cA. \]
                   Now, let \( \epsilon > 0 \). Since \( \inf(A)  \) exists, there exists \( \beta \in A  \) such that   
                   \[  \beta < \inf(A) + \frac{ \epsilon }{ c }.  \]
                   Multiplying by \( c > 0  \), we must have
                   \[  c \beta > c \inf(A) + \epsilon \ \ \text{for some } c \beta \in cA.   \]
                   Thus, \( c \inf(A) = \inf(cA) \) by the lemma found in {\hyperref[Problem 2]{Problem 2}}. 
            \end{proof}
        \item[9-9)] \( \sup(cA) = c \inf(A) \) if \( c < 0  \).
            \begin{proof}
            Let \( c < 0  \). We will show that \( \sup(cA) = c \inf(A) \). We will first show that \( cA  \) is bounded above by \( c \inf (A) \). Note that \( \inf(A) \) exists. Thus, \( a \geq \inf(A) \) for all \( a \in A  \). Since \( c < 0  \), \( ca \leq c \inf(A)  \) for all \( ca \in cA  \). Thus, \( c \inf(A) \) is an upper bound of \( cA \).
           
            Now, let \( \epsilon > 0 \). Since \( \inf(A) \) exists, there must exists an \( \alpha \in A  \) such that  
            \[  \alpha < \inf(A) - \frac{ \epsilon }{ c }. \]
            By multiplying through by \( c < 0  \), we have
            \[  c \alpha > c \inf(A) - \epsilon \]
            for some \( c \alpha \in cA  \). Thus, we conclude that \( \sup(cA) = c \inf(A) \) by {\hyperref[Problem 2]{Problem 2}}.
            \end{proof}
        \item[9-10)] \( \inf(cA) = c \sup (A) \) if \( c < 0  \).
            \begin{proof}
            Let \( c < 0  \). We will show that \( \inf(cA) = c \sup(A) \). We will first show that \( cA  \) is bounded below by \( c \sup (A) \). Note that \( \sup(A) \) exists. Thus, \( a \leq \sup(A) \) for all \( a \in A  \). Since \( c < 0  \), \( ca \geq c \sup(A)  \) for all \( ca \in cA  \). Thus, \( c \sup(A) \) is a lower bound of \( cA \).
           
            Now, let \( \epsilon > 0 \). Since \( \sup(A) \) exists, there must exists an \( \omega \in A  \) such that  
            \[  \omega > \sup(A) + \frac{ \epsilon }{ c }. \]
            By multiplying through by \( c < 0  \), we have
            \[  c \omega < c \sup(A) +  \epsilon \]
            for some \( c \omega \in cA  \). Thus, we conclude that \( \inf(cA) = c \sup(A) \) by {\hyperref[Problem 3]{Problem 3}}.
            \end{proof}
        \item[9-11)] Is it true that \( \sup(A \cdot B) = \sup(A) \cdot \sup(B) \).
            \begin{proof}
            This is false. Consider the sets \( A = \{ 1, 2, 4, 8  \}  \) and \( B = \{ -5,-2, -3 , -1 \}  \). Then \( AB = \{ -5, -4, -12, -8 \}  \). Observe that \( \sup(A) \cdot \sup(B) = 8 \cdot -1 = -8 \), but \( \sup(AB) = -5 \neq -8 = \sup(A) \cdot \sup(B)  \). 
            \end{proof}
    \end{enumerate}
\item Recall that \( \Q  \) is a field; in particular, \( \Q  \) is closed under addition and multiplication. 
    \begin{enumerate}
        \item[10-1)] Prove that if \( p \in \Q  \) and \( t \) is an irrational number, then \(  p + t  \) is an irrational number.
            \begin{proof}
            Suppose \( p \in \Q  \) and \( t  \) is an irrational number. Suppose for sake of contradiction that \(  p + t  \) is an rational number. Then there exists \( x,y \in \Z  \) with \( y \neq 0  \) such that 
            \[  p + t = \frac{ x }{ y }. \]
           Subtracting \( p  \) on both sides, we get 
           \[  t = \frac{ x }{ y }  - p. \]
           Since \( \Q  \) is a field and \( \frac{ x }{ y } , p \in \Q  \), we get that \( \frac{ x }{ y } - p \in \Q    \). But this tells us that \( t  \) is a rational number which is a contradiction. Thus, \( t  \) must be irrational.
            \end{proof}
        \item[10-2)] Complete the following proof. \label{10-2)}
            \begin{theorem}
                Given any two real numbers \( x < y  \), there exists an irrational number \( t  \) satisfying \( x < t < y  \).
            \end{theorem}
            \begin{proof}
            It follows from \( x < y  \) that \( x - \sqrt{ 2 }  < y - \sqrt{ 2 }  \). Since \( \Q  \) is dense in \( \R  \), there exists \( p \in \Q  \) such that \( x - \sqrt{ 2 }  < p < y - \sqrt{ 2 }  \). Adding \( \sqrt{ 2 }  \) on both sides gives us   
            \[  x < p + \sqrt{ 2 } < y. \]
            Since \( p \in \Q   \) and \( \sqrt{ 2 }  \) is irrational, we get that \( t = p + \sqrt{ 2 }  \) is irrational from {\hyperref[10-2)]{10-2)}}. Thus, \( x < t < y  \) for some irrational number \( t  \).

            \end{proof}
    \end{enumerate}
    \item Prove the following:
        \begin{theorem}[Nested Interval Property]
            For each \( n \in \N  \), assume we are given a closed interval \( {I}_{n} = [{a}_{n}, {b}_{n}] = \{ x \in \R : {a}_{n} \leq x \leq {b}_{n} \}  \). Assume also that each \( {I}_{n} \) contains \( {I}_{n+1} \). Then, the resulting nested sequence of closed intervals
            \[  {I}_{1} \supseteq {I}_{2} \supseteq {I}_{3} \supseteq {I}_{4} \supseteq \cdots \]
            has a nonempty intersection, that is \( \cap_{n}^{\infty } {I}_{n} \neq \emptyset \).
        \end{theorem}
        \begin{proof}
            Let \( A = \{ {a}_{n} : n \in \N  \}  \). Note that \( A \neq \emptyset  \) because \( {a}_{1} \in A  \). Since \( {I}_{n} = [{a}_{n}, {b}_{n}]  \) for all \( n \in \N  \) and that each \( {I}_{n+1} \) is contained within each \( {I}_{n} \) , we see that each \( {a}_{n} \in A  \) must be bounded above by \( {b}_{n} \) for all \( n \in \N  \). Since \( A \neq \emptyset  \) and bounded above, there must exists a number \( x  \) such that \( x = \sup(A) \). Since \( x  \) is an upper bound of \( A  \), we have \( {a}_{n} \leq x \leq {b}_{n}  \). Thus, \( x \in {I}_{n}  \) for all \( n \in \N  \) which means that  
            \[  x \in \bigcap_{ n=1  }^{ \infty  }  {I}_{n}, \]
            proving that 
            \[  \bigcap_{ n = 1  }^{ \infty  } {I}_{n} \neq \emptyset. \]
        \end{proof}
    \item Complete the following proof:

        Prove that \( \bigcap_{ n = 1  }^{ \infty  } (0, \frac{ 1 }{ n } ) = \emptyset \). 
        \begin{proof}
        Suppose for sake of contradiction \( \bigcap_{ n=1  }^{ \infty  } (  0, 1/n   ) \neq \emptyset  \). So there exists \( x \in \bigcap_{ n=1 }^{ \infty  } (0, 1/n) \). This implies that \( x > 0  \) and \( x < 1/n \) for all \( n \in \N  \). By the Archimedean Property, there exists an \( {n}_{0} \in \N  \) such that   
        \[  \frac{ 1 }{ {n}_{0} }  < x.   \]
        But this contradicts our assumption that \( x < 1 /n   \) holds for all \( n \in \N  \). Thus, the intersection
        \[  \bigcap_{ n=1  }^{ \infty  }  \Big(  0 , \frac{ 1 }{ n }  \Big)  \]
        must be empty.
        \end{proof}
\end{enumerate}



\end{document}

