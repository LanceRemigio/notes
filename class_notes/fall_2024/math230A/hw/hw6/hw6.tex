\documentclass[a4paper]{article}
\usepackage{standalone}
\usepackage{import}
\usepackage[utf8]{inputenc}
\usepackage[T1]{fontenc}
\usepackage{textcomp}
\usepackage{hyperref}
% \usepackage{fourier}
% \usepackage[dutch]{babel}
\usepackage{url}
% \usepackage{hyperref}
% \hypersetup{
%     colorlinks,
%     linkcolor={black},
%     citecolor={black},
%     urlcolor={blue!80!black}
% }
\usepackage{graphicx}
\usepackage{float}
\usepackage{booktabs}
\usepackage{enumitem}
% \usepackage{parskip}
\usepackage{emptypage}
\usepackage{subcaption}
\usepackage{multicol}
\usepackage[usenames,dvipsnames]{xcolor}

% \usepackage{cmbright}


\usepackage[margin=1in]{geometry}
\usepackage{amsmath, amsfonts, mathtools, amsthm, amssymb}
\usepackage{mathrsfs}
\usepackage{cancel}
\usepackage{bm}
\newcommand\N{\ensuremath{\mathbb{N}}}
\newcommand\R{\ensuremath{\mathbb{R}}}
\newcommand\Z{\ensuremath{\mathbb{Z}}}
\renewcommand\O{\ensuremath{\emptyset}}
\newcommand\Q{\ensuremath{\mathbb{Q}}}
\newcommand\C{\ensuremath{\mathbb{C}}}
\DeclareMathOperator{\sgn}{sgn}
\usepackage{systeme}
\let\svlim\lim\def\lim{\svlim\limits}
\let\implies\Rightarrow
\let\impliedby\Leftarrow
\let\iff\Leftrightarrow
\let\epsilon\varepsilon
\usepackage{stmaryrd} % for \lightning
\newcommand\contra{\scalebox{1.1}{$\lightning$}}
% \let\phi\varphi
\renewcommand\qedsymbol{$\blacksquare$}




% correct
\definecolor{correct}{HTML}{009900}
\newcommand\correct[2]{\ensuremath{\:}{\color{red}{#1}}\ensuremath{\to }{\color{correct}{#2}}\ensuremath{\:}}
\newcommand\green[1]{{\color{correct}{#1}}}



% horizontal rule
\newcommand\hr{
    \noindent\rule[0.5ex]{\linewidth}{0.5pt}
}


% hide parts
\newcommand\hide[1]{}



% si unitx
\usepackage{siunitx}
\sisetup{locale = FR}
% \renewcommand\vec[1]{\mathbf{#1}}
\newcommand\mat[1]{\mathbf{#1}}


% tikz
\usepackage{tikz}
\usepackage{tikz-cd}
\usetikzlibrary{intersections, angles, quotes, calc, positioning}
\usetikzlibrary{arrows.meta}
\usepackage{pgfplots}
\pgfplotsset{compat=1.13}


\tikzset{
    force/.style={thick, {Circle[length=2pt]}-stealth, shorten <=-1pt}
}

% theorems
\makeatother
\usepackage{thmtools}
\usepackage[framemethod=TikZ]{mdframed}
\mdfsetup{skipabove=1em,skipbelow=0em}


\theoremstyle{definition}

\declaretheoremstyle[
    headfont=\bfseries\sffamily\color{ForestGreen!70!black}, bodyfont=\normalfont,
    mdframed={
        linewidth=2pt,
        rightline=false, topline=false, bottomline=false,
        linecolor=ForestGreen, backgroundcolor=ForestGreen!5,
    }
]{thmgreenbox}

\declaretheoremstyle[
    headfont=\bfseries\sffamily\color{NavyBlue!70!black}, bodyfont=\normalfont,
    mdframed={
        linewidth=2pt,
        rightline=false, topline=false, bottomline=false,
        linecolor=NavyBlue, backgroundcolor=NavyBlue!5,
    }
]{thmbluebox}

\declaretheoremstyle[
    headfont=\bfseries\sffamily\color{NavyBlue!70!black}, bodyfont=\normalfont,
    mdframed={
        linewidth=2pt,
        rightline=false, topline=false, bottomline=false,
        linecolor=NavyBlue
    }
]{thmblueline}

\declaretheoremstyle[
    headfont=\bfseries\sffamily\color{RawSienna!70!black}, bodyfont=\normalfont,
    mdframed={
        linewidth=2pt,
        rightline=false, topline=false, bottomline=false,
        linecolor=RawSienna, backgroundcolor=RawSienna!5,
    }
]{thmredbox}

\declaretheoremstyle[
    headfont=\bfseries\sffamily\color{RawSienna!70!black}, bodyfont=\normalfont,
    numbered=no,
    mdframed={
        linewidth=2pt,
        rightline=false, topline=false, bottomline=false,
        linecolor=RawSienna, backgroundcolor=RawSienna!1,
    },
    qed=\qedsymbol
]{thmproofbox}

\declaretheoremstyle[
    headfont=\bfseries\sffamily\color{NavyBlue!70!black}, bodyfont=\normalfont,
    numbered=no,
    mdframed={
        linewidth=2pt,
        rightline=false, topline=false, bottomline=false,
        linecolor=NavyBlue, backgroundcolor=NavyBlue!1,
    },
]{thmexplanationbox}

\declaretheorem[style=thmgreenbox, numberwithin = section, name=Definition]{definition}
\declaretheorem[style=thmbluebox, name=Example]{eg}
\declaretheorem[style=thmredbox, numberwithin = section, name=Proposition]{prop}
\declaretheorem[style=thmredbox, numberwithin = section, name=Theorem]{theorem}
\declaretheorem[style=thmredbox, numberwithin = section,  name=Lemma]{lemma}
\declaretheorem[style=thmredbox, numberwithin = section,  numbered=no, name=Corollary]{corollary}


\declaretheorem[style=thmproofbox, name=Proof]{replacementproof}
\renewenvironment{proof}[1][\proofname]{\vspace{-10pt}\begin{replacementproof}}{\end{replacementproof}}


\declaretheorem[style=thmexplanationbox, name=Proof]{tmpexplanation}
\newenvironment{explanation}[1][]{\vspace{-10pt}\begin{tmpexplanation}}{\end{tmpexplanation}}


\declaretheorem[style=thmblueline, numbered=no, name=Remark]{remark}
\declaretheorem[style=thmblueline, numbered=no, name=Note]{note}

\newtheorem*{uovt}{UOVT}
\newtheorem*{notation}{Notation}
\newtheorem*{previouslyseen}{As previously seen}
\newtheorem*{problem}{Problem}
\newtheorem*{observe}{Observe}
\newtheorem*{property}{Property}
\newtheorem*{intuition}{Intuition}


\usepackage{etoolbox}
\AtEndEnvironment{vb}{\null\hfill$\diamond$}%
\AtEndEnvironment{intermezzo}{\null\hfill$\diamond$}%
% \AtEndEnvironment{opmerking}{\null\hfill$\diamond$}%

% http://tex.stackexchange.com/questions/22119/how-can-i-change-the-spacing-before-theorems-with-amsthm
\makeatletter
% \def\thm@space@setup{%
%   \thm@preskip=\parskip \thm@postskip=0pt
% }
\newcommand{\oefening}[1]{%
    \def\@oefening{#1}%
    \subsection*{Oefening #1}
}

\newcommand{\suboefening}[1]{%
    \subsubsection*{Oefening \@oefening.#1}
}

\newcommand{\exercise}[1]{%
    \def\@exercise{#1}%
    \subsection*{Exercise #1}
}

\newcommand{\subexercise}[1]{%
    \subsubsection*{Exercise \@exercise.#1}
}


\usepackage{xifthen}

\def\testdateparts#1{\dateparts#1\relax}
\def\dateparts#1 #2 #3 #4 #5\relax{
    \marginpar{\small\textsf{\mbox{#1 #2 #3 #5}}}
}

\def\@lesson{}%
\newcommand{\lesson}[3]{
    \ifthenelse{\isempty{#3}}{%
        \def\@lesson{Lecture #1}%
    }{%
        \def\@lesson{Lecture #1: #3}%
    }%
    \subsection*{\@lesson}
    \testdateparts{#2}
}

% \renewcommand\date[1]{\marginpar{#1}}


% fancy headers
\usepackage{fancyhdr}
\pagestyle{fancy}

\fancyhead[LE,RO]{Lance Remigio}
\fancyhead[RO,LE]{\@lesson}
\fancyhead[RE,LO]{}
\fancyfoot[LE,RO]{\thepage}
\fancyfoot[C]{\leftmark}

\makeatother




% notes
\usepackage{todonotes}
\usepackage{tcolorbox}

\tcbuselibrary{breakable}
\newenvironment{verbetering}{\begin{tcolorbox}[
    arc=0mm,
    colback=white,
    colframe=green!60!black,
    title=Opmerking,
    fonttitle=\sffamily,
    breakable
]}{\end{tcolorbox}}

\newenvironment{noot}[1]{\begin{tcolorbox}[
    arc=0mm,
    colback=white,
    colframe=white!60!black,
    title=#1,
    fonttitle=\sffamily,
    breakable
]}{\end{tcolorbox}}




% figure support
\usepackage{import}
\usepackage{xifthen}
\pdfminorversion=7
\usepackage{pdfpages}
\usepackage{transparent}
\newcommand{\incfig}[1]{%
    \def\svgwidth{\columnwidth}
    \import{./figures/}{#1.pdf_tex}
}

% %http://tex.stackexchange.com/questions/76273/multiple-pdfs-with-page-group-included-in-a-single-page-warning
\pdfsuppresswarningpagegroup=1



% \usepackage{fourier}

\pagestyle{fancy} \fancyhf{}

\title{Math 230A: Homework 6}
\author{Lance Remigio}

\begin{document}
\maketitle    
\lhead{Math 230A: Homework 5}
\chead{Lance Remigio}
\rhead{\thepage}

\begin{problem}
    In class we proved that every \( 2- \)cell is a compact subset of \( \R^{2} \). Use the same procedure to prove that every closed and bounded interval \( [a,b] \) (that is, every \( 1- \)cell) is a compact subset of \( \R  \).
\end{problem}



\newpage

\ 

\newpage


\begin{problem}
    Complete the proof of Theorem 2.41, that is, let \( E \subseteq  \R^{k} \) and prove that if every infinite subset of \( E  \) has a limit point in \( E  \), then \( E  \) is closed and bounded.
\end{problem}

\newpage

\ 

\newpage
\begin{problem}
    Give an example of a metric space \( (Y,d) \) and a set \( E \subseteq Y \) where \( E  \) is closed and bounded but not compact.
\end{problem}

\newpage

\

\newpage


\begin{problem}
    Prove that a discrete space is \textbf{totally disconnected}. That is, prove that in a metric space equipped with the discrete metric, the only subsets are singletons and \( \emptyset \).
\end{problem}
\newpage

\

\newpage
\begin{problem}
    Let \( E \subseteq \R .  \) Prove that \( E  \) is connected if and only if it has the following property. If \( x \in E  \) and \( y \in E  \), and \( x < z < y  \), then \( z \in E  \).
\end{problem}
\newpage

\

\newpage
\begin{problem}
    Let \( (X,d) \) be a metric space. Let \( A  \) and \( B  \) be two connected subsets of \( X  \). Prove that if \( A \cap B \neq \emptyset  \), then \( A \cup B  \) is connected.
\end{problem}
\newpage

\

\newpage
\begin{problem}
    Recall that in class we mentioned that path-connectedness implies connectedness. Hence, every circle in \( \R^{2} \) is a connected subset of \( \R^{2} \). Use this fact to show that intersection of two connected sets in \( \R^{2} \) is not necessarily connected.
\end{problem}
\newpage 

\

\newpage
\begin{problem}
    \begin{enumerate}
        \item[(a)] If \( A  \) and \( B  \) are disjoint closed sets in some metric space \( (X,d) \), prove that they are separated. 
        \item[(b)] Prove the same for disjoint open sets.
        \item[(c)] Fix \( p \in X  \), \( \delta > 0  \), define \( A  \) to be the set of all \( q \in X  \) for which \( d(p,q) < \delta \), define \( B  \) similarly, with \( >  \) in place of \( <  \). Prove that \( A  \) and \( B  \) are separated.
        \item[(d)] Prove that every connected metric space with at least two points is uncountable. (Hint: Use (c)).
    \end{enumerate}
\end{problem}




\newpage 

\

\newpage




\begin{problem}
    Recall that in class we mentioned that implies connectedness. Hence, every closed disk in \( \R^{2} \) is a connected subset of \( \R^{2} \). Use this fact to show that the interior of a connected set in \( \R^{2} \) is not necessarily connected.
\end{problem}

\newpage 

\

\newpage


\begin{problem}
    Let \( (X,d) \) be a metric space. Suppose that \( E \subseteq X  \) is connected. Prove that \( \overline{E} \) is connected.
\end{problem}
\newpage 

\

\newpage
\begin{problem}
    Let \( (X,d) \) be a metric space. Prove that \( X  \) is connected if and only if \( \emptyset \) are the only subsets of \( X  \) which are both open and closed.
\end{problem}
\newpage 

\

\newpage
\begin{problem}
    Let \( E  \) be the set of all \( x \in [0,1] \) whose decimal expansion contains only the digits \( 4  \) and \( 7  \). Is \( E   \) countable? Is \( E  \) compact? Is \( E  \) perfect? (For this problem, you may use the fact that if \( x = 0.{x}_{1} {x}_{2} \cdots {x}_{n} \cdots \) is an element of \( E  \), then \( x = \sum_{ k=1  }^{ \infty   } \frac{ {x}_{k} }{  10^{k} }  \)).
\end{problem}
\newpage

\

\newpage


\begin{problem}
    Let \( (X,d) \) be a metric space.
        \begin{enumerate}
            \item[(a)] A subset of \( A \subseteq  X  \) is called \textbf{nowhere dense in \( X  \)} if the interior of the closure of \( A  \) is empty, that is \( (\overline{A})^{\circ} = \emptyset \). Prove that \( A  \) is nowhere dense if and only if \( A^{c} \) contains a dense open set in \( X  \).
            \item[(b)] Prove that if \( \R^{k} = \bigcup_{ n=1  }^{ \infty    } {F}_{n} \) where each \( {F}_{n} \) is a closed subset of \( \R^{k }  \), then at least one \( {F}_{n} \) is NOT nowhere dense.
        \end{enumerate}
\end{problem}

\newpage

\

\newpage


\end{document}
