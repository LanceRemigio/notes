\documentclass[11pt,a4paper]{article} 
\usepackage{standalone}
\usepackage{import}
\usepackage[utf8]{inputenc}
\usepackage[T1]{fontenc}
\usepackage{textcomp}
\usepackage{hyperref}
% \usepackage{fourier}
% \usepackage[dutch]{babel}
\usepackage{url}
% \usepackage{hyperref}
% \hypersetup{
%     colorlinks,
%     linkcolor={black},
%     citecolor={black},
%     urlcolor={blue!80!black}
% }
\usepackage{graphicx}
\usepackage{float}
\usepackage{booktabs}
\usepackage{enumitem}
% \usepackage{parskip}
\usepackage{emptypage}
\usepackage{subcaption}
\usepackage{multicol}
\usepackage[usenames,dvipsnames]{xcolor}

% \usepackage{cmbright}


\usepackage[margin=1in]{geometry}
\usepackage{amsmath, amsfonts, mathtools, amsthm, amssymb}
\usepackage{mathrsfs}
\usepackage{cancel}
\usepackage{bm}
\newcommand\N{\ensuremath{\mathbb{N}}}
\newcommand\R{\ensuremath{\mathbb{R}}}
\newcommand\Z{\ensuremath{\mathbb{Z}}}
\renewcommand\O{\ensuremath{\emptyset}}
\newcommand\Q{\ensuremath{\mathbb{Q}}}
\newcommand\C{\ensuremath{\mathbb{C}}}
\DeclareMathOperator{\sgn}{sgn}
\usepackage{systeme}
\let\svlim\lim\def\lim{\svlim\limits}
\let\implies\Rightarrow
\let\impliedby\Leftarrow
\let\iff\Leftrightarrow
\let\epsilon\varepsilon
\usepackage{stmaryrd} % for \lightning
\newcommand\contra{\scalebox{1.1}{$\lightning$}}
% \let\phi\varphi
\renewcommand\qedsymbol{$\blacksquare$}




% correct
\definecolor{correct}{HTML}{009900}
\newcommand\correct[2]{\ensuremath{\:}{\color{red}{#1}}\ensuremath{\to }{\color{correct}{#2}}\ensuremath{\:}}
\newcommand\green[1]{{\color{correct}{#1}}}



% horizontal rule
\newcommand\hr{
    \noindent\rule[0.5ex]{\linewidth}{0.5pt}
}


% hide parts
\newcommand\hide[1]{}



% si unitx
\usepackage{siunitx}
\sisetup{locale = FR}
% \renewcommand\vec[1]{\mathbf{#1}}
\newcommand\mat[1]{\mathbf{#1}}


% tikz
\usepackage{tikz}
\usepackage{tikz-cd}
\usetikzlibrary{intersections, angles, quotes, calc, positioning}
\usetikzlibrary{arrows.meta}
\usepackage{pgfplots}
\pgfplotsset{compat=1.13}


\tikzset{
    force/.style={thick, {Circle[length=2pt]}-stealth, shorten <=-1pt}
}

% theorems
\makeatother
\usepackage{thmtools}
\usepackage[framemethod=TikZ]{mdframed}
\mdfsetup{skipabove=1em,skipbelow=0em}


\theoremstyle{definition}

\declaretheoremstyle[
    headfont=\bfseries\sffamily\color{ForestGreen!70!black}, bodyfont=\normalfont,
    mdframed={
        linewidth=2pt,
        rightline=false, topline=false, bottomline=false,
        linecolor=ForestGreen, backgroundcolor=ForestGreen!5,
    }
]{thmgreenbox}

\declaretheoremstyle[
    headfont=\bfseries\sffamily\color{NavyBlue!70!black}, bodyfont=\normalfont,
    mdframed={
        linewidth=2pt,
        rightline=false, topline=false, bottomline=false,
        linecolor=NavyBlue, backgroundcolor=NavyBlue!5,
    }
]{thmbluebox}

\declaretheoremstyle[
    headfont=\bfseries\sffamily\color{NavyBlue!70!black}, bodyfont=\normalfont,
    mdframed={
        linewidth=2pt,
        rightline=false, topline=false, bottomline=false,
        linecolor=NavyBlue
    }
]{thmblueline}

\declaretheoremstyle[
    headfont=\bfseries\sffamily\color{RawSienna!70!black}, bodyfont=\normalfont,
    mdframed={
        linewidth=2pt,
        rightline=false, topline=false, bottomline=false,
        linecolor=RawSienna, backgroundcolor=RawSienna!5,
    }
]{thmredbox}

\declaretheoremstyle[
    headfont=\bfseries\sffamily\color{RawSienna!70!black}, bodyfont=\normalfont,
    numbered=no,
    mdframed={
        linewidth=2pt,
        rightline=false, topline=false, bottomline=false,
        linecolor=RawSienna, backgroundcolor=RawSienna!1,
    },
    qed=\qedsymbol
]{thmproofbox}

\declaretheoremstyle[
    headfont=\bfseries\sffamily\color{NavyBlue!70!black}, bodyfont=\normalfont,
    numbered=no,
    mdframed={
        linewidth=2pt,
        rightline=false, topline=false, bottomline=false,
        linecolor=NavyBlue, backgroundcolor=NavyBlue!1,
    },
]{thmexplanationbox}

\declaretheorem[style=thmgreenbox, numberwithin = section, name=Definition]{definition}
\declaretheorem[style=thmbluebox, name=Example]{eg}
\declaretheorem[style=thmredbox, numberwithin = section, name=Proposition]{prop}
\declaretheorem[style=thmredbox, numberwithin = section, name=Theorem]{theorem}
\declaretheorem[style=thmredbox, numberwithin = section,  name=Lemma]{lemma}
\declaretheorem[style=thmredbox, numberwithin = section,  numbered=no, name=Corollary]{corollary}


\declaretheorem[style=thmproofbox, name=Proof]{replacementproof}
\renewenvironment{proof}[1][\proofname]{\vspace{-10pt}\begin{replacementproof}}{\end{replacementproof}}


\declaretheorem[style=thmexplanationbox, name=Proof]{tmpexplanation}
\newenvironment{explanation}[1][]{\vspace{-10pt}\begin{tmpexplanation}}{\end{tmpexplanation}}


\declaretheorem[style=thmblueline, numbered=no, name=Remark]{remark}
\declaretheorem[style=thmblueline, numbered=no, name=Note]{note}

\newtheorem*{uovt}{UOVT}
\newtheorem*{notation}{Notation}
\newtheorem*{previouslyseen}{As previously seen}
\newtheorem*{problem}{Problem}
\newtheorem*{observe}{Observe}
\newtheorem*{property}{Property}
\newtheorem*{intuition}{Intuition}


\usepackage{etoolbox}
\AtEndEnvironment{vb}{\null\hfill$\diamond$}%
\AtEndEnvironment{intermezzo}{\null\hfill$\diamond$}%
% \AtEndEnvironment{opmerking}{\null\hfill$\diamond$}%

% http://tex.stackexchange.com/questions/22119/how-can-i-change-the-spacing-before-theorems-with-amsthm
\makeatletter
% \def\thm@space@setup{%
%   \thm@preskip=\parskip \thm@postskip=0pt
% }
\newcommand{\oefening}[1]{%
    \def\@oefening{#1}%
    \subsection*{Oefening #1}
}

\newcommand{\suboefening}[1]{%
    \subsubsection*{Oefening \@oefening.#1}
}

\newcommand{\exercise}[1]{%
    \def\@exercise{#1}%
    \subsection*{Exercise #1}
}

\newcommand{\subexercise}[1]{%
    \subsubsection*{Exercise \@exercise.#1}
}


\usepackage{xifthen}

\def\testdateparts#1{\dateparts#1\relax}
\def\dateparts#1 #2 #3 #4 #5\relax{
    \marginpar{\small\textsf{\mbox{#1 #2 #3 #5}}}
}

\def\@lesson{}%
\newcommand{\lesson}[3]{
    \ifthenelse{\isempty{#3}}{%
        \def\@lesson{Lecture #1}%
    }{%
        \def\@lesson{Lecture #1: #3}%
    }%
    \subsection*{\@lesson}
    \testdateparts{#2}
}

% \renewcommand\date[1]{\marginpar{#1}}


% fancy headers
\usepackage{fancyhdr}
\pagestyle{fancy}

\fancyhead[LE,RO]{Lance Remigio}
\fancyhead[RO,LE]{\@lesson}
\fancyhead[RE,LO]{}
\fancyfoot[LE,RO]{\thepage}
\fancyfoot[C]{\leftmark}

\makeatother




% notes
\usepackage{todonotes}
\usepackage{tcolorbox}

\tcbuselibrary{breakable}
\newenvironment{verbetering}{\begin{tcolorbox}[
    arc=0mm,
    colback=white,
    colframe=green!60!black,
    title=Opmerking,
    fonttitle=\sffamily,
    breakable
]}{\end{tcolorbox}}

\newenvironment{noot}[1]{\begin{tcolorbox}[
    arc=0mm,
    colback=white,
    colframe=white!60!black,
    title=#1,
    fonttitle=\sffamily,
    breakable
]}{\end{tcolorbox}}




% figure support
\usepackage{import}
\usepackage{xifthen}
\pdfminorversion=7
\usepackage{pdfpages}
\usepackage{transparent}
\newcommand{\incfig}[1]{%
    \def\svgwidth{\columnwidth}
    \import{./figures/}{#1.pdf_tex}
}

% %http://tex.stackexchange.com/questions/76273/multiple-pdfs-with-page-group-included-in-a-single-page-warning
\pdfsuppresswarningpagegroup=1




\pagestyle{fancy}
\fancyhf{}
\fancyhead[EL]{\nouppercase\leftmark}
\fancyhead[OR]{\nouppercase\rightmark}
\fancyhead[ER,OL]{\thepage}


\title{Math 230A: Homework 2}
\author{Lance Remigio}
\begin{document}
\maketitle

\section*{Problem 1} Mark each statement True or False.

\begin{enumerate}
    \item If \( Y  \) is a set and \( \{ {G}_{\alpha} : \alpha \in A  \}   \) is a family of subsets of \( Y  \), then \( \bigcup_{ \alpha \in A  }^{  }  {G}_{\alpha} \) is a subset of \( Y  \). \textbf{True.}
    \item The set \( \{ q \in \Q : q^{2} < 2  \}  \) is countable. \textbf{False.}     
    \item Every infinite subset of an uncountable set is uncountable. \textbf{False.} 
    \item The union of any collection of at most countable sets is at most countable. \textbf{True.}
    \item If \( A,B \subseteq \R  \) and \( A  \) is countable, then \( A \cap B  \) is at most countable. \textbf{True.}
\end{enumerate}

\section*{Problem 2} Find a bijection from \( \N  \) to the set of odd integers.
\begin{solution}
Define \( f: \N \to \{ 2n + 1: n \in \N \}  \) as \( f(n) = 2n + 1  \) for \( n \in \N \). 
\end{solution}

\section*{Problem 3} For the following problem, you may use your grade school knowledge that every real number has a decimal representation. Although there is a careful description of decimal representation on page 11 of Rudin, it will not be needed to solve this problem. 


Recall that in class we proved that the set of all binary sequences is uncountable. Use that fact to show that interval \( [0,0.2) \) is uncountable. Explain how this result proves that \( [0,1] \) and \( \R  \) are countable as well.
\begin{proof}

\end{proof}

\section*{Problem 4} Is the set of all irrational numbers countable?
\begin{solution}
False. It is uncountable.
\end{solution}

\section*{Problem 5} Show any collection of disjoint intervals of positive length is at most countable.
\begin{proof}
Let \( A  \) be the collection of disjoint intervals of positive length. Suppose for sake of contradiction that this collection of intervals is uncountable. However, notice how each of the disjoint interval contains a rational number \( q  \) by the density of \( \Q  \) in \( \R \). Since \( A  \) is uncountable, this means that there are an uncountable number of rational points. But we know that \( \Q  \) is countable which is a contradiction.
\end{proof}

\section*{Problem 6} A real number is said to be algebraic if it is a root of some nonzero polynomial with rational coefficients. Prove that the set of all algebraic numbers is countable.
\begin{proof}
Define the set
\[  \alpha = \{ p(x) = 0 : x \in \R  \}  \] with \( p(x)  \) being any polynomial with rational coefficients. Let  
\[  \bigcup_{ \alpha  }^{  }  \alpha  \] be the union of all \( \alpha  \) representing the set of all real algebraic numbers coming from \( p(x) \). Since \( p(x) \) is of \( n \)th degree, we know that \( p(x) \) must contain a finite number of roots, namely, \( n  \) number of them. Thus, \( \alpha  \) is countable and hence, the union of all \( \alpha  \) must also be countable.  
\end{proof}

\section*{Problem 7} Prove that not all real numbers are algebraic.
\begin{proof}
    Suppose for sake of contradiction that all real numbers are algebraic. Let \( A  \) be the set of all real numbers that are algebraic. By the last problem, this set is countable. However, \( \R  \) is an uncountable set by Problem 3 which is a contradiction. Thus, not all real numbers are algebraic.  
\end{proof}

\section*{Problem 8} Choose (specific) distinct real numbers \( a,b,c,d  \) such that \( a < b  \) and \( c < d  \). Prove that \( \card{[a,b]} = \card{[c,d]}  \). (Hint: Find a linear polynomial \( f  \) such that \( f(a) = c  \) and \( f(b) = d  \).)
\begin{proof}

\end{proof}

\section*{Problem 9} Consider the function \( f: (0,1) \to \R  \) given by 
\[  f(x) = 
\begin{cases}
    \frac{ 1 }{ n - 1  }  &\text{if} \ x = \frac{ 1 }{ n }, \  \text{for some} \ n \in \N \\ 
    x &\text{otherwise}
\end{cases}. 
\]
\begin{enumerate}
    \item Prove that \( f  \) is a bijection from \( (0,1)  \) to \( (0,1] \).
        \begin{proof}
        
        \end{proof}
    \item Find a bijection from \( (0,1]  \) to \( [0,1) \).
        \begin{solution}
            
        \end{solution}
    \item Find a bijection from \(  [0,1)  \) to \( [0,1] \).
        \begin{solution}
        
        \end{solution}

\end{enumerate}


\section{Problem 10} Prove the following theorem.

\begin{theorem}[ ]
    Let \( A  \) and \( B  \) be two nonempty sets. If \( B \subseteq A  \) and there exists a one-to-one function \( f: A \to B  \), then \( A \sim B  \).
\end{theorem}
\begin{proof}

\end{proof}

\section{Problem 11} Prove the following theorem.
\begin{theorem}[Schroder-Bernstein]
    Let \( A  \) and \( B  \) be two nonempty sets. If there exists two one-to-one functions \( f: A \to B  \) and \( g: B \to A  \), then \( B \sim A  \).
\end{theorem}
\begin{proof}

\end{proof}

\section{Problem 12} Prove the following theorem.

\begin{theorem}[ ]
    Every infinite set contains a proper countable subset.
\end{theorem}
\begin{proof}

\end{proof}


\end{document}
