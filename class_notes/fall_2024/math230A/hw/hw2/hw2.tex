\documentclass[11pt,a4paper]{article} 
\usepackage{standalone}
\usepackage{import}
\usepackage[utf8]{inputenc}
\usepackage[T1]{fontenc}
% \usepackage{fourier}
\usepackage{textcomp}
\usepackage{hyperref}
\usepackage[english]{babel}
\usepackage{url}
% \usepackage{hyperref}
% \hypersetup{
%     colorlinks,
%     linkcolor={black},
%     citecolor={black},
%     urlcolor={blue!80!black}
% }
\usepackage{graphicx} \usepackage{float}
\usepackage{booktabs}
\usepackage{enumitem}
% \usepackage{parskip}
% \usepackage{parskip}
\usepackage{emptypage}
\usepackage{subcaption}
\usepackage{multicol}
\usepackage[usenames,dvipsnames]{xcolor}
\usepackage{ocgx}
% \usepackage{cmbright}


\usepackage[margin=1in]{geometry}
\usepackage{amsmath, amsfonts, mathtools, amsthm, amssymb}
\usepackage{thmtools}
\usepackage{mathrsfs}
\usepackage{cancel}
\usepackage{bm}
\newcommand\N{\ensuremath{\mathbb{N}}}
\newcommand\R{\ensuremath{\mathbb{R}}}
\newcommand\Z{\ensuremath{\mathbb{Z}}}
\renewcommand\O{\ensuremath{\emptyset}}
\newcommand\Q{\ensuremath{\mathbb{Q}}}
\newcommand\C{\ensuremath{\mathbb{C}}}
\newcommand\F{\ensuremath{\mathbb{F}}}
\DeclareMathOperator{\sgn}{sgn}
\DeclareMathOperator{\diam}{diam}
\DeclareMathOperator{\LO}{LO}
\DeclareMathOperator{\UP}{UP}
\DeclareMathOperator{\card}{card}
\DeclareMathOperator{\Arg}{Arg}
\DeclareMathOperator{\Dom}{Dom}
\DeclareMathOperator{\Log}{Log}
\DeclareMathOperator{\dist}{dist}
% \DeclareMathOperator{\span}{span}
\usepackage{systeme}
\let\svlim\lim\def\lim{\svlim\limits}
\renewcommand\implies\Longrightarrow
\let\impliedby\Longleftarrow
\let\iff\Longleftrightarrow
\let\epsilon\varepsilon
\usepackage{stmaryrd} % for \lightning
\newcommand\contra{\scalebox{1.1}{$\lightning$}}
% \let\phi\varphi
\renewcommand\qedsymbol{$\blacksquare$}

% correct
\definecolor{correct}{HTML}{009900}
\newcommand\correct[2]{\ensuremath{\:}{\color{red}{#1}}\ensuremath{\to }{\color{correct}{#2}}\ensuremath{\:}}
\newcommand\green[1]{{\color{correct}{#1}}}

% horizontal rule
\newcommand\hr{
    \noindent\rule[0.5ex]{\linewidth}{0.5pt}
}

% hide parts
\newcommand\hide[1]{}

% si unitx
\usepackage{siunitx}
\sisetup{locale = FR}
% \renewcommand\vec[1]{\mathbf{#1}}
\newcommand\mat[1]{\mathbf{#1}}

% tikz
\usepackage{tikz}
\usepackage{tikz-cd}
\usetikzlibrary{intersections, angles, quotes, calc, positioning}
\usetikzlibrary{arrows.meta}
\usepackage{pgfplots}
\pgfplotsset{compat=1.13}

\tikzset{
    force/.style={thick, {Circle[length=2pt]}-stealth, shorten <=-1pt}
}

% theorems
\makeatother
\usepackage{thmtools}
\usepackage[framemethod=TikZ]{mdframed}
\mdfsetup{skipabove=1em,skipbelow=1em}

\theoremstyle{definition}

\declaretheoremstyle[
    headfont=\bfseries\sffamily\color{ForestGreen!70!black}, bodyfont=\normalfont,
    mdframed={
        linewidth=1pt,
        rightline=false, topline=false, bottomline=false,
        linecolor=ForestGreen, backgroundcolor=ForestGreen!5,
    }
]{thmgreenbox}

\declaretheoremstyle[
    headfont=\bfseries\sffamily\color{NavyBlue!70!black}, bodyfont=\normalfont,
    mdframed={
        linewidth=1pt,
        rightline=false, topline=false, bottomline=false,
        linecolor=NavyBlue, backgroundcolor=NavyBlue!5,
    }
]{thmbluebox}

\declaretheoremstyle[
    headfont=\bfseries\sffamily\color{NavyBlue!70!black}, bodyfont=\normalfont,
    mdframed={
        linewidth=1pt,
        rightline=false, topline=false, bottomline=false,
        linecolor=NavyBlue
    }
]{thmblueline}

\declaretheoremstyle[
    headfont=\bfseries\sffamily, bodyfont=\normalfont,
    numbered = no,
    mdframed={
        rightline=true, topline=true, bottomline=true,
    }
]{thmbox}

\declaretheoremstyle[
    headfont=\bfseries\sffamily, bodyfont=\normalfont,
    numbered=no,
    % mdframed={
    %     rightline=true, topline=false, bottomline=true,
    % },
    qed=\qedsymbol
]{thmproofbox}

\declaretheoremstyle[
    headfont=\bfseries\sffamily\color{NavyBlue!70!black}, bodyfont=\normalfont,
    numbered=no,
    mdframed={
        rightline=false, topline=false, bottomline=false,
        linecolor=NavyBlue, backgroundcolor=NavyBlue!1,
    },
]{thmexplanationbox}

\declaretheorem[
    style=thmbox, 
    % numberwithin = section,
    numbered = no,
    name=Definition
    ]{definition}

\declaretheorem[
    style=thmbox, 
    name=Example,
    ]{eg}

\declaretheorem[
    style=thmbox, 
    % numberwithin = section,
    name=Proposition]{prop}

\declaretheorem[
    style = thmbox,
    numbered=yes,
    name =Problem
    ]{problem}

\declaretheorem[style=thmbox, name=Theorem]{theorem}
\declaretheorem[style=thmbox, name=Lemma]{lemma}
\declaretheorem[style=thmbox, name=Corollary]{corollary}

\declaretheorem[style=thmproofbox, name=Proof]{replacementproof}

\declaretheorem[style=thmproofbox, 
                name = Solution
                ]{replacementsolution}

\renewenvironment{proof}[1][\proofname]{\vspace{-1pt}\begin{replacementproof}}{\end{replacementproof}}

\newenvironment{solution}
    {
        \vspace{-1pt}\begin{replacementsolution}
    }
    { 
            \end{replacementsolution}
    }

\declaretheorem[style=thmexplanationbox, name=Proof]{tmpexplanation}
\newenvironment{explanation}[1][]{\vspace{-10pt}\begin{tmpexplanation}}{\end{tmpexplanation}}

\declaretheorem[style=thmbox, numbered=no, name=Remark]{remark}
\declaretheorem[style=thmbox, numbered=no, name=Note]{note}

\newtheorem*{uovt}{UOVT}
\newtheorem*{notation}{Notation}
\newtheorem*{previouslyseen}{As previously seen}
% \newtheorem*{problem}{Problem}
\newtheorem*{observe}{Observe}
\newtheorem*{property}{Property}
\newtheorem*{intuition}{Intuition}

\usepackage{etoolbox}
\AtEndEnvironment{vb}{\null\hfill$\diamond$}%
\AtEndEnvironment{intermezzo}{\null\hfill$\diamond$}%
% \AtEndEnvironment{opmerking}{\null\hfill$\diamond$}%

% http://tex.stackexchange.com/questions/22119/how-can-i-change-the-spacing-before-theorems-with-amsthm
\makeatletter
% \def\thm@space@setup{%
%   \thm@preskip=\parskip \thm@postskip=0pt
% }
\newcommand{\oefening}[1]{%
    \def\@oefening{#1}%
    \subsection*{Oefening #1}
}

\newcommand{\suboefening}[1]{%
    \subsubsection*{Oefening \@oefening.#1}
}

\newcommand{\exercise}[1]{%
    \def\@exercise{#1}%
    \subsection*{Exercise #1}
}

\newcommand{\subexercise}[1]{%
    \subsubsection*{Exercise \@exercise.#1}
}


\usepackage{xifthen}

\def\testdateparts#1{\dateparts#1\relax}
\def\dateparts#1 #2 #3 #4 #5\relax{
    \marginpar{\small\textsf{\mbox{#1 #2 #3 #5}}}
}

\def\@lesson{}%
\newcommand{\lesson}[3]{
    \ifthenelse{\isempty{#3}}{%
        \def\@lesson{Lecture #1}%
    }{%
        \def\@lesson{Lecture #1: #3}%
    }%
    \subsection*{\@lesson}
    \testdateparts{#2}
}

% \renewcommand\date[1]{\marginpar{#1}}


% fancy headers
\usepackage{fancyhdr}
\pagestyle{fancy}

\makeatother

% notes
\usepackage{todonotes}
\usepackage{tcolorbox}

\tcbuselibrary{breakable}
\newenvironment{verbetering}{\begin{tcolorbox}[
    arc=0mm,
    colback=white,
    colframe=green!60!black,
    title=Opmerking,
    fonttitle=\sffamily,
    breakable
]}{\end{tcolorbox}}

\newenvironment{noot}[1]{\begin{tcolorbox}[
    arc=0mm,
    colback=white,
    colframe=white!60!black,
    title=#1,
    fonttitle=\sffamily,
    breakable
]}{\end{tcolorbox}}

% figure support
\usepackage{import}
\usepackage{xifthen}
\pdfminorversion=7
\usepackage{pdfpages}
\usepackage{transparent}
\newcommand{\incfig}[1]{%
    \def\svgwidth{\columnwidth}
    \import{./figures/}{#1.pdf_tex}
}

% %http://tex.stackexchange.com/questions/76273/multiple-pdfs-with-page-group-included-in-a-single-page-warning
\pdfsuppresswarningpagegroup=1



\pagestyle{fancy}
\fancyhf{}
\fancyhead[EL]{\nouppercase\leftmark}
\fancyhead[OR]{\nouppercase\rightmark}
\fancyhead[C]{Homework 2}
\fancyhead[ER,OL]{\thepage}

% 1, 2, 3, 4, 5, 6, 7, 8, 9, 10, 11, 12, 13, 14, 15, 16, 17


\title{Math 230A: Homework 2}
\author{Lance Remigio}
\begin{document}
\maketitle

\section*{Problem 1} Mark each statement True or False.

\begin{enumerate} 
    \item If \( Y  \) is a set and \( \{ {G}_{\alpha} : \alpha \in A  \}   \) is a family of subsets of \( Y  \), then \( \bigcup_{ \alpha \in A  }^{  }  {G}_{\alpha} \) is a subset of \( Y  \). \textbf{True.}
    \item The range of every sequence is at most countable. \textbf{True.}
    \item The set \( \{ q \in \Q : q^{2} < 2  \}  \) is countable. \textbf{True.}     
    \item Every infinite subset of an uncountable set is uncountable. \textbf{False.} 
    \item The union of any collection of at most countable sets is at most countable. \textbf{False.}
    \item If \( A,B \subseteq \R  \) and \( A  \) is countable, then \( A \cap B  \) is at most countable. \textbf{True.}
\end{enumerate}

\section*{Problem 2} Find a bijection from \( \N  \) to the set of odd integers.
\begin{solution}
    Define the following bijection \( f: \N \to \{ \text{odd integers}  \}  \) as
    \[  f(n) = 
    \begin{cases}
        n &\text{if} \ n \ \text{is odd} \\   
        -n + 1 &\text{if} \  n \ \text{is} \ \text{even}.
    \end{cases} \]
\end{solution}

\section*{Problem 3} For the following problem, you may use your grade school knowledge that every real number has a decimal representation. Although there is a careful description of decimal representation on page 11 of Rudin, it will not be needed to solve this problem. 

Recall that in class we proved that the set of all binary sequences is uncountable. Use that fact to show that interval \( [0,0.2) \) is uncountable. Explain how this result proves that \( [0,1] \) and \( \R  \) are uncountable as well.

\begin{proof}
Consider the set of binary sequences
\[ A = \{ n \in \N : ({a}_{n})  \} \]
 Suppose we have the map 
\[  f: A   \to [0,0.2). \]
Note that \( f  \) is injective since for any \( ({a}_{1}, {a}_{2}, \dots ), ({b}_{1}, {b}_{2}, \dots) \in A   \), we have   
\begin{align*}
    f(({a}_{1}, {a}_{2}, \dots)) &= f(({b}_{1}, {b}_{2}, \dots )) \\
    0. {a}_{1} {a}_{2} \dots &= 0. {b}_{1} {b}_{2} \dots \ . 
\end{align*}
Hence, \( ({a}_{1}, {a}_{2}, \dots ) = ( {b}_{1}, {b}_{2}, \dots ) \). Thus, \( f  \) is injective.
Consider the proper subset \( f(A)  \) of \( [0,0.2) \). Recall that \( A  \) is uncountable and so, \( f(A)  \) is uncountable since \( f: A \to [0,0.2) \) is uncountable. Thus, \( [0,0.2) \) is also uncountable. 

Note that \( [0,0,2) \subseteq [0,1] \). But \( [0,0,2) \) is an uncountable set. Thus, \( [0,1] \) is uncountable. Similarly, we see that \( [0,1] \subseteq  \R  \). Since \( [0,1] \) is uncountable, we see that \( \R  \) is uncountable.  
\end{proof}

\section*{Problem 4} Is the set of all irrational numbers countable?
\begin{solution}
    Suppose for sake of contradiction that the set of all irrational numbers \( \mathbb{I} \) is countable. Note that \( \Q  \) is also countable and subsequently that the union \( \mathbb{I} \cup \Q = \R   \) is countable. However, \( \R  \) is uncountable from Problem 3 and so we have a contradiction. Thus, the set of irrational numbers must uncountable.  
\end{solution}

\section*{Problem 5} Show any collection of disjoint intervals of positive length is at most countable.
\begin{proof}
    Let \( \{ {I}_{\alpha} \}_{\alpha \in \Lambda} \) be a collection of disjoint intervals of positive length. By the Density of \( \Q  \) in \( \R  \), for each \( \alpha \in \Lambda \), there exists a \( {q}_{\alpha} \in \Q  \) such that \( {q}_{\alpha} \in {I}_{\alpha} \). Define the map \( f: \{ {I}_{\alpha} \}_{\alpha \in \Lambda} \to \Q  \) by \( f({I}_{\alpha}) = {q}_{\alpha} \). Since every interval \( {I}_{\alpha} \) is disjoint, we must have that \( f  \) is injective. If \( f  \) is injective, then the map \( f: \{ {I}_{\alpha} \}_{\alpha \in \Lambda} \to f(\{ {I}_{\alpha} \}_{\Lambda}) \) is bijective. Thus, \[ \{ {I}_{\alpha} \}_{ \alpha \in \Lambda} \sim f(\{ {I}_{\alpha} \}_{ \alpha \in \Lambda}). \]
    Furthermore, we see that \( f(\{ {I}_{\alpha} \}_{\alpha \in \Lambda}) \subseteq  \Q  \). Since \( \Q  \) is countable, we know that \( f(\{ {I}_{\alpha} \}_{\alpha \in \Lambda}) \) is at most countable. Therefore, we conclude that \( \{ {I}_{\alpha} \}_{\alpha \in \Lambda} \) is at most countable.
\end{proof}

\section*{Problem 6} A real number is said to be algebraic if it is a root of some nonzero polynomial with rational coefficients. Prove that the set of all algebraic numbers is countable. Define the map .
\begin{proof}
    Let 
    \[  A = \text{the set of all (real) algebraic numbers}.  \]
    For all \( n \geq 1  \), let 
    \[  \hat{P}_{n} = \text{the set of all polynomials of degree exactly \( n \) with rational coefficients}  \]
    and that
    \[  {B}_{n} = \{ \alpha \in \R :  \text{there exists} \ f \in {\hat{P}}_{n} \ \text{with} \ f(\alpha) = 0  \}  \]
    to be the collection of all polynomials in \( {\hat{P}}_{n} \).
    Also, for any polynomial \( f  \), we let 
    \[  R(f) = \{ \alpha \in \R : f(\alpha) = 0 \} \]
    to be the collection of all roots of \( f \). Here are some observations 
    \begin{enumerate}
        \item[(i)] \( A = \bigcup_{ n=1  }^{ \infty  }  {B}_{n} \)
        \item[(ii)] For all \( n \geq 1  \) and for all \( f \in {\hat{P}}_{n} \), we have that \( R(f) \) is at most countable.
    \end{enumerate}
    We will show the following statements:
    \begin{enumerate}
        \item[(1)] We will show that for each \( n \geq 1  \), \( {\hat{P}}_{n} \) is countable.
        \item[(2)] Using (1) and (ii), we will show that \( {B}_{n}  \) is at most countable for each \( n \geq 1  \). 
        \item[(3)] Using (2) and (i), we will show that \( A  \) is at most countable.
        \item[(4)] Show that \( A  \) is infinite.
    \end{enumerate}

    First, we will show fact (1).

    Let \( n \geq 1  \). Define the function \( g: {\hat{P}}_{n} \to \Q^{n+1} \) by
    \[  g({a}_{n} x^{n} + {a}_{n-1} x^{n-1} + \cdots + {a}_{1} x + {a}_{0}) = ({a}_{n}, {a}_{n-1}, \dots, {a}_{0}). \] 
    Clearly, this function is 1-1. So, \( g: {\hat{P}}_{n} \to g({\hat{P}}_{n}) \) is bijective. Thus, we have that \( {\hat{P}}_{n} \sim g({\hat{P}}_{n}) \) and \( g({\hat{P}}_{n}) \subseteq \Q^{n+1} \). Note that \( \Q  \) is countable implies that the product \( \Q^{n+1} \) is countable. This implies that \( g({\hat{P}}_{n}) \) is at most countable. Now, we see that \( j: \N \to {\hat{P}}_{n} \) defined by \( h(m) = m x^{n} \) is a 1-1 function. Hence, we see that \( h(\N) \) is an infinite subset of \( {\hat{P}}_{n} \). So, \( {\hat{P}}_{n} \) must be infinite. Since \( {\hat{P}}_{n} \) is at most countable and \( {\hat{P}}_{n} \) is infinite, we must have that \( {\hat{P}}_{n} \) is countable. 

    Next, we will show that for all \( n \geq 1  \), \( {B}_{n}  \) is at most countable. Let \( n \geq 1  \). Since \( {\hat{P}}_{n} \) is countable, we can enumerate its elements in the following way:
    \[  {\hat{P}}_{n} = \{ {f}_{1}^{n}, {f}_{2}^{n}, \dots  \}. \]
    We have 
    \[  {B}_{n} = \bigcup_{ f \in {\hat{P}}_{n} }^{  }  R(f) = \bigcup_{ m=1  }^{ \infty  }  R({f}_{m}^{n}). \]
    For each \( m  \), we see that \( R({f}_{m}^{n}) \) is finite (and thus, at most countable). Since a countable union of at most countable sets is at most countable, we can conclude that \( {B}_{n} \) is at most countable. 

    Next, we will show that \( A  \) is at most countable. Since each \( {B}_{n} \) for all \( n \geq 1  \) is at most countable and the fact that
    \[  A = \bigcup_{ n = 1  }^{ \infty  }  {B}_{n}, \]
    we have that \( A  \) is at most countable (A countable union of at most countable sets is at most countable). 

    Lastly, we will show that \( A  \) is infinite. For each \( \alpha \in \Q  \), clearly, \( \alpha  \) is a root of the polynomial \( x - \alpha  \) (which is a polynomial with rational coefficients). Thus, each rational number is algebraic. Since \( \Q \subseteq  A   \) and \( \Q  \) is countable, we know that \( \Q  \) is infinite. Thus, \( A  \) must be infinite, and thus countable.    
\end{proof} 

% Have a mapping f that sends each polynomial with rational coefficients to a finite product of rational numbers. This mapping is injective.  

\section*{Problem 7} Prove that not all real numbers are algebraic.
\begin{proof}
    Suppose for sake of contradiction that all real numbers are algebraic. Let \( A  \) be the set of all real numbers that are algebraic. By the last problem, this set is countable. However, \( \R  \) is an uncountable set by Problem 3 which is a contradiction. Thus, not all real numbers are algebraic.  
\end{proof}

\section*{Problem 8} Choose (specific) distinct real numbers \( a,b,c,d  \) such that \( a < b  \) and \( c < d  \). Prove that \( \card{[a,b]} = \card{[c,d]}  \). (Hint: Find a linear polynomial \( f  \) such that \( f(a) = c  \) and \( f(b) = d  \).)
\begin{proof}
    Our goal is to show that \( \card{[a,b]} = \card{[c,d]} \). It suffices to show that \( [a,b] \sim [c,d] \). Consider the map \( f: [a,b] \to [c,d] \) defined by
\[  f(x) = \frac{ d - c  }{  b - a  } (x-a) + c.    \]
We wan to show that \( f  \) is a bijective function.
Note that \( f(a) = c  \) and \( f(b) = d   \). First, we show that \( f  \) is injective. Let \( x,y \in [a,b]  \). Suppose \( f(x) = f(y) \). Then we have that  
\begin{align*}
    \frac{ d - c  }{  b - a  } ( x- a) + c &= \frac{ d - c  }{ b - a  }  (y - a) + c  \\
    \frac{ d - c  }{  b - a  }  (x -a) &= \frac{ d - c  }{  b - a  }  (y-a) \\
    x - a &= y - a \\
    x &= y.
\end{align*}
Thus, \( f  \) is injective. Consider \( x \in [c,d] \). Then 
\[  x = \frac{ (y-c) (b-a) }{ d - c  } + a. \]
Then we have that 
\[  y = \frac{ d - c  }{  b - a  }  (x - a ) + c. \]
But this means that \( f(x) = y \) for some \( x \in [c,d] \). Hence, \( f  \) is surjective. Since \( f: [a,b] \sim [c,d] \) is bijective, we must have that \( \card{[a,b]} = \card{[c,d]} \).
\end{proof}

\section*{Problem 9} Consider the function \( f: (0,1) \to \R  \) given by 
\[  f(x) = 
\begin{cases}
    \frac{ 1 }{ n - 1  }  &\text{if} \ x = \frac{ 1 }{ n }, \  \text{for some} \ n \in \N \\ 
    x &\text{otherwise}
\end{cases}. 
\]
\begin{enumerate}
    \item Prove that \( f  \) is a bijection from \( (0,1)  \) to \( (0,1] \).
        \begin{proof}
            We will show that \( f: (0,1) \to (0,1] \) where 
            \[  f(x) = 
            \begin{cases}
                \frac{ 1 }{ n - 1  } &\text{if} \ \text{for some} \ n \in \N \\ 
                x &\text{otherwise}
            \end{cases} \]
is a bijective map. First, we will show that \( f  \) is injective. Suppose \( x,y \in (0,1) \). Then we have \( x = 1/n  \) and \( y = 1/ m \) for some \( n,m \in \N  \), respectively, such that \( f(x) = \frac{ 1  }{ n - 1  }  \) and \( f(y) = \frac{ 1  }{ m - 1 }  \). Suppose \( f(x) = f(y) \). Then observe that  
\begin{align*} 
    \frac{ 1 }{ n - 1  } &= \frac{ 1  }{  m - 1  }  \\
    m - 1 &= n - 1 \\
    m &= n.
\end{align*}
Thus, we see that \( x = y  \) and we conclude that \( f  \) is injective. 

Now, we will show that \( f  \) is surjective. Note that for any \( n \in \N  \) such that 
\[  0 < \frac{ 1 }{ n }  < \frac{ 1 }{ n - 1  }  < 1. \]
Then we can find some \( x = \frac{ 1 }{ n } \in (0,1)  \) such that  
\[  f(x) = \frac{ 1 }{ n - 1  }.  \]
Thus, \( f  \) must be surjective.

        \end{proof}
    \item Find a bijection from \( (0,1]  \) to \( [0,1) \).
        \begin{solution}
          Define the map  
          \[  f(x) = 
          \begin{cases}
              \frac{ 1 }{ n - 1  }  &\text{if} \ x = \frac{ 1 }{ n }  \ \text{for some} \ n \in \N  \\
              0 &\text{if} \ x = 1 \\
              x &\text{otherwise}
          \end{cases}  \]
        \end{solution}
    \item Find a bijection from \(  [0,1)  \) to \( [0,1] \).
        \begin{solution}
          Define the function  
          \[  f(x) = 
          \begin{cases}
              \frac{ 1 }{ n - 1   } &\text{if} \  x = \frac{ 1 }{ n } \ \text{for some} \  n \geq 3 \\   
              0 &\text{if} \ x = 0 \\ 
              1 &\text{if} \ x = 1/2 \\
              x &\text{otherwise} 
          \end{cases} \]
        \end{solution}

\end{enumerate}


\section{Problem 10} Prove the following theorem.

\begin{theorem}[ ]
    Let \( A  \) and \( B  \) be two nonempty sets. If \( B \subseteq A  \) and there exists a one-to-one function \( f: A \to B  \), then \( A \sim B  \).
\end{theorem}
\begin{proof}
\begin{itemize}
    \item If \( {x}_{1} \in C  \) and \( {x}_{2} \in C  \), then \( h({x}_{1}) = h({x}_{2})   \) implies that  
        \[  h({x}_{1}) = h({x}_{2}) \Longrightarrow f({x}_{1}) = f({x}_{2}). \]
        Since \( f \) is injective, we must have \( {x}_{1} = {x}_{2} \).
    \item To show that \( h(A) \subseteq B  \), we want to show that for all \( a \in h(A)  \), \( a  \in B \). Let \( a \in h(A) \). By definition of \( h  \), we either have \( a \in C  \) or \( a \in A \setminus  C  \). If \( a \in C  \), then we have \( h(a) = f(a) \). Since \( f: A \to B   \), \( h(a) \in  B \). Thus, \( h(A) \subseteq B  \). If \( a \in A \setminus  C  \), then \( h(a) = a  \) by definition of \( h  \).         
    \item If \( y \notin C  \), then for all \( n \geq 2  \), \( y \notin {C}_{n} \) which further implies that \( y \in A \setminus  {C}_{n} \). Since \( {C}_{n} \subseteq  C  \), we see that \( y \in A \setminus  C \). By definition of \( h  \), we see that \( h(y) = y  \) and so, \( h  \) is surjective.
\end{itemize}
\end{proof}

\section{Problem 11} Prove the following theorem.
\begin{theorem}[Schroder-Bernstein]
    Let \( A  \) and \( B  \) be two nonempty sets. If there exists two one-to-one functions \( f: A \to B  \) and \( g: B \to A  \), then \( B \sim A  \).
\end{theorem}
\begin{proof}
Since \( g(B) \subseteq  A  \) and the fact that the function \( g \circ f : A \to g(B)  \) is injective, we can use Theorem 1 to conclude that \( A \sim g(B) \). Since \( B \sim g(B) \) and \( g(B) \sim A  \), we conclude that \( B \sim A  \).
\end{proof}

\section*{Problem 12} Prove the following theorem.

\begin{theorem}[ ]
    Every infinite set contains a proper countable subset.
\end{theorem}
\begin{proof}
    \textbf{Continuing the proof presented on homework sheet.} In this way, we have constructed a one-to-one function \( f: \N \to S  \). Note that \( f: \N \to f(\N) \) will be bijective. Since \( \N \sim f(\N) \) and \( f(\N)  \) is a proper subset of \( S  \), we know that that \( f(\N) \) is countable.
\end{proof}

\section*{Problem 13} Prove the following theorem.

\begin{theorem}[ ]
    A set \( X  \) is infinite if and only if there exists a proper subset \( Y \subseteq  X  \) such that \( X \sim Y  \).
\end{theorem}

\begin{proof}
Assume that \( X  \) is finite. Thus, there exists \( n \in \N  \) such that \( X \sim {\N}_{n} \). Let \( h: X \to {\N}_{n} \) be a bijective map. Assume for contradiction that there exists a proper subset \( Y \subset X  \) and a bijective map \( f: X \to Y  \). Thus, we have \( Y \sim X  \). Since \( Y \sim X  \) and \( X \sim {\N}_{n} \), we have \( Y \sim {\N}_{n} \). But by our lemma, we know that this cannot happen since there exists no one-to-one mapping of \( {\N}_{n} \) onto \( Y \subset {\N}_{n} \). Hence, we have a contradiction and we must conclude that \( X  \) must be infinite.
\end{proof}

\section*{Problem 14} Prove the following theorem.
\begin{theorem}[ ]
    A set \( X  \) is infinite if and only if there exists a one-to-one function \( f: X \to X  \) that is not onto.
\end{theorem}
\begin{proof}
By Theorem 4, it is enough to prove that there exists a proper subset \( Y \subseteq X   \) such that \(X \sim Y  \). Note that \( f(X) \subseteq  X   \). Note that, by assumption, there exists a one-to-one function \( f: X \to X  \) that is NOT onto. Thus, we can restrict our codomain \( X  \) by its proper subset \( f(X) \) and produce the bijective function    \( f: X \to f(X) \). Thus, we see that \( X \sim f(X) \) and \( f(X) \subseteq X  \). By Theorem 4, we conclude that \( X  \) is infinite.
\end{proof}

\section*{Problem 15} Prove the following theorem.
\begin{theorem}[Cantor's Theorem]
    For any set \( A  \), the power set \( P(A) \) has strictly greater cardinality than \( A  \). (Hint: If \( f: A \to P(A) \) were bijective, consider \( B = \{ x \in A : x \notin f(x) \}  \)).
\end{theorem}
\begin{proof}
    Let \( A  \) be any set. Consider the map \( f: A \to P(A) \). We will show that \( \card A < \card P(A) \); that is, there exists no bijection \( f: A \to P(A) \). 

    First, we note that \( f : A \to P(A) \) is defined by  
    \[  g(x) = \{ x \}  \]
    which is a one-to-one function.
    Suppose for sake of contradiction that \( f: A \to P(A) \) is a bijection.  
Thus, we see that \( f  \) is surjective, by definition. Now, consider the set
\[  B = \{ x \in A : x \notin f(x)   \}  \]
note that \( B \subseteq  A  \). Since \( f  \) is surjective, there exists \( y \in A  \) such that \( f(y) = B  \). There are two cases to consider: either \( y \in B  \) or \( y \notin B  \). If \( y \in B  \), then \( y \notin f(y) \). But note that \( f(y) = B  \), so \( y \notin B  \) which is a contradiction. On the other hand, if \( y \notin B   \), then \( y \in f(y) \). Again, \( f(y) = B  \) implies \( y \) is also in \( B  \) which is a contradiction. Thus, there exists no bijective map \( f: A \to P(A) \) and so we conclude that  
\[  \card A < \card P(A). \]
\end{proof}

\section*{Problem 16} Prove that \( A  \) is infinite if and only if there exists a one-to-one function \( f: \N \to A  \). 
\begin{proof}
\( (\Rightarrow) \) Suppose \( A  \) is infinite. Our goal is to show that there exists a one-to-one function \( f: \N \to A  \). Since \( A  \) is infinite, \( A  \) is countable. Thus, there exists a bijective map \( f: \N \to A  \). By definition, this map is also injective.

\( (\Leftarrow) \) Suppose there exists a one-to-one function \( f: \N \to A  \). Our goal is to show that \( A  \) is infinite. Notice that \( f: \N \to f(\N) \) is a bijective map. Hence, \( \N \sim f(\N)  \) implies that \( f(\N) \) infinite. Since \( f(\N) \subseteq  A  \), we also have that \( A  \) is infinite.
\end{proof}

\section*{Problem 17} Prove that \( A  \) is at most countable if and only if there exists a one-to-one function \( f: A \to \N  \).
\begin{proof}
\( (\Rightarrow) \) Suppose \( A  \) is at most countable. By definition, \( A  \) is either countable or finite. If \( A  \) is finite, then there exists a bijective map such that \( f: A \to \N_n   \) for some \( n \in \N \). By definition of bijection, we know that \( f  \) is also injective. Suppose \( A  \) is countable, then we have \( A \sim \N  \) and so there exists a bijective map \( f: A \to \N  \) which is also injective by definition. 


\( (\Leftarrow) \) Suppose there exists a one-to-one function \( f: A \to \N \). Note that \( f(A)    \) is a proper subset of \( \N \) and that \( \N  \) is countable. Since \( f(A)  \) is a proper subset of \( \N  \), \( f(A)  \) must be at most countable. Furthermore, \( f: A \to f(A) \) is a bijective map and so, \( A \sim f(A) \). Since \( A \sim f(A) \) and \( f(A) \) is at most countable, we must have that \( A  \) is at most countable.
\end{proof}


\end{document}
