\documentclass[a4paper]{article}
\usepackage{standalone}
\usepackage{import}
\usepackage[utf8]{inputenc}
\usepackage[T1]{fontenc}
\usepackage{textcomp}
\usepackage{hyperref}
% \usepackage{fourier}
% \usepackage[dutch]{babel}
\usepackage{url}
% \usepackage{hyperref}
% \hypersetup{
%     colorlinks,
%     linkcolor={black},
%     citecolor={black},
%     urlcolor={blue!80!black}
% }
\usepackage{graphicx}
\usepackage{float}
\usepackage{booktabs}
\usepackage{enumitem}
% \usepackage{parskip}
\usepackage{emptypage}
\usepackage{subcaption}
\usepackage{multicol}
\usepackage[usenames,dvipsnames]{xcolor}

% \usepackage{cmbright}


\usepackage[margin=1in]{geometry}
\usepackage{amsmath, amsfonts, mathtools, amsthm, amssymb}
\usepackage{mathrsfs}
\usepackage{cancel}
\usepackage{bm}
\newcommand\N{\ensuremath{\mathbb{N}}}
\newcommand\R{\ensuremath{\mathbb{R}}}
\newcommand\Z{\ensuremath{\mathbb{Z}}}
\renewcommand\O{\ensuremath{\emptyset}}
\newcommand\Q{\ensuremath{\mathbb{Q}}}
\newcommand\C{\ensuremath{\mathbb{C}}}
\DeclareMathOperator{\sgn}{sgn}
\usepackage{systeme}
\let\svlim\lim\def\lim{\svlim\limits}
\let\implies\Rightarrow
\let\impliedby\Leftarrow
\let\iff\Leftrightarrow
\let\epsilon\varepsilon
\usepackage{stmaryrd} % for \lightning
\newcommand\contra{\scalebox{1.1}{$\lightning$}}
% \let\phi\varphi
\renewcommand\qedsymbol{$\blacksquare$}




% correct
\definecolor{correct}{HTML}{009900}
\newcommand\correct[2]{\ensuremath{\:}{\color{red}{#1}}\ensuremath{\to }{\color{correct}{#2}}\ensuremath{\:}}
\newcommand\green[1]{{\color{correct}{#1}}}



% horizontal rule
\newcommand\hr{
    \noindent\rule[0.5ex]{\linewidth}{0.5pt}
}


% hide parts
\newcommand\hide[1]{}



% si unitx
\usepackage{siunitx}
\sisetup{locale = FR}
% \renewcommand\vec[1]{\mathbf{#1}}
\newcommand\mat[1]{\mathbf{#1}}


% tikz
\usepackage{tikz}
\usepackage{tikz-cd}
\usetikzlibrary{intersections, angles, quotes, calc, positioning}
\usetikzlibrary{arrows.meta}
\usepackage{pgfplots}
\pgfplotsset{compat=1.13}


\tikzset{
    force/.style={thick, {Circle[length=2pt]}-stealth, shorten <=-1pt}
}

% theorems
\makeatother
\usepackage{thmtools}
\usepackage[framemethod=TikZ]{mdframed}
\mdfsetup{skipabove=1em,skipbelow=0em}


\theoremstyle{definition}

\declaretheoremstyle[
    headfont=\bfseries\sffamily\color{ForestGreen!70!black}, bodyfont=\normalfont,
    mdframed={
        linewidth=2pt,
        rightline=false, topline=false, bottomline=false,
        linecolor=ForestGreen, backgroundcolor=ForestGreen!5,
    }
]{thmgreenbox}

\declaretheoremstyle[
    headfont=\bfseries\sffamily\color{NavyBlue!70!black}, bodyfont=\normalfont,
    mdframed={
        linewidth=2pt,
        rightline=false, topline=false, bottomline=false,
        linecolor=NavyBlue, backgroundcolor=NavyBlue!5,
    }
]{thmbluebox}

\declaretheoremstyle[
    headfont=\bfseries\sffamily\color{NavyBlue!70!black}, bodyfont=\normalfont,
    mdframed={
        linewidth=2pt,
        rightline=false, topline=false, bottomline=false,
        linecolor=NavyBlue
    }
]{thmblueline}

\declaretheoremstyle[
    headfont=\bfseries\sffamily\color{RawSienna!70!black}, bodyfont=\normalfont,
    mdframed={
        linewidth=2pt,
        rightline=false, topline=false, bottomline=false,
        linecolor=RawSienna, backgroundcolor=RawSienna!5,
    }
]{thmredbox}

\declaretheoremstyle[
    headfont=\bfseries\sffamily\color{RawSienna!70!black}, bodyfont=\normalfont,
    numbered=no,
    mdframed={
        linewidth=2pt,
        rightline=false, topline=false, bottomline=false,
        linecolor=RawSienna, backgroundcolor=RawSienna!1,
    },
    qed=\qedsymbol
]{thmproofbox}

\declaretheoremstyle[
    headfont=\bfseries\sffamily\color{NavyBlue!70!black}, bodyfont=\normalfont,
    numbered=no,
    mdframed={
        linewidth=2pt,
        rightline=false, topline=false, bottomline=false,
        linecolor=NavyBlue, backgroundcolor=NavyBlue!1,
    },
]{thmexplanationbox}

\declaretheorem[style=thmgreenbox, numberwithin = section, name=Definition]{definition}
\declaretheorem[style=thmbluebox, name=Example]{eg}
\declaretheorem[style=thmredbox, numberwithin = section, name=Proposition]{prop}
\declaretheorem[style=thmredbox, numberwithin = section, name=Theorem]{theorem}
\declaretheorem[style=thmredbox, numberwithin = section,  name=Lemma]{lemma}
\declaretheorem[style=thmredbox, numberwithin = section,  numbered=no, name=Corollary]{corollary}


\declaretheorem[style=thmproofbox, name=Proof]{replacementproof}
\renewenvironment{proof}[1][\proofname]{\vspace{-10pt}\begin{replacementproof}}{\end{replacementproof}}


\declaretheorem[style=thmexplanationbox, name=Proof]{tmpexplanation}
\newenvironment{explanation}[1][]{\vspace{-10pt}\begin{tmpexplanation}}{\end{tmpexplanation}}


\declaretheorem[style=thmblueline, numbered=no, name=Remark]{remark}
\declaretheorem[style=thmblueline, numbered=no, name=Note]{note}

\newtheorem*{uovt}{UOVT}
\newtheorem*{notation}{Notation}
\newtheorem*{previouslyseen}{As previously seen}
\newtheorem*{problem}{Problem}
\newtheorem*{observe}{Observe}
\newtheorem*{property}{Property}
\newtheorem*{intuition}{Intuition}


\usepackage{etoolbox}
\AtEndEnvironment{vb}{\null\hfill$\diamond$}%
\AtEndEnvironment{intermezzo}{\null\hfill$\diamond$}%
% \AtEndEnvironment{opmerking}{\null\hfill$\diamond$}%

% http://tex.stackexchange.com/questions/22119/how-can-i-change-the-spacing-before-theorems-with-amsthm
\makeatletter
% \def\thm@space@setup{%
%   \thm@preskip=\parskip \thm@postskip=0pt
% }
\newcommand{\oefening}[1]{%
    \def\@oefening{#1}%
    \subsection*{Oefening #1}
}

\newcommand{\suboefening}[1]{%
    \subsubsection*{Oefening \@oefening.#1}
}

\newcommand{\exercise}[1]{%
    \def\@exercise{#1}%
    \subsection*{Exercise #1}
}

\newcommand{\subexercise}[1]{%
    \subsubsection*{Exercise \@exercise.#1}
}


\usepackage{xifthen}

\def\testdateparts#1{\dateparts#1\relax}
\def\dateparts#1 #2 #3 #4 #5\relax{
    \marginpar{\small\textsf{\mbox{#1 #2 #3 #5}}}
}

\def\@lesson{}%
\newcommand{\lesson}[3]{
    \ifthenelse{\isempty{#3}}{%
        \def\@lesson{Lecture #1}%
    }{%
        \def\@lesson{Lecture #1: #3}%
    }%
    \subsection*{\@lesson}
    \testdateparts{#2}
}

% \renewcommand\date[1]{\marginpar{#1}}


% fancy headers
\usepackage{fancyhdr}
\pagestyle{fancy}

\fancyhead[LE,RO]{Lance Remigio}
\fancyhead[RO,LE]{\@lesson}
\fancyhead[RE,LO]{}
\fancyfoot[LE,RO]{\thepage}
\fancyfoot[C]{\leftmark}

\makeatother




% notes
\usepackage{todonotes}
\usepackage{tcolorbox}

\tcbuselibrary{breakable}
\newenvironment{verbetering}{\begin{tcolorbox}[
    arc=0mm,
    colback=white,
    colframe=green!60!black,
    title=Opmerking,
    fonttitle=\sffamily,
    breakable
]}{\end{tcolorbox}}

\newenvironment{noot}[1]{\begin{tcolorbox}[
    arc=0mm,
    colback=white,
    colframe=white!60!black,
    title=#1,
    fonttitle=\sffamily,
    breakable
]}{\end{tcolorbox}}




% figure support
\usepackage{import}
\usepackage{xifthen}
\pdfminorversion=7
\usepackage{pdfpages}
\usepackage{transparent}
\newcommand{\incfig}[1]{%
    \def\svgwidth{\columnwidth}
    \import{./figures/}{#1.pdf_tex}
}

% %http://tex.stackexchange.com/questions/76273/multiple-pdfs-with-page-group-included-in-a-single-page-warning
\pdfsuppresswarningpagegroup=1




\pagestyle{fancy}
\fancyhf{}

\title{Math 230A: Homework 4}
\author{Lance Remigio}

\begin{document}
\maketitle    
\lhead{Math 230A: Homework 4}
\chead{Lance Remigio}
\rhead{\thepage}

\begin{enumerate}
    \item Consider \( \R  \) with its standard metric. Let \( E = \Big\{ \frac{ 1 }{ n }  : n \in \N \Big\}.    \) Prove that \( E' = \{ 0 \}  \).
        \begin{proof}
        Note that \( 0 \notin E \). To show that \( E' = \{ 0  \}  \), it suffices to show that for \( \epsilon > 0  \)
        \[  {N}_{\epsilon}(0) \cap E \neq \emptyset \]
        which is equivalent to 
        \[  (- \epsilon, \epsilon) \cap E \neq \emptyset. \]
        Let \( \epsilon > 0 \). By the Archimedean Property of \( \R  \), there exists \( m \in \N  \) such that \( \frac{ 1 }{ m }  < \epsilon \). Thus, \( 1/m \in (-\epsilon, \epsilon) \cap E  \) and so \( 0 \in E' \). 
        \end{proof}
    \item Consider \( \R  \) with its standard metric.
        \begin{enumerate}
            \item[(a)] Prove that \( \N' = \emptyset \). Is \( \N  \) closed?
                \begin{solution}
                Suppose for sake of contradiction that \( \N' \neq \emptyset \). Let \( n \in \N' \). Let \( \epsilon = 1  \). Then \( {N}_{1}(n) \cap \N \setminus  \{ n \}  \neq \emptyset \) where \( {N}_{1}(n) = (n-1, n+1) \). But note that \( (n-1, n+1) \) only contains one point, namely, \( n \) and nothing else. Hence, \( n  \) must be an isolated point be an isolated point of \( \N  \) which is a contradiction. Thus, \( \N = \emptyset \). Because \( \emptyset \subseteq  \N  \), \( \N  \) must be a closed set.
                \end{solution}
            \item[(b)] Prove that \( \Q' = \R  \). Is \( \Q  \) closed?
                \begin{solution}
                Let \( a \in \R  \) and let \( \epsilon > 0 \). Consider the open interval \( (a - \epsilon, a + \epsilon) \). Since \( \Q  \) is dense in \( \R  \), we can find an \( x \in \Q  \) such that \( x \in (a- \epsilon, a + \epsilon) \) which is equivalent to saying that \( | x - a  | < \epsilon \). So, any \( a \in \R  \) must be a limit point of \( \Q  \). Therefore, we conclude that \( \Q'= \R  \). Consequently, \( \Q  \) must not be closed since its limit points take are outside of \( \Q \). 
                \end{solution}
        \end{enumerate}
    \item Consider \( \R^{2}  \) with its standard metric. Let \( E = \{ (x,y) \in \R^{2} : x^{2} + y^{2} < 4  \}  \). Prove that \( E' = \{ (x,y) \in \R^{2} : x^{2} + y^{2} \leq 4  \}  \).
        \begin{proof}
        We want to show that 
        \[  E' = \{ (x,y) \in \R^{2} : x^{2} + y^{2} \leq 4 \}. \]
         Set \( p = (a,b) \). Choose \( \delta = \frac{ 1 }{ 2 }  (\sqrt{ a^{2} + b^{2} } - 2 ) > 0 \). Then we must have \( {N}_{\delta}(p) \cap E \setminus  \{ p \} = \emptyset \), showing that \( p \notin E' \) and that \( a^{2} + b^{2} > 4  \). For \( p \in E' \), then we must have \( {N}_{\delta}(p) \cap E \setminus  \{ p  \} \neq \emptyset \); that is, \( a^{2} + b^{2} \leq 4 \). Thus, 
         \[  E' = \{ (x,y) \in \R^{2} : x^{2} + y^{2} \leq 4 \}. \]
        \end{proof}
    \item Let \( (X,d)  \) be a metric space. Prove that \( X  \) and \( \emptyset  \) are both open sets.
        \begin{proof}
        We want to show the following: 
        \begin{enumerate}
            \item[(i)] \( X  \) is open 
            \item[(ii)] \( \emptyset \) is open
        \end{enumerate}
        Let \( x \in X \). For (i), we want to show that there exists \( \delta > 0 \) such that \( {N}_{\delta}(x) \subseteq X  \). Pick another point \( q \neq x  \) such that \( q \in X  \). Choose \( \delta = \frac{ 1 }{ 2 } d(x,q) \). Since \( q,x \in X  \), it follows that \( {N}_{\delta}(x) \subseteq X  \). So, \( X  \) is open. 

        Since there are no points to consider in \( \emptyset \), it must immediately follow that \( \emptyset  \) is an open set.
        \end{proof}
    \item Let \( (X,d)  \) be a metric space. Prove that \( X  \) and \( \emptyset  \) are both closed sets.
        \begin{proof}
        We will proving the following statements: 
        \begin{enumerate}
            \item[(i)] \( X  \) is closed.
            \item[(ii)] \( \emptyset \) is closed.
        \end{enumerate}
        Let \( x  \) be a limit point of \( X  \). Then for all \( \epsilon > 0 \), we have \( {N}_{\epsilon}(x) \cap X \setminus  \{ x \} \neq \emptyset  \). But note that \( {N}_{\epsilon}(x) \) is an open set that is contained in \( X  \). Thus, \( x  \) must be contained within \( X  \) which proves (i).

        For (ii), note that the complement of \( \emptyset \) is just \( X  \) itself. Since \( X  \) is open by part (a), we see that \( X^{c} = \emptyset \) must be closed.
        \end{proof}
    \item Consider \( \R  \) equipped with the discrete metric. Let \( E = \{ 1,2,3 \} \). Prove that \( E  \) is open, that is, \( E^{\circ} = E  \).
        \begin{proof}
        We will check that for every element \( x  \) in \( E  \) that there exists \( \delta > 0  \) such that \( {N}_{\delta}(x) \subseteq E  \). Choose \( \delta = 1  \), then we have \( d(1,x) = 0 < 1   \). Since this is the only point that satisfies \( \delta \), we must have that \( {N}_{1}(1) = \{ 1 \} \subseteq E   \). Similarly, \( {N}_{1}(2) = \{ 2 \}  \subseteq  E  \) and \( {N}_{1}(3) = \{ 3 \}  \subseteq E  \). Thus, \( E  \) must be open.
        \end{proof}
    \item Consider \( X = \R  \) equipped with the standard metric. Let \( a < b  \). 
        \begin{enumerate}
            \item[(a)] Prove that the sets \( (a,b), (a,\infty ),  \) and \( (-\infty , a) \) are open.
                \begin{proof}
                We will show that the following intervals in \( \R  \) are open:                
                \begin{enumerate}
                    \item[(i)] \( (a,b) \)
                    \item[(ii)] \( (a,\infty ) \)
                    \item[(iii)] \( (-\infty , a) \)
                \end{enumerate}
                Starting with (i), we want to show that there exists a \( \delta > 0  \) such that \( {N}_{\delta}(x) \subseteq (a,b) \) where \( {N}_{\delta}(x) = (x - \delta, x + \delta) \). Let \( x \in (a,b) \). Choose \( \delta = \frac{ 1 }{ 2 } \min \{ b - x, x - a  \}  \). Then we have \( {N}_{\delta}(x) \subseteq (a,b)  \). Thus, we conclude that \( (a,b) \) is open.

                With (ii), observe that the neighborhood \( {N}_{\delta}(x) \) constructed in the proof of (i) is contained within \( (a,\infty ) \). Thus, \( (a,\infty ) \) is open as well. 

                Lastly and similarly, observe that the neighborhood \( {N}_{\delta}(x) \) constructed in (i) that is contained in \( (a,b) \) is also contained within \( (-\infty, b  ) \). Thus, \( {N}_{\delta}(x) \subseteq (- \infty, b) \) for some \( \delta > 0  \) implying that \( (- \infty , b) \) is open.
                \end{proof}
            \item[(b)] Prove that the sets \( [a,b], [a,\infty), \) and \( (-\infty, b ] \) are closed.
                \begin{proof}
                We will showing that the following intervals are closed in \( \R  \):
                \begin{enumerate}
                    \item[(i)] \( [a,b] \)
                    \item[(ii)] \( [a,\infty ) \)
                    \item[(iii)] \( (-\infty,b] \)
                \end{enumerate}
                Starting with (i), we want to show that every limit point \( x \) of \( [a,b] \) is contained in \( [a,b] \). Thus, let \( x  \) be a limit point of \( x  \). By definition, for all \( \delta > 0  \), we have \( {N}_{\delta}(x) \cap [a,b] \setminus  \{ x \} \neq \emptyset \). So, \( {N}_{\delta}(x) \) contains a point \( q \neq x  \) such that \( q \in [a,b]  \). This means that \( a \leq q \leq b  \). Using the fact that \( q \in {N}_{\delta}(x) \), we have  
                \begin{align*}
                    |  x-  q  |  < \delta &\Longleftrightarrow q - \delta < x < q + \delta  \\
                                          &\Longrightarrow a - \delta \leq x \leq b + \delta.
                \end{align*}
                Since \( \delta > 0  \) is arbitrary, we conclude that \( a \leq x \leq b  \) which means that \( x \in [a,b] \). Thus, \( [a,b] \) is closed.

                With (ii), we will employ the same process. Let \( x  \) be a limit point of \( [a,\infty)  \). Then for all \( \varphi > 0  \), we see that \( {N}_{\varphi}(x) \cap [a,\infty) \setminus  \{ x \} \neq \emptyset  \). To this end, pick \( p \neq x  \) such that \( p \in [a,\infty) \). Since \( p  \) is also contained within \( {N}_{\varphi}(x) \), we see that  
                \begin{align*}  | x - p  |  < \varphi &\Longleftrightarrow p - \varphi < x < p + \varphi \\
                    &\Longleftrightarrow a - \varphi \leq x < p + \varphi.
                \end{align*}
                This implies that \( a \leq x < p  \) since \( \varphi > 0 \) is arbitrary. Thus, we have \( x \in [a,\infty ) \) and so \( [a,\infty ) \) is closed.  

                The last case is similar to (ii). Let \( \epsilon > 0 \). If \( x  \) is a limit point of \( (-\infty, b] \), then \( {N}_{\epsilon}(x) \cap [- \infty, b) \setminus  \{ x  \} \neq \emptyset \) implies that we can pick \( p \neq x  \) such that \( p \in [- \infty ,  b) \). Since \( \epsilon > 0 \) is arbitrary, we have
                \[  q - \epsilon < x < q + \epsilon \leq b + \epsilon \]
                implies 
                \[  q < x \leq b.  \]
                Hence, \( x \in (-\infty, b] \) and so \( (- \infty , b]  \) is a closed set.
                \end{proof}
        \end{enumerate}
    \item Let \( (X,d) \) be a metric space. Let \( p \in X   \) and \( \delta > 0  \). Prove that the closed ball \( {C}_{\delta}(p) = \{ x \in X : d(x,p) \leq \delta  \}  \) is indeed closed.
        \begin{proof}
        Let \( p \in X  \) and \( \delta > 0  \). Our goal is to show that 
        \[  {C}_{\delta}(p) = \{ x \in X   : d(x,p) \leq \delta \} \]
        is a closed set; that is, we need to show that every limit point of \( {C}_{\delta}(p) \) is contained in \( {C}_{\delta}(p) \). Let \( x  \) be a limit point of \( {C}_{\delta}(p) \). By definition, choose \( \epsilon = \frac{ \delta }{ 2 }  \) such that  
        \[  {N}_{\frac{ \delta }{ 2 } }(x) \cap {C}_{\delta}(p) \setminus  \{ x  \}.  \]
        Choose this point to be \( y  \) such that \( y \neq x  \) and \( y \in {C}_{\delta}(p) \). Thus, we see that \( d(y,p) \leq \frac{ \delta }{ 2 }  \). Likewise, \( y  \) being contained in \( {N}_{\frac{ \delta }{ 2 } }(x)  \) implies that \( d(x,y) < \frac{ \delta }{ 2 }  \). In order for \( x  \) to be contained in \( {C}_{\delta}(p) \), we need to show that         
        \begin{center}
            for all \( \delta > 0  \) that \( d(x,p) \leq \delta  \).
        \end{center}
        Using the triangle inequality, we see that 
        \[  d(x,p) \leq d(x,y) + d(y,p) \leq \frac{ \delta }{ 2 }  + \frac{ \delta }{ 2 }  = \delta.  \]
        Thus, we see that \( x \in {C}_{\delta}(p) \) and so we can now conclude that, indeed, \( C_{\delta}(p) \) is a closed set.
    \end{proof}
    \item Give an example that shows that the equality \( \overline{{N}_{\delta}(p)} = {C}_{\delta}(p)  \) is not always true.
        \begin{solution}
            Consider the interval \( [0,1] \) in \( \R  \) with the discrete metric. Clearly, we see that \( 1/2 \in [0,1] \). If we let \( \epsilon = 1   \), then 
            \[  {N}_{1}(1/2) = \{ x \in \R : d(x,1/2) < 1 \} = \{ 1/2 \}  \]
            since the only case when the inequality is satisfied is when \( x =  1/2  \). If we consider the closure of this neighborhood, we just get 
            \[  \overline{{N}_{1}(1/2)} = \{ 1/2 \}. \]
            Now, consider the closed ball 
            \[  {C}_{1}(1/2) = \{ x \in \R : d(x,1/2) \leq 1  \}. \]
            Observe that for any \( x \in \R  \), either \( x = 1/2 \) or \( x \neq 1/2 \) in \( [0,1] \), the inequality of the set above we always be satisfied; that is, the set will just be all elements contained in \( [0,1]  \). Thus, we see that \( {C}_{1}(1/2) = [0,1] \) and, in this case, that \( {C}_{1}(1/2) \neq \overline{{N}_{1}(1/2)} \).  
        \end{solution}
    \item Let \( (X,d) \) be a metric space. Prove that an arbitrary intersection of closed sets is closed. Prove that a finite union of closed sets is closed. 
        \begin{proof}
        Let \( (X,d) \) be a metric space. We will prove that
        \begin{enumerate}
            \item[(1)] An arbitrary intersection of closed sets is closed.
            \item[(2)] A finite union of closed sets is closed.
        \end{enumerate}
        To this end, we will proceed by showing the propositions above. 
        \begin{enumerate}
            \item[(1)] Let \( \{ {F}_{\alpha} \}  \) be an arbitrary collection of closed sets. We want to show that 
                \[  \bigcap_{ \alpha }^{  }  {F}_{\alpha} \ \text{is closed.} \]
               It suffices to show that the complement of this set is open. Consider the complement
               \[  \Big(  \bigcap_{ \alpha }^{  }  {F}_{\alpha} \Big)^{c} = \bigcup_{ \alpha }^{  } {F}_{\alpha}^{c}. \tag{1} \]
               Observe that each \( {F}_{\alpha}^{c} \) is open since each \( {F}_{\alpha} \) is closed. But note that the arbitrary union of open sets is open and so the union of the right-hand side of (1) is open. Thus, the left-hand side of (1) is open. Hence, the complement  
               \[  \Big(  \bigcap_{ \alpha }^{  } {F}_{\alpha} \Big)^{c} \ \text{is open} \Longleftrightarrow \bigcap_{ \alpha }^{  }  {F}_{\alpha} \ \text{is closed}  \]
               and we are done.
            \item[(2)] Let \( \{ {F}_{i} : 1 \leq i \leq n  \}  \) be a finite collection of closed sets \( {F}_{i} \). We want to show that \( \bigcup_{ i=1 }^{ n } {F}_{i} \) is a closed set. It suffices to show that the complement of this set, that is \( \Big(  \bigcup_{ i=1 }^{ n } {F}_{i} \Big)^{c} \) is open. Observe that
                \[ \Big(  \bigcup_{ i=1 }^{ n } {F}_{i} \Big)^{c} = \bigcap_{ i=1 }^{ n } {F}_{i}^{c}.  \]
                Note that each \( {F}_{i}^{c} \) is open if and only if each \( {F}_{i} \) is closed. Thus, the finite intersection of each open set \( {F}_{i}^{c}  \) is open; that is,
                \[ \Big(  \bigcup_{ i=1 }^{ n } {F}_{i} \Big)^{c} = \bigcap_{ i=1 }^{ n } {F}_{i}^{c} \ \text{is open}.  \]
                Hence, the finite union of closed sets
                \[  \bigcup_{ i=1 }^{ n } {F}_{i} \ \text{is closed}. \]
        \end{enumerate}
        \end{proof}
    \item Let \( E  \) be a subset of a metric space \( (X,d ) \). Show that \( \overline{E} = \{ x \in X : \ \forall x > 0 \ {N}_{\epsilon}(x) \cap E \neq \emptyset \}  \).
    \begin{proof}
        Let \( E  \) be a subset of a metric space \( (X,d) \). We want to show that 
        \[ \overline{E} =\{  x \in X : \forall   \epsilon > 0 \ {N}_{\epsilon}(x) \cap E \neq \emptyset \}.  \] 
        Denote 
        \[  A = \{  x \in X : \forall  \epsilon > 0 \ {N}_{\epsilon}(x) \cap E \neq \emptyset \}.   \]
        It suffices to show the following two inclusions:
        \begin{enumerate}
            \item[(1)] \( \overline{E} \subseteq A \).
            \item[(2)] \( A \subseteq \overline{E} \).
        \end{enumerate}
        Starting with (1), suppose \( x \in \overline{E} \). Then either \( x \in E  \) or \( x \in E' \) (or both). If \( x \in E'  \), then \( x  \) is a limit point of \( E  \). Then for all \( \epsilon > 0  \),
        \[  {N}_{\epsilon}(x) \cap E \setminus  \{ x \}  \neq \emptyset. \]
        Since \( E \setminus  \{ x \} \subseteq E  \), we we see that 
        \[  {N}_{\epsilon}(x) \cap E \neq \emptyset \]
        which implies that \( x \in A \). Suppose \( x \in E  \). Then for all \( \epsilon > 0 \), we see that  
        \[  {N}_{\epsilon}(x) \cap E \neq \emptyset. \]
        If \( x  \) is a limit point and \( x \in E  \), then we immediately have that 
        \[  {N}_{\epsilon}(x) \cap E \neq \emptyset.  \]
        by definition of limit point. Thus, we see that in the three cases outlined that \( \overline{E} \subseteq A  \).
       
        With (2), suppose \( x \in A  \). We want to show that \( x \in \overline{E} \); that is, \( x  \) is either a limit point of \( E  \) or an element of \( E  \). Suppose that \( x  \) is NOT an element of \( E  \). By definition of \( A  \), we see that for all \( \epsilon > 0 \), 
        \[  {N}_{\epsilon}(x) \cap E \neq \emptyset.  \]
        If we take away \( x  \) from \( E  \), then the intersection above is nonempty; that is, 
        \[  {N}_{\epsilon}(x) \cap E \setminus  \{ x \} \neq \emptyset. \]
        Thus, \( x  \) is a limit point of \( E  \) and so \( x \in \overline{E} \). Otherwise, when \( x  \) is not a limit point of \( E  \), then the nonempty intersection
        \[  {N}_{\epsilon}(x) \cap E  \]
        implies that \( x \in E  \) and so \( x \in \overline{E} \). Thus, \( A \subseteq \overline{E} \).
    \end{proof}
\item Let \( (X,d) \) be a metric space. Show that the closure of a bounded set is bounded.
    \begin{proof}
    Let \( E \subseteq  X   \) be a bounded set and let \( \overline{E} \) be the closure of \( E  \); that is, \( \overline{E} = E \cup E' \). Our goal is to show that \( \overline{E}  \) is bounded; that is, there exists \( M \in \R  \) and \( q \in X  \) such that \( d(p,q) < M  \) for all \( p \in \overline{E} \). Let \( p \in \overline{E} \). Let us consider two cases:
    \begin{enumerate}
        \item[(1)] \( p \in E  \) 
        \item[(2)] \( p \in E' \); that is, \( p  \) is a limit point of \( E  \).
    \end{enumerate}
    Suppose \( p \in E  \). Since \( E  \) is bounded, there exists \( M \in \R  \) and \( q \in X  \) such that \( d(p,q) < M  \). Hence, \( \overline{E} \) is bounded. Now, suppose \( p \in E' \). Then we know that \( p  \) is a limit point of \( E  \); that is, choose \( \epsilon = M > 0  \) such that 
    \[ {N}_{M}(p) \cap E \setminus  \{ p \} \neq \emptyset.  \]
    So, pick \( q \neq p  \) such that \( q \in E  \). Since \( q \in {N}_{\epsilon}(p) \), we have that \( d(p,q) < \epsilon \). But since \( \overline{E} \) is a closed set, we know that \( p \in \overline{E} \), this must also mean that \( \overline{E} \) is bounded.  
    \end{proof}
\item Prove that a nonmepty subset \( A  \) of a metric space \( (X,d) \) is open if and only if it is a union of neighborhoods (open balls).
    \begin{proof}
    Let \( A  \) be a nonempty subset of a metric space \( (X,d) \) that is an open set. Thus, define the collection of open balls as
    \[  {V}_{\epsilon}(x) = \{ {N}_{\epsilon}(x) : x \in A   \}. \]
    Our goal is to show that 
    \[  A = \bigcup_{ x \in A  }^{  } {N}_{\epsilon}(x); \]
    that is, we need to show the following two inclusions:
    \begin{enumerate}
        \item[(1)] \( A \subseteq \bigcup_{ x \in A  }^{  }  {N}_{\epsilon}(x) \)
        \item[(2)] \( \bigcup_{ x \in A  }^{  }  {N}_{\epsilon}(x) \subseteq A. \)
    \end{enumerate}
    Now, we proceed with the proof of the result for the two cases above:
    \begin{enumerate}
        \item[(1)] Let \( x \in A  \). Since \( A  \) is open, there exists an \( \epsilon > 0  \) such that \( {N}_{\epsilon}(x) \subseteq  A  \). But note that \( {N}_{\epsilon}(x) \subseteq  \bigcup_{ x \in A  }^{  }  {N}_{\epsilon}(x) \). Since \( x \in {N}_{\epsilon}(x) \), we must have \( x \in \bigcup_{ x \in A  }^{  } {N}_{\epsilon}(x)  \). Hence, 
            \[  A \subseteq \bigcup_{ x \in A  }^{  } {N}_{\epsilon}(x). \]
        \item[(2)] Let \( p \in \bigcup_{ x \in A  }^{  } {N}_{\epsilon}(x) \). Then for some \( \epsilon > 0 \), then \( y \in {N}_{\epsilon}  \). But since \( {N}_{\epsilon} \) is an open set of \( A  \), we must have \( y \in A  \). Thus,
            \[  \bigcup_{ x \in A  }^{   }  {N}_{\epsilon}(x) \subseteq  A. \]
    \end{enumerate}
    Thus, with (1) and (2) we have that \( A  \) is the union of open balls in \( A  \).

    \( (\Longleftarrow) \) Suppose 
    \[  A = \bigcup_{ x \in A  }^{  } {N}_{\epsilon}(x). \tag{*} \]
    We want to show that \( A  \) is open; that is, we need to find a \( \delta > 0  \) such that \( {N}_{\delta}(x) \subseteq A  \). Let \( x \in A  \). Then by (*), we know that \( x \in \bigcup_{ x \in A  }^{  }  {N}_{\epsilon}(x) \). By definition of the arbitrary union, we see that we can find \(  \delta  > 0 \) such that \( {N}_{\delta}(x) \subseteq \bigcup_{ x \in A  }^{   }  {N}_{\epsilon}(x) = A  \). Thus, \( A  \) must be an open set. 
    \end{proof}
\item On page 9, Rudin implicitly defines a subset \( E \subseteq \R   \) to be "dense" if it satisfies:
    \begin{enumerate}
        \item[(i)] For all \( x,y \in \R  \) with \( x < y  \), there exists \( p \in E  \) such that \( x < p < y  \).
    \end{enumerate}
    On page 32, he defines a subset \( E  \) of a metric space \( X  \) to be "dense" if it satisfies:
    \begin{enumerate}
        \item[(ii)] Every point of \( X  \) is either a limit point of \( E  \) or a point of \( E  \) (that is, \( \overline{E} = X  \)).
    \end{enumerate}
    Prove that \( E \subseteq \R  \) satisfies (i) if and only if it satisfies (ii).
    \begin{proof}
    \( (\Longrightarrow) \) Suppose \( E \subseteq \R  \) satisfies the property (i). We want to show that every point in \( \R  \) must either be a limit point of \( E  \) or a point in \( E  \). To this end, let \( x \in \R  \) be an arbitrary point. Suppose that \( x  \) is not an element of \( E  \). Thus, we want to show that \( x  \) is a limit point of \( E  \); that is, we will show that 
    \[  \forall \delta > 0 \ \ {N}_{\delta}(x) \cap E \setminus  \{ x  \}  \neq \emptyset. \]
    Let \( \delta > 0 \). Then denote the open neighborhood of \( x  \) as 
    \[  {N}_{\delta}(x) = \{ y \in \R : | x - y | < \delta \} = (x - \delta, x  + \delta).  \]
    But note that \( x - \delta < x + \delta  \) implies that there exists a \( p \neq x  \in E  \) such that \( x - \delta < p < x + \delta \). Thus, we see that \( |  x - p  |  < \delta  \) and hence, we see that 
    \[  {N}_{\delta}(x) \cap E \setminus  \{ x \} \neq \emptyset.  \]
    Hence, \( x  \) is a limit point of \( E  \). Now, if \( x  \) is NOT a limit point, then \( x  \) is an isolated point of \( E  \) and that \(  x \in E  \).
    
    \( (\Longleftarrow ) \) Let \( x,y \in \R  \) such that \( x < y  \). We will show that there exists \( p \in E  \) such that \( x < p < y \). Suppose \( E \subseteq \R   \) satisfies property (ii); that is, every point of \( X  \) is either limit point of \( E  \) or a point of \( E  \). We will consider a few cases:     
    \begin{enumerate}
        \item[(1)] \( x  \) is a limit point and \( y \in E  \)
        \item[(2)] \( x \in E  \) and \( y \in E  \)
        \item[(3)] \( x \in E  \) but \( y  \) is a limit point.
        \item[(4)] \( x  \) and \( y  \) are limit points of \( E  \) 
    \end{enumerate}
    Suppose case (1). If \( x  \) is a limit point, then for all \( \delta > 0  \), we have 
    \[  (x - \delta, x + \delta) \cap E \setminus  \{ x  \}  \neq \emptyset. \]
    Hence, pick \( p \neq x  \) such that \( p \in E  \). Thus, \( x - \delta < p < x + \delta \). Since \( y > x  \), we see that 
    \[  x - \delta < p < y  + \delta. \]
    Since \( \delta> 0  \) is arbitrary, we can see that \( x < p < y  \).

    Now, consider case (2). Then immediately if \( x < y  \), then choose \( p = \frac{  y -x  }{ 2  }  \). Clearly, we have \( x < p < y \).

    Next, consider the case (3) when \( x \in E   \), but \( y  \) is a limit point of \( E  \). Since \( y  \) is a limit point of \( E  \), we have for all \( \delta > 0  \), 
    \[  {N}_{\delta}(y) \cap E \setminus  \{ y \} \neq \emptyset. \]
    That is, 
    \[  (y - \delta, y + \delta)  \cap E \setminus  \{ y \} \neq \emptyset. \]
    To this end, pick \( p \neq y  \) in \( E  \) such that 
    \[  y - \delta < p < y + \delta. \]
    If \( x < y  \), then we see that
    \[  x  - \delta < p < y + \delta. \]
    Since \( \delta > 0  \) is arbitrary, we have that \( x < p < y \).

    Finally, suppose \( x  \) and \( y  \) are limit points of \( E  \). By definition, for all \( \epsilon > 0  \) and \( \delta > 0  \), we have
    \begin{align*}
     (x - \epsilon, x + \epsilon) \cap E \setminus  \{ x  \} &\neq \emptyset \\
     (y - \delta, y  + \delta) \cap E \setminus  \{ y \} &\neq \emptyset. 
    \end{align*}
    Without loss of generality, let \( \epsilon = \delta > 0 \) and pick \( p \neq x  \in E  \) such that \( p \in (x- \epsilon , x + \epsilon) \). Since \( x < y  \), we have that  
    \[  x - \epsilon < p < x + \epsilon < y + \epsilon. \]
    Thus, we see that 
    \[  x < p < y. \]
    \end{proof}
    \item Let \( A  \) and \( B  \) be subsets of a metric space \( (X,d) \).
        \begin{enumerate}
            \item[(a)] If \( A \subseteq B  \), then \( A' \subseteq B' \).
                \begin{proof}
                Let \( x \in A' \). We want to show that \( x \in B' \); that is, \( x  \)is a limit point of \( B  \). By definition, \( x \in A'  \) implies that for all \( \epsilon > 0  \), \( {N}_{\epsilon}(x) \cap A \setminus  \{ x \} \neq \emptyset \). So, pick \( p \neq x  \) such that \( p \in A  \). But note that \( A \subseteq B  \). Thus, \( p \in  B \) and we must have for all \( \epsilon > 0 \), 
                \[  {N}_{\epsilon}(x) \cap B \setminus  \{ x \} \neq \emptyset. \]
                Thus, \( x  \) is also a limit point of \( B  \) (that is, \( x \in B' \)) and we conclude that \( A' \subseteq  B' \).
                \end{proof}
            \item[(b)] If \( A \subseteq  B  \), then \( \overline{A} \subseteq \overline{B} \).
                \begin{proof}
                Suppose \( A \subseteq  B  \). By part (i), we see that \( A' \subseteq  B ' \). Thus, 
                \[  \overline{A} = A \cup A' \subseteq  B \cup B' = \overline{B}. \]
                Hence, \( \overline{A} \subseteq \overline{B} \).
                \end{proof}
            \item[(c)] Prove that \( \overline{A \cap B} \subseteq \overline{A} \cap \overline{B} \).
                \begin{proof}
                Our goal is to show that \( \overline{A \cap B} \subseteq \overline{A} \cap \overline{B} \). Let \( x \in \overline{A \cap B} \). Then either \( x \in A \cap B  \) or \( x \in (A \cap B)' \). Suppose \( x \in A \cap  B  \). Then both \( x \in A  \) and \( x \in  B \) and so \( x \in \overline{A} \) and \( x \in \overline{B} \) since \( A \subseteq  \overline{A} \) and \( B \subseteq \overline{B} \). Now, suppose \( x \in (A \cap B)' \). Then \( x  \) is a limit point of \( A \cap B  \); that is, for all \( \epsilon > 0  \), \( {N}_{\epsilon}(x) \cap (A \cap B) \setminus  \{ x \} \neq \emptyset \). Since \( A \cap B \subseteq  A  \) and \( A \cap B \subseteq  B  \), we must have that
                \[  {N}_{\epsilon}(x) \cap A \setminus  \{ x \} \neq \emptyset  \tag{1}\]
                and 
                \[  {N}_{\epsilon}(x) \cap B \setminus  \{ x \} \neq \emptyset. \tag{2}  \]
                Thus, (1) and (2) imply that \( x \in A'   \) and \( x \in B'  \) meaning that \( x  \) is both a limit of point of \( A  \) and \(  B \). Hence, \( x \in \overline{A}  \) and \( x \in \overline{B} \). Thus, we have \( \overline{A \cap B} \subseteq \overline{A} \cap \overline{B} \) in both of these cases.
                \end{proof}
        \end{enumerate}
    \item Let \( (X,d) \) be a metric space and let \( E \subseteq X  \).
        \begin{enumerate}
            \item[(a)] Prove that \( E' \) is closed.
                \begin{proof}
                Our goal is to show that \( E' \) is closed; that is, any limit point of \( E'  \) is contained within \( E' \). Let \( x  \) be a limit point of \( E'  \). Then for all \( \epsilon > 0 \), \( {N}_{\epsilon}(x) \cap E' \setminus  \{ x \}  \neq \emptyset  \). To this end, pick \( y \neq x  \) such that \( y \in E' \). But this tells us that \( y  \) is a limit point of \( E  \). Thus, for all \( \delta > 0  \), \( {N}_{\delta}(y) \cap E \setminus  \{ y \} \neq \emptyset \). Thus, we can pick \( p \neq y  \) such that \( p \in E  \). Note that \( p \neq y \neq x  \) and \( p \in E  \) imply that  
                \[  {N}_{\epsilon}(x) \cap E \setminus  \{ x \} \neq \emptyset. \]
                Thus, \( x  \) is a limit point of \( E  \) and so \( x \in E' \). Hence, \( E' \) is closed.
                \end{proof}
            \item[(b)] Prove that \( E  \) and \( \overline{E} \) have the same limit points.
                \begin{proof}
                To prove that \( E  \) and \( \overline{E} \) have the same limit points, we need to show that \( E' = (\overline{E})' \). We will prove the following two inclusions:
                \begin{enumerate}
                    \item[(1)] \( E' \subseteq  (\overline{E})' \)
                    \item[(2)] \( (\overline{E})' \subseteq E' \).
                \end{enumerate}
                Starting with (1), let \( x \in E' \). Then \( x  \) is a limit point of \( E \). Thus, for all \( \epsilon > 0  \), \( {N}_{\epsilon}(x) \cap E \setminus  \{ x \}  \neq \emptyset \). Since \( E \subseteq \overline{E} \), we must have that   
                \[  {N}_{\epsilon}(x) \cap \overline{E} \setminus  \{ x \} \neq \emptyset. \]
                Thus, \( x \in \overline{E}' \) which proves (1).

                Now with (2), let \( x \in \overline{E}' \). Then for all \( \epsilon > 0  \), \( {N}_{\epsilon}(x) \cap \overline{E} \setminus  \{ x \} \neq \emptyset \). From this nonempty intersection, pick \( y \neq x   \) such that \( y \in \overline{E} \); that is, \( y \in E  \) or \( y \in E' \). If \( y \in E  \), then \( x  \) is a limit point of \( E  \) and thus \( x \in E' \). If \( y \in E' \), then \( y  \) is a limit point of \( E  \). Hence, for all \( \delta > 0  \) \( {N}_{\delta}(y) \cap E \setminus  \{ y \} \neq \emptyset \). From this, we can pick \( q \neq y \neq x  \) such that \( q \in E  \). But this tell us that  
                \[  {N}_{\epsilon}(x) \cap E \setminus \{ x \} \neq \emptyset, \]
                proving that \( x  \) is a limit point of \( E \) and so \( x \in E' \). 
                \end{proof}
            \item[(c)] Construct an example that shows \( E  \) and \( E' \) do not necessarily have the same limit points?
                \begin{proof}
                Suppose we have the following set \( E = \{ x < \frac{ 1 }{ n }  : n \in \N, x \in \R  \}  \) and its set of limit points \( E' = \{ \frac{ 1 }{ n }  : n \in \N  \}   \). Do these two sets necessarily have the same limit points?
                \end{proof}
        \end{enumerate}
        \item Construct a bounded set of real numbers with exactly three limit points.
            \begin{solution}
            Consider \( (a,b) \subseteq \R  \). Then the three limit points of this set are \( a,b \) and a \( p \in \Q  \) with \( a < p < b \) since \( \Q  \) is dense in \( \R  \).
            \end{solution}
        \item Let \( (X,d)  \) be a metric space and \( E \subseteq X  \).
            \begin{enumerate}
                \item[(a)] Prove that \( E^{\circ} \) is always open, that is \( (E^{\circ})^{\circ} = E^{\circ} \).
                    \begin{proof}
                    To show that \( E^{\circ} \) is open, we will show that \( (E^{\circ})^{\circ} \). To do this, we will need to show the following two inclusions:
                    \begin{enumerate}
                        \item[(1)] \( (E^{\circ})^{\circ} \subseteq E^{\circ} \)
                        \item[(2)] \( E^{\circ} \subseteq (E^{\circ})^{\circ} \).
                    \end{enumerate}
                    Starting with (1), let \( x \in (E^{\circ})^{\circ} \). By definition of interior points, there must exists a \( \delta > 0  \) such that \( {N}_{\delta}(x) \subseteq  E^{\circ} \). Clearly, we have \( x \in E^{\circ} \) which proves (1).   

                    Now, let \( x \in E^{\circ} \). Then there exists \( \epsilon > 0  \) such that \( {N}_{\epsilon}(x) \subseteq E  \) where \( x \in E  \). Suppose for sake of contradiction that \( x  \) is NOT in the interior of \( E^{\circ}  \). Then for all \( \delta > 0  \), \( {N}_{\delta}(x) \not\subseteq E^{\circ}  \). That is, \( {N}_{\delta}(x) \cap (E^{\circ})^{c} \neq \emptyset  \). Since \( E^{\circ} \subseteq E  \), it follows that \( x \in E^{c} \). But this is a contradiction because we had assume that \( x \in E^{\circ}  \) and so \( x \in E  \). Thus, we must have \( x  \) lie in the interior of \( E^{\circ} \). 
                From (1) and (2), we conclude that \( E^{\circ} \) is open.
                    \end{proof}
                \item[(b)] If \( G \subseteq E   \) and \( G  \) is open, prove that \( G \subseteq E^{\circ} \).
                    \begin{proof}
                    Suppose \( G \subseteq E  \) and \( G  \) is open. We will show that \( G \subseteq E^{\circ} \). Let \( x \in G  \). Since \( G  \) is open, there exists \( \delta > 0  \) such that \( {N}_{\epsilon}(x) \subseteq G \subseteq E   \). Thus, \( x  \) must be in the interior of \( E  \); that is, \( x \in E^{\circ} \). Hence, \( G \subseteq E^{\circ} \).
                    \end{proof}
                \item[(c)] Prove that \( E^{\circ} = \overline{E^{c}}^{c} \).
                    \begin{proof}
                    We will show that \( E^{\circ} = (\overline{E^{c}})^{c} \). We will show the following two inclusions:        
                    \begin{enumerate}
                        \item[(1)] \( E^{\circ} \subseteq  (\overline{E^{c}})^{c} \),
                        \item[(2)] \(  (\overline{E^{c}})^{c} \subseteq E^{\circ}\).
                    \end{enumerate}
                    Starting with (1), let \( x \in E^{\circ} \). Then there exists a \( \delta > 0 \) such that \( {N}_{\delta}(x) \subseteq  E   \). This implies that \( x  \) cannot be in \( E^{c} \). Otherwise, \( x  \) is not an interior point of \( E  \). Subsequently, we have that for all \( \delta > 0  \), \( {N}_{\delta}(x) \cap E^{c} = \emptyset \). Thus, neither \( x \in E^{c}  \) nor \( x \in (E^{c})' \). Hence, \( x \in (\overline{E^{c}})^{c} \).    

                    With (2), let \( x \in (\overline{E^{c}})^{c}  \). Then \( x \notin \overline{E^{c}} \). Since \( \overline{E^{c}} = E^{c} \cup (E^{c})'  \), we have \( x \notin E^{c} \) and \( x \notin (E^{c})' \). Then there exists \( \delta > 0  \) such that  
                    \[   {N}_{\delta}(x) \cap E^{c} \setminus  \{ x \}  = \emptyset \Longleftrightarrow {N}_{\delta}(x) \cap E^{c} = \emptyset.    \]
                    Since \( x \notin E^{c} \), we must have \( x \in E  \). Thus, \( {N}_{\delta}(x) \subseteq E  \) for some \( \delta > 0  \) and so \( x \in E^{\circ} \) which proves (2). 
                    \end{proof}
                \item[(d)] Give an example that shows that \( E  \) and \( \overline{E} \) do not always have the same interiors.
                    \begin{proof}
                    Let \( x \in \R  \). Consider the following open ball in the set of real numbers
                    \[  {N}_{1/n}(x) = \Big\{ y \in \R : |  x - y | < \frac{ 1 }{ n }  \Big\}.    \]
                    Since \( {N}_{1/n}(x) \) is open, we know that \( {N}_{1/n}(x) = ({N}_{1/n}(x))^{\circ} \). Note that the closure of this set is 
                    \[  \overline{{N}_{1/n}(x)} = \Big\{ y \in \R : | x - y |  \leq \frac{ 1 }{ n } \Big\}. \]
                    But this is also equal to the interior of the closure \( {N}_{1/n}(x) \). Clearly, we see that
                    \[  (\overline{{N}_{1/n}(x)})^{\circ} \neq {N}_{1/n}(x) \]
                    \end{proof}
                \item[(e)] Give an example that shows that \( E  \) and \( E^{\circ} \) do not always have the same closures.
                    \begin{proof}
                        Consider the set of rational numbers \( \Q \). Then observe that \( \Q^{\circ} \) is empty since for all \( \epsilon > 0  \), \( {N}_{\epsilon}(x)  \) contains irrational numbers which are not contained in \( \Q  \). Hence, \( \overline{\Q^{\circ}} = \emptyset \). But notice that the closure \( \overline{\Q} = \R  \). Clearly, \( \overline{\Q} \neq \overline{\Q^{\circ}} \).  
                    \end{proof}
            \end{enumerate}
        \item Let \( (X,d) \) be a metric space and let \( A \subseteq X  \). Prove that \( A  \) is dense in \( X  \) if and only if any nonempty open set in \( X  \) has a nonempty intersection with \( A  \).
            \begin{proof}
            \( (\Longrightarrow) \) Let \( B  \) be a nonempty open set in \( X  \). Let \( b \in B  \). We will show that \( A \cap B \neq \emptyset \). Since \( b \in X  \) and \( A  \) is dense in \( X  \), either 
            \begin{enumerate}
                \item[(1)] \( b   \) is a limit point of \( A  \) or
                \item[(2)] \( b \in A  \).
            \end{enumerate}
            If \( b  \) is a limit point of \( A  \), then for all \( \delta > 0  \), 
            \[  {N}_{\delta}(b) \cap A \setminus  \{ b \} \neq \emptyset.  \]
            Since \( B \) is an open set, we know that \( {N}_{\delta}(b) \subseteq B \). Together with the fact that \( A \setminus  \{ b \} \subseteq  A  \), we can conclude that 
            \[  B \cap A \neq \emptyset. \]
            On the other hand, \( b \in A  \) immediately implies that \( B \cap A \neq \emptyset \).

            \( (\Longleftarrow) \) Let \( p \in X  \) and let \( A \subseteq X  \). We need to show that \( A  \) is dense in \( X  \); that is, we need to show that \( p  \) is either a limit point of \( A  \) or \( p \in A  \). Suppose that \( p \notin A  \). We will show that \( p  \) is a limit point of \( A  \). Let \( \epsilon > 0 \) and consider the open neighborhood \( {N}_{\epsilon}(p) \). By assumption, the open ball \( {N}_{\epsilon}(p) \) contains a nonempty intersection with \( A  \); that is,  
            \[  {N}_{\epsilon}(p) \cap A \neq \emptyset. \]
            Subsequently, we have
            \[  {N}_{\epsilon}(p) \cap A \setminus  \{ p\} \neq \emptyset.  \]
            Thus, \( p  \) is a limit point of \( A  \). On the other hand, if \( p  \) is not a limit point of \( A  \), then \( p \) is an isolated point of \( A  \). Thus, \( p \in A  \). 
        \end{proof}
        \item Let \( (X,d) \) be a metric space and \( {A}_{i} \subseteq X \) for all \( i \in \N \). 
            \begin{enumerate}
                \item[(a)] Prove that for all \( n \in \N  \), we have \( \overline{\bigcup_{ i=1  }^{ n }  } {A}_{i} = \bigcup_{ i=1 }^{ n }  \overline{{A}_{i}}. \)
                    \begin{proof}
                    For all \( i \in \N  \), we see that 
                    \begin{align*}
                        \overline{\bigcup_{ i=1  }^{ n }  {A}_{i}} &= \Big(  \bigcup_{ i =1  }^{ n }  {A}_{i} \Big) \cup \Big(  \bigcup_{ i=1 }^{ n } {A}_{i} \Big)' \\
                                                                   &= \Big( \bigcup_{ i= 1}^{ n }  {A}_{i} \Big) \cup \Big(  \bigcup_{ i=1  }^{ n }  {A}_{i}' \Big) \\
                                                                   &= \bigcup_{ i=1  }^{ n }  {A}_{i} \cup {A}_{i}' \\
                &= \bigcup_{ i=1  }^{ n }  \overline{{A}_{i}}.
                    \end{align*}
                    \end{proof}
                \item[(b)] Prove that \( \bigcup_{ i=1 }^{ \infty  }  \overline{{A}_{i}} \subseteq  \overline{\bigcup_{ i=1 }^{ \infty  } } {A}_{i} \).
                    \begin{proof}
                    Our goal is to show that \( \bigcup_{ i=1 }^{ \infty  }  \overline{{A}_{i}} \subseteq  \overline{\bigcup_{ i=1 }^{ \infty  } } {A}_{i} \). Let \( x \in \bigcup_{ i=1  }^{ \infty  }  \overline{{A}_{i}} \). Then for some \( i \in \N  \), we see that \( x \in \overline{{A}_{i}} \). Then either \( x \in {A}_{i} \) or \( x \in {A}_{i}' \). If \( x \in {A}_{i} \) and \( {A}_{i} \subseteq \bigcup_{ i=1  }^{ \infty  } {A}_{i} \), we see that \( x \in \overline{\bigcup_{ i=1 }^{ \infty  } {A}_{i}} \). Suppose \( x \in {A}_{i}' \). Note that \( {A}_{i}' \subseteq \Big(  \bigcup_{ i=1  }^{ \infty  } {A}_{i} \Big)' \). So, \( x \in \Big(  \bigcup_{ i=1  }^{ \infty  } {A}_{i} \Big)' \). Thus, \( x \in \overline{\bigcup_{ i=1  }^{ \infty  } {A}_{i}} \). Hence, we have 
                    \[   \bigcup_{ i=1 }^{ \infty  }  \overline{{A}_{i}} \subseteq  \overline{\bigcup_{ i=1 }^{ \infty  } } {A}_{i}.  \]
                    \end{proof}
                \item[(c)] Show, by an example, that the inclusion in part (b) can be proper.
                    \begin{solution}
                    Consider the sequence of open intervals in \( \R  \), \( {A}_{i} = (  0 , 1 + \frac{ 1 }{ i }) \). Observe that  
                    \[  \bigcup_{ i=1  }^{ \infty  } \overline{\Big(0, 1 + \frac{ 1 }{ i }  \Big)} = \bigcup_{ i=1  }^{ \infty  }  \Big[ 0, 1 + \frac{ 1 }{ i } \Big] = [0,2).   \]
                    But we have
                    \[  \overline{\bigcup_{ i=1  }^{ \infty  }\Big( 0, 1 + \frac{ 1 }{ i }  \Big)} = \overline{(0,2)} = [0,2]. \]
                    Clearly, \( [0,2) \) is a proper subset of \( [0,2] \).
                    \end{solution}
            \end{enumerate}
        \item Let \( (X,d) \) be a metric space and \( {A}_{i} \subseteq X  \), for all \( i \in \N  \).
            \begin{enumerate}
                \item[(a)] Prove that for all \( n \in \N  \), we have \( \Big(  \bigcap_{ i =1  }^{ n }  {A}_{i} \Big)^{\circ} = \bigcap_{ i = 1  }^{ n }  {A}_{i}^{\circ} \).
                    \begin{proof}
                    Our goal is to show that \( \Big(  \bigcap_{ i =1  }^{ n }  {A}_{i} \Big)^{\circ} = \bigcap_{ i = 1  }^{ n }  {A}_{i}^{\circ} \); that is, we need to show the following two inclusions: 
                    \begin{enumerate}
                        \item[(1)] \( \Big(  \bigcap_{ i=1  }^{ n }  {A}_{i} \Big)^{\circ} \subseteq \bigcap_{ i=1 }^{ n }  {A}_{i}^{\circ} \).
                        \item[(2)] \( \bigcap_{ i=1 }^{ n }  {A}_{i}^{\circ} \subseteq \Big(  \bigcap_{ i=1  }^{ n }  {A}_{i} \Big)^{\circ} \). 
                    \end{enumerate}
           Starting with (1), let \( x \in \Big(  \bigcap_{ i=1 }^{ n }  {A}_{i} \Big)^{\circ}  \). Then there exists \( \delta > 0  \) such that \( {N}_{\delta}(x) \subseteq \bigcap_{  i =1  }^{ n }  {A}_{i}  \). Thus, \( x \in \bigcap_{ i=1 }^{ n } {A}_{i} \) implies that \( x \in {A}_{i} \) for all \( 1 \leq i \leq n \). But this tell us that \( {N}_{\epsilon}(x) \subseteq {A}_{i} \) for all \( 1 \leq i \leq n  \), and so \( x \in A_{i} \). Hence,  
           \[  x \in \bigcap_{ i=1 }^{ n } {A}_{i}^{\circ}. \]
           Thus, we conclude that 
           \[  \Big(  \bigcap_{ i=1 }^{ n } {A}_{i} \Big)^{\circ} \subseteq \bigcap_{ i=1 }^{ n } {A}_{i}^{\circ}. \]

           With (2), let \( x \in \bigcap_{ i=1 }^{ n } {A}_{i}^{\circ} \). Then for all \( 1 \leq i \leq n  \), we have \( x \in {A}_{i}^{\circ} \). By definition, there exists \( \delta > 0  \) such that \( {N}_{\epsilon}(x) \subseteq {A}_{i} \) for all \( 1 \leq i \leq n \). Thus, \( x \in {A}_{i} \) for all \( 1 \leq i \leq n \) and so                    
           \[  x \in \bigcap_{ i=1 }^{ n } {A}_{i}. \]
          Subsequently, \( {N}_{\delta}(x) \subseteq \bigcap_{ i=1 }^{ n } {A}_{i} \) and so \( x \in \Big(  \bigcap_{ i=1 }^{ n } {A}_{i} \Big)^{\circ} \). Therefore, we have 
          \[  \bigcap_{ i=1 }^{ n }  {A}_{i}^{\circ} \subseteq \Big(  \bigcap_{ i=1 }^{ n } {A}_{i} \Big)^{\circ}.  \]
          Hence, (1) and (2) gives us our desired result.
       \end{proof}
                \item[(b)] Prove that \( \Big(  \bigcap_{ i \in \N  }^{   }  {A}_{i} \Big)^{\circ} \subseteq \bigcap_{ i \in \N  }^{  } {A}_{i}^{\circ} \).
                    \begin{proof}
                    Let \( x \in \Big(  \bigcap_{ i \in \N  }^{  }  {A}_{i} \Big)^{\circ} \). Then there exists \( \delta > 0  \) such that \( {N}_{\delta}(x) \subseteq \bigcap_{ i \in \N  }^{  }  {A}_{i} \). Hence, \( x \in \bigcap_{ i \in \N }^{   } {A}_{i} \) if and only if \( x \in {A}_{i} \) for all \( i \in \N \). But this also tells us that \( {N}_{\delta}(x) \subseteq {A}_{i} \). Hence, \( x \in {A}_{i}^{\circ} \) for all \( i \in \N \). Thus,   
                    \[  \Big(  \bigcup_{ i \in \N  }^{  } {A}_{i} \Big)^{\circ} \subseteq  \bigcup_{ i \in \N  }^{  }  {A}_{i}^{\circ}. \]
                    \end{proof}
                \item[(c)] Give an example where equality does not hold in (b).
                    \begin{solution}
                    Consider the sequence of closed intervals \( {A}_{i}  \) in \( \R  \) defined by 
                    \[  {A}_{i} = \Big[ a - \frac{ 1 }{ i } , b\Big]. \]
                    Then the interior of these closed intervals are
                    \[  {A}_{i}^{\circ} = \Big(  a - \frac{ 1 }{ i } , b \Big). \]
                    Now, observe that
                    \[  \bigcup_{ i \in \N  }^{  }  {A}_{i}^{\circ} = [a,b). \]
                    But 
                    \[  \Big(  \bigcup_{ i \in \N  }^{  } {A}_{i} \Big)^{\circ} = [a,b]^{\circ} = (a,b). \]
                    Hence, we have \( (a,b) \subset [a,b) \).
                    \end{solution}
            \end{enumerate}
        \item Let \( (X,d) \) be a metric space and \( {A}_{i} \subseteq X  \), for all \( i \in \N  \).
            \begin{enumerate}
                \item[(a)] \( \bigcup_{ i \in \N  }^{  }  {A}_{i}^{\circ} \subseteq  \Big(  \bigcup_{ i \in \N  }^{  }  {A}_{i} \Big)^{\circ} \).
                    \begin{proof}
                    Let \( x \in \bigcup_{ i \in \N  }^{  }  {A}_{i}^{\circ} \). Then for some \( i \in \N  \), we have \( x \in {A}_{i}^{\circ} \). Then there exists a \( \delta > 0  \) such that \( {N}_{\delta}(x) \subseteq {A}_{i} \). But note that 
                    \[  {A}_{i} \subseteq \bigcup_{ i \in \N  }^{  }  {A}_{i} \]
                    and so
                    \[ {N}_{\delta}(x) \subseteq \bigcup_{ i \in \N  }^{  } {A}_{i}.  \]
                    This tells us that \( x \in \Big(  \bigcup_{ i \in \N  }^{  } {A}_{i} \Big)^{\circ}  \). Therefore, 
                    \[  \bigcup_{ i \in \N  }^{  } {A}_{i}^{\circ} \subseteq \Big(  \bigcup_{ i \in \N  }^{  } {A}_{i} \Big)^{\circ}. \]
                    \end{proof}
                \item[(b)] Give an example of a finite collection in which equality does not hold in (a).
                    \begin{solution}
                        Consider the two closed intervals \( A = [0,1] \) and \( B = [1,2] \). Note that \( A^{\circ} = (0,1) \) and \( B^{\circ} = (1,2) \). Thus,
                        \[  A^{\circ} \cup B^{\circ} = (0,1) \cup (1,2) \]
                        whereas
                        \[  (A \cup B)^{\circ} = ([0,2])^{\circ} = (0,2). \]
                        So, \( (0,1) \cup (1,2) \subseteq (0,2) \) since \( 1 \) is not included in \( (0,1) \cup (1,2) \).
                    \end{solution}
            \end{enumerate}
\end{enumerate}



\end{document}
