\documentclass[a4paper]{article}
\usepackage{standalone}
\usepackage{import}
\usepackage[utf8]{inputenc}
\usepackage[T1]{fontenc}
% \usepackage{fourier}
\usepackage{textcomp}
\usepackage{hyperref}
\usepackage[english]{babel}
\usepackage{url}
% \usepackage{hyperref}
% \hypersetup{
%     colorlinks,
%     linkcolor={black},
%     citecolor={black},
%     urlcolor={blue!80!black}
% }
\usepackage{graphicx} \usepackage{float}
\usepackage{booktabs}
\usepackage{enumitem}
% \usepackage{parskip}
% \usepackage{parskip}
\usepackage{emptypage}
\usepackage{subcaption}
\usepackage{multicol}
\usepackage[usenames,dvipsnames]{xcolor}
\usepackage{ocgx}
% \usepackage{cmbright}


\usepackage[margin=1in]{geometry}
\usepackage{amsmath, amsfonts, mathtools, amsthm, amssymb}
\usepackage{thmtools}
\usepackage{mathrsfs}
\usepackage{cancel}
\usepackage{bm}
\newcommand\N{\ensuremath{\mathbb{N}}}
\newcommand\R{\ensuremath{\mathbb{R}}}
\newcommand\Z{\ensuremath{\mathbb{Z}}}
\renewcommand\O{\ensuremath{\emptyset}}
\newcommand\Q{\ensuremath{\mathbb{Q}}}
\newcommand\C{\ensuremath{\mathbb{C}}}
\newcommand\F{\ensuremath{\mathbb{F}}}
\DeclareMathOperator{\sgn}{sgn}
\DeclareMathOperator{\diam}{diam}
\DeclareMathOperator{\LO}{LO}
\DeclareMathOperator{\UP}{UP}
\DeclareMathOperator{\card}{card}
\DeclareMathOperator{\Arg}{Arg}
\DeclareMathOperator{\Dom}{Dom}
\DeclareMathOperator{\Log}{Log}
\DeclareMathOperator{\dist}{dist}
% \DeclareMathOperator{\span}{span}
\usepackage{systeme}
\let\svlim\lim\def\lim{\svlim\limits}
\renewcommand\implies\Longrightarrow
\let\impliedby\Longleftarrow
\let\iff\Longleftrightarrow
\let\epsilon\varepsilon
\usepackage{stmaryrd} % for \lightning
\newcommand\contra{\scalebox{1.1}{$\lightning$}}
% \let\phi\varphi
\renewcommand\qedsymbol{$\blacksquare$}

% correct
\definecolor{correct}{HTML}{009900}
\newcommand\correct[2]{\ensuremath{\:}{\color{red}{#1}}\ensuremath{\to }{\color{correct}{#2}}\ensuremath{\:}}
\newcommand\green[1]{{\color{correct}{#1}}}

% horizontal rule
\newcommand\hr{
    \noindent\rule[0.5ex]{\linewidth}{0.5pt}
}

% hide parts
\newcommand\hide[1]{}

% si unitx
\usepackage{siunitx}
\sisetup{locale = FR}
% \renewcommand\vec[1]{\mathbf{#1}}
\newcommand\mat[1]{\mathbf{#1}}

% tikz
\usepackage{tikz}
\usepackage{tikz-cd}
\usetikzlibrary{intersections, angles, quotes, calc, positioning}
\usetikzlibrary{arrows.meta}
\usepackage{pgfplots}
\pgfplotsset{compat=1.13}

\tikzset{
    force/.style={thick, {Circle[length=2pt]}-stealth, shorten <=-1pt}
}

% theorems
\makeatother
\usepackage{thmtools}
\usepackage[framemethod=TikZ]{mdframed}
\mdfsetup{skipabove=1em,skipbelow=1em}

\theoremstyle{definition}

\declaretheoremstyle[
    headfont=\bfseries\sffamily\color{ForestGreen!70!black}, bodyfont=\normalfont,
    mdframed={
        linewidth=1pt,
        rightline=false, topline=false, bottomline=false,
        linecolor=ForestGreen, backgroundcolor=ForestGreen!5,
    }
]{thmgreenbox}

\declaretheoremstyle[
    headfont=\bfseries\sffamily\color{NavyBlue!70!black}, bodyfont=\normalfont,
    mdframed={
        linewidth=1pt,
        rightline=false, topline=false, bottomline=false,
        linecolor=NavyBlue, backgroundcolor=NavyBlue!5,
    }
]{thmbluebox}

\declaretheoremstyle[
    headfont=\bfseries\sffamily\color{NavyBlue!70!black}, bodyfont=\normalfont,
    mdframed={
        linewidth=1pt,
        rightline=false, topline=false, bottomline=false,
        linecolor=NavyBlue
    }
]{thmblueline}

\declaretheoremstyle[
    headfont=\bfseries\sffamily, bodyfont=\normalfont,
    numbered = no,
    mdframed={
        rightline=true, topline=true, bottomline=true,
    }
]{thmbox}

\declaretheoremstyle[
    headfont=\bfseries\sffamily, bodyfont=\normalfont,
    numbered=no,
    % mdframed={
    %     rightline=true, topline=false, bottomline=true,
    % },
    qed=\qedsymbol
]{thmproofbox}

\declaretheoremstyle[
    headfont=\bfseries\sffamily\color{NavyBlue!70!black}, bodyfont=\normalfont,
    numbered=no,
    mdframed={
        rightline=false, topline=false, bottomline=false,
        linecolor=NavyBlue, backgroundcolor=NavyBlue!1,
    },
]{thmexplanationbox}

\declaretheorem[
    style=thmbox, 
    % numberwithin = section,
    numbered = no,
    name=Definition
    ]{definition}

\declaretheorem[
    style=thmbox, 
    name=Example,
    ]{eg}

\declaretheorem[
    style=thmbox, 
    % numberwithin = section,
    name=Proposition]{prop}

\declaretheorem[
    style = thmbox,
    numbered=yes,
    name =Problem
    ]{problem}

\declaretheorem[style=thmbox, name=Theorem]{theorem}
\declaretheorem[style=thmbox, name=Lemma]{lemma}
\declaretheorem[style=thmbox, name=Corollary]{corollary}

\declaretheorem[style=thmproofbox, name=Proof]{replacementproof}

\declaretheorem[style=thmproofbox, 
                name = Solution
                ]{replacementsolution}

\renewenvironment{proof}[1][\proofname]{\vspace{-1pt}\begin{replacementproof}}{\end{replacementproof}}

\newenvironment{solution}
    {
        \vspace{-1pt}\begin{replacementsolution}
    }
    { 
            \end{replacementsolution}
    }

\declaretheorem[style=thmexplanationbox, name=Proof]{tmpexplanation}
\newenvironment{explanation}[1][]{\vspace{-10pt}\begin{tmpexplanation}}{\end{tmpexplanation}}

\declaretheorem[style=thmbox, numbered=no, name=Remark]{remark}
\declaretheorem[style=thmbox, numbered=no, name=Note]{note}

\newtheorem*{uovt}{UOVT}
\newtheorem*{notation}{Notation}
\newtheorem*{previouslyseen}{As previously seen}
% \newtheorem*{problem}{Problem}
\newtheorem*{observe}{Observe}
\newtheorem*{property}{Property}
\newtheorem*{intuition}{Intuition}

\usepackage{etoolbox}
\AtEndEnvironment{vb}{\null\hfill$\diamond$}%
\AtEndEnvironment{intermezzo}{\null\hfill$\diamond$}%
% \AtEndEnvironment{opmerking}{\null\hfill$\diamond$}%

% http://tex.stackexchange.com/questions/22119/how-can-i-change-the-spacing-before-theorems-with-amsthm
\makeatletter
% \def\thm@space@setup{%
%   \thm@preskip=\parskip \thm@postskip=0pt
% }
\newcommand{\oefening}[1]{%
    \def\@oefening{#1}%
    \subsection*{Oefening #1}
}

\newcommand{\suboefening}[1]{%
    \subsubsection*{Oefening \@oefening.#1}
}

\newcommand{\exercise}[1]{%
    \def\@exercise{#1}%
    \subsection*{Exercise #1}
}

\newcommand{\subexercise}[1]{%
    \subsubsection*{Exercise \@exercise.#1}
}


\usepackage{xifthen}

\def\testdateparts#1{\dateparts#1\relax}
\def\dateparts#1 #2 #3 #4 #5\relax{
    \marginpar{\small\textsf{\mbox{#1 #2 #3 #5}}}
}

\def\@lesson{}%
\newcommand{\lesson}[3]{
    \ifthenelse{\isempty{#3}}{%
        \def\@lesson{Lecture #1}%
    }{%
        \def\@lesson{Lecture #1: #3}%
    }%
    \subsection*{\@lesson}
    \testdateparts{#2}
}

% \renewcommand\date[1]{\marginpar{#1}}


% fancy headers
\usepackage{fancyhdr}
\pagestyle{fancy}

\makeatother

% notes
\usepackage{todonotes}
\usepackage{tcolorbox}

\tcbuselibrary{breakable}
\newenvironment{verbetering}{\begin{tcolorbox}[
    arc=0mm,
    colback=white,
    colframe=green!60!black,
    title=Opmerking,
    fonttitle=\sffamily,
    breakable
]}{\end{tcolorbox}}

\newenvironment{noot}[1]{\begin{tcolorbox}[
    arc=0mm,
    colback=white,
    colframe=white!60!black,
    title=#1,
    fonttitle=\sffamily,
    breakable
]}{\end{tcolorbox}}

% figure support
\usepackage{import}
\usepackage{xifthen}
\pdfminorversion=7
\usepackage{pdfpages}
\usepackage{transparent}
\newcommand{\incfig}[1]{%
    \def\svgwidth{\columnwidth}
    \import{./figures/}{#1.pdf_tex}
}

% %http://tex.stackexchange.com/questions/76273/multiple-pdfs-with-page-group-included-in-a-single-page-warning
\pdfsuppresswarningpagegroup=1



\pagestyle{fancy}
\fancyhf{}

\title{Math 230A: Homework 5}
\author{Lance Remigio}

\begin{document}
\maketitle    
\lhead{Math 230A: Homework 5}
\chead{Lance Remigio}
\rhead{\thepage}

\begin{problem}
   Mark each statement True or False. Let \( (X,d) \) be a metric space and \( K \subseteq X  \). 
   \begin{enumerate}
       \item If \( Y \subseteq X   \) and \( \{ {G}_{\alpha} \}  \) is a collection of subsets of \( Y  \) that are open relative to \( Y  \), then \( \bigcup_{ \alpha }^{  } {G}_{\alpha} \) is open relative to \( Y  \).
        \item If there exists some open cover of \( K   \) which has a finite subcover, then \( K  \) is compact. 
        \item If \( K  \) is compact, then \( K' \subseteq K  \).
        \item If \( K  \) is closed, then \( K  \) is compact.
        \item If \( K  \) is compact and \( E \subseteq K \), then \( E  \) is compact.
        \item Consider \( E = [-10,10] \subseteq \R  \) and the open cover of \( E  \) by \( \Phi = \{ (x-1,x + 1) : x \in E  \}  \). Then the collection \( \{ (x - \frac{ 1 }{ 2 }  , x + \frac{ 1 }{ 2 } ) : x \in E  \}  \) is a subcover of \( \Phi \).
        \item Let \( E  \) and \( \Phi \) be as above. Then the collection \( \{ (x-1, x+1): x = -10,-9,-8, \dots, 8, 9, 10 \}  \) is a subcover of \( \Phi \).
        \item Let \( E  \) and \( \Phi \) be as above. Then the collection \( \{ (x-1, x+1): x = -10,-8, -6, \dots, 6, 8, 10 \}  \) is a subcover of \( \Phi \).
   \end{enumerate}
\end{problem}
   \begin{solution}
   \begin{enumerate}
       \item True.
        \item False.
        \item True.
        \item False.
        \item 
        \item 
        \item
   \end{enumerate}
   \end{solution}

   \begin{problem}
       Show that compact implies bounded.
   \end{problem}
   \begin{proof}
    Let \( E \subseteq X  \). Suppose \( E  \) is compact. Our goal is to show that \( E  \) is bounded set in \( X  \); that is, there exists \( p \in X  \) and \( \delta > 0  \) such that \( E \subseteq {N}_{\delta}(q) \). Let \( x \in E  \). To this end, we will show that there exists a \( p \in X  \) and \( \delta > 0  \) such that \( E \subseteq  {N}_{\delta}(x) \). Since \( E  \) is compact, \( E  \) is a closed set. Hence, we know that 
    \[  E = \overline{E}  =  \{ x \in X : \forall \epsilon > 0 \ {N}_{\epsilon}(x) \cap E \neq \emptyset \}.  \]
    Choose \( \delta = \epsilon + 1  \). Since \( E  \) is closed, then there exists \(  \) a \(p \in {N}_{\epsilon}(x) \cap E  \) such that \( p \in {N}_{\epsilon}(x) \) and \( p \in E  \) (that is, \( p \in X  \)). Since \( p \in {N}_{\epsilon}(x) \), we have
    \[  d(x,p) < \epsilon < \epsilon + 1 = \delta. \]
    Hence, we see that \( p \in {N}_{\delta}(x) \). Thus, \( E \subseteq  {N}_{\delta}(x) \) and so \( E  \) is bounded.
   \end{proof}

   \begin{problem}
      Show the union of finitely many compact sets is compact. 
   \end{problem}
   \begin{proof}
       Let \( (X,d)  \) be a metric space. Set \( K =  \bigcup_{ i=1 }^{ n } {K}_{i} \) with \( {K}_{i} \) compact for all \( 1 \leq i \leq n  \). Our goal is to show that \( K  \) is compact. Let \( \{ {O}_{\alpha} \}_{\alpha \in \Lambda} \) be a collection of open sets in \( X  \) that forms an open cover for \( K  \). Since \( {K}_{i} \) is compact for all \( 1 \leq i \leq n  \), we can find a finite subcover for each \( {K}_{i} \). Thus, we have  
       \[  {K}_{i} \subseteq \bigcup_{ j=1  }^{ {m}_{i} } {O}_{i,j} \ \ 1 \leq j \leq {n}_{i}.  \]
       Note that each \( \bigcup_{ j=1 }^{ {m}_{i} } {O}_{i,j} \) is open since each \( {O}_{i,j} \) is open. Furthermore, 
       \[  \sum {m}_{i} = n.  \]
       Denote each of these unions as 
       \[  {Q}_{\alpha_i} = \bigcup_{ j=1  }^{ {m}_{i} }  {O}_{i,j}. \]
       Thus, we see that 
       \begin{align*}
           K = \bigcup_{ i=1  }^{ n }  {K}_{i} &\subseteq \bigcup_{ i=1  }^{ n } {Q}_{{\alpha}_{i}}
       \end{align*}
       and so \( K  \) is compact.
   \end{proof}

   \begin{problem}
       Show that an arbitrary intersection of compact sets is compact. (Hint. A closed subset of a compact set is compact)
   \end{problem}
   \begin{proof}
       Let \( \{ {O}_{\alpha} \}_{\alpha \in \Lambda} \) be a collection of open sets in \( X  \) and let \( \{ {K}_{\alpha} \}_{\alpha \in \Lambda} \) be a collection of compact sets in \( X  \). Define
       \[  K = \bigcap_{  \alpha  }^{  } {K}_{\alpha}. \tag{1} \]
       Since each \( {K}_{\alpha} \) is compact, we know that each \( {K}_{\alpha} \) must be closed. Therefore, the arbitrary intersection above must be closed and so \( K  \) is closed. But observe that 
       \[  \bigcup_{ \alpha }^{  } {K}_{\alpha} \subseteq  {K}_{\alpha} \]
       and that \( K_{\alpha}  \) is a closed set. Thus, \( K  \) must be compact as well! 
   \end{proof}

    \begin{problem}
        A metric space \( (X,d) \) is called \textbf{separable} if it contains a countable subset \( E  \) which is dense in \( X  \). For example, \( \R  \) is separable because \( \Q  \) is a countable set which is dense in \( \R  \). Show that \( \R^{2}  \) is separable. (Hint: Consider the set of points which have only rational coordinates.) 
    \end{problem}
    \begin{proof}
        Note that \( \R^{2} \) is just \( \R \times \R  \). Recall that a finite product of countable sets is countable. Thus, we see that \( \Q^{2} = \Q \times \Q  \) is countable. All that is left to show is that \( \Q^{2} \) is dense in \( \R^{2} \). Let \( {x}_{1}, {x}_{2}, {y}_{1}, {y}_{2} \in \R  \) with \( {x}_{1} < {x}_{2} \) and \( {y}_{1} < {y}_{2} \). Since \( \Q  \) is dense in \( \R  \), there must exists \( p \in \Q  \) and \( q \in \Q  \) such that \( {x}_{1} < p < {x}_{2} \) and \( {y}_{1} < q < {y}_{2} \), respectively. Denote the points in \( \R^{2} \) as \( {p}_{1} = ({x}_{1}, {y}_{1}) \),  \( {p}_{2} = ({x}_{2}, {y}_{2}) \) and \( r = (p,q) \). Thus, we have that   
        \[  {p}_{1} < r < {p}_{2}.  \]
        Thus, \( \Q^{2}  \) must be dense in \( \R^{2} \).
    \end{proof}

   \begin{problem}
       Let \( (X,d) \) be a separable metric space and \( \O \neq A \subseteq X  \). Prove that the collection of the isolated points of \( A  \) is at most countable.
   \end{problem}
   \begin{proof}
       Let \( (X,d) \) be a separable metric space and \( \O = A \subseteq  X  \). Denote the set of isolated points as 
       \[  {A}_{I} = \{ x \in X : \exists \epsilon > 0 \ \text{such that} \ {N}_{\epsilon}(x) \cap A = \emptyset \}.  \]
       By assumption, \( X  \) contains a subset \( E  \) such that \( E  \) is countable and \( \overline{E} = X  \). Our goal is to show that \( {A}_{I} \) is at most countable. It suffices to show that \( {A}_{I} \subseteq E  \). Let \( x \in {A}_{I} \). Then there exists \( \epsilon > 0  \) such that \( {N}_{\epsilon}(x) \cap A = \emptyset \). Hence, \( {N}_{\epsilon}(x) \subseteq  A^{c} \) for some \( \epsilon > 0  \) and so \( A^{c} \) is an open set in \( X  \). By problem 19 of homework 4, we see that  
       \[  A^{c} \cap E \neq \emptyset. \]
       Thus, \( x  \) must be contained in the intersection above. So, \( {A}_{I} \subseteq E  \). Since \( E  \) is countable, \( {A}_{I} \) must be at most countable and we are done. 
   \end{proof}

   \begin{problem}
       Let \( (X,d) \) be a metric space. A collection \( \{ {V}_{\alpha} \} \) of open subsets of \( X  \) is said to be a \textbf{base} for \( X  \) if the following is true: For every \( x \in X  \) and every open set \( G \subseteq X   \) such that \( x \in G  \), we have \( {V}_{\alpha} \subseteq G   \) for some \( \alpha \). In other words, every open set in \( X  \) is the union of a subcollection of \( \{ {V}_{\alpha} \}  \). 

       Prove that the every separable metric space has a countable base. (Hint: Take all neighborhoods with rational radius and center in some countable dense subset of \( X  \).)
   \end{problem}
   \begin{proof}
   Since \( (X,d) \) is a separable metric space, we know that \( X  \) contains a countable dense subset \( E  \); that is, we have 
   \[  X = \overline{E} = E = \{ x \in X : \forall \epsilon > 0 \ {N}_{\epsilon}(x) \cap E \neq \emptyset \}. \]
   Let \( \{ {V}_{\alpha} \}_{\alpha \in \Lambda} \). Let \( x \in X  \) and let \( G \subseteq  X   \) be an open set such that \( x \in G  \). Our goal is to show that \( {V}_{\alpha} \subseteq  G  \) for some \( \alpha \). Since \( X = \overline{E } \), we must have \( x \in \overline{E} \); that is, for all \( \epsilon > 0  \),
   \[  {N}_{\epsilon}(x) \cap E \neq \emptyset. \]
   To this end, let \( \epsilon = \frac{ 1 }{ n }  \) and pick a point \( y \in {N}_{\epsilon}(x) \cap E  \). Then
   \begin{center}
       \( y \in {N}_{1/n}(x) \) and \( y \in E  \).
   \end{center}
    Note that \( {N}_{1/n}(x) \) is an open set in \( X  \). Thus, we can write this set in the following way
    \[  {N}_{1/n}(x) = \bigcup_{ \alpha }^{  } {V}_{\alpha}. \]
    Thus, \( y  \) must be contained in the union above and so \( y \in {V}_{\alpha_0} \) for some \( {\alpha}_{0} \in \Lambda \). Since \( y \in G  \), we must also have \( {V}_{\alpha} \subseteq G  \) and we are done.
   \end{proof}

   \begin{problem}
      Let \( X  \) be a metric space in which every infinite subset has a limit point. Prove that \( X  \) is separable. (Hint: Fix \( \delta > 0  \), and pick \( {x}_{1} \in X  \).) Having chosen \( {x}_{1}, \dots, {x}_{j} \in X  \), choose \( {x}_{j+1} \in X  \), if possible, so that \( d({x}_{i}, {x}_{j+1}) \geq \delta \) for \( i = 1,\dots, j \). Show that this process must stop after a finite number of steps, and that \( X  \) can therefore be covered by finitely many neighborhoods of radius \( \delta  \). Take \( \delta = 1/n \ (n = 1,2,3,\dots) \), and consider the centers of the corresponding neighborhoods.
   \end{problem}
   \begin{proof}
   
   \end{proof}

   \begin{problem}[Extra Credit]
      Let \( (X,d) \) be a metric space and \( Y  \) be a nonempty subset of \( X  \). Let \( E \subseteq Y \). Prove that   
      \begin{center}
          \( E  \) is closed relative to \( Y  \) \( \iff \) \( E = A \cap Y  \) for some closed set \( A \subseteq  X \).
      \end{center}
   \end{problem} 
   \begin{proof}
    \( (\Longrightarrow) \) Assume that \( E  \) is closed relative to \( Y  \). Our goal is to show that \( E = A \cap Y  \); that is, we need to show two inclusions:
    \begin{enumerate}
        \item[(1)] \( E \subseteq A \cap Y  \),
        \item[(2)] \( A \cap Y \subseteq E  \).
    \end{enumerate}
    By assumption, we know that 
    \[  E = \overline{E} = \{ \forall x \in Y, \forall \epsilon > 0 : {N}_{\epsilon}^{Y}(x) \cap E \neq \emptyset \}. \]
    Note that \( {N}_{\epsilon}(x) \subseteq  \overline{{N}_{\epsilon}(x)} \). Set 
    \[  A = \bigcup_{ i=1 }^{ n } \overline{{N}_{{\epsilon}_{i}}(x)}  \]
    with radius \( {\epsilon}_{i} > 0  \) for each \( {N}_{{\epsilon}_{i}} \). Notice that \( A  \) must be a closed set in \( Y  \) because each \( \overline{{N}_{{\epsilon}_{i}}(x)} \) is a closed set. Since 
    \[ {N}_{\epsilon}(x) \cap Y \subseteq  \overline{{N}_{\epsilon}(x)} \cap Y \subseteq E,   \]
    we must have 
    \begin{align*}
        A \cap Y = \Big(  \bigcup_{ i=1  }^{ n }  \overline{{N}_{{\epsilon}_{i}}(x)}  \Big) \cap Y &\subseteq \bigcup_{ i=1 }^{ n } (\overline{{N}_{{\epsilon}_{i}}(x)} \cap Y)  \\
                                                                                                   &\subseteq \bigcup_{ i=1 }^{ n }  E \\
                                                                                                   &= E.
    \end{align*}
    Thus, we see that \( A \cap Y \subseteq E   \) which proves (1).

    Now, we want to show (2). Let \( p \in E  \). Using our definition of \( A  \) and the fact that \( {N}_{\epsilon}(p) \subseteq \overline{{N}_{\epsilon}(p)} \), 
    \begin{align*}
        E \subseteq  {N}_{\epsilon}(p) \cap Y \subseteq \overline{{N}_{\epsilon}(p)} \cap Y &\subseteq \Big[ \bigcup_{ i=1  }^{ n }  \overline{{N}_{{\epsilon}_{i}}(x)}\Big] \cap Y   \\
                                                                                            &= A \cap Y
    \end{align*}
    which shows (2). Hence, (1) and (2) imply that \( E = A \cap Y  \) for some closed set \( A \subseteq  Y \).

    \( (\impliedby) \) Suppose \( E = A \cap Y  \) for some closed set \( A \subseteq X  \). Our goal is to show that \( E  \) is closed relative to \( Y  \). Let \( p  \) be a limit point of \( E  \). 



   \end{proof}
   
   \begin{problem}[Extra Credit]
   Let \( (X,d) \) be a metric space. Let \( E \subseteq  X \). Prove that the following definitions of boundedness are equivalent:
    \begin{itemize}
        \item Rudin's Definition: There exists \( q \in X  \) and \( \epsilon > 0  \) such that \( E \subseteq {N}_{\epsilon}(q) \).
        \item Anthony's Definition: There exists \( R > 0  \) such that for all \( x  \) and \( y  \) in \( E  \), we have \( d(x,y) < R  \).
    \end{itemize}
   \end{problem}
   \begin{proof}
     
   \end{proof}







\end{document}
