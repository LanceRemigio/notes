\documentclass[a4paper]{article}
\usepackage[utf8]{inputenc}
\usepackage[T1]{fontenc}
\usepackage{textcomp}
\usepackage{hyperref}
% \usepackage{fourier}
% \usepackage[dutch]{babel}
\usepackage{url}
% \usepackage{hyperref}
% \hypersetup{
%     colorlinks,
%     linkcolor={black},
%     citecolor={black},
%     urlcolor={blue!80!black}
% }
\usepackage{graphicx}
\usepackage{float}
\usepackage{booktabs}
\usepackage{enumitem}
% \usepackage{parskip}
\usepackage{emptypage}
\usepackage{subcaption}
\usepackage{multicol}
\usepackage[usenames,dvipsnames]{xcolor}

% \usepackage{cmbright}


\usepackage[margin=1in]{geometry}
\usepackage{amsmath, amsfonts, mathtools, amsthm, amssymb}
\usepackage{mathrsfs}
\usepackage{cancel}
\usepackage{bm}
\newcommand\N{\ensuremath{\mathbb{N}}}
\newcommand\R{\ensuremath{\mathbb{R}}}
\newcommand\Z{\ensuremath{\mathbb{Z}}}
\renewcommand\O{\ensuremath{\emptyset}}
\newcommand\Q{\ensuremath{\mathbb{Q}}}
\newcommand\C{\ensuremath{\mathbb{C}}}
\DeclareMathOperator{\sgn}{sgn}
\usepackage{systeme}
\let\svlim\lim\def\lim{\svlim\limits}
\let\implies\Rightarrow
\let\impliedby\Leftarrow
\let\iff\Leftrightarrow
\let\epsilon\varepsilon
\usepackage{stmaryrd} % for \lightning
\newcommand\contra{\scalebox{1.1}{$\lightning$}}
% \let\phi\varphi
\renewcommand\qedsymbol{$\blacksquare$}




% correct
\definecolor{correct}{HTML}{009900}
\newcommand\correct[2]{\ensuremath{\:}{\color{red}{#1}}\ensuremath{\to }{\color{correct}{#2}}\ensuremath{\:}}
\newcommand\green[1]{{\color{correct}{#1}}}



% horizontal rule
\newcommand\hr{
    \noindent\rule[0.5ex]{\linewidth}{0.5pt}
}


% hide parts
\newcommand\hide[1]{}



% si unitx
\usepackage{siunitx}
\sisetup{locale = FR}
% \renewcommand\vec[1]{\mathbf{#1}}
\newcommand\mat[1]{\mathbf{#1}}


% tikz
\usepackage{tikz}
\usepackage{tikz-cd}
\usetikzlibrary{intersections, angles, quotes, calc, positioning}
\usetikzlibrary{arrows.meta}
\usepackage{pgfplots}
\pgfplotsset{compat=1.13}


\tikzset{
    force/.style={thick, {Circle[length=2pt]}-stealth, shorten <=-1pt}
}

% theorems
\makeatother
\usepackage{thmtools}
\usepackage[framemethod=TikZ]{mdframed}
\mdfsetup{skipabove=1em,skipbelow=0em}


\theoremstyle{definition}

\declaretheoremstyle[
    headfont=\bfseries\sffamily\color{ForestGreen!70!black}, bodyfont=\normalfont,
    mdframed={
        linewidth=2pt,
        rightline=false, topline=false, bottomline=false,
        linecolor=ForestGreen, backgroundcolor=ForestGreen!5,
    }
]{thmgreenbox}

\declaretheoremstyle[
    headfont=\bfseries\sffamily\color{NavyBlue!70!black}, bodyfont=\normalfont,
    mdframed={
        linewidth=2pt,
        rightline=false, topline=false, bottomline=false,
        linecolor=NavyBlue, backgroundcolor=NavyBlue!5,
    }
]{thmbluebox}

\declaretheoremstyle[
    headfont=\bfseries\sffamily\color{NavyBlue!70!black}, bodyfont=\normalfont,
    mdframed={
        linewidth=2pt,
        rightline=false, topline=false, bottomline=false,
        linecolor=NavyBlue
    }
]{thmblueline}

\declaretheoremstyle[
    headfont=\bfseries\sffamily\color{RawSienna!70!black}, bodyfont=\normalfont,
    mdframed={
        linewidth=2pt,
        rightline=false, topline=false, bottomline=false,
        linecolor=RawSienna, backgroundcolor=RawSienna!5,
    }
]{thmredbox}

\declaretheoremstyle[
    headfont=\bfseries\sffamily\color{RawSienna!70!black}, bodyfont=\normalfont,
    numbered=no,
    mdframed={
        linewidth=2pt,
        rightline=false, topline=false, bottomline=false,
        linecolor=RawSienna, backgroundcolor=RawSienna!1,
    },
    qed=\qedsymbol
]{thmproofbox}

\declaretheoremstyle[
    headfont=\bfseries\sffamily\color{NavyBlue!70!black}, bodyfont=\normalfont,
    numbered=no,
    mdframed={
        linewidth=2pt,
        rightline=false, topline=false, bottomline=false,
        linecolor=NavyBlue, backgroundcolor=NavyBlue!1,
    },
]{thmexplanationbox}

\declaretheorem[style=thmgreenbox, numberwithin = section, name=Definition]{definition}
\declaretheorem[style=thmbluebox, name=Example]{eg}
\declaretheorem[style=thmredbox, numberwithin = section, name=Proposition]{prop}
\declaretheorem[style=thmredbox, numberwithin = section, name=Theorem]{theorem}
\declaretheorem[style=thmredbox, numberwithin = section,  name=Lemma]{lemma}
\declaretheorem[style=thmredbox, numberwithin = section,  numbered=no, name=Corollary]{corollary}


\declaretheorem[style=thmproofbox, name=Proof]{replacementproof}
\renewenvironment{proof}[1][\proofname]{\vspace{-10pt}\begin{replacementproof}}{\end{replacementproof}}


\declaretheorem[style=thmexplanationbox, name=Proof]{tmpexplanation}
\newenvironment{explanation}[1][]{\vspace{-10pt}\begin{tmpexplanation}}{\end{tmpexplanation}}


\declaretheorem[style=thmblueline, numbered=no, name=Remark]{remark}
\declaretheorem[style=thmblueline, numbered=no, name=Note]{note}

\newtheorem*{uovt}{UOVT}
\newtheorem*{notation}{Notation}
\newtheorem*{previouslyseen}{As previously seen}
\newtheorem*{problem}{Problem}
\newtheorem*{observe}{Observe}
\newtheorem*{property}{Property}
\newtheorem*{intuition}{Intuition}


\usepackage{etoolbox}
\AtEndEnvironment{vb}{\null\hfill$\diamond$}%
\AtEndEnvironment{intermezzo}{\null\hfill$\diamond$}%
% \AtEndEnvironment{opmerking}{\null\hfill$\diamond$}%

% http://tex.stackexchange.com/questions/22119/how-can-i-change-the-spacing-before-theorems-with-amsthm
\makeatletter
% \def\thm@space@setup{%
%   \thm@preskip=\parskip \thm@postskip=0pt
% }
\newcommand{\oefening}[1]{%
    \def\@oefening{#1}%
    \subsection*{Oefening #1}
}

\newcommand{\suboefening}[1]{%
    \subsubsection*{Oefening \@oefening.#1}
}

\newcommand{\exercise}[1]{%
    \def\@exercise{#1}%
    \subsection*{Exercise #1}
}

\newcommand{\subexercise}[1]{%
    \subsubsection*{Exercise \@exercise.#1}
}


\usepackage{xifthen}

\def\testdateparts#1{\dateparts#1\relax}
\def\dateparts#1 #2 #3 #4 #5\relax{
    \marginpar{\small\textsf{\mbox{#1 #2 #3 #5}}}
}

\def\@lesson{}%
\newcommand{\lesson}[3]{
    \ifthenelse{\isempty{#3}}{%
        \def\@lesson{Lecture #1}%
    }{%
        \def\@lesson{Lecture #1: #3}%
    }%
    \subsection*{\@lesson}
    \testdateparts{#2}
}

% \renewcommand\date[1]{\marginpar{#1}}


% fancy headers
\usepackage{fancyhdr}
\pagestyle{fancy}

\fancyhead[LE,RO]{Lance Remigio}
\fancyhead[RO,LE]{\@lesson}
\fancyhead[RE,LO]{}
\fancyfoot[LE,RO]{\thepage}
\fancyfoot[C]{\leftmark}

\makeatother




% notes
\usepackage{todonotes}
\usepackage{tcolorbox}

\tcbuselibrary{breakable}
\newenvironment{verbetering}{\begin{tcolorbox}[
    arc=0mm,
    colback=white,
    colframe=green!60!black,
    title=Opmerking,
    fonttitle=\sffamily,
    breakable
]}{\end{tcolorbox}}

\newenvironment{noot}[1]{\begin{tcolorbox}[
    arc=0mm,
    colback=white,
    colframe=white!60!black,
    title=#1,
    fonttitle=\sffamily,
    breakable
]}{\end{tcolorbox}}




% figure support
\usepackage{import}
\usepackage{xifthen}
\pdfminorversion=7
\usepackage{pdfpages}
\usepackage{transparent}
\newcommand{\incfig}[1]{%
    \def\svgwidth{\columnwidth}
    \import{./figures/}{#1.pdf_tex}
}

% %http://tex.stackexchange.com/questions/76273/multiple-pdfs-with-page-group-included-in-a-single-page-warning
\pdfsuppresswarningpagegroup=1




\begin{document}

\section{Topics}

\begin{enumerate}
    \item {\hyperref[Types of Discontinuities]{Types of Discontinuities (discontinuity of the first kind, discontinuity of the second kind)}} 
    \item {\hyperref[Limits of Monotone Functions]{Limits of Monotone Functions}}
    \item {\hyperref[Limits Involving Infinity]{Limits Involving Infinity}}
\end{enumerate}

\section{Types of Discontinuities}\label{Types of Discontinuities}

\begin{definition}[Right-hand limit, left-hand limit]
    Let \( f: (a,b) \to (Y, \tilde{d}) \). 
    \begin{enumerate}
        \item[(i)] Let \( a \leq c < b \). We write
            \[  \lim_{ x \to c^{+} } f(x) = L \]
            if any of the following equivalent conditions holds:
            \begin{enumerate}
                \item[(1)] For all sequences \( ({x}_{n}) \) in \( (c,b) \) satisfying \( {x}_{n} \to c  \), we have \( f({x}_{n}) \to L \).
                \item[(2)] \( \forall \epsilon > 0  \), \( \exists \delta > 0  \) such that if \( c < x < c + \delta  \) then \( \tilde{d}(f(x), L) < \epsilon \).
            \end{enumerate}
        \item[(ii)] Let \( a < c \leq b  \). We write
            \[  \lim_{ x \to c^{-} }  f(x) = L \]
            if any of the following equivalent conditions hold.
            \begin{enumerate}
                \item[(1)] For all sequences \( ({x}_{n}) \) in \( (a,c) \) such that \( {x}_{n} \to c  \) we have \( f({x}_{n}) \to L  \).
                \item[(2)] For all \( \epsilon > 0  \), there exists \( \delta > 0  \) such that if \( c - \delta < x < c  \) then \( \tilde{d}(f(x), L) < \epsilon \). 
            \end{enumerate} 
    \end{enumerate}
\end{definition}

\begin{definition}[Classification of Discontinuities]
    Let \( f: (a,b) \to (Y, \tilde{d}) \). Let \( c \in (a,b) \). Suppose that \( f  \) is discontinuous at \( c  \).
    \begin{enumerate}
        \item[(i)] \( f  \) is said to have a discontinuity of the first kind, or a simple discontinuity, at \( c  \) if \( \lim_{ x \to c^{-} } f(x) \) and \( \lim_{ x \to c^{+} } f(x) \) exist. We outline the two possible scenarios below:
            \begin{enumerate}
                \item[(1)] \( \lim_{ x \to c^{-} } f(x) = \lim_{ x \to c^{+} } f(x) \neq f(c) \). This is a removable discontinuity.
                \item[(2)] \( \lim_{ x \to c^{-} } f(x) \neq \lim_{ x \to c^{+} } f(x)  \). This is a Jump discontinuity.
            \end{enumerate}
        \item[(ii)] \( f  \) is said to have a discontinuity of the second kind at \( c  \) if at least one of \( \lim_{ x \to c^{-} } f(x) \) or \( \lim_{ x \to c^{+} } f(x) \) does not exist. 
    \end{enumerate}
\end{definition}

A quick example is \( f: (-1,1) \to \R  \) and \( f(x) = \frac{ 1 }{ x }  \) with \( x \neq 0 \) and \( f(0) = 0 \).


\section{Limits of Monotone Functions}\label{Limits of Monotone Functions}


Recall that  
\begin{enumerate}
    \item[(1)] \( f: (a,b) \to \R  \) is said to be \textbf{increasing} on \( (a,b) \) if 
        \[  {x}_{1} < {x}_{2} \implies f({x}_{1}) \leq f({x}_{2}). \]
    \item[(2)] \( f: (a,b) \to \R  \) is said to be \textbf{decreasing} on \( (a,b) \) if  
        \[  {x}_{1} < {x}_{2} \implies f({x}_{1}) \geq  f({x}_{2}). \]
    \item[(3)] \( f: (a,b) \to \R  \) is said to be \textbf{monotone} on \( (a,b) \) if it is either increasing or decreasing.
\end{enumerate}

\begin{theorem}[One-sided Limits of Monotone Functions]
   Let \( f: (a,b) \to \R  \) be an increasing function. Then at every point \( c \in (a,b) \), the one-sided limits exist, and 
   \begin{enumerate}
       \item[(i)] \( \lim_{ x \to c^{-} } f(x) = \sup_{a < x < c} f(x) \leq f(c) \).
       \item[(ii)] \( \lim_{ x \to c^{+} } f(x) = \inf_{c < x < b} f(x) \geq f(c) \) 
       \item[(iii)] If \( a <  c < d < b \), then
           \[  \lim_{ x \to c^{+} }  f(x) \leq \lim_{ x \to c^{-} } f(x). \]
           (A similar statement holds for decreasing functions)
   \end{enumerate} 
\end{theorem}

\section{Limits Involving Infinity}\label{Limits Involving Infinity}

Let \( f: \R \to \R  \), \( c \in \R  \) and \( L \in \R  \). 

\begin{enumerate}
    \item[(1)] \( \lim_{ x \to + \infty  } f(x) = L \iff \forall \epsilon > 0 \ \exists M > 0 \ \text{such that} \ \forall x > M \ | f(x) - L | < \epsilon \).
    \item[(2)] \( \lim_{ x \to - \infty   }  f(x) = L \iff \forall \epsilon > 0 \ \exists M > 0 \ \forall x > M \ | f(x) - L  | < \epsilon   \).
    \item[(3)] \( \lim_{ x \to \infty  } f(x) = \infty \iff \forall R > 0 \ \exists M > 0 \ \text{such that} \ \forall x > M \ f(x) > R  \).
    \item[(4)] \( \lim_{ x \to - \infty   }  f(x) = \infty \iff \forall R > 0 \ \exists M > 0 \ \text{such that} \ \forall x < - M  \ f(x) > R  \).
    \item[(5)] \( \lim_{ x \to + \infty  } f(x) = - \infty \iff \forall R > 0 \ \exists M > 0 \ \text{such that} \ \forall x > M \ f(x) < - R   \) 
    \item[(6)] \( \lim_{ x \to - \infty  } f(x) = - \infty \iff \forall R > 0 \ \exists M > 0 \ \text{such that} \ \forall x < - M  \ f(x) < - R  \)
    \item[(7)] \( \lim_{ x \to c }  f(x) = + \infty \iff \forall R > 0 \ \exists \delta > 0 \ \text{such that} \ \text{if} \ 0 < |  x - c  | < \delta \ \text{then} \ f(x) > R  \).
    \item[(8)] \( \lim_{ x \to c } f(x) = - \infty \iff \forall R > 0 \ \exists \delta  > 0 \ \text{such that if} \ 0 < | x - c  |  < \delta \ \text{then} \ f(x) < - R  \).
\end{enumerate}


\end{document}

