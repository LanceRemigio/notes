\documentclass[a4paper]{report}
\usepackage{standalone}
\usepackage{import}
\usepackage[utf8]{inputenc}
\usepackage[T1]{fontenc}
% \usepackage{fourier}
\usepackage{textcomp}
\usepackage{hyperref}
\usepackage[english]{babel}
\usepackage{url}
% \usepackage{hyperref}
% \hypersetup{
%     colorlinks,
%     linkcolor={black},
%     citecolor={black},
%     urlcolor={blue!80!black}
% }
\usepackage{graphicx} \usepackage{float}
\usepackage{booktabs}
\usepackage{enumitem}
% \usepackage{parskip}
% \usepackage{parskip}
\usepackage{emptypage}
\usepackage{subcaption}
\usepackage{multicol}
\usepackage[usenames,dvipsnames]{xcolor}
\usepackage{ocgx}
% \usepackage{cmbright}


\usepackage[margin=1in]{geometry}
\usepackage{amsmath, amsfonts, mathtools, amsthm, amssymb}
\usepackage{thmtools}
\usepackage{mathrsfs}
\usepackage{cancel}
\usepackage{bm}
\newcommand\N{\ensuremath{\mathbb{N}}}
\newcommand\R{\ensuremath{\mathbb{R}}}
\newcommand\Z{\ensuremath{\mathbb{Z}}}
\renewcommand\O{\ensuremath{\emptyset}}
\newcommand\Q{\ensuremath{\mathbb{Q}}}
\newcommand\C{\ensuremath{\mathbb{C}}}
\newcommand\F{\ensuremath{\mathbb{F}}}
\DeclareMathOperator{\sgn}{sgn}
\DeclareMathOperator{\diam}{diam}
\DeclareMathOperator{\LO}{LO}
\DeclareMathOperator{\UP}{UP}
\DeclareMathOperator{\card}{card}
\DeclareMathOperator{\Arg}{Arg}
\DeclareMathOperator{\Dom}{Dom}
\DeclareMathOperator{\Log}{Log}
\DeclareMathOperator{\dist}{dist}
% \DeclareMathOperator{\span}{span}
\usepackage{systeme}
\let\svlim\lim\def\lim{\svlim\limits}
\renewcommand\implies\Longrightarrow
\let\impliedby\Longleftarrow
\let\iff\Longleftrightarrow
\let\epsilon\varepsilon
\usepackage{stmaryrd} % for \lightning
\newcommand\contra{\scalebox{1.1}{$\lightning$}}
% \let\phi\varphi
\renewcommand\qedsymbol{$\blacksquare$}

% correct
\definecolor{correct}{HTML}{009900}
\newcommand\correct[2]{\ensuremath{\:}{\color{red}{#1}}\ensuremath{\to }{\color{correct}{#2}}\ensuremath{\:}}
\newcommand\green[1]{{\color{correct}{#1}}}

% horizontal rule
\newcommand\hr{
    \noindent\rule[0.5ex]{\linewidth}{0.5pt}
}

% hide parts
\newcommand\hide[1]{}

% si unitx
\usepackage{siunitx}
\sisetup{locale = FR}
% \renewcommand\vec[1]{\mathbf{#1}}
\newcommand\mat[1]{\mathbf{#1}}

% tikz
\usepackage{tikz}
\usepackage{tikz-cd}
\usetikzlibrary{intersections, angles, quotes, calc, positioning}
\usetikzlibrary{arrows.meta}
\usepackage{pgfplots}
\pgfplotsset{compat=1.13}

\tikzset{
    force/.style={thick, {Circle[length=2pt]}-stealth, shorten <=-1pt}
}

% theorems
\makeatother
\usepackage{thmtools}
\usepackage[framemethod=TikZ]{mdframed}
\mdfsetup{skipabove=1em,skipbelow=1em}

\theoremstyle{definition}

\declaretheoremstyle[
    headfont=\bfseries\sffamily\color{ForestGreen!70!black}, bodyfont=\normalfont,
    mdframed={
        linewidth=1pt,
        rightline=false, topline=false, bottomline=false,
        linecolor=ForestGreen, backgroundcolor=ForestGreen!5,
    }
]{thmgreenbox}

\declaretheoremstyle[
    headfont=\bfseries\sffamily\color{NavyBlue!70!black}, bodyfont=\normalfont,
    mdframed={
        linewidth=1pt,
        rightline=false, topline=false, bottomline=false,
        linecolor=NavyBlue, backgroundcolor=NavyBlue!5,
    }
]{thmbluebox}

\declaretheoremstyle[
    headfont=\bfseries\sffamily\color{NavyBlue!70!black}, bodyfont=\normalfont,
    mdframed={
        linewidth=1pt,
        rightline=false, topline=false, bottomline=false,
        linecolor=NavyBlue
    }
]{thmblueline}

\declaretheoremstyle[
    headfont=\bfseries\sffamily, bodyfont=\normalfont,
    numbered = no,
    mdframed={
        rightline=true, topline=true, bottomline=true,
    }
]{thmbox}

\declaretheoremstyle[
    headfont=\bfseries\sffamily, bodyfont=\normalfont,
    numbered=no,
    % mdframed={
    %     rightline=true, topline=false, bottomline=true,
    % },
    qed=\qedsymbol
]{thmproofbox}

\declaretheoremstyle[
    headfont=\bfseries\sffamily\color{NavyBlue!70!black}, bodyfont=\normalfont,
    numbered=no,
    mdframed={
        rightline=false, topline=false, bottomline=false,
        linecolor=NavyBlue, backgroundcolor=NavyBlue!1,
    },
]{thmexplanationbox}

\declaretheorem[
    style=thmbox, 
    % numberwithin = section,
    numbered = no,
    name=Definition
    ]{definition}

\declaretheorem[
    style=thmbox, 
    name=Example,
    ]{eg}

\declaretheorem[
    style=thmbox, 
    % numberwithin = section,
    name=Proposition]{prop}

\declaretheorem[
    style = thmbox,
    numbered=yes,
    name =Problem
    ]{problem}

\declaretheorem[style=thmbox, name=Theorem]{theorem}
\declaretheorem[style=thmbox, name=Lemma]{lemma}
\declaretheorem[style=thmbox, name=Corollary]{corollary}

\declaretheorem[style=thmproofbox, name=Proof]{replacementproof}

\declaretheorem[style=thmproofbox, 
                name = Solution
                ]{replacementsolution}

\renewenvironment{proof}[1][\proofname]{\vspace{-1pt}\begin{replacementproof}}{\end{replacementproof}}

\newenvironment{solution}
    {
        \vspace{-1pt}\begin{replacementsolution}
    }
    { 
            \end{replacementsolution}
    }

\declaretheorem[style=thmexplanationbox, name=Proof]{tmpexplanation}
\newenvironment{explanation}[1][]{\vspace{-10pt}\begin{tmpexplanation}}{\end{tmpexplanation}}

\declaretheorem[style=thmbox, numbered=no, name=Remark]{remark}
\declaretheorem[style=thmbox, numbered=no, name=Note]{note}

\newtheorem*{uovt}{UOVT}
\newtheorem*{notation}{Notation}
\newtheorem*{previouslyseen}{As previously seen}
% \newtheorem*{problem}{Problem}
\newtheorem*{observe}{Observe}
\newtheorem*{property}{Property}
\newtheorem*{intuition}{Intuition}

\usepackage{etoolbox}
\AtEndEnvironment{vb}{\null\hfill$\diamond$}%
\AtEndEnvironment{intermezzo}{\null\hfill$\diamond$}%
% \AtEndEnvironment{opmerking}{\null\hfill$\diamond$}%

% http://tex.stackexchange.com/questions/22119/how-can-i-change-the-spacing-before-theorems-with-amsthm
\makeatletter
% \def\thm@space@setup{%
%   \thm@preskip=\parskip \thm@postskip=0pt
% }
\newcommand{\oefening}[1]{%
    \def\@oefening{#1}%
    \subsection*{Oefening #1}
}

\newcommand{\suboefening}[1]{%
    \subsubsection*{Oefening \@oefening.#1}
}

\newcommand{\exercise}[1]{%
    \def\@exercise{#1}%
    \subsection*{Exercise #1}
}

\newcommand{\subexercise}[1]{%
    \subsubsection*{Exercise \@exercise.#1}
}


\usepackage{xifthen}

\def\testdateparts#1{\dateparts#1\relax}
\def\dateparts#1 #2 #3 #4 #5\relax{
    \marginpar{\small\textsf{\mbox{#1 #2 #3 #5}}}
}

\def\@lesson{}%
\newcommand{\lesson}[3]{
    \ifthenelse{\isempty{#3}}{%
        \def\@lesson{Lecture #1}%
    }{%
        \def\@lesson{Lecture #1: #3}%
    }%
    \subsection*{\@lesson}
    \testdateparts{#2}
}

% \renewcommand\date[1]{\marginpar{#1}}


% fancy headers
\usepackage{fancyhdr}
\pagestyle{fancy}

\makeatother

% notes
\usepackage{todonotes}
\usepackage{tcolorbox}

\tcbuselibrary{breakable}
\newenvironment{verbetering}{\begin{tcolorbox}[
    arc=0mm,
    colback=white,
    colframe=green!60!black,
    title=Opmerking,
    fonttitle=\sffamily,
    breakable
]}{\end{tcolorbox}}

\newenvironment{noot}[1]{\begin{tcolorbox}[
    arc=0mm,
    colback=white,
    colframe=white!60!black,
    title=#1,
    fonttitle=\sffamily,
    breakable
]}{\end{tcolorbox}}

% figure support
\usepackage{import}
\usepackage{xifthen}
\pdfminorversion=7
\usepackage{pdfpages}
\usepackage{transparent}
\newcommand{\incfig}[1]{%
    \def\svgwidth{\columnwidth}
    \import{./figures/}{#1.pdf_tex}
}

% %http://tex.stackexchange.com/questions/76273/multiple-pdfs-with-page-group-included-in-a-single-page-warning
\pdfsuppresswarningpagegroup=1



\begin{document}

\section{Lecture 10}

\begin{theorem}[\( E  \) is open relative to \( Y  \)]
Let \( (X,d) \) be a metric space and \( E \subseteq Y \subseteq X  \) and \( Y \neq \emptyset \). \( E  \) is open relative to \( Y  \) if and only if there exists an open set \( {G}_{0} \subseteq X   \) such that \( E = G \cap Y  \).
\end{theorem}
\begin{proof}
\( (\Longrightarrow) \) Assume that \( E  \) is open relative to \( Y  \). Our goal is to show that there exists an open set \( G \subseteq X  \) such that \( E = G \cap Y  \). Since \( E  \) is open relative to \( Y  \), every \( a \in E  \), we have that \( a  \) is an interior point of \( E  \); that is, there exists \( {\epsilon}_{0} > 0  \) such that \( {N}_{{\epsilon}_{a}}^{Y}(a) \subseteq E  \). Hence, for all \( a \in E  \), there exists \( {\epsilon}_{a} > 0  \) such that \( {N}_{{\epsilon}_{a}}(a) \cap Y \subseteq  E  \). Let \( G = \bigcup_{ a \in E  }^{  }  {N}_{{\epsilon}_{a}}(a) \). Clearly, \( G  \) is open in \( X  \) since 
\begin{enumerate}
    \item[(1)] For all \( a \in E  \), \( {N}_{{\epsilon}_{a}}(a)  \) is a neighborhood and so it is open in \( X  \).
    \item[(2)] A union of open sets is open in any metric space.
\end{enumerate}
In what follows, we will prove that \( E  = G \cap Y  \). Note that 
\begin{align*}
    G \cap Y &= \Big( \bigcup_{ a \in E  }^{  }  {N}_{{\epsilon}_{a}} (a) \Big) \cap Y = \bigcup_{ a \in E  }^{   }  \Big(  {N}_{\epsilon_a}  \cap Y\Big) \subseteq  \bigcup_{ a \in E  }^{  } E = E.
\end{align*}
Suppose \( b \in E  \). We have \( b \in {N}_{{\epsilon}_{b}}(b)  \) and thus \( b \in G  \). Furthermore, \(  b \in E  \) implies \( b \in Y  \) since \( E \subseteq Y  \). Thus, we see that \( E \subseteq G \cap Y  \).

\( (\Longleftarrow) \) Assume that there exists \( G \subseteq  X  \) such that \( E = G \cap Y  \). We want to show that \( E  \) is open relative to \( Y  \). Our goal is to show that for all \( a \in E  \), there exists \( \epsilon > 0  \) such that \( {N}_{{\epsilon}_{a}}^{Y}(a) \subseteq E  \).

So, let \( a \in E  \) be given. Our goal is to find \( \epsilon > 0  \) such that 
\[  {N}_{\epsilon}(a) \cap Y \subseteq  E. \]
By assumption, we have 
\begin{align*}
    a \in E = G \cap Y &\Longrightarrow a \in G  \\
                       &\Longrightarrow_{G \ \text{is open}} \exists \epsilon > 0 \ \text{such that} \ {N}_{\epsilon}(a) \subseteq G. 
\end{align*}
Hence, we see that 
\[  {N}_{\epsilon}(a) \cap Y \subseteq  G \cap Y = E. \]
\end{proof}

\begin{theorem}[ ]
    \begin{enumerate}
        \item[(i)] If \( E \subseteq \R   \) is bounded above, then \( \sup E \in \overline{E} \).
        \item[(ii)] If \( E \subseteq \R  \) is bounded below, then \( \inf E \in \overline{E} \).
    \end{enumerate}
\end{theorem}

\begin{proof}
    Here we will prove (1) and the proof of (2) is completely analogous.

    Since \( E  \) is bounded above, we have that \( \sup E  \) exists and is a real number by the least upper bound property of \( \R  \). Let \( \alpha = \sup E  \). Our goal is to show that \( \alpha \in \overline{E} \); that is, we want to show that for all \( \epsilon > 0  \),
    \[  {N}_{\epsilon}(\alpha) \cap E \neq \emptyset. \tag{Exercise 11 of HW4} \] 
    Let \( \epsilon  > 0  \) be given. Since \( \alpha = \sup E  \), we know that there exists \( x \in E  \) such that \( \alpha - \epsilon < x  \). Hence, there exists \( x \in E  \) such that 
    \[  \alpha - \epsilon <   x \leq \alpha < \alpha + \epsilon. \]
    Hence, \( \alpha - \epsilon < x < \alpha + \epsilon \); that is, \( x \in {N}_{\epsilon}(\alpha) \). Therefore, \( {N}_{\epsilon}(\alpha) \cap E \neq \emptyset \).
\end{proof}

\begin{definition}[Open Covers]
    Let \( (X,d) \) be a metric space and \( E \subseteq X  \). A collection of sets \( \{ {O}_{\alpha}   \}_{\alpha \in \Lambda} \) is said to be an \textbf{open cover} of \( E  \) if   
    \begin{enumerate}
        \item[(i)] for every \( \alpha \in \Lambda \), \( {O}_{\alpha}  \) is open in \( X \).
        \item[(ii)] \( E \subseteq \bigcup_{ \alpha \in \Lambda }^{  } {O}_{\alpha} \).
    \end{enumerate}
\end{definition}

\begin{eg}
    Consider \( (\R , | \bullet | ) \) and the subset \( E = [0,\infty ) \). The collection \( \{ {E}_{n} \}_{n \in \N } \) defined by for all \( n \in \N  \), we have 
    \[  {E}_{n} = \Big(  \frac{ -1 }{ n } , n  \Big) \]
    is an open cover of \( E  \). The reason is as follows:
    \begin{enumerate}
        \item[(i)] In homework 4, we proved that every open interval is an open set with the standard metric on \( \R  \).
        \item[(ii)] \( E \subseteq \bigcup_{ n=1  }^{ \infty  }  {E}_{n} \). Indeed, let \( a \in E  \). If \( a = 0  \), then \( a  \) belongs \( {E}_{n} \) for all \( n \in \N \). So, we have 
            \[  a \in \bigcup_{ n=1  }^{ \infty  } {E}_{n}. \]
            On the other hand, if \( a \neq 0  \), then \( \frac{ 1 }{ a }  > 0 \). By the Archimedean Property of \( \R  \), there exists an \( n \in \N  \) such that \( \frac{ 1 }{ n }  < \frac{ 1 }{ a }  \). So, \( a < n  \). Clearly, 
            \[  \frac{ -1 }{ n }  < 0 < a <  n. \]
            Hence, \( a \in {E}_{n} = \Big(  \frac{ -1 }{ n } , n  \Big) \). Consequently, 
            \[  a \in \bigcup_{ m=1  }^{ \infty   } {E}_{m}. \]

    \end{enumerate}
\end{eg}

\begin{definition}[Compactness]
    Let \( (X,d) \) be a metric space and \( K \subseteq X  \). We say that \(  K \) is \textbf{compact} if every open cover of \(  K  \) has a finite subcover; that is, if \( \{ {O}_{\alpha} \}_{\alpha \in \Lambda}  \) is an open cover of \( K  \), then there exists \( {\alpha}_{1}, {\alpha}_{2}, \dots, {\alpha}_{n} \), we have
    \[  K \subseteq \bigcup_{ i = 1  }^{ n }  {O}_{{\alpha}_{i}}. \]
\end{definition}

\begin{eg}
    Let \( (X,d) \) be a metric space and \( E \subseteq  X \). If \( E  \) is finite, then \( E  \) is compact. The reason is as follows: 

    Let \( \{ {O}_{\alpha} \}_{\alpha \in \Lambda} \) be any open cover. Our goal is to show that this open cover has a finite subcover; that is, we can choose \( {\alpha}_{1}, \dots, {\alpha}_{n} \) such that 
    \[  E \subseteq \bigcup_{ i=1  }^{ n }  {O}_{{\alpha}_{i}}. \]
    If \( E = \emptyset \), there is nothing to prove. Otherwise, \( E \neq \emptyset \), we can denote the elements of \( E  \) by \( {x}_{1}, \dots, {x}_{n} \). So, 
    \[ E = \{ {x}_{1}, \dots, {x}_{n} \}. \]
    We have
    \begin{align*}
        {x}_{1} \in E \subseteq  \bigcup_{ \alpha \in \Lambda }^{  } {O}_{\alpha} &\Longrightarrow \exists {\alpha}_{1} \in \Lambda \ \text{such that} \ {x}_{1} \in {O}_{{\alpha}_{1}} \\
        {x}_{2} \in E \subseteq  \bigcup_{ \alpha \in \Lambda }^{  }  {O}_{\alpha} &\Longrightarrow \exists {\alpha}_{2} \in \Lambda \ \text{such that} \ {x}_{2} \in {O}_{{\alpha}_{2}}
    \end{align*}
    and so continue this process until you have reached the \( n \)th element of \( E \). Hence, we have
    \[  {x}_{n} \in E \subseteq \bigcup_{ \alpha \in \Lambda }^{  } {O}_{\alpha} \Longrightarrow \exists {\alpha}_{n} \in \Lambda \ \text{such that} \ {x}_{n} \in {O}_{{\alpha}_{n}}. \]
    Thus, we have that
    \[  E \subseteq \bigcup_{ i=1  }^{ n } {O}_{{\alpha}_{i}} \]
    and so \( E  \) is compact.
\end{eg}

\begin{eg}
   Let \( (\R, | \cdot | ) \) and  
   \[  E = \Big\{\frac{ 1 }{ n }  : n \in \N \Big\} \cup \{ 0  \}. \]
   Prove that \( E  \) is compact (In general, if \( {a}_{n} \to a  \) in \( \R  \), then the set \( F = \{ {a}_{n} : n \in \N  \} \cup \{ a \}   \) is compact).
    
   Let \( \{ {O}_{\alpha} \}_{\alpha \in \Lambda} \) be any open cover of \( E  \). Our goal is to show that this open cover has a finite subcover. Notice that \( 0 \in E  \) and \( E \subseteq \bigcup_{ \alpha \in \Lambda }^{  } {O}_{\alpha} \) implies that 
   \[  O \in \bigcup_{  \alpha \in \Lambda }^{  }  {O}_{\alpha} \Longrightarrow \exists {\alpha}_{0} \in \Lambda \ \text{such that} \ O \in {O}_{{\alpha}_{0}}. \]
   So, if \( O \in {O}_{{\alpha}_{0}}  \) and \( {O}_{{\alpha}_{0}}  \) is open, we have 
   \[  \exists \epsilon > 0 \ \text{such that} \ (- \epsilon , \epsilon) \subseteq  {O}_{{\alpha}_{0}}. \]
   By the Archimedean Property of \( \R  \), we see that there exists \( m \in \N  \) such that \( \frac{ 1 }{ m   }  < \epsilon \). So, for all \( n \geq m  \), we have \( \frac{ 1 }{ n }  < \epsilon  \). Hence, for all \( n \geq m  \), we have \( \frac{ 1 }{ n }  \in (- \epsilon , \epsilon ) \subseteq  {O}_{{\alpha}_{0}} \). Note that \( 1 \in E  \) so there exists \( {\alpha}_{1} \in \Lambda \) such that \( 1 \in {O}_{{\alpha}_{1}} \). Similarly, \( \frac{ 1 }{ 2 }  \in E  \) implies that there exists \( {\alpha}_{2} \in \Lambda \) such that \( \frac{ 1 }{ 2 }  \in {O}_{{\alpha}_{2}} \). Continue this process until, we have 
   \[  \frac{ 1 }{ m - 1  } \in E \Longrightarrow \exists {\alpha}_{m} \in \Lambda \ \text{such that} \ \frac{ 1 }{ m - 1  }  \in {O}_{{\alpha}_{m-1}}. \]
   Thus, we conclude that 
   \[  E \subseteq \bigcup_{ n=0  }^{ \infty   }  {O}_{{\alpha}_{n}} \]
   and so \( E  \) is compact.

\end{eg}

\begin{remark}
    If \( X  \) itself is compact, we say that \( (X,d) \) is a compact metric space; that is, if \( \{ {O}_{\alpha} \}_{\alpha \in \Lambda} \) is any collection of open sets such that  
    \[  X = \bigcup_{ \alpha \in \Lambda }^{  }  {O}_{\alpha} \]
    then there exists \( {\alpha}_{1}, \dots, {\alpha}_{n} \in \Lambda \) such that 
    \[  X = \bigcup_{ i=1  }^{ n }  {O}_{{\alpha}_{i}}. \]
\end{remark}

\begin{theorem}[Compactness implies Closed]
Let \( (X,d) \) be a metric space and \( K \subseteq X   \) is compact. Then \(  K  \) is compact.
\end{theorem}
\begin{proof}
It is enough to show that \( K^{c} \) is open; that is, we need to find \( \epsilon > 0 \) such that \( {N}_{\epsilon}(x) \subseteq K^{c} \) for every \( x \in K^{c} \). That is, we want to show that   
\[ \exists \epsilon > 0 \ \text{such that} \ {N}_{\epsilon}(a) \cap K = \emptyset.   \]
We have 
\begin{align*}
    a \in K^{c} &\Longrightarrow a \notin K  \\
                &\Longrightarrow \forall x \in K \ d(x,a) > 0 \\
\end{align*}
For all \( x \in K  \), let \( {\epsilon}_{x} = \frac{ 1 }{ 4 }  d(x,a) \). Clearly, we have 
\[ \forall x \in K \ {N}_{{\epsilon}_{x}} \cap {N}_{{\epsilon}_{x}}(a) = \emptyset.  \]
Note that \( \{ {N}_{{\epsilon}_{x}}(x) \}_{x \in K} \) is an open cover for \( K   \). Since \(  K  \) is compact, there is a finite subcover. That is, there exists \( {x}_{1}, \dots {x}_{n} \in  K  \) such that 
\[  K \subseteq  \bigcup_{ i =1  }^{ n }  {N}_{{\epsilon}_{{x}_{i}}}({x}_{i}) \]
and of course 
\[  {N}_{{\epsilon}_{{x}_{1}}} ({x}_{1}) \cap {N}_{{\epsilon}_{{x}_{1}}}(a) = \emptyset. \]
If we continue this process until the \( n \)th step, we have that 
\[  {N}_{{\epsilon}_{{x}_{n}}}(x) \cap {N}_{{\epsilon}_{{x}_{n}}}(a) = \emptyset. \]
Let \( \epsilon = \min \{ {\epsilon}_{{x}_{1}}, \dots, {\epsilon}_{{x}_{n}}  \}.   \)
Clearly, \( {N}_{\epsilon}(a) \subseteq  {N}_{{\epsilon}_{{x}_{i}}} (a) \forall 1 \leq i \leq n  \). Hence, we have 
\[  {N}_{{\epsilon}_{{x}_{1}}} \cap {N}_{\epsilon}(a) = \emptyset \]
and similarly for the \( n \)th step, we have 
\[  {N}_{{\epsilon}_{{x}_{n}}}(x) \cap {N}_{\epsilon}(a) = \emptyset. \]
Therefore, we have 
\[  {N}_{\epsilon}(a) \cap \Big[ {N}_{{\epsilon}_{{x}_{1}}} \cup \cdots \cup {N}_{{\epsilon}_{{x}_{n}}} \Big] = \emptyset. \]
So, 
\[  {N}_{\epsilon}(a) \cap K = \emptyset. \]

\end{proof}

\section{Lecture 11}


\subsection{Topics}

\begin{itemize}
    \item Metric subspace
    \item {\hyperref[Theorem 2.35]{Theorem 2.35}} 
    \item {\hyperref[Theorem 2.33]{Theorem 2.33}} 
    \item {\hyperref[Theorem 2.37]{Theorem 2.37}} 
    \item {\hyperref[Theorem 2.36]{Theorem 2.36}} 
\end{itemize}

\begin{theorem}[ ]\label{Theorem 2.35}
    Closed subsets of compact sets are compact.
\end{theorem}
\begin{proof}

\end{proof}

\begin{corollary}
    If \( F  \) is closed and \( K  \) is compact, then \( F \cap K  \) is compact.
\end{corollary}
\begin{proof}

\end{proof}


\begin{theorem}[ ]\label{Theorem 2.33}
    Suppose \( K \subseteq Y \subseteq X  \). We have \( E  \) is compact if and only if \( K  \) is compact relative to \( Y  \).
\end{theorem}
\begin{proof}

\end{proof}

\begin{theorem}[ ]\label{Theorem 2.37}
    If \( E  \) is an infinite subset of a compact set \( K  \), then \( E  \) has a limit point in \( K  \).
\end{theorem}
\begin{proof}

\end{proof}


\begin{remark}
    Let \( (X,d) \) be a metric space and \( K \subseteq X . \) The following statements are equivalent:
    \begin{enumerate}
        \item[(i)] \( K  \) is compact.
        \item[(ii)] Every infinite subset of \( K  \) has a limit point in \( K  \).
        \item[(iii)] Every sequence in \( K  \) has a subsequence that converges to a point in \( K  \).
    \end{enumerate}
\end{remark}

\begin{corollary}
If \( {K}_{1} \supseteq {K}_{2} \supseteq \dots  \) is a sequence of nonempty compact sets, then \( \bigcap_{ n=1  }^{ \infty   }  {K }_{n} \) is nonempty.
\end{corollary}
\begin{proof}

\end{proof}



\begin{theorem}[ ]\label{Theorem 2.36}
    Let \( \{ {K}_{\alpha} \}   \) a collection of compact subsets of \( X  \). If the intersection of any finite subcollection is nonempty, then
    \[  \bigcap_{ \alpha \in \Lambda }^{  } {K}_{\alpha} \neq \emptyset. \]
\end{theorem}
\begin{proof}

\end{proof}


\begin{theorem}[Nested Interval Property]
    If \( {I}_{n} = [{a}_{n}, {b}_{n}] \) is a sequence of closed intervals in \( \R  \) such that \( {I}_{1} \supseteq {I}_{2} \supseteq {I}_{3} \supseteq \dots  \), then \( \bigcap_{ n=1  }^{ \infty   }  {I}_{n} \) is nonempty. 
\end{theorem}
\begin{proof}

\end{proof}




\end{document}
