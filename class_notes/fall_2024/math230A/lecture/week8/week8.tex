\documentclass[a4paper]{book}

\usepackage[utf8]{inputenc}
\usepackage[T1]{fontenc}
% \usepackage{fourier}
\usepackage{textcomp}
\usepackage{hyperref}
\usepackage[english]{babel}
\usepackage{url}
% \usepackage{hyperref}
% \hypersetup{
%     colorlinks,
%     linkcolor={black},
%     citecolor={black},
%     urlcolor={blue!80!black}
% }
\usepackage{graphicx} \usepackage{float}
\usepackage{booktabs}
\usepackage{enumitem}
% \usepackage{parskip}
% \usepackage{parskip}
\usepackage{emptypage}
\usepackage{subcaption}
\usepackage{multicol}
\usepackage[usenames,dvipsnames]{xcolor}
\usepackage{ocgx}
% \usepackage{cmbright}


\usepackage[margin=1in]{geometry}
\usepackage{amsmath, amsfonts, mathtools, amsthm, amssymb}
\usepackage{thmtools}
\usepackage{mathrsfs}
\usepackage{cancel}
\usepackage{bm}
\newcommand\N{\ensuremath{\mathbb{N}}}
\newcommand\R{\ensuremath{\mathbb{R}}}
\newcommand\Z{\ensuremath{\mathbb{Z}}}
\renewcommand\O{\ensuremath{\emptyset}}
\newcommand\Q{\ensuremath{\mathbb{Q}}}
\newcommand\C{\ensuremath{\mathbb{C}}}
\newcommand\F{\ensuremath{\mathbb{F}}}
\DeclareMathOperator{\sgn}{sgn}
\DeclareMathOperator{\diam}{diam}
\DeclareMathOperator{\LO}{LO}
\DeclareMathOperator{\UP}{UP}
\DeclareMathOperator{\card}{card}
\DeclareMathOperator{\Arg}{Arg}
\DeclareMathOperator{\Dom}{Dom}
\DeclareMathOperator{\Log}{Log}
\DeclareMathOperator{\dist}{dist}
% \DeclareMathOperator{\span}{span}
\usepackage{systeme}
\let\svlim\lim\def\lim{\svlim\limits}
\renewcommand\implies\Longrightarrow
\let\impliedby\Longleftarrow
\let\iff\Longleftrightarrow
\let\epsilon\varepsilon
\usepackage{stmaryrd} % for \lightning
\newcommand\contra{\scalebox{1.1}{$\lightning$}}
% \let\phi\varphi
\renewcommand\qedsymbol{$\blacksquare$}

% correct
\definecolor{correct}{HTML}{009900}
\newcommand\correct[2]{\ensuremath{\:}{\color{red}{#1}}\ensuremath{\to }{\color{correct}{#2}}\ensuremath{\:}}
\newcommand\green[1]{{\color{correct}{#1}}}

% horizontal rule
\newcommand\hr{
    \noindent\rule[0.5ex]{\linewidth}{0.5pt}
}

% hide parts
\newcommand\hide[1]{}

% si unitx
\usepackage{siunitx}
\sisetup{locale = FR}
% \renewcommand\vec[1]{\mathbf{#1}}
\newcommand\mat[1]{\mathbf{#1}}

% tikz
\usepackage{tikz}
\usepackage{tikz-cd}
\usetikzlibrary{intersections, angles, quotes, calc, positioning}
\usetikzlibrary{arrows.meta}
\usepackage{pgfplots}
\pgfplotsset{compat=1.13}

\tikzset{
    force/.style={thick, {Circle[length=2pt]}-stealth, shorten <=-1pt}
}

% theorems
\makeatother
\usepackage{thmtools}
\usepackage[framemethod=TikZ]{mdframed}
\mdfsetup{skipabove=1em,skipbelow=1em}

\theoremstyle{definition}

\declaretheoremstyle[
    headfont=\bfseries\sffamily\color{ForestGreen!70!black}, bodyfont=\normalfont,
    mdframed={
        linewidth=1pt,
        rightline=false, topline=false, bottomline=false,
        linecolor=ForestGreen, backgroundcolor=ForestGreen!5,
    }
]{thmgreenbox}

\declaretheoremstyle[
    headfont=\bfseries\sffamily\color{NavyBlue!70!black}, bodyfont=\normalfont,
    mdframed={
        linewidth=1pt,
        rightline=false, topline=false, bottomline=false,
        linecolor=NavyBlue, backgroundcolor=NavyBlue!5,
    }
]{thmbluebox}

\declaretheoremstyle[
    headfont=\bfseries\sffamily\color{NavyBlue!70!black}, bodyfont=\normalfont,
    mdframed={
        linewidth=1pt,
        rightline=false, topline=false, bottomline=false,
        linecolor=NavyBlue
    }
]{thmblueline}

\declaretheoremstyle[
    headfont=\bfseries\sffamily, bodyfont=\normalfont,
    numbered = no,
    mdframed={
        rightline=true, topline=true, bottomline=true,
    }
]{thmbox}

\declaretheoremstyle[
    headfont=\bfseries\sffamily, bodyfont=\normalfont,
    numbered=no,
    % mdframed={
    %     rightline=true, topline=false, bottomline=true,
    % },
    qed=\qedsymbol
]{thmproofbox}

\declaretheoremstyle[
    headfont=\bfseries\sffamily\color{NavyBlue!70!black}, bodyfont=\normalfont,
    numbered=no,
    mdframed={
        rightline=false, topline=false, bottomline=false,
        linecolor=NavyBlue, backgroundcolor=NavyBlue!1,
    },
]{thmexplanationbox}

\declaretheorem[
    style=thmbox, 
    % numberwithin = section,
    numbered = no,
    name=Definition
    ]{definition}

\declaretheorem[
    style=thmbox, 
    name=Example,
    ]{eg}

\declaretheorem[
    style=thmbox, 
    % numberwithin = section,
    name=Proposition]{prop}

\declaretheorem[
    style = thmbox,
    numbered=yes,
    name =Problem
    ]{problem}

\declaretheorem[style=thmbox, name=Theorem]{theorem}
\declaretheorem[style=thmbox, name=Lemma]{lemma}
\declaretheorem[style=thmbox, name=Corollary]{corollary}

\declaretheorem[style=thmproofbox, name=Proof]{replacementproof}

\declaretheorem[style=thmproofbox, 
                name = Solution
                ]{replacementsolution}

\renewenvironment{proof}[1][\proofname]{\vspace{-1pt}\begin{replacementproof}}{\end{replacementproof}}

\newenvironment{solution}
    {
        \vspace{-1pt}\begin{replacementsolution}
    }
    { 
            \end{replacementsolution}
    }

\declaretheorem[style=thmexplanationbox, name=Proof]{tmpexplanation}
\newenvironment{explanation}[1][]{\vspace{-10pt}\begin{tmpexplanation}}{\end{tmpexplanation}}

\declaretheorem[style=thmbox, numbered=no, name=Remark]{remark}
\declaretheorem[style=thmbox, numbered=no, name=Note]{note}

\newtheorem*{uovt}{UOVT}
\newtheorem*{notation}{Notation}
\newtheorem*{previouslyseen}{As previously seen}
% \newtheorem*{problem}{Problem}
\newtheorem*{observe}{Observe}
\newtheorem*{property}{Property}
\newtheorem*{intuition}{Intuition}

\usepackage{etoolbox}
\AtEndEnvironment{vb}{\null\hfill$\diamond$}%
\AtEndEnvironment{intermezzo}{\null\hfill$\diamond$}%
% \AtEndEnvironment{opmerking}{\null\hfill$\diamond$}%

% http://tex.stackexchange.com/questions/22119/how-can-i-change-the-spacing-before-theorems-with-amsthm
\makeatletter
% \def\thm@space@setup{%
%   \thm@preskip=\parskip \thm@postskip=0pt
% }
\newcommand{\oefening}[1]{%
    \def\@oefening{#1}%
    \subsection*{Oefening #1}
}

\newcommand{\suboefening}[1]{%
    \subsubsection*{Oefening \@oefening.#1}
}

\newcommand{\exercise}[1]{%
    \def\@exercise{#1}%
    \subsection*{Exercise #1}
}

\newcommand{\subexercise}[1]{%
    \subsubsection*{Exercise \@exercise.#1}
}


\usepackage{xifthen}

\def\testdateparts#1{\dateparts#1\relax}
\def\dateparts#1 #2 #3 #4 #5\relax{
    \marginpar{\small\textsf{\mbox{#1 #2 #3 #5}}}
}

\def\@lesson{}%
\newcommand{\lesson}[3]{
    \ifthenelse{\isempty{#3}}{%
        \def\@lesson{Lecture #1}%
    }{%
        \def\@lesson{Lecture #1: #3}%
    }%
    \subsection*{\@lesson}
    \testdateparts{#2}
}

% \renewcommand\date[1]{\marginpar{#1}}


% fancy headers
\usepackage{fancyhdr}
\pagestyle{fancy}

\makeatother

% notes
\usepackage{todonotes}
\usepackage{tcolorbox}

\tcbuselibrary{breakable}
\newenvironment{verbetering}{\begin{tcolorbox}[
    arc=0mm,
    colback=white,
    colframe=green!60!black,
    title=Opmerking,
    fonttitle=\sffamily,
    breakable
]}{\end{tcolorbox}}

\newenvironment{noot}[1]{\begin{tcolorbox}[
    arc=0mm,
    colback=white,
    colframe=white!60!black,
    title=#1,
    fonttitle=\sffamily,
    breakable
]}{\end{tcolorbox}}

% figure support
\usepackage{import}
\usepackage{xifthen}
\pdfminorversion=7
\usepackage{pdfpages}
\usepackage{transparent}
\newcommand{\incfig}[1]{%
    \def\svgwidth{\columnwidth}
    \import{./figures/}{#1.pdf_tex}
}

% %http://tex.stackexchange.com/questions/76273/multiple-pdfs-with-page-group-included-in-a-single-page-warning
\pdfsuppresswarningpagegroup=1



\begin{document}

\section{Lecture 14}

\subsection{Topics}

\begin{itemize}
    \item Convergence of a sequence
    \item Bounded Sequence
    \item Theorem: \( {x}_{n} \to x \iff \) every neighborhood of \( x  \) contains \( {x}_{n} \) for all but at most finitely many \( n \).
    \item Theorem: A sequence cannot have more than one limit.
    \item Theorem: Convergent sequence \( \implies \) Bounded sequence.
    \item If \( x \in E' \), there is a sequence \( ({x}_{n}) \) of distinct points in \( E \setminus  \{ x  \}  \) such that \( {x}_{n} \to x  \).
\end{itemize}


\begin{definition}[Convergence of Sequence]
  Let \( (X,d) \) be a metric space. Let \( ({x}_{n}) \) be a sequence in \( X  \). We say that \( ({x}_{n}) \) converges to a limit \( x \in X  \) if for every \( \epsilon > 0  \), we can find an \( N \in \N  \) such that if \( n > N  \), \( d({x}_{n}, x) < \epsilon \).
\end{definition}

\begin{definition}[Bounded Sequence]
    Let \( (X,d) \) be a metric space. Let \( ({x}_{n}) \) be a sequence in \( X  \). We say that \( ({x}_{n}) \) is \textbf{bounded} if the set \( \{ {x}_{n} : n \in \N  \}  \) is a bounded set in the metric space \( X  \); that is,
    \begin{center}
        \( ({x}_{n}) \) is bounded if and only if there exists \( q \in X  \), there exists \( r > 0  \) such that \( \{ {x}_{n}: n \in \N  \}  \subseteq  {N}_{r}(q) \).
    \end{center}
\end{definition}


\begin{theorem}[An Equivalent Characterization of Convergence]
   Let \( (X,d) \) be a metric space. We say that \( ({x}_{n}) \to x  \) if and only if for all \( \epsilon > 0  \), \( {N}_{\epsilon}(x)  \) contains \( {x}_{n}  \) for all but at most finitely many \( n \). 
\end{theorem}

\begin{proof}
We have \( ({x}_{n}) \to x  \) if and only if for all \( \epsilon > 0 \), there exists \( N \in \N  \) such that for any \( n > N  \), we have \( d({x}_{n} , x ) < \epsilon \). This holds if and only if for all \( \epsilon > 0  \), there exists an \( N \in \N  \) such that for any \( n > N  \), we have \( {x}_{n} \in {N}_{\epsilon}(x)  \). This holds if and only if for all \( \epsilon > 0  \) there exists an \( N \in \N  \) such that \( {N}_{\epsilon}(x) \) contains \( {x}_{n} \) for all \( n > N  \). This is equivalent to saying that for all \( \epsilon > 0  \), \( {N}_{\epsilon}(x) \) contains \( {x}_{n} \) for all but at most finitely many \( n \). 
\end{proof}

\begin{theorem}[Uniqueness of Limit]
    Let \( (X,d) \) be a metric space. Let \( ({x}_{n}) \) be a sequence in \( X  \). If \( {x}_{n} \to x  \) in \( X  \) and \( {x}_{n} \to \tilde{x} \), then \( x = \tilde{x} \).
\end{theorem}

\begin{proof}
    In order to show that \( \tilde{x} = d  \), it suffices to show that \( d(\tilde{x}, x) = 0  \). That is, it suffices to show that for all \( \epsilon > 0  \), \( d(x, \tilde{x}) < \epsilon \). Let \( \epsilon > 0  \) be given. Since \( {x}_{n} \to x  \), there exists an \( {N}_{1} \in \N  \) such that for any \( n > {N}_{1} \), we have \[ d({x}_{n}, x) < \frac{ \epsilon }{ 2 }.  \] Similarly, \( {x}_{n} \to \tilde{x} \) implies that there exists an \( {N}_{2} \in \N  \) such that for any \( n > {N}_{2} \), we have 
    \[  d({x}_{n}, \tilde{x}) < \frac{ \epsilon }{ 2 }. \]
    Choose \( N = \max \{{N}_{1}, {N}_{2}\} \). So, for any \( n > N  \), we have 
    \[  d(x, \tilde{x}) \leq d(x, {x}_{n}) + d({x}_{n}, \tilde{x}) < \frac{ \epsilon }{ 2 }  + \frac{ \epsilon }{ 2 }  = \epsilon. \]
    Hence, we have \( x = \tilde{x} \).
\end{proof}

\begin{theorem}[Convergent \( \implies \) Bounded]
   Let \( (X,d) \) be a metric space. Let \( ({x}_{n}) \) be a sequence in \( X  \). If \( {x}_{n}  \to x \) in \( X  \), then \( ({x}_{n}) \) is bounded.  
\end{theorem}
\begin{proof}
By the definition of convergence with (\( \epsilon = 1  \)), we have  
\[  \exists N \in \N \ \text{such that} \ \forall n > N \ {x}_{n} \in {N}_{1}(x). \]
Now, let 
\[  r = \max \{  1 , d({x}_{1}, x), d({x}_{2},x), \dots , d({x}_{N}, x) \} + 1. \]
Then, clearly, we have 
\[  \forall n \in \N \ d({x}_{n},x) < r. \]
Hence, 
\[  \forall n \in \N  \ {x}_{n} \in {N}_{r}(x). \]

\end{proof}

\begin{corollary}[Contrapositive]
    If \( ({x}_{n}) \) is NOT bounded, then \( ({x}_{n}) \) does NOT converge. 
\end{corollary}

\begin{theorem}[Limit Point is a Limit of a Sequence]
    Let \( (X,d) \) be a metric space. Let \( E \subseteq  X \). Suppose \( x \in E' \). Then there exists a sequence \( {x}_{1}, {x}_{2}, \dots  \) of distinct points in \( E \setminus  \{ x \}  \) that converges to \( x  \). 
\end{theorem}
\begin{proof}
Since \( x \in E' \), we have 
\[  \forall \epsilon > 0  \ \ {N}_{\epsilon}(x) \cap (E \setminus  \{ x \} ) \ \text{is infinite}. \]
In particular, we have 
\begin{align*}
    &\text{For} \ \epsilon = 1 \ \exists {x}_{1} \in E \setminus  \{ x \} \ \text{such that} \ d({x}_{1}, x) < 1  \\
    &\text{For} \ \epsilon = \frac{ 1 }{ 2 }  \ \exists {x}_{2} \in E \setminus  \{ x \}  \ \text{such that} \ {x}_{2} \neq {x}_{1} \ \text{and} \ d({x}_{2}, x) < \frac{ 1 }{ 2 }  \\
    &\text{For} \ \epsilon = \frac{ 1 }{ 3 }  \ \exists {x}_{3} \in E \setminus  \{ x \}  \ \text{such that} \ {x}_{3} \neq {x}_{1}, {x}_{2} \ \text{and} \ d({x}_{3}, x) < \frac{ 1 }{ 3 }  \\
    &\vdots \\
    &\text{For} \ \epsilon = \frac{ 1 }{ n } \ \exists {x}_{n} \in E \setminus  \{ x \}  \ \text{such that} \ {x}_{n} \neq {x}_{1}, {x}_{2}, \dots, {x}_{n-1} \ \text{and} \ d({x}_{n}, x) < \frac{ 1 }{ n } \\ 
    &\vdots
\end{align*}
In this way, we claim that the sequence \( {x}_{1}, {x}_{2}, {x}_{3}, \dots \) of distinct points in \( E \setminus  \{ x \}  \) converges to \( x  \). 

Indeed, let \( \epsilon > 0   \) be given. We need to find an \( N \in \N  \) such that for any \( n > N  \), \( d({x}_{n}, x) < \epsilon \). Choose \( N  \) such that \( \frac{ 1 }{ N } < \epsilon  \). Then for all \( n > N  \), we have 
\[  d({x}_{n},x) < \frac{ 1 }{ n }  < \frac{ 1 }{ N }  < \epsilon \]
as desired.
\end{proof}

\section{Lecture 15}

\begin{definition}[Subsequence]
    Let \( (X,d) \) be a metric space. Let \( ({x}_{n}) \) be a sequence in \( X  \) and let 
    \[  {n}_{1} < {n}_{2} < {n}_{3} < \cdots    \]
    be a strictly increasing sequence of natural numbers. Then \( ({x}_{{n}_{1}}, {x}_{{n}_{2}}, {x}_{{n}_{3}}, \dots ) \) is called a \textbf{subsequence} of \( ({x}_{1}, {x}_{2}, {x}_{3}, \dots ) \) and is denoted by \( ({x}_{{n}_{k }}) \), where \( k \in \N  \) indexes the subsequence.
\end{definition}

\begin{theorem}[ ]
    Let \( (X,d) \) be a metric space. Let \( ({x}_{n}) \) be a sequence in \( X  \). If \( \lim_{ n \to \infty  } {x}_{n} = x  \), then every subsequence of \( ({x}_{n}) \) converges to \( x  \).
\end{theorem}

\begin{proof}
Let \( ({x}_{{n}_{k }}) \) be a subsequence of \( ({x}_{n}) \). Our goal is to show that \( \lim_{ k  \to  \infty  }  {x}_{{n}_{k }} = x  \). That is, we want to show that 
\[  \forall \epsilon > 0 \ \ \exists N \in \N \ \text{such that} \ \forall k > N \  d({x}_{{n}_{k }}, x) < \epsilon.  \]

Let \( \epsilon > 0  \) be given. Our goal is to find an \( N \in \N  \) such that
\[ \text{if} \ k > N \ \text{then} \ d({x}_{{n}_{k }}, x) < \epsilon. \tag{I} \]
Since \( {x}_{n} \to x  \), we have
\[  \exists \hat{N} \ \text{such that} \ \forall n > \hat{N} \ d({x}_{n}, x) < \epsilon. \tag{II} \]
We claim that this \( \hat{N} \) can be used as the same \( N  \) we were looking for. Indeed, if we let \( N = \hat{N} \), then for any \( k > N  \), we can conclude that \( {n}_{k} \geq k > N  \) and so, by (II) we can write \( d({x}_{{n}_{k}},x) < \epsilon \).
\end{proof}

\begin{corollary}
    \begin{enumerate}
        \item[(1)] If  a subsequence of \( ({x}_{n}) \) does NOT converge to \( x  \), then \( ({x}_{n})  \) does NOT converge to \( x  \).
        \item[(2)] If \( ({x}_{n})  \) has a pair of subsequences converging to different limits, then \( ({x}_{n}) \) does not converge.
    \end{enumerate}
\end{corollary}

\begin{theorem}[ ]
    Let \( (X,d) \) be a metric space. Let \( ({x}_{n}) \) be a sequence in \( X  \). The subsequential limits of \( ({x}_{n}) \) form a closed set in \( X  \).
\end{theorem}

\begin{proof}
    Let \( E = \{ b \in X : b \ \text{is a limit of a subsequence of} \ ({x}_{n}) \}  \). Our goal is to show that \( E' \subseteq E  \). To this end, we pick an arbitrary element \( a \in E' \) and we will prove that \( a \in E  \), that is, we will show that there is a subsequence of \( ({x}_{n}) \) that converges to some \( a \in X  \). We may considering two cases:
    \begin{enumerate}
        \item[(1)] \( \forall n \in \N \), we have \( {x}_{n} = a  \). In this case, \( ({x}_{n}) \) and any subsequence of \( ({x}_{n}) \) converges to \( a  \). So, \( a \in E  \).
        \item[(2)] \( \exists {n}_{1} \in \N  \) such that \( {x}_{{n}_{1}} \neq a  \). In this case, let \( \delta = d(a, {x}_{{n}_{1}}) > 0  \). Since \( a \in E' \), we have \( {N}_{\frac{ \delta }{ 2^{2} } }(a) \cap (E \setminus  \{ a \} ) \neq \emptyset \). So,  
            \[  \exists b \in E \setminus  \{ a \}  \ \text{such that} \ d(b,a) < \frac{ \delta }{ 2^{3} }. \]
            Since \( n \in E  \), \( b  \) is a limit of a subsequence of \( ({x}_{n}) \), so
            \[  \exists {n}_{3} > {n}_{2} \ \text{such that} \ d({x}_{{n}_{3}}, b) < \frac{ \delta }{ 2^{3} }. \]
            By the triangle inequality, we have that 
            \[  d({x}_{{n}_{3}}, a) \leq d({x}_{{n}_{3}}, b) + d(b,a) < \frac{ \delta }{ 2^{3} }  + \frac{ \delta }{ 2^{3} }  = \frac{ \delta }{ 2^{2} }. \]
            In this way, we obtain a subsequence \( {x}_{{n}_{1}}, {x}_{{n}_{2}}, {x}_{{n}_{3}}, \dots  \) of \( ({x}_{n}) \) such that
            \[  \forall k \geq 2 \ \ d({x}_{{n}_{k }} , a) < \frac{ \delta }{ 2^{k-1} }. \]
           From this construction, we can see that \( {x}_{{n}_{k }} \to a  \). Thus, \( a \in E  \). 
    \end{enumerate}
\end{proof}

\begin{theorem}[Compactness implies Sequential Compactness]\label{SeqCompact}
   Let \( (X,d) \) be a compact metric space. Then every sequence in \( X  \) has a convergent subsequence. 
\end{theorem}

\begin{proof}
Let \( ({x}_{n}) \) be a sequence is the compact metric space \( X  \). Let \( E = \{ {x}_{1}, {x}_{2}, \dots  \}  \). If \( E  \) is finite, then there exists \( x \in X  \) and \( {n}_{1} < {n}_{2} < {n}_{3} < \cdots     \) such that 
\[  {x}_{{n}_{1}} = {x}_{{n}_{2}} = {x}_{{n}_{3}} = \cdots = x. \]
Clearly, the subsequence \( ({x}_{{n}_{1}}, {x}_{{n}_{2}}, {x}_{{n}_{3}}, \dots ) \) converges (it converges to \( x  \)).

If \( E  \) is infinite, then since \( X  \) is compact by Theorem 2.3.7, \( E  \) has a limit point \( x \in X  \) (that is, there exists \( x \in X  \) such that \( x \in E' \)). Thus, we have 
\[  \forall \epsilon > 0 \ \ {N}_{\epsilon}(x) \cap (E \setminus  \{ x \} ) \ \text{is infinite.} \]
In particular, 
\begin{align*}
    &\text{For} \ \epsilon = 1, \ \exists {n}_{1} \in \N \ \text{such that} \ d({x}_{{n}_{1}}, x) < 1  \\
    &\text{For} \ \epsilon = \frac{ 1 }{ 2 } , \ \exists {n}_{2} \in \N \ \text{such that} \ d({x}_{{n}_{2}}, x) < \frac{ 1 }{ 2 }   \\
    &\text{For} \ \epsilon = \frac{ 1 }{ 3 } , \ \exists {n}_{3} \in \N \ \text{such that} \ d({x}_{{n}_{3}}, x) < \frac{ 1 }{ 3 }   \\
    &\vdots \\
    &\text{For} \ \epsilon = \frac{ 1 }{ m } , \ \exists {n}_{m} \in \N \ \text{such that} \ d({x}_{{n}_{m}}, x) < \frac{ 1 }{ m }   \\
    &\vdots
\end{align*}
In this way, we obtain a subsequence \( {x}_{{n}_{1}}, {x}_{{n}_{2}}, \dots  \) of \( ({x}_{n}) \) that converges (it converges to \( x \)). 
\end{proof}

\begin{corollary}
    Every bounded sequence in \( \R^{k} \) has a convergent subsequence.
\end{corollary}

\begin{proof}
Let \( ({x}_{n}) \) be a bounded sequence in \( \R^{k} \). Then we have
\[  \exists q \in \R^{k} \ \text{and} \ r > 0 \ \text{such that} \ \{ {x}_{n} : n \in \N  \} \subseteq {N}_{r}(q). \]
Note that \( {N}_{r}(q) \) is a bounded set and so \( \overline{{N}_{r}(q)} \) is closed and bounded. So, \( \overline{{N}_{r}(q)} \) is a compact subset of \( \R^{k} \). So, we have \( \overline{{N}_{r}(q)} \) is a compact metric space and \( ({x}_{n}) \) is a sequence in \( \overline{{N}_{r}(q)} \) implies, by the {\hyperref[SeqCompact]{previous theorem}}, there exists a subsequence \( ({x}_{{n}_{k}}) \) of \( ({x}_{n}) \) that converges in the metric space \( \overline{{N}_{r}(q)} \). Since the distance \( \overline{{N}_{r}(q)} \) is the same as distance function in \( \R^{k } \), we can conclude that \( ({x}_{{n}_{k}}) \) converges in \( \R^{k } \) as well.
\end{proof}

\begin{definition}[Cauchy Sequence]
    Let \( (X,d) \) be a metric space. A sequence \( ({x}_{n}) \) in \( X  \) is said to be a \textbf{Cauchy Sequence} if 
    \[  \forall \epsilon > 0 \ \exists N \in \N \ \text{such that} \ \forall n,m > N  \ d({x}_{n}, {x}_{m}) < \epsilon. \]
\end{definition}

\begin{theorem}[Convergent implies Cauchy]
    Let \( (X,d) \) be a metric space. Let \( ({x}_{n}) \) be a sequence in \( X  \). Then
    \[  ({x}_{n}) \ \text{converges} \ \implies \ ({x}_{n}) \ \text{is Cauchy}.  \]
\end{theorem}

\begin{proof}
Suppose \( ({x}_{n}) \to x  \) for some \( x \in X  \). Our goal is to show that \( ({x}_{n}) \) is a Cauchy sequence. Our goal is to show that  
\[  \forall \epsilon > 0 \ \exists N \in \N \ \text{such that} \ \forall n,m > N \ d({x}_{n}, {x}_{m}) < \epsilon. \]
By assumption, there exists an \( \hat{N} \in \N  \) such that for any \( n > \hat{N}  \), we have
\[ d({x}_{n} ,x) < \frac{ \epsilon }{ 2 }. \tag{1}  \]
Similarly, for any \( m > \hat{N}  \)
\[  d({x}_{m}, x) < \frac{ \epsilon }{ 2 }.  \tag{2} \]
We claim that \( \hat{N} \) can be used as the \( N  \) we were looking for. Set \( N = \hat{N} \) and suppose for any \( n,m > N \), we have 
\begin{align*}
   d({x}_{n}, {x}_{m})  \leq d({x}_{n}, x) + d(x, {x}_{m}) 
                        < \frac{ \epsilon }{ 2 }  + \frac{ \epsilon }{ 2 }  
                        = \epsilon.
\end{align*}
\end{proof}

\begin{remark}
    The converse of the theorem above does not necessarily hold. For example, consider \( \Q  \) as a subspace of \( \R  \). In \( \Q  \), it is not true that every Cauchy sequence is convergent. For example, let \( ({q}_{n}) \) be a sequence in \( \Q  \) such that \( {q}_{n} \to \sqrt{ 2 }  \). That is,  
    \begin{align*}
        {q}_{n} \to \sqrt{ 2 } \ \text{in} \ \R &\implies ({q}_{n}) \ \text{is convergent in} \ \R  \\
                                                &\implies ({q}_{n}) \ \text{is Cauchy in} \ \R \\ 
                                                &\implies ({q}_{n})  \ \text{is Cauchy in} \ \Q.
    \end{align*}
    However, we know that \( ({q}_{n})  \) does not converge in \( \Q  \).
\end{remark}

\begin{definition}[Complete Metric Space]
    A metric space in which every Cauchy sequence is convergent is called a \textbf{complete metric space}.
\end{definition}


\end{document}
