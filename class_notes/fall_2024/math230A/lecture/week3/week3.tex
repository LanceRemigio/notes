\documentclass[a4paper]{report}
\usepackage{standalone}
\usepackage{import}
\usepackage[utf8]{inputenc}
\usepackage[T1]{fontenc}
\usepackage{textcomp}
\usepackage{hyperref}
% \usepackage{fourier}
% \usepackage[dutch]{babel}
\usepackage{url}
% \usepackage{hyperref}
% \hypersetup{
%     colorlinks,
%     linkcolor={black},
%     citecolor={black},
%     urlcolor={blue!80!black}
% }
\usepackage{graphicx}
\usepackage{float}
\usepackage{booktabs}
\usepackage{enumitem}
% \usepackage{parskip}
\usepackage{emptypage}
\usepackage{subcaption}
\usepackage{multicol}
\usepackage[usenames,dvipsnames]{xcolor}

% \usepackage{cmbright}


\usepackage[margin=1in]{geometry}
\usepackage{amsmath, amsfonts, mathtools, amsthm, amssymb}
\usepackage{mathrsfs}
\usepackage{cancel}
\usepackage{bm}
\newcommand\N{\ensuremath{\mathbb{N}}}
\newcommand\R{\ensuremath{\mathbb{R}}}
\newcommand\Z{\ensuremath{\mathbb{Z}}}
\renewcommand\O{\ensuremath{\emptyset}}
\newcommand\Q{\ensuremath{\mathbb{Q}}}
\newcommand\C{\ensuremath{\mathbb{C}}}
\DeclareMathOperator{\sgn}{sgn}
\usepackage{systeme}
\let\svlim\lim\def\lim{\svlim\limits}
\let\implies\Rightarrow
\let\impliedby\Leftarrow
\let\iff\Leftrightarrow
\let\epsilon\varepsilon
\usepackage{stmaryrd} % for \lightning
\newcommand\contra{\scalebox{1.1}{$\lightning$}}
% \let\phi\varphi
\renewcommand\qedsymbol{$\blacksquare$}




% correct
\definecolor{correct}{HTML}{009900}
\newcommand\correct[2]{\ensuremath{\:}{\color{red}{#1}}\ensuremath{\to }{\color{correct}{#2}}\ensuremath{\:}}
\newcommand\green[1]{{\color{correct}{#1}}}



% horizontal rule
\newcommand\hr{
    \noindent\rule[0.5ex]{\linewidth}{0.5pt}
}


% hide parts
\newcommand\hide[1]{}



% si unitx
\usepackage{siunitx}
\sisetup{locale = FR}
% \renewcommand\vec[1]{\mathbf{#1}}
\newcommand\mat[1]{\mathbf{#1}}


% tikz
\usepackage{tikz}
\usepackage{tikz-cd}
\usetikzlibrary{intersections, angles, quotes, calc, positioning}
\usetikzlibrary{arrows.meta}
\usepackage{pgfplots}
\pgfplotsset{compat=1.13}


\tikzset{
    force/.style={thick, {Circle[length=2pt]}-stealth, shorten <=-1pt}
}

% theorems
\makeatother
\usepackage{thmtools}
\usepackage[framemethod=TikZ]{mdframed}
\mdfsetup{skipabove=1em,skipbelow=0em}


\theoremstyle{definition}

\declaretheoremstyle[
    headfont=\bfseries\sffamily\color{ForestGreen!70!black}, bodyfont=\normalfont,
    mdframed={
        linewidth=2pt,
        rightline=false, topline=false, bottomline=false,
        linecolor=ForestGreen, backgroundcolor=ForestGreen!5,
    }
]{thmgreenbox}

\declaretheoremstyle[
    headfont=\bfseries\sffamily\color{NavyBlue!70!black}, bodyfont=\normalfont,
    mdframed={
        linewidth=2pt,
        rightline=false, topline=false, bottomline=false,
        linecolor=NavyBlue, backgroundcolor=NavyBlue!5,
    }
]{thmbluebox}

\declaretheoremstyle[
    headfont=\bfseries\sffamily\color{NavyBlue!70!black}, bodyfont=\normalfont,
    mdframed={
        linewidth=2pt,
        rightline=false, topline=false, bottomline=false,
        linecolor=NavyBlue
    }
]{thmblueline}

\declaretheoremstyle[
    headfont=\bfseries\sffamily\color{RawSienna!70!black}, bodyfont=\normalfont,
    mdframed={
        linewidth=2pt,
        rightline=false, topline=false, bottomline=false,
        linecolor=RawSienna, backgroundcolor=RawSienna!5,
    }
]{thmredbox}

\declaretheoremstyle[
    headfont=\bfseries\sffamily\color{RawSienna!70!black}, bodyfont=\normalfont,
    numbered=no,
    mdframed={
        linewidth=2pt,
        rightline=false, topline=false, bottomline=false,
        linecolor=RawSienna, backgroundcolor=RawSienna!1,
    },
    qed=\qedsymbol
]{thmproofbox}

\declaretheoremstyle[
    headfont=\bfseries\sffamily\color{NavyBlue!70!black}, bodyfont=\normalfont,
    numbered=no,
    mdframed={
        linewidth=2pt,
        rightline=false, topline=false, bottomline=false,
        linecolor=NavyBlue, backgroundcolor=NavyBlue!1,
    },
]{thmexplanationbox}

\declaretheorem[style=thmgreenbox, numberwithin = section, name=Definition]{definition}
\declaretheorem[style=thmbluebox, name=Example]{eg}
\declaretheorem[style=thmredbox, numberwithin = section, name=Proposition]{prop}
\declaretheorem[style=thmredbox, numberwithin = section, name=Theorem]{theorem}
\declaretheorem[style=thmredbox, numberwithin = section,  name=Lemma]{lemma}
\declaretheorem[style=thmredbox, numberwithin = section,  numbered=no, name=Corollary]{corollary}


\declaretheorem[style=thmproofbox, name=Proof]{replacementproof}
\renewenvironment{proof}[1][\proofname]{\vspace{-10pt}\begin{replacementproof}}{\end{replacementproof}}


\declaretheorem[style=thmexplanationbox, name=Proof]{tmpexplanation}
\newenvironment{explanation}[1][]{\vspace{-10pt}\begin{tmpexplanation}}{\end{tmpexplanation}}


\declaretheorem[style=thmblueline, numbered=no, name=Remark]{remark}
\declaretheorem[style=thmblueline, numbered=no, name=Note]{note}

\newtheorem*{uovt}{UOVT}
\newtheorem*{notation}{Notation}
\newtheorem*{previouslyseen}{As previously seen}
\newtheorem*{problem}{Problem}
\newtheorem*{observe}{Observe}
\newtheorem*{property}{Property}
\newtheorem*{intuition}{Intuition}


\usepackage{etoolbox}
\AtEndEnvironment{vb}{\null\hfill$\diamond$}%
\AtEndEnvironment{intermezzo}{\null\hfill$\diamond$}%
% \AtEndEnvironment{opmerking}{\null\hfill$\diamond$}%

% http://tex.stackexchange.com/questions/22119/how-can-i-change-the-spacing-before-theorems-with-amsthm
\makeatletter
% \def\thm@space@setup{%
%   \thm@preskip=\parskip \thm@postskip=0pt
% }
\newcommand{\oefening}[1]{%
    \def\@oefening{#1}%
    \subsection*{Oefening #1}
}

\newcommand{\suboefening}[1]{%
    \subsubsection*{Oefening \@oefening.#1}
}

\newcommand{\exercise}[1]{%
    \def\@exercise{#1}%
    \subsection*{Exercise #1}
}

\newcommand{\subexercise}[1]{%
    \subsubsection*{Exercise \@exercise.#1}
}


\usepackage{xifthen}

\def\testdateparts#1{\dateparts#1\relax}
\def\dateparts#1 #2 #3 #4 #5\relax{
    \marginpar{\small\textsf{\mbox{#1 #2 #3 #5}}}
}

\def\@lesson{}%
\newcommand{\lesson}[3]{
    \ifthenelse{\isempty{#3}}{%
        \def\@lesson{Lecture #1}%
    }{%
        \def\@lesson{Lecture #1: #3}%
    }%
    \subsection*{\@lesson}
    \testdateparts{#2}
}

% \renewcommand\date[1]{\marginpar{#1}}


% fancy headers
\usepackage{fancyhdr}
\pagestyle{fancy}

\fancyhead[LE,RO]{Lance Remigio}
\fancyhead[RO,LE]{\@lesson}
\fancyhead[RE,LO]{}
\fancyfoot[LE,RO]{\thepage}
\fancyfoot[C]{\leftmark}

\makeatother




% notes
\usepackage{todonotes}
\usepackage{tcolorbox}

\tcbuselibrary{breakable}
\newenvironment{verbetering}{\begin{tcolorbox}[
    arc=0mm,
    colback=white,
    colframe=green!60!black,
    title=Opmerking,
    fonttitle=\sffamily,
    breakable
]}{\end{tcolorbox}}

\newenvironment{noot}[1]{\begin{tcolorbox}[
    arc=0mm,
    colback=white,
    colframe=white!60!black,
    title=#1,
    fonttitle=\sffamily,
    breakable
]}{\end{tcolorbox}}




% figure support
\usepackage{import}
\usepackage{xifthen}
\pdfminorversion=7
\usepackage{pdfpages}
\usepackage{transparent}
\newcommand{\incfig}[1]{%
    \def\svgwidth{\columnwidth}
    \import{./figures/}{#1.pdf_tex}
}

% %http://tex.stackexchange.com/questions/76273/multiple-pdfs-with-page-group-included-in-a-single-page-warning
\pdfsuppresswarningpagegroup=1




\begin{document}

\section{Lecture 4}

\subsection{Topics}
    \begin{enumerate}
        \item[(1)] Sequences
        \item[(2)] Infinite subset of a countable set is countable.
    \end{enumerate}
\subsection{Sequences}

\begin{definition}[Sequence]
    We call a \textbf{sequence}, we mean a function \( f  \) on the set \( \N  \).    
\end{definition}

\begin{itemize}
    \item We can let \( {x}_{n} = f(n) \). Then it is customary to denote the sequence \( f  \) by \( ({x}_{n})_{n \geq 1} \) or \( {x}_{1}, {x}_{2}, \dots  \).
    \item Note that \( {x}_{1}, {x}_{2}, \dots  \) need not be distinct.
    \item If for all \( n \in \N  \), \( {x}_{n} \in A  \), then we say \( ({x}_{n})_{n \geq1} \) is a sequence in \( A  \).
    \item Sometimes it is convenient to replace \( \N  \) in the definition above with \( \{ 0,1,2,\dots \}  \) or \( \{ -1,0,1,2, \dots \}  \).
\end{itemize}


\subsection{Infinite subset of a countable set is countable}

\begin{theorem}[ ]\label{Subsets of countable sets is countable}
    Every infinite subset of a countable set is countable.
\end{theorem}

\begin{proof}
Let \( A  \) be a countable set. Let \( E \subseteq A  \) and \( E  \) is infinite. Our goal is to show that \( E  \) is countable. Since \( A  \) is countable, there exists a bijective function \( g: \N \to A  \), so 
\[  A = \{ g(n) : n \in \N \} = \{ {x}_{n} : n \in \N \} \]
with \( {x}_{n} = g(n) \) for all \( n \in \N \).
Now, let us construct the sequence \( {n}_{1}, {n}_{2}, \dots  \) as follows: 
\begin{enumerate}
    \item[(1)] Let \( {n}_{1}  \) be the smallest positive integer such that \( {x}_{{n}_{1}} \in E  \). 
    \item[(2)] Let \( {n}_{2} \) be the smallest positive integer greater than \( {n}_{1}  \) such that \( {x}_{{n}_{2}} \in E  \). 
    \item[\( \vdots \)]
    \item[(k)] Let \( {n}_{k} \) be the smallest integer greater than \( {n}_{k - 1}  \) such that \( {x}_{{n}_{k}} \in E  \).
\end{enumerate}
Observe that the set 
\[  \{ m \in \N :  m > {n}_{k-1} \wedge {x}_{m} \in E  \}   \]
is both nonempty (since \( E  \) is infinite) and bounded below (by the well ordering property of natural numbers). Thus, we know that  
\[  {n}_{k} = \min \{ m \in \N : m > {n}_{k-1} \wedge {x}_{m} \in E  \}. \]
Now, define the function \( f: \N \to E  \) as follows:
\[  f(k) = {x}_{{n}_{k}}. \]
We claim that this is a bijective map. If we can prove this fact, then surely \( E  \) is countable. We need to prove a two things:
\begin{enumerate}
    \item[(1)] \( f \) is injective and
    \item[(2)] \( f  \) is surjective.
\end{enumerate}
Starting with (1), suppose \( {k}_{1} \neq {k}_{2}  \). Then we have \( {n}_{{k}_{1}} \neq {n}_{{k}_{2}} \) and thus, \( {x}_{{n}_{{k}_{1}}} \neq {x}_{{n}_{{k}_{2}}} \). Hence, we see that \( f({k}_{1}) \neq f({k}_{2}) \). Therefore, \( f  \) is injective which proves (1).

With (2), let \( b \in E  \). Since \( E \subseteq  A  \), we must have \( b \in A  \) and thus there exists an \( m \in \N  \) such that \( b = {x}_{m} \). Hence, there exists \( 1 \leq k \leq m  \) such that \( {x}_{{n}_{k}} = {x}_{m} = b \). Thus, \( f(k) = b \) which shows that \( f  \) is surjective and so (2) is satisfied.

Therefore, \( f  \) must be a bijective map and we can now conclude that \( E \) is countable.
\end{proof}

\begin{remark}
   Consider the contrapositive of the theorem above:  
   \begin{center}
       Let \( E  \subseteq  A  \). If \( E  \) is not countable, then \( A  \) is not countable. 
   \end{center}
   When we say that a set is NOT countable, we do not necessarily mean the set is uncountable. However, in the case in that \( E  \) is an infinite set, we can say that \( E  \) being not countable is equivalent to saying that \( E  \) is uncountable. So, only in the case in which \( E  \) is infinite then the contrapositive is 
   \begin{center}
       Let \( E \subseteq A   \) be infinite. If \( E  \) is uncountable, then \( A  \) is uncountable. By the {\hyperref[Subsets of countable sets is countable]{theorem}} we just proved, we see that \( f(A) \subseteq S   \) and \( S  \) being countable implies that \( f(A) \) must be at most countable. Thus, \( A  \) must be at most countable.
   \end{center}
   
\end{remark}

\begin{corollary}[ ]\label{Injective mapping implies at most countability}
   Let \( A  \) be any set and let \( S  \) be a countable set. If there exists an injective mapping \( f: A \to S  \), then \( A  \) is at most countable. 
\end{corollary}
\begin{proof}
Let \( A  \) be any set and let \( S  \) be a countable set. Suppose there exists an injective mapping \( f: A \to S  \). We can restrict the codomain \( S  \) of \( f  \) to its range \( f(A) \). Because \( f  \) is injective, the mapping \( f: A \to f(A) \) will be bijective. Thus, we have that \( A \sim f(A) \).
\end{proof}

\begin{eg}[\( \N \times \N  \) is countable]
   \begin{enumerate}
       \item[(1)] The function \( f: \N \times \N \to \N  \) defined by
           \[  f(x,y) = 2^{x} 3^{y} \]
           is injective. By the corollary we have just proved, we see taht \( \N \times \N  \) is at most countable. 
        \item[(2)] Notice that \( g: \N \to \{ 1 \}  \times \N  \) is a bijection \( g(a) = (1,a) \). Hence, \( \{ 1 \}  \times \N  \) is countable. Now, we have that  
            \begin{enumerate}
                \item[(i)] \( \{ 1 \}  \times \N \subseteq  \N \times \N   \)  
                \item[(ii)] \( \{ 1 \} \times \N  \) is countable implies that \( \{ 1 \}  \times \N  \) is infinite.
            \end{enumerate}
            Thus, (i) and (ii) imply that \( \N \times \N  \) is countable.
   \end{enumerate} 
\end{eg}

Proving that a set \( A  \) is countable can be done by finding a subset \( B  \) that is countable which forces \( A  \) to be countable. 

\begin{eg}[\( \Q  \) is countable]
   \begin{enumerate}
       \item[(1)] The function \( f: \Q \to \N \) defined by 
           \[  f(x) = 
           \begin{cases}
               2(2^{p} 3^{q}) &\text{if} \ x = \frac{ p }{ q } \ p \in \N, q \in \N, (p,q) = 1 \\
               2(2^{p} 3^{q}) + 1 &\text{if} \ x = \frac{ -p }{ q } \ p \in \N, q \in \N, (p,q) = 1 \\
               1 &\text{if} \ x = 0
           \end{cases} \]
           is an injective mapping. So, \( \Q  \) is at most countable.
        \item[(2)] Observe that \( \N \subseteq \Q   \) and \( \N  \) being a countable (and thus \( \N  \) is infinite) set implies that \( \Q  \) is infinite. Thus, \( \Q  \) is is countable.
   \end{enumerate} 
\end{eg}

\subsection{Countable union of at most countable sets is at most countable}

\begin{theorem}[ ]\label{Countable union of at most countable sets is at most countable}
   Countable union of at most countable sets is at most countable.
\end{theorem}
\begin{proof}
    Let \( \{ {A}_{n} : n \in \N  \} \) be a countable family of at most countable sets; that is, for each \( n \in \N  \), \( {A}_{n} \) is at most countable. Our goal is to show that 
    \[  K = \bigcup_{ n \in \N  }^{  }  {A}_{n} \ \text{is at most countable.} \]
    To this end, it suffices to show that there exists an injective map \( f: K \to \N \times \N  \). Let 
    \begin{align*}
        {B}_{1} &= {A}_{1} \\
        {B}_{2} &= {A}_{2} \setminus  {A}_{1} \\
        {B}_{3} &= {A}_{3} \setminus  ({A}_{1} \cup {A}_{2}) \\
                &\vdots \\
        {B}_{n+1} &= {A}_{n+1} \setminus  \Big(  \bigcup_{ k=1  }^{ n }  {A}_{k} \Big).
    \end{align*}
    We leave as an exercise to show that 
        \[  \bigcup_{ n \in \N  }^{   }  {B}_{n} = \bigcup_{ n \in \N  }^{  }  {A}_{n} \ \text{are pairwise disjoint}. \]
       Note that for all \( n \in \N  \), we have \( {B}_{n} \subseteq  {A}_{n} \) and \( {A}_{n}  \) is at most countable. Then for each \( n \in \N  \). \( {B}_{n} \) must be at most countable. So, for each \( n \in \N  \), there exists an injective mapping \( {f}_{n}: {B}_{n} \to \N \). Our goal is to show that  
       \[  \bigcup_{ n =1  }^{ \infty  } {B}_{n} \ \text{is at most countable}. \]
       To this end, we define the function 
       \[  f: \bigcup_{ n=1  }^{ \infty  }  {B}_{n} \to \N \times \N \]
       as follows:
       \begin{center}
           For each \( x \in \bigcup_{ n = 1  }^{ \infty  } {B}_{n} \), there is exactly one \( n \in \N  \) such that \( x \in {B}_{n} \); let us denote this \( n  \) by \( {n}_{x} \).  
       \end{center}
       Thus, we define 
       \[  f(x) = ({n}_{x}, {f}_{{n}_{x}}(x)). \]
       This function is injective because
       \begin{align*}
           f(x) = f(y) &\Longrightarrow ({n}_{x}, {f}_{{n}_{x}}(x)) = ({n}_{y}, {f}_{{n}_{y}}(y)) \\
                       &\Longrightarrow {n}_{x} = {n}_{y} \ \wedge \ {f}_{{n}_{x}}(x) = {f}_{{n}_{y}}(y) \\
                       &\Longrightarrow {f}_{{n}_{k}}(x) \wedge {f}_{{n}_{x}}(y) \\
                       &\Longrightarrow x =y \tag{Since \( {f}_{{n}_{x}}  \) is injective}
       \end{align*}
       By the {\hyperref[Injective mapping implies at most countability]{corollary}}, we see that the set 
       \[  \bigcup_{ n =1  }^{  \infty  }  {B}_{n} \] is at most countable.
\end{proof}

\begin{corollary}
A countable union of countable sets is countable.    
\end{corollary}
\begin{proof}
Let \( \{ {A}_{n} : n \in \N  \}  \) be a countable collection of countable sets. By the previous {\hyperref[Countable union of at most countable sets is at most countable]{Theorem}}, we see that \( \bigcup_{ n \in \N  }^{   }  {A}_{n} \) is at most countable. Note that \( {A}_{1} \subseteq \bigcup_{ n \in \N  }^{   } {A}_{n} \) and that \( {A}_{1} \) is countable. Thus, \( {A}_{1} \) must be infinite and so 
\[  \bigcup_{ n \in \N  }^{} {A}_{n}  \ \text{is infinite}. \]
Thus, we see that 
\[  \bigcup_{ n \in \N  }^{  } {A}_{n}  \]
must be countable.
\end{proof}

\begin{corollary}
   If \( A  \) and \( B  \) are at most countable, then \( A \cup B  \) at most countable. 
\end{corollary}
\begin{proof}
Let \( {A}_{1} = A  \) and \( {A}_{2} = B  \) where \( {A}_{2} = {A}_{3} = \cdots = {A}_{n} = \emptyset \). Then \( A \cup B  \) is at most countable by the previous {\hyperref[Countable union of at most countable sets is at most countable]{Theorem}}. 
\end{proof}

\begin{theorem}[ ]
    If \( A  \) is countable, then \( A \times A  \) is countable.
\end{theorem}
\begin{proof}
    \begin{enumerate}
        \item[(1)] Note that \( A \times A  = \bigcup_{ b \in A  }^{  }  \{ b  \}  \times A  \) for each \( b \in A  \).
        \item[(2)] The function \( f: A \to \{ b \}  \times A  \) defined by \( f(x) = (b,x) \) is bijective. So, \( A \sim \{ b \}  \times A  \). Hence, \( \{ b  \}  \times A  \) is countable.
    \end{enumerate}
    Since a countable union of countable sets is countable, we see that \( A \times A  \) must be countable by (1) and (2).
\end{proof}




\section{Lecture 5}

\subsection{Topics}
\begin{itemize}
    \item Summary of last lecture
    \item Collection of all binary sequences is uncountable.
    \item Preliminary Remarks
    \item Inequalities
\end{itemize}

\subsection{Summary of Last Lecture}


\begin{enumerate}
    \item How to prove \( A  \) is at most countable:
        \begin{itemize}
            \item Show that \( A \subseteq S   \) where \( S  \) is countable.
            \item Find an injective function \( f: A \to S  \) where \( S  \) is countable.  
        \end{itemize}
    \item How to prove \( A  \) is infinite:
        \begin{itemize}
            \item Prove that \( A  \) has a countable subset.
            \item Find an injective function \( f: \N \to A  \).
            \item Find an injective function \( f: A \to A  \) that is not onto.
            \item Find a proper subset \( B \subseteq  A  \) such that \( A \sim B  \).
        \end{itemize}
    \item Find a proper subset \( B \subseteq  A   \) such that \( A \sim B  \).
        \begin{itemize}
            \item find a bijective function \( f: A \to B  \).
            \item (Shroder-Bernstein) 
                \begin{itemize}
                    \item Find an injective function \( g: A \to B  \)
                    \item Find an injective function \( h: B \to A  \).
                \end{itemize}
        \end{itemize}
\item A countable union of countable sets is countable \textbf{AND}
\item A finite product of countable sets is countable. 
\end{enumerate}

\begin{theorem}[D]
Let \( A  \) be the set of all sequences whose terms are the digits \( 0  \) and \( 1  \) (that is \( A  \) is the collection of all binary sequences). This set \( A  \) is uncountable. 
\end{theorem}
\begin{proof}
First, notice that \( A  \) is infinite. Let \( h: \N \to A   \) be the function defined by 
\[  \text{for all} \ h(n) = \text{The binary sequence whose \( n \)th term is the digit 1 and all other terms are zero}. \]
Clearly, \( h  \) is an injective map. Hence, \( h: \N \to h(\N) \) is bijective. We have \( \N \sim h(\N) \), and so \( h(\N)  \) is infinite. Note that \( h(\N) \subseteq A  \). Thus, \( A  \) is infinite 

Suppose for sake of contradiction that \( A  \) is NOT uncountable. Since \( A  \) is infinite and not uncountable, this assumption tells us that \( A  \) must be countable. Thus, there exists a bijective map \( f: \N \to A  \). So, we can write
\[  A = \{ f(n) : n \in \N  \}.  \]
This means that for each \( n \in \N  \), \( f(n) \) is a binary sequence. Let 
\begin{align*}
    f(1) &= ({a}_{1}^{1}, {a}_{2}^{1}, \dots) \\
    f(2) &= ({a}_{1}^{2}, {a}_{2}^{2} , \dots ) \\
    f(3) &= ({a}_{1}^{3}, {a}_{2}^{3} , \dots ) \\
         &\vdots \\
    f(n) &= ({a}_{1}^{n}, {a}_{2}^{n}, \dots) \\
        &\vdots
\end{align*}
The goal is to construct a binary sequence \( ({b}_{1}, {b}_{2}, \dots ) \) that is not in this list! This will contradict the fact that \( A  \) contains all the binary sequences.  

Define 
\[  {b}_{1} = 
\begin{cases}
    0 &\text{if} \ {a}_{1}^{1} = 1  \\
    1 &\text{if} \ {a}_{1}^{1} = 0 
\end{cases}  \]
Similarly, for the second term, we see that 
\[  {b}_{2} = 
\begin{cases}
    0 &\text{if} \ {a}_{2}^{2} = 1  \\
    1 &\text{if} \ {a}_{2}^{2} = 0 
\end{cases}  \]
and so on. More generally, 
\[  \text{for all} \ i \in \N \ {b}_{i} = 
\begin{cases}
    0 &\text{if} \ {a}_{i}^{i} = 1 \\
    1 &\text{if} \ {a}_{i}^{i} = 0 
\end{cases}. \]
Clearly, this sequence \( ({b}_{n}) \) is not the same as any of the sequences of the list above; that is, for all \( i \in \N  \), \( {b}_{i} \neq f(i)  \).
\end{proof} 

\subsection{Preliminary Remarks}

\begin{itemize}
    \item \( \R  \) is NOT just an ordered field, it has more extra structures.
    \item In \( \R  \), there is a standard notion of length (size, norm) and a corresponding notion of distance.
    \item Given \( a \in \R  \), the \textbf{size} of \( a  \) is defined to be \( | a  |  \). But note that the \textbf{absolute value of \( a \)} is defined as follows:
        \[  | a  |  = 
        \begin{cases}
            a &\text{if} \ a \geq 0 \\
            -a &\text{if} \ a \leq 0.
        \end{cases} \]
    \item Given two numbers \( a  \) and \( b  \) in \( \R  \), the \textbf{standard distance} between \( a  \) and \( b  \) is 
        \[  \dist{(a,b)} = | a - b  |.  \]
    \item The notion of distance plays an essential role in the development of calculus.
    \item When there is a notion of distance, then it will be possible to make sense of statements such as \textit{as \( h \) gets close to zero...}. 
    \item Note that size, length, or norm are all equivalent in describing the size of the difference between two numbers/elements.
\end{itemize}

\begin{theorem}[Basic Properties of Standard Size in \( \R  \)]
    Let \( a \in \R  \). Then
    \begin{enumerate}
        \item[(i)] \( |  a  |  \geq 0  \)  
        \item[(ii)] \( |  a  |  = 0  \) if and only if \( a = 0  \). 
        \item [(iii)] \( |  \alpha a  |  = | \alpha  |  | a |  \) for all \( \alpha \in \R  \) and \( a \in \R  \).
        \item[(iv)] \( |  a + b  |  \leq | a  |  + | b  |  \) for all \( a,b \in \R  \).
    \end{enumerate}
\end{theorem}
\begin{remark}
    Note that (iv) is equivalent to saying that 
    \[  -(| a | + | b | ) \leq a + b \leq | a  |  + | b  |.  \]
    This is an immediate consequence of the following inequalities:
    \begin{enumerate}
        \item[(i)] Every \( a \in \R  \), we have \( - | a  |  \leq a \leq | a  |  \) and \( - | b  |  \leq b \leq |  b |  \).
    \end{enumerate}
\end{remark}

Is it possible to generalize the notion of "size"? \textbf{YES!}

\begin{definition}[ ]
    Let \( V  \) be a (real) vector space. A function from \( \| \cdot \|:   V \to \R  \) is called a \textbf{norm} on \( V  \) if it satisfies the following properties:
    \begin{enumerate}
        \item[(i)] For all \( x \in V  \), \( \|x \| \geq 0  \) 
        \item[(ii)] For all \( x \in V  \), \( \|x \| = 0  \) if and only if \( x = 0  \).
        \item[(iii)] For all \( \alpha \in \R  \) and \( x \in V  \), \( \| \alpha x \| = | \alpha |  \|x\| \).
        \item[(iv)] For all \( x,y \in V  \), we have
            \[  \|x + y \| \leq \|x\| + \|y\| \ \tag{triangle inequality} \]
\end{enumerate}
A vector space \( V  \) equipped with notion of norm is called a normed space. (Sometimes, we write \( (V , \|\cdot\|) \) is a normed space). So, call \( d(x,y) = \| x - y \| \).
\end{definition}

\begin{theorem}[Basic Properties of Standard Distance in \( \R  \)]
    For all \( a,b \in \R  \). 
    \begin{enumerate}
        \item[(i)] \( \dist(a,b) \geq 0  \) 
        \item[(ii)] \( \dist(a,b) = 0  \) if and only if \( a = b  \).
        \item[(iii)] \( \dist(a,b) = \dist(b,a)  \).
        \item[(iv)] For all \( a,b,c \in \R   \), we have  \( \dist(a,c) \leq \dist(a,b) + \dist(b,c) \).
    \end{enumerate}
\end{theorem}

\subsection{Metric Spaces}

\begin{definition}[Metric Spaces]
Let \(  X \neq \emptyset  \). A function \( d: X \times X \to \R  \) is said to be a \textbf{distance function} or a \textbf{metric} if it satisfies the following properties:
\begin{enumerate}
    \item[(i)] For all \( x,y \in X  \), \( d(x,y) \geq 0  \).
    \item[(ii)] For all \( x,y \in X  \), \( d(x,y) = 0  \) if and only if \( x = y  \).
    \item[(iii)] For all \( x,y \in X  \), \( d(x,y) = d(y,x) \).
    \item[(iv)] For all \( x,y,z \in X  \), \( d(x,y) \leq d(x,z) + d(z,y) \).
\end{enumerate}
A set \( X  \) equipped with a metric \( d  \) is called a \textbf{metric space}. (Sometimes we write \( (X,d)  \) is a metric space)
\end{definition}
\begin{remark}
    \( X  \) does \textbf{NOT} need to be a vector space. Furthermore, \( d  \) is a more general notion of size than \( \| \cdot \| \).
\end{remark}


\begin{eg}
    \( (\R , d)  \) where \( d: \R \times \R \to [0,\infty)   \) is defined by 
    \[  d(x,y) = |  x - y  |. \]
    (or \( d(x,y) = \alpha |  x - y  |  \) where \( \alpha > 0  \) is a fixed real number)


\end{eg}






\end{document}
