\documentclass[a4paper]{report}
\usepackage{standalone}
\usepackage{import}
\usepackage[utf8]{inputenc}
\usepackage[T1]{fontenc}
% \usepackage{fourier}
\usepackage{textcomp}
\usepackage{hyperref}
\usepackage[english]{babel}
\usepackage{url}
% \usepackage{hyperref}
% \hypersetup{
%     colorlinks,
%     linkcolor={black},
%     citecolor={black},
%     urlcolor={blue!80!black}
% }
\usepackage{graphicx} \usepackage{float}
\usepackage{booktabs}
\usepackage{enumitem}
% \usepackage{parskip}
% \usepackage{parskip}
\usepackage{emptypage}
\usepackage{subcaption}
\usepackage{multicol}
\usepackage[usenames,dvipsnames]{xcolor}
\usepackage{ocgx}
% \usepackage{cmbright}


\usepackage[margin=1in]{geometry}
\usepackage{amsmath, amsfonts, mathtools, amsthm, amssymb}
\usepackage{thmtools}
\usepackage{mathrsfs}
\usepackage{cancel}
\usepackage{bm}
\newcommand\N{\ensuremath{\mathbb{N}}}
\newcommand\R{\ensuremath{\mathbb{R}}}
\newcommand\Z{\ensuremath{\mathbb{Z}}}
\renewcommand\O{\ensuremath{\emptyset}}
\newcommand\Q{\ensuremath{\mathbb{Q}}}
\newcommand\C{\ensuremath{\mathbb{C}}}
\newcommand\F{\ensuremath{\mathbb{F}}}
\DeclareMathOperator{\sgn}{sgn}
\DeclareMathOperator{\diam}{diam}
\DeclareMathOperator{\LO}{LO}
\DeclareMathOperator{\UP}{UP}
\DeclareMathOperator{\card}{card}
\DeclareMathOperator{\Arg}{Arg}
\DeclareMathOperator{\Dom}{Dom}
\DeclareMathOperator{\Log}{Log}
\DeclareMathOperator{\dist}{dist}
% \DeclareMathOperator{\span}{span}
\usepackage{systeme}
\let\svlim\lim\def\lim{\svlim\limits}
\renewcommand\implies\Longrightarrow
\let\impliedby\Longleftarrow
\let\iff\Longleftrightarrow
\let\epsilon\varepsilon
\usepackage{stmaryrd} % for \lightning
\newcommand\contra{\scalebox{1.1}{$\lightning$}}
% \let\phi\varphi
\renewcommand\qedsymbol{$\blacksquare$}

% correct
\definecolor{correct}{HTML}{009900}
\newcommand\correct[2]{\ensuremath{\:}{\color{red}{#1}}\ensuremath{\to }{\color{correct}{#2}}\ensuremath{\:}}
\newcommand\green[1]{{\color{correct}{#1}}}

% horizontal rule
\newcommand\hr{
    \noindent\rule[0.5ex]{\linewidth}{0.5pt}
}

% hide parts
\newcommand\hide[1]{}

% si unitx
\usepackage{siunitx}
\sisetup{locale = FR}
% \renewcommand\vec[1]{\mathbf{#1}}
\newcommand\mat[1]{\mathbf{#1}}

% tikz
\usepackage{tikz}
\usepackage{tikz-cd}
\usetikzlibrary{intersections, angles, quotes, calc, positioning}
\usetikzlibrary{arrows.meta}
\usepackage{pgfplots}
\pgfplotsset{compat=1.13}

\tikzset{
    force/.style={thick, {Circle[length=2pt]}-stealth, shorten <=-1pt}
}

% theorems
\makeatother
\usepackage{thmtools}
\usepackage[framemethod=TikZ]{mdframed}
\mdfsetup{skipabove=1em,skipbelow=1em}

\theoremstyle{definition}

\declaretheoremstyle[
    headfont=\bfseries\sffamily\color{ForestGreen!70!black}, bodyfont=\normalfont,
    mdframed={
        linewidth=1pt,
        rightline=false, topline=false, bottomline=false,
        linecolor=ForestGreen, backgroundcolor=ForestGreen!5,
    }
]{thmgreenbox}

\declaretheoremstyle[
    headfont=\bfseries\sffamily\color{NavyBlue!70!black}, bodyfont=\normalfont,
    mdframed={
        linewidth=1pt,
        rightline=false, topline=false, bottomline=false,
        linecolor=NavyBlue, backgroundcolor=NavyBlue!5,
    }
]{thmbluebox}

\declaretheoremstyle[
    headfont=\bfseries\sffamily\color{NavyBlue!70!black}, bodyfont=\normalfont,
    mdframed={
        linewidth=1pt,
        rightline=false, topline=false, bottomline=false,
        linecolor=NavyBlue
    }
]{thmblueline}

\declaretheoremstyle[
    headfont=\bfseries\sffamily, bodyfont=\normalfont,
    numbered = no,
    mdframed={
        rightline=true, topline=true, bottomline=true,
    }
]{thmbox}

\declaretheoremstyle[
    headfont=\bfseries\sffamily, bodyfont=\normalfont,
    numbered=no,
    % mdframed={
    %     rightline=true, topline=false, bottomline=true,
    % },
    qed=\qedsymbol
]{thmproofbox}

\declaretheoremstyle[
    headfont=\bfseries\sffamily\color{NavyBlue!70!black}, bodyfont=\normalfont,
    numbered=no,
    mdframed={
        rightline=false, topline=false, bottomline=false,
        linecolor=NavyBlue, backgroundcolor=NavyBlue!1,
    },
]{thmexplanationbox}

\declaretheorem[
    style=thmbox, 
    % numberwithin = section,
    numbered = no,
    name=Definition
    ]{definition}

\declaretheorem[
    style=thmbox, 
    name=Example,
    ]{eg}

\declaretheorem[
    style=thmbox, 
    % numberwithin = section,
    name=Proposition]{prop}

\declaretheorem[
    style = thmbox,
    numbered=yes,
    name =Problem
    ]{problem}

\declaretheorem[style=thmbox, name=Theorem]{theorem}
\declaretheorem[style=thmbox, name=Lemma]{lemma}
\declaretheorem[style=thmbox, name=Corollary]{corollary}

\declaretheorem[style=thmproofbox, name=Proof]{replacementproof}

\declaretheorem[style=thmproofbox, 
                name = Solution
                ]{replacementsolution}

\renewenvironment{proof}[1][\proofname]{\vspace{-1pt}\begin{replacementproof}}{\end{replacementproof}}

\newenvironment{solution}
    {
        \vspace{-1pt}\begin{replacementsolution}
    }
    { 
            \end{replacementsolution}
    }

\declaretheorem[style=thmexplanationbox, name=Proof]{tmpexplanation}
\newenvironment{explanation}[1][]{\vspace{-10pt}\begin{tmpexplanation}}{\end{tmpexplanation}}

\declaretheorem[style=thmbox, numbered=no, name=Remark]{remark}
\declaretheorem[style=thmbox, numbered=no, name=Note]{note}

\newtheorem*{uovt}{UOVT}
\newtheorem*{notation}{Notation}
\newtheorem*{previouslyseen}{As previously seen}
% \newtheorem*{problem}{Problem}
\newtheorem*{observe}{Observe}
\newtheorem*{property}{Property}
\newtheorem*{intuition}{Intuition}

\usepackage{etoolbox}
\AtEndEnvironment{vb}{\null\hfill$\diamond$}%
\AtEndEnvironment{intermezzo}{\null\hfill$\diamond$}%
% \AtEndEnvironment{opmerking}{\null\hfill$\diamond$}%

% http://tex.stackexchange.com/questions/22119/how-can-i-change-the-spacing-before-theorems-with-amsthm
\makeatletter
% \def\thm@space@setup{%
%   \thm@preskip=\parskip \thm@postskip=0pt
% }
\newcommand{\oefening}[1]{%
    \def\@oefening{#1}%
    \subsection*{Oefening #1}
}

\newcommand{\suboefening}[1]{%
    \subsubsection*{Oefening \@oefening.#1}
}

\newcommand{\exercise}[1]{%
    \def\@exercise{#1}%
    \subsection*{Exercise #1}
}

\newcommand{\subexercise}[1]{%
    \subsubsection*{Exercise \@exercise.#1}
}


\usepackage{xifthen}

\def\testdateparts#1{\dateparts#1\relax}
\def\dateparts#1 #2 #3 #4 #5\relax{
    \marginpar{\small\textsf{\mbox{#1 #2 #3 #5}}}
}

\def\@lesson{}%
\newcommand{\lesson}[3]{
    \ifthenelse{\isempty{#3}}{%
        \def\@lesson{Lecture #1}%
    }{%
        \def\@lesson{Lecture #1: #3}%
    }%
    \subsection*{\@lesson}
    \testdateparts{#2}
}

% \renewcommand\date[1]{\marginpar{#1}}


% fancy headers
\usepackage{fancyhdr}
\pagestyle{fancy}

\makeatother

% notes
\usepackage{todonotes}
\usepackage{tcolorbox}

\tcbuselibrary{breakable}
\newenvironment{verbetering}{\begin{tcolorbox}[
    arc=0mm,
    colback=white,
    colframe=green!60!black,
    title=Opmerking,
    fonttitle=\sffamily,
    breakable
]}{\end{tcolorbox}}

\newenvironment{noot}[1]{\begin{tcolorbox}[
    arc=0mm,
    colback=white,
    colframe=white!60!black,
    title=#1,
    fonttitle=\sffamily,
    breakable
]}{\end{tcolorbox}}

% figure support
\usepackage{import}
\usepackage{xifthen}
\pdfminorversion=7
\usepackage{pdfpages}
\usepackage{transparent}
\newcommand{\incfig}[1]{%
    \def\svgwidth{\columnwidth}
    \import{./figures/}{#1.pdf_tex}
}

% %http://tex.stackexchange.com/questions/76273/multiple-pdfs-with-page-group-included-in-a-single-page-warning
\pdfsuppresswarningpagegroup=1




\begin{document}

\section{Lecture 12}

\subsection{Topics}

\begin{itemize}
    \item Definition of \( K- \)cell. 
    \item Theorem: If \( {I}_{1} \supseteq {I}_{2} \supseteq {I}_{3} \supseteq \dots  \) is a sequence of \( k- \)cells, then \( \bigcap_{ n = 1  }^{ \infty    }  {I}_{n} \) is nonempty.
    \item Theorem: Every \( k - \)cell is compact.
    \item Theorem: Suppose \( E \subseteq \R^{k} \). Then
    \item Connected sets
        \begin{center}
            \( E  \) is closed and bounded \( \iff  \) \( E  \) is compact \( \iff  \) Every infinite subset of \( E  \) has a limit point in \( E  \).
        \end{center}
    \item Theorem: Every bounded infinite subset of \( \R^{k} \) has a limit point in \( \R^{k } \). 
\end{itemize}

\begin{corollary}
If \( {I}_{1} \supseteq {I}_{2} \supseteq {I}_{3} \supseteq \dots  \) is a sequence of compact sets, then \( \bigcap_{ n = 1  }^{ \infty    }  {I}_{n} \) is nonempty.
\end{corollary}


\begin{theorem}[Nested Interval Property]
    If \( {I}_{n} = [{a}_{n}, {b}_{n}] \) is a sequence of closed intervals in \( \R  \) such that \( {I}_{1} \supseteq {I}_{2} \supseteq {I}_{3} \supseteq \dots  \), then \( \bigcap_{  n = 1 }^{ \infty   }  {I}_{n} \) is nonempty.
\end{theorem}

\begin{definition}[K-cell]
    The set \( I = [{a}_{1}, {b}_{1}] \times \cdots \times [{a}_{k}, {b}_{k}] \) is called a \( k- \)cell in \( \R^{k} \).  
\end{definition}
\begin{eg}
 Let \( I = [{a}_{1}, {a}_{2}]  \times [{a}_{2}, {b}_{2}] \) is a \( 2- \)cell in \( \R^{2} \).
\end{eg}

\begin{theorem}[Nested Cell Property]
   If \( {I}_{1} \supseteq {I}_{2} \supseteq {I}_{3} \cdots  \) is a nested sequence of \( k - \)cells, then \( \bigcap_{ n=1 }^{ \infty  }  {I}_{n} \neq \emptyset \). 
\end{theorem}
\begin{proof}
    For each \( n \in \N  \), let 
    \begin{align*}
        {I}_{n} = [{a}_{1}^{(n)}, {b}_{1}^{(n)}] \times \cdots \times [{a}_{k}^{(n)}, {b}_{k}^{(n)}]  \\
    \end{align*}
    Also, let 
    \begin{center}
        \( \forall n \in \N  \) and \( \forall 1 \leq i \leq k  \), we have \( {A}_{i}^{(n)} = [{a}_{i}^{(n)}, {b}_{i}^{(n)}] \)
    \end{center}
    Since for each \( n \in \N  \), \( {I}_{n} \supseteq {I}_{n+1} \), we have
    \[  {A}_{i}^{(n) }  \supseteq {A}_{i}^{(n+1)}\ \forall 1 \leq i \leq k.  \]
    That is,
    \begin{align*}
       {I}_{1} &= {A}_{1}^{(1)} \times \cdots \times {A}_{k }^{(1)} \\
       {I}_{2} &= {A}_{1}^{(2)} \times \cdots \times {A}_{k}^{(2)} \\
               &\vdots  \\
        {I}_{n} &= {A}_{1}^{(n)} \times \cdots {A}_{k}^{(n)}.
    \end{align*}
    Hence, it follows from the nested interval property that there exists 
    \begin{align*}
        \exists {x}_{1} &\in \bigcap_{ n=1  }^{ \infty  } {A}_{1}^{(n)} \\
        \exists {x}_{2} &\in \bigcap_{ i=1  }^{ \infty   }  {A}_{2}^{(n)} \\
                        &\vdots \\
        \exists {x}_{k} &\in \bigcap_{ n=1  }^{ \infty  }  {A}_{k}^{(n)}. 
    \end{align*}
    Thus, by a fact in set theory; that is,  
    \[  (A \cap B) \times (C \cap D ) \subseteq  (A \times C ) \cap (B \times D). \]
    \begin{align*}
        ({x}_{1}, \dots, {x}_{k}) &\in \Big[\bigcup_{ n=1  }^{ \infty   }  {A}_{1}^{(n)} \Big] \times \Big[\bigcap_{ n=1  }^{ \infty  } {A}_{2}^{(n)} \Big] \times \cdots \times \Big[ \bigcap_{  n = 1 }^{ \infty  }  {A}_{k}^{(n)}\Big] \\
                                  &\subseteq \bigcap_{ n=1  }^{ \infty  }  [{A}_{1}^{(n)} \times \cdots \times {A}_{k}^{(n)}] \\
                                  &= \bigcap_{ n=1  }^{ \infty  }  {I}_{n}.
    \end{align*}
    Hence, we see that 
    \[  \bigcap_{ n=1  }^{ \infty  }  {I}_{n} \neq \emptyset. \]
\end{proof}

\begin{theorem}[ ]
Every \( k- \)cell in \( \R^{k} \) is compact.    
\end{theorem}
\begin{proof}
    Here we will prove the claim for \( 2- \)cells. The proof for a general \( k- \)cell is completely analogous. Let \( I = [{a}_{1}, {b}_{1}] \times [{a}_{2}, {b}_{2}] \) be a \( 2- \)cell. Let \( a = ({a}_{1}, {a}_{2})  \) and \( b = ({b}_{1}, {b}_{2}) \). Let \[ \delta = d(a,b) = \| a - b\| = \sqrt{ | {a}_{1} - {b}_{1} |^{2} + | {a}_{2} - {b}_{2} |^{2} }  \]. Note that if \( x = ({x}_{1}, {x}_{2}) \) and \( y = ({y}_{1}, {y}_{2}) \) are any two points in \( I  \), then
    \begin{align*}
    {x}_{1}, {y}_{1} \in [{a}_{1}, {b}_{2}] &\implies | {x}_{1} - {y}_{1}  |  \leq | {b}_{1} - {a}_{1} |  \\
    {x}_{2}, {y}_{2} \in [{a}_{2}, {b}_{2}]                     &\implies  | {x}_{2} - {y}_{2} |  \leq | {b}_{2} - {a}_{2} |
\end{align*}
which implies that 
\[ \sqrt{ | {x}_{1} - {y}_{1} |^{2} + | {x}_{2} - {y}_{2} |^{2} } \leq \sqrt{ | {a}_{1} - {b}_{1} |^{2} + | {a}_{2} - {b}_{2}  |^{2} } = \delta.  \]
So, \( d(x,y) \leq \delta \).
Let us assume for contradiction that \( I  \) is NOT compact. So, there exists an open cover \( \{ {G}_{\alpha} \}_{\alpha \in \Lambda} \) of \( I  \) that does NOT have a finite subcover; that is, \( I \subseteq \bigcup_{ \alpha \in \Lambda }^{  } {G}_{\alpha} \). For each \( 1 \leq i \leq 2  \), divide \( [{a}_{i}, {b}_{i}] \) into two subintervals of equal length:
\[  {c}_{i} = \frac{ {a}_{i} + {b}_{i} }{ 2 }  \ \ [{a}_{i}, {b}_{i}] = [{a}_{i}, {c}_{i}] \cup [{c}_{i}, {b}_{i}]. \]
These subintervals determine \( 4  \) \( 2 - \)cells. There is at least one of these \( 4  \) \( 2- \)cells that is not covered by any finite subcollection of \( \{ {G}_{\alpha} \}_{\alpha \in \Lambda} \). Let us call this \( 2- \)cell as \( {I}_{1} \). Notice that 
\begin{align*}
    \forall x,y \in {I}_{1}  \ \  \|x - y\|_2 \leq \frac{ \delta }{ 2 }.
\end{align*}
Now, subdivide \( {I}_{1} \) into \( 4  \) \( 2- \)cells and continue this process inductively. In this manner, we will obtain a sequence of \( 2- \)cells 
\[ I, {I}_{1}, {I}_{2}, {I}_{3}, \dots  \]
such that 
\begin{align*}
    &I \supseteq {I}_{1} \supseteq {I}_{2} \supseteq {I}_{3} \supseteq \dots  \tag{1} \\
    &\forall x,y \in {I}_{n},  \ \ \| x - y \| \leq \frac{ \delta }{ 2 }  \tag{2} \\
    &\forall n \in \N  \ \   {I}_{n} \  \text{cannot be covered by a finite subcollection of} \ \  \{ {G}_{\alpha} \}_{\alpha \in I}
\end{align*}

\end{proof}


\begin{theorem}[Heine-Borel Theorem]
   Let \( E \subseteq \R^{k} \). The following statements are equivalent:
   \begin{enumerate}
       \item[(a)] \( E  \) is closed and bounded. 
        \item[(b)] \( E  \) is compact.
        \item[(c)] Every infinite subset of \( E  \) has a limit point of \( E  \). 
   \end{enumerate}
\end{theorem}
\begin{proof}

\end{proof}


\begin{remark}
    Note that in any general metric space, we have \( (a) \implies (b) \) is not necessarily true.
\end{remark}


\begin{theorem}[Bolzano-Weierstrass Theorem]
 Let \( E \subseteq \R^{k }  \) and \( E  \) is an infinite set and bounded. Then \( E' \neq \emptyset \).   
\end{theorem}
\begin{proof}
Suppose that \( E  \) is bounded. Then there exists a \( k- \)cell \( I  \) such that \( E \subseteq  I  \). By Theorem 2.40, we know that \( I  \) is a compact set. Furthermore, we know that \( I  \) is limit point compact by Theorem 2.41. So, every infinite set in \( I  \) has a limit point in \( I  \). In particular, \( E  \) has a limit point in \( I  \). So, \( E \neq \emptyset \).
\end{proof}

\begin{definition}[Connected Sets, Disconnected, connected]
    Let \( (X,d) \) be a metric space. 
    \begin{enumerate}
        \item[(i)] Two sets \( A,B \subseteq X   \) are aid to be disjoint if \( A \cap B = \emptyset \).
        \item[(ii)] Two sets \( A,B \subseteq X  \) are said to be \textbf{separated} if \( \overline{A} \cap B  \) and \( A \cap \overline{B}  \) are both empty.
        \item [(iii)] A set \( E \subseteq X   \) is said to \textbf{disconnected} if it can be written as a union of tow nonempty separated sets \( A  \) and \( B  \); that is, \( E = A \cup B  \).
        \item[(iv)] A set \( E \subseteq X  \) is said to be connected if it is NOT disconnected.
    \end{enumerate}
\end{definition}

\begin{eg}[\( \R  \) with the standard metric]
\begin{enumerate}
    \item[(*)] If we have \( A = (1,2)  \) and \( B = (2,5) \) are separated, then
        \begin{align*}
            \overline{A} \cap B &= [1,2] \cap (2,5) = \emptyset \\
            A \cap \overline{B} &= (1,2) \cap [2,5] = \emptyset.
        \end{align*}
        Hence, \( E = A \cup B  \) is disconnected.
    \item[(*)] We have \( C = (1,2] \) and \( D = (2,5) \) are disjoint but not separated; that is, we have
        \begin{align*}
            C \cap \overline{D} &= (1,2] \cap [2,5] = \{ 2 \}  \\
            C \cup D &= (1,5) \ \text{is indeed connected.}
        \end{align*}
\end{enumerate}    
\end{eg}

\section{Lecture 12}




\end{document}

