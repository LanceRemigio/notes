\documentclass[a4paper]{article}

\usepackage[utf8]{inputenc}
\usepackage[T1]{fontenc}
\usepackage{textcomp}
\usepackage{hyperref}
% \usepackage{fourier}
% \usepackage[dutch]{babel}
\usepackage{url}
% \usepackage{hyperref}
% \hypersetup{
%     colorlinks,
%     linkcolor={black},
%     citecolor={black},
%     urlcolor={blue!80!black}
% }
\usepackage{graphicx}
\usepackage{float}
\usepackage{booktabs}
\usepackage{enumitem}
% \usepackage{parskip}
\usepackage{emptypage}
\usepackage{subcaption}
\usepackage{multicol}
\usepackage[usenames,dvipsnames]{xcolor}

% \usepackage{cmbright}


\usepackage[margin=1in]{geometry}
\usepackage{amsmath, amsfonts, mathtools, amsthm, amssymb}
\usepackage{mathrsfs}
\usepackage{cancel}
\usepackage{bm}
\newcommand\N{\ensuremath{\mathbb{N}}}
\newcommand\R{\ensuremath{\mathbb{R}}}
\newcommand\Z{\ensuremath{\mathbb{Z}}}
\renewcommand\O{\ensuremath{\emptyset}}
\newcommand\Q{\ensuremath{\mathbb{Q}}}
\newcommand\C{\ensuremath{\mathbb{C}}}
\DeclareMathOperator{\sgn}{sgn}
\usepackage{systeme}
\let\svlim\lim\def\lim{\svlim\limits}
\let\implies\Rightarrow
\let\impliedby\Leftarrow
\let\iff\Leftrightarrow
\let\epsilon\varepsilon
\usepackage{stmaryrd} % for \lightning
\newcommand\contra{\scalebox{1.1}{$\lightning$}}
% \let\phi\varphi
\renewcommand\qedsymbol{$\blacksquare$}




% correct
\definecolor{correct}{HTML}{009900}
\newcommand\correct[2]{\ensuremath{\:}{\color{red}{#1}}\ensuremath{\to }{\color{correct}{#2}}\ensuremath{\:}}
\newcommand\green[1]{{\color{correct}{#1}}}



% horizontal rule
\newcommand\hr{
    \noindent\rule[0.5ex]{\linewidth}{0.5pt}
}


% hide parts
\newcommand\hide[1]{}



% si unitx
\usepackage{siunitx}
\sisetup{locale = FR}
% \renewcommand\vec[1]{\mathbf{#1}}
\newcommand\mat[1]{\mathbf{#1}}


% tikz
\usepackage{tikz}
\usepackage{tikz-cd}
\usetikzlibrary{intersections, angles, quotes, calc, positioning}
\usetikzlibrary{arrows.meta}
\usepackage{pgfplots}
\pgfplotsset{compat=1.13}


\tikzset{
    force/.style={thick, {Circle[length=2pt]}-stealth, shorten <=-1pt}
}

% theorems
\makeatother
\usepackage{thmtools}
\usepackage[framemethod=TikZ]{mdframed}
\mdfsetup{skipabove=1em,skipbelow=0em}


\theoremstyle{definition}

\declaretheoremstyle[
    headfont=\bfseries\sffamily\color{ForestGreen!70!black}, bodyfont=\normalfont,
    mdframed={
        linewidth=2pt,
        rightline=false, topline=false, bottomline=false,
        linecolor=ForestGreen, backgroundcolor=ForestGreen!5,
    }
]{thmgreenbox}

\declaretheoremstyle[
    headfont=\bfseries\sffamily\color{NavyBlue!70!black}, bodyfont=\normalfont,
    mdframed={
        linewidth=2pt,
        rightline=false, topline=false, bottomline=false,
        linecolor=NavyBlue, backgroundcolor=NavyBlue!5,
    }
]{thmbluebox}

\declaretheoremstyle[
    headfont=\bfseries\sffamily\color{NavyBlue!70!black}, bodyfont=\normalfont,
    mdframed={
        linewidth=2pt,
        rightline=false, topline=false, bottomline=false,
        linecolor=NavyBlue
    }
]{thmblueline}

\declaretheoremstyle[
    headfont=\bfseries\sffamily\color{RawSienna!70!black}, bodyfont=\normalfont,
    mdframed={
        linewidth=2pt,
        rightline=false, topline=false, bottomline=false,
        linecolor=RawSienna, backgroundcolor=RawSienna!5,
    }
]{thmredbox}

\declaretheoremstyle[
    headfont=\bfseries\sffamily\color{RawSienna!70!black}, bodyfont=\normalfont,
    numbered=no,
    mdframed={
        linewidth=2pt,
        rightline=false, topline=false, bottomline=false,
        linecolor=RawSienna, backgroundcolor=RawSienna!1,
    },
    qed=\qedsymbol
]{thmproofbox}

\declaretheoremstyle[
    headfont=\bfseries\sffamily\color{NavyBlue!70!black}, bodyfont=\normalfont,
    numbered=no,
    mdframed={
        linewidth=2pt,
        rightline=false, topline=false, bottomline=false,
        linecolor=NavyBlue, backgroundcolor=NavyBlue!1,
    },
]{thmexplanationbox}

\declaretheorem[style=thmgreenbox, numberwithin = section, name=Definition]{definition}
\declaretheorem[style=thmbluebox, name=Example]{eg}
\declaretheorem[style=thmredbox, numberwithin = section, name=Proposition]{prop}
\declaretheorem[style=thmredbox, numberwithin = section, name=Theorem]{theorem}
\declaretheorem[style=thmredbox, numberwithin = section,  name=Lemma]{lemma}
\declaretheorem[style=thmredbox, numberwithin = section,  numbered=no, name=Corollary]{corollary}


\declaretheorem[style=thmproofbox, name=Proof]{replacementproof}
\renewenvironment{proof}[1][\proofname]{\vspace{-10pt}\begin{replacementproof}}{\end{replacementproof}}


\declaretheorem[style=thmexplanationbox, name=Proof]{tmpexplanation}
\newenvironment{explanation}[1][]{\vspace{-10pt}\begin{tmpexplanation}}{\end{tmpexplanation}}


\declaretheorem[style=thmblueline, numbered=no, name=Remark]{remark}
\declaretheorem[style=thmblueline, numbered=no, name=Note]{note}

\newtheorem*{uovt}{UOVT}
\newtheorem*{notation}{Notation}
\newtheorem*{previouslyseen}{As previously seen}
\newtheorem*{problem}{Problem}
\newtheorem*{observe}{Observe}
\newtheorem*{property}{Property}
\newtheorem*{intuition}{Intuition}


\usepackage{etoolbox}
\AtEndEnvironment{vb}{\null\hfill$\diamond$}%
\AtEndEnvironment{intermezzo}{\null\hfill$\diamond$}%
% \AtEndEnvironment{opmerking}{\null\hfill$\diamond$}%

% http://tex.stackexchange.com/questions/22119/how-can-i-change-the-spacing-before-theorems-with-amsthm
\makeatletter
% \def\thm@space@setup{%
%   \thm@preskip=\parskip \thm@postskip=0pt
% }
\newcommand{\oefening}[1]{%
    \def\@oefening{#1}%
    \subsection*{Oefening #1}
}

\newcommand{\suboefening}[1]{%
    \subsubsection*{Oefening \@oefening.#1}
}

\newcommand{\exercise}[1]{%
    \def\@exercise{#1}%
    \subsection*{Exercise #1}
}

\newcommand{\subexercise}[1]{%
    \subsubsection*{Exercise \@exercise.#1}
}


\usepackage{xifthen}

\def\testdateparts#1{\dateparts#1\relax}
\def\dateparts#1 #2 #3 #4 #5\relax{
    \marginpar{\small\textsf{\mbox{#1 #2 #3 #5}}}
}

\def\@lesson{}%
\newcommand{\lesson}[3]{
    \ifthenelse{\isempty{#3}}{%
        \def\@lesson{Lecture #1}%
    }{%
        \def\@lesson{Lecture #1: #3}%
    }%
    \subsection*{\@lesson}
    \testdateparts{#2}
}

% \renewcommand\date[1]{\marginpar{#1}}


% fancy headers
\usepackage{fancyhdr}
\pagestyle{fancy}

\fancyhead[LE,RO]{Lance Remigio}
\fancyhead[RO,LE]{\@lesson}
\fancyhead[RE,LO]{}
\fancyfoot[LE,RO]{\thepage}
\fancyfoot[C]{\leftmark}

\makeatother




% notes
\usepackage{todonotes}
\usepackage{tcolorbox}

\tcbuselibrary{breakable}
\newenvironment{verbetering}{\begin{tcolorbox}[
    arc=0mm,
    colback=white,
    colframe=green!60!black,
    title=Opmerking,
    fonttitle=\sffamily,
    breakable
]}{\end{tcolorbox}}

\newenvironment{noot}[1]{\begin{tcolorbox}[
    arc=0mm,
    colback=white,
    colframe=white!60!black,
    title=#1,
    fonttitle=\sffamily,
    breakable
]}{\end{tcolorbox}}




% figure support
\usepackage{import}
\usepackage{xifthen}
\pdfminorversion=7
\usepackage{pdfpages}
\usepackage{transparent}
\newcommand{\incfig}[1]{%
    \def\svgwidth{\columnwidth}
    \import{./figures/}{#1.pdf_tex}
}

% %http://tex.stackexchange.com/questions/76273/multiple-pdfs-with-page-group-included-in-a-single-page-warning
\pdfsuppresswarningpagegroup=1




\begin{document}

\section{Lecture 16}

\subsection{Topics}

\begin{itemize}
    \item Diameter of a set 
    \item Theorem: \( \diam \overline{E} = \diam E   \)
    \item Theorem: Nested sequence of nonempty compact sets \( {K}_{n} \) and if \( \diam {K}_{n} \to 0   \), then \( \bigcap_{ n=1  }^{ \infty  }  {K}_{n} \) consists of exactly one point.
    \item Theorem: Every compact metric space is complete.
    \item Theorem: \( \R^{k} \) is a complete metric space.
\end{itemize}

\begin{definition}[Diameter of a Set]
    Let \( (X,d) \) be a metric space. Let \( E  \) be a nonmempty set in \( X  \). The diameter of \( E  \), denoted by \( \diam E  \), is defined as follows:
    \[  \diam E = \sup \{ d(a,b) : a,d \in E  \}.  \]
\end{definition}

\begin{remark}
    Note that if \( \emptyset \neq A \subseteq  B \subseteq X   \), then 
    \[  \{ d(a,b) : a,b \in A  \}  \subseteq  \{ d(a,b) : a,b  \in B\}. \]
    Thus, we have 
    \[  \sup \{ d(a,b) : a,b  \in A \}  \leq \sup \{ d(a,b) : a,b \in B \}. \]
    That is, 
    \[  \diam A \leq \diam B. \]
\end{remark}

\begin{theorem}[Interesting Observation]
   Let \( (X,d) \) be a metric space. Let \( ({x}_{n}) \) be a sequence in \( X  \). For all \( n \in \N  \), let \( {E}_{n} = \{ {x}_{n+1}, {x}_{n+1}, \dots  \}  \). Then 
   \begin{center}
       \( ({x}_{n})  \) is Cauchy \( \iff  \) \( \lim_{ n \to \infty  }  \diam {E}_{n} = 0  \).
   \end{center}
\end{theorem}
\begin{proof}
\( ( \Longrightarrow ) \) Assume that \( ({x}_{n}) \) is a Cauchy sequence. Note that 
\begin{align*}
    {E}_{1} &= \{ {x}_{2}, {x}_{3}, {x}_{4}, {x}_{5}, \dots  \}  \\
    {E}_{2} &= \{ {x}_{3}, {x}_{4}, {x}_{5}, \dots  \}  \\
    {E}_{3}  &= \{ {x}_{4}, {x}_{5}, \dots  \}  \\
    {E}_{4} &= \dots \\
            &\vdots
\end{align*}
Clearly, we have 
\[  {E}_{1} \supseteq {E}_{2} \supseteq {E}_{3} \supseteq {E}_{4} \supseteq \cdots .  \]
So, we have 
\[  \diam {E}_{1} \geq \diam {E}_{2} \geq \diam {E}_{3} \geq \cdots . \]
Our goal is to show that 
\[  \forall \ \epsilon > 0 \ \exists N \in \N \ \forall n > N \ \underbrace{| \diam {E}_{n} - 0 |}_{\diam {E}_{n} \geq 0}  < \epsilon. \tag{*} \]
So, it suffices to show that \( \diam {E}_{n} < \epsilon \). To this end, let \( \epsilon > 0  \) be given. Since \( ({x}_{n}) \) is Cauchy, there exists \( \hat{N} \in \N \) such that 
\[  \forall n,m > \hat{N} \ \ d({x}_{n}, {x}_{m}) < \frac{ \epsilon }{ 2 }. \]
We claim that this \( \hat{N} \) is the same \( N  \) we were looking for. Letting \( N = \hat{N} \), we have 
\[  {E}_{\hat{N}} = \{ {x}_{\hat{N} + 1} , {x}_{\hat{N} + 2}, {x}_{\hat{N} + 3}, \dots  \}. \]
Hence, we have
\[  \forall a,b \in {E}_{\hat{N}} \ \ d(a,b) < \frac{ \epsilon }{ 2 }. \]
Thus, we have 
\[  \diam {E}_{\hat{N}} = \sup \{ d(a,b) : a,b \in {E}_{\hat{N}} \} \leq \frac{ \epsilon }{ 2 }  < \epsilon. \]
If \( n > \hat{N} \), then
\[  \diam {E}_{n} \leq \diam {E}_{\hat{N}} < \epsilon \]
as desired.

\( (\Longleftarrow) \) Assume that \( \lim_{ n \to \infty  }  \diam {E}_{n} = 0  \). Our goal is to show that \( ({x}_{n}) \) is Cauchy; that is, 
\[  \forall \epsilon > 0 \ \exists N \in \N \ \text{such that} \ \forall n,m > N \ d({x}_{n}, {x}_{m}) < \epsilon. \tag{*} \]
Since \( \lim_{ n \to  \infty  }  \diam {E}_{n} = 0 \), for this given \( \epsilon \), there exists \( \hat{N} \) such that 
\[  \forall n > \hat{N} \ \ \diam {E}_{n} < \epsilon. \]
In particular, we have \( \diam {E}_{\hat{N} + 1} < \epsilon \).
Now, we claim that \( N = \hat{N} + 1  \) can be used as the same \( N  \) we were looking for. Indeed, if we let \( N = \hat{N} = 1 \), we have
\[  \forall n,m > \hat{N} + 1 , \ {x}_{n}, {x}_{m} \in {E}_{\hat{N} +1}  \]
and thus
\[  d({x}_{n}, {x}_{m}) \leq \diam {E}_{\hat{N} + 1} < \epsilon. \]
\end{proof}

\begin{theorem}[ ]
    Let \( (X,d) \) be a metric space. Let \( E  \) be a nonempty subset of \( X  \). Then 
    \[  \diam \overline{E} = \diam E.  \]
\end{theorem}
\begin{proof}
Note that, since \( E \subseteq \overline{E} \), we have \( \diam E \leq \diam \overline{E} \). In what follows, we will prove that \( \diam \overline{E} \leq \diam E  \) by showing that 
\[  \forall \epsilon > 0  \ \ \diam \overline{E} \leq \diam E + \epsilon. \]
Let \( \epsilon > 0  \) be given. Our goal is to show that 
\[  \sup \{ d(a,b) : a, b \in \overline{E} \} \leq \diam E + \epsilon. \]
To this end, it suffices to show that \( \diam E + \epsilon  \) is an upper bound for \( \{ d(a,b) : a,b \in \overline{E} \}  \). Suppose \( a,b \in \overline{E} \). We have 
\begin{align*}
    &a \in \overline{E} \implies {N}_{\frac{ \epsilon }{ 2 } }(a) \cap E \neq \emptyset \implies \exists x \in E \ \text{such that} \ d(x,a) < \frac{ \epsilon }{ 2 }  \\
    &b \in \overline{E} \implies  {N}_{\frac{ \epsilon }{ 2 } }(b) \cap E \neq \emptyset  \implies \exists y \in E \ \text{such that} \ d(y,b) < \frac{ \epsilon }{ 2 }.
\end{align*}
Therefore, we have that 
\begin{align*}
    d(a,b) &\leq d(a,x) + d(x,y) + d(y,b) \\
           &< \frac{ \epsilon  }{ 2 }  + d(x,y) + \frac{ \epsilon }{ 2 } \\
           &< \frac{ \epsilon  }{ 2  }  + \diam E + \frac{ \epsilon }{ 2 }  \\
           &= \epsilon + \diam E 
\end{align*}
which is our desired result.
\end{proof}

\begin{theorem}[ ]
Let \( (X,d) \) be a metric space and let \( {K}_{1} \supseteq {K}_{2} \supseteq {K}_{3} \supseteq \cdots  \) be a nested sequence of nonempty compact sets where
\( \lim_{ n \to \infty  }  \diam {K}_{n} = 0  \). Then 
\begin{center}
    \( \bigcap_{ n=1  }^{ \infty  }  {K}_{n} \) consists of exactly one point.    
\end{center}
\end{theorem}
\begin{proof}
Let \( K = \bigcap_{ n=1  }^{ \infty  }  {K}_{n} \). By Theorem 2.3.6, we know that \( K \neq \emptyset \). In order to show that \( K  \) has only one element, we suppose that \( a,b \in K  \) and we will prove \( a = b \). In order to show that \( a = b  \), we will prove that \( d(a,b) = 0  \). Showing this is equivalent to showing that 
\[  \forall \epsilon > 0 \ \ d(a,b) < \epsilon.  \]
Let \( \epsilon > 0  \) be given. Since \( \lim_{ n \to \infty  }  \diam {K}_{n} = 0  \), there exists \( N \in \N  \) such that 
\[  \forall n > N  \ \ \diam {K}_{n} < \epsilon. \]
In particular, \( \diam {K}_{N+1} < \epsilon \). Now, we have 
\[  a \in \bigcap_{ n=1  }^{ \infty  }  \implies a \in {K}_{N+1}  \]
and 
\[  b \in \bigcap_{ n=1  }^{ \infty  }  {K}_{n} \implies b \in {K}_{N+1}  \]
which implies further that 
\[  d(a,b) \leq \diam {K}_{N+1} < \epsilon \]
which is our desired result.
\end{proof}

\begin{theorem}[Compact Space \( \implies \) Complete Space]
   Any compact metric space is complete. 
\end{theorem}
\begin{proof}
Let \( (X,d) \) be a compact metric space. Let \( ({x}_{n}) \) be a Cauchy sequence in \( X  \). Our goal is to show that \( ({x}_{n}) \) converges in \( X  \). For each \( n \in \N  \), let \( {E}_{n} =\{  {x}_{n+1}, {x}_{n+2}, {x}_{n+3} \dots  \}  \). We know that 
\begin{enumerate}
    \item[(1)] \( {E}_{1} \supseteq {E}_{2} \supseteq {E}_{3} \supseteq \cdots  \)
    \item[(2)] \( ({x}_{n}) \) is Cauchy \( \implies  \) \( \lim_{ n \to \infty  } \diam {E}_{n} = 0 \).
\end{enumerate}
It follows from (1) that
\[  \overline{{E}_{1}} \supseteq \overline{{E}_{2}} \supseteq \overline{{E}_{3}} \supseteq \cdots \tag{\(\dagger\)} \]
Since closed subsets of a compact metric space are compact, we know that \( (\dagger) \) is a nested sequence of nonempty compact sets. Since \( \diam {E}_{n} = \diam \overline{{E}_{n}} \), it follows from (2) that \( \lim_{ n \to \infty  }  \diam \overline{{E}_{n}} = 0  \). Hence, we know by the previous theorem that \( \bigcap_{ n=1  }^{ \infty  }  \overline{{E}_{n}} \) has exactly one point. Let us denote this point as \( a  \) where
\[  \bigcap_{ n=1 }^{ \infty  } \overline{{E}_{n}} = \{ a \}.  \]
In what follows, we will prove that \( \lim_{ n \to \infty  } {x}_{n} = a  \). To this end, it suffices to show that 
\[  \forall \epsilon > 0 \ \exists N \in \N \ \text{such that} \ \forall n > N \ d({x}_{n},a) < \epsilon. \tag{*} \]
Let \( \epsilon > 0  \) be give. Since \( \lim_{ n \to \infty  }  \diam \overline{{E}_{n}} = 0  \), for this given \( \epsilon \), there exists \( \hat{N} \) such that 
\[  \forall n > \hat{N} \ \ \diam \overline{{E}_{n}} < \epsilon. \]
We claim that \( \hat{N} + 1 \) can be used as the \( N  \) that we were looking for. Indeed, if we let \( N = \hat{N} + 1 \), then (*) holds. The reason is as follows: If \( n > \hat{N} + 1 \), then
\[  {x}_{n} \in {E}_{\hat{N} + 1} \implies {x}_{n} \in \overline{{E}_{\hat{N} + 1}}. \tag{I} \]
Furthermore, 
\[  a \in \bigcap_{  n = 1  }^{ \infty   }  \overline{{E}_{n}} \implies a \in \overline{{E}_{\hat{N} + 1}}. \tag{II}   \]
Thus, (I) and (II) imply that 
\[  d({x}_{n},a) \leq \diam \overline{{E}_{\hat{N} + 1}} < \epsilon. \]
\end{proof}

\begin{theorem}[\( \R^{k} \) is complete]
    \( \R^{k} \) is a complete metric space.
\end{theorem}

\begin{proof}
Let \( ({x}_{n}) \) be a Cauchy sequence in \( \R^{k} \). By homework 7, we see that \( ({x}_{n}) \) must be bounded; that is,
\[  \exists p \in \R^{k } , \epsilon > 0 \ \text{such that} \ \forall n \in \N \ {x}_{n} \in {N}_{\epsilon}(p). \]
Note that \( \overline{{N}_{\epsilon}(p)}  \) is a closed and bounded set in \( \R^{k} \), and so it must be compact by the Heine-Borel Theorem. Thus, if \( \overline{{N}_{\epsilon}(p)}  \) is a compact metric space and \( ({x}_{n}) \) is a Cauchy sequence in \( \overline{{N}_{\epsilon}(p)} \), we have that \( ({x}_{n}) \) converges to a point \( x \in \overline{{N}_{\epsilon}(p)} \) by the previous theorem. Since the metric in \( \overline{{N}_{\epsilon}(p)} \) is exactly the same as the metric in \( \R^{k} \), we can conclude that \( {x}_{n} \to x  \) in \( \R^{k} \).

\end{proof}

\section{Lecture 17}

\subsection{Topics}
\begin{itemize}
    \item Algebraic Limit Theorem (for sequences of numbers)
    \item Divergent sequences
    \item Monotone sequences 
    \item Monotone Convergence Theorem
    \item Extended real numbers
    \item Properties of limit in \( \overline{\R} \) (Extended Real Numbers)
\end{itemize}

\begin{theorem}[Algebraic Limit Theorem]
    Suppose \( ({a}_{n}) \) and \( ({b}_{n}) \) are sequences of real numbers, and \( \lim_{ n \to \infty  }  {a}_{n} = a   \), \( \lim_{ n \to \infty  } {b}_{n} = b  \). Then
    \begin{enumerate}
        \item[(i)] \( \lim_{ n \to \infty  }  ({a}_{n} + {b}_{n}) = a + b \)
        \item[(ii)] n\( \lim_{ n \to \infty  } ({ca}_{n}) = ca \) for any real number \( c  \).
        \item[(iii)] \( \lim_{ n \to \infty  }  ({a}_{n}{b}_{n}) = ab \)
        \item[(iv)] \( \lim_{ n \to \infty  } \frac{ {a}_{n} }{ {b}_{n} }  = \frac{ a }{ b }  \), provided that \( b \neq 0  \).
    \end{enumerate}
\end{theorem}
\begin{proof}
    \textit{Proof left to the reader.}
\end{proof}

\begin{definition}[ ]
    Consider \( \R  \) with its standard metric. Let \( ({x}_{n}) \) be a sequence of real numbers. If \( ({x}_{n}) \) does not converge, we say \( ({x}_{n}) \) \textbf{diverges}.
\end{definition}

Divergence can come in three different forms:
\begin{enumerate}
    \item[(i)] \( ({x}_{n}) \) becomes arbitrarily large as \( n \to \infty   \). More precisely, 
        \[  \forall M > 0 \ \exists N \in \N \ \text{such that} \ \forall n > N \ {x}_{n} > M.  \]
        From this, we say that \( {x}_{n} \to \infty   \) or \( \lim_{ n \to \infty  }  {x}_{n} = \infty  \).
    \item[(ii)] \( (-{x}_{n}) \) becomes arbitrarily large as \( n \to \infty   \). More precisely, 
        \[  \forall M > 0 \ \exists N \in \N \ \text{such that} \ \forall n > N \ \underbrace{-{x}_{n} > M}_{{x}_{n} < -M}.  \]
        From this, we say that \( {x}_{n} \to - \infty  \) or \( \lim_{ n \to \infty  }  {x}_{n} = - \infty  \).
    \item[(iii)] \( ({x}_{n}) \) is NOT convergent and it does NOT diverge to \( \infty  \) nor \( - \infty   \). For example, 
        \[  ({x}_{n}) = ((-1)^{n}) = (-1,+1,-1,+1, \dots) \]
        or
        \[  ({x}_{n}) = (n (-1)^{n}) = (-1,2, -3, 4, -5, 6, \dots). \]
\end{enumerate}

\begin{remark}[Strategies to Prove Divergence]
    Let \( ({a}_{n}) \) be a sequence of real numbers.
    \begin{enumerate}
        \item[(1)] If \( ({a}_{n}) \) is unbounded, then \( ({a}_{n}) \) diverges.
        \item[(2)] If \( ({a}_{n}) \) has a pair of subsequences converging to different limits, then \( ({a}_{n}) \) diverges.
        \item[(3)] Let \( a \in \R  \) be fixed but arbitrary. Show that the assumption \( {a}_{n} \to a  \) leads to a contradiction.
    \end{enumerate}
\end{remark}

\begin{definition}[Increasing, Decreasing, Monotone]
    Consider \( \R  \) with the standard metric.
    \begin{enumerate}
        \item[(i)] \( ({a}_{n}) \) is said to be \textbf{increasing} if for all \( n \in \N  \), \( {a}_{n} \leq {a}_{n+1}  \).
        \item[(ii)] \( ({a}_{n}) \) is said to be \textbf{decreasing} if for all \( n \in \N  \), \( {a}_{n+1} \leq {a}_{n} \).
        \item[(iii)] \( ({a}_{n}) \) is said to be \textbf{monotone} if it is either increasing or decreasing or both.
        \item[(iv)] \( ({a}_{n}) \) is said to be \textbf{strictly increasing} if for all \( n \in \N  \), \( {a}_{n} < {a}_{n+1}.  \)
        \item[(v)] \( ({a}_{n})  \) is said to be \textbf{strictly decreasing} if for all \( n \in \N  \), \( {a}_{n+1} < {a}_{n} \).
    \end{enumerate}
\end{definition}

\begin{eg}
    \begin{enumerate}
        \item[(i)] \( 1,-1,1,-1,1,-1, \dots \) is neither an increasing nor decreasing sequence.
        \item[(ii)] \( 2,4,6,8,10, \dots \) is strictly increasing.
        \item[(iii)] \( 6,6,7,7,8,8, \dots \) is an increasing sequence but not strictly increasing.
        \item[(iv)] \( -2,-4,-6,-8,-10, -12, \dots \) is a strictly decreasing sequence.
    \end{enumerate}
\end{eg}

\begin{theorem}[Monotone Convergence Theorem]
    Consider \( \R  \) with its standard metric. 
    \begin{enumerate}
        \item[(i)] If \( ({a}_{n}) \) is increasing and bounded, then \( ({a}_{n}) \) converges to \( \sup \{ {a}_{n} : n \in \N \}  \).
        \item[(ii)] If \( ({a}_{n}) \) is decreasing and bounded, then \( ({a}_{n}) \) converges to \( \inf \{ {a}_{n} : n \in \N \}  \).
        \item[(iii)] If \( ({a}_{n}) \) is increasing and unbounded, then \( {a}_{n} \to \infty  \).
        \item[(iv)] If \( ({a}_{n}) \) is decreasing and unbounded, then \( {a}_{n} \to - \infty  \).
    \end{enumerate}
\end{theorem}
\begin{proof}
Here will prove item (i). The rest of the parts follow analogously.

Suppose that \( ({a}_{n}) \) is increasing (\( \forall n \  {a}_{n} \leq {a}_{n+1} \)) and \( ({a}_{n}) \) is a bounded sequence. Our goal is to show that \( {a}_{n} \to s  \) where
\[  s = \sup \{ {a}_{n} : n \in \N \}.  \]
Notice that since \( \{ {a}_{n} : n \in \N \}  \) is a bounded set, we know that \( \{ {a}_{n} : n \in \N \}  \) must exists in \( \R \). Our goal is to show that  
\[  \forall \epsilon > 0 \ \exists N \in \N \ \text{such that} \ \forall n > N \ | {a}_{n} - s  |  < \epsilon. \tag{*} \]
Let \( \epsilon > 0  \) be given. To show (*), it suffices to show that 
\begin{center}
    if \( n > N  \), then \( s - \epsilon < {a}_{n} < s + \epsilon \).
\end{center}
Since \( s = \sup \{ {a}_{n} : n \in \N  \}  \), we see that \( s - \epsilon  \) must NOT be an upper bound of \( \{ {a}_{n} : n \in \N  \}  \). Hence, there exists an element in \( \{ {a}_{n} : n \in \N \}  \) that is larger than \( s - \epsilon  \). That is, there exists \( \hat{N} \in \N \) such that \( {a}_{\hat{N}} > s - \epsilon  \). We claim that this \( \hat{N} \) can be used as the \( N  \) that were looking for. Indeed, if we let \( N = \hat{N} \), then 
\begin{enumerate}
    \item[(1)] if \( n > \hat{N} \), then \( {a}_{n} \geq {a}_{\hat{N}} > s - \epsilon \)
    \item[(2)] if \( n > \hat{N} \), then \( {a}_{n} \leq \sup \{ {a}_{n} : n \in \N  \}  = s < s + \epsilon. \)
\end{enumerate}
Thus, (1) and (2) imply that if \( n > \hat{N} \), then \( s - \epsilon < {a}_{n} < s + \epsilon \) as desired.
\end{proof}

\begin{remark}
    Consider \( \R  \) with its standard metric.
    \begin{enumerate}
        \item[(*)] Boundedness does not generally imply convergence. An example of such a bounded sequence that does not converge is \( {a}_{n} = (-1)^{n} \).
        \item[(*)] Monotone does not necessarily imply convergence. An example of such a sequence is \( {a}_{n} = n  \).
        \item[(*)] Convergence does not generally imply being monotone; that is, consider \( {a}_{n} = \frac{ (-1)^{n} }{ n }  \) which converges to \( 0  \) but does not have a monotone behavior.
    \end{enumerate}
\end{remark}

\begin{eg}
   Define the sequence \( ({a}_{n}) \) recursively by \( {a}_{1} = 1  \) and  
   \[  {a}_{n+1} = \frac{ 1 }{ 2 } {a}_{n} + 1.  \]
   \begin{enumerate}
       \item[(i)] Use induction to show that \( {a}_{n} \leq 2  \) for every \( n  \). We claim that \( \forall n \geq 1  \), \( {a}_{n} \leq 2  \). Let our base case be \( n = 1   \). Then \( {a}_{1} =1  \leq 2  \). Suppose for our induction hypothesis that the claim holds for \( n = k (k \geq 1)  \), that is, suppose that \( {a}_{k } \leq 2  \). Our goal is to show that the claim holds for \( n = k + 1  \), that is, prove that \( {a}_{k+1} \leq 2  \). Hence, we have
           \[  {a}_{k+1} = \frac{ 1 }{ 2 }  {a}_{k } + 1 \underbrace{\leq}_{{a}_{k} \leq 2} \frac{ 1 }{ 2 }(2) + 1  = 1 + 1 = 2.  \]
       \item[(ii)] Use induction to show that \( ({a}_{n}) \) is an increasing sequence. Our claim that for all \( n  \), \( {a}_{n} \leq {a}_{n+1} \). Let our base case be \( n = 1  \). Then \( {a}_{1} = 1  \) and 
           \[ {a}_{2} = \frac{ 1 }{ 2 }  {a}_{1} + 1 = \frac{ 1 }{ 2 }(1) + 1 = \frac{ 3 }{ 2 }.    \]
           Hence, this implies that \( {a}_{1} \leq {a}_{2} \). Suppose that the claim for \( n = k + 1 \), that is, prove that \( {a}_{k+1} \leq {a}_{k+2} \). Now, we have 
           \begin{align*}
               {a}_{k } \leq {a}_{k+1} &\implies \frac{ 1 }{ 2 }  {a}_{k } \leq \frac{ 1 }{ 2 }  {a}_{k+1}  \\
                                       &\implies \frac{ 1 }{ 2 }  {a}_{k } + 1 \leq \frac{ 1 }{ 2 }  {a}_{k+1} + 1 \\
                                       &\implies {a}_{k+1} \leq {a}_{k+2}.
       \end{align*}
    \item[(iii)] We will show that (i) and (ii) will imply that \( ({a}_{n}) \) converges. We have 
        \begin{enumerate}
            \item[(1)] \( \forall n  \), \( 1 = {a}_{1} \leq {a}_{n} \leq 2 \implies ({a}_{n})   \) is bounded,
            \item[(2)] \( ({a}_{n}) \) is increasing.
        \end{enumerate}
        By the Monotone Convergence Theorem, we see that \( ({a}_{n}) \) converges.
    \item[(iv)] Now, we will show that \( \lim_{ n \to \infty  }  {a}_{n} = 2 \). Let \( A = \lim_{ n \to \infty  }  {a}_{n} \). We have
        \begin{align*}
            \lim_{ n \to \infty  }  {a}_{n+1} = \lim_{ n \to \infty  }  \Big[ \frac{ 1 }{ 2 }  {a}_{n} + 1\Big] &\implies \lim_{ n \to \infty  }  {a}_{n+1} = \frac{ 1 }{ 2 }  (\lim_{ n \to \infty  }  {a}_{n}) + 1 \\
                                                                                                                &\implies A = \frac{ 1 }{ 2 } A + 1 \\
                                                                                                                &\implies \frac{ 1 }{ 2 } A = 1 \\
                                                                                                                &\implies A = 2.
        \end{align*}
   \end{enumerate}
\end{eg}

\subsection{Extended Real Numbers}

The set of extended real number, denoted by \( \overline{\R} \), consists of all real numbers and two symbols \( +\infty, - \infty  \):
\[  \overline{\R} = \R \cup \{ - \infty  , + \infty  \}.  \]
\begin{enumerate}
    \item[(*)] \( \overline{\R} \) is equipped with an order. We preserve the original order in \( \R  \) and we define
        \[  \forall x \in \R  \ \ -\infty  < x < \infty. \]
    \item[(*)] \( \overline{\R} \) is not a field, but it is customary to make the following conventions:
    \begin{align*}
        &\forall x \in \overline{\R} \ \text{with} \ x > 0 \ x \cdot (+ \infty ) = + \infty \ \ x \cdot (- \infty ) = - \infty    \\
        &\forall x \in \overline{\R} \ \text{with} \ x < 0, x \cdot (+ \infty ) = - \infty  \ x \cdot (-\infty ) = +\infty \\ 
        &\forall x \in \R, \  x + \infty  = + \infty \\
        &\forall x \in \R,  \ x - \infty  = - \infty.
    \end{align*}
    As for the last convention, we define
    \[ +  \infty  + \infty = + \infty  \ \ - \infty  - \infty  = - \infty.   \]
    Furthermore, we have 
    \[  \forall x \in \R \ \ \frac{ x  }{  + \infty  }  = 0 \ \ \frac{ x  }{  - \infty   }  = 0.  \]
    Please notice that we did \underline{not} define the following
    \[  - \infty  + \infty, + \infty  - \infty , \frac{ \infty    }{  \infty   } , \dots , 0 \cdot \infty  , \infty  \cdot 0 , 0 \cdot - \infty , - \infty  \cdot 0.  \]
\item[(*)] \( \sup A = \infty  \iff  \) either \( \infty  \in A  \) or \(  A \subseteq  \R \cup \{ - \infty \}    \) and \( A  \) is not bounded above in \( \R \cup \{ - \infty   \}  \). 
\item[(*)] \( \inf A = - \infty \iff   \) \( - \infty   \) or \( A \subseteq \R \cup \{  \infty   \}  \) and \( A  \) is not bounded below in \( \R \cup \{ \infty  \}  \). 
\item[(*)] \( \sup \emptyset = - \infty  \) and \( \inf \emptyset = + \infty  \).
\end{enumerate}
\begin{remark}
    Let \( ({a}_{n}) \) bea  sequence in \( \overline{\R} \). Let \( a \in \R  \).
    \begin{enumerate}
        \item[(i)] \( \lim_{ n \to \infty  }  {a}_{n} = a \iff \forall \epsilon > 0 \ \exists N \in \N \ \text{such that} \ \forall n > N \ | {a}_{n}- a   | < \epsilon \) \\
        \item[(ii)] \( \lim_{ n \to \infty  }  {a}_{n} = \infty \iff \forall M > 0 \ \exists N \in \N \ \text{such that} \ \forall n > N \ {a}_{n} > M  \) 
        \item[(iii)] \( \lim_{ n \to \infty  }  {a}_{n} = - \infty \iff \forall M > 0 \ \exists N \in \N \ \text{such that} \ \forall n > N \ - {a}_{n} > M  \).
    \end{enumerate}
\end{remark}

\begin{theorem}[Algebraic Limit Theorem for \( \overline{\R} \)]
    Suppose \( {a}_{n} \to a  \) in \( \overline{\R} \) and \( {b}_{n} \to b  \) in \( \overline{\R} \). Then
    \begin{enumerate}
        \item[(i)] If \( c \in \R  \), then \( {ca}_{n} \to ca \)
        \item[(ii)] \( {a}_{n} + {b}_{n} \to a + b \) (provided that \( \infty  - \infty   \) does not appear)
        \item[(iii)] \( {a}_{n}{b}_{n} \to ab \) (provided that \( (\pm \infty ) \cdot 0  \) or \( 0 \cdot (\pm \infty)  \) do not appear).
        \item[(iv)] If \( a = \infty   \), then \( \frac{ 1 }{ {a}_{n} }  \to 0  \). If \( a = - \infty  \), then \( \frac{ 1 }{ {a}_{n}  } \to 0  \).
        \item[(v)] If \( {a}_{n} \to 0  \) and \( {a}_{n} > 0  \), then \( \frac{ 1 }{ {a}_{n} }  \to \infty  \). If \( {a}_{n} \to 0  \) and \( {a}_{n} < 0  \), then \( \frac{ 1 }{ {a}_{n} } \to - \infty  \).
    \end{enumerate}
\end{theorem}

\begin{theorem}[Order Limit Theorem for \( \overline{\R} \)]
    Suppose \( {a}_{n} \to a  \) in \( \overline{\R} \) and \( {b}_{n} \to b  \) in \( \overline{\R} \). Then
    \begin{enumerate}
        \item[(i)] If \( {a}_{n} \leq {b}_{n}  \), then \( a \leq b\).
        \item[(ii)] If \( {a}_{n} \leq {e}_{n} \) and \( {a}_{n} \to \infty   \), then \( {e}_{n} \to \infty   \).
        \item[(iii)] If \( {e}_{n} \leq {a}_{n}  \) and \( {a}_{n} \to - \infty  \), then \( {e}_{n} \to - \infty  \).
    \end{enumerate}
\end{theorem}

\begin{theorem}[Monotone Convergence Theorem in \( \overline{\R} \)]
    Let \( ({a}_{n}) \) be a sequence in \( \overline{\R} \).
    \begin{enumerate}
        \item[(i)] If \( ({a}_{n}) \) is increasing, then \( {a}_{n} \to \sup \{ {a}_{n} : n \in \N \}  \).
        \item[(ii)] If \( ({a}_{n})  \) is decreasing, then \( {a}_{n} \to \inf \{  {a}_{n} : n \in \N \}  \).
    \end{enumerate}
\end{theorem}

Note that if \( {a}_{n} \to \infty   \), then \( \frac{ 1 }{ {a}_{n}  }  \to 0  \), however, \( \frac{ 1 }{ {a}_{n} }  \to 0  \) does not imply that \( {a}_{n} \to \infty  \). For example, consider \( ({a}_{n}) = (-n) \) or \( ({a}_{n}) = ((-1)^{n} n) \).

\begin{remark}
    \begin{enumerate}
        \item[(i)] \( \overline{\R} \) can be equipped with the following metric:
    \begin{center}
        Let \( f: \overline{\R} \to [\frac{ -\pi }{ 2 } , \frac{ \pi }{ 2 } ] \) be defined by 
        \[  f(x) = 
        \begin{cases}
            \frac{ -\pi }{ 2 }  &x = \infty  \\
            \arctan(x) &-\infty < x < \infty  \\ 
            \frac{ \pi }{ 2 }  &x = \infty 
        \end{cases}  \]
    \end{center}
    Define \( \overline{d}(x,y) = | f(x) - f(y) | \ \forall x,y \in \overline{\R}  \). The closure of \( \R  \) in \( (\overline{\R}, \overline{d}) \) is \( \overline{\R} \).
        \item[(ii)] One can show that if \( ({a}_{n}) \) is a sequence in \( \R  \), then
            \[  {a}_{n} \to a \in \overline{\R} \iff \ \text{The sequence \( {a}_{n} \) converges to \( a  \) in the metric space \( (\overline{\R}, \overline{d}) \).} \]
        \item[(iii)] The closure of \( \R  \) in the metric space \( (\overline{\R}, \overline{d}) \) is \( \overline{\R} \).
        \item[(iv)] Every set in \( (\overline{\R}, \overline{d}) \) is bounded:
            \[ \forall x,y \in \overline{\R} \ \ \overline{d}(x,y) \leq \pi. \]
    \end{enumerate}
\end{remark}
\end{document}


