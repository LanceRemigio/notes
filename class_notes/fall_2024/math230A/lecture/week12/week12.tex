\documentclass[a4paper]{article}
\usepackage[utf8]{inputenc}
\usepackage[T1]{fontenc}
\usepackage{textcomp}
\usepackage{hyperref}
% \usepackage{fourier}
% \usepackage[dutch]{babel}
\usepackage{url}
% \usepackage{hyperref}
% \hypersetup{
%     colorlinks,
%     linkcolor={black},
%     citecolor={black},
%     urlcolor={blue!80!black}
% }
\usepackage{graphicx}
\usepackage{float}
\usepackage{booktabs}
\usepackage{enumitem}
% \usepackage{parskip}
\usepackage{emptypage}
\usepackage{subcaption}
\usepackage{multicol}
\usepackage[usenames,dvipsnames]{xcolor}

% \usepackage{cmbright}


\usepackage[margin=1in]{geometry}
\usepackage{amsmath, amsfonts, mathtools, amsthm, amssymb}
\usepackage{mathrsfs}
\usepackage{cancel}
\usepackage{bm}
\newcommand\N{\ensuremath{\mathbb{N}}}
\newcommand\R{\ensuremath{\mathbb{R}}}
\newcommand\Z{\ensuremath{\mathbb{Z}}}
\renewcommand\O{\ensuremath{\emptyset}}
\newcommand\Q{\ensuremath{\mathbb{Q}}}
\newcommand\C{\ensuremath{\mathbb{C}}}
\DeclareMathOperator{\sgn}{sgn}
\usepackage{systeme}
\let\svlim\lim\def\lim{\svlim\limits}
\let\implies\Rightarrow
\let\impliedby\Leftarrow
\let\iff\Leftrightarrow
\let\epsilon\varepsilon
\usepackage{stmaryrd} % for \lightning
\newcommand\contra{\scalebox{1.1}{$\lightning$}}
% \let\phi\varphi
\renewcommand\qedsymbol{$\blacksquare$}




% correct
\definecolor{correct}{HTML}{009900}
\newcommand\correct[2]{\ensuremath{\:}{\color{red}{#1}}\ensuremath{\to }{\color{correct}{#2}}\ensuremath{\:}}
\newcommand\green[1]{{\color{correct}{#1}}}



% horizontal rule
\newcommand\hr{
    \noindent\rule[0.5ex]{\linewidth}{0.5pt}
}


% hide parts
\newcommand\hide[1]{}



% si unitx
\usepackage{siunitx}
\sisetup{locale = FR}
% \renewcommand\vec[1]{\mathbf{#1}}
\newcommand\mat[1]{\mathbf{#1}}


% tikz
\usepackage{tikz}
\usepackage{tikz-cd}
\usetikzlibrary{intersections, angles, quotes, calc, positioning}
\usetikzlibrary{arrows.meta}
\usepackage{pgfplots}
\pgfplotsset{compat=1.13}


\tikzset{
    force/.style={thick, {Circle[length=2pt]}-stealth, shorten <=-1pt}
}

% theorems
\makeatother
\usepackage{thmtools}
\usepackage[framemethod=TikZ]{mdframed}
\mdfsetup{skipabove=1em,skipbelow=0em}


\theoremstyle{definition}

\declaretheoremstyle[
    headfont=\bfseries\sffamily\color{ForestGreen!70!black}, bodyfont=\normalfont,
    mdframed={
        linewidth=2pt,
        rightline=false, topline=false, bottomline=false,
        linecolor=ForestGreen, backgroundcolor=ForestGreen!5,
    }
]{thmgreenbox}

\declaretheoremstyle[
    headfont=\bfseries\sffamily\color{NavyBlue!70!black}, bodyfont=\normalfont,
    mdframed={
        linewidth=2pt,
        rightline=false, topline=false, bottomline=false,
        linecolor=NavyBlue, backgroundcolor=NavyBlue!5,
    }
]{thmbluebox}

\declaretheoremstyle[
    headfont=\bfseries\sffamily\color{NavyBlue!70!black}, bodyfont=\normalfont,
    mdframed={
        linewidth=2pt,
        rightline=false, topline=false, bottomline=false,
        linecolor=NavyBlue
    }
]{thmblueline}

\declaretheoremstyle[
    headfont=\bfseries\sffamily\color{RawSienna!70!black}, bodyfont=\normalfont,
    mdframed={
        linewidth=2pt,
        rightline=false, topline=false, bottomline=false,
        linecolor=RawSienna, backgroundcolor=RawSienna!5,
    }
]{thmredbox}

\declaretheoremstyle[
    headfont=\bfseries\sffamily\color{RawSienna!70!black}, bodyfont=\normalfont,
    numbered=no,
    mdframed={
        linewidth=2pt,
        rightline=false, topline=false, bottomline=false,
        linecolor=RawSienna, backgroundcolor=RawSienna!1,
    },
    qed=\qedsymbol
]{thmproofbox}

\declaretheoremstyle[
    headfont=\bfseries\sffamily\color{NavyBlue!70!black}, bodyfont=\normalfont,
    numbered=no,
    mdframed={
        linewidth=2pt,
        rightline=false, topline=false, bottomline=false,
        linecolor=NavyBlue, backgroundcolor=NavyBlue!1,
    },
]{thmexplanationbox}

\declaretheorem[style=thmgreenbox, numberwithin = section, name=Definition]{definition}
\declaretheorem[style=thmbluebox, name=Example]{eg}
\declaretheorem[style=thmredbox, numberwithin = section, name=Proposition]{prop}
\declaretheorem[style=thmredbox, numberwithin = section, name=Theorem]{theorem}
\declaretheorem[style=thmredbox, numberwithin = section,  name=Lemma]{lemma}
\declaretheorem[style=thmredbox, numberwithin = section,  numbered=no, name=Corollary]{corollary}


\declaretheorem[style=thmproofbox, name=Proof]{replacementproof}
\renewenvironment{proof}[1][\proofname]{\vspace{-10pt}\begin{replacementproof}}{\end{replacementproof}}


\declaretheorem[style=thmexplanationbox, name=Proof]{tmpexplanation}
\newenvironment{explanation}[1][]{\vspace{-10pt}\begin{tmpexplanation}}{\end{tmpexplanation}}


\declaretheorem[style=thmblueline, numbered=no, name=Remark]{remark}
\declaretheorem[style=thmblueline, numbered=no, name=Note]{note}

\newtheorem*{uovt}{UOVT}
\newtheorem*{notation}{Notation}
\newtheorem*{previouslyseen}{As previously seen}
\newtheorem*{problem}{Problem}
\newtheorem*{observe}{Observe}
\newtheorem*{property}{Property}
\newtheorem*{intuition}{Intuition}


\usepackage{etoolbox}
\AtEndEnvironment{vb}{\null\hfill$\diamond$}%
\AtEndEnvironment{intermezzo}{\null\hfill$\diamond$}%
% \AtEndEnvironment{opmerking}{\null\hfill$\diamond$}%

% http://tex.stackexchange.com/questions/22119/how-can-i-change-the-spacing-before-theorems-with-amsthm
\makeatletter
% \def\thm@space@setup{%
%   \thm@preskip=\parskip \thm@postskip=0pt
% }
\newcommand{\oefening}[1]{%
    \def\@oefening{#1}%
    \subsection*{Oefening #1}
}

\newcommand{\suboefening}[1]{%
    \subsubsection*{Oefening \@oefening.#1}
}

\newcommand{\exercise}[1]{%
    \def\@exercise{#1}%
    \subsection*{Exercise #1}
}

\newcommand{\subexercise}[1]{%
    \subsubsection*{Exercise \@exercise.#1}
}


\usepackage{xifthen}

\def\testdateparts#1{\dateparts#1\relax}
\def\dateparts#1 #2 #3 #4 #5\relax{
    \marginpar{\small\textsf{\mbox{#1 #2 #3 #5}}}
}

\def\@lesson{}%
\newcommand{\lesson}[3]{
    \ifthenelse{\isempty{#3}}{%
        \def\@lesson{Lecture #1}%
    }{%
        \def\@lesson{Lecture #1: #3}%
    }%
    \subsection*{\@lesson}
    \testdateparts{#2}
}

% \renewcommand\date[1]{\marginpar{#1}}


% fancy headers
\usepackage{fancyhdr}
\pagestyle{fancy}

\fancyhead[LE,RO]{Lance Remigio}
\fancyhead[RO,LE]{\@lesson}
\fancyhead[RE,LO]{}
\fancyfoot[LE,RO]{\thepage}
\fancyfoot[C]{\leftmark}

\makeatother




% notes
\usepackage{todonotes}
\usepackage{tcolorbox}

\tcbuselibrary{breakable}
\newenvironment{verbetering}{\begin{tcolorbox}[
    arc=0mm,
    colback=white,
    colframe=green!60!black,
    title=Opmerking,
    fonttitle=\sffamily,
    breakable
]}{\end{tcolorbox}}

\newenvironment{noot}[1]{\begin{tcolorbox}[
    arc=0mm,
    colback=white,
    colframe=white!60!black,
    title=#1,
    fonttitle=\sffamily,
    breakable
]}{\end{tcolorbox}}




% figure support
\usepackage{import}
\usepackage{xifthen}
\pdfminorversion=7
\usepackage{pdfpages}
\usepackage{transparent}
\newcommand{\incfig}[1]{%
    \def\svgwidth{\columnwidth}
    \import{./figures/}{#1.pdf_tex}
}

% %http://tex.stackexchange.com/questions/76273/multiple-pdfs-with-page-group-included-in-a-single-page-warning
\pdfsuppresswarningpagegroup=1




\begin{document}

\section{Topics}

\begin{itemize}
    \item {\hyperref[Limits and Continuity]{Limits and Continuity}} 
    \item {\hyperref[Limits of Functions]{Limits of Functions}}
    \item {\hyperref[Sequential Criterion for Limits of Functions]{Sequential Criterion for Limits of Functions}}
    \item {\hyperref[Algebraic Limit Theorem]{Algebraic Limit Theorem}}
    \item {\hyperref[Continuous Functions]{Continuous Functions}}
    \item {\hyperref[Characterization of Continuity via Sequences]{Characterization of Continuity via Sequences}}
    \item {\hyperref[Algebraic Continuity Theorem]{Algebraic Continuity Theorem}}
    \item {\hyperref[A Composition of Continuous Functions is Continuous]{A Composition of Continuous is Continuous}}
\end{itemize}

\section{Limits and Continuity}\label{Limits and Continuity}

One of the most important concepts in Calculus is the limit of a function. Our goal is to understand what is meant by:
\begin{enumerate}
    \item[(1)] \( \lim_{ x \to c }  f(x) = L \).
    \item[(2)] "\( f(x) \) is continuous at \( c \)"
\end{enumerate}

\section{Limits of Functions}\label{Limits of Functions}

The usual set up for discussing continuity of functions is that we need to have  
\begin{enumerate}
    \item[(1)] Two metric spaces, \( (X,d) \) and \( (Y, \tilde{d}) \), for the domain and codomain of the function, respectively.
    \item[(2)] Let \( E  \) be some nonempty subset of \( X  \).
    \item[(3)] Let \( c  \) be some limit point of \( E  \); that is, \( c \in E' \).
    \item[(4)] Let \( f: E \to Y \).
\end{enumerate}

Recall that \( c \in E' \) means that there exists a sequence in \( E \setminus  \{ c  \}  \) such that \( {x}_{n} \to c  \). In particular, \( c  \) does not necessarily belong to \( E  \). Also, it is not the case that every point in \( E  \) necessarily belongs to \( E' \).

Some common descriptions made about limits of functions can be made using the following statements:

\begin{itemize}
    \item \( f(x)  \) approaches \( L  \) as \( x  \) approaches \( c  \) (\( x \in E  \) and \( x \neq c  \)).
    \item For every \( \epsilon > 0  \), no matter how small it is, \( \tilde{d}(f(x), L) < \epsilon  \) provided that \( d(x,c) \) is sufficiently small (i.e \( x \in E  \) and \( x \neq c \)).
\end{itemize}

\begin{definition}[Limit of a Function]
    Let \( (X,d) \) and \( (Y,\tilde{d}) \) be metric spaces. Let \( \emptyset \neq E \subseteq X  \). Let \( c \in E' \). Let \( f: E \to Y \). We say that \( \lim_{ x  \to c  }  f(x) = L  \) if 
    \begin{center}
        \( \forall \epsilon > 0  \), \( \exists \delta > 0  \) such that if \( 0 < d(x,c) < \delta  \) (with \( x \in E  \)), then \( \tilde{d}(f(x), L) < \epsilon. \)
    \end{center}
\end{definition}

\begin{remark}
    Just like in our definition of sequential limits and how our position \( N \)is dependent upon our choice of \( \epsilon  \), we see a similar notion here when it comes to our \( \delta > 0  \) being dependent upon our choice of \( \epsilon  \).
\end{remark}

The reason we require \( x \in E  \) is to ensure that \( f(x)  \) makes sense. That is, we may not need to state the condition that \( x \in E  \) provided that there is no confusion about the setup. 

Furthermore, we require the condition that \( {d}_{x}(x,c) > 0  \) because we need only concern ourselves with what happens as the points of our domain approach the limit point \( c  \), not necessarily what happens at the point \( c  \). In this way, we can see that \( f(c) \) could be changed to anything (or could not even exist) and the limit would still be unaffected.

Here are some equivalent statements we can make about limits of functions

\begin{enumerate}
    \item[(1)] \( \lim_{ x  \to c  } f(x) = L  \)
    \item[(2)] \( \forall \epsilon > 0  \), \( \exists \delta > 0  \) such that if \( 0 < d(x,c) < \delta \) (with \( c \in E  \)), then \( \tilde{d}(f(x), L) < \epsilon \).
    \item[(3)] \( \forall \epsilon > 0  \), \( \exists > 0  \) such that \( \forall x \in E \setminus  \{ c  \}  \) satisfying \( d(x,c) < \delta \), we have \( \tilde{d}(f(x), L) < \epsilon \).
    \item[(4)] \( \forall \epsilon > 0  \), \( \exists \delta > 0  \) such that \( \forall x \in ({N}_{\delta}^{X}(c) \cap (E \setminus  \{ c \} )) \), \( \tilde{d}(f(x), L) < \epsilon \).
    \item[(5)] \( \forall \epsilon> 0  \), \( \exists \delta > 0  \), \( \forall x \in ({N}_{\delta}^{X}(c) \cap (E \setminus  \{ c \} )) \), \( f(x) \in {N}_{\epsilon}^{Y}(L) \).
    \item[(6)] Given any \( \epsilon- \)neighborhood \( {N}_{\epsilon}^{Y}(L) \) of \( L \), there exists a \( \delta- \)neighborhood \( {N}_{\delta}^{X}(c) \) of \( c  \) such that the image of the part of \( {N}_{\delta}^{C}(c) \) contained in \( E \setminus  \{ c \}  \) is contained in \( {N}_{\epsilon}^{Y}(L) \).
\end{enumerate}

Keep in mind that, given \( \epsilon > 0  \), the choice of \( \delta  \) is NOT unique. If \( \delta = r  \) where (\( r  \) is a fixed number) works, then any positive number less than \( r  \) will also work. That is, given any \( \epsilon > 0  \), we can always assume that the \( \delta  \) we are looking for is less than \( 1  \).

\begin{eg}
    Let \( f: \R \to \R  \) be defined by \( f(x) = 2x + 5  \) for all \( x \in \R  \). Prove that \( \lim_{ x \to 3  } f(x) = 11 \).

    Our goal is to show that for all \( \epsilon > 0  \), there exists \( \delta > 0  \) such that if \( 0 < | x - 3  |  < \delta \), then \( | f(x) - 11  |  < \epsilon  \). Let \( \delta = \frac{ \epsilon  }{  2  }   \). For any \( x  \) with \( 0 < |  x - 3  |  < \delta  \), we have   
    \[  | f(x) - 11  |  =  |  (2x +5) - 11  |  = 2 | x - 3  | < 2 \delta = 2 \Big(  \frac{ \epsilon }{ 2 }  \Big) = \epsilon \]
    as desired.
\end{eg}

\begin{eg}
    Let \( f: \R \to \R  \) by \( f(x) = x^{2} \) for all \( x \in \R  \). Prove that \( \lim_{ x \to 2 }  f(x) = 4 \). 

    Our goal is to show that for all \( \epsilon > 0  \), there exists \( \delta > 0  \) such that if \( 0 < | x - 2  | < \delta \), then \( | f(x) - 4  |  < \epsilon  \). Let \( \epsilon > 0  \) be given. Recall that, we may assume that the \( \delta  \) we are looking for is not greater than \( 1  \). Let \( \delta = \min \{ 1 , \frac{ 3 }{ 5 }  \}  \). For any \( x  \) with \( 0 < | x - c  | < \delta \), we have 
    \[  | f(x) - 4  |   = | x^{2} - 4  |  = | x - 2  |  | x + 2  | < 5 | x - 2  |  < 5 \Big( \frac{ \epsilon  }{ 5  } \Big)  = \epsilon  \]
    as desired.
\end{eg}

\begin{eg}[Limits with Discrete Metric]
    Let \( f: \R \to (\R, \tilde{d}) \) defined by \( f(x) = x^{2} \). Prove that \( \lim_{ x \to 2 } f(x)  \) does not exist. 
\end{eg}
\begin{proof}
Assume for sake of contradiction that, \( \lim_{ x \to 2 }  f(x) =  L \). So, 
\begin{center}
    \( \forall \epsilon > 0  \), \( \exists \delta > 0  \) such that if \(  0 < |  x - 2  |  < \delta  \), then \( \tilde{d}(f(x), L) < \epsilon \).
\end{center}
In particular, for \( \epsilon = \frac{ 1 }{ 2 }  \), there exists \( \delta > 0  \) such that if \( 0 < | x - 2  |  < \delta  \), then \( \tilde{d}(f(x), L) < \frac{ 1 }{ 2 }  \). Note that 
\[  \tilde{d}(f(x), L) < \frac{ 1 }{ 2 } \implies \tilde{d}(f(x), L) = 0 \implies f(x) = L.  \]
Hence, there exists \( \delta > 0  \) such that if \( 2 - \delta < x < 2 + \delta  \), then \( x^{2} = L  \). Obviously, it is not the case that for all \( x \in (2-  \delta, 2 + \delta) \) that \( x^{2}  \) is equal to the fixed number \( 1  \). 
\end{proof}

\section{Sequential Criterion for Limits of Functions}\label{Sequential Criterion for Limits of Functions}

\begin{theorem}[Sequential Criterion for Limits of Functions]
    Let \( (X,d)  \) and \( (Y, \tilde{d}) \) be metric spaces, let \( E  \) be a nonempty set in \( X  \), let \( c \in E' \), and \( f: E \to Y \). The following statements are equivalent:
    \begin{enumerate}
        \item[(i)] \( \lim_{ x \to c } f(x)= L  \).
        \item[(ii)] For all sequences \( ({a}_{n}) \) in \( E \setminus  \{ c  \}  \) satisfying \( {a}_{n} \to c  \), we have \( f({a}_{n}) \to L  \).
    \end{enumerate}
\end{theorem}
\begin{proof}
    (\( (i) \implies (ii) \)) Let \( ({a}_{n}) \) be a sequence in \( E \setminus  \{ c  \}  \) such that \( {a}_{n} \to c  \). Our goal is to show that \( f({a}_{n}) \to L  \). That is, we want to show that 
    \[  \forall \epsilon > 0 \ \exists N \in \N \ \text{such that} \ \forall n > N \ \tilde{d}(f({a}_{n}), L) < \epsilon. \tag{*} \]
    Let \( \epsilon > 0  \) be given. Since \( \lim_{ x \to c }  f(x) = L  \), we see that 
    \[  \exists \delta > 0 \ \text{such that} \ \forall x \in {N}_{\delta}^{X}(c) \cap (E \setminus  \{ c \} ) \ f(x) \in {N}_{\epsilon}^{Y}(L). \tag{1} \]
    Since \( \lim {a}_{n} = c \), we also see that
    \[  \exists \hat{N} \in \N \ \text{such that} \ \forall n > \hat{N} \ {a}_{n} \in {N}_{\delta}^{X}(c). \tag{2} \]
    Our claim is the \( \hat{N} \) can be used as the same \( N  \) we were looking for. Indeed, if we let \( N = \hat{N} \) such that for any \( n > N  \), we can see that from (2) that \( {a}_{n} \in {N}_{\delta}^{X}(c) \) and \( {a}_{n} \in E \setminus \{ c  \}  \) imply that \( {a}_{n} \in {N}_{\delta}^{X}(c) \cap (E \setminus  \{ c  \} ) \). By part (1), we can see that \( f({a}_{n}) \in {N}_{\epsilon}^{Y}(L) \).

    (\( (ii) \implies (i) \)) Suppose for all sequences \( ({a}_{n}) \) in \( E \setminus  \{ c  \}  \) satisfying \( {a}_{n} \to c  \), we have \(  f({a}_{n}) \to L \). Our goal is to show that \( \lim_{ x \to c } f(x) = L  \). Suppose for sake of contradiction that \( \lim_{ x \to c  } \neq L    \); that is,
    \[  \exists \epsilon > 0 \ \text{such that} \ \forall \delta > 0 \ \exists x \in {N}_{\delta}^{X}(c) \cap (E \setminus \{  c \} ) \ \text{such that} \ f(x) \notin {N}_{\epsilon}^{Y}(L). \]
    So,
    \begin{align*}
        \delta = 1 \  &\exists {x}_{1} \in E \setminus \{ c \} \ \text{such that} \ d({x}_{1}, c) < 1 \ \text{but} \ \tilde{d}(f({x}_{1}) , L) \geq \epsilon  \\
        \delta = \frac{ 1 }{ 2 } \  &\exists {x}_{2} \in E \setminus \{ c \} \ \text{such that} \ d({x}_{2}, c) < 1 \ \text{but} \ \tilde{d}(f({x}_{2}) , L) \geq \epsilon  \\
        \delta = \frac{ 1 }{ 3 } \  &\exists {x}_{3} \in E \setminus \{ c \}  \ \text{satisfying} \ d({x}_{3},c) < \frac{ 1 }{ 3 }  \ \text{but} \ \tilde{d}(f({x}_{3}), L) \geq \epsilon \\
                                 &\vdots
    \end{align*}
    In this way we will obtain a sequence \( ({x}_{n}) \) in \( E \setminus \{ c  \}  \) such that \( {x}_{n} \to c  \) but for which \( \tilde{d}(f({x}_{n}),L) \geq \epsilon  \), and so \( f({x}_{n}) \not\to L \). This contradicts our assumption.

\end{proof}

\begin{remark}
    The previous theorem gives us a nice way of establishing that certain limits do NOT exist. If you can find two sequences \( ({a}_{n})  \) in \( E \setminus \{ c  \}  \) and \( ({b}_{n}) \) in \( E \setminus \{ c  \}  \) such that \( \lim {a}_{n} = c  \) and \( \lim {b}_{n} = c  \) but \( \lim f({a}_{n}) \neq \lim f({b}_{n}) \), then you can conclude that \( \lim_{ x \to c } f(x) \) does NOT exist.
\end{remark}

\begin{eg}
    Let \( f: \R \setminus \{  0 \} \to \R  \) be defined by \( f(x) = \sin \frac{ 1 }{ x }  \). Prove that \( \lim_{ x \to 0 } f(x) \) doe NOT exist.
\end{eg}
\begin{proof}
Let \( {a}_{n} = \frac{ 1 }{ 2n \pi }  \) and \( {b}_{n} = \frac{ 1 }{ 2n \pi + \frac{ \pi }{ 2 }  }  \). Clearly, \( ({a}_{n}) \) and \( ({b}_{n}) \) are sequences in \( \R \setminus \{ 0 \}  \) and \( \lim_{ n \to \infty  }  {a}_{n} = 0  \) and \( \lim_{ n \to \infty  }  {b}_{n} = 0 \). However,
\[ \lim_{ n \to \infty  }  f({a}_{n}) = \lim_{ n \to \infty  }  \sin \frac{ 1 }{ {a}_{n} }  = \lim_{ n \to \infty  }  \sin (2n \pi) = \lim_{ n \to \infty  } 0 = 0   \]
and
\[  \lim_{ n \to \infty  } f({b}_{n}) = \lim_{ n \to \infty  } \sin \frac{ 1 }{ {b}_{n} } = \lim_{ n \to \infty  }  \sin \Big(2n\pi + \frac{ \pi }{ 2 }  \Big) = \lim_{ n \to \infty  }  \sin \frac{ \pi }{ 2 }  = \lim_{ n \to \infty  } 1 = 1. \]
So, \( \lim f({a}_{n}) \neq \lim f({b}_{n}) \). Therefore, \( \lim_{ x \to 0 }  \sin \frac{ 1 }{ x }  \) does not exist.
\end{proof}

\section{Algebraic Limit Theorem}\label{Algebraic Limit Theorem}

\begin{theorem}[Algebraic Limit Theorem]
    Let \( (X,d) \) be a metric space. Let \( \emptyset \neq E \subseteq  X \). Let \( c \in E' \). Let \( f,g: E \to \R  \). Assume that 
    \[  \lim_{ x \to c } f(x) = L \ \text{and} \ \lim_{ x \to c }  g(x) = M. \]
    Then
    \begin{enumerate}
        \item[(i)] \( \forall k \in \R  \) \( \lim_{ x \to c } (kf(x)) =  kL \),
        \item[(ii)] \( \lim_{ x \to c } (f(x) + g(x))  = L + M\),
        \item[(iii)] \( \lim_{ x \to c } f(x)g(x) = LM \)
        \item[(iv)] \( \lim_{ x \to c } \frac{ f(x) }{ g(x) }  = \frac{ L }{ M }  \) provided that \( M \neq 0 \).
    \end{enumerate}
\end{theorem}
\begin{proof}
All these items follow immediately from the Algebraic Limit Theorem for sequences and the sequential criterion for limits of functions.
\end{proof}

\section{Continuous Functions}\label{Continuous Functions}

\begin{enumerate}
    \item[(1)] Let \( E \subseteq X  \). First, we will explain what it means to say that \( f: E \to Y \) is continuous at a point \( c \in E' \cap E  \).
    \item[(2)] Then we will explain what it means to say that \( f: E \to Y \) is continuous at a point \( c \in E  \). 
    \item[(3)] Finally, we will explain what it means to say that \( f: E \to Y \) is continuous.
\end{enumerate}

In what follows, we will present a three definitions of continuity that will best represent the three situations above.

\begin{definition}[Version 1 of Continuity]
    Let \( (X,d) \) and \( (Y,\tilde{d}) \) be metric spaces. Let \( E \subseteq X  \). Let \( c \in E' \). Let \( f: E \to Y \). We say that \( f  \) is continuous at \( c  \) if all the following three conditions hold:
    \begin{enumerate}
        \item[(1)] \( c \in E  \) (\( f \) is defined at \( c \)),
        \item[(2)] \( \lim_{ x \to c } f(x)  \) exists,
        \item[(3)] \( \lim_{ x \to c } f(x) = f(c) \).
    \end{enumerate}
\end{definition}

The definition above is mostly seen in introductory Calculus books. 

\begin{definition}[Version 2 of Continuity]
    Let \( (X,d) \) and \( (Y,\tilde{d}) \) be two metric spaces. Let \( E  \) be a nonempty set in \( X  \). Let \( c \in E  \). Let \( f: E \to Y \). We say \( f  \) is continuous at \( c  \) if any of the following equivalent statements holds:
\begin{enumerate}
    \item[(1)] \( \lim_{ x \to c } f(x) = f(c) \).
    \item[(2)] \( \forall \epsilon > 0 \), \( \exists \delta > 0 \) such that if \( d(x,c) < \delta \) (with \( x \in E  \)), then \( \tilde{d}(f(x), f(c)) < \epsilon \).
    \item[(3)] \( \forall \epsilon > 0 \), \( \exists \delta > 0  \) such that \( \forall x \in {N}_{\delta}^{X}(c) \cap E  \), \( f(x) \in {N}_{\epsilon}^{Y}(f(c)) \).
    \item[(4)] For every \( \epsilon- \)neighborhood \( {N}_{\epsilon}^{Y}(f(c))  \) of \( f(c) \), there exists a \( \delta- \)neighborhood \( {N}_{\delta}^{X}(c)  \) of \( c  \) such that the image of \( {N}_{\delta}^{X}(c) \cap E  \) is contained in \( {N}_{\epsilon}^{Y}(f(c)) \).
\end{enumerate}  
\end{definition}
 
\begin{remark}
    \begin{enumerate}
        \item[(i)] It is a direct consequence of definition \( 2  \) that if \( c  \) is an isolated point of \( E  \), then \( f  \) is continuous at \( c  \).
        \item[(ii)] As we discussed, if \( c \in E \cap E' \), then definition \( 2 \) is equivalent to saying that \( \lim_{ x \to c } f(x) = f(c) \).
    \end{enumerate}
\end{remark}

\begin{definition}[Version 3 of Continuity]
   Let \( (X,d)\) be a metric space and let \( E \subseteq X  \). Suppose \( f: E \to Y \). We say that \( f  \) is \textbf{continuous} if it is continuous at every point of \( E  \).
\end{definition}

\section{Characterization of Continuity via Sequences}\label{Characterization of Continuity via Sequences}

\begin{theorem}[Characterization of Continuity via Sequences]
    Let \( f: E \subseteq X \to Y \). Let \( c \in E  \). The following two statements are equivalent:
    \begin{enumerate}
        \item[(i)] \( f  \) is continuous at \( c  \)
        \item[(ii)] For all sequences \( ({a}_{n}) \) in \( E  \) satisfying \( {a}_{n} \to c  \), we have \( f({a}_{n}) \to f(c) \).
    \end{enumerate}
\end{theorem}
\begin{proof}
    (\( (i) \implies (ii) \)) Let \( ({a}_{n}) \) be a sequence in \( E  \) such that \( {a}_{n} t\to c \). Our goal is to show that \( f({a}_{n}) \to f(c) \), that is, we want to show that 
    \[  \forall \epsilon > 0 \ \exists N \in \N \ \text{such that} \ \forall n > N \ \tilde{d}(f({a}_{n}), f(c)) < \epsilon. \tag{*} \]
    Let \( \epsilon > 0  \) be given. Since \( f  \) is continuous at \( c  \),
    \[  \exists \delta > 0 \ \text{such that} \ \forall x \in {N}_{\delta}^{X}(c) \cap E  \  f(x) \in {N}_{\epsilon}^{Y}(f(c)). \tag{1} \]
    Furthermore, we have 
    \[  {a}_{n} \to c \implies \exists \hat{N} \ \text{such that} \ \forall n > \hat{N} \ {a}_{n} \in {N}_{\delta}^{X}(c). \tag{2} \]
    We claim that we can use \( \hat{N} \) as the same \( N  \) we were looking for. Indeed, if \( n > \hat{N} \), then (2) implies \( {a}_{n} \in {N}_{\delta}^{X}(c) \). Additionally, since \( {a}_{n} \in E  \), we have \( {a}_{n} \in {N}_{\delta}^{X}(c) \cap E  \). With (1), we have \( f({a}_{n}) \in {N}_{\epsilon}^{Y}(f(c)) \).

    (\( (ii) \implies (i) \)) Assume that for all sequences \( ({a}_{n}) \) in \( E  \) such that \( {a}_{n} \to c  \), we have \( f({a}_{n}) \to f(c) \). Our goal is to show that \( f  \) is continuous at \( c  \). We may consider two cases.
    \begin{enumerate}
        \item[(1)] Suppose \( c \in E \setminus  E' \); that is, \( c  \) is an isolated point of \( E  \). Indeed, it follows immediately that \( f  \) is continuous at any isolated point of its domain.
        \item[(2)] Suppose \( c \in E' \). It suffices to show that \( \lim_{ x \to c } f(x) = f(c) \). By the Sequential Criterion for Limits of functions, it suffices to show that 
            \begin{center}
                if \( ({a}_{n}) \) is a sequence in \( E \setminus \{ c  \}  \) such that \( {a}_{n} \to c  \), then \( f({a}_{n}) \to f(c) \).
            \end{center}
            But this is a direct consequence of the assumption that 
            \begin{center}
                if \( ({a}_{n}) \) is a sequence in \( E  \) such that \( {a}_{n} \to c  \), then \( f({a}_{n}) \to f(c) \).
            \end{center}
    \end{enumerate}
\end{proof}

Note that if \( f  \) is continuous at \( c  \) and 
\[ {a}_{1} \ {a}_{2} \ {a}_{3} \dots \to c, \]
then
\[  f({a}_{1}) \ f({a}_{2}) \ f({a}_{3}) \ \dots \to f(c) \]
that is, if \( f  \) is continuous at \( c  \), then
\[  \lim_{ n \to \infty  }  f({a}_{n}) = f(\lim_{ n \to \infty  }  {a}_{n}). \]
The above is saying that we can bring a limit inside a continuous function.

For example, we will see in Homework 10 that if \( f: \R \to \R  \) defined by \( f(x) = \sin x  \) is continuous. So, we have 
\[  \lim_{ n \to \infty  } \sin \Big(  \frac{ 1 }{ n }  \Big) = \sin \Big(\lim_{ n \to \infty  }  \frac{ 1 }{ n } \Big) = \sin 0 = 0.  \]

Note that if we can even find one sequence \( ({a}_{n}) \) in \( E  \) such that \( {a}_{n} \to c  \) but \( f({a}_{n}) \not\to f(c) \), that shows \( f  \) is NOT continuous at \( c  \).

\begin{eg}[Dirichlet Function]
   Prove that the Dirichlet Function 
   \[  f: \R \to \R  \ \ f(x) = 
   \begin{cases}
       1 &\text{if} \  x \in \Q \\
       0 &\text{if} \  x \notin \Q 
   \end{cases} \]
   is discontinuous everywhere.
\end{eg}
\begin{proof}
Let \( c \in \R  \). We will show that \( f  \) is discontinuous at \( c  \). We will consider two cases:
\begin{enumerate}
    \item[(1)] Suppose \( c \in  \R \setminus \Q  \); that is, \( f(c) = 0  \). Let \( ({q}_{n})  \) be a sequence of rational numbers such that \( {q}_{n} \to c  \). Note that  
        \[  \forall n \ \  {q}_{n} \in \Q \implies \forall n f({q}_{n}) = 1  \implies \lim_{ n \to \infty  }  f({q}_{n}) = \lim_{ n \to \infty  } 1 = 1. \]
        Therefore, we have
        \[  {q}_{n} \to c \wedge  f({q}_{n}) \not\to f(c) = 0 \implies \ \text{\( f \) is NOT continuous at \( c \)}.  \]
    \item[(2)] Suppose \( c \in \Q  \); that is, \( f(c) = 1  \). Let \( ({r}_{n}) \) be sequence of irrational numbers such that \( {r}_{n} \to c  \). Note that 
        \begin{align*}
            \forall n \ {r}_{n} \in \R \setminus \Q  &\implies \forall n \ f({r}_{n}) = 0 \\
                                                 &\implies \lim_{ n \to \infty  }  f({r}_{n}) = \lim_{ n \to \infty  }  0 = 0.
        \end{align*}
        Therefore, we have
        \[  {r}_{n} \to c \wedge f({r}_{n}) \not\to f(c) \implies \text{\( f  \) is continuous at \( c \)}. \]
\end{enumerate}
\end{proof}

\begin{eg}
    Prove that \( f: (\R,d) \to \R  \) defined by 
    \[  f(x) = 
    \begin{cases}
        1 &\text{if} \ x \in \Q \\
        0 &\text{if} \ x \notin \Q
    \end{cases} \]
    is continuous everywhere.
\end{eg}
\begin{proof}
Let \( c \in \R  \). Our goal is to show that \( f  \) is continuous at \( c  \), that is, we want to show 
\begin{center}
    \( \forall \epsilon > 0  \), \( \exists \delta > 0  \) such that if \( d(x,c) < \delta \), then \( | f(x) - f(c) | < \epsilon \). 
\end{center}
Let \( \epsilon > 0  \) be given. Regardless of the expression of \( f  \), the above holds with \( \delta = \frac{ 1 }{ 2 }  \). Indeed, if \( d(x,c) < \frac{ 1 }{ 2 }  \), then \( d(x,c) = 0  \), so \( x = c  \), and therefore, \( | f(x) - f(c) |  =  | f(c) - f(c) |  = | 0 | = 0 < \epsilon \) as desired.
\end{proof}

\begin{eg}
    Let \( (X, \|\cdot\|) \) be a normed space. Prove that 
    \[  \|\cdot\| : X \to \R  \]
    is continuous.
\end{eg}
\begin{proof}
Let \( c \in X  \). We will prove that \( \|\cdot\| \) is continuous at \( c  \), that is, we will show that  
\begin{align*}
    \text{\( \forall \epsilon > 0  \), \( \exists \delta > 0  \) such that if \( \|x - c \| < \delta \) then \( | \|x \| - \|c\| |  < \epsilon \)}. \tag{*}
\end{align*}
Let \( \epsilon > 0  \) be given. By the reverse triangle inequality, we can see that 
\[ \Big| \|x\| - \|c\| \Big| \leq \|x - c\|  \]
implies that (*) holds for example with \( \delta = \epsilon \).
\end{proof}
\begin{corollary}
    If \( {x}_{n} \to x  \) in \( X  \), then \( \|{x}_{n}\| \to \|x\|  \) in \( \R  \).
\end{corollary}

\begin{eg}
    Let \( (X,d) \) be a metric space. Let \( p \in X  \). Define \( f: X \to \R  \) by \( f(x) = d(p,x) \). Prove that \( f  \) is continuous. 
\end{eg}
\begin{proof}
Let \( c \in X  \). Our goal is to show that 
\[ \text{\( \forall \epsilon > 0   \), \( \exists \delta > 0  \) such that if \( d(x,c) < \delta \) then \( | d(p,x) - d(p,c) | < \epsilon \). } \tag{*} \] 
Let \( \epsilon > 0  \) be given. Note that (*) follows immediately from the inequality 
\[  | d(p,x) - d(p,c) | \leq d(x,c) \iff -d(x,c) \leq d(p,x) - d(p,c) \leq d(x,c) \]
that (*) holds with \( \delta = \epsilon \).
\end{proof}
\begin{corollary}
    If \( {x}_{n} \to x  \) in \( (X,d) \), and \( p \in X  \), then \( d(p,{x}_{n}) \to d(p,x) \) (\( \lim_{ n \to \infty  } d(p,{x}_{n}) = d(p,\lim_{ n \to \infty  } {x}_{n}) \)). 
\end{corollary}

\begin{eg}[Continuity with Linear Operators]
    Consider \( C[0,1] = \{ f: [0,1] \to \R : \ \text{\( f \) is continuous} \}  \) equipped with the norm 
    \[  \|f\|_{\infty } = \max_{0 \leq x \leq 1} | f(x) |.  \]
    Prove that \( T: C[0,1] \to \R  \) defined by \( T(f) = f(1/2) \) is continuous.
\end{eg}
\begin{proof}
    Let \( g \in C[0,1] \). Our goal is to show that \( T  \) is continuous at \( g  \). To this end, it suffices to show that if \( {g}_{n} \to g  \) (in the normed space \( (C[0,1], \|\cdot\|_{\infty})  \)), then \( T({g}_{n}) \to T(g) \) in \( \R  \). Applying the Squeeze Theorem to 
    \[  0 \leq | {g}_{n}(1/2) - g(1/2) | \leq \max | {g}_{n} - g(x) |,  \]
    we can see that 
    \[  | {g}_{n}(1/2) - g(1/2) | \to 0 \]
    and so, we have
    \begin{align*}
        {g}_{n} \to g &\implies \|{g}_{n} - g\|_{\infty} \to 0 \ \text{as \( n \to \infty  \)} \\
                      &\implies \max_{0 \leq x \leq 1} | {g}_{n}(x) - g(x) | \to \  0 \  \text{as} \ n \to \infty  \\
                      &\implies | {g}_{n} (1/2) - g(1/2) |  \to 0 \ \text{as} \ n \to \infty  \\
                      &\implies {g}_{n}(1/2) \to g(1/2) \ \text{in \( \R \)} \\
                      &\implies T({g}_{n}) \to T(g) \ \text{in \( \R \)}.
    \end{align*}
\end{proof}

\section{Algebraic Continuity Theorem}\label{Algebraic Continuity Theorem}

\begin{theorem}[Algebraic Continuity Theorem]
    Assume \( f: E \subseteq (X,d) \to \R   \) and \( g: E \subseteq (X,d) \to \R   \) are continuous at \( c \in E  \). Then
    \begin{enumerate}
        \item[(i)] \( k \cdot f (x) \) is continuous at \( c  \) for all \( k \in  \R  \)
        \item[(ii)] \( f(x) + g(x)  \) is continuous at \( c  \)
        \item[(iii)] \( f(x) g(x)  \) is continuous at \( c  \)
        \item[(iv)] \( \frac{ f(x) }{  g(x) }  \) is continuous at \( c  \) provided that the denominator is not zero.
    \end{enumerate}
\end{theorem}
\begin{proof}
These items are direct consequences of the ALT for sequences and the characterization of continuity via sequences. For example, let's prove (iii).

By characterization of continuity via sequences, it suffices to show that if \( ({a}_{n}) \) is a sequence in \( E  \) such that \( {a}_{n} \to c  \), then
\[  f({a}_{n}) g({a}_{n}) \to f(c) g(c). \]
Let \( ({a}_{n}) \) be such a sequence. Since \( f \) is continuous at \( c  \) and \( {a}_{n} \to c  \), we have \( f({a}_{n}) \to f(c) \). Similarly, \( g  \) is continuous at \( c  \) and \( {a}_{n} \to c  \) implies that \( g({a}_{n}) \to g(c) \). Using these two facts, we can use ALT for sequences of real numbers to conclude that 
\[  f({a}_{n}) g({a}_{n}) \to f(c) g(c) \]
as desired.
\end{proof}

\section{A Composition of Continuous Functions is Continuous}\label{A Composition of Continuous Functions is Continuous}

\begin{theorem}[Composition of Continuous Functions is Continuous]
    Let \( (X,d), (Y, \tilde{d}), (Z,\overline{d}) \) are metric spaces. Assume that \( A,B \neq \emptyset \) are subsets of \( X  \) and \( Y  \), respectively. Let \( f: A \to Y  \) and \( g: B \to Z  \) and let \( f(A) \) be contained in \( \R  \), \( f  \) is continuous at \( c \in A  \), and \( g \) is continuous at \( f(c) \in B \). Then \( g \circ f : A \to Z  \) is continuous at \( c \in A  \).
\end{theorem}

\begin{proof}
It suffices to show that if \( ({a}_{n}) \) is a sequence in \( A  \) such that \( {a}_{n} \to c  \), then \( (g \circ f)({a}_{n}) \to (g \circ f)(c) \). Let \( ({a}_{n}) \) be such a sequence. Since \( f \) is continuous at \( c  \) and \( {a}_{n} \to c  \), we have \( f({a}_{n}) \to f(c) \). Similarly, \( g  \) being continuous at \( f(c)  \) and \( f({a}_{n}) \to f(c) \) implies that \( g(f({a}_{n})) \to g(f(c)) \). So, we have \( (g \circ f)({a}_{n}) \to (g \circ f)(c) \) as desired.
\end{proof}

\begin{eg}
    If \( f: X \to \R  \) and \( g: X \to \R \) are continuous, then \( \max \{ f,g \}  \) and \( \min \{ f,g \}  \) are also continuous. The following are explicit formulas of the max and min functions: 
    \begin{enumerate}
        \item[(1)] \( \max \{ a,b \}  = \frac{ a+ b }{ 2  }  + \frac{ | a - b |  }{ 2  }  \).
        \item[(2)] \( \min \{ a,b \}  = \frac{ a + b  }{  2  }  - \frac{ | a- b |  }{ 2 }  \).
    \end{enumerate}
\end{eg}

\begin{eg}
    \begin{enumerate}
        \item[(1)] If \( E  \) is a metric subspace of \( X  \), then
            \[  i : E \to X   \ \ , \ \ i(x) = x \]
            is continuous (think of the sequential criterion for continuity).
        \item[(2)] If \( f: x \to Y \) is continuous and \( E \subseteq  X  \), then \( f |_{E} : E \to X  \) is continuous.
        \item[(3)] If \( f: X \to Y \) is continuous and \( Y  \) is a metric subspace of \( Z  \), then \( f: X \to Z \) is continuous.
    \end{enumerate} 
\end{eg}







\end{document}

