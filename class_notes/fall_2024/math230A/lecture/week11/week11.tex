\documentclass[a4paper]{article}
\usepackage[utf8]{inputenc}
\usepackage[T1]{fontenc}
\usepackage{textcomp}
\usepackage{hyperref}
% \usepackage{fourier}
% \usepackage[dutch]{babel}
\usepackage{url}
% \usepackage{hyperref}
% \hypersetup{
%     colorlinks,
%     linkcolor={black},
%     citecolor={black},
%     urlcolor={blue!80!black}
% }
\usepackage{graphicx}
\usepackage{float}
\usepackage{booktabs}
\usepackage{enumitem}
% \usepackage{parskip}
\usepackage{emptypage}
\usepackage{subcaption}
\usepackage{multicol}
\usepackage[usenames,dvipsnames]{xcolor}

% \usepackage{cmbright}


\usepackage[margin=1in]{geometry}
\usepackage{amsmath, amsfonts, mathtools, amsthm, amssymb}
\usepackage{mathrsfs}
\usepackage{cancel}
\usepackage{bm}
\newcommand\N{\ensuremath{\mathbb{N}}}
\newcommand\R{\ensuremath{\mathbb{R}}}
\newcommand\Z{\ensuremath{\mathbb{Z}}}
\renewcommand\O{\ensuremath{\emptyset}}
\newcommand\Q{\ensuremath{\mathbb{Q}}}
\newcommand\C{\ensuremath{\mathbb{C}}}
\DeclareMathOperator{\sgn}{sgn}
\usepackage{systeme}
\let\svlim\lim\def\lim{\svlim\limits}
\let\implies\Rightarrow
\let\impliedby\Leftarrow
\let\iff\Leftrightarrow
\let\epsilon\varepsilon
\usepackage{stmaryrd} % for \lightning
\newcommand\contra{\scalebox{1.1}{$\lightning$}}
% \let\phi\varphi
\renewcommand\qedsymbol{$\blacksquare$}




% correct
\definecolor{correct}{HTML}{009900}
\newcommand\correct[2]{\ensuremath{\:}{\color{red}{#1}}\ensuremath{\to }{\color{correct}{#2}}\ensuremath{\:}}
\newcommand\green[1]{{\color{correct}{#1}}}



% horizontal rule
\newcommand\hr{
    \noindent\rule[0.5ex]{\linewidth}{0.5pt}
}


% hide parts
\newcommand\hide[1]{}



% si unitx
\usepackage{siunitx}
\sisetup{locale = FR}
% \renewcommand\vec[1]{\mathbf{#1}}
\newcommand\mat[1]{\mathbf{#1}}


% tikz
\usepackage{tikz}
\usepackage{tikz-cd}
\usetikzlibrary{intersections, angles, quotes, calc, positioning}
\usetikzlibrary{arrows.meta}
\usepackage{pgfplots}
\pgfplotsset{compat=1.13}


\tikzset{
    force/.style={thick, {Circle[length=2pt]}-stealth, shorten <=-1pt}
}

% theorems
\makeatother
\usepackage{thmtools}
\usepackage[framemethod=TikZ]{mdframed}
\mdfsetup{skipabove=1em,skipbelow=0em}


\theoremstyle{definition}

\declaretheoremstyle[
    headfont=\bfseries\sffamily\color{ForestGreen!70!black}, bodyfont=\normalfont,
    mdframed={
        linewidth=2pt,
        rightline=false, topline=false, bottomline=false,
        linecolor=ForestGreen, backgroundcolor=ForestGreen!5,
    }
]{thmgreenbox}

\declaretheoremstyle[
    headfont=\bfseries\sffamily\color{NavyBlue!70!black}, bodyfont=\normalfont,
    mdframed={
        linewidth=2pt,
        rightline=false, topline=false, bottomline=false,
        linecolor=NavyBlue, backgroundcolor=NavyBlue!5,
    }
]{thmbluebox}

\declaretheoremstyle[
    headfont=\bfseries\sffamily\color{NavyBlue!70!black}, bodyfont=\normalfont,
    mdframed={
        linewidth=2pt,
        rightline=false, topline=false, bottomline=false,
        linecolor=NavyBlue
    }
]{thmblueline}

\declaretheoremstyle[
    headfont=\bfseries\sffamily\color{RawSienna!70!black}, bodyfont=\normalfont,
    mdframed={
        linewidth=2pt,
        rightline=false, topline=false, bottomline=false,
        linecolor=RawSienna, backgroundcolor=RawSienna!5,
    }
]{thmredbox}

\declaretheoremstyle[
    headfont=\bfseries\sffamily\color{RawSienna!70!black}, bodyfont=\normalfont,
    numbered=no,
    mdframed={
        linewidth=2pt,
        rightline=false, topline=false, bottomline=false,
        linecolor=RawSienna, backgroundcolor=RawSienna!1,
    },
    qed=\qedsymbol
]{thmproofbox}

\declaretheoremstyle[
    headfont=\bfseries\sffamily\color{NavyBlue!70!black}, bodyfont=\normalfont,
    numbered=no,
    mdframed={
        linewidth=2pt,
        rightline=false, topline=false, bottomline=false,
        linecolor=NavyBlue, backgroundcolor=NavyBlue!1,
    },
]{thmexplanationbox}

\declaretheorem[style=thmgreenbox, numberwithin = section, name=Definition]{definition}
\declaretheorem[style=thmbluebox, name=Example]{eg}
\declaretheorem[style=thmredbox, numberwithin = section, name=Proposition]{prop}
\declaretheorem[style=thmredbox, numberwithin = section, name=Theorem]{theorem}
\declaretheorem[style=thmredbox, numberwithin = section,  name=Lemma]{lemma}
\declaretheorem[style=thmredbox, numberwithin = section,  numbered=no, name=Corollary]{corollary}


\declaretheorem[style=thmproofbox, name=Proof]{replacementproof}
\renewenvironment{proof}[1][\proofname]{\vspace{-10pt}\begin{replacementproof}}{\end{replacementproof}}


\declaretheorem[style=thmexplanationbox, name=Proof]{tmpexplanation}
\newenvironment{explanation}[1][]{\vspace{-10pt}\begin{tmpexplanation}}{\end{tmpexplanation}}


\declaretheorem[style=thmblueline, numbered=no, name=Remark]{remark}
\declaretheorem[style=thmblueline, numbered=no, name=Note]{note}

\newtheorem*{uovt}{UOVT}
\newtheorem*{notation}{Notation}
\newtheorem*{previouslyseen}{As previously seen}
\newtheorem*{problem}{Problem}
\newtheorem*{observe}{Observe}
\newtheorem*{property}{Property}
\newtheorem*{intuition}{Intuition}


\usepackage{etoolbox}
\AtEndEnvironment{vb}{\null\hfill$\diamond$}%
\AtEndEnvironment{intermezzo}{\null\hfill$\diamond$}%
% \AtEndEnvironment{opmerking}{\null\hfill$\diamond$}%

% http://tex.stackexchange.com/questions/22119/how-can-i-change-the-spacing-before-theorems-with-amsthm
\makeatletter
% \def\thm@space@setup{%
%   \thm@preskip=\parskip \thm@postskip=0pt
% }
\newcommand{\oefening}[1]{%
    \def\@oefening{#1}%
    \subsection*{Oefening #1}
}

\newcommand{\suboefening}[1]{%
    \subsubsection*{Oefening \@oefening.#1}
}

\newcommand{\exercise}[1]{%
    \def\@exercise{#1}%
    \subsection*{Exercise #1}
}

\newcommand{\subexercise}[1]{%
    \subsubsection*{Exercise \@exercise.#1}
}


\usepackage{xifthen}

\def\testdateparts#1{\dateparts#1\relax}
\def\dateparts#1 #2 #3 #4 #5\relax{
    \marginpar{\small\textsf{\mbox{#1 #2 #3 #5}}}
}

\def\@lesson{}%
\newcommand{\lesson}[3]{
    \ifthenelse{\isempty{#3}}{%
        \def\@lesson{Lecture #1}%
    }{%
        \def\@lesson{Lecture #1: #3}%
    }%
    \subsection*{\@lesson}
    \testdateparts{#2}
}

% \renewcommand\date[1]{\marginpar{#1}}


% fancy headers
\usepackage{fancyhdr}
\pagestyle{fancy}

\fancyhead[LE,RO]{Lance Remigio}
\fancyhead[RO,LE]{\@lesson}
\fancyhead[RE,LO]{}
\fancyfoot[LE,RO]{\thepage}
\fancyfoot[C]{\leftmark}

\makeatother




% notes
\usepackage{todonotes}
\usepackage{tcolorbox}

\tcbuselibrary{breakable}
\newenvironment{verbetering}{\begin{tcolorbox}[
    arc=0mm,
    colback=white,
    colframe=green!60!black,
    title=Opmerking,
    fonttitle=\sffamily,
    breakable
]}{\end{tcolorbox}}

\newenvironment{noot}[1]{\begin{tcolorbox}[
    arc=0mm,
    colback=white,
    colframe=white!60!black,
    title=#1,
    fonttitle=\sffamily,
    breakable
]}{\end{tcolorbox}}




% figure support
\usepackage{import}
\usepackage{xifthen}
\pdfminorversion=7
\usepackage{pdfpages}
\usepackage{transparent}
\newcommand{\incfig}[1]{%
    \def\svgwidth{\columnwidth}
    \import{./figures/}{#1.pdf_tex}
}

% %http://tex.stackexchange.com/questions/76273/multiple-pdfs-with-page-group-included-in-a-single-page-warning
\pdfsuppresswarningpagegroup=1




\begin{document}

\section{Lecture 20-21}
\subsection{Topics}

\begin{itemize}
    \item Infinite series (Basic definitions)
    \item Telescoping Series, Geometric Series
    \item Algebraic Limit Theorem for Series
    \item Divergence Test
    \item Cauchy Criterion for Series
    \item Absolute Convergence Test
\end{itemize}

Consider the following expression:
\[  \frac{ 1  }{  2  }  + \frac{ 1  }{  4  }  + \frac{ 1  }{  8  }  + \frac{ 1 }{ 16  }  + \cdots \ . \]
How can we make sense of the infinite sum above? More generally, let \( ({a}_{n}) \) be a sequence of real numbers. Then what does the following expression mean?
\[  \sum_{ n=1  }^{ \infty  } {a}_{n} = {a}_{1} + {a}_{2} + {a}_{3} + \cdots \ ? \]
\begin{definition}[Infinite Series]
    Let \( (X, \|\cdot\|) \) be a normed space. Let \( ({x}_{n}) \) be a sequence in \( X  \). 
    \begin{enumerate}
        \item[(*)] An expression of the form 
            \[  \sum_{ n=1  }^{ \infty  } {x}_{n} = {x}_{1} + {x}_{2} + {x}_{3} + \cdots  \]
            is called an \textbf{infinite series}.
        \item[(*)] \( {x}_{1}, {x}_{2}, \dots  \) are called the \textbf{terms} of this infinite series.
        \item[(*)] The corresponding sequence of \textbf{partial sums} is defined by
            \[  \forall m \in \N  \ \ {s}_{m} = \text{(finite) sum of the first \(  m  \) terms of the series}; \]
            that is,
            \begin{align*}
                {s}_{1} &= {x}_{1} \\
                {s}_{2} &= {x}_{1} + {x}_{2} \\
                {s}_{3} &= {x}_{1} + {x}_{2} + {x}_{3} \\
                        &\vdots \\
                {s}_{m} &= {x}_{1} + {x}_{2} + \cdots + {x}_{m} \\
                        &\vdots
            \end{align*}
        \item We say that the infinite series \( \sum_{ n=1  }^{ \infty  } {x}_{n}  \) converges to \( L \in X  \) (and we write \( \sum_{ n=1  }^{ \infty  } {x}_{n} = L  \)) if \( \lim_{ n \to \infty  }  {s}_{m} =  L  \).
        \item We say that the infinite series \textbf{diverges}, if \( ({s}_{m}) \) diverges.
        \item If \( X = \R  \) and \( {s}_{m} \to \infty   \), we write \( \sum_{ n=1  }^{ \infty  } {x}_{n} = \infty  \).
        \item[(*)] If \( X = \R  \) and \( {s}_{m} \to - \infty   \), we write \( \sum_{ n=1  }^{ \infty  } {x}_{n} = - \infty  \).
    \end{enumerate}
\end{definition}

\begin{remark}[1]
    Given an infinite series \( \sum_{ n=1  }^{ \infty  } {x}_{n}  \), it is important to keep a clear distinction between
    \begin{enumerate}
        \item[(a)] the sequence of terms: \( ({x}_{1}, {x}_{2}, {x}_{3}, \dots ) \)
        \item[(b)] the sequence of partial sums: \( ({s}_{1}, {s}_{2}, {s}_{3}, \dots ) \).
    \end{enumerate}
\end{remark}

\begin{remark}[2]
    We may sometimes consider infinite series where the summation begins with \( n = 0  \) or \( n = {n}_{0} \) for some integer \( {n}_{0}  \) different from \( 1  \).
\end{remark}

As we shall see, some of our theorems apply specifically to series in \( \R  \) or to series with terms in \( [0,\infty) \). Also, in our examples, we will primarily focus on series in \( \R  \); however, we will also consider encounter highly useful theorems that hold in more general normed spaces.  

In most cases it is difficult (or even impossible)  to find a simple formula for the partial sum \( {s}_{m} \). However, there are two types of series for which we can easily find a simple formula for the partial sums. These two types are:
\begin{enumerate}
    \item[(1)] Telescoping Series
    \item[(2)] Geometric Series
\end{enumerate}


\subsection{Telescoping Series}

\begin{eg}
   Consider the following series 
   \[  \sum_{ n=1  }^{ \infty  } \Big(  \frac{ 1 }{ n } - \frac{ 1 }{ n + 1 }  \Big). \]
   Notice that \( {x}_{n} = \frac{ 1 }{ n }  - \frac{ 1 }{ n+1 }  \). The corresponding sequence of partial sums is  
   \begin{align*}
       {s}_{1} &= 1 - \frac{ 1 }{ 2 }  \\
       {s}_{2} &= \Big(  1 - \frac{ 1 }{ 2 }  \Big) + \Big(  \frac{ 1 }{ 2 }  - \frac{ 1 }{ 3 }  \Big) = 1 - \frac{ 1 }{ 3 } \\
       {s}_{3} &= \Big(  1 - \frac{ 1 }{ 2 }  \Big) + \Big(  \frac{ 1 }{ 2 }  - \frac{ 1 }{ 3 }  \Big) + \Big(  \frac{ 1 }{ 3 }  - \frac{ 1 }{ 4 }  \Big) = 1 - \frac{ 1 }{ 4 } \\
               &\vdots \\
       {s}_{m} &= \sum_{ n=1  }^{ m } \Big(  \frac{ 1 }{ n }  - \frac{ 1 }{ n + 1 }  \Big) = \Big(  \sum_{ n=1  }^{ m  } \frac{ 1 }{ n }  \Big) - \Big(  \sum_{ n=1  }^{ m } \frac{ 1 }{ n+1 }  \Big) \\ 
               &= 1 - \frac{ 1 }{ m + 1 }.
   \end{align*}
   Clearly, we see that 
   \[  \lim_{ m \to \infty  }  {s}_{m} = \lim_{ m \to \infty  }  \Big[ 1 - \frac{ 1 }{  m + 1  } \Big] = 1.  \]
   Hence, \( \sum_{ n=1  }^{ \infty   \frac{ 1 }{ n (n+1) }  } \) converges to \( 1  \).
\end{eg}

In general, a telescoping series is an infinite series where partial sums eventually have a finite number of terms after cancellation. For example, if \( ({y}_{n}) \) is a sequence in the normed space \( (X, \|\cdot\|) \), then \( \sum_{ n=1  }^{ \infty   } ({y}_{n} - {y}_{n+1}) \) is a telescoping series; that is, 
\begin{align*}
    {s}_{m} = \sum_{ n=1  }^{ m } ({y}_{n} - {y}_{n+1}) = \Big(  \sum_{ n=1  }^{ m } {y}_{n} \Big) - \Big(  \sum_{ n=1  }^{ m } {y}_{n+1} \Big) &= [{y}_{1} + {y}_{2} + \cdots + {y}_{m}] - [{y}_{2} + {y}_{3} + \cdots + {y}_{m+1}] \\
                                                                                                                                                &= {y}_{1} - {y}_{m}.
\end{align*}

\subsection{Geometric Series}

Let \( k  \) be a fixed integer and let \( r \neq 0  \) be a fixed real number. The infinite series \( \sum_{ n= k  }^{  \infty   } r^{n} = r^{k } + r^{k +1} + r^{k + 2 } + \cdots   \) is called a \textbf{geometric series} with common ration \( "r" \). For example, 
\[ \sum_{ n=1  }^{ \infty   } \Big(  \frac{ 1 }{ 2 }  \Big)^{n} = \frac{ 1 }{ 2 }  + \frac{ 1 }{ 4 }  + \frac{ 1 }{ 8 }  + \cdots \ \text{is a geometric series with common ratio} \  \frac{ 1 }{ 2 }. \]
Another example is that
\[  \sum_{ n=1  }^{ \infty  } \frac{ 7^{n} }{  29^{n} }  \ \text{is a geometric series with common ratio} \ \frac{ 7  }{  2 9 }. \]
A non-example is the following:
\[  \sum_{ n=1  }^{ \infty   } \frac{ 1 }{ n^{2} }. \]
We can easily find a formula for the \( m \)th partial sum of \( \sum_{ n= k  }^{  \infty   } r^{k } \) where
\begin{align*}
    {s}_{1} &= r^{k} \\
    {s}_{2}&= r^{k } + r^{k+1} \\
    {s}_{3} &= r^{k } + r^{k+1} + r^{k+2} \\
            &\vdots \\
    {s}_{m} &= r^{k } + r^{k+1} + \cdots + r^{k + m - 1} \tag{*}
\end{align*}

Now, if \( r = 1  \), we have
\[  {s}_{m} = \underbrace{1 + 1 + \cdots + 1}_{m \text{summands}} = m.  \]

If \( r \neq 1  \), then multiply both sides of (*) by \( r  \):
\[  r {s}_{m} = r^{k+1} + r^{k+2} + \cdots+ r^{k+m} \tag{**}. \]
Subtracting (**) from (*), we get
\[  {s}_{m} - r {s}_{m} = r^{k} - r^{k + m}. \]
Since \( r \neq 1  \), we have
\[  {s}_{m} = \frac{ r^{k } - r^{k + m} }{ 1 - r  }  = \frac{ r^{k } (1 - r^{m}) }{  1 - r  }. \]

Note that
\begin{enumerate}
    \item[(i)] If \( | r  |  < 1  \), then \( \lim r^{m} = 0  \).
    \item[(ii)] If \( | r  |  > 1  \) or \( r = -1  \), then \( \lim_{ n \to \infty  }  r^{m}  \) does not exists.
\end{enumerate}
Hence, we have 
\[  \lim_{ m \to \infty  }  {s}_{m} = 
\begin{cases}
    \frac{ r^{k } }{  1 - r  }  &\text{if} | r  |  < 1 \\
    \text{DNE} &\text{if} | r  |  \geq 1. 
\end{cases} \]
Thus, 
\[  \sum_{ n= k  }^{  \infty   } r^{n } = 
\begin{cases}
    \frac{ r^{k }  }{  1 - r  }  &\text{if} | r  |  < 1 \\
    \text{diverges} &\text{if} | r  |  \geq 1.
\end{cases} \]

\begin{eg}
    \begin{itemize}
        \item \( \sum_{ n=1  }^{ \infty   } \Big(  \frac{ 1 }{ 2 }  \Big)^{n} \)

            Observe that 
            \[  \sum_{ n=1  }^{ \infty  } \Big(  \frac{ 1 }{ 2 }  \Big)^{n} = \frac{ \Big(  \frac{ 1 }{ 2 }  \Big)^{1} }{  1 - \frac{ 1 }{ 2 }  }  = \frac{ \frac{ 1 }{ 2 }  }{ \frac{ 1 }{ 2 }  }  = 1.  \]

        \item \( \sum_{ n=4 }^{ \infty  } \Big(  \frac{ 1 }{ 2 }  \Big)^{n}  \)

            Observe that 
            \[  \sum_{ n=4 }^{ \infty  } \Big(  \frac{ 1 }{ 2 }  \Big)^{n} = \frac{ \Big(  \frac{ 1 }{ 2 }  \Big)^{4} }{  1 - \frac{ 1 }{ 2 }  } = \frac{ \Big(  \frac{ 1 }{ 2 }  \Big)^{4} }{  \Big(  \frac{ 1 }{ 2 }  \Big) }  = \frac{ 1 }{ 8 }. \]
    \end{itemize}
\end{eg}

\begin{theorem}[ ]
    Let \( (X , \|\cdot\|) \) be a normed space. Let \( ({a}_{n}) \) and \( ({b}_{n}) \) be two sequence in \( X  \). Suppose that 
    \[  \sum_{ n=1  }^{ \infty  } {a}_{n} = A \ \ (A \in X) , \ \ \sum_{ n=1  }^{ \infty  } {b}_{n} = B \ \ (B \in X). \]
    Then
    \begin{enumerate}
        \item[(i)] For any scalar \( \lambda  \), \( \sum_{ n=1  }^{ \infty   } (\lambda {a}_{n}) = \lambda A  \).
        \item[(ii)] \( \sum_{ n=1  }^{ \infty  } ({a}_{n} + {b}_{n}) = A + B \).
    \end{enumerate}
\end{theorem}
\begin{proof}
Can easily be proven via the Algebraic Limit Theorem for Sequences.
\end{proof}

\begin{theorem}[Divergence Test]
   Let \( (X,\|\cdot\|) \) be a normed space. Let \( ({x}_{n}) \) be a sequence in \( X  \). If \( \sum_{ n=1  }^{ \infty  } {x}_{n}  \) converges, then \( \lim_{ n \to \infty  }  {x}_{n} = 0  \). 
\end{theorem}
\begin{proof}
Let \( {s}_{n} = {x}_{1} + \cdots + {x}_{n} \). Let \( L = \sum_{ n=1  }^{ \infty  } {x}_{n} \). Note that 
\[  \sum_{ n=1  }^{ \infty  } {x}_{n} = L \implies \lim_{ n \to \infty  }  {s}_{n} = L.  \]
Also, note that 
\[  \forall n \geq 2 \ \ {x}_{n} = {s}_{n} - {s}_{n-1}. \]
Note that \( \lim {s}_{n} = L  \) and \( \lim {s}_{n-1} = L  \). Therefore, 
\[  \lim_{ n \to \infty  }  {x}_{n} = \lim_{ n \to \infty  }  ({s}_{n} - {s}_{n-1}) = L - L =  0  \]
by the Algebraic Limit Theorem for normed spaces.
\end{proof}
\begin{remark}
    Note that the divergence test is just the contrapositive of the above.
\end{remark}

\begin{eg}
    \begin{itemize}
        \item \( \sum_{ n=1  }^{ \infty  } (-1)^{n} \) diverges because \( \lim_{ n \to \infty  }  (-1)^{n} \) does not exist.
        \item \( \sum_{ n=1  }^{ \infty  } \frac{ 3n+1 }{ 7n-4 }  \) diverges because \( \lim_{ n \to \infty  }  \frac{ 3n + 1 }{ 7 n - 4  }  = \frac{ 3 }{ 7 } \neq 0  \).
    \end{itemize}
\end{eg}

From the above statements, we can now see make two key observations:

\begin{itemize}
    \item If \( \lim_{ n \to \infty  }  {x}_{n} = 0  \), then \( \sum_{ n=1  }^{ \infty  } {x}_{n} \) may or may not converge.
    \item If \( \lim_{ n \to \infty  }  {x}_{n} \neq 0  \), then \( \sum_{ n=1  }^{ \infty  } {x}_{n}  \) diverges.
\end{itemize}

As for the first observation above, we see that \( \sum_{  }^{  } \frac{ 1 }{ n }  \) diverges, but \( \sum_{   }^{  } \frac{ 1 }{ n^{2} }   \) converges.

\begin{theorem}[Cauchy Criterion]
    Let \( (X,\|\cdot\|) \) be a complete normed space. Let \( ({x}_{n}) \) be a sequence in \( X  \). Then
    \[  \sum_{ k=1  }^{ \infty  } {x}_{k } \ \text{converges} \ \iff \forall \epsilon > 0 \ \exists N \in \N \ \text{such that} \ \forall n > m > N \ \ \|\sum_{ k=1  }^{ n } {x}_{k } \| < \epsilon. \]
\end{theorem}
\begin{proof}
Let \( {s}_{n} = {x}_{1} + \cdots + {x}_{k} \). Assuming that  
\[  \|{s}_{n} - {s}_{m} \| = \Big\|\sum_{ k=m+1 }^{ n } {x}_{k} \Big\| \]
where \( n > m  \) and from the fact that 
\begin{align*}
    {s}_{n} - {s}_{m} &= ({x}_{1} + \cdots + {x}_{m} + \cdots + {x}_{n}) - ({x}_{1} +  \cdots + {x}_{m}) \\
                      &= \sum_{ k= m+1 }^{ n } {s}_{k}.
\end{align*}
Then we have 
\begin{align*}
    \sum_{ k=1  }^{ \infty  } {x}_{k } \ \text{converges} &\iff ({s}_{k}) \ \text{converges} \\
                                                          &\iff ({s}_{k}) \ \text{is Cauchy} \\
                                                          &\iff \forall \epsilon > 0 \ \exists N \in \N \ \text{such that} \ \forall n > m > N \  \ \|{s}_{n} - {s}_{m} \| < \epsilon \\
                                                          &\iff \forall \epsilon > 0 \ \exists N \in \N \ \text{such that} \ \forall n > m > N \ \ \|\sum_{ k=m+1 }^{ n } {x}_{k} \| < \epsilon
\end{align*}
as desired.
\end{proof}

From here, we will refer to complete normed spaces as Banach spaces.

\begin{theorem}[Absolute Convergence Test]
    Let \( (X,\|\cdot\|) \) be a Banach Space. Let \( ({x}_{n}) \) be a sequence in \( X  \). If \( \underbrace{\sum_{ n=1  }^{ \infty  } \|{x}_{n}\| }_{\text{a sum in \( \R  \)}} \) converges, then \( \sum_{ n=1  }^{ \infty  } {x}_{n} \) converges.
\end{theorem}
\begin{proof}
    By the Cauchy Criterion for series, it suffices to show that 
    \[  \forall \epsilon > 0 \ \exists N \in \N \ \text{such that} \ \forall n > m > N \Big\|\sum_{ k=m+1 }^{ n } {x}_{k }\Big\| < \epsilon. \tag{*} \]
    Since \( \sum_{ k=1  }^{ \infty   } \|{x}_{k }\|  \) converges, and since \( \R  \) is complete, it follows from the Cauchy Criterion for series that there exists \( \hat{N} \) such that 
    \[ \forall n> m > \hat{N} \ \ \Big| \sum_{ k= m +1 }^{ n  } \|{x}_{k }\| \Big| < \epsilon.  \]
    We claim that \( \hat{N} \) is the same \( N  \) we were looking for. Hence, if \( n > m > \hat{N} \), then we have
    \[  \Big\| \sum_{ k=m+1 }^{ n  } {x}_{k }  \Big\| \leq \sum_{ k= m+1 }^{ n  } \| {x}_{k} \| = \Big|  \sum_{ k= m+1 }^{ n  } \|{x}_{k }\| \Big|  < \epsilon  \]
    as desired.
\end{proof}

Please take note of the following observations: 

\begin{enumerate}
    \item[(1)] If \( \sum_{ n=1  }^{ \infty  } \|{x}_{n}\| \) converges, then \( \sum_{ n=1  }^{ \infty  } {x}_{n} \) converges (in Banach spaces).
    \item[(2)] If \( \sum_{ n=1  }^{ \infty  } \|{x}_{n}\|  \) diverges, then \( \sum_{ n=1  }^{ \infty  } {x}_{n} \) may converge or diverge.
\end{enumerate}

From (2), we shall see (in the next lecture) that
\begin{enumerate}
    \item[(1)] \( \sum_{ n=1  }^{ \infty  } \Big| \frac{ (-1)^{n+1} }{ n }  \Big|   \) diverges but \( \sum_{ n=1  }^{ \infty   } \frac{ (-1)^{n+1} }{ n }   \) converges.
    \item[(2)] \( \sum_{ n=1  }^{ \infty  } | (-1)^{n} |  \) diverges, also \( \sum_{ n=1  }^{ \infty  } (-1)^{n} \) diverges (by the divergence test).
\end{enumerate}

\begin{definition}[Absolute Convergence and Conditional Convergence]
    We say that a series \( \sum_{  }^{  }{x}_{n} \) \textbf{absolutely converges} if \( \sum_{  }^{  } \|{x}_{n}\| \) converges and \( \sum_{  }^{  } {x}_{n} \) converges. We say that \( \sum_{  }^{  }{x}_{n} \) \textbf{conditionally converges} if \( \sum_{  }^{  } \|{x}_{n}\|  \) diverges but \( \sum_{  }^{  } {x}_{n} \) converges. 
\end{definition}

\begin{eg}[Conditionally Convergent]
    Consider \( \sum_{ n=1  }^{ \infty  } \frac{ (-1)^{n+1} }{ n }  \). We see that this series is conditionally convergent since  
    \[  \Big| \frac{ (-1)^{n+1} }{ n }  \Big|  = \frac{ 1 }{ n } \to 0 \]
    which tell us that the above series diverges by the divergence test. But the above series converges via the Leibniz Test (As we shall see in the next lecture).
\end{eg}

\section{Lecture 21-22}

\subsection{Topics}

\begin{itemize}
    \item Cauchy Condensation Test
    \item Comparison Test
    \item More on \( \lim \sup   \) and \( \lim \inf  \)
    \item Root Test
    \item Ratio Test
    \item Dirichlet's Test
\end{itemize}


\end{document}
