\documentclass[a4paper]{article}

\usepackage[utf8]{inputenc}
\usepackage[T1]{fontenc}
% \usepackage{fourier}
\usepackage{textcomp}
\usepackage{hyperref}
\usepackage[english]{babel}
\usepackage{url}
% \usepackage{hyperref}
% \hypersetup{
%     colorlinks,
%     linkcolor={black},
%     citecolor={black},
%     urlcolor={blue!80!black}
% }
\usepackage{graphicx} \usepackage{float}
\usepackage{booktabs}
\usepackage{enumitem}
% \usepackage{parskip}
% \usepackage{parskip}
\usepackage{emptypage}
\usepackage{subcaption}
\usepackage{multicol}
\usepackage[usenames,dvipsnames]{xcolor}
\usepackage{ocgx}
% \usepackage{cmbright}


\usepackage[margin=1in]{geometry}
\usepackage{amsmath, amsfonts, mathtools, amsthm, amssymb}
\usepackage{thmtools}
\usepackage{mathrsfs}
\usepackage{cancel}
\usepackage{bm}
\newcommand\N{\ensuremath{\mathbb{N}}}
\newcommand\R{\ensuremath{\mathbb{R}}}
\newcommand\Z{\ensuremath{\mathbb{Z}}}
\renewcommand\O{\ensuremath{\emptyset}}
\newcommand\Q{\ensuremath{\mathbb{Q}}}
\newcommand\C{\ensuremath{\mathbb{C}}}
\newcommand\F{\ensuremath{\mathbb{F}}}
\DeclareMathOperator{\sgn}{sgn}
\DeclareMathOperator{\diam}{diam}
\DeclareMathOperator{\LO}{LO}
\DeclareMathOperator{\UP}{UP}
\DeclareMathOperator{\card}{card}
\DeclareMathOperator{\Arg}{Arg}
\DeclareMathOperator{\Dom}{Dom}
\DeclareMathOperator{\Log}{Log}
\DeclareMathOperator{\dist}{dist}
% \DeclareMathOperator{\span}{span}
\usepackage{systeme}
\let\svlim\lim\def\lim{\svlim\limits}
\renewcommand\implies\Longrightarrow
\let\impliedby\Longleftarrow
\let\iff\Longleftrightarrow
\let\epsilon\varepsilon
\usepackage{stmaryrd} % for \lightning
\newcommand\contra{\scalebox{1.1}{$\lightning$}}
% \let\phi\varphi
\renewcommand\qedsymbol{$\blacksquare$}

% correct
\definecolor{correct}{HTML}{009900}
\newcommand\correct[2]{\ensuremath{\:}{\color{red}{#1}}\ensuremath{\to }{\color{correct}{#2}}\ensuremath{\:}}
\newcommand\green[1]{{\color{correct}{#1}}}

% horizontal rule
\newcommand\hr{
    \noindent\rule[0.5ex]{\linewidth}{0.5pt}
}

% hide parts
\newcommand\hide[1]{}

% si unitx
\usepackage{siunitx}
\sisetup{locale = FR}
% \renewcommand\vec[1]{\mathbf{#1}}
\newcommand\mat[1]{\mathbf{#1}}

% tikz
\usepackage{tikz}
\usepackage{tikz-cd}
\usetikzlibrary{intersections, angles, quotes, calc, positioning}
\usetikzlibrary{arrows.meta}
\usepackage{pgfplots}
\pgfplotsset{compat=1.13}

\tikzset{
    force/.style={thick, {Circle[length=2pt]}-stealth, shorten <=-1pt}
}

% theorems
\makeatother
\usepackage{thmtools}
\usepackage[framemethod=TikZ]{mdframed}
\mdfsetup{skipabove=1em,skipbelow=1em}

\theoremstyle{definition}

\declaretheoremstyle[
    headfont=\bfseries\sffamily\color{ForestGreen!70!black}, bodyfont=\normalfont,
    mdframed={
        linewidth=1pt,
        rightline=false, topline=false, bottomline=false,
        linecolor=ForestGreen, backgroundcolor=ForestGreen!5,
    }
]{thmgreenbox}

\declaretheoremstyle[
    headfont=\bfseries\sffamily\color{NavyBlue!70!black}, bodyfont=\normalfont,
    mdframed={
        linewidth=1pt,
        rightline=false, topline=false, bottomline=false,
        linecolor=NavyBlue, backgroundcolor=NavyBlue!5,
    }
]{thmbluebox}

\declaretheoremstyle[
    headfont=\bfseries\sffamily\color{NavyBlue!70!black}, bodyfont=\normalfont,
    mdframed={
        linewidth=1pt,
        rightline=false, topline=false, bottomline=false,
        linecolor=NavyBlue
    }
]{thmblueline}

\declaretheoremstyle[
    headfont=\bfseries\sffamily, bodyfont=\normalfont,
    numbered = no,
    mdframed={
        rightline=true, topline=true, bottomline=true,
    }
]{thmbox}

\declaretheoremstyle[
    headfont=\bfseries\sffamily, bodyfont=\normalfont,
    numbered=no,
    % mdframed={
    %     rightline=true, topline=false, bottomline=true,
    % },
    qed=\qedsymbol
]{thmproofbox}

\declaretheoremstyle[
    headfont=\bfseries\sffamily\color{NavyBlue!70!black}, bodyfont=\normalfont,
    numbered=no,
    mdframed={
        rightline=false, topline=false, bottomline=false,
        linecolor=NavyBlue, backgroundcolor=NavyBlue!1,
    },
]{thmexplanationbox}

\declaretheorem[
    style=thmbox, 
    % numberwithin = section,
    numbered = no,
    name=Definition
    ]{definition}

\declaretheorem[
    style=thmbox, 
    name=Example,
    ]{eg}

\declaretheorem[
    style=thmbox, 
    % numberwithin = section,
    name=Proposition]{prop}

\declaretheorem[
    style = thmbox,
    numbered=yes,
    name =Problem
    ]{problem}

\declaretheorem[style=thmbox, name=Theorem]{theorem}
\declaretheorem[style=thmbox, name=Lemma]{lemma}
\declaretheorem[style=thmbox, name=Corollary]{corollary}

\declaretheorem[style=thmproofbox, name=Proof]{replacementproof}

\declaretheorem[style=thmproofbox, 
                name = Solution
                ]{replacementsolution}

\renewenvironment{proof}[1][\proofname]{\vspace{-1pt}\begin{replacementproof}}{\end{replacementproof}}

\newenvironment{solution}
    {
        \vspace{-1pt}\begin{replacementsolution}
    }
    { 
            \end{replacementsolution}
    }

\declaretheorem[style=thmexplanationbox, name=Proof]{tmpexplanation}
\newenvironment{explanation}[1][]{\vspace{-10pt}\begin{tmpexplanation}}{\end{tmpexplanation}}

\declaretheorem[style=thmbox, numbered=no, name=Remark]{remark}
\declaretheorem[style=thmbox, numbered=no, name=Note]{note}

\newtheorem*{uovt}{UOVT}
\newtheorem*{notation}{Notation}
\newtheorem*{previouslyseen}{As previously seen}
% \newtheorem*{problem}{Problem}
\newtheorem*{observe}{Observe}
\newtheorem*{property}{Property}
\newtheorem*{intuition}{Intuition}

\usepackage{etoolbox}
\AtEndEnvironment{vb}{\null\hfill$\diamond$}%
\AtEndEnvironment{intermezzo}{\null\hfill$\diamond$}%
% \AtEndEnvironment{opmerking}{\null\hfill$\diamond$}%

% http://tex.stackexchange.com/questions/22119/how-can-i-change-the-spacing-before-theorems-with-amsthm
\makeatletter
% \def\thm@space@setup{%
%   \thm@preskip=\parskip \thm@postskip=0pt
% }
\newcommand{\oefening}[1]{%
    \def\@oefening{#1}%
    \subsection*{Oefening #1}
}

\newcommand{\suboefening}[1]{%
    \subsubsection*{Oefening \@oefening.#1}
}

\newcommand{\exercise}[1]{%
    \def\@exercise{#1}%
    \subsection*{Exercise #1}
}

\newcommand{\subexercise}[1]{%
    \subsubsection*{Exercise \@exercise.#1}
}


\usepackage{xifthen}

\def\testdateparts#1{\dateparts#1\relax}
\def\dateparts#1 #2 #3 #4 #5\relax{
    \marginpar{\small\textsf{\mbox{#1 #2 #3 #5}}}
}

\def\@lesson{}%
\newcommand{\lesson}[3]{
    \ifthenelse{\isempty{#3}}{%
        \def\@lesson{Lecture #1}%
    }{%
        \def\@lesson{Lecture #1: #3}%
    }%
    \subsection*{\@lesson}
    \testdateparts{#2}
}

% \renewcommand\date[1]{\marginpar{#1}}


% fancy headers
\usepackage{fancyhdr}
\pagestyle{fancy}

\makeatother

% notes
\usepackage{todonotes}
\usepackage{tcolorbox}

\tcbuselibrary{breakable}
\newenvironment{verbetering}{\begin{tcolorbox}[
    arc=0mm,
    colback=white,
    colframe=green!60!black,
    title=Opmerking,
    fonttitle=\sffamily,
    breakable
]}{\end{tcolorbox}}

\newenvironment{noot}[1]{\begin{tcolorbox}[
    arc=0mm,
    colback=white,
    colframe=white!60!black,
    title=#1,
    fonttitle=\sffamily,
    breakable
]}{\end{tcolorbox}}

% figure support
\usepackage{import}
\usepackage{xifthen}
\pdfminorversion=7
\usepackage{pdfpages}
\usepackage{transparent}
\newcommand{\incfig}[1]{%
    \def\svgwidth{\columnwidth}
    \import{./figures/}{#1.pdf_tex}
}

% %http://tex.stackexchange.com/questions/76273/multiple-pdfs-with-page-group-included-in-a-single-page-warning
\pdfsuppresswarningpagegroup=1



\title{Math 234A Take Home Quiz}
\author{Lance Remigio}
\begin{document}
\maketitle   

\begin{enumerate}
    \item Indicate whether the following statements are True or False. You \textbf{do not} need to justify your answer.
        \begin{enumerate}
            \item[(a)] Let \( f: D \to \C  \) where \( D \subseteq  \C   \) is open. Let \( a \in D  \) and suppose that \( f  \) is holomorphic at \( a  \). Then, we can find an open set \( D' \subseteq D  \) such that \( a \in D' \) and \( f  \) is complex differentiable at each \( z \in D' \). \textbf{True}. 
            \item[(b)] Let \( f: D \to \C  \) where \( D \subseteq  \C   \) is an open set. Let \( f(z) = u + iv \) with \( u: D \to \R  \) and \( v : D \to \R  \). Suppose that \( u  \) and \( v  \) satisfy the Cauchy-Riemann Equations on \( D  \). Then, \( f  \) is holomorphic on \( D  \). \textbf{False.}
            \item[(c)] Let \( f: \C \to \C  \) defined by \( f(z) = \sin (\overline{z}) \). Then, \( f  \) is holomorphic at \( 0 \in \C  \). \textbf{False.}
            \item[(d)] Let \( f: [0,1] \to \C  \) be defined by \( f(t) = 2z(1-t) + 2tw \) where \( z,w \in \C  \) are fixed complex numbers. Then \( \int_{ 0 }^{ 1 } f(t) \ dt = w - z  \). \textbf{False.}
        \end{enumerate}
    \item Give definitions of the following terms.
        \begin{enumerate}
            \item[(a)] Cauchy-Riemann equations for a pair of functions \( u,v: D \to \R  \) where \( D  \) is an open subset of \( \R^{2} \).
                \begin{solution}
                Suppose that \( u,v : D \to \R  \) where \( D  \) is an open subset of \( \R^{2} \) and \( u \) and \( v  \) contain partial derivatives that exists and are continuous on every point of \( D  \). Then the Cauchy-Riemann equations are
                \[  \frac{ \partial u  }{ \partial x  }  = \frac{ \partial v }{ \partial y } \ \ \text{and} \ \ \frac{ \partial u }{  \partial y  }  = -\frac{ \partial v   }{ \partial x   }.  \]
                \end{solution}
            \item[(b)] A piecewise smooth curve in \( \C  \).
                \begin{solution}
                    A curve \( \alpha: [a,b] \to \C  \) is \textbf{piecewise smoooth} if there is a partition 
                    \[  a = {a}_{0} < {a}_{1} < \cdots < {a}_{n} = b \]
                    such that \( \alpha \big|_{[{a}_{i-1}, {a}_{i}]} \) is smooth for \( i = 1,2,\dots, n \).
                \end{solution}
            \item[(c)] A complex line integral of a continuous function \( f: D \to \C  \) over a piecewise smooth curve in \( D  \), where \( D \subseteq \C . \)
                \begin{solution}
                    Assume \( \alpha: [a,b] \to \C  \) is a piecewise smooth curve with partition
                    \[  a = {a}_{0} < {a}_{1} < \dots, < {a}_{n} = b \]
                    such that \( \alpha \big|_{[{a}_{i-1}, {a}_{i}]} \) smooth for \( i = 1,2,\dots, n \). Let \( f: D \to \C  \) continuous and \( \alpha([a,b]) \subseteq  D  \). Then we define
                    \[  \int_{ \alpha }^{  } f(z) \ dz = \sum_{ i=1  }^{ n } \int_{ \alpha |_{[{a}_{i-1}, {a}_{i}]} } f(z) \ dz \]
                    to be the \textbf{complex line integral of \( f  \) over a piecewise smoooth curve in \( D  \), where \( D \subseteq \C  \)}.
                    
                \end{solution}
        \end{enumerate}
    \item Suppose that \( D \subseteq \C  \) is open and connected. Assume that \( f: D \to \C  \) is holomorphic. Assume that \( g: D \to \C  \) defined by \( g(z) = \overline{f(z)} \) is also holomorphic. Prove that \( f  \) and \( g  \) are both constant functions. Use this to deduce that \( f: \C \to \C  \) defined by \( f(z) = \overline{\sin z} \) is not holomorphic.
        \begin{proof}
        Our goal is to show that \( f  \) and \( g  \) are both constant. It suffices to show that \( \Re(f) \) is constant and \( \Re(g) = \Re(\overline{f}) \) is constant. Note that \( f = u + iv  \) with \( u,v  \) are real-valued functions. Since \( f  \) and \( g  \) are holomorphic, we 
        \begin{align*}
        \frac{\partial f }{\partial x }  &= \frac{\partial u }{\partial x }  + i \frac{\partial v }{\partial x } \\   
        \frac{\partial \overline{f} }{\partial x  }             &= \frac{\partial u }{\partial x }  - i\frac{\partial v }{\partial x } 
    \end{align*}
        
        Note that if \( f  \) and \( g  \) are holomorphic, then we have
        \begin{align*}
            \frac{ \partial f  }{  \partial \overline{z} } &= \frac{ 1 }{ 2 }  \Big[ \frac{ \partial f  }{  \partial x  }  + i \frac{ \partial f  }{  \partial y  } \Big] = 0 \tag{1}  \\
            \frac{ \partial \overline{f} }{ \partial \overline{z } } &= \frac{ 1 }{ 2 }  \Big[ \frac{ \partial \overline{f} }{ \partial x  }  + i \frac{ \partial \overline{f} }{ \partial y  } \Big] = 0 \tag{2}. 
        \end{align*}
        Adding equations (1) and (2), we have 
        \begin{align*}
            \frac{ 1 }{ 2 }  \Big[ \frac{\partial f }{\partial x }  + i \frac{\partial f }{\partial y } \Big] = - \frac{ 1 }{ 2 }  \Big[ \frac{\partial \overline{f} }{\partial x }  + i \frac{\partial \overline{f} }{\partial y } \Big] &\implies \frac{\partial f }{\partial x }  + i \frac{\partial f }{\partial y }  = -\Big[ \frac{\partial \overline{f} }{\partial x  }  + i \frac{\partial \overline{f} }{\partial y } \Big].
        \end{align*} 
        Furthermore, we have 
        \begin{align*}
          \Big(  \frac{\partial u }{\partial x }  + i \frac{\partial v }{\partial x }   \Big) + i \Big( \frac{\partial u }{\partial y }  - i \frac{\partial v }{\partial y }    \Big) = -\Big(  \frac{\partial u }{\partial x }  - i \frac{\partial v }{\partial x }  \Big) - i  \Big( \frac{\partial u }{\partial y } - i\frac{\partial v }{\partial y }    \Big)   
        \end{align*}
        and hence,
        \[  \frac{\partial u }{\partial x }  + i \frac{\partial v }{\partial x }  = - \frac{\partial u }{\partial x }  + i \frac{\partial v }{\partial x } \implies 2 \frac{\partial u }{\partial x }  = 0 \implies \frac{\partial u }{\partial x }  = 0.    \]
        Since \( u = \Re(f) \), we can see that \( \Re(f) \) is a constant function. Since \( \Re(f) = \Re(\overline{f}) \) (By Exercise 4(a) of Homework 5), we also have that \( \Re(\overline{f}) \) is a constant function. Thus, \( f  \) and \( g  \) are constant functions.

        To show that \( f(z) = \overline{\sin z } \) is not holomorphic, we can just show that the real and imaginary parts of \( \overline{\sin z } \) are not constant. By Exercise 6 (a) of homework 5, we see that  
        \[  h(z) = \sin z = \frac{ 1 }{ 2 }  (e^{-y} + e^{y}) \sin x  + i \Big[ - \frac{ 1 }{ 2 }  (e^{-y} + e^{y}) \cos x \Big] \]
        is a holomorphic function on \( \C  \). Note that \( f(z) = \overline{h(z)} \).
        Hence, we see that 
        \[  \overline{\sin z} = \frac{ 1 }{ 2 }  (e^{-y} + e^{y}) \sin x  + i \Big[ \frac{ 1 }{ 2 }  (e^{-y} + e^{y}) \cos x \Big]. \]
        Clearly, both real and imaginary parts of \( h(z) = \overline{\sin z } \) and \( f(z) = \sin z  \) are NOT constant. Hence, \( f(z) = \overline{\sin z } \) is NOT a holomorphic function.   
        \end{proof}
    \item Assume that \( f: [a,b] \to \C  \) be integrable.
        \begin{enumerate}
            \item[(a)] Prove that \( \Big| \int_{ a }^{ b } f(t) \ dt \Big| \leq \int_{ a }^{ b }  | f(t) |  \ dt \).
                \begin{proof}
                    Let \( x: [a,b] \to \R  \) and \( y: [a,b] \to \R  \) such that \( f(t) = x(t) + i y(t) \). Our goal is to show that 
                    \[  \Big| \int_{ a }^{ b }  f(t) \ dt \Big| \leq \int_{ a }^{ b }  | f(t) |  \ dt. \]
                    It suffices to show that 
                \[  \Big| \int_{ a }^{ b } f(t) \ dt \Big|^{2} \leq \Big( \int_{ a }^{ b }  | f(t) | \ dt \Big)^{2}.  \]
                Observe that 
                \begin{align*}
                    \Big| \int_{ a }^{ b }  f(t) \ dt \Big|^{2} &= \Big(  \int_{ a }^{ b }  f(t) \ dt \Big) \overline{\Big(  \int_{ a }^{ b } f(s) \ ds \Big)} \\
                                                                &= \Big[ \int_{ a }^{ b }  (x(t) + i y(t)) \ dt \Big] \cdot \Big[ \int_{ a }^{ b } (x(s) - i y(s)) \ ds \Big] \\
                                                                &= \int_{ a }^{ b } \int_{ a }^{ b }  (x(t) + iy(t) ) (x(s) - i y(s)) \ dt  ds \\
                                                                &= \int_{ a }^{ b }  \int_{ a }^{ b } (x(t) x(s) + i (y(t)x(s) - x(t)y(s)) + y(t)y(s) ) \ dt ds \\
                                                                &= \int_{ a }^{ b }  \int_{ a }^{ b } [x(t) x(s) + y(s) x(t)] \ dt ds + i \int_{ a }^{ b } \int_{ a }^{ b } [y(t) x(s) - x(t) y(s)] \ dt ds.
                \end{align*}
                Note that 
                \[ \int_{ a }^{ b } \int_{ a }^{ b }  [y(t) x(s) - x(t) y(s)] \ dt \ ds =  \int_{ a }^{ b }  \int_{ a }^{ b }  y(t) x(s)  \ dt ds - \int_{ a }^{ b }  \int_{ a }^{ b }  x(t) y(s) \ dt \ ds = 0. \]
                Hence, we see that
                \[ \Big| \int_{ a }^{ b } f(t) \ dt  \Big|^{2} = \int_{ a }^{ b } \int_{ a }^{ b } [x(t) x(s) + y(s) x(t)] \ dt ds. \tag{*} \]
                Now, notice we have
                \begin{align*}
                | x(t)x(s) + y(s) x(t) | &\leq \sqrt{ ((x(t))^{2} + (y(t))^{2}) ((x(s))^{2} + (y(s))^{2}) } \\      
                                             &= \sqrt{ ((x(t))^{2} + (y(t))^{2}) } \cdot \sqrt{ ((x(s))^{2} + (y(s))^{2}) } 
            \end{align*}
            by the Cauchy-Schwarz inequality. So, (*) implies that 
            \begin{align*}
                \Big| \int_{ a }^{ b } f(t) \ dt \Big|^{2} &= \int_{ a }^{ b }  \int_{ a }^{ b }  [x(t) x(s) + y(s) x(t)] \ dt ds \\
            &\leq \int_{ a }^{ b }  \int_{ a }^{ b } \sqrt{ ((x(t))^{2} + (y(t))^{2}) }  \cdot \sqrt{ ((x(s))^{2} + (y(s))^{2}) }   \ dt  ds  \\ 
                                                       &= \Big( \int_{ a }^{ b }  \sqrt{ ((x(t))^{2} + (y(t))^{2}) }  \ dt \Big) \cdot \Big(  \int_{ a }^{ b }  \sqrt{ (x(s))^{2} + (y(s))^{2} }  \ ds  \Big) \\
                                                       &= \Big(  \int_{ a }^{ b }  \sqrt{ (x(t))^{2} + (y(t))^{2} }  \ dt \Big)^{2} \\
                                                       &= \Big( \int_{ a }^{ b } | f(t) |  \ dt \Big)^{2}.
        \end{align*}
        Hence, squaring both sides will gives us our desired result; that is,
        \[  \Big| \int_{ a }^{ b } f(t) \ dt \Big|^{2} \leq \int_{ a }^{ b } | f(t) |  \ dt. \]
                \end{proof}
            \item[(b)] Prove that \( \int_{ a }^{ b } f(t) \ dt = - \int_{ a }^{ b }  f(s) \ ds \). 
                \begin{proof}
                    By definition, \( f(t) = x(t) + i y(t) \) with \( x,y : [a,b] \to \R  \). Our goal is to show that 
                \[  \int_{ a }^{ b } f(t) \ dt = - \int_{ a }^{ b }  f(s) \ ds.  \tag{*} \]
                By definition of the complex integral, we see that 
                \[  \int_{ a }^{ b } f(t) \ dt = \int_{ a }^{ b } x(t) \ dt + i \int_{ a }^{ b }  y(t) \ dt. \]
                Since \( x(t) \) and \( y(t)  \) are real-integrable functions for all \( t \in [a,b] \), we see that 
                \[  -\int_{ a }^{ b } x(s) \ ds = \int_{ a }^{ b } x(t) \ dt  \ \ \text{and} \ \ - \int_{ a }^{ b } y(s) \ ds = \int_{ a }^{ b } y(t) \ dt.  \]
                Starting with the right-hand side of (*), we get
                \begin{align*}
                    - \int_{ a }^{ b } f(s) \ ds &= - \Big[ \int_{ a }^{ b } x(s)  \ ds + i \int_{ a }^{ b }  y(s) \ ds \Big] \\
                                                 &= - \int_{ a }^{ b }  x(s) \ ds + i \Big(  - \int_{ a }^{ b } y(s) \ ds \Big) \\
                                                 &= \int_{ a }^{ b } x(t) \ dt + i \int_{ a }^{ b } y(t) \ dt \\
                                                 &= \int_{ a }^{ b } f(t) \ dt.
                \end{align*}
               Thus, we have established (*).
                \end{proof}
        \end{enumerate}
    \item Compute the integral \( \displaystyle \int_{ 0 }^{ 2 } f(t) \ dt \), where \( f: [0,2] \to \C  \) is defined by 
        \[  f(t) = 
        \begin{cases}
            (1+i)t  &\text{if} \ 0 \leq t \leq 1 \\
            1 + i t^{2} &\text{if} \ 1 < t \leq 2. 
        \end{cases} \]
        \begin{solution}
        Observe that 
        \begin{align*}
            \int_{ 0 }^{ 2 } f(t) \ dt &= \int_{ 0 }^{ 1 }  (1+i)t \ dt + \int_{ 1 }^{ 2 }  (1+i) t^{2} \ dt \\
                                       &= \Big[ \frac{ (1+i)t^{2} }{ 2 } \Big]_{0}^{1} + \Big[ \frac{ (1+i)t^{3} }{ 3 } \Big]^{2}_{1} \\
                                       &= \frac{ 1+i }{ 2 }  + \Big(  \frac{ (1+i) 8  }{ 3 }  + \frac{ 1 + i  }{ 3 }  \Big) \\
                                       &= \frac{ 17 (1+i) }{ 6 }.
        \end{align*}
        Hence, we have
        \[  \int_{ 0 }^{ 2 }  f(t) \ dt = \frac{ 17 (1 + i) }{ 6 }. \]
        \end{solution}
\end{enumerate}




\end{document}
