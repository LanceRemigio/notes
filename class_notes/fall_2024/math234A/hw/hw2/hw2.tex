\documentclass[a4paper]{article}
\usepackage{standalone}
\usepackage{import}
\usepackage[utf8]{inputenc}
\usepackage[T1]{fontenc}
\usepackage{textcomp}
\usepackage{hyperref}
% \usepackage{fourier}
% \usepackage[dutch]{babel}
\usepackage{url}
% \usepackage{hyperref}
% \hypersetup{
%     colorlinks,
%     linkcolor={black},
%     citecolor={black},
%     urlcolor={blue!80!black}
% }
\usepackage{graphicx}
\usepackage{float}
\usepackage{booktabs}
\usepackage{enumitem}
% \usepackage{parskip}
\usepackage{emptypage}
\usepackage{subcaption}
\usepackage{multicol}
\usepackage[usenames,dvipsnames]{xcolor}

% \usepackage{cmbright}


\usepackage[margin=1in]{geometry}
\usepackage{amsmath, amsfonts, mathtools, amsthm, amssymb}
\usepackage{mathrsfs}
\usepackage{cancel}
\usepackage{bm}
\newcommand\N{\ensuremath{\mathbb{N}}}
\newcommand\R{\ensuremath{\mathbb{R}}}
\newcommand\Z{\ensuremath{\mathbb{Z}}}
\renewcommand\O{\ensuremath{\emptyset}}
\newcommand\Q{\ensuremath{\mathbb{Q}}}
\newcommand\C{\ensuremath{\mathbb{C}}}
\DeclareMathOperator{\sgn}{sgn}
\usepackage{systeme}
\let\svlim\lim\def\lim{\svlim\limits}
\let\implies\Rightarrow
\let\impliedby\Leftarrow
\let\iff\Leftrightarrow
\let\epsilon\varepsilon
\usepackage{stmaryrd} % for \lightning
\newcommand\contra{\scalebox{1.1}{$\lightning$}}
% \let\phi\varphi
\renewcommand\qedsymbol{$\blacksquare$}




% correct
\definecolor{correct}{HTML}{009900}
\newcommand\correct[2]{\ensuremath{\:}{\color{red}{#1}}\ensuremath{\to }{\color{correct}{#2}}\ensuremath{\:}}
\newcommand\green[1]{{\color{correct}{#1}}}



% horizontal rule
\newcommand\hr{
    \noindent\rule[0.5ex]{\linewidth}{0.5pt}
}


% hide parts
\newcommand\hide[1]{}



% si unitx
\usepackage{siunitx}
\sisetup{locale = FR}
% \renewcommand\vec[1]{\mathbf{#1}}
\newcommand\mat[1]{\mathbf{#1}}


% tikz
\usepackage{tikz}
\usepackage{tikz-cd}
\usetikzlibrary{intersections, angles, quotes, calc, positioning}
\usetikzlibrary{arrows.meta}
\usepackage{pgfplots}
\pgfplotsset{compat=1.13}


\tikzset{
    force/.style={thick, {Circle[length=2pt]}-stealth, shorten <=-1pt}
}

% theorems
\makeatother
\usepackage{thmtools}
\usepackage[framemethod=TikZ]{mdframed}
\mdfsetup{skipabove=1em,skipbelow=0em}


\theoremstyle{definition}

\declaretheoremstyle[
    headfont=\bfseries\sffamily\color{ForestGreen!70!black}, bodyfont=\normalfont,
    mdframed={
        linewidth=2pt,
        rightline=false, topline=false, bottomline=false,
        linecolor=ForestGreen, backgroundcolor=ForestGreen!5,
    }
]{thmgreenbox}

\declaretheoremstyle[
    headfont=\bfseries\sffamily\color{NavyBlue!70!black}, bodyfont=\normalfont,
    mdframed={
        linewidth=2pt,
        rightline=false, topline=false, bottomline=false,
        linecolor=NavyBlue, backgroundcolor=NavyBlue!5,
    }
]{thmbluebox}

\declaretheoremstyle[
    headfont=\bfseries\sffamily\color{NavyBlue!70!black}, bodyfont=\normalfont,
    mdframed={
        linewidth=2pt,
        rightline=false, topline=false, bottomline=false,
        linecolor=NavyBlue
    }
]{thmblueline}

\declaretheoremstyle[
    headfont=\bfseries\sffamily\color{RawSienna!70!black}, bodyfont=\normalfont,
    mdframed={
        linewidth=2pt,
        rightline=false, topline=false, bottomline=false,
        linecolor=RawSienna, backgroundcolor=RawSienna!5,
    }
]{thmredbox}

\declaretheoremstyle[
    headfont=\bfseries\sffamily\color{RawSienna!70!black}, bodyfont=\normalfont,
    numbered=no,
    mdframed={
        linewidth=2pt,
        rightline=false, topline=false, bottomline=false,
        linecolor=RawSienna, backgroundcolor=RawSienna!1,
    },
    qed=\qedsymbol
]{thmproofbox}

\declaretheoremstyle[
    headfont=\bfseries\sffamily\color{NavyBlue!70!black}, bodyfont=\normalfont,
    numbered=no,
    mdframed={
        linewidth=2pt,
        rightline=false, topline=false, bottomline=false,
        linecolor=NavyBlue, backgroundcolor=NavyBlue!1,
    },
]{thmexplanationbox}

\declaretheorem[style=thmgreenbox, numberwithin = section, name=Definition]{definition}
\declaretheorem[style=thmbluebox, name=Example]{eg}
\declaretheorem[style=thmredbox, numberwithin = section, name=Proposition]{prop}
\declaretheorem[style=thmredbox, numberwithin = section, name=Theorem]{theorem}
\declaretheorem[style=thmredbox, numberwithin = section,  name=Lemma]{lemma}
\declaretheorem[style=thmredbox, numberwithin = section,  numbered=no, name=Corollary]{corollary}


\declaretheorem[style=thmproofbox, name=Proof]{replacementproof}
\renewenvironment{proof}[1][\proofname]{\vspace{-10pt}\begin{replacementproof}}{\end{replacementproof}}


\declaretheorem[style=thmexplanationbox, name=Proof]{tmpexplanation}
\newenvironment{explanation}[1][]{\vspace{-10pt}\begin{tmpexplanation}}{\end{tmpexplanation}}


\declaretheorem[style=thmblueline, numbered=no, name=Remark]{remark}
\declaretheorem[style=thmblueline, numbered=no, name=Note]{note}

\newtheorem*{uovt}{UOVT}
\newtheorem*{notation}{Notation}
\newtheorem*{previouslyseen}{As previously seen}
\newtheorem*{problem}{Problem}
\newtheorem*{observe}{Observe}
\newtheorem*{property}{Property}
\newtheorem*{intuition}{Intuition}


\usepackage{etoolbox}
\AtEndEnvironment{vb}{\null\hfill$\diamond$}%
\AtEndEnvironment{intermezzo}{\null\hfill$\diamond$}%
% \AtEndEnvironment{opmerking}{\null\hfill$\diamond$}%

% http://tex.stackexchange.com/questions/22119/how-can-i-change-the-spacing-before-theorems-with-amsthm
\makeatletter
% \def\thm@space@setup{%
%   \thm@preskip=\parskip \thm@postskip=0pt
% }
\newcommand{\oefening}[1]{%
    \def\@oefening{#1}%
    \subsection*{Oefening #1}
}

\newcommand{\suboefening}[1]{%
    \subsubsection*{Oefening \@oefening.#1}
}

\newcommand{\exercise}[1]{%
    \def\@exercise{#1}%
    \subsection*{Exercise #1}
}

\newcommand{\subexercise}[1]{%
    \subsubsection*{Exercise \@exercise.#1}
}


\usepackage{xifthen}

\def\testdateparts#1{\dateparts#1\relax}
\def\dateparts#1 #2 #3 #4 #5\relax{
    \marginpar{\small\textsf{\mbox{#1 #2 #3 #5}}}
}

\def\@lesson{}%
\newcommand{\lesson}[3]{
    \ifthenelse{\isempty{#3}}{%
        \def\@lesson{Lecture #1}%
    }{%
        \def\@lesson{Lecture #1: #3}%
    }%
    \subsection*{\@lesson}
    \testdateparts{#2}
}

% \renewcommand\date[1]{\marginpar{#1}}


% fancy headers
\usepackage{fancyhdr}
\pagestyle{fancy}

\fancyhead[LE,RO]{Lance Remigio}
\fancyhead[RO,LE]{\@lesson}
\fancyhead[RE,LO]{}
\fancyfoot[LE,RO]{\thepage}
\fancyfoot[C]{\leftmark}

\makeatother




% notes
\usepackage{todonotes}
\usepackage{tcolorbox}

\tcbuselibrary{breakable}
\newenvironment{verbetering}{\begin{tcolorbox}[
    arc=0mm,
    colback=white,
    colframe=green!60!black,
    title=Opmerking,
    fonttitle=\sffamily,
    breakable
]}{\end{tcolorbox}}

\newenvironment{noot}[1]{\begin{tcolorbox}[
    arc=0mm,
    colback=white,
    colframe=white!60!black,
    title=#1,
    fonttitle=\sffamily,
    breakable
]}{\end{tcolorbox}}




% figure support
\usepackage{import}
\usepackage{xifthen}
\pdfminorversion=7
\usepackage{pdfpages}
\usepackage{transparent}
\newcommand{\incfig}[1]{%
    \def\svgwidth{\columnwidth}
    \import{./figures/}{#1.pdf_tex}
}

% %http://tex.stackexchange.com/questions/76273/multiple-pdfs-with-page-group-included-in-a-single-page-warning
\pdfsuppresswarningpagegroup=1





\pagestyle{fancy}
\fancyhf{}


\title{Math 234A: Homework 2}
\author{Lance Remigio}
\begin{document}
   \maketitle 


\lhead{Math 234A: Homework 2}
\chead{Lance Remigio}
\rhead{\thepage}


\section*{Problem 1} 
\begin{definition}[Cauchy Sequence]
    A sequence \( ({z}_{n}) \) is called a \textbf{Cauchy Sequence} if for all \( \epsilon > 0  \), there exists a positive interger \( \N  \) such that for all \( m,n \geq N  \),   
    \[  | {z}_{m} - {z}_{n}  |  < \epsilon. \]
\end{definition}
Show that a sequence \( ({z}_{n}) \) in \( \C  \) is convergent if and only if it is Cauchy.
\begin{proof}
    \( (\Longrightarrow) \) Let \( \epsilon > 0  \). If \( ({z}_{n}) \) is Cauchy, then we can find an \( N \in \N  \) such that for any \( n,m \geq N  \) such that 
    \[  | {z}_{n} - {z}_{m} | < \epsilon. \]
    Then observe that for any \( n,m \geq N  \), we have
    \[ | \Re({z}_{n}) - \Re({z}_{m}) | = | \Re({z}_{n} - {z}_{m}) | \leq | {z}_{n} - {z}_{m} | < \epsilon \]
    and similarly,
    \[ | \Im({z}_{n}) - \Im({z}_{m}) | = | \Im({z}_{n} - {z}_{m}) |  \leq | {z}_{n} - {z}_{m} | < \epsilon.   \]
    Therefore, the real and imaginary parts of \( ({z}_{n}) \) are Cauchy. Since \( \Re({z}_{n}) \) and \( \Im({z}_{n}) \) are real-valued sequences, they must converge as well. But this holds if and only if \( ({z}_{n}) \) converges. 

    \( (\Longleftarrow) \) Let \( \epsilon > 0  \). Suppose \( ({z}_{n}) \) converges. This holds if and only if \( \Re({z}_{n}) \) and \( \Im({z}_{n}) \) converge. But this holds if and only if \( \Re({z}_{n}) \) and \( \Im({z}_{n}) \) are both Cauchy since they are real-valued sequences. Thus, we can find an \( N \in \N  \) such that for any \( n,m \geq N  \), we see that 
    \[  | \Re({z}_{n}) - \Re({z}_{m}) | < \frac{ \epsilon  }{ 2  } \tag{1}  \]
    and 
    \[  | \Im({z}_{n}) - \Im({z}_{m}) | < \frac{ \epsilon  }{  2  }. \tag{2} \]
    Using (1) and (2), we see that
    \begin{align*}
        | {z}_{n} - {z}_{m} | &= | \Re({z}_{n}) + i \Im({z}_{n}) - (\Re({z}_{m}) + i \Im({z}_{m})) |  \\
                              &= | (\Re({z}_{n}) - \Re({z}_{m})) + i (\Im({z}_{n}) - \Im({z}_{m})) | \\
                              &\leq | \Re({z}_{n}) - \Re({z}_{m}) | + | \Im({z}_{n}) - \Im({z}_{m}) | \\ 
                              &< \frac{ \epsilon  }{ 2 } + \frac{ \epsilon }{ 2 } \\
                              &= \epsilon.
    \end{align*}
    Thus, we conclude that \( ({z}_{n}) \) is Cauchy.
\end{proof}

\section*{Problem 2}
\begin{enumerate}
    \item[(i)] Let \( {z}_{0} = {x}_{0} + i {\zeta}_{0} \in \C  \): Define a sequence \( ({z}_{n}) \) by \( {z}_{n+1} = \frac{ 1 }{ 2 }  ({z}_{n} + \frac{ 1 }{ {z}_{n} } ) \), \( n \geq 1  \), and \( {z}_{1} = \frac{ 1 }{ 2 }  ({z}_{0} + \frac{ 1 }{ {z}_{0} } ) \). 
        Show that 
        \[  \lim_{ n  \to \infty  } {z}_{n} = 
        \begin{cases}
            1 &\text{if} \ {x}_{0} > 0 \\
            -1 &\text{if}  \ {x}_{0} < 0 
        \end{cases}   \]
    \begin{proof}
     Let \( {z}_{0} = {x}_{0} + i {\zeta}_{0} \in \C  \). Define a sequence \( ({z}_{n}) \) by     
     \[  {z}_{n+1} = \frac{ 1 }{ 2 }  \Big(  {z}_{n} + \frac{ 1 }{ {z}_{n} }  \Big), \ n \geq 1  \]
     and
     \[  {z}_{1} = \frac{ 1 }{ 2 } \Big(  {z}_{0} + \frac{ 1 }{ {z}_{0} }  \Big). \]
     Our goal is to show that 
        \[  \lim_{ n  \to \infty  } {z}_{n} = 
        \begin{cases}
            1 &\text{if} \ {x}_{0} > 0 \\
            -1 &\text{if}  \ {x}_{0} < 0. 
        \end{cases}   \]
        Suppose \( {x}_{0} > 0  \). First, we need to show that \( {z}_{n} \neq 1  \) for all \( n \in \N  \). We proceed via induction. Let our base case be \( n = 1  \). Then we want to show that \( \Re({z}_{1}) > 0  \). Using the definition of \( {z}_{0} \), we see that  
        \begin{align*}
            {z}_{1} = \frac{ {z}_{0}^{2} + 1  }{ 2 {z}_{0} } &= \frac{ 1 }{ 2({x}_{0}^{2} + {\zeta}_{0}^{2}) } \cdot [({x}_{0} + i {\zeta}_{0})^{2}({x}_{0} - i {\zeta}_{0}) + ({x}_{0} - i {\delta}_{0}) ] \\
                                                             &= \frac{ {x}_{0}^{3} + {x}_{0} {\zeta}_{0}^{2} + {x}_{0} }{ 2 ({x}_{0}^{2} + {\zeta}_{0}^{2}) } +  i \frac{ {\zeta}_{0} {x}_{0}^{3} - {\zeta}_{0} }{ 2 ({x}_{0}^{2} + {\zeta}_{0}^{2}) }.
        \end{align*}
        Since \( {z}_{1} = \Re({z}_{1}) + i \Im({z}_{1}) \) and \( {x}_{0} > 0  \), we see that 
        \[  \Re({z}_{1}) = \frac{ {x}_{0}^{3} + {x}_{0} {\zeta}_{0}^{2} + {x}_{0} }{ 2 ({x}_{0}^{2} + {\zeta}_{0}^{2})  } > 0.   \]
        Now, suppose that the result holds for the \( n \)th case. We will show that the result holds for the \( n + 1  \) case. Set \( \alpha = \Re({z}_{n})  \) and \( \beta = \Im({z}_{n}) \). Then by definition of \( {z}_{n+1} \), we see that
        \begin{align*}
            {z}_{n+1} = \frac{ {z}_{n}^{2} + 1 }{ 2 {z}_{n} }                                                                 &= \frac{ 1 }{ 2(\alpha^{2} + \beta^{2})  }  \cdot [(\alpha + i \beta)(\alpha - i \beta) + (\alpha - i \beta)] \\
                                                                                                                              &= \frac{ \alpha^{3} + \alpha \beta^{3} + \alpha }{ 2(\alpha^{2} + \beta^{2}) } + i \frac{ \beta \alpha^{3} - \beta }{ 2(\alpha^{2} + \beta^{2}) }. 
        \end{align*}
        By equating the real and imaginary parts of \( {z}_{n+1} \) with the right-hand side of the equation above and using the inductive hypothesis that \(  \alpha = \Re({z}_{n})  > 0\), we see that
        \[  \Re({z}_{n+1}) = \frac{ \alpha^{3} + \alpha \beta^{3} + \alpha }{ 2 (\alpha^{2} + \beta^{2}) } > 0.  \]
Thus, we see that \( \Re({z}_{n}) > 0  \) for all \( n \in \N \). Consequently, this tells us that \( \Re({z}_{n}) + 1 >  1 \) which further tells us that \( | {z}_{n+1} + 1  | > 1  \). Now, define a new sequence \( ({w}_{n})  \) such that
\[  {w}_{n+1} = \frac{ {z}_{n+1} - 1  }{  {z}_{n+1} + 1  }   \]
        and note that 
        \[  {z}_{n+1} = \frac{ 1 + {z}_{n}^{2} }{  2 {z}_{n} }.  \]
        Our next step is to show that \( {w}_{n+1} = {w}_{n}^{2} \). Thus, observe that
        \begin{align*}
            {w}_{n+1} = \frac{ \frac{ 1 + {z}_{n}^{2} }{ 2 {z}_{n} }  - 1  }{ \frac{ 1 + {z}_{n}^{2} }{ 2 {z}_{n}} + 1   }  
                      = \frac{ \frac{ {z}_{n}^{2} - 2 {z}_{n} + 1}{2 {z}_{n}}  }{ \frac{ {z}_{n}^{2} + 2 {z}_{n} + 1  }{ 2 {z}_{n} }  }  
                      &= \frac{ {z}_{n}^{2} - 2 {z}_{n} + 1  }{  {z}_{n}^{2} + 2 {z}_{n} + 1  } \\
                      &= \frac{ ({z}_{n} - 1)^{2} }{ ({z}_{n} + 1)^{2} }  \\
                      &= \Big(  \frac{ {z}_{n} -1  }{ {z}_{n} + 1  }  \Big)^{2} \\
                      &= {w}_{n}^{2}.
        \end{align*}
        Now, we want to show, as a consequence of the result  above, that \( ({w}_{n}) \) is a decreasing sequence; that is, we want to show that  \( 0 < |  {w}_{n + 1}  |  < | {w}_{n} |  < 1 \). We will induct on \( n \in \N  \) to show this. Thus, let \( n = 1  \) be our base case. Observe that
        \begin{align*}
            0 < | {w}_{1 + 1} |  = | {w}_{2} | = | {w}_{1}^{2} | &= \Big| \Big(  \frac{ {z}_{1} - 1  }{  {z}_{1} + 1  } \Big)^{2} \Big|  \\
                                                             &< \Big|  \frac{ {z}_{1} - 1  }{ {z}_{1} + 1  }  \Big|  \\
                                                             &< 1.
        \end{align*}
        Let us now assume that this result holds for the \( n \)th case. Thus, we see that
        \begin{align*}
            0 < | {w}_{n+2} |  = | {w}_{n+1}^{2} | = \Big| \Big( \frac{ {z}_{n+1} - 1  }{  {z}_{n+1} + 1  }    \Big)^{2} \Big|  < | {w}_{n+1}   | &< | {w}_{n} | < 1. \\
        \end{align*}
        Thus, we see that \( 0 < | {w}_{n+1} | < | {w}_{n} |  < 1  \) holds for all \( n \in \N  \). Clearly, we see that \( ({w}_{n}) \) is bounded and decreasing for all \( n  \). Hence, \( ({w}_{n})  \) must converge, and in this case, we claim that it converges to \( 0  \). Observe that 
        \[  | {w}_{n} | = e^{\ln | {w}_{n} | }. \]
        Now, since \( 0 < | {w}_{n} |  < 1  \), we must have \( \ln | {w}_{n} | < 0  \). By applying the limit as \( n \to \infty  \), we can see that 
        \[  \lim_{ n \to \infty  }  | {w}_{n} |  = \lim_{ n \to \infty  }  e^{\ln | {w}_{n} | } = 0.  \]
        Since \( | {z}_{n+1} + 1  | > 1  \) for all \( n \in \N  \), we must have that 
        \[  | {z}_{n+1} - 1  | \to 0  \]
        as \( n \to \infty  \). Thus, we conclude that \( \lim_{ n \to \infty  } {z}_{n} = 1  \).
    Now, assume that \( {x}_{0} < 0  \). Consider the sequence \( (-{z}_{n}) \). Then observe that  
    \[  \lim_{ n \to \infty  } - {z}_{n} = - \lim_{ n \to \infty  } {z}_{n} = -1. \]
        


    \end{proof}
    \item[(ii)] Discuss convergence and divergence of 
        \[  {z}_{n} = 1 + i \frac{ (-1)^{n} }{ n^{2} } \ \text{for} \ n = 1,2,\dots \ . \]
        Let \( \Phi_n = \Arg({z}_{n}) \). Show that \( ({\Phi}_{n}) \to 0 \).
        \begin{proof}
        Observe that \( \Re({z}_{n}) \to 1  \) and \( \Im({z}_{n}) = \frac{ (-1)^{n} }{ n^{2} } \to 0 \) as \( n \to \infty  \). Since the real part and imaginary part of \( {z}_{n} \) converge, we know that \( ({z}_{n}) \) must also converge. In fact, it converges to the following value
        \[  {z}_{n} \to 1 + 0i = 1. \]
        By definition, we know that 
        \[  {\Phi}_{n} = \Arg({z}_{n}) = \tan^{-1} \Bigg(  \frac{ \Im({z}_{n}) }{ \Re({z}_{n})  }  \Bigg) \to \tan^{-1}( 0 ) = 0 \ \text{as} \ n \to \infty. \]
        \end{proof}
    \item[(iii)] Assume that \( 0 < \gamma < 1  \). Show that 
        \begin{align*}
            \sum_{ n=1  }^{ \infty  } \gamma^{n} \cos n \theta &= \frac{ \gamma \cos \theta - \gamma^{2} }{ 1 - 2 \gamma \cos \theta + \gamma^{2} } \\
            \sum_{ n=1  }^{ \infty  } \gamma^{n} \sin n \theta &= \frac{  \gamma \sin \theta   }{  1 - 2 \gamma \cos \theta + \gamma^{2} }. 
    \end{align*}
        \begin{proof}
        Set \( z = \gamma e^{i\theta} \). Consider the series  
        \[  \sum_{ n=1  }^{ \infty  } z^{n} \]
        which converges absolutely for \( | z  | < 1  \) such that 
        \[  \sum_{ n=1  }^{ \infty  } z^{n} = \frac{ z  }{  1 - z  }. \tag{1} \]
        By rearranging terms and using the polar representation of \( z \), we see that
        \begin{align*}
            \sum_{ n=1  }^{ \infty  } z^{n} &= \sum_{ n=1  }^{ \infty  } (\gamma e^{i\theta})^{n} \\
                                            &= \sum_{ n=1  }^{ \infty  } [\gamma (\cos \theta + i \sin \theta)]^{n} \\ 
                                            &= \sum_{ n=1  }^{ \infty  } \gamma^{n} (\cos \theta + i \sin \theta )^{n} \\
                                            &= \sum_{ n=1  }^{ \infty  } \gamma^{n} (\cos n \theta + i \sin n \theta ) \tag{De Moivre's Theorem} \\
                                            &= \sum_{ n=1  }^{ \infty  } \gamma^{n} \cos n \theta + i \sum_{ n=1 }^{ \infty  } \gamma^{n} \sin n \theta.  
        \end{align*}
        Thus, we have
        \[ \sum_{ n=1  }^{ \infty  } z^{n} = \sum_{ n=1  }^{ \infty  } \gamma^{n} \cos n \theta + i \sum_{ n=1  }^{ \infty  } \gamma^{n} \sin n \theta. \tag{2}    \]
        Working with the sum in (1) and using the polar representation of \( z  \), we see that
        \begin{align*}
            \frac{ z  }{  1 - z  } = \frac{ \gamma e^{i \theta} }{ 1 - \gamma e^{i \theta}  }  
                                   &= \frac{  ( \gamma \cos \theta + i \gamma \sin \theta)  }{ (1 - \gamma \cos \theta) -   i \gamma \sin \theta  }   \\
                                   &= \frac{  ( \gamma \cos \theta + i \gamma \sin \theta)  }{ (1 - \gamma \cos \theta) - i \gamma \sin \theta  } \cdot \frac{ ( 1 - \gamma \cos \theta) + i \gamma \sin \theta }{ (1 - \gamma \cos \theta ) + i \gamma \sin \theta  }    \\
                                   &= \frac{ \gamma \cos \theta - \gamma^{2}( \sin^{2} \theta + \cos^{2} \theta ) + i \gamma \sin \theta  }{  1 -  2 \gamma \cos \theta + \gamma^{2} (\sin^{2} \theta + \cos^{2} \theta ) } \\
                                   &= \frac{ \gamma \cos \theta - \gamma^{2} }{ 1 - 2 \gamma \cos \theta + \gamma^{2} } + i \frac{ \gamma \sin \theta  }{ 1 - 2 \gamma \cos \theta + \gamma^{2} }.
        \end{align*}
        Thus, we see that 
        \[  \frac{ z  }{  1 - z  }  =  \frac{ \gamma \cos \theta - \gamma^{2} }{ 1 - 2 \gamma \cos \theta + \gamma^{2} } + i \frac{ \gamma \sin \theta  }{ 1 - 2 \gamma \cos \theta + \gamma^{2} }. \tag{3}
\]
Equating the real and imaginary parts of (2) and (3), we see that
\begin{align*}
    \Re \Big(  \sum_{ n=1  }^{ \infty  } z^{n}  \Big) &= \sum_{ n=1  }^{ \infty  } \gamma^{n} \cos n \theta = \frac{ \gamma \cos \theta - \gamma^{2} }{ 1 - 2 \gamma \cos \theta + \gamma^{2} },   \\
    \Im \Big(  \sum_{ n=1  }^{ \infty  } z^{n} \Big) &= \sum_{ n=1  }^{ \infty  } \gamma^{n} \sin n \theta = \frac{ \gamma \sin \theta  }{ 1 - 2 \gamma \cos \theta + \gamma^{2} }.
\end{align*}
        \end{proof}
\end{enumerate}

\section*{Problem 3}
Let \( ({z}_{n}) \) be a sequence of non-zero complex numbers. Suppose that 
\[  \lim_{ n \to \infty  }  \Big| \frac{ {z}_{n+1}  }{ {z}_{n} }  \Big|  = L. \]
Show that \( \lim_{ n \to \infty  } \sqrt[n]{ | {z}_{n} |   }  = L  \).

\begin{proof}
Let \( \epsilon > 0  \). Our goal is to show that there exists an \( N \in \N  \) such that for any \( n \geq N  \), we have
\[ | | {z}_{n} |^{\frac{ 1 }{ n } } - L  | < \epsilon.  \]
Indeed, we can use the fact that 
\[  \lim_{ n \to \infty  }  \Big| \frac{ {z}_{n+1} }{ {z}_{n} }  \Big|  = L,  \]
to find an \( N \in \N  \) such that whenever \(  n \geq  N \), we have
\[  \Big| \frac{ {z}_{n+1} }{ {z}_{n} }  \Big| < L + \epsilon.  \]
Now, observe that
\begin{align*}
    | {z}_{n} |^{\frac{ 1 }{ n } } &= \Big| \frac{ {z}_{n} }{ {z}_{n-1} } \cdot \frac{ {z}_{n-1} }{ {z}_{n-2} } \cdot \frac{ {z}_{n-2} }{ {z}_{n-3} } \cdots \frac{ {z}_{N+1} }{ {z}_{N} }  \Big|^{\frac{ 1 }{ n } } \\ 
                                   &< \Big[ (L+\epsilon)^{n - N} \Big]^{\frac{ 1 }{ n }} | {z}_{N} |^{1/n} \\
                                   &= \Big[(L+\epsilon)\Big]^{1- \frac{ N }{ n } } | {z}_{N} |^{\frac{ 1 }{ n }}.
\end{align*}
Taking the limit as \( n \to \infty  \) on the right-hand side, we can see that
\[  | {z}_{n} |^{\frac{ 1 }{ n }} < L + \epsilon \Longleftrightarrow  | | {z}_{n} |^{1/n} - L | < \epsilon     \]
which is our desired result.

\end{proof}

\section*{Problem 4}
Determine all \( z \in \C  \) such that 
\begin{enumerate}
    \item[(i)] \( \exp(z) = -2 \)
        \begin{solution}
        Note that \( -2 = -2 + 0i \) which means that 
        \[  \Arg(-2) = \tan^{-1}(-2) = \pi. \]
        Thus, we see that 
        \begin{align*}
            z = \Log(-2) &= \ln | -2 |  + i \Arg(-2) \\
                     &= \ln 2 + i \Big( \pi + 2\pi k    \Big) \  \text{for} \ k \in \Z. 
        \end{align*}
        \end{solution}
    \item[(ii)] \( \exp(z) = -i \)
        \begin{solution}
        Note that \( -i = 0 - i \) which means that 
        \[  \Arg(-i) = \frac{ 3 \pi }{ 2 }. \]
        Thus, we see that
        \begin{align*}
            z =  \Log(-i) &= \ln | -i |  + i \Arg(-i) \\
                          &=  \ln (1) + i \Big[\frac{ 3 \pi  }{  2  }  + 2 \pi k \Big] \ \text{for} \ k \in \Z \\ 
                          &= i\Big[\frac{ 3 \pi  }{  2  }  + 2 \pi k \Big] \ \text{for} \ k \in \Z. 
        \end{align*}
        \end{solution}
    \item[(iii)] \( \sin z = 1 - i \)
        \begin{solution}
        Using the fact that 
        \[  \sin z = \frac{ e^{iz} - e^{-iz} }{ 2i }  \]
        we see that
        \begin{align*}
            \frac{ e^{iz} - e^{-iz} }{ 2i  } = 1 - i &\Longrightarrow  e^{iz} - e^{-iz} = 2i + 2  \\
                                                     &\Longrightarrow (e^{iz})^{2} - 2(i+1) e^{iz} = 1 \\ 
                                                     &\Longrightarrow w^{2} - 2(i+1)w = 1  \tag{\( w = e^{iz} \)}\\
                                                     &\Longrightarrow w^{2} -2(1+i) w + (1+i)^{2} = 1 + (1+i)^{2} \\
                                                     &\Longrightarrow (w - (1+i))^{2} = 1 + (1+i)^{2} \\
                                                     &\Longrightarrow (w - (1+i))^{2 } = 1 + 2i \\
                                                     &\Longrightarrow  w_{1,2} = \pm \sqrt{ 1 + 2i } + (1+i). 
        \end{align*}
        Now, we will convert the first term on the right-hand side in terms of its respective polar representation. Thus, we have
        \[  \pm \sqrt{ 1 + 2i }  = \pm  5^{\frac{1  }{ 4 } } e^{i \frac{ \tan^{-1}(2) }{ 2 } } = \pm 5^{1/4} \Big(  \cos \Big(  \frac{ \tan^{-1}(2) }{ 2 }  \Big) + i \sin \Big(  \frac{ \tan^{-1}(2) }{ 2 }  \Big) \Big).  \]
        Now, set 
        \[  \alpha = \Big(  5^{1/4} \cos \Big(  \frac{ \tan^{-1}(2) }{ 2 }  \Big) + 1  \Big) + i \Big(  5^{1/4} \sin \Big(  \frac{ \tan^{-1}(2) }{ 2 }  \Big) + 1  \Big). \]
        and
        \[  \zeta = \Big(  1 - 5^{1/4} \cos \Big(  \frac{ \tan^{-1}(2) }{ 2 }  \Big)  \Big) + i \Big(  1 - 5^{1/4} \sin \Big(  \frac{ \tan^{-1}(2) }{ 2 }  \Big)  \Big). \]
        \begin{align*}
            e^{iz} = \alpha  &\Longrightarrow z = \frac{ 1 }{ i } \log (\alpha)  \\
                              &\Longrightarrow z = -i [\ln | \alpha | + i \Arg(\alpha) ] \\ 
                              &\Longrightarrow z = \Arg(\alpha) - i \ln | \alpha |
        \end{align*}
        and similarly
    \[  e^{iz} = \zeta \Longrightarrow z = \{  \Arg(\zeta) - i \ln | \zeta | \}.  \]
        \end{solution}
    \item[(iv)] \( \cos z = 3 + 4i \)
        \begin{solution}
        Using the fact that 
        \[  \cos z = \frac{ e^{iz} + e^{-iz} }{ 2 },  \]
        we have
        \begin{align*}
            e^{iz} + e^{-iz} = 2 (3 + 4i) &\Longrightarrow (e^{iz})^{2} + 1 = (6+8i) e^{iz}  \\
                                          &\Longrightarrow w^{2} + 1 = (6+8i)w \\ \tag{\( w = e^{iz} \)}
                                          &\Longrightarrow w^{2} - (6+8i)w = -1 \\
                                          &\Longrightarrow w^{2} - (6+8i)w + (3+4i)^{2} = (3+4i)^{2} - 1 \\
                                          &\Longrightarrow (w - (3+4i))^{2} = -1 + (3+4i)^{2} \\
                                          &\Longrightarrow w = \pm \sqrt{ 8(3i-1) } + (3+4i)  
        \end{align*}
        Now, let us convert the first term on the right-hand side into it's polar form which is 
        \[ \pm \sqrt{ 8 (3i - 1) } = 8^{1/2} \cdot 10^{1/2} e^{i \frac{ 3 \pi }{ 8 } } = 8^{1/2} \cdot 10^{1/2} \Big(  \cos \Big(  \frac{ 3 \pi }{ 8 }  \Big) + i \sin \Big(  \frac{ 3 \pi  }{ 8 }  \Big) \Big). \]
        Let us set 
        \[  \beta = \Big(  3 + 8^{1/2}\cdot 10^{1/2} \cos \Big(  \frac{ 3 \pi }{ 8 }  \Big) \Big) + i \Big(  4 + 8^{1/2} \cdot 10^{1/2} \sin \Big(  \frac{ 3 \pi  }{ 8 }  \Big) \Big)\]
        and
        \[  \omega = \Big(  3 - 8^{1/2}\cdot 10^{1/2} \cos \Big(  \frac{ 3 \pi }{ 8 }  \Big) \Big) + i \Big(  4 - 8^{1/2} \cdot 10^{1/2} \sin \Big(  \frac{ 3 \pi  }{ 8 }  \Big) \Big)\]
        Now, we have
        \begin{align*}
            e^{iz} = \beta &\Longrightarrow z  = \frac{ 1 }{ i }  \log (\beta )  \\
                           &\Longrightarrow z = \{ -i [\ln | \beta  |  + i \Arg(\beta)] \} \\  
                           &\Longrightarrow z = \{ \Arg(\beta) - i \ln | \beta | \}.
        \end{align*}
        and
        \begin{align*}
            e^{iz} = \omega &\Longrightarrow z  = \frac{ 1 }{ i }  \log (\omega )  \\
                           &\Longrightarrow z = \{ -i [\ln | \omega  |  + i \Arg(\omega)] \} \\  
                           &\Longrightarrow z = \{ \Arg(\omega) - i \ln | \omega | \}.
        \end{align*}
        \end{solution}
\end{enumerate}

\section*{Problem 5}
Let \( ({z}_{n}) \) and \( ({w}_{n}) \) be two sequences of complex numbers. Assume that \( {z}_{n} = {w}_{n} - {w}_{n +1}  \). Show that \( \sum_{ n=1  }^{ \infty   } {z}_{n}  \) converges if and only if \( ({w}_{n}) \) converges. If \( \sum_{ n=1  }^{ \infty } {z}_{n} \) converges, show that 
\[  \sum_{ n=0 }^{ \infty  } {z}_{n} = {w}_{1} - \lim_{ n \to \infty  } {w}_{n+1}. \]
Use this to compute 
\[  \sum_{ n=1  }^{  \infty  } \frac{ 1 }{ (n+i)((n+1)+i) }.  \]
\begin{proof}
    \( (\Rightarrow) \) Let \( ({z}_{n}) \) and \( {w}_{n} \) be two sequences of complex numbers. Our goal is to show that \( ({w}_{n}) \) converges. Assume that \( {z}_{n} = {w}_{n} - {w}_{n+1} \). Since \( \sum_{ n=1  }^{ \infty  } {z}_{n} \) converges , we know that \( ({z}_{n}) \) must converge to \( 0 \). But we must also have that 
\[  | {z}_{n} |  = | {w}_{n} - {w}_{n+1} | \to 0  \]
as \( n \to \infty   \). Hence, \( ({w}_{n}) \) must be a Cauchy sequence and so, it must converge by problem 1.   

\( (\Leftarrow) \) Suppose that \( ({w}_{n}) \) converges to some \( w \in \C  \). Let
\[  {s}_{n} = \sum_{ k=1  }^{ n } {z}_{k}. \]
Our goal is to show that \( ({s}_{n}) \to 0  \) as \( n \to \infty  \). Since \( {z}_{n} = {w}_{n} - {w}_{n+1} \), we can see that      
\begin{align*}
    | {s}_{n} - {s}_{m} | = \Big| \sum_{ k=m+1  }^{ n } {z}_{k}  \Big| &= \Big| \sum_{ k=m+1  }^{ n } {w}_{k} - {w}_{k+1} \Big|   \\
                &= | ({w}_{m+1} - {w}_{m+2}) + ({w}_{m+2} - {w}_{m+3}) + \cdots + ({w}_{n} - {w}_{n+1}) | \\
                &= | {w}_{m+1} - {w}_{n+1} |.
\end{align*}
Since \( ({w}_{n})  \) converges, we must also have that \( ({w}_{n}) \) is Cauchy. Thus, we see that 
\[  | {s}_{n} - {s}_{m} | = | {w}_{m+1} - {w}_{n+1}  |  \to 0 \ \text{as} \ n  \to \infty.   \]
Thus, \( ({s}_{n}) \) must converge and hence the infinite series \( \sum_{ k=1  }^{ \infty  } {z}_{n}  \) converges. Using this result, we can now compute
\begin{align*}
    {s}_{n} = \sum_{ k=1  }^{ n } {z}_{k} &= \sum_{ k=1  }^{ n } ({w}_{k} - {w}_{k+1})  \\
                                          &=  ({w}_{1} - {w}_{2}) + ({w}_{2} - {w}_{3}) + \cdots + ({w}_{n} - {w}_{n+1}) \\
                                          &= {w}_{1} - {w}_{n+1}
\end{align*}
whereby taking the limit gives us
\[  \lim_{ n \to \infty  }  {s}_{n} = \lim_{ n \to \infty  }  ({w}_{1} - {w}_{n+1}) = {w}_{1} - \lim_{ n \to \infty  }  {w}_{n+1}. \]
Now, let us compute the following infinite series
\[  \sum_{ n=1  }^{  \infty  } \frac{ 1 }{ (n+i)((n+1)+i) }.  \]
Using the formula we have just proved, we can see that
\begin{align*}
\sum_{ n=1 }^{ \infty  } \frac{ 1  }{  (n+i)((n+1)+i) }  &= \frac{ 1 }{ (1+i) (2 + i) }  + \lim_{ n \to \infty  }  \frac{ 1 }{ (n+i)((n+1) + i) }   \\
                                                         &= \frac{ 1 }{ (1+i) (2+i) }.
\end{align*}
\end{proof}

\section*{Problem 6}

For \( z,w \in \C^{\cdot} \). Show that 
\[  \Log(zw) = \Log(z) + \Log(w) + 2 \pi i k  \]
where 
\[  k = 
\begin{cases}
    0 &\text{if} \ - \pi < \Arg(z) + \Arg(w) \leq \pi \\ 
    1 &\text{if} \ -2 \pi < \Arg(z) + \Arg(w) \leq - \pi \\
    -1 &\text{if} \ \pi < \Arg z + \Arg w \leq 2 \pi.
\end{cases} \]
\begin{proof}
Let \( z,w \in \C  \). We would like to consider three cases: 
    \begin{enumerate}
        \item[(i)] \( - \pi < \Arg(z) + \Arg(w) \leq \pi \)
        \item[(ii)] \( -2 \pi < \Arg(z) + \Arg(w) \leq - \pi \).
        \item[(iii)] \( \pi < \Arg z + \Arg w \leq 2 \pi \). 
    \end{enumerate}
Starting with case (i), we see that
\[  \Log(z) = \ln | z |  + \Arg(z) \]
and that \( k = 0  \). Thus, we see that 
we can see that 
\begin{align*}
    \Log(zw) &= \ln | zw |  + \Arg(zw) \\
             &= \ln | z | | w |  + (\Arg(z) + \Arg(w) ) \\
             &= [\ln(z) + \ln(w)] + (\Arg(z) + \Arg(w) )\\  
             &= \Big(  \ln(z) + \Arg(z)  \Big) + \Big(  \ln(w) + \Arg(w) \Big)\\
             &= \Log(z) + \Log(w). 
\end{align*}
With case (ii), we have \( k = 1  \) so, we have
\begin{align*}
    \Log(zw) &= \ln | zw |  + \Arg(zw) \\
             &= \ln | z | | w |  + (\Arg(z) + \Arg(w) + 2\pi i ) \\
             &= [\ln(z) + \ln(w)] + (\Arg(z) + \Arg(w) + 2\pi i )\\  
             &= \Big(  \ln(z) + \Arg(z)  \Big) + \Big(  \ln(w) + \Arg(w) \Big) + 2\pi i   \\
             &= \Log(z) + \Log(w) + 2\pi i k. 
\end{align*}
Lastly, with case (iii), we have \( k = -1 \) which implies that 
\begin{align*}
    \Log(zw) &= \ln | zw |  + \Arg(zw) \\
             &= \ln | z | | w |  + (\Arg(z) + \Arg(w) - 2\pi i ) \\
             &= [\ln(z) + \ln(w)] + (\Arg(z) + \Arg(w) - 2\pi i )\\  
             &= \Big(  \ln(z) + \Arg(z)  \Big) + \Big(  \ln(w) + \Arg(w) \Big) - 2\pi i   \\
             &= \Log(z) + \Log(w) - 2\pi i . 
\end{align*}

\end{proof}

\section*{Problem 7 (Hyperbolic Functions)}
For \( z \in \C  \), define 
\[  \cosh(z) = \frac{ \exp(z) + \exp(-z) }{ 2 }  \]
and
\[  \sinh(z) = \frac{ \exp(z) - \exp(-z) }{ 2 }.  \]
Show that 
\begin{enumerate}
    \item[(i)] \( \sinh(z) = -i \sin(iz) \), \( \cosh(z) = \cos(iz) \) for all \( z \in \C  \).
        \begin{proof}
        Let \( z \in \C  \). Using the formulas above, we can see that
        \begin{align*}
            -i \sin (iz) &= -i \Bigg[ \frac{ \exp(i^{2} z) - \exp(-i^{2}z) }{ 2i } \Bigg] \\
                         &= \frac{ \exp(z) - \exp(-z) }{ 2 } \\
                         &= \sinh(z)
        \end{align*}
        and
        \begin{align*}
            \cos(iz) &= \frac{ \exp(i^{2} z) + \exp(- i^{2} z) }{ 2 }  \\
                     &= \frac{ \exp(-z) + \exp(z) }{ 2 } \\
                     &= \cosh(z).
        \end{align*}
        \end{proof}
    \item[(ii)] \( \sinh(z+w) = \sin(z) \cosh(w) + \cosh(z) \sinh(w) \).
        \begin{proof}
            We will show the formulas above by using the sum formulas for cosine and sine. By part (i), we see that
            \begin{align*}
                \sinh(z+w) &= -i \sin(i (z+w)) \\
                           &= -i \sin (iz + iw) \\
                           &= -i \Big[ \sin(iz) \cos(iw) + \sin(iw) \cos(iz) \Big] \\
                           &= (  - i \sin(iz) ) \cos(iw) + (-i \sin(iw)) \cos(iz) \\
                           &= \sinh(z) \cosh(w) + \sinh(w) \cosh(z)
            \end{align*}
            and
            \begin{align*}
                \cosh(z+w) &= \cos(i(z+w)) \\
                           &= \cos(iz + iw) \\
                           &= \cos(iz)\cos(iw) - \sin(iz) \sin(iw) \\
                           &= \cosh(z)\cosh(w) + i^{2} \sin(iz) \sin(iw) \\
                           &= \cosh(z) \cosh(w) + (-i \sin(iz))(-i \sin(iw)) \\
                           &= \cosh(z) \cosh(w) + \sinh(z) \sinh(w).
            \end{align*}
        \end{proof}
    \item[(iii)] \( \cosh^{2}(z) - \sinh^{2}(z) = 1 \) for all \( z \in \C  \). 
        \begin{proof}
        Let \( z \in \C  \). Using part (i), we see that
        \begin{align*}
            \cosh^{2}(z) - \sinh^{2}(z) &= \cos^{2}(iz) - (- \sin^{2}(iz)) \\
                                        &= \cos^{2}(iz) + \sin^{2}(iz) \\ 
                                        &= 1.
        \end{align*}
        \end{proof}
    \item[(iv)] \( \cosh(z + 2 \pi i ) = \cosh(z) \) and \( \sinh(z + 2 \pi i ) = \sinh(z) \) for all \( z \in \C  \).
        \begin{proof}
        Let \( z \in \C  \). Observe that
        \[  \sinh(2\pi i) = -i \sin(2\pi i^{2}) = -i \sin(-2\pi) = \sin(2\pi) = 0 \]
        and 
        \[  \cosh(2\pi i ) = \cos(2\pi i^{2}) = \cos(-2 \pi) = \cos(2\pi) = 1. \]
        Using part (ii), we can see that
        \begin{align*}
            \cosh(z+2\pi i) &= \cosh(z) \cosh(2\pi i) + \sinh(z) \sinh(2\pi i )   \\
                            &= \cosh(z)
        \end{align*}
        and
        \begin{align*}
            \sinh(z + 2\pi i) &= \sinh(z)\cosh(2 \pi i) + \cosh(z) \sinh(2 \pi i) \\
                              &= \sinh(z).
        \end{align*}

        \end{proof}
    \item[(v)] \( \cosh(z) = \sum_{ n=0 }^{ \infty  } \frac{ z^{2n} }{ (2n)! }   \) and \( \sum_{ n=0 }^{ \infty  } \frac{ z^{2n+1} }{ (2n+1)! }  \).
        \begin{proof}
        By using the series representation of cosine and sine, we can see that  
        \begin{align*}
            \cosh(z) = \cos(iz) &= \sum_{ n=0 }^{ \infty  } \frac{ (-1)^{n} }{ (2n)! } (iz)^{2n} \\
                                &= \sum_{ n=0  }^{ \infty  } \frac{ (-1)^{n} }{ (2n)! } (-1)^{n} z^{2n} \\
                                &= \sum_{ n=0 }^{ \infty  } \frac{ z^{2n} }{ (2n)! } 
        \end{align*}
        and 
        \begin{align*}
            \sinh(z) = -i \sin(iz) &= -i \sum_{ n=0 }^{ \infty  } \frac{ (-1)^{n} }{ (2n+1)! } (iz)^{2n+1} \\
                                   &=  \sum_{ n=0  }^{ \infty  } \frac{ (-i) (-1)^{n}  }{  (2n+1)! } i^{2n+1} z^{2n+1} \\ 
                                   &= \sum_{ n=0 }^{ \infty  } \frac{ (-i) (-1)^{n} }{ (2n+1)! } (-1)^{n} i z^{2n+1} \\
                                   &= \sum_{ n=0 }^{ \infty  } \frac{ z^{2n+1} }{ (2n+1)! }.
        \end{align*}
        \end{proof}
\end{enumerate}

\section*{Problem 8}
\begin{enumerate}
    \item[(a)] Find all possible \( z \in \C  \) so that 
        \[  \sum_{ n=1  }^{ \infty  } n | z |^{n}  \] 
        converges. Use it to discuss the convergence of the series \( \sum_{ n=1  }^{  \infty  } n z^{n} \).
        \begin{solution}
        Using the ratio test, denote \( {a}_{n} = n | z |^{n} \). Then observe that         
        \begin{align*}
            \Big| \frac{ {a}_{n+1} }{ {a}_{n} }  \Big|  = \Big| \frac{ (n+1) | z |^{n+1} }{ n | z |^{n} }  \Big| &= \frac{ n+1 }{ n } \cdot | z |  \\
                                                                                                                 &= \Big(  1 + \frac{ 1 }{ n }  \Big) | z |. 
        \end{align*}
        If we take the limit as \( n \to \infty  \), we see that 
        \[  \lim_{ n \to \infty  }  \Big| \frac{ {a}_{n+1} }{ {a}_{n} }  \Big| = \lim_{ n \to \infty  } \Big( 1 + \frac{ 1 }{ n }  \Big) | z | = | z |. \]
        Note that by the ratio test, the series 
        \[  \sum_{ n=1 }^{ \infty  }n | z |^{n} \]
        converges if \( | z |  < 1  \). Thus, the series  
        \[  \sum_{ n=1 }^{ \infty  } n z^{n} \] converges for all 
        \( | z | < 1  \).
        \end{solution}
    \item Show that the series \( \sum_{ n=0 }^{ \infty  } \frac{ z^{4n+1} }{ (2n)! }  \)converges absolutely for all \( z \in \C  \).
        \begin{proof}
        Let \( z \in \C  \). We proceed via the ratio test to show that 
        \[  \sum_{ n=0  }^{  \infty  } \frac{ z^{4n+1} }{ (2n)! }  \]
        converges absolutely. Thus, observe that
        \[  {r}_{n} = \frac{ z^{4n+1} }{ (2n)! }. \]
        Hence, we have
        \begin{align*}
            \Big| \frac{ {r}_{n+1} }{ {r}_{n} }  \Big| &= \Big| \frac{ z^{4n+5} }{ (2n+2)! } \cdot \frac{ (2n)! }{ z^{4n+1} }   \Big|  \\
                                                       &= \Big| \frac{ z }{ (2n+2)(2n+1) }  \Big| \\
                                                       &= \frac{ | z |  }{ (2n+2)(2n+1) } .
        \end{align*}
        Taking the limit as \( n \to \infty  \), we see that for any fixed \( z \in \C  \) that
        \[ \lim_{ n \to \infty  }  \Big| \frac{ {r}_{n+1} }{ {r}_{n} }  \Big|  = \lim_{ n \to \infty  }  \frac{ | z |  }{ (2n+2)(2n+1) } = 0 < 1.   \]
        Thus, we see that the infinite series
        \[  \sum_{ n=0 }^{ \infty  } \frac{ z^{4n+1} }{ (2n)! }  \]
        converges. Now, observe that
        \[  \sum_{ n=0 }^{ \infty  } \frac{ z^{4n+1} }{ (2n)! } = z \sum_{ n=0 }^{ \infty  } \frac{ (z^{2} )^{2n}  }{ (2n)! } = z \cosh(z^{2}) \]
        and so we can find the sum of this infinite series granted that we pick a \( z \in \C  \).
        \end{proof}
\end{enumerate}

\end{document}
