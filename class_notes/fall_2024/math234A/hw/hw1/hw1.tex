\documentclass[a4paper]{article}
\usepackage{standalone}
\usepackage{import}
\usepackage[utf8]{inputenc}
\usepackage[T1]{fontenc}
\usepackage{textcomp}
\usepackage{hyperref}
% \usepackage{fourier}
% \usepackage[dutch]{babel}
\usepackage{url}
% \usepackage{hyperref}
% \hypersetup{
%     colorlinks,
%     linkcolor={black},
%     citecolor={black},
%     urlcolor={blue!80!black}
% }
\usepackage{graphicx}
\usepackage{float}
\usepackage{booktabs}
\usepackage{enumitem}
% \usepackage{parskip}
\usepackage{emptypage}
\usepackage{subcaption}
\usepackage{multicol}
\usepackage[usenames,dvipsnames]{xcolor}

% \usepackage{cmbright}


\usepackage[margin=1in]{geometry}
\usepackage{amsmath, amsfonts, mathtools, amsthm, amssymb}
\usepackage{mathrsfs}
\usepackage{cancel}
\usepackage{bm}
\newcommand\N{\ensuremath{\mathbb{N}}}
\newcommand\R{\ensuremath{\mathbb{R}}}
\newcommand\Z{\ensuremath{\mathbb{Z}}}
\renewcommand\O{\ensuremath{\emptyset}}
\newcommand\Q{\ensuremath{\mathbb{Q}}}
\newcommand\C{\ensuremath{\mathbb{C}}}
\DeclareMathOperator{\sgn}{sgn}
\usepackage{systeme}
\let\svlim\lim\def\lim{\svlim\limits}
\let\implies\Rightarrow
\let\impliedby\Leftarrow
\let\iff\Leftrightarrow
\let\epsilon\varepsilon
\usepackage{stmaryrd} % for \lightning
\newcommand\contra{\scalebox{1.1}{$\lightning$}}
% \let\phi\varphi
\renewcommand\qedsymbol{$\blacksquare$}




% correct
\definecolor{correct}{HTML}{009900}
\newcommand\correct[2]{\ensuremath{\:}{\color{red}{#1}}\ensuremath{\to }{\color{correct}{#2}}\ensuremath{\:}}
\newcommand\green[1]{{\color{correct}{#1}}}



% horizontal rule
\newcommand\hr{
    \noindent\rule[0.5ex]{\linewidth}{0.5pt}
}


% hide parts
\newcommand\hide[1]{}



% si unitx
\usepackage{siunitx}
\sisetup{locale = FR}
% \renewcommand\vec[1]{\mathbf{#1}}
\newcommand\mat[1]{\mathbf{#1}}


% tikz
\usepackage{tikz}
\usepackage{tikz-cd}
\usetikzlibrary{intersections, angles, quotes, calc, positioning}
\usetikzlibrary{arrows.meta}
\usepackage{pgfplots}
\pgfplotsset{compat=1.13}


\tikzset{
    force/.style={thick, {Circle[length=2pt]}-stealth, shorten <=-1pt}
}

% theorems
\makeatother
\usepackage{thmtools}
\usepackage[framemethod=TikZ]{mdframed}
\mdfsetup{skipabove=1em,skipbelow=0em}


\theoremstyle{definition}

\declaretheoremstyle[
    headfont=\bfseries\sffamily\color{ForestGreen!70!black}, bodyfont=\normalfont,
    mdframed={
        linewidth=2pt,
        rightline=false, topline=false, bottomline=false,
        linecolor=ForestGreen, backgroundcolor=ForestGreen!5,
    }
]{thmgreenbox}

\declaretheoremstyle[
    headfont=\bfseries\sffamily\color{NavyBlue!70!black}, bodyfont=\normalfont,
    mdframed={
        linewidth=2pt,
        rightline=false, topline=false, bottomline=false,
        linecolor=NavyBlue, backgroundcolor=NavyBlue!5,
    }
]{thmbluebox}

\declaretheoremstyle[
    headfont=\bfseries\sffamily\color{NavyBlue!70!black}, bodyfont=\normalfont,
    mdframed={
        linewidth=2pt,
        rightline=false, topline=false, bottomline=false,
        linecolor=NavyBlue
    }
]{thmblueline}

\declaretheoremstyle[
    headfont=\bfseries\sffamily\color{RawSienna!70!black}, bodyfont=\normalfont,
    mdframed={
        linewidth=2pt,
        rightline=false, topline=false, bottomline=false,
        linecolor=RawSienna, backgroundcolor=RawSienna!5,
    }
]{thmredbox}

\declaretheoremstyle[
    headfont=\bfseries\sffamily\color{RawSienna!70!black}, bodyfont=\normalfont,
    numbered=no,
    mdframed={
        linewidth=2pt,
        rightline=false, topline=false, bottomline=false,
        linecolor=RawSienna, backgroundcolor=RawSienna!1,
    },
    qed=\qedsymbol
]{thmproofbox}

\declaretheoremstyle[
    headfont=\bfseries\sffamily\color{NavyBlue!70!black}, bodyfont=\normalfont,
    numbered=no,
    mdframed={
        linewidth=2pt,
        rightline=false, topline=false, bottomline=false,
        linecolor=NavyBlue, backgroundcolor=NavyBlue!1,
    },
]{thmexplanationbox}

\declaretheorem[style=thmgreenbox, numberwithin = section, name=Definition]{definition}
\declaretheorem[style=thmbluebox, name=Example]{eg}
\declaretheorem[style=thmredbox, numberwithin = section, name=Proposition]{prop}
\declaretheorem[style=thmredbox, numberwithin = section, name=Theorem]{theorem}
\declaretheorem[style=thmredbox, numberwithin = section,  name=Lemma]{lemma}
\declaretheorem[style=thmredbox, numberwithin = section,  numbered=no, name=Corollary]{corollary}


\declaretheorem[style=thmproofbox, name=Proof]{replacementproof}
\renewenvironment{proof}[1][\proofname]{\vspace{-10pt}\begin{replacementproof}}{\end{replacementproof}}


\declaretheorem[style=thmexplanationbox, name=Proof]{tmpexplanation}
\newenvironment{explanation}[1][]{\vspace{-10pt}\begin{tmpexplanation}}{\end{tmpexplanation}}


\declaretheorem[style=thmblueline, numbered=no, name=Remark]{remark}
\declaretheorem[style=thmblueline, numbered=no, name=Note]{note}

\newtheorem*{uovt}{UOVT}
\newtheorem*{notation}{Notation}
\newtheorem*{previouslyseen}{As previously seen}
\newtheorem*{problem}{Problem}
\newtheorem*{observe}{Observe}
\newtheorem*{property}{Property}
\newtheorem*{intuition}{Intuition}


\usepackage{etoolbox}
\AtEndEnvironment{vb}{\null\hfill$\diamond$}%
\AtEndEnvironment{intermezzo}{\null\hfill$\diamond$}%
% \AtEndEnvironment{opmerking}{\null\hfill$\diamond$}%

% http://tex.stackexchange.com/questions/22119/how-can-i-change-the-spacing-before-theorems-with-amsthm
\makeatletter
% \def\thm@space@setup{%
%   \thm@preskip=\parskip \thm@postskip=0pt
% }
\newcommand{\oefening}[1]{%
    \def\@oefening{#1}%
    \subsection*{Oefening #1}
}

\newcommand{\suboefening}[1]{%
    \subsubsection*{Oefening \@oefening.#1}
}

\newcommand{\exercise}[1]{%
    \def\@exercise{#1}%
    \subsection*{Exercise #1}
}

\newcommand{\subexercise}[1]{%
    \subsubsection*{Exercise \@exercise.#1}
}


\usepackage{xifthen}

\def\testdateparts#1{\dateparts#1\relax}
\def\dateparts#1 #2 #3 #4 #5\relax{
    \marginpar{\small\textsf{\mbox{#1 #2 #3 #5}}}
}

\def\@lesson{}%
\newcommand{\lesson}[3]{
    \ifthenelse{\isempty{#3}}{%
        \def\@lesson{Lecture #1}%
    }{%
        \def\@lesson{Lecture #1: #3}%
    }%
    \subsection*{\@lesson}
    \testdateparts{#2}
}

% \renewcommand\date[1]{\marginpar{#1}}


% fancy headers
\usepackage{fancyhdr}
\pagestyle{fancy}

\fancyhead[LE,RO]{Lance Remigio}
\fancyhead[RO,LE]{\@lesson}
\fancyhead[RE,LO]{}
\fancyfoot[LE,RO]{\thepage}
\fancyfoot[C]{\leftmark}

\makeatother




% notes
\usepackage{todonotes}
\usepackage{tcolorbox}

\tcbuselibrary{breakable}
\newenvironment{verbetering}{\begin{tcolorbox}[
    arc=0mm,
    colback=white,
    colframe=green!60!black,
    title=Opmerking,
    fonttitle=\sffamily,
    breakable
]}{\end{tcolorbox}}

\newenvironment{noot}[1]{\begin{tcolorbox}[
    arc=0mm,
    colback=white,
    colframe=white!60!black,
    title=#1,
    fonttitle=\sffamily,
    breakable
]}{\end{tcolorbox}}




% figure support
\usepackage{import}
\usepackage{xifthen}
\pdfminorversion=7
\usepackage{pdfpages}
\usepackage{transparent}
\newcommand{\incfig}[1]{%
    \def\svgwidth{\columnwidth}
    \import{./figures/}{#1.pdf_tex}
}

% %http://tex.stackexchange.com/questions/76273/multiple-pdfs-with-page-group-included-in-a-single-page-warning
\pdfsuppresswarningpagegroup=1




\title{Math 234A: Homework 1}
\author{Lance Remigio}
\begin{document}
   \maketitle 

\begin{enumerate}
    \item  
        \begin{enumerate}
            \item[(i)] \textbf{(Parallelogram identity)} Let \( z,w \in \C  \). Show that 
                \[  | z + w  |^{2} + | z - w  |^{2} = 2 (| z  |^{2} + | w |^{2}).  \]
                \begin{proof}
                Let \( z,w \in \C  \) with \( z = x + iy  \) and \( w = u + iv  \) with \( x,y \in \R  \) and \( u,v \in \R  \). Our goal is to show that  
                \[   | z + w  |^{2} + | z - w  |^{2} = 2 (| z  |^{2} + | w |^{2}).    \]
                Consider \( | z - w  |^{2} \) and notice that 
                \[  z - w = (x - u) + i(y - v). \]
                By definition of the modulus, we have 
                \begin{align*}
                    | z - w  |^{2} &= (z - w) \overline{z - w} \\
                                   &= ((x-u) + i(y-v))((x-u) - i(y -v)) \\
                                   &= (x - u)^{2} + (y -v)^{2} \\
                                   &= x^{2} - 2xu + u^{2} + y^{2} - 2yv + u^{2} \\
                                   &= (x^{2} + y^{2}) - 2(xu + yv) + (u^{2} + v^{2}) \\
                                   &= | z |^{2} - 2(xu + yv) + | w |^{2}.
                \end{align*}
                Note that 
                \[  z + w = (x+u) + i(y+v). \]
                \begin{align*}
                    | z + w  |^{2} &= (z + w) \overline{(z+w)}   \\
                                   &= ((x+u) + i(y+v))((x+u) - i(y+v)) \\
                                   &= (x+u)^{2} + (y +v)^{2} \\
                                   &= x^{2} + 2xu + u^{2} + y^{2} + 2yv + v^{2} \\ 
                                   &= | z |^{2} + 2(xu + yv) + | w  |^{2}.
                \end{align*}
                Adding these two moduli together gives us
                \[ | z + w  |^{2} + | z - w  |^{2} = 2 | z  |^{2} + 2 | w |^{2} = 2 (| z | ^{2} + | w |^{2})    \]
                which is our desired result.

                \end{proof}
            \item[(ii)] \textbf{(Binomial Expansion)}: Let \( z,w \in \C  \) and \( n  \) be a positive integer. Show that 
                \[  (z + w)^{n} = \sum_{ k=  0  }^{  n  } \begin{pmatrix} 
                           n \\
                           k 
                          \end{pmatrix} z^{k} w^{n-k}    \]
                          where \( \begin{pmatrix} 
                                     n \\
                                     k 
                                    \end{pmatrix}  = \frac{ n!  }{ k! (n-k)! }. \)
        \end{enumerate}
        \begin{proof}
        Let \( z,w \in \C  \). We proceed via induction on \( n \in \Z^{+}  \) to show that 
                \[  (z + w)^{n} = \sum_{ k=  0  }^{  n  } \begin{pmatrix} 
                           n \\
                           k 
                          \end{pmatrix} z^{k} w^{n-k}    \]
                          where \( \begin{pmatrix} 
                                     n \\
                                     k 
                                    \end{pmatrix}  = \frac{ n!  }{ k! (n-k)! }. \)
        Let \( n =1  \) be our base case. Then we have  
        \begin{align*}
            \sum_{ k=0  }^{ 1 } \begin{pmatrix} 
                       n \\
                       k 
                      \end{pmatrix} z^{k } w^{n-k} &= \begin{pmatrix} 
                                1 \\ 
                                0 
                                \end{pmatrix} z^{0} w + \begin{pmatrix} 
                                           1 \\
                                           1
                                          \end{pmatrix} z^{1} w^{0}  \\
                                          &= (z + w)^{1},
        \end{align*}
        which tells us that the result holds in our base case. Now, suppose the result holds for \( n  \)th case. We will show the result holds for the \( n + 1  \) case. By our induction hypothesis, we see that
        \begin{align*}
            (z+w)^{n+1} &= (z+w) (z+w)^{n} \\
                        &= (z+w) \sum_{ k=0  }^{ n } \begin{pmatrix} 
                                   n \\
                                   k 
                                  \end{pmatrix} z^{k} w^{n-k} \\
                        &=  \sum_{ k=0  }^{ n } \begin{pmatrix} 
                                   n \\ 
                                   k 
                                  \end{pmatrix} z^{k+1} w^{n-k} + w \sum_{ k=0  }^{ n } \begin{pmatrix} 
                                             n \\
                                             k 
                                            \end{pmatrix} z^{k} w^{n-k + 1}.  
        \end{align*}
        Reordering indices in the first summation by setting \( m = k +1 \), we have
        \begin{align*}
            \sum_{ k=0  }^{ n } \begin{pmatrix} 
                                   n \\ 
                                   k 
                                  \end{pmatrix} z^{k+1} w^{n-k} + w \sum_{ k=0  }^{ n } \begin{pmatrix} 
                                             n \\
                                             k 
                                            \end{pmatrix} z^{k} w^{n-k + 1} &= \sum_{ m=1  }^{ n+1 } \begin{pmatrix} 
                                                       n \\
                                                       m-1
                                                      \end{pmatrix}  z^{m} w^{(n+1) -m} \\
                                                      &+ \sum_{ k=0  }^{ n } \begin{pmatrix} 
                                                                 n \\
                                                                 k 
                                                                \end{pmatrix}  z^{k} w^{(n-k) + 1}.
        \end{align*}
       Then separating the first and last term of each summation, respectively, we have
        \begin{align*}
            (z + w)^{n+1} &=  \begin{pmatrix} 
                                               n \\
                                               n 
                                              \end{pmatrix} z^{n} w + \sum_{ m=1  }^{ n } \begin{pmatrix} 
                                                         n \\
                                                         m-1
                                                     \end{pmatrix} z^{m} w^{(n-k)+1} + \sum_{ k=1 }^{ n } \begin{pmatrix} 
                                                                n \\
                                                                k 
                                                               \end{pmatrix}  z^{k } w^{n -k  + 1}  +  \begin{pmatrix} 
                                                                n \\
                                                                0  
                                                            \end{pmatrix} w^{n+1} \\
                                                            &=  \begin{pmatrix} 
                                               n \\
                                               n 
                                               \end{pmatrix} z^{n} w + \sum_{ k=1  }^{ n } \Bigg[ \begin{pmatrix} n \\ k  \end{pmatrix} + \begin{pmatrix} n \\ k - 1  \end{pmatrix}\Bigg] z^{k } w^{(n+1)-k}  +  \begin{pmatrix} 
                                                                n \\
                                                                0  
                                                            \end{pmatrix} w^{n+1}.
        \end{align*}
        Using the fact that \textbf{(Need to show this!)} 
        \[  \begin{pmatrix} n \\ k  \end{pmatrix}  + \begin{pmatrix} n \\ k - 1  \end{pmatrix}  = \begin{pmatrix} n + 1 \\ k  \end{pmatrix}     \]
        and collecting the first and last terms of the summation, we see that 
        \begin{align*}
            (z+w)^{n+1} &= \begin{pmatrix} n \\ 0  \end{pmatrix} z^{0} w^{n+1} + \sum_{ k=1  }^{ n } \begin{pmatrix} n + 1 \\ k  \end{pmatrix}  z^{k } w^{(n+1)- k} + \begin{pmatrix} n \\ n  \end{pmatrix} z^{n+1} w^{0}  \\
                        &= \sum_{ k=0  }^{ n + 1   } \begin{pmatrix} n + 1 \\ k  \end{pmatrix} z^{k } w^{(n+1) - k} 
        \end{align*}
        which completes our induction argument. 
        \end{proof}
    \item  For \( z,w \in \C  \). Define \( \langle z , w \rangle = \Re(z \overline{w}) \). (If we think of \(  \C  \) as two dimensional real vector space, then \( \langle \cdot  ,  \cdot  \rangle  \) defines an inner product on \( \C  \)). 
        \begin{enumerate}
            \item[(i)] Cauchy Schwarz Inequality: 
                Show that \( | \langle z , w \rangle |^{2} \leq | z |^{2} | w |^{2} \) for all \( z,w \in \C  \).
                \begin{proof}
                First, we would like to show that for any \( z,w \in \C  \)
                \[ \langle z , w \rangle^{2} + \langle iz , w \rangle^{2} = | z |^{2} | w |^{2}.    \]
                Let \( z,w \in \C  \). By definition of \( \langle z , w \rangle \), we see that 
                \begin{align*}
                    \langle z , w \rangle^{2} + \langle iz  , w  \rangle^{2} &= (xu + yv)^{2} + (xv - uy)^{2}. \tag{1}
    \end{align*}
    Observe that 
    \begin{align*}
        (xu + yv)^{2} &= (xu)^{2} + 2xuyv + (yv)^{2} \tag{2}
    \end{align*}
    and
    \begin{align*}
        (xv - uy)^{2} &= (xv)^{2} - 2xuyv + (uy)^{2}. \tag{3}
    \end{align*}
    Then adding (2) and (3) results in 
    \begin{align*}
        \langle z , w \rangle^{2} + \langle iz , w \rangle^{2} &= (xu)^{2} + (yv)^{2} + (xv)^{2} + (uy)^{2} \\
                                                               &= x^{2} u^{2} + y^{2} v^{2} + x^{2} v^{2} + u^{2} y^{2} \\
                                                               &= u^{2} (x^{2} + y^{2}) + v^{2} (x^{2} + y^{2}) \\
                                                               &= (x^{2} + y^{2}) (u^{2} + v^{2}) \\
                                                               &= | z  |^{2} | w |^{2}.
    \end{align*}
    Now, we need to show that \( | \langle z , w \rangle |^{2} \leq | z |^{2} | w |^{2}  \). By using the result above, we can write      
    \begin{align*}
       | \langle z , w \rangle |^{2}   &\leq | \langle z , w \rangle |^{2} + | \langle iz , w \rangle |^{2}  
                                       = | z |^{2} | w |^{2} 
    \end{align*}
    and we are done.
                \end{proof}
            \item[(ii)] Triangle Inequalities: Show 
                \[  | z + w  | \leq | z  |  + | w  | \]
                and 
                \[  | | z  |  - | w |  | \leq | z - w  |  \]
                for all \( z,w \in \C  \).
                \begin{proof}
                Let \( z,w \in \C  \). We will first show that \(  | z + w  | \leq | z  |  + | w |  \). First, we will show the following results:
                \[  | z + w  |^{2} = | z |^{2} + 2 | \langle z , w \rangle | + | w |^{2} \tag{1} \]
                and 
                \[  | z - w  |^{2} = | z |^{2} - 2 | \langle z , w \rangle | + | w |^{2}. \tag{2}  \]
                Let \( z = x + iy \) and \( w = u + i v  \) for \( x,y,u,v \in \R  \). Observe that   
                \[  z + w = (x+u) + i(y +v) \]
                and 
                \[  z - w = (x-u) + i(y - v). \]
                Using the definition of the modulus, we see that
                \begin{align*}
                    | z + w  |^{2} = (z+w) \overline{(z+w)} &= ((x+u) + i(y+v)) ((x+u) - i(y+v)) \\
                                                            &= (x+u)^{2} + (y +v)^{2} \\
                                                            &= x^{2} 2xu + u^{2} + y^{2} + 2yv + v^{2} \\
                                                            &= (x^{2} + y^{2}) + 2 (xu + yv) + (y^{2} + v^{2}) \tag{since \( \Re(z \overline{w}) = xu + yv \)} \\
                                                            &= | z |^{2} + 2 \langle z , w \rangle + | w |^{2}. \tag{since \( \langle z , w \rangle = \Re(z \overline{w}) \)}
                \end{align*} 
                Similarly, we have
                \begin{align*}
                    | z - w  |^{2} = (z-w)\overline{(z-w)} &= ((x-u) + i (y - v)) ((x-u) - i (y - v)) \\
                                                           &= (x-u)^{2} + (y - v)^{2} \\
                                                           &=  x^{2} -2xu + u^{2} + y^{2} -2yv + v^{2} \\
                                                           &= x^{2} + y^{2} -2(xu + yv) + v^{2} \\
                                                           &= (x^{2}  + y^{2}) -2 \langle z , w \rangle + (v^{2} + u^{2}) \\ 
                                                           &= | z |^{2} - 2 \langle z , w \rangle + | w |^{2}.
\end{align*}
Now, let us prove that \( | z + w  | \leq | z  |  + | w |  \). Consider \( | z + w  |^{2} \). By part (a), we see that
\begin{align*}
    | z + w  |^{2} &= | z  |^{2} + 2 \langle z , w \rangle + | w |^{2} \\
                   &\leq | z |^{2} + 2 z  w + | w |^{2} \\ 
                   &\leq | z |^{2} + 2 | z | | w |  + | w |^{2} \\
                   &= (| z |  + | w | )^{2}.
\end{align*}
By taking the square root of both sides, we see that
\[  | z + w  | \leq | z  |  + | w |. \]
To show the second inequality, consider \( | z - w  |^{2} \). Then using part (a) again, we have
\begin{align*}
    | z - w  |^{2} &= | z |^{2} - 2 \langle z , w \rangle + | w  |^{2} \\
                   & \geq | z |^{2} - 2 | z | | w |  + | w |^{2} \\ 
                   &= (| z |  - | w |)^{2}. 
\end{align*}
By taking the square root of both sides, we see that 
\[  | z - w  | \geq |  | z  |  - | w |  |.  \]



                \end{proof}
        \end{enumerate}
    \item \textbf{(Lagrange Identity)} Let \( {z}_{1}, \dots, {z}_{n}, {w}_{1}, \dots, {w}_{n} \in \C   \). Show that    
        \[  \Big| \sum_{ k=1  }^{ n } {z}_{k} {w}_{k} \Big|^{2} = \sum_{ k=1  }^{ n } | {z}_{k} |^{2} \sum_{ k=1  }^{ n } | {w}_{k } |^{2} - \sum_{ 1 \leq i < j \leq n  }^{  } | {z}_{i} \overline{{w}_{j}} - {z}_{j} \overline{{w}_{i}} |^{2}. \]
        Use this to deduce that 
        \[  \Big| \sum_{ k=1  }^{ n } {z}_{k } {w}_{k } \Big|^{2} \leq \sum_{ k=1  }^{ n } | {z}_{k} |^{2} \sum_{ k=1  }^{ n } | {w}_{k } |^{2}. \]
        \begin{proof}
        We will show that 
        \[  \sum_{ 1 \leq i < j \leq n  }^{  } | {z}_{i} \overline{{w}_{j}} - {z}_{j} \overline{{w}_{i}} |^{2} = \sum_{ k=1  }^{ n } | {z}_{k } |^{2} \sum_{ k=1  }^{ n } | {w}_{k } |^{2} - \Big|  \sum_{ k=1  }^{ n } {z}_{k } {w}_{k } \Big|^{2}. \]
        Observe that 
        \begin{align*}
            \sum_{ 1 \leq i < j \leq n  }^{  } | {z}_{i} \overline{{w}_{j}} - {z}_{j} \overline{{w}_{i}}  |^{2} &= \sum_{ i=1  }^{ n } \sum_{ j=1  }^{ n } | {z}_{i} \overline{{w}_{j}} - {z}_{j} \overline{{w}_{i}} |^{2}  \\
                                                                                                                &= \sum_{ i=1  }^{ n } \sum_{ j=1  }^{ n } ({z}_{i} \overline{{w}_{j}} - {z}_{j} \overline{{w}_{i}})(\overline{{z}_{i}} {w}_{j} - \overline{{z}_{j}} {w}_{i}) \\
                                                                                                                &= \sum_{ i=1  }^{ n } \sum_{ j=1  }^{ n } \Big[ | {z}_{i} |^{2} | {w}_{j} |^{2} - 2 {z}_{j} \overline{{z}_{i}} \overline{{w}_{i}} {w}_{j} + | {z}_{j} |^{2} | {w}_{i} |^{2} \Big] \\
                                                                                                                &= \sum_{ i=1  }^{ n } \sum_{ j=1  }^{ n } | {z}_{i} |^{2} | {w}_{j} |^{2} - 2 \sum_{ i=1  }^{ n }\sum_{ j=1  }^{ n } {z}_{j} \overline{{z}_{i}} \overline{{w}_{i}} {w}_{j}
                                                                                                                + \sum_{ i=1  }^{ n } \sum_{ j=1  }^{ n } | {z}_{j} |^{2} | {w}_{i} |^{2} \\
                                                                                                                &= \sum_{ i=1  }^{ n } | {z}_{i} |^{2} \sum_{ j=1  }^{ n } | {w}_{i} |^{2} + \sum_{ i=1  }^{ n } | {w}_{i} |^{2} \sum_{ j=1  }^{ n } | {z}_{j} |^{2} - 2 \sum_{ i=1  }^{ n } \sum_{ j = 1   }^{ n } {z}_{j} \overline{{z}_{i}} \overline{{w}_{i}} {w}_{j} \\
                                                                                                                &=  \sum_{ i=1  }^{ n  }| {z}_{i} | \sum_{ i=1  }^{ n   } | {w}_{i} |^{2} - \sum_{ i=1  }^{ n }  \overline{{z}_{i} {w}_{i}} \sum_{ i=1  }^{ n } {z}_{i} {w}_{i} \tag{for \( i = j  \)} \\  
                                                                                                                &=  \sum_{ i=1  }^{ n } | {z}_{i} |^{2} \sum_{ i=1  }^{ n  } | {w}_{i} |^{2} - \Big| \sum_{ i=1  }^{ n } {z}_{i} {w}_{i}  \Big|^{2}.
        \end{align*}
        Thus, we conclude that 
        \begin{align*}   \Big| \sum_{ k=1  }^{ n } {z}_{k} {w}_{k} \Big|^{2} &= \sum_{ k=1  }^{ n } | {z}_{k} |^{2} \sum_{ k=1  }^{ n } | {w}_{k } |^{2} - \sum_{ 1 \leq i < j \leq n  }^{  } | {z}_{i} \overline{{w}_{j}} - {z}_{j} \overline{{w}_{i}} |^{2} \\  
            &\leq \sum_{ k=1  }^{ n } | {z}_{k } |^{2} \sum_{ k=1  }^{ n } | {w}_{k } |^{2}.  
        \end{align*}

        \end{proof}
    \item Express the following complex number in the form \( \alpha + i \beta  \):
        \begin{enumerate}
            \item[(i)] \( (1 + i)^{-1} \)
                \begin{solution}
                   Observe that  
                   \[  (1 + i)^{-1} = \frac{ 1 }{  1 + i }  \]
                   and that 
                   \begin{align*}
                      \frac{ 1 }{  1 + i  }  \cdot \frac{ (1 - i)  }{  (1- i) } = \frac{ (1- i) }{ 1 - i^{2} } = \frac{ 1 - i  }{ 2 } = \frac{ 1 }{ 2 }  - \frac{ 1 }{ 2 }  i.
                   \end{align*}
                \end{solution}
            \item[(ii)] \( (1 + i) / 2i \)
                \begin{solution}
                 Observe that    
                 \begin{align*}
                     \frac{ (1 + i) }{ 2i  }  = \frac{ 1 }{ 2i }  (1 + i) = \frac{ 1 }{ 2i }  + \frac{ 1 }{ 2 } = \frac{ 1 }{ 2 }  - \frac{ 1 }{ 2 }  i.  
                 \end{align*}
                \end{solution}
            \item[(iii)] \( (5 + 5i)^{10} \)
                \begin{solution}
                 Let \( z = 1 + i \). Observe that we can write   
                 \[  (5 + 5i)^{10} = 5^{10} (1 + i)^{10}. \]
                 Note that  
                 \[  r = \sqrt{ 1^{2} + 1^{2} }  = \sqrt{ 2 }. \]
                 Furthermore, we have
                 \[  \tan^{-1}(  1 / 1  ) = \tan^{-1}(1) = \frac{ \pi  }{ 4 }. \]
                 Using De Moivre's formula, we can write
                 \begin{align*}
                     z^{10} &= (\sqrt{ 2 })^{10}  (\cos(10 \theta ) + i \sin(10 \theta) \\
                           &=  (\sqrt{ 2 } )^{10} (\cos(5 \pi / 2 ) + i \sin( 5 \pi / 2))
                           &= (\sqrt{ 2 } )^{10} i.
                \end{align*}
                Then we have
                \[  (5 + 5i)^{10} = 5^{10} (\sqrt{ 2 } )^{10} i = 312500000 i. \]
            \end{solution}
            \item[(iv)] \( \Big(  \frac{ 2 + i  }{ 3 -2i }  \Big)^{2} \)
                \begin{solution}
                Our first step is to get \( \frac{ 2 + i  }{  3 -2i }  \) in terms of \( \alpha + i \beta  \). Thus, observe that
                \[  \frac{ 2 + i  }{ 3 - 2i } = \frac{ 2 + i  }{ 3 - 2i } \cdot \frac{ 3 + 2i }{ 3 + 2i } = \frac{ 7i + 4  }{ 13 } = \frac{ 4 }{ 13 }  + i \frac{ 7 }{ 13 }.    \]
               Furthermore, we have 
               \[  \Big(  \frac{ 4 }{ 13 } + i \frac{ 7 }{ 13 }  \Big)^{2} = \frac{ 1 }{ 169 } ( 4 + 7i)^{2} = \frac{ 1 }{ 169  } (16 + 46i - 49) = \frac{ 1 }{ 169  }(-33 + 46i).    \]
               Thus, we have that 
               \[  \Big(  \frac{ 2 + i  }{ 3 -2i }  \Big)^{2} = \frac{ -33 }{ 169  } + \frac{ 46 }{ 169 }i  \]
                \end{solution}
            \item[(v)] \( \Big(  \frac{ - 1 + i \sqrt{ 3 }  }{ 2 }  \Big)^{3} \).
                \begin{solution}
                   Denote \( z = \frac{ -1 }{ 2 }  + i \frac{ \sqrt{ 3 }  }{ 2 }  \). Then observe that  
                   \[  \theta = \tan^{-1} \Big(  \frac{ \sqrt{ 3 } / 2 }{ -1 / 2 }  \Big) = \tan^{-1}(- \sqrt{ 3 } ) = \frac{ 2 \pi  }{ 3 }.   \]
                   Furthermore, we have
                   \[  r = \sqrt{ \Big(  \frac{ 1 }{ 2 }  \Big)^{2} + \Big(  \frac{ \sqrt{ 3 }  }{ 2 }  \Big)^{2} } = 1.   \]
                   Using De Moivre's formula, we have that 
                   \[  z^{3} = 1^{3} \cdot \Big(  \cos \Big( 3 \cdot \frac{ 2 \pi  }{ 3 }    \Big) + i \sin \Big(  3 \cdot \frac{ 2 \pi  }{ 3 }  \Big) \Big) = \cos(2 \pi) + i \sin(2 \pi ) = 1 + i0 = 1.  \]
                \end{solution}
        \end{enumerate}
    \item Let \( z \in \C^{\cdot} \) and \( z = \gamma ( \cos \varphi + i \sin \varphi)\) where \( n \in \Z^{+}  \) and  
        \[  w = \gamma^{1/n} \Bigg[ \cos \Big(  \frac{ \varphi + 2 \pi k  }{ n  } \Big)  + i \sin \Big(  \frac{ \varphi + 2 \pi k  }{ n  }  \Big)\Bigg]  \]
        where \( k \in \Z  \). Show that \( w^{n} = z  \).
        \begin{proof}
        Note that for any \( n \in \N  \) that 
        \[ z^{n} = ( \cos \varphi + i \sin \varphi )^{n} = \cos n \varphi + i \sin n \varphi.    \]
        Thus, we have
        \begin{align*}
            w^{n} &= \Bigg(  \gamma^{1/n} \Bigg[  \cos \Bigg(  \frac{ \varphi + 2 \pi k  }{ n  } \Big)  + i \sin \Big(  \frac{ \varphi + 2 \pi k  }{ n  }  \Big) \Bigg] \Bigg)^{n} \\
                  &= (\gamma^{1/n})^{n} \Bigg[ \cos \Big(  \frac{ \varphi + 2 \pi k  }{ n  } \Big)  + i \sin \Big(  \frac{ \varphi + 2 \pi k  }{ n  }  \Big) \Bigg]^{n} \\
                  &= \gamma \Bigg[ \cos \Big( n \cdot  \frac{ \varphi + 2 \pi k  }{ n }  \Big) + i \sin \Big(  n \cdot \frac{  \varphi + 2 \pi k  }{ n  }  \Big) \Bigg] \\   
                  &= \gamma (\cos \varphi + i \sin \varphi) \\
                  &= z
        \end{align*}
        which ends our proof.
        \end{proof}
    \item \textbf{(Computing fourth roots)}: Find your distinct complex numbers \( w  \) such that \( w^{4} = z  \) for 
        \begin{enumerate}
            \item[(i)] \( z = i  \).
                \begin{solution}
                
                \end{solution}
            \item[(ii)] \( z = -i \).
                \begin{solution}
                
                \end{solution}
            \item[(iii)] \( z = 1  \).
                \begin{solution}
                
                \end{solution}
            \item[(iv)] \( z = -1  \).
                \begin{solution}
                
                \end{solution}
        \end{enumerate}
    \item Sketch the following sets in \( \C  \).
        \begin{enumerate}
            \item[(i)] \( \zeta = \{ z \in \C : \Re((1+i)z - 2) = 0 \}   \).
                \begin{solution}
                
                \end{solution}
            \item[(ii)] Let \( a,c \in \R  \) and \( b \in \C  \) with \( b \overline{b } - a > 0   \) and
                \[  \zeta = \{ z \in \C : a | z  |^{2} + \overline{b}z + b \overline{z} + c = 0    \}. \]
                \begin{solution}
                
                \end{solution}
            \item[(iii)] \( \zeta  = \{ z \in \C : | z - i  |  = 2  \}  \).
                \begin{solution}
                    
                \end{solution}
        \end{enumerate}
    \item Let \( z,a \in \C  \). 
        \begin{enumerate}
            \item[(i)] Show that \( |  1 - z \overline{a} |^{2} - | z - a  |^{2} = (1 - | z | )^{2} (1 - | a |^{2}) \).
                \begin{proof}
                
                \end{proof}
            \item[(ii)] Assume that \( | a  |  < 1  \). Show that 
                \[  | z  |  < 1 \iff \Big| \frac{ z - a  }{ 1 - \overline{a} z  } \Big| < 1     \]
                and
                \[  | z  |  = 1 \iff \Big| \frac{ z - a  }{  1 - \overline{a} z  }  \Big|  = 1. \]
                \begin{proof}
                
                \end{proof}
        \end{enumerate}

\end{enumerate}




\end{document}
