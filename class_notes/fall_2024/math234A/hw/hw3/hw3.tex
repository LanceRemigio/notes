\documentclass[a4paper]{article}
\usepackage{standalone}
\usepackage{import}
\usepackage[utf8]{inputenc}
\usepackage[T1]{fontenc}
% \usepackage{fourier}
\usepackage{textcomp}
\usepackage{hyperref}
\usepackage[english]{babel}
\usepackage{url}
% \usepackage{hyperref}
% \hypersetup{
%     colorlinks,
%     linkcolor={black},
%     citecolor={black},
%     urlcolor={blue!80!black}
% }
\usepackage{graphicx} \usepackage{float}
\usepackage{booktabs}
\usepackage{enumitem}
% \usepackage{parskip}
% \usepackage{parskip}
\usepackage{emptypage}
\usepackage{subcaption}
\usepackage{multicol}
\usepackage[usenames,dvipsnames]{xcolor}
\usepackage{ocgx}
% \usepackage{cmbright}


\usepackage[margin=1in]{geometry}
\usepackage{amsmath, amsfonts, mathtools, amsthm, amssymb}
\usepackage{thmtools}
\usepackage{mathrsfs}
\usepackage{cancel}
\usepackage{bm}
\newcommand\N{\ensuremath{\mathbb{N}}}
\newcommand\R{\ensuremath{\mathbb{R}}}
\newcommand\Z{\ensuremath{\mathbb{Z}}}
\renewcommand\O{\ensuremath{\emptyset}}
\newcommand\Q{\ensuremath{\mathbb{Q}}}
\newcommand\C{\ensuremath{\mathbb{C}}}
\newcommand\F{\ensuremath{\mathbb{F}}}
\DeclareMathOperator{\sgn}{sgn}
\DeclareMathOperator{\diam}{diam}
\DeclareMathOperator{\LO}{LO}
\DeclareMathOperator{\UP}{UP}
\DeclareMathOperator{\card}{card}
\DeclareMathOperator{\Arg}{Arg}
\DeclareMathOperator{\Dom}{Dom}
\DeclareMathOperator{\Log}{Log}
\DeclareMathOperator{\dist}{dist}
% \DeclareMathOperator{\span}{span}
\usepackage{systeme}
\let\svlim\lim\def\lim{\svlim\limits}
\renewcommand\implies\Longrightarrow
\let\impliedby\Longleftarrow
\let\iff\Longleftrightarrow
\let\epsilon\varepsilon
\usepackage{stmaryrd} % for \lightning
\newcommand\contra{\scalebox{1.1}{$\lightning$}}
% \let\phi\varphi
\renewcommand\qedsymbol{$\blacksquare$}

% correct
\definecolor{correct}{HTML}{009900}
\newcommand\correct[2]{\ensuremath{\:}{\color{red}{#1}}\ensuremath{\to }{\color{correct}{#2}}\ensuremath{\:}}
\newcommand\green[1]{{\color{correct}{#1}}}

% horizontal rule
\newcommand\hr{
    \noindent\rule[0.5ex]{\linewidth}{0.5pt}
}

% hide parts
\newcommand\hide[1]{}

% si unitx
\usepackage{siunitx}
\sisetup{locale = FR}
% \renewcommand\vec[1]{\mathbf{#1}}
\newcommand\mat[1]{\mathbf{#1}}

% tikz
\usepackage{tikz}
\usepackage{tikz-cd}
\usetikzlibrary{intersections, angles, quotes, calc, positioning}
\usetikzlibrary{arrows.meta}
\usepackage{pgfplots}
\pgfplotsset{compat=1.13}

\tikzset{
    force/.style={thick, {Circle[length=2pt]}-stealth, shorten <=-1pt}
}

% theorems
\makeatother
\usepackage{thmtools}
\usepackage[framemethod=TikZ]{mdframed}
\mdfsetup{skipabove=1em,skipbelow=1em}

\theoremstyle{definition}

\declaretheoremstyle[
    headfont=\bfseries\sffamily\color{ForestGreen!70!black}, bodyfont=\normalfont,
    mdframed={
        linewidth=1pt,
        rightline=false, topline=false, bottomline=false,
        linecolor=ForestGreen, backgroundcolor=ForestGreen!5,
    }
]{thmgreenbox}

\declaretheoremstyle[
    headfont=\bfseries\sffamily\color{NavyBlue!70!black}, bodyfont=\normalfont,
    mdframed={
        linewidth=1pt,
        rightline=false, topline=false, bottomline=false,
        linecolor=NavyBlue, backgroundcolor=NavyBlue!5,
    }
]{thmbluebox}

\declaretheoremstyle[
    headfont=\bfseries\sffamily\color{NavyBlue!70!black}, bodyfont=\normalfont,
    mdframed={
        linewidth=1pt,
        rightline=false, topline=false, bottomline=false,
        linecolor=NavyBlue
    }
]{thmblueline}

\declaretheoremstyle[
    headfont=\bfseries\sffamily, bodyfont=\normalfont,
    numbered = no,
    mdframed={
        rightline=true, topline=true, bottomline=true,
    }
]{thmbox}

\declaretheoremstyle[
    headfont=\bfseries\sffamily, bodyfont=\normalfont,
    numbered=no,
    % mdframed={
    %     rightline=true, topline=false, bottomline=true,
    % },
    qed=\qedsymbol
]{thmproofbox}

\declaretheoremstyle[
    headfont=\bfseries\sffamily\color{NavyBlue!70!black}, bodyfont=\normalfont,
    numbered=no,
    mdframed={
        rightline=false, topline=false, bottomline=false,
        linecolor=NavyBlue, backgroundcolor=NavyBlue!1,
    },
]{thmexplanationbox}

\declaretheorem[
    style=thmbox, 
    % numberwithin = section,
    numbered = no,
    name=Definition
    ]{definition}

\declaretheorem[
    style=thmbox, 
    name=Example,
    ]{eg}

\declaretheorem[
    style=thmbox, 
    % numberwithin = section,
    name=Proposition]{prop}

\declaretheorem[
    style = thmbox,
    numbered=yes,
    name =Problem
    ]{problem}

\declaretheorem[style=thmbox, name=Theorem]{theorem}
\declaretheorem[style=thmbox, name=Lemma]{lemma}
\declaretheorem[style=thmbox, name=Corollary]{corollary}

\declaretheorem[style=thmproofbox, name=Proof]{replacementproof}

\declaretheorem[style=thmproofbox, 
                name = Solution
                ]{replacementsolution}

\renewenvironment{proof}[1][\proofname]{\vspace{-1pt}\begin{replacementproof}}{\end{replacementproof}}

\newenvironment{solution}
    {
        \vspace{-1pt}\begin{replacementsolution}
    }
    { 
            \end{replacementsolution}
    }

\declaretheorem[style=thmexplanationbox, name=Proof]{tmpexplanation}
\newenvironment{explanation}[1][]{\vspace{-10pt}\begin{tmpexplanation}}{\end{tmpexplanation}}

\declaretheorem[style=thmbox, numbered=no, name=Remark]{remark}
\declaretheorem[style=thmbox, numbered=no, name=Note]{note}

\newtheorem*{uovt}{UOVT}
\newtheorem*{notation}{Notation}
\newtheorem*{previouslyseen}{As previously seen}
% \newtheorem*{problem}{Problem}
\newtheorem*{observe}{Observe}
\newtheorem*{property}{Property}
\newtheorem*{intuition}{Intuition}

\usepackage{etoolbox}
\AtEndEnvironment{vb}{\null\hfill$\diamond$}%
\AtEndEnvironment{intermezzo}{\null\hfill$\diamond$}%
% \AtEndEnvironment{opmerking}{\null\hfill$\diamond$}%

% http://tex.stackexchange.com/questions/22119/how-can-i-change-the-spacing-before-theorems-with-amsthm
\makeatletter
% \def\thm@space@setup{%
%   \thm@preskip=\parskip \thm@postskip=0pt
% }
\newcommand{\oefening}[1]{%
    \def\@oefening{#1}%
    \subsection*{Oefening #1}
}

\newcommand{\suboefening}[1]{%
    \subsubsection*{Oefening \@oefening.#1}
}

\newcommand{\exercise}[1]{%
    \def\@exercise{#1}%
    \subsection*{Exercise #1}
}

\newcommand{\subexercise}[1]{%
    \subsubsection*{Exercise \@exercise.#1}
}


\usepackage{xifthen}

\def\testdateparts#1{\dateparts#1\relax}
\def\dateparts#1 #2 #3 #4 #5\relax{
    \marginpar{\small\textsf{\mbox{#1 #2 #3 #5}}}
}

\def\@lesson{}%
\newcommand{\lesson}[3]{
    \ifthenelse{\isempty{#3}}{%
        \def\@lesson{Lecture #1}%
    }{%
        \def\@lesson{Lecture #1: #3}%
    }%
    \subsection*{\@lesson}
    \testdateparts{#2}
}

% \renewcommand\date[1]{\marginpar{#1}}


% fancy headers
\usepackage{fancyhdr}
\pagestyle{fancy}

\makeatother

% notes
\usepackage{todonotes}
\usepackage{tcolorbox}

\tcbuselibrary{breakable}
\newenvironment{verbetering}{\begin{tcolorbox}[
    arc=0mm,
    colback=white,
    colframe=green!60!black,
    title=Opmerking,
    fonttitle=\sffamily,
    breakable
]}{\end{tcolorbox}}

\newenvironment{noot}[1]{\begin{tcolorbox}[
    arc=0mm,
    colback=white,
    colframe=white!60!black,
    title=#1,
    fonttitle=\sffamily,
    breakable
]}{\end{tcolorbox}}

% figure support
\usepackage{import}
\usepackage{xifthen}
\pdfminorversion=7
\usepackage{pdfpages}
\usepackage{transparent}
\newcommand{\incfig}[1]{%
    \def\svgwidth{\columnwidth}
    \import{./figures/}{#1.pdf_tex}
}

% %http://tex.stackexchange.com/questions/76273/multiple-pdfs-with-page-group-included-in-a-single-page-warning
\pdfsuppresswarningpagegroup=1



\pagestyle{fancy}
\fancyhf{}

\title{Math 234A: Homework 3}
\author{Lance Remigio}

\begin{document}
\maketitle    
\lhead{Math 234A: Homework 3}
\chead{Lance Remigio}
\rhead{\thepage}

\begin{problem}[Complex Logarithms]
    Compute the following:
    \begin{enumerate}
        \item[(i)] \( \Log(i)  \) and \( \log(i) \)
        \item[(ii)] \( \Log(1+i)   \) and \( \log(1+i) \).
        \item[(iii)] \( \Log(-1)  \) and \( \log(-1) \).
    \end{enumerate}
    Note: for \( z \in \C^{\bullet} \), \( \log z  \) is a set not a single number.
\end{problem}

\begin{solution}
    
\end{solution}

\begin{problem}[Complex Powers]
   Compute the following: 
   \begin{enumerate}
       \item[(i)] \( (1 + i)^{3 +i} \)
        \item[(ii)] \( \Big(  \frac{ 1 + i  }{  1 - i  }  \Big)^{i} \)
        \item[(iii)] \( (-e)^{i/2} \).
   \end{enumerate}
\end{problem}
\begin{solution}

\end{solution}

\begin{problem}
   \begin{enumerate}
       \item[(a)] Let \( A \subseteq  \C  \). Show that the following statements are equivalent.
           \begin{enumerate}
               \item[(i)] \( A  \) is closed.
                \item[(ii)] For any sequence \( ({a}_{n}) \) in \( A  \) such that \( {a}_{n} \to a \in \C  \) implies that \( a \in A  \).
                \item[(iii)] \( A  \) contains all its accumulation points;  that is, if \( a \in \C  \) is an accumulation point of \( A  \), then \( a \in A  \). 
           \end{enumerate}
        \item[(b)] Given a set \( A \subseteq \C \), we define 
            \[  {\mathcal{F}}_{A} = \{ F \subseteq \C : F \ \text{is closed and} \ A \subseteq F  \}. \]
            Define \( \overline{A} = \bigcup_{ F \in {\mathcal{F}}_{A} }^{  }  F  \). Show that \( \overline{A} = A \cup A' \) where 
            \[  A' = \{ z \in \C : z \ \text{is an accumulation point} \}.  \]
   \end{enumerate} 
\end{problem}
\begin{proof}
\begin{enumerate}
    \item[(a)] To show that all the statements are equivalent, we will show that \( (i) \implies (ii) \implies (iii) \implies (i) \).  

        \( (i) \implies (ii) \) Suppose \( A  \) is closed. Let \( ({a}_{n}) \) be a sequence in \( A  \) where \( {a}_{n} \to a \in \C  \) (note that \( {a}_{n} \neq a  \). Our goal is to show that \( a \in A  \). Suppose for sake of contradiction that \( a \notin A  \). Then there exists some \( \epsilon > 0  \) such that \( {N}_{\epsilon}(a) \cap E = \emptyset  \); that is, \( {N}_{\epsilon}(a) \subseteq A^{c} \). Hence, we have that \( a \in A^{c} \). But \( ({a}_{n}) \to a \in \C  \) implies that there exists at least one \( {a}_{n} \neq a   \) such that \( {a}_{n} \in A^{c} \). However, the sequence \( ({a}_{n}) \) must be entirely contained in \( A  \) by assumption which is a contradiction. Thus, \( a \in A   \).

        \( (ii) \implies (iii) \) Let \( ({a}_{n}) \) be a sequence in \( A  \) where \( {a}_{n} \neq a \in \C  \) where \( a \in A  \). Our goal is to show that \( A  \) contains all of its limit points. Let \( a  \) be a limit point of \( A  \). Choose \( \epsilon = 1/n \) and choose \( {a}_{n} \neq a  \) to be a sequence of points in \( A  \). Then by assumption, the sequence \( ({a}_{n}) \to a \in \C   \) implies that \( a \in A  \); that is, we have that 
        \[  {N}_{1/n}(a) \cap A  \neq \emptyset.   \]
        Because \( a \in A  \), we can conclude that \( A  \) must contain all of it's accumulation points.
       
        \( (iii) \implies (i) \) Suppose \( A  \) contains all of its accumulation points. Our goal is to show that \( A  \) is closed. It suffices to show that \( A^{c} \) is an open set; that is, we need to find an \( \delta > 0  \) such that \( {N}_{\delta}(x) \subseteq A^{c} \) for all \( x \in A^{c} \). To this end, let \( x \in A^{c} \). Then \( x \notin A  \). This tells us that \( x  \) cannot be a limit point of \( A  \). That is, there exists an \( \delta > 0  \) such that \( {N}_{\delta}(x) \cap A = \emptyset \). This implies that \( {N}_{\delta}(x) \subseteq A^{c} \) for some \( \delta > 0  \), and so \( A^{c} \) must be open. Hence, \( A  \) must be closed.
    \item[(b)] Our goal is to show that \( \overline{A} = A \cup A' \). First, we would like to show two lemmas:   
        \begin{enumerate}
            \item[(*)] \( A \cup A' \) is a closed set.
            \item[(**)] If \( F  \) is a closed set and \( A \subseteq F \), then \( A \cup A' \subseteq F  \) as well. 
        \end{enumerate}
        To show that (*) holds, let \( x  \) be a accumulation point of \( A \cup A' \). Our goal is to show that this accumulation point is contained in \( A \cup A' \). By definition, we see that for all \( \epsilon > 0  \), we have 
        \[  {B}(x,\epsilon) \cap ((A \cup A') \setminus  \{ x \} ) \neq \emptyset. \]
        To this end, pick a point in this intersection, say, \( a  \) such that \( a \in {B(x,\epsilon})  \) and \( a \in (A \cup A') \setminus  \{ x \}  \). That is, we have \( a \in A  \) or \( a \in A' \). If \( a \in A  \), then \( x  \) is a accumulation point of \( A  \), and so \( x \in A \cup A' \). If \( a \in A' \), then \( a  \) is a accumulation point of \( A' \). That is, for all \( \delta> 0  \), we have 
        \[ B(a,\delta) \cap A' \setminus  \{ a \} \neq \emptyset.  \]
        Pick a point in this intersection, say, \( p \neq a  \) such that \( p \in A' \). But this implies that \( x  \) must be a limit point of \( A  \), and so \( x  \in A'  \) and thus \( A \cup A' \) must be a closed set.

        To show that (**) holds, suppose \( F  \) is a closed set and that \( A \subseteq  F  \). Our goal is to show that \( A \cup A' \subseteq F  \). Let \( x \in A \cup A' \). Then either \( x \in A  \) or \( x \in A' \). If \( x \in A  \), then \( x \in F  \) since \( A \subseteq F  \). On the other hand, if \( x \in A' \), then \( x  \) is a limit point of \( A  \). That is, for all \( \delta > 0  \), we have 
        \[  B(x,\delta) \cap (A \setminus  \{ x \} ) \neq \emptyset. \]
         Since \( A \subseteq F  \), we can see that 
         \[  B(x,\delta) \cap (F \setminus  \{ x \} ) \neq \emptyset \]
         which implies that \( x  \) is a limit point of \( F  \). But \( F  \) is closed, so \( x  \) must be contained in \( F  \). Thus, we have \( A \cup A' \subseteq  F  \) in both cases.

    In what follows, we will show that \( \overline{A} = A \cup A' \). To do this, we need to show two inclusions:
    \begin{enumerate}
        \item[(1)] \( \overline{A} \subseteq  A \cup A'  \)
        \item[(2)] \( A \cup A'  \subseteq \overline{A}\).
    \end{enumerate}
    Starting with (1), we see that \( A \cup A'  \subseteq F \) by (*). But this implies that \( A \cup A' \) is the smallest closed set containing \( F  \), we must have that 
    \[  A \cup A' \subseteq  \bigcap_{ F \in {\mathcal{F}}_{A} }^{  }  F = \overline{A} \]
    which satisfies (1).
    
    With (2), we want to show that \( \overline{A} \subseteq A \cup A' \).

\end{enumerate}
\end{proof}






\end{document}
