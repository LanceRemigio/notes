\documentclass[a4paper]{report}
\usepackage{standalone}
\usepackage{import}

\usepackage[utf8]{inputenc}
\usepackage[T1]{fontenc}
\usepackage{textcomp}
\usepackage{hyperref}
% \usepackage{fourier}
% \usepackage[dutch]{babel}
\usepackage{url}
% \usepackage{hyperref}
% \hypersetup{
%     colorlinks,
%     linkcolor={black},
%     citecolor={black},
%     urlcolor={blue!80!black}
% }
\usepackage{graphicx}
\usepackage{float}
\usepackage{booktabs}
\usepackage{enumitem}
% \usepackage{parskip}
\usepackage{emptypage}
\usepackage{subcaption}
\usepackage{multicol}
\usepackage[usenames,dvipsnames]{xcolor}

% \usepackage{cmbright}


\usepackage[margin=1in]{geometry}
\usepackage{amsmath, amsfonts, mathtools, amsthm, amssymb}
\usepackage{mathrsfs}
\usepackage{cancel}
\usepackage{bm}
\newcommand\N{\ensuremath{\mathbb{N}}}
\newcommand\R{\ensuremath{\mathbb{R}}}
\newcommand\Z{\ensuremath{\mathbb{Z}}}
\renewcommand\O{\ensuremath{\emptyset}}
\newcommand\Q{\ensuremath{\mathbb{Q}}}
\newcommand\C{\ensuremath{\mathbb{C}}}
\DeclareMathOperator{\sgn}{sgn}
\usepackage{systeme}
\let\svlim\lim\def\lim{\svlim\limits}
\let\implies\Rightarrow
\let\impliedby\Leftarrow
\let\iff\Leftrightarrow
\let\epsilon\varepsilon
\usepackage{stmaryrd} % for \lightning
\newcommand\contra{\scalebox{1.1}{$\lightning$}}
% \let\phi\varphi
\renewcommand\qedsymbol{$\blacksquare$}




% correct
\definecolor{correct}{HTML}{009900}
\newcommand\correct[2]{\ensuremath{\:}{\color{red}{#1}}\ensuremath{\to }{\color{correct}{#2}}\ensuremath{\:}}
\newcommand\green[1]{{\color{correct}{#1}}}



% horizontal rule
\newcommand\hr{
    \noindent\rule[0.5ex]{\linewidth}{0.5pt}
}


% hide parts
\newcommand\hide[1]{}



% si unitx
\usepackage{siunitx}
\sisetup{locale = FR}
% \renewcommand\vec[1]{\mathbf{#1}}
\newcommand\mat[1]{\mathbf{#1}}


% tikz
\usepackage{tikz}
\usepackage{tikz-cd}
\usetikzlibrary{intersections, angles, quotes, calc, positioning}
\usetikzlibrary{arrows.meta}
\usepackage{pgfplots}
\pgfplotsset{compat=1.13}


\tikzset{
    force/.style={thick, {Circle[length=2pt]}-stealth, shorten <=-1pt}
}

% theorems
\makeatother
\usepackage{thmtools}
\usepackage[framemethod=TikZ]{mdframed}
\mdfsetup{skipabove=1em,skipbelow=0em}


\theoremstyle{definition}

\declaretheoremstyle[
    headfont=\bfseries\sffamily\color{ForestGreen!70!black}, bodyfont=\normalfont,
    mdframed={
        linewidth=2pt,
        rightline=false, topline=false, bottomline=false,
        linecolor=ForestGreen, backgroundcolor=ForestGreen!5,
    }
]{thmgreenbox}

\declaretheoremstyle[
    headfont=\bfseries\sffamily\color{NavyBlue!70!black}, bodyfont=\normalfont,
    mdframed={
        linewidth=2pt,
        rightline=false, topline=false, bottomline=false,
        linecolor=NavyBlue, backgroundcolor=NavyBlue!5,
    }
]{thmbluebox}

\declaretheoremstyle[
    headfont=\bfseries\sffamily\color{NavyBlue!70!black}, bodyfont=\normalfont,
    mdframed={
        linewidth=2pt,
        rightline=false, topline=false, bottomline=false,
        linecolor=NavyBlue
    }
]{thmblueline}

\declaretheoremstyle[
    headfont=\bfseries\sffamily\color{RawSienna!70!black}, bodyfont=\normalfont,
    mdframed={
        linewidth=2pt,
        rightline=false, topline=false, bottomline=false,
        linecolor=RawSienna, backgroundcolor=RawSienna!5,
    }
]{thmredbox}

\declaretheoremstyle[
    headfont=\bfseries\sffamily\color{RawSienna!70!black}, bodyfont=\normalfont,
    numbered=no,
    mdframed={
        linewidth=2pt,
        rightline=false, topline=false, bottomline=false,
        linecolor=RawSienna, backgroundcolor=RawSienna!1,
    },
    qed=\qedsymbol
]{thmproofbox}

\declaretheoremstyle[
    headfont=\bfseries\sffamily\color{NavyBlue!70!black}, bodyfont=\normalfont,
    numbered=no,
    mdframed={
        linewidth=2pt,
        rightline=false, topline=false, bottomline=false,
        linecolor=NavyBlue, backgroundcolor=NavyBlue!1,
    },
]{thmexplanationbox}

\declaretheorem[style=thmgreenbox, numberwithin = section, name=Definition]{definition}
\declaretheorem[style=thmbluebox, name=Example]{eg}
\declaretheorem[style=thmredbox, numberwithin = section, name=Proposition]{prop}
\declaretheorem[style=thmredbox, numberwithin = section, name=Theorem]{theorem}
\declaretheorem[style=thmredbox, numberwithin = section,  name=Lemma]{lemma}
\declaretheorem[style=thmredbox, numberwithin = section,  numbered=no, name=Corollary]{corollary}


\declaretheorem[style=thmproofbox, name=Proof]{replacementproof}
\renewenvironment{proof}[1][\proofname]{\vspace{-10pt}\begin{replacementproof}}{\end{replacementproof}}


\declaretheorem[style=thmexplanationbox, name=Proof]{tmpexplanation}
\newenvironment{explanation}[1][]{\vspace{-10pt}\begin{tmpexplanation}}{\end{tmpexplanation}}


\declaretheorem[style=thmblueline, numbered=no, name=Remark]{remark}
\declaretheorem[style=thmblueline, numbered=no, name=Note]{note}

\newtheorem*{uovt}{UOVT}
\newtheorem*{notation}{Notation}
\newtheorem*{previouslyseen}{As previously seen}
\newtheorem*{problem}{Problem}
\newtheorem*{observe}{Observe}
\newtheorem*{property}{Property}
\newtheorem*{intuition}{Intuition}


\usepackage{etoolbox}
\AtEndEnvironment{vb}{\null\hfill$\diamond$}%
\AtEndEnvironment{intermezzo}{\null\hfill$\diamond$}%
% \AtEndEnvironment{opmerking}{\null\hfill$\diamond$}%

% http://tex.stackexchange.com/questions/22119/how-can-i-change-the-spacing-before-theorems-with-amsthm
\makeatletter
% \def\thm@space@setup{%
%   \thm@preskip=\parskip \thm@postskip=0pt
% }
\newcommand{\oefening}[1]{%
    \def\@oefening{#1}%
    \subsection*{Oefening #1}
}

\newcommand{\suboefening}[1]{%
    \subsubsection*{Oefening \@oefening.#1}
}

\newcommand{\exercise}[1]{%
    \def\@exercise{#1}%
    \subsection*{Exercise #1}
}

\newcommand{\subexercise}[1]{%
    \subsubsection*{Exercise \@exercise.#1}
}


\usepackage{xifthen}

\def\testdateparts#1{\dateparts#1\relax}
\def\dateparts#1 #2 #3 #4 #5\relax{
    \marginpar{\small\textsf{\mbox{#1 #2 #3 #5}}}
}

\def\@lesson{}%
\newcommand{\lesson}[3]{
    \ifthenelse{\isempty{#3}}{%
        \def\@lesson{Lecture #1}%
    }{%
        \def\@lesson{Lecture #1: #3}%
    }%
    \subsection*{\@lesson}
    \testdateparts{#2}
}

% \renewcommand\date[1]{\marginpar{#1}}


% fancy headers
\usepackage{fancyhdr}
\pagestyle{fancy}

\fancyhead[LE,RO]{Lance Remigio}
\fancyhead[RO,LE]{\@lesson}
\fancyhead[RE,LO]{}
\fancyfoot[LE,RO]{\thepage}
\fancyfoot[C]{\leftmark}

\makeatother




% notes
\usepackage{todonotes}
\usepackage{tcolorbox}

\tcbuselibrary{breakable}
\newenvironment{verbetering}{\begin{tcolorbox}[
    arc=0mm,
    colback=white,
    colframe=green!60!black,
    title=Opmerking,
    fonttitle=\sffamily,
    breakable
]}{\end{tcolorbox}}

\newenvironment{noot}[1]{\begin{tcolorbox}[
    arc=0mm,
    colback=white,
    colframe=white!60!black,
    title=#1,
    fonttitle=\sffamily,
    breakable
]}{\end{tcolorbox}}




% figure support
\usepackage{import}
\usepackage{xifthen}
\pdfminorversion=7
\usepackage{pdfpages}
\usepackage{transparent}
\newcommand{\incfig}[1]{%
    \def\svgwidth{\columnwidth}
    \import{./figures/}{#1.pdf_tex}
}

% %http://tex.stackexchange.com/questions/76273/multiple-pdfs-with-page-group-included-in-a-single-page-warning
\pdfsuppresswarningpagegroup=1




\pagestyle{fancy}
\fancyhf{}

\begin{document}
    
\lhead{Math 234A}
\chead{Lance Remigio}
\rhead{\thepage}
\subsection{Topics}

\begin{itemize}
    \item Open sets
    \item Closed sets
    \item Interior Points
    \item Closure of a set.
\end{itemize}

Let's recall the Euclidean metric on \( \C  \):
\[  d(z,w) = | \vec{ z }  - \vec{ w }   |.  \]
For real vector spaces, we can identify \( \C  \) with \( \R^{2} \) using the map \( \C \to \R^{2}  \) with \( \alpha + i \beta \to (\alpha, \beta) \). Thus, we can visualize \( \C  \) using \( \R^{2} \).

Recall that definition of continuity for functions in \( \R  \).

\begin{definition}[continuity in \( \R  \)]
    Let \( f: \R \to \R  \). We say that \( f  \) is \textbf{continuous} for each \( a \in \R  \) if for all \( \epsilon > 0 \), there exists \( \delta > 0  \) such that \( | x - a  |  < \delta \) implies
    \[  | f(x) - f(a) |  < \epsilon. \]
\end{definition}

Note that whenever we say \( | x - a  |  < \delta  \), we can also say that \( x \in (a - \delta, a + \delta) \). Similarly, we have \( | f(x) - f(a) | < \epsilon  \) is equivalent to \( f(x) \in (f(a) - \epsilon, f(a) + \epsilon) \).

If we want to write the former using set notation, we can denote it as 
\[  \{ x \in \R : d(x,a) < \delta \}  \]
which is a more general statement than the one found in the definition above.
\begin{definition}[Open Ball]
  Consider the usual metric \( d(z,w) = | z - w  |   \) in \( \C  \). Let \( \epsilon > 0  \) and \( a \in \C  \). By an \textbf{open ball} centered at \( a  \), and radius \( \epsilon > 0  \), we mean the set
  \[ B(a,\epsilon)  =  \{ z \in \C : d(z,a) < \epsilon \}. \]
\end{definition}

\begin{definition}[Open Set]
   Let \( D \subseteq  \C .  \) We say \( D  \) is \textbf{open} if for each \( a \in D  \), we can find \( \epsilon > 0  \) such that  
   \(  B(a,\epsilon) \subseteq D  \).
\end{definition}

\begin{eg}
    \begin{enumerate}
        \item[(i)] \( \C  \) is open.
        \item[(ii)] Let \( \epsilon > 0  \) and \( a \in \C  \). Then \( B(a,\epsilon)  \) is also open; that is, open balls are open.
        \item[(iii)] Let \( {D}_{1} \) and \( {D}_{2} \) be open sets.
            Show that \( {D}_{1} \cap {D}_{2} \) is open. Let \( x \in {D}_{1} \cap {D}_{2} \) To show that \( {D}_{1} \cap {D}_{2} \) is open, we need to find \( \delta > 0  \) such that \( B(x,\delta) \subseteq {D}_{1} \cap {D}_{2} \). Since \( x \in {D}_{1} \cap {D}_{2} \), then \( x \in {D}_{1} \) and \( x \in {D}_{2} \). Since \( {D}_{1} \) and \( {D}_{2} \) are open sets, we can find \( \epsilon > 0  \) and \( \epsilon' > 0  \) such that 
            \begin{center}
               \( B(x,\epsilon) \subseteq {D}_{1} \) and \( B(x, \epsilon') \subseteq  {D}_{2} \), respectively. 
            \end{center} By definition, we see that \( d(x,p) < \epsilon \) for all \( p \in {D}_{1} \) and \( d(x,q) < \epsilon' \). Then pick \( \delta = \min \{ d(x,p), d(x,q) \}  \). Since \( x  \) is in both \( {D}_{1} \) and \( {D}_{2} \), we have that 
            \[  B(x,\delta) \subseteq  {D}_{1} \cap {D}_{2}. \]
            Hence, \( {D}_{1} \cap {D}_{2} \) is open.
        \item The empty set is open.
        \item Let \( \{ {D}_{\lambda}  \}_{\lambda \in \Lambda} \) be a collection of open ets, where \( \Lambda \) is an indexing set. Define the set
            \[  \bigcup_{ \lambda \in \Lambda }^{  }  {D}_{\lambda} = \{ z \in \C : z \in {D}_{\lambda} \ \text{for some} \ \lambda \in \Lambda \}. \]
            Show that \( \bigcup_{ \lambda \in \Lambda }^{  } {D}_{\lambda}  \) is also open.
    \end{enumerate}
\end{eg}

\begin{definition}[Interior Point]
  Let \( D \subseteq \C \) and \( a \in D  \). We say that \( a  \) is an \textbf{interior point} of \( D  \) if we can find \( \epsilon > 0  \) such that \( B(a, \epsilon) \subseteq  D  \).
\end{definition}

\begin{prop}
    \( D  \) is open if and only if each \( a \in D  \) is an interior point of \( D  \).
\end{prop}

\begin{eg}
    Consider the following intersection 
    \[  \bigcap_{ n=1  }^{ \infty  }  B(0,1/n)   \tag{1} \]
    Let us compute (1). This is just the singleton \( \{ 0 \}  \). Show that \( \{ 0  \}  \) is not open. Note that the intersection of arbitrary many open sets may not be open. 
\end{eg}

\begin{definition}[Closed Set]
    We say \( A \subseteq \C   \) \textbf{closed} if its complement     
    \[  \C \setminus  A = \{ z \in \C : z \notin A  \}   \] is open.
\end{definition}
\begin{eg}
    Define a closed ball with center at \( a \in \C  \) and radius \( \epsilon > 0  \) by
    \[  \overline{B}(a, \epsilon) = \{ z \in \C : | z - a  | \leq \epsilon \}. \]
    Note that this is not the same thing as the closure of the open ball! Show that \( \overline{B}(a,\epsilon) \) is closed.
\end{eg}
\begin{eg}
    \begin{itemize}
\item For any \( a \in \C  \), we say that the singleton \( \{ a  \}   \) is closed.
\item If \( {A}_{1}, {A}_{2}, \dots, {A}_{n} \) are closed sets. Then 
    \[  \bigcup_{ i =1  }^{ n }  {A}_{i} \ \text{is also closed.} \]
    We can just show that the complement of this set is open.
\item Let \( \{ {A}_{\Lambda} \}_{\lambda \in \Lambda}   \) be a collection of closed sets. Define
    \[  \bigcap_{ \lambda \in \Lambda }^{  }  {A}_{\lambda} = \{ z \in \C : z \in {A}_{\Lambda} \ \forall \lambda \in \Lambda \},  \]
    then \( \bigcap_{  \lambda \in \Lambda }^{  } {A}_{\lambda } \) is closed.
\end{itemize}
\end{eg}

\begin{definition}[Boundary Points]
   Let \( D \subseteq  \C  \). We say \( a \in \C  \) is a \textbf{boundary point} of \( D  \) if for all \( \epsilon > 0  \), \( B(a,\epsilon) \cap D \neq \emptyset \) and \( B(a,\epsilon) \cap (\C \setminus  D ) \neq \emptyset  \); that is, \( B(a,\epsilon) \) intersects \( D  \) and \( \C \setminus  D   \) non-trivially.  
\end{definition}

\begin{eg}
   Any point on the circle \( \{ z \in \C : | z - a  | = \epsilon \}  \) is a boundary point of \( B(a,\epsilon) \), where \( \epsilon > 0  \).
\end{eg}

\begin{eg}[Boundary Points of Singletons]
    \begin{itemize}
  \item  Note that the boundary point of the singleton is just the singleton itself.
    \item Let \( D = \C \setminus  \{ a  \}  \). Just like the first item, the boundary points of this set is just \( \{ a  \}  \). 
    \end{itemize}
\end{eg}

\begin{definition}[Accumulation Point]
    
\end{definition}







\end{document}
