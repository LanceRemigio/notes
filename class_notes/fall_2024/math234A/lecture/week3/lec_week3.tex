\documentclass[a4paper]{report}
\usepackage{standalone}
\usepackage{import}

\usepackage[utf8]{inputenc}
\usepackage[T1]{fontenc}
\usepackage{textcomp}
\usepackage{hyperref}
% \usepackage{fourier}
% \usepackage[dutch]{babel}
\usepackage{url}
% \usepackage{hyperref}
% \hypersetup{
%     colorlinks,
%     linkcolor={black},
%     citecolor={black},
%     urlcolor={blue!80!black}
% }
\usepackage{graphicx}
\usepackage{float}
\usepackage{booktabs}
\usepackage{enumitem}
% \usepackage{parskip}
\usepackage{emptypage}
\usepackage{subcaption}
\usepackage{multicol}
\usepackage[usenames,dvipsnames]{xcolor}

% \usepackage{cmbright}


\usepackage[margin=1in]{geometry}
\usepackage{amsmath, amsfonts, mathtools, amsthm, amssymb}
\usepackage{mathrsfs}
\usepackage{cancel}
\usepackage{bm}
\newcommand\N{\ensuremath{\mathbb{N}}}
\newcommand\R{\ensuremath{\mathbb{R}}}
\newcommand\Z{\ensuremath{\mathbb{Z}}}
\renewcommand\O{\ensuremath{\emptyset}}
\newcommand\Q{\ensuremath{\mathbb{Q}}}
\newcommand\C{\ensuremath{\mathbb{C}}}
\DeclareMathOperator{\sgn}{sgn}
\usepackage{systeme}
\let\svlim\lim\def\lim{\svlim\limits}
\let\implies\Rightarrow
\let\impliedby\Leftarrow
\let\iff\Leftrightarrow
\let\epsilon\varepsilon
\usepackage{stmaryrd} % for \lightning
\newcommand\contra{\scalebox{1.1}{$\lightning$}}
% \let\phi\varphi
\renewcommand\qedsymbol{$\blacksquare$}




% correct
\definecolor{correct}{HTML}{009900}
\newcommand\correct[2]{\ensuremath{\:}{\color{red}{#1}}\ensuremath{\to }{\color{correct}{#2}}\ensuremath{\:}}
\newcommand\green[1]{{\color{correct}{#1}}}



% horizontal rule
\newcommand\hr{
    \noindent\rule[0.5ex]{\linewidth}{0.5pt}
}


% hide parts
\newcommand\hide[1]{}



% si unitx
\usepackage{siunitx}
\sisetup{locale = FR}
% \renewcommand\vec[1]{\mathbf{#1}}
\newcommand\mat[1]{\mathbf{#1}}


% tikz
\usepackage{tikz}
\usepackage{tikz-cd}
\usetikzlibrary{intersections, angles, quotes, calc, positioning}
\usetikzlibrary{arrows.meta}
\usepackage{pgfplots}
\pgfplotsset{compat=1.13}


\tikzset{
    force/.style={thick, {Circle[length=2pt]}-stealth, shorten <=-1pt}
}

% theorems
\makeatother
\usepackage{thmtools}
\usepackage[framemethod=TikZ]{mdframed}
\mdfsetup{skipabove=1em,skipbelow=0em}


\theoremstyle{definition}

\declaretheoremstyle[
    headfont=\bfseries\sffamily\color{ForestGreen!70!black}, bodyfont=\normalfont,
    mdframed={
        linewidth=2pt,
        rightline=false, topline=false, bottomline=false,
        linecolor=ForestGreen, backgroundcolor=ForestGreen!5,
    }
]{thmgreenbox}

\declaretheoremstyle[
    headfont=\bfseries\sffamily\color{NavyBlue!70!black}, bodyfont=\normalfont,
    mdframed={
        linewidth=2pt,
        rightline=false, topline=false, bottomline=false,
        linecolor=NavyBlue, backgroundcolor=NavyBlue!5,
    }
]{thmbluebox}

\declaretheoremstyle[
    headfont=\bfseries\sffamily\color{NavyBlue!70!black}, bodyfont=\normalfont,
    mdframed={
        linewidth=2pt,
        rightline=false, topline=false, bottomline=false,
        linecolor=NavyBlue
    }
]{thmblueline}

\declaretheoremstyle[
    headfont=\bfseries\sffamily\color{RawSienna!70!black}, bodyfont=\normalfont,
    mdframed={
        linewidth=2pt,
        rightline=false, topline=false, bottomline=false,
        linecolor=RawSienna, backgroundcolor=RawSienna!5,
    }
]{thmredbox}

\declaretheoremstyle[
    headfont=\bfseries\sffamily\color{RawSienna!70!black}, bodyfont=\normalfont,
    numbered=no,
    mdframed={
        linewidth=2pt,
        rightline=false, topline=false, bottomline=false,
        linecolor=RawSienna, backgroundcolor=RawSienna!1,
    },
    qed=\qedsymbol
]{thmproofbox}

\declaretheoremstyle[
    headfont=\bfseries\sffamily\color{NavyBlue!70!black}, bodyfont=\normalfont,
    numbered=no,
    mdframed={
        linewidth=2pt,
        rightline=false, topline=false, bottomline=false,
        linecolor=NavyBlue, backgroundcolor=NavyBlue!1,
    },
]{thmexplanationbox}

\declaretheorem[style=thmgreenbox, numberwithin = section, name=Definition]{definition}
\declaretheorem[style=thmbluebox, name=Example]{eg}
\declaretheorem[style=thmredbox, numberwithin = section, name=Proposition]{prop}
\declaretheorem[style=thmredbox, numberwithin = section, name=Theorem]{theorem}
\declaretheorem[style=thmredbox, numberwithin = section,  name=Lemma]{lemma}
\declaretheorem[style=thmredbox, numberwithin = section,  numbered=no, name=Corollary]{corollary}


\declaretheorem[style=thmproofbox, name=Proof]{replacementproof}
\renewenvironment{proof}[1][\proofname]{\vspace{-10pt}\begin{replacementproof}}{\end{replacementproof}}


\declaretheorem[style=thmexplanationbox, name=Proof]{tmpexplanation}
\newenvironment{explanation}[1][]{\vspace{-10pt}\begin{tmpexplanation}}{\end{tmpexplanation}}


\declaretheorem[style=thmblueline, numbered=no, name=Remark]{remark}
\declaretheorem[style=thmblueline, numbered=no, name=Note]{note}

\newtheorem*{uovt}{UOVT}
\newtheorem*{notation}{Notation}
\newtheorem*{previouslyseen}{As previously seen}
\newtheorem*{problem}{Problem}
\newtheorem*{observe}{Observe}
\newtheorem*{property}{Property}
\newtheorem*{intuition}{Intuition}


\usepackage{etoolbox}
\AtEndEnvironment{vb}{\null\hfill$\diamond$}%
\AtEndEnvironment{intermezzo}{\null\hfill$\diamond$}%
% \AtEndEnvironment{opmerking}{\null\hfill$\diamond$}%

% http://tex.stackexchange.com/questions/22119/how-can-i-change-the-spacing-before-theorems-with-amsthm
\makeatletter
% \def\thm@space@setup{%
%   \thm@preskip=\parskip \thm@postskip=0pt
% }
\newcommand{\oefening}[1]{%
    \def\@oefening{#1}%
    \subsection*{Oefening #1}
}

\newcommand{\suboefening}[1]{%
    \subsubsection*{Oefening \@oefening.#1}
}

\newcommand{\exercise}[1]{%
    \def\@exercise{#1}%
    \subsection*{Exercise #1}
}

\newcommand{\subexercise}[1]{%
    \subsubsection*{Exercise \@exercise.#1}
}


\usepackage{xifthen}

\def\testdateparts#1{\dateparts#1\relax}
\def\dateparts#1 #2 #3 #4 #5\relax{
    \marginpar{\small\textsf{\mbox{#1 #2 #3 #5}}}
}

\def\@lesson{}%
\newcommand{\lesson}[3]{
    \ifthenelse{\isempty{#3}}{%
        \def\@lesson{Lecture #1}%
    }{%
        \def\@lesson{Lecture #1: #3}%
    }%
    \subsection*{\@lesson}
    \testdateparts{#2}
}

% \renewcommand\date[1]{\marginpar{#1}}


% fancy headers
\usepackage{fancyhdr}
\pagestyle{fancy}

\fancyhead[LE,RO]{Lance Remigio}
\fancyhead[RO,LE]{\@lesson}
\fancyhead[RE,LO]{}
\fancyfoot[LE,RO]{\thepage}
\fancyfoot[C]{\leftmark}

\makeatother




% notes
\usepackage{todonotes}
\usepackage{tcolorbox}

\tcbuselibrary{breakable}
\newenvironment{verbetering}{\begin{tcolorbox}[
    arc=0mm,
    colback=white,
    colframe=green!60!black,
    title=Opmerking,
    fonttitle=\sffamily,
    breakable
]}{\end{tcolorbox}}

\newenvironment{noot}[1]{\begin{tcolorbox}[
    arc=0mm,
    colback=white,
    colframe=white!60!black,
    title=#1,
    fonttitle=\sffamily,
    breakable
]}{\end{tcolorbox}}




% figure support
\usepackage{import}
\usepackage{xifthen}
\pdfminorversion=7
\usepackage{pdfpages}
\usepackage{transparent}
\newcommand{\incfig}[1]{%
    \def\svgwidth{\columnwidth}
    \import{./figures/}{#1.pdf_tex}
}

% %http://tex.stackexchange.com/questions/76273/multiple-pdfs-with-page-group-included-in-a-single-page-warning
\pdfsuppresswarningpagegroup=1





\begin{document}

\section{Lecture 4}

\subsection{Topics}

\begin{enumerate}
    \item[(i)] {\hyperref[Convergence of Series]{Convergence of Series}} 
    \item[(ii)] {\hyperref[Exponential, Sine, and Cosine functions]{Exponential, Sine, and Cosine functions}} 
\end{enumerate}


\subsection{Convergence of Series}\label{Convergence of Series}

Infinite series of complex numbers. Let \( \{ {z}_{n} \}_{n=1}^{\infty } \) be a sequence of complex numbers. 

Goal: To give meaning to \( \sum_{ n=1  }^{ \infty  } {z}_{n} \).

Define \[ {s}_{n} = {z}_{1} + {z}_{2} + \cdots + {z}_{n} = \sum_{ k = 1  }^{ n } {z}_{k} \] 
\begin{definition}[Convergence of Series]
    If the sequence \( \{ {s}_{n} \}_{n=1}^{\infty } \) converges, we say that the series  
    \[  \sum_{ n=1  }^{ \infty  } {z}_{n} \]
    converges and we write
    \[  \sum_{ n = 1  }^{  \infty   } {z}_{n} = s. \]
\end{definition}

\begin{eg}
    Consider the geometric series
    \[  \sum_{ n=1  }^{ \infty  } \frac{ 1  }{ 2^{n} \cdot i  }. \]
    Let 
    \begin{align*}
    {s}_{n} &= \frac{ 1 }{ 2 i  }  + \frac{ 1 }{ 2^{2} i   } + \cdots + \frac{ 1  }{ 2^{n} i  }   \\
            &=  \frac{ 1 }{ i }  \Big[ \frac{ 1 }{ 2 }  + \frac{ 1 }{ 2^{2} }  + \cdots + \frac{ 1 }{ 2^{n} } \Big].
\end{align*}
\begin{align*}
    i {s}_{n} &= \frac{ 1 }{ 2 }  + \frac{ 1 }{ 2^{2}  } + \cdots + \frac{ 1 }{ 2^{n} } \tag{1}  \\
    \frac{ 1 }{ 2 }  i {s}_{n} &= \frac{ 1 }{ 2^{2}  }  + \cdots + \frac{ 1 }{ 2^{n}  } + \frac{ 1 }{ 2^{n+1} } \tag{2}
\end{align*}
Consider
\[  \frac{ 1 }{ 2 }  i \cdot {s}_{n} = \frac{ 1 }{ 2^{2}  }  + \cdots + \frac{ 1 }{ 2^{n} }  + \frac{ 1 }{  2^{n+1} }.  \]
Then subtracting (1) and (2), we see that
\[ i {s}_{n} - \frac{ 1 }{ 2 }  i {s}_{n} =   \frac{ 1 }{ 2 }  i {s}_{n} = \frac{ 1 }{ 2 }  - \frac{ 1 }{ 2^{n+1} } \implies {s}_{n} = \frac{ 1 }{ i }  \Big[ 1 - \frac{ 1 }{ 2^{n} } \Big].   \]
Thus, taking the limit as \( n \to \infty   \) gives us 
\[  \lim_{ n \to \infty  }  {s}_{n} = \frac{ 1 }{ i }. \]
\end{eg}

\begin{eg}
    Assume that \( | z  |  < 1  \), \( z \in \C  \). Compute
    \[  \sum_{ n=1  }^{ \infty   } z^{n}. \]
    Since \( | z  |  < 1  \), we know that 
    \[  \sum_{ n-1  }^{ \infty  } z^{n}   \]
    is a geometric series and that it converges. Using the geometric series sum formula, we see that
    \[  \sum_{ n=1  }^{ \infty  } z^{n} = \lim_{ n \to \infty  }  \sum_{  k =1  }^{ n  } z^{k } = \frac{ z  }{  1 - z }.  \]
    Consider this alternative way. Suppose we have
    \[ {s}_{n} = z + z^{2} + z^{3} + \cdots + z^{n}.    \]
    and
    \[  z {s}_{n} =  z^{2} + z^{3} + \cdots + z^{n} + z^{n+1}.  \]
    Then subtracting these two equations, we see that
    \[  (1 - z ) {s}_{n} = z - z^{n+1}. \]
    Then we have
    \[  {s}_{n} = \frac{ z  }{ 1 - z  } - \frac{ z^{n+1} }{ (1 - z) }.  \]
    Then taking the limit as \( n \to \infty  \), we must have
    \[  \sum_{ n=1  }^{ \infty   } z^{n} = \frac{ z  }{  1 - z  }. \]
\end{eg}

Now, we will discuss absolute convergence of complex series.

\begin{definition}[Absolute Convergence of Infinite Series]
    Let \( ({z}_{n}) \) be a sequence of complex numbers. We say that the series \( \sum_{  n = 1  }^{  \infty  } {z}_{n} \) converges absolutely if the series of real numbers 
    \[  \sum_{ n=1  }^{ \infty   } | {z}_{n} |  \] converges.
\end{definition}


\begin{remark}
    If \( \sum_{ n=1  }^{ \infty  } {z}_{n} \) converges absolutely, then the series
    \[  \sum_{ n=1  }^{ \infty  } {z}_{n} \] converges.
    But note that the converse may not be true.
\end{remark}

\begin{eg}
Let \( z \in \C  \). The series  
\[  \sum_{ n=0  }^{ \infty  } \frac{ z^{n} }{ n!  }   \]
converges absolutely. Note that this is just the series expansion of \( e^{z} \) found in calculus courses. Consider   
\[  \sum_{ n=0  }^{ \infty  } \frac{ | z |^{n} }{ n!  }  = e^{| z | }.  \]
Note that this infinite series is a real series. If use the ratio test, we can consider 
\[  {a}_{n+1} = \frac{ | z |^{n+1} }{ (n+1)! }  \]
and 
\[  {a}_{n} = \frac{ | z |^{n} }{ n!  }  \]
and then show that 
\[  \lim_{ n \to \infty  }  \Big| \frac{ {a}_{n+1} }{ {a}_{n} }  \Big|  < 1.  \]
\end{eg}

Next, we shall define \( e^{z} \) in \( \C  \). 

\subsection{Exponential, Sine, and Cosine}\label{Exponential, Sine, and Cosine functions}

\begin{definition}[Complex Version of Exponential Function]
    For \( z \in \C  \), we define 
    \[ \exp(z) = \sum_{ n=0  }^{  \infty  } \frac{ z^{n} }{ n!  }   \]
\end{definition}
Note that the series 
            \[  \sum_{ n=0 }^{ \infty  } (-1)^{n} \frac{ z^{2n+1} }{ (2n+1)!  }  \]
            and 
            \[  \sum_{ n=0  }^{ \infty  } (-1)^{n} \frac{ z^{2n} }{ (2n)! }  \]
            converges absolutely for all \( z \in \C  \).
\begin{definition}[Sine and Cosine Series]
         We define  \( \sin z  \) as
            \[ \sin z = \sum_{ n=0 }^{ \infty  } (-1)^{2n+1} \frac{ z^{2n+1} }{ (2n+1)! } \]
            and \( \cos z  \) as 
            \[  \cos z  =  \sum_{ n = 0  }^{ \infty  } (-1)^{n} \frac{ z^{2n} }{ (2n)! }.  \]
\end{definition}

\begin{definition}[ ]
    For \( z \in \C  \), we define 
    \[  \exp(z) = \sum_{ n=0 }^{ \infty  } \frac{ z^{n} }{ n! }. \]
\end{definition}

Now, we want to find out whether \( \exp(z + w) = \exp(z) \cdot \exp(w) \). Note that this is true, but we want to prove this rigorously! Recall that the Euler Formula is
\[  e^{iz} = \cos z + i \sin z \  \text{\textbf{Show this!}}. \]




\end{document}
