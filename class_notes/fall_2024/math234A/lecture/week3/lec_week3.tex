\documentclass[a4paper]{report}
\usepackage{standalone}
\usepackage{import}

\usepackage[utf8]{inputenc}
\usepackage[T1]{fontenc}
% \usepackage{fourier}
\usepackage{textcomp}
\usepackage{hyperref}
\usepackage[english]{babel}
\usepackage{url}
% \usepackage{hyperref}
% \hypersetup{
%     colorlinks,
%     linkcolor={black},
%     citecolor={black},
%     urlcolor={blue!80!black}
% }
\usepackage{graphicx} \usepackage{float}
\usepackage{booktabs}
\usepackage{enumitem}
% \usepackage{parskip}
% \usepackage{parskip}
\usepackage{emptypage}
\usepackage{subcaption}
\usepackage{multicol}
\usepackage[usenames,dvipsnames]{xcolor}
\usepackage{ocgx}
% \usepackage{cmbright}


\usepackage[margin=1in]{geometry}
\usepackage{amsmath, amsfonts, mathtools, amsthm, amssymb}
\usepackage{thmtools}
\usepackage{mathrsfs}
\usepackage{cancel}
\usepackage{bm}
\newcommand\N{\ensuremath{\mathbb{N}}}
\newcommand\R{\ensuremath{\mathbb{R}}}
\newcommand\Z{\ensuremath{\mathbb{Z}}}
\renewcommand\O{\ensuremath{\emptyset}}
\newcommand\Q{\ensuremath{\mathbb{Q}}}
\newcommand\C{\ensuremath{\mathbb{C}}}
\newcommand\F{\ensuremath{\mathbb{F}}}
\DeclareMathOperator{\sgn}{sgn}
\DeclareMathOperator{\diam}{diam}
\DeclareMathOperator{\LO}{LO}
\DeclareMathOperator{\UP}{UP}
\DeclareMathOperator{\card}{card}
\DeclareMathOperator{\Arg}{Arg}
\DeclareMathOperator{\Dom}{Dom}
\DeclareMathOperator{\Log}{Log}
\DeclareMathOperator{\dist}{dist}
% \DeclareMathOperator{\span}{span}
\usepackage{systeme}
\let\svlim\lim\def\lim{\svlim\limits}
\renewcommand\implies\Longrightarrow
\let\impliedby\Longleftarrow
\let\iff\Longleftrightarrow
\let\epsilon\varepsilon
\usepackage{stmaryrd} % for \lightning
\newcommand\contra{\scalebox{1.1}{$\lightning$}}
% \let\phi\varphi
\renewcommand\qedsymbol{$\blacksquare$}

% correct
\definecolor{correct}{HTML}{009900}
\newcommand\correct[2]{\ensuremath{\:}{\color{red}{#1}}\ensuremath{\to }{\color{correct}{#2}}\ensuremath{\:}}
\newcommand\green[1]{{\color{correct}{#1}}}

% horizontal rule
\newcommand\hr{
    \noindent\rule[0.5ex]{\linewidth}{0.5pt}
}

% hide parts
\newcommand\hide[1]{}

% si unitx
\usepackage{siunitx}
\sisetup{locale = FR}
% \renewcommand\vec[1]{\mathbf{#1}}
\newcommand\mat[1]{\mathbf{#1}}

% tikz
\usepackage{tikz}
\usepackage{tikz-cd}
\usetikzlibrary{intersections, angles, quotes, calc, positioning}
\usetikzlibrary{arrows.meta}
\usepackage{pgfplots}
\pgfplotsset{compat=1.13}

\tikzset{
    force/.style={thick, {Circle[length=2pt]}-stealth, shorten <=-1pt}
}

% theorems
\makeatother
\usepackage{thmtools}
\usepackage[framemethod=TikZ]{mdframed}
\mdfsetup{skipabove=1em,skipbelow=1em}

\theoremstyle{definition}

\declaretheoremstyle[
    headfont=\bfseries\sffamily\color{ForestGreen!70!black}, bodyfont=\normalfont,
    mdframed={
        linewidth=1pt,
        rightline=false, topline=false, bottomline=false,
        linecolor=ForestGreen, backgroundcolor=ForestGreen!5,
    }
]{thmgreenbox}

\declaretheoremstyle[
    headfont=\bfseries\sffamily\color{NavyBlue!70!black}, bodyfont=\normalfont,
    mdframed={
        linewidth=1pt,
        rightline=false, topline=false, bottomline=false,
        linecolor=NavyBlue, backgroundcolor=NavyBlue!5,
    }
]{thmbluebox}

\declaretheoremstyle[
    headfont=\bfseries\sffamily\color{NavyBlue!70!black}, bodyfont=\normalfont,
    mdframed={
        linewidth=1pt,
        rightline=false, topline=false, bottomline=false,
        linecolor=NavyBlue
    }
]{thmblueline}

\declaretheoremstyle[
    headfont=\bfseries\sffamily, bodyfont=\normalfont,
    numbered = no,
    mdframed={
        rightline=true, topline=true, bottomline=true,
    }
]{thmbox}

\declaretheoremstyle[
    headfont=\bfseries\sffamily, bodyfont=\normalfont,
    numbered=no,
    % mdframed={
    %     rightline=true, topline=false, bottomline=true,
    % },
    qed=\qedsymbol
]{thmproofbox}

\declaretheoremstyle[
    headfont=\bfseries\sffamily\color{NavyBlue!70!black}, bodyfont=\normalfont,
    numbered=no,
    mdframed={
        rightline=false, topline=false, bottomline=false,
        linecolor=NavyBlue, backgroundcolor=NavyBlue!1,
    },
]{thmexplanationbox}

\declaretheorem[
    style=thmbox, 
    % numberwithin = section,
    numbered = no,
    name=Definition
    ]{definition}

\declaretheorem[
    style=thmbox, 
    name=Example,
    ]{eg}

\declaretheorem[
    style=thmbox, 
    % numberwithin = section,
    name=Proposition]{prop}

\declaretheorem[
    style = thmbox,
    numbered=yes,
    name =Problem
    ]{problem}

\declaretheorem[style=thmbox, name=Theorem]{theorem}
\declaretheorem[style=thmbox, name=Lemma]{lemma}
\declaretheorem[style=thmbox, name=Corollary]{corollary}

\declaretheorem[style=thmproofbox, name=Proof]{replacementproof}

\declaretheorem[style=thmproofbox, 
                name = Solution
                ]{replacementsolution}

\renewenvironment{proof}[1][\proofname]{\vspace{-1pt}\begin{replacementproof}}{\end{replacementproof}}

\newenvironment{solution}
    {
        \vspace{-1pt}\begin{replacementsolution}
    }
    { 
            \end{replacementsolution}
    }

\declaretheorem[style=thmexplanationbox, name=Proof]{tmpexplanation}
\newenvironment{explanation}[1][]{\vspace{-10pt}\begin{tmpexplanation}}{\end{tmpexplanation}}

\declaretheorem[style=thmbox, numbered=no, name=Remark]{remark}
\declaretheorem[style=thmbox, numbered=no, name=Note]{note}

\newtheorem*{uovt}{UOVT}
\newtheorem*{notation}{Notation}
\newtheorem*{previouslyseen}{As previously seen}
% \newtheorem*{problem}{Problem}
\newtheorem*{observe}{Observe}
\newtheorem*{property}{Property}
\newtheorem*{intuition}{Intuition}

\usepackage{etoolbox}
\AtEndEnvironment{vb}{\null\hfill$\diamond$}%
\AtEndEnvironment{intermezzo}{\null\hfill$\diamond$}%
% \AtEndEnvironment{opmerking}{\null\hfill$\diamond$}%

% http://tex.stackexchange.com/questions/22119/how-can-i-change-the-spacing-before-theorems-with-amsthm
\makeatletter
% \def\thm@space@setup{%
%   \thm@preskip=\parskip \thm@postskip=0pt
% }
\newcommand{\oefening}[1]{%
    \def\@oefening{#1}%
    \subsection*{Oefening #1}
}

\newcommand{\suboefening}[1]{%
    \subsubsection*{Oefening \@oefening.#1}
}

\newcommand{\exercise}[1]{%
    \def\@exercise{#1}%
    \subsection*{Exercise #1}
}

\newcommand{\subexercise}[1]{%
    \subsubsection*{Exercise \@exercise.#1}
}


\usepackage{xifthen}

\def\testdateparts#1{\dateparts#1\relax}
\def\dateparts#1 #2 #3 #4 #5\relax{
    \marginpar{\small\textsf{\mbox{#1 #2 #3 #5}}}
}

\def\@lesson{}%
\newcommand{\lesson}[3]{
    \ifthenelse{\isempty{#3}}{%
        \def\@lesson{Lecture #1}%
    }{%
        \def\@lesson{Lecture #1: #3}%
    }%
    \subsection*{\@lesson}
    \testdateparts{#2}
}

% \renewcommand\date[1]{\marginpar{#1}}


% fancy headers
\usepackage{fancyhdr}
\pagestyle{fancy}

\makeatother

% notes
\usepackage{todonotes}
\usepackage{tcolorbox}

\tcbuselibrary{breakable}
\newenvironment{verbetering}{\begin{tcolorbox}[
    arc=0mm,
    colback=white,
    colframe=green!60!black,
    title=Opmerking,
    fonttitle=\sffamily,
    breakable
]}{\end{tcolorbox}}

\newenvironment{noot}[1]{\begin{tcolorbox}[
    arc=0mm,
    colback=white,
    colframe=white!60!black,
    title=#1,
    fonttitle=\sffamily,
    breakable
]}{\end{tcolorbox}}

% figure support
\usepackage{import}
\usepackage{xifthen}
\pdfminorversion=7
\usepackage{pdfpages}
\usepackage{transparent}
\newcommand{\incfig}[1]{%
    \def\svgwidth{\columnwidth}
    \import{./figures/}{#1.pdf_tex}
}

% %http://tex.stackexchange.com/questions/76273/multiple-pdfs-with-page-group-included-in-a-single-page-warning
\pdfsuppresswarningpagegroup=1




\begin{document}

\section{Lecture 4}

\subsection{Topics}

\begin{enumerate}
    \item[(i)] Continue discussion of convergence of sequence/series.
    \item[(ii)] Discuss exponential, sine, and cosine function.
\end{enumerate}


\subsection{Convergence of Series}

Infinite series of complex numbers. Let \( \{ {z}_{n} \}_{n=1}^{\infty } \) be a sequence of complex numbers. 

Goal: To give meaning to \( \sum_{ n=1  }^{ \infty  } {z}_{n} \).

Define \[ {s}_{n} = {z}_{1} + {z}_{2} + \cdots + {z}_{n} = \sum_{ k = 1  }^{ n } {z}_{k} \] 
\begin{definition}[Convergence of Series]
    If the sequence \( \{ {s}_{n} \}_{n=1}^{\infty } \) converges, we say that the series  
    \[  \sum_{ n=1  }^{ \infty  } {z}_{n} \]
    converges and we write
    \[  \sum_{ n = 1  }^{  \infty   } {z}_{n} = s. \]
\end{definition}

\begin{eg}
    Consider the geometric series
    \[  \sum_{ n=1  }^{ \infty  } \frac{ 1  }{ 2^{n} \cdot i  }. \]
    Let 
    \begin{align*}
    {s}_{n} &= \frac{ 1 }{ 2 i  }  + \frac{ 1 }{ 2^{2} i   } + \cdots + \frac{ 1  }{ 2^{n} i  }   \\
            &=  \frac{ 1 }{ i }  \Big[ \frac{ 1 }{ 2 }  + \frac{ 1 }{ 2^{2} }  + \cdots + \frac{ 1 }{ 2^{n} } \Big].
\end{align*}
\begin{align*}
    i {s}_{n} &= \frac{ 1 }{ 2 }  + \frac{ 1 }{ 2^{2}  } + \cdots + \frac{ 1 }{ 2^{n} } \tag{1}  \\
    \frac{ 1 }{ 2 }  i {s}_{n} &= \frac{ 1 }{ 2^{2}  }  + \cdots + \frac{ 1 }{ 2^{n}  } + \frac{ 1 }{ 2^{n+1} } \tag{2}
\end{align*}
Consider
\[  \frac{ 1 }{ 2 }  i \cdot {s}_{n} = \frac{ 1 }{ 2^{2}  }  + \cdots + \frac{ 1 }{ 2^{n} }  + \frac{ 1 }{  2^{n+1} }.  \]
Then subtracting (1) and (2), we see that
\[ i {s}_{n} - \frac{ 1 }{ 2 }  i {s}_{n} =   \frac{ 1 }{ 2 }  i {s}_{n} = \frac{ 1 }{ 2 }  - \frac{ 1 }{ 2^{n+1} } \implies {s}_{n} = \frac{ 1 }{ i }  \Big[ 1 - \frac{ 1 }{ 2^{n} } \Big].   \]
Thus, taking the limit as \( n \to \infty   \) gives us 
\[  \lim_{ n \to \infty  }  {s}_{n} = \frac{ 1 }{ i }. \]
\end{eg}

\begin{eg}
    Assume that \( | z  |  < 1  \), \( z \in \C  \). Compute
    \[  \sum_{ n=1  }^{ \infty   } z^{n}. \]
    Since \( | z  |  < 1  \), we know that 
    \[  \sum_{ n-1  }^{ \infty  } z^{n}   \]
    is a geometric series and that it converges. Using the geometric series sum formula, we see that
    \[  \sum_{ n=1  }^{ \infty  } z^{n} = \lim_{ n \to \infty  }  \sum_{  k =1  }^{ n  } z^{k } = \frac{ z  }{  1 - z }.  \]
    Consider this alternative way. Suppose we have
    \[ {s}_{n} = z + z^{2} + z^{3} + \cdots + z^{n}.    \]
    and
    \[  z {s}_{n} =  z^{2} + z^{3} + \cdots + z^{n} + z^{n+1}.  \]
    Then subtracting these two equations, we see that
    \[  (1 - z ) {s}_{n} = z - z^{n+1}. \]
    Then we have
    \[  {s}_{n} = \frac{ z  }{ 1 - z  } - \frac{ z^{n+1} }{ (1 - z) }.  \]
    Then taking the limit as \( n \to \infty  \), we must have
    \[  \sum_{ n=1  }^{ \infty   } z^{n} = \frac{ z  }{  1 - z  }. \]
\end{eg}

Now, we will discuss absolute convergence of complex series.

\begin{definition}[Absolute Convergence of Infinite Series]
    Let \( ({z}_{n}) \) be a sequence of complex numbers. We say that the series \( \sum_{  n = 1  }^{  \infty  } {z}_{n} \) converges absolutely if the series of real numbers 
    \[  \sum_{ n=1  }^{ \infty   } | {z}_{n} |  \] converges.
\end{definition}


\begin{remark}
    If \( \sum_{ n=1  }^{ \infty  } {z}_{n} \) converges absolutely, then the series
    \[  \sum_{ n=1  }^{ \infty  } {z}_{n} \] converges.
    But note that the converse may not be true.
\end{remark}

\begin{eg}
Let \( z \in \C  \). The series  
\[  \sum_{ n=0  }^{ \infty  } \frac{ z^{n} }{ n!  }   \]
converges absolutely. Note that this is just the series expansion of \( e^{z} \) found in calculus courses. Consider   
\[  \sum_{ n=0  }^{ \infty  } \frac{ | z |^{n} }{ n!  }  = e^{| z | }.  \]
Note that this infinite series is a real series. If use the ratio test, we can consider 
\[  {a}_{n+1} = \frac{ | z |^{n+1} }{ (n+1)! }  \]
and 
\[  {a}_{n} = \frac{ | z |^{n} }{ n!  }  \]
and then show that 
\[  \lim_{ n \to \infty  }  \Big| \frac{ {a}_{n+1} }{ {a}_{n} }  \Big|  < 1.  \]
\end{eg}

Next, we shall define \( e^{z} \) in \( \C  \). 

\subsection{Exponential, Sine, and Cosine}

\begin{definition}[Complex Version of Exponential Function]
    For \( z \in \C  \), we define 
    \[ \exp(z) = \sum_{ n=0  }^{  \infty  } \frac{ z^{n} }{ n!  }   \]
\end{definition}
Note that the series 
            \[  \sum_{ n=0 }^{ \infty  } (-1)^{n} \frac{ z^{2n+1} }{ (2n+1)!  }  \]
            and 
            \[  \sum_{ n=0  }^{ \infty  } (-1)^{n} \frac{ z^{2n} }{ (2n)! }  \]
            converges absolutely for all \( z \in \C  \).
\begin{definition}[Sine and Cosine Series]
         We define  \( \sin z  \) as
            \[ \sin z = \sum_{ n=0 }^{ \infty  } (-1)^{2n+1} \frac{ z^{2n+1} }{ (2n+1)! } \]
            and \( \cos z  \) as 
            \[  \cos z  =  \sum_{ n = 0  }^{ \infty  } (-1)^{n} \frac{ z^{2n} }{ (2n)! }.  \]
\end{definition}

\begin{definition}[ ]
    For \( z \in \C  \), we define 
    \[  \exp(z) = \sum_{ n=0 }^{ \infty  } \frac{ z^{n} }{ n! }. \]
\end{definition}

Now, we want to find out whether \( \exp(z + w) = \exp(z) \cdot \exp(w) \). Note that this is true, but we want to prove this rigorously! Recall that the Euler Formula is
\[  e^{iz} = \cos z + i \sin z \  \text{\textbf{Show this!}}. \]

\section{Lecture 5}

\subsection{Topics}
\begin{itemize}
    \item Continue discussion of some important functions.
    \item Go over some topological notion briefly.
\end{itemize}

\subsection{Sine and Cosine Functions}

Recall that we defined 
\begin{align*}
    \sin(z) &= \sum_{ n=0  }^{ \infty  } (-1)^{n} \frac{ z^{2n+1} }{ (2n+1)! } , \ \text{for} \ z \in \C  \\
    \cos(z) &= \sum_{ n = 0  }^{ \infty  } (-1)^{n} \frac{ z^{2n}  }{ (2n)! } , \ \text{for} \ z \in \C. 
\end{align*}
We also know,
\[  \exp(z) = \sum_{ n=0 }^{ \infty  } \frac{ z^{n} }{ n! }.  \]

\begin{lemma}
    For any \( z \in \C  \), observe that
    \begin{itemize}
        \item \( \exp(iz) = \cos z + i \sin z  \)
        \item \( \cos(z) = (\exp(iz) + \exp(-iz)) / 2 \) 
        \item \( \sin(z) = (\exp(iz) - \exp(-iz)) / 2 \).
    \end{itemize}
\end{lemma}

\begin{proof}
Let \( z \in \C  \). We know the series
\[  \sum_{ n=0 }^{ \infty  } \frac{ (iz)^{n} }{ n! }  \] converges absolutely.
Thus, we see that
\begin{align*}
    \sum_{ n=0 }^{ \infty  } \frac{ (iz)^{n} }{ n! } &= \sum_{ n=0 }^{ \infty  } \frac{ (i)^{2n+1} z^{2n+1} }{ (2n+1)! }  + \sum_{ n=0 }^{ \infty  } \frac{ (i)^{2n} z^{2n} }{ (2n)! }  \\
                                                     &= \sum_{ n=0 }^{ \infty  } \frac{ (-1)^{n} z^{2n} }{ (2n)! }  +  i  \sum_{ n=0 }^{ \infty  } \frac{ (-1)^{n} z^{2n+1} }{ (2n+1)! } \\
                                                     &= \cos z + i \sin z.
\end{align*}
Take the real and imaginary part the formulas stated in the lemma will follow.
\end{proof}

\begin{corollary}
    For any \( z \in \C  \), we see that 
    \begin{itemize}
        \item \( \exp(x + iy) = e^{x} (\cos y + i \sin y) \).  
        \item \( \Re(\exp(z)) = e^{x} \cos y \).
        \item \( \Im(\exp(z)) = e^{x} \sin y, \ | \exp(z) |  = e^{x} \).
        \item \( \sin(z +w) = \sin z \cos w + \cos z \sin w \).
        \item  \( \cos(z + w) = \cos z \cos w - \sin z \sin w  \).
    \end{itemize}
\end{corollary}

\begin{prop}
   Show that \( \exp(z) = \exp(w) \) if and only if \( z - w \in 2 \pi i \Z  \). 
   \textbf{Hint:} Let \( z = x + iy \) and \( w = u + i v \). Then 
   \begin{align*}
       \exp(z) &= e^{x} (\cos y + i \sin y) \\
       \exp(w) &= e^{v} (\cos v + i \sin v)
   \end{align*}
   and show that \( u  = x  \) and \(  u - v \in 2 \pi \Z  \).
\end{prop} 

\begin{remark}
    This exercise shows that \( \exp: \C \to \C^{\cdot} \) is not injective.
    \begin{itemize}
        \item Note that \( \exp: \C \to \C^{\cdot} \) is a group homomorphism. From this exercise, we know that \( \ker(\exp) = 2\pi i \Z  \). This means that \( \exp  \) is periodic with period \( 2 \pi i  \). 
    \end{itemize}
\end{remark}

Our next goal is find an "inverse" of \( \exp \). We just learned that it is not possible unless we change the "domain" of \( \exp \).

Let \( S = \{ z \in \C : -\pi < \Im(z) \leq \pi \}. \)

\begin{lemma}
   \( \exp: S \to \C^{\cdot} \) is a bijective map. 
\end{lemma}
\begin{proof}
Let \( z \in \C^{\cdot} \) and \( z = x + iy \). Let \( z = \gamma (\cos \varphi + i \sin \varphi) \) be the polar representation of \( z  \) such that \( -\pi < \varphi \leq \pi \). Define \( w = \ln \gamma + i \varphi \). Then \( \exp(w) = z  \) and so, \( \exp  \) is surjective. Moreover, \( \exp  \) is injective (on \( S \)) as well.
\end{proof}

\begin{corollary}
    For any \( z \in \C^{\cdot} \), we can find a unique \( w \in S  \) such that \( \exp(w) = z \).
\end{corollary}

\begin{definition}[ ]
    Given \( z \in \C^{\cdot} \), the unique \( w \in S  \) is called the principal value of the logarithm of \( z  \) and we write \( w = \Log(z) \).
\end{definition}

\begin{theorem}[ ]
    There exists a mapping 
    \[  \Log: \C^{\cdot} \to \C  \] such that 
    \begin{enumerate}
        \item[(i)] \( \exp(\Log z) = z  \)
        \item[(ii)] \( - \pi < \Im(\Log z) \leq \pi \).
    \end{enumerate}
\end{theorem}

\begin{definition}[Principal Branch]
    The function \( \Log Z  \) is called the principal branch of the logarithm. 
\end{definition}

\begin{prop}
    Let \( z \in \C^{\cdot} \). Find all possible solutions of \( \exp(w) = z  \).
\end{prop}

\begin{lemma}
   For \( z \in \C^{\cdot} \), we have  
   \[  \Log(z) = \ln | z |  + i \Arg(z). \]
\end{lemma}

\subsection{Notion of logarithm}
For the proposition above, we can now see that the solutions given by the equation \( \exp(w) = z  \) are given by
\[  w = \Log(z) + 2 \pi i k  \ \text{for} \ z \in \Z   \]
where 
\[  \log z = \Log z + 2\pi i k  \]
is a \textbf{multivalued function} and \( \log z = \Log z  \) if we want the solution to be in \( S  \).




\end{document}
