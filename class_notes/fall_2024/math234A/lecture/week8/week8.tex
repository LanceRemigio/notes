\documentclass[a4paper]{article}

\usepackage[utf8]{inputenc}
\usepackage[T1]{fontenc}
% \usepackage{fourier}
\usepackage{textcomp}
\usepackage{hyperref}
\usepackage[english]{babel}
\usepackage{url}
% \usepackage{hyperref}
% \hypersetup{
%     colorlinks,
%     linkcolor={black},
%     citecolor={black},
%     urlcolor={blue!80!black}
% }
\usepackage{graphicx} \usepackage{float}
\usepackage{booktabs}
\usepackage{enumitem}
% \usepackage{parskip}
% \usepackage{parskip}
\usepackage{emptypage}
\usepackage{subcaption}
\usepackage{multicol}
\usepackage[usenames,dvipsnames]{xcolor}
\usepackage{ocgx}
% \usepackage{cmbright}


\usepackage[margin=1in]{geometry}
\usepackage{amsmath, amsfonts, mathtools, amsthm, amssymb}
\usepackage{thmtools}
\usepackage{mathrsfs}
\usepackage{cancel}
\usepackage{bm}
\newcommand\N{\ensuremath{\mathbb{N}}}
\newcommand\R{\ensuremath{\mathbb{R}}}
\newcommand\Z{\ensuremath{\mathbb{Z}}}
\renewcommand\O{\ensuremath{\emptyset}}
\newcommand\Q{\ensuremath{\mathbb{Q}}}
\newcommand\C{\ensuremath{\mathbb{C}}}
\newcommand\F{\ensuremath{\mathbb{F}}}
\DeclareMathOperator{\sgn}{sgn}
\DeclareMathOperator{\diam}{diam}
\DeclareMathOperator{\LO}{LO}
\DeclareMathOperator{\UP}{UP}
\DeclareMathOperator{\card}{card}
\DeclareMathOperator{\Arg}{Arg}
\DeclareMathOperator{\Dom}{Dom}
\DeclareMathOperator{\Log}{Log}
\DeclareMathOperator{\dist}{dist}
% \DeclareMathOperator{\span}{span}
\usepackage{systeme}
\let\svlim\lim\def\lim{\svlim\limits}
\renewcommand\implies\Longrightarrow
\let\impliedby\Longleftarrow
\let\iff\Longleftrightarrow
\let\epsilon\varepsilon
\usepackage{stmaryrd} % for \lightning
\newcommand\contra{\scalebox{1.1}{$\lightning$}}
% \let\phi\varphi
\renewcommand\qedsymbol{$\blacksquare$}

% correct
\definecolor{correct}{HTML}{009900}
\newcommand\correct[2]{\ensuremath{\:}{\color{red}{#1}}\ensuremath{\to }{\color{correct}{#2}}\ensuremath{\:}}
\newcommand\green[1]{{\color{correct}{#1}}}

% horizontal rule
\newcommand\hr{
    \noindent\rule[0.5ex]{\linewidth}{0.5pt}
}

% hide parts
\newcommand\hide[1]{}

% si unitx
\usepackage{siunitx}
\sisetup{locale = FR}
% \renewcommand\vec[1]{\mathbf{#1}}
\newcommand\mat[1]{\mathbf{#1}}

% tikz
\usepackage{tikz}
\usepackage{tikz-cd}
\usetikzlibrary{intersections, angles, quotes, calc, positioning}
\usetikzlibrary{arrows.meta}
\usepackage{pgfplots}
\pgfplotsset{compat=1.13}

\tikzset{
    force/.style={thick, {Circle[length=2pt]}-stealth, shorten <=-1pt}
}

% theorems
\makeatother
\usepackage{thmtools}
\usepackage[framemethod=TikZ]{mdframed}
\mdfsetup{skipabove=1em,skipbelow=1em}

\theoremstyle{definition}

\declaretheoremstyle[
    headfont=\bfseries\sffamily\color{ForestGreen!70!black}, bodyfont=\normalfont,
    mdframed={
        linewidth=1pt,
        rightline=false, topline=false, bottomline=false,
        linecolor=ForestGreen, backgroundcolor=ForestGreen!5,
    }
]{thmgreenbox}

\declaretheoremstyle[
    headfont=\bfseries\sffamily\color{NavyBlue!70!black}, bodyfont=\normalfont,
    mdframed={
        linewidth=1pt,
        rightline=false, topline=false, bottomline=false,
        linecolor=NavyBlue, backgroundcolor=NavyBlue!5,
    }
]{thmbluebox}

\declaretheoremstyle[
    headfont=\bfseries\sffamily\color{NavyBlue!70!black}, bodyfont=\normalfont,
    mdframed={
        linewidth=1pt,
        rightline=false, topline=false, bottomline=false,
        linecolor=NavyBlue
    }
]{thmblueline}

\declaretheoremstyle[
    headfont=\bfseries\sffamily, bodyfont=\normalfont,
    numbered = no,
    mdframed={
        rightline=true, topline=true, bottomline=true,
    }
]{thmbox}

\declaretheoremstyle[
    headfont=\bfseries\sffamily, bodyfont=\normalfont,
    numbered=no,
    % mdframed={
    %     rightline=true, topline=false, bottomline=true,
    % },
    qed=\qedsymbol
]{thmproofbox}

\declaretheoremstyle[
    headfont=\bfseries\sffamily\color{NavyBlue!70!black}, bodyfont=\normalfont,
    numbered=no,
    mdframed={
        rightline=false, topline=false, bottomline=false,
        linecolor=NavyBlue, backgroundcolor=NavyBlue!1,
    },
]{thmexplanationbox}

\declaretheorem[
    style=thmbox, 
    % numberwithin = section,
    numbered = no,
    name=Definition
    ]{definition}

\declaretheorem[
    style=thmbox, 
    name=Example,
    ]{eg}

\declaretheorem[
    style=thmbox, 
    % numberwithin = section,
    name=Proposition]{prop}

\declaretheorem[
    style = thmbox,
    numbered=yes,
    name =Problem
    ]{problem}

\declaretheorem[style=thmbox, name=Theorem]{theorem}
\declaretheorem[style=thmbox, name=Lemma]{lemma}
\declaretheorem[style=thmbox, name=Corollary]{corollary}

\declaretheorem[style=thmproofbox, name=Proof]{replacementproof}

\declaretheorem[style=thmproofbox, 
                name = Solution
                ]{replacementsolution}

\renewenvironment{proof}[1][\proofname]{\vspace{-1pt}\begin{replacementproof}}{\end{replacementproof}}

\newenvironment{solution}
    {
        \vspace{-1pt}\begin{replacementsolution}
    }
    { 
            \end{replacementsolution}
    }

\declaretheorem[style=thmexplanationbox, name=Proof]{tmpexplanation}
\newenvironment{explanation}[1][]{\vspace{-10pt}\begin{tmpexplanation}}{\end{tmpexplanation}}

\declaretheorem[style=thmbox, numbered=no, name=Remark]{remark}
\declaretheorem[style=thmbox, numbered=no, name=Note]{note}

\newtheorem*{uovt}{UOVT}
\newtheorem*{notation}{Notation}
\newtheorem*{previouslyseen}{As previously seen}
% \newtheorem*{problem}{Problem}
\newtheorem*{observe}{Observe}
\newtheorem*{property}{Property}
\newtheorem*{intuition}{Intuition}

\usepackage{etoolbox}
\AtEndEnvironment{vb}{\null\hfill$\diamond$}%
\AtEndEnvironment{intermezzo}{\null\hfill$\diamond$}%
% \AtEndEnvironment{opmerking}{\null\hfill$\diamond$}%

% http://tex.stackexchange.com/questions/22119/how-can-i-change-the-spacing-before-theorems-with-amsthm
\makeatletter
% \def\thm@space@setup{%
%   \thm@preskip=\parskip \thm@postskip=0pt
% }
\newcommand{\oefening}[1]{%
    \def\@oefening{#1}%
    \subsection*{Oefening #1}
}

\newcommand{\suboefening}[1]{%
    \subsubsection*{Oefening \@oefening.#1}
}

\newcommand{\exercise}[1]{%
    \def\@exercise{#1}%
    \subsection*{Exercise #1}
}

\newcommand{\subexercise}[1]{%
    \subsubsection*{Exercise \@exercise.#1}
}


\usepackage{xifthen}

\def\testdateparts#1{\dateparts#1\relax}
\def\dateparts#1 #2 #3 #4 #5\relax{
    \marginpar{\small\textsf{\mbox{#1 #2 #3 #5}}}
}

\def\@lesson{}%
\newcommand{\lesson}[3]{
    \ifthenelse{\isempty{#3}}{%
        \def\@lesson{Lecture #1}%
    }{%
        \def\@lesson{Lecture #1: #3}%
    }%
    \subsection*{\@lesson}
    \testdateparts{#2}
}

% \renewcommand\date[1]{\marginpar{#1}}


% fancy headers
\usepackage{fancyhdr}
\pagestyle{fancy}

\makeatother

% notes
\usepackage{todonotes}
\usepackage{tcolorbox}

\tcbuselibrary{breakable}
\newenvironment{verbetering}{\begin{tcolorbox}[
    arc=0mm,
    colback=white,
    colframe=green!60!black,
    title=Opmerking,
    fonttitle=\sffamily,
    breakable
]}{\end{tcolorbox}}

\newenvironment{noot}[1]{\begin{tcolorbox}[
    arc=0mm,
    colback=white,
    colframe=white!60!black,
    title=#1,
    fonttitle=\sffamily,
    breakable
]}{\end{tcolorbox}}

% figure support
\usepackage{import}
\usepackage{xifthen}
\pdfminorversion=7
\usepackage{pdfpages}
\usepackage{transparent}
\newcommand{\incfig}[1]{%
    \def\svgwidth{\columnwidth}
    \import{./figures/}{#1.pdf_tex}
}

% %http://tex.stackexchange.com/questions/76273/multiple-pdfs-with-page-group-included-in-a-single-page-warning
\pdfsuppresswarningpagegroup=1



\begin{document}

\section{Handout-11}

In this handout, we will relate complex differentiability with differentiability in multivariable calculus.

\begin{definition}[ ]
    Let \( f: D \to \C  \), \( D \subseteq \C  \) be a function. We say that \( f  \) is complex differentiable on \( D  \) if \( f \) is differentiable at each \( a \in D  \).
\end{definition}
 If \( f  \) is differentiable on \( D  \), then we can define a function \( f' : D \to \C  \) by \( z \mapsto f'(z) \). The function \( f'  \) is called the complex derivative of \( f \).

 \begin{remark}
     Assume that \( D = [a,b] \subseteq \R  \). By abusing notation, let us write \( f(x) \) with \( x \in [a,b] \). Let \( f(x) = u(x) + i v(x) \). Then, one can show that \( f  \) is differentiable if and only if \( u  \) and \( v  \) are differentiable. If \( f  \) is differentiable, then \( f'(x) = u'(x) + i v'(x)  \).
 \end{remark}

 \begin{eg}[Computing Complex Derivatives]
    \begin{enumerate}
        \item[(i)] Let \( f: \C \to \C  \), \( f(z) = z^{n} \), \( n \in \Z^{+} \). Then \( f'(z) = n z^{n-1} \). The result follows from 
            \[  z^{n} - {w}_{n} = (z-w) (z^{n-1} + z^{n-2} w + \cdots + z w^{n-2} + w^{n-1}). \]
        \item[(ii)] Let \( P(z) = {a}_{0} + {a}_{1} z + \cdots + {a}_{n-1} z^{n-1} + {a}_{n} z^{n} \) for \( z \in \C  \) and \( {a}_{0}, \dots , {a}_{n} \in \C  \) are constants and \( n \in \Z^{+} \). Then
            \[  P'(z) = \sum_{ k=1  }^{ n } k {a}_{k} z^{k-1}.  \]
        \item[(iii)] Power series:
            Let \( {c}_{0}, {c}_{1}, {c}_{2}, \dots  \) be a sequence of complex numbers and \( a \in \C  \). A series of the form \( \sum_{ n=0  }^{ \infty} {c}_{n} (z- a )^{n}   \) is called a power series centered at \( a  \) with coefficients \( \{ {c}_{n} \}_{n=0}^{\infty } \). Assume that the series converges on an open ball \( B(a,R) \) for some \( R  \); that is, for each \( z \in B(a,R) \), the series \( \sum_{ n=0 }^{ \infty  } {c}_{n} (z-a)^{n} \) exists. Then we can define \( f: B(a,R) \to \C  \) by
            \[  f(z) = \sum_{ n=0  }^{ \infty  } {c}_{n} (z-a)^{n}. \]
        \item[(iv)] Let \( f: \C \to \C  \), \( f(z) = \overline{z} \). We claim that \( f \) is not complex differentiable at \( z = 0  \). We have \( f(z) - f(0) = \overline{z} \) and thus,
            \[  \lim_{ z \to 0 }  \frac{ f(z) }{ z    }  =  \lim_{ z \to 0 } \frac{ \overline{z} }{ z } = 
            \begin{cases}
                1 &\text{if} \ z \to 0 \ \text{along the real axis} \\
                -1 &\text{if} \ z \to 0 \  \text{along the imaginary axis}
            \end{cases} \]
    \end{enumerate} 
    \item[(v)] Consider the function \( \Log: \C^{\bullet} \to \C  \). We say that it is NOT continuous along the negative real axis. Let 
        \[  D = {\C}_{-} = \C^{\bullet} \setminus  \{ z \in \C | z < 0 \}. \]
        Then, \( \Log: D \to \C  \) is complex differentiable and 
        \[  (\Log z)' = \frac{ 1 }{ z }   \]
        on \( D  \).
        Assuming complex differentiability, we can use \( e^{\Log z } = z  \) and chain rule:
        \begin{align*}
            &\implies e^{\Log z} \cdot (\Log z)' = 1 \\
            &\implies (\Log z )' = \frac{ 1 }{ e^{\Log z}  }  = \frac{ 1 }{ z }. 
        \end{align*}
    \item[(vi)] Show that \( f: \C \to \C  \), \( f(z) = | z |^{2}  = z \overline{z} \) not complex differentiable except for \( z = 0 \). Also, show that \( f'(0) = 0  \).
 \end{eg}

\begin{remark} 
    When \( D \subseteq \C   \) is open, the notion of complex differentiability becomes interesting and has very different behavior from that of \( f: U \to \R  \), \( U \subseteq \R   \) open, \( f  \) differentiable. We will expand on this later. 
\end{remark}

Let us start with recalling some multivariable calculus concepts. Let \( D \subseteq \R^{p} \) be open and \( f: D \to \R^{q} \) be a function. We say that \( f  \) is differentiable at \( \vec{ a }  \in D  \) if we can find a linear function \( A: \R^{p} \to \R^{q} \) such that
\begin{enumerate}
    \item[(i)] \( f(\vec{ x }) = f(\vec{ a } ) + A (\vec{ x }  - \vec{ a } ) + \gamma(\vec{ x } )  \)
    \item[(ii)] \( \lim_{ \vec{ x }  \to \vec{ a }  } \frac{ \gamma(\overline{x}) }{ \|\vec{ x } - \vec{ a } \| = 0 }. \)
\end{enumerate}
The linear map is called the Jacobian of \( f  \) at \( a  \) and it is written as \( J(f;a) \). The following proposition gives us a connection between complex differentiability and differentiability in the multivariable sense when \( D \subseteq \C   \) is open.

\begin{prop}[ ]
   Let \( D \subseteq \C   \) be an open set and \( f: D \to \C  \). Let \( a \in D  \) and \( \ell \in \C  \). Then the following statements are equivalent.
   \begin{enumerate}
       \item[(i)] \( f  \) is complex differentiable at \( a  \) and \( f'(a) = \ell \).
        \item[(ii)] \( f  \) is differentiable at \( a \in D   \) in the sense of multivariable calculus (here we think of \( D \subseteq  \R^{2} \) open and \( \C  \) is identified with \( \R^{2} \)) and   
            \[  J(f,a) = \begin{pmatrix} \Re(\ell) & - \Im(\ell) \\ \Im(\ell) & \Re(\ell) \end{pmatrix}.  \]
   \end{enumerate} 
\end{prop}

In fact, if we indetify \( \begin{pmatrix} x \\ y \end{pmatrix}  \) with \( x + i y   \), then \( J(f,a) \begin{pmatrix} x \\ y \end{pmatrix} = \begin{pmatrix} \Re(\ell) x - \Im(\ell) y \\ \Re(\ell)y + \Im(\ell) x \end{pmatrix}  \); that is, \( J(f,a) z = \ell \cdot z \).

\section{Handout-12}

\subsection{Plan}

Learn the necessary and sufficient condition for complex differentiability of a function \( f: D \to \C  \) when \( D  \) is open in \( \C  \).

\subsection{Recap}

Let \( f: D \to \C  \), \( D \subseteq \C   \) open, \( a \in D  \) and \( \ell \in \C  \). We saw that \( f \) is differentiability at \( a \in D  \) and \( f'(a) = \ell \) is equivalent to \( f  \) is differentiable at \( a  \) in the sense of multivariate calculus and   
\[  J(f,a) = \begin{pmatrix} \Re(\ell) & - \Im(\ell) \\ \Im(\ell) & \Re(\ell) \end{pmatrix}
\]
when \( f  \) is regarded as a function from a subset of \( \R^{2} \) to \( \R^{2} \).

This suggests the following question: Consider the setting as above. Assume that \( f  \) is differentiable in the sense of multivariate calculus at \( a \in D  \). Is is necessarily true that \( f  \) is complex differentiable at \( a  \)? \textbf{Ans. NO}

\begin{eg}
    Define \( f: \C \to \C  \) by \( f(z) = \overline{z} \); that is, \( f(x,y) = (x,-y) \) when regarded as a map from \( \R^{2} \to \R^{2} \). Let \( a = (0,0) \). Recall that \( f  \) is not complex differentiable at \( 0  \).
\end{eg}

Since both components of \( f  \) have continuous partial derivatives, \( f  \) is differentiable in the sense of multivariate calculus. Our next question then becomes:
\begin{center}
    \textit{What other conditions do we need in addition to our differentiability in the sense of multivariate calculus in order to guarantee complex differentiability?}
\end{center}

This is the content of the Cauchy-Riemann Equations. Before we discuss this we make a digression to linear algebra. 

\begin{problem}
    Let \( \ell \in \C  \) and define a map \( L: \C \to \C  \) by \( L(z) = \ell z  \).
    \begin{enumerate}
        \item[(i)] Show that \( L  \) is \( \R- \)linear and \( \C- \)linear.
        \item[(ii)] Think of \( L  \) as a map from \( \R^{2} \to \R^{2} \) and find a matrix representation of \( L  \) with respect to the standard bases \( \Big\{ \begin{pmatrix} 1 \\ 0  \end{pmatrix} , \begin{pmatrix} 0 \\ 1  \end{pmatrix} \Big\}  \) on both domain and co-domain. Indeed, we have
            \[  L = \begin{pmatrix} \Re(\ell) & - \Im(\ell) \\ \Im(\ell) & \Re(\ell)  \end{pmatrix}. \]
        \item[(iii)] Let \( L = \begin{pmatrix} a & b \\ c & d  \end{pmatrix}  \) with \( a,b,c,d \in \R  \). Define \( L: \R^{2} \to \R^{2} \) by \( \begin{pmatrix} x \\ y  \end{pmatrix}  \mapsto \begin{pmatrix} ax + by \\ cx + dy \end{pmatrix}  \); that is, 
            \[  L(x+iy) = (ax+by) + i (cx+dy). \]
            Then, we know that \( L  \) is \( \R- \)linear. Find out a necessary and sufficient condition that \( L  \) is also \( \C  \)-linear.
    \end{enumerate}
\end{problem}

\begin{lemma}[Warm-up to Cauchy-Riemann]
   Let \( D \subseteq \R^{2} \) open, \( a \in D  \), \( f: D \to \R^{2} \) be differentiable at \( a \in D  \) in the sense of multivariable calculus. Let \( f \Big(  \begin{pmatrix} x \\ y  \end{pmatrix}  \Big) = \begin{pmatrix} u(x,y) \\ v(x,y) \end{pmatrix}  \). Then, we know \( J(f,a) : \R^{2} \to \R^{2} \) is \( \R- \)linear. Now, suppose that \( \C \to \C  \) is complex linear. Then 
   \[  \frac{\partial u }{\partial x } (a) = \frac{\partial v }{\partial y } \ \ \text{and} \ \ \frac{\partial u }{\partial y } (a) = - \frac{\partial v }{\partial x }  (a).  \]
\end{lemma}





\end{document}
