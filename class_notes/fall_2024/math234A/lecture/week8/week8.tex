\documentclass[a4paper]{article}

\usepackage[utf8]{inputenc}
\usepackage[T1]{fontenc}
\usepackage{textcomp}
\usepackage{hyperref}
% \usepackage{fourier}
% \usepackage[dutch]{babel}
\usepackage{url}
% \usepackage{hyperref}
% \hypersetup{
%     colorlinks,
%     linkcolor={black},
%     citecolor={black},
%     urlcolor={blue!80!black}
% }
\usepackage{graphicx}
\usepackage{float}
\usepackage{booktabs}
\usepackage{enumitem}
% \usepackage{parskip}
\usepackage{emptypage}
\usepackage{subcaption}
\usepackage{multicol}
\usepackage[usenames,dvipsnames]{xcolor}

% \usepackage{cmbright}


\usepackage[margin=1in]{geometry}
\usepackage{amsmath, amsfonts, mathtools, amsthm, amssymb}
\usepackage{mathrsfs}
\usepackage{cancel}
\usepackage{bm}
\newcommand\N{\ensuremath{\mathbb{N}}}
\newcommand\R{\ensuremath{\mathbb{R}}}
\newcommand\Z{\ensuremath{\mathbb{Z}}}
\renewcommand\O{\ensuremath{\emptyset}}
\newcommand\Q{\ensuremath{\mathbb{Q}}}
\newcommand\C{\ensuremath{\mathbb{C}}}
\DeclareMathOperator{\sgn}{sgn}
\usepackage{systeme}
\let\svlim\lim\def\lim{\svlim\limits}
\let\implies\Rightarrow
\let\impliedby\Leftarrow
\let\iff\Leftrightarrow
\let\epsilon\varepsilon
\usepackage{stmaryrd} % for \lightning
\newcommand\contra{\scalebox{1.1}{$\lightning$}}
% \let\phi\varphi
\renewcommand\qedsymbol{$\blacksquare$}




% correct
\definecolor{correct}{HTML}{009900}
\newcommand\correct[2]{\ensuremath{\:}{\color{red}{#1}}\ensuremath{\to }{\color{correct}{#2}}\ensuremath{\:}}
\newcommand\green[1]{{\color{correct}{#1}}}



% horizontal rule
\newcommand\hr{
    \noindent\rule[0.5ex]{\linewidth}{0.5pt}
}


% hide parts
\newcommand\hide[1]{}



% si unitx
\usepackage{siunitx}
\sisetup{locale = FR}
% \renewcommand\vec[1]{\mathbf{#1}}
\newcommand\mat[1]{\mathbf{#1}}


% tikz
\usepackage{tikz}
\usepackage{tikz-cd}
\usetikzlibrary{intersections, angles, quotes, calc, positioning}
\usetikzlibrary{arrows.meta}
\usepackage{pgfplots}
\pgfplotsset{compat=1.13}


\tikzset{
    force/.style={thick, {Circle[length=2pt]}-stealth, shorten <=-1pt}
}

% theorems
\makeatother
\usepackage{thmtools}
\usepackage[framemethod=TikZ]{mdframed}
\mdfsetup{skipabove=1em,skipbelow=0em}


\theoremstyle{definition}

\declaretheoremstyle[
    headfont=\bfseries\sffamily\color{ForestGreen!70!black}, bodyfont=\normalfont,
    mdframed={
        linewidth=2pt,
        rightline=false, topline=false, bottomline=false,
        linecolor=ForestGreen, backgroundcolor=ForestGreen!5,
    }
]{thmgreenbox}

\declaretheoremstyle[
    headfont=\bfseries\sffamily\color{NavyBlue!70!black}, bodyfont=\normalfont,
    mdframed={
        linewidth=2pt,
        rightline=false, topline=false, bottomline=false,
        linecolor=NavyBlue, backgroundcolor=NavyBlue!5,
    }
]{thmbluebox}

\declaretheoremstyle[
    headfont=\bfseries\sffamily\color{NavyBlue!70!black}, bodyfont=\normalfont,
    mdframed={
        linewidth=2pt,
        rightline=false, topline=false, bottomline=false,
        linecolor=NavyBlue
    }
]{thmblueline}

\declaretheoremstyle[
    headfont=\bfseries\sffamily\color{RawSienna!70!black}, bodyfont=\normalfont,
    mdframed={
        linewidth=2pt,
        rightline=false, topline=false, bottomline=false,
        linecolor=RawSienna, backgroundcolor=RawSienna!5,
    }
]{thmredbox}

\declaretheoremstyle[
    headfont=\bfseries\sffamily\color{RawSienna!70!black}, bodyfont=\normalfont,
    numbered=no,
    mdframed={
        linewidth=2pt,
        rightline=false, topline=false, bottomline=false,
        linecolor=RawSienna, backgroundcolor=RawSienna!1,
    },
    qed=\qedsymbol
]{thmproofbox}

\declaretheoremstyle[
    headfont=\bfseries\sffamily\color{NavyBlue!70!black}, bodyfont=\normalfont,
    numbered=no,
    mdframed={
        linewidth=2pt,
        rightline=false, topline=false, bottomline=false,
        linecolor=NavyBlue, backgroundcolor=NavyBlue!1,
    },
]{thmexplanationbox}

\declaretheorem[style=thmgreenbox, numberwithin = section, name=Definition]{definition}
\declaretheorem[style=thmbluebox, name=Example]{eg}
\declaretheorem[style=thmredbox, numberwithin = section, name=Proposition]{prop}
\declaretheorem[style=thmredbox, numberwithin = section, name=Theorem]{theorem}
\declaretheorem[style=thmredbox, numberwithin = section,  name=Lemma]{lemma}
\declaretheorem[style=thmredbox, numberwithin = section,  numbered=no, name=Corollary]{corollary}


\declaretheorem[style=thmproofbox, name=Proof]{replacementproof}
\renewenvironment{proof}[1][\proofname]{\vspace{-10pt}\begin{replacementproof}}{\end{replacementproof}}


\declaretheorem[style=thmexplanationbox, name=Proof]{tmpexplanation}
\newenvironment{explanation}[1][]{\vspace{-10pt}\begin{tmpexplanation}}{\end{tmpexplanation}}


\declaretheorem[style=thmblueline, numbered=no, name=Remark]{remark}
\declaretheorem[style=thmblueline, numbered=no, name=Note]{note}

\newtheorem*{uovt}{UOVT}
\newtheorem*{notation}{Notation}
\newtheorem*{previouslyseen}{As previously seen}
\newtheorem*{problem}{Problem}
\newtheorem*{observe}{Observe}
\newtheorem*{property}{Property}
\newtheorem*{intuition}{Intuition}


\usepackage{etoolbox}
\AtEndEnvironment{vb}{\null\hfill$\diamond$}%
\AtEndEnvironment{intermezzo}{\null\hfill$\diamond$}%
% \AtEndEnvironment{opmerking}{\null\hfill$\diamond$}%

% http://tex.stackexchange.com/questions/22119/how-can-i-change-the-spacing-before-theorems-with-amsthm
\makeatletter
% \def\thm@space@setup{%
%   \thm@preskip=\parskip \thm@postskip=0pt
% }
\newcommand{\oefening}[1]{%
    \def\@oefening{#1}%
    \subsection*{Oefening #1}
}

\newcommand{\suboefening}[1]{%
    \subsubsection*{Oefening \@oefening.#1}
}

\newcommand{\exercise}[1]{%
    \def\@exercise{#1}%
    \subsection*{Exercise #1}
}

\newcommand{\subexercise}[1]{%
    \subsubsection*{Exercise \@exercise.#1}
}


\usepackage{xifthen}

\def\testdateparts#1{\dateparts#1\relax}
\def\dateparts#1 #2 #3 #4 #5\relax{
    \marginpar{\small\textsf{\mbox{#1 #2 #3 #5}}}
}

\def\@lesson{}%
\newcommand{\lesson}[3]{
    \ifthenelse{\isempty{#3}}{%
        \def\@lesson{Lecture #1}%
    }{%
        \def\@lesson{Lecture #1: #3}%
    }%
    \subsection*{\@lesson}
    \testdateparts{#2}
}

% \renewcommand\date[1]{\marginpar{#1}}


% fancy headers
\usepackage{fancyhdr}
\pagestyle{fancy}

\fancyhead[LE,RO]{Lance Remigio}
\fancyhead[RO,LE]{\@lesson}
\fancyhead[RE,LO]{}
\fancyfoot[LE,RO]{\thepage}
\fancyfoot[C]{\leftmark}

\makeatother




% notes
\usepackage{todonotes}
\usepackage{tcolorbox}

\tcbuselibrary{breakable}
\newenvironment{verbetering}{\begin{tcolorbox}[
    arc=0mm,
    colback=white,
    colframe=green!60!black,
    title=Opmerking,
    fonttitle=\sffamily,
    breakable
]}{\end{tcolorbox}}

\newenvironment{noot}[1]{\begin{tcolorbox}[
    arc=0mm,
    colback=white,
    colframe=white!60!black,
    title=#1,
    fonttitle=\sffamily,
    breakable
]}{\end{tcolorbox}}




% figure support
\usepackage{import}
\usepackage{xifthen}
\pdfminorversion=7
\usepackage{pdfpages}
\usepackage{transparent}
\newcommand{\incfig}[1]{%
    \def\svgwidth{\columnwidth}
    \import{./figures/}{#1.pdf_tex}
}

% %http://tex.stackexchange.com/questions/76273/multiple-pdfs-with-page-group-included-in-a-single-page-warning
\pdfsuppresswarningpagegroup=1




\begin{document}

\section{Handout-11}

In this handout, we will relate complex differentiability with differentiability in multivariable calculus.

\begin{definition}[ ]
    Let \( f: D \to \C  \), \( D \subseteq \C  \) be a function. We say that \( f  \) is complex differentiable on \( D  \) if \( f \) is differentiable at each \( a \in D  \).
\end{definition}
 If \( f  \) is differentiable on \( D  \), then we can define a function \( f' : D \to \C  \) by \( z \mapsto f'(z) \). The function \( f'  \) is called the complex derivative of \( f \).

 \begin{remark}
     Assume that \( D = [a,b] \subseteq \R  \). By abusing notation, let us write \( f(x) \) with \( x \in [a,b] \). Let \( f(x) = u(x) + i v(x) \). Then, one can show that \( f  \) is differentiable if and only if \( u  \) and \( v  \) are differentiable. If \( f  \) is differentiable, then \( f'(x) = u'(x) + i v'(x)  \).
 \end{remark}

 \begin{eg}[Computing Complex Derivatives]
    \begin{enumerate}
        \item[(i)] Let \( f: \C \to \C  \), \( f(z) = z^{n} \), \( n \in \Z^{+} \). Then \( f'(z) = n z^{n-1} \). The result follows from 
            \[  z^{n} - {w}_{n} = (z-w) (z^{n-1} + z^{n-2} w + \cdots + z w^{n-2} + w^{n-1}). \]
        \item[(ii)] Let \( P(z) = {a}_{0} + {a}_{1} z + \cdots + {a}_{n-1} z^{n-1} + {a}_{n} z^{n} \) for \( z \in \C  \) and \( {a}_{0}, \dots , {a}_{n} \in \C  \) are constants and \( n \in \Z^{+} \). Then
            \[  P'(z) = \sum_{ k=1  }^{ n } k {a}_{k} z^{k-1}.  \]
        \item[(iii)] Power series:
            Let \( {c}_{0}, {c}_{1}, {c}_{2}, \dots  \) be a sequence of complex numbers and \( a \in \C  \). A series of the form \( \sum_{ n=0  }^{ \infty} {c}_{n} (z- a )^{n}   \) is called a power series centered at \( a  \) with coefficients \( \{ {c}_{n} \}_{n=0}^{\infty } \). Assume that the series converges on an open ball \( B(a,R) \) for some \( R  \); that is, for each \( z \in B(a,R) \), the series \( \sum_{ n=0 }^{ \infty  } {c}_{n} (z-a)^{n} \) exists. Then we can define \( f: B(a,R) \to \C  \) by
            \[  f(z) = \sum_{ n=0  }^{ \infty  } {c}_{n} (z-a)^{n}. \]
        \item[(iv)] Let \( f: \C \to \C  \), \( f(z) = \overline{z} \). We claim that \( f \) is not complex differentiable at \( z = 0  \). We have \( f(z) - f(0) = \overline{z} \) and thus,
            \[  \lim_{ z \to 0 }  \frac{ f(z) }{ z    }  =  \lim_{ z \to 0 } \frac{ \overline{z} }{ z } = 
            \begin{cases}
                1 &\text{if} \ z \to 0 \ \text{along the real axis} \\
                -1 &\text{if} \ z \to 0 \  \text{along the imaginary axis}
            \end{cases} \]
    \end{enumerate} 
    \item[(v)] Consider the function \( \Log: \C^{\bullet} \to \C  \). We say that it is NOT continuous along the negative real axis. Let 
        \[  D = {\C}_{-} = \C^{\bullet} \setminus  \{ z \in \C | z < 0 \}. \]
        Then, \( \Log: D \to \C  \) is complex differentiable and 
        \[  (\Log z)' = \frac{ 1 }{ z }   \]
        on \( D  \).
        Assuming complex differentiability, we can use \( e^{\Log z } = z  \) and chain rule:
        \begin{align*}
            &\implies e^{\Log z} \cdot (\Log z)' = 1 \\
            &\implies (\Log z )' = \frac{ 1 }{ e^{\Log z}  }  = \frac{ 1 }{ z }. 
        \end{align*}

 \end{eg}




\end{document}
