\documentclass[a4paper]{report}
\usepackage{standalone}
\usepackage{import}

\usepackage[utf8]{inputenc}
\usepackage[T1]{fontenc}
% \usepackage{fourier}
\usepackage{textcomp}
\usepackage{hyperref}
\usepackage[english]{babel}
\usepackage{url}
% \usepackage{hyperref}
% \hypersetup{
%     colorlinks,
%     linkcolor={black},
%     citecolor={black},
%     urlcolor={blue!80!black}
% }
\usepackage{graphicx} \usepackage{float}
\usepackage{booktabs}
\usepackage{enumitem}
% \usepackage{parskip}
% \usepackage{parskip}
\usepackage{emptypage}
\usepackage{subcaption}
\usepackage{multicol}
\usepackage[usenames,dvipsnames]{xcolor}
\usepackage{ocgx}
% \usepackage{cmbright}


\usepackage[margin=1in]{geometry}
\usepackage{amsmath, amsfonts, mathtools, amsthm, amssymb}
\usepackage{thmtools}
\usepackage{mathrsfs}
\usepackage{cancel}
\usepackage{bm}
\newcommand\N{\ensuremath{\mathbb{N}}}
\newcommand\R{\ensuremath{\mathbb{R}}}
\newcommand\Z{\ensuremath{\mathbb{Z}}}
\renewcommand\O{\ensuremath{\emptyset}}
\newcommand\Q{\ensuremath{\mathbb{Q}}}
\newcommand\C{\ensuremath{\mathbb{C}}}
\newcommand\F{\ensuremath{\mathbb{F}}}
\DeclareMathOperator{\sgn}{sgn}
\DeclareMathOperator{\diam}{diam}
\DeclareMathOperator{\LO}{LO}
\DeclareMathOperator{\UP}{UP}
\DeclareMathOperator{\card}{card}
\DeclareMathOperator{\Arg}{Arg}
\DeclareMathOperator{\Dom}{Dom}
\DeclareMathOperator{\Log}{Log}
\DeclareMathOperator{\dist}{dist}
% \DeclareMathOperator{\span}{span}
\usepackage{systeme}
\let\svlim\lim\def\lim{\svlim\limits}
\renewcommand\implies\Longrightarrow
\let\impliedby\Longleftarrow
\let\iff\Longleftrightarrow
\let\epsilon\varepsilon
\usepackage{stmaryrd} % for \lightning
\newcommand\contra{\scalebox{1.1}{$\lightning$}}
% \let\phi\varphi
\renewcommand\qedsymbol{$\blacksquare$}

% correct
\definecolor{correct}{HTML}{009900}
\newcommand\correct[2]{\ensuremath{\:}{\color{red}{#1}}\ensuremath{\to }{\color{correct}{#2}}\ensuremath{\:}}
\newcommand\green[1]{{\color{correct}{#1}}}

% horizontal rule
\newcommand\hr{
    \noindent\rule[0.5ex]{\linewidth}{0.5pt}
}

% hide parts
\newcommand\hide[1]{}

% si unitx
\usepackage{siunitx}
\sisetup{locale = FR}
% \renewcommand\vec[1]{\mathbf{#1}}
\newcommand\mat[1]{\mathbf{#1}}

% tikz
\usepackage{tikz}
\usepackage{tikz-cd}
\usetikzlibrary{intersections, angles, quotes, calc, positioning}
\usetikzlibrary{arrows.meta}
\usepackage{pgfplots}
\pgfplotsset{compat=1.13}

\tikzset{
    force/.style={thick, {Circle[length=2pt]}-stealth, shorten <=-1pt}
}

% theorems
\makeatother
\usepackage{thmtools}
\usepackage[framemethod=TikZ]{mdframed}
\mdfsetup{skipabove=1em,skipbelow=1em}

\theoremstyle{definition}

\declaretheoremstyle[
    headfont=\bfseries\sffamily\color{ForestGreen!70!black}, bodyfont=\normalfont,
    mdframed={
        linewidth=1pt,
        rightline=false, topline=false, bottomline=false,
        linecolor=ForestGreen, backgroundcolor=ForestGreen!5,
    }
]{thmgreenbox}

\declaretheoremstyle[
    headfont=\bfseries\sffamily\color{NavyBlue!70!black}, bodyfont=\normalfont,
    mdframed={
        linewidth=1pt,
        rightline=false, topline=false, bottomline=false,
        linecolor=NavyBlue, backgroundcolor=NavyBlue!5,
    }
]{thmbluebox}

\declaretheoremstyle[
    headfont=\bfseries\sffamily\color{NavyBlue!70!black}, bodyfont=\normalfont,
    mdframed={
        linewidth=1pt,
        rightline=false, topline=false, bottomline=false,
        linecolor=NavyBlue
    }
]{thmblueline}

\declaretheoremstyle[
    headfont=\bfseries\sffamily, bodyfont=\normalfont,
    numbered = no,
    mdframed={
        rightline=true, topline=true, bottomline=true,
    }
]{thmbox}

\declaretheoremstyle[
    headfont=\bfseries\sffamily, bodyfont=\normalfont,
    numbered=no,
    % mdframed={
    %     rightline=true, topline=false, bottomline=true,
    % },
    qed=\qedsymbol
]{thmproofbox}

\declaretheoremstyle[
    headfont=\bfseries\sffamily\color{NavyBlue!70!black}, bodyfont=\normalfont,
    numbered=no,
    mdframed={
        rightline=false, topline=false, bottomline=false,
        linecolor=NavyBlue, backgroundcolor=NavyBlue!1,
    },
]{thmexplanationbox}

\declaretheorem[
    style=thmbox, 
    % numberwithin = section,
    numbered = no,
    name=Definition
    ]{definition}

\declaretheorem[
    style=thmbox, 
    name=Example,
    ]{eg}

\declaretheorem[
    style=thmbox, 
    % numberwithin = section,
    name=Proposition]{prop}

\declaretheorem[
    style = thmbox,
    numbered=yes,
    name =Problem
    ]{problem}

\declaretheorem[style=thmbox, name=Theorem]{theorem}
\declaretheorem[style=thmbox, name=Lemma]{lemma}
\declaretheorem[style=thmbox, name=Corollary]{corollary}

\declaretheorem[style=thmproofbox, name=Proof]{replacementproof}

\declaretheorem[style=thmproofbox, 
                name = Solution
                ]{replacementsolution}

\renewenvironment{proof}[1][\proofname]{\vspace{-1pt}\begin{replacementproof}}{\end{replacementproof}}

\newenvironment{solution}
    {
        \vspace{-1pt}\begin{replacementsolution}
    }
    { 
            \end{replacementsolution}
    }

\declaretheorem[style=thmexplanationbox, name=Proof]{tmpexplanation}
\newenvironment{explanation}[1][]{\vspace{-10pt}\begin{tmpexplanation}}{\end{tmpexplanation}}

\declaretheorem[style=thmbox, numbered=no, name=Remark]{remark}
\declaretheorem[style=thmbox, numbered=no, name=Note]{note}

\newtheorem*{uovt}{UOVT}
\newtheorem*{notation}{Notation}
\newtheorem*{previouslyseen}{As previously seen}
% \newtheorem*{problem}{Problem}
\newtheorem*{observe}{Observe}
\newtheorem*{property}{Property}
\newtheorem*{intuition}{Intuition}

\usepackage{etoolbox}
\AtEndEnvironment{vb}{\null\hfill$\diamond$}%
\AtEndEnvironment{intermezzo}{\null\hfill$\diamond$}%
% \AtEndEnvironment{opmerking}{\null\hfill$\diamond$}%

% http://tex.stackexchange.com/questions/22119/how-can-i-change-the-spacing-before-theorems-with-amsthm
\makeatletter
% \def\thm@space@setup{%
%   \thm@preskip=\parskip \thm@postskip=0pt
% }
\newcommand{\oefening}[1]{%
    \def\@oefening{#1}%
    \subsection*{Oefening #1}
}

\newcommand{\suboefening}[1]{%
    \subsubsection*{Oefening \@oefening.#1}
}

\newcommand{\exercise}[1]{%
    \def\@exercise{#1}%
    \subsection*{Exercise #1}
}

\newcommand{\subexercise}[1]{%
    \subsubsection*{Exercise \@exercise.#1}
}


\usepackage{xifthen}

\def\testdateparts#1{\dateparts#1\relax}
\def\dateparts#1 #2 #3 #4 #5\relax{
    \marginpar{\small\textsf{\mbox{#1 #2 #3 #5}}}
}

\def\@lesson{}%
\newcommand{\lesson}[3]{
    \ifthenelse{\isempty{#3}}{%
        \def\@lesson{Lecture #1}%
    }{%
        \def\@lesson{Lecture #1: #3}%
    }%
    \subsection*{\@lesson}
    \testdateparts{#2}
}

% \renewcommand\date[1]{\marginpar{#1}}


% fancy headers
\usepackage{fancyhdr}
\pagestyle{fancy}

\makeatother

% notes
\usepackage{todonotes}
\usepackage{tcolorbox}

\tcbuselibrary{breakable}
\newenvironment{verbetering}{\begin{tcolorbox}[
    arc=0mm,
    colback=white,
    colframe=green!60!black,
    title=Opmerking,
    fonttitle=\sffamily,
    breakable
]}{\end{tcolorbox}}

\newenvironment{noot}[1]{\begin{tcolorbox}[
    arc=0mm,
    colback=white,
    colframe=white!60!black,
    title=#1,
    fonttitle=\sffamily,
    breakable
]}{\end{tcolorbox}}

% figure support
\usepackage{import}
\usepackage{xifthen}
\pdfminorversion=7
\usepackage{pdfpages}
\usepackage{transparent}
\newcommand{\incfig}[1]{%
    \def\svgwidth{\columnwidth}
    \import{./figures/}{#1.pdf_tex}
}

% %http://tex.stackexchange.com/questions/76273/multiple-pdfs-with-page-group-included-in-a-single-page-warning
\pdfsuppresswarningpagegroup=1




\begin{document}

\section{Lecture 4}

\subsection{Topics}

\begin{itemize}
    \item Continue discussion of convergence of sequences and series.
    \item Discuss exponential, sine, and cosine function.
\end{itemize}

\subsection{Class Exercises}

\begin{enumerate}
    \item[(i)] Show that \( \sum_{ n=0  }^{  \infty  } \frac{ z^{n} }{ n! }  \) converges for all \( z  \).
    \item[(ii)] Show that \( \sum_{ n= 0  }^{  \infty  } (-1)^{n} \frac{ z^{2n} }{ (2n)! }  \) converges for all \( z \).
    \item[(iii)] Show that \( \sum_{ n=0 }^{ \infty  } (-1)^{n} \frac{ z^{2n+1} }{ (2n+1)! }  \) converges for all \( z \).
\end{enumerate}

\begin{definition}[Exponential, Cosine, and Sine]
    We define 
    \begin{align*}
        \exp(z) &= \sum_{ n= 0  }^{  \infty  } \frac{ z^{n} }{ n! }  \\
        \cos(z) &= \sum_{ n=0 }^{ \infty  } \frac{ (-1)^{n} z^{2n} }{ (2n)! } \\
        \sin(z) &= \sum_{ n=0 }^{ \infty   } (-1)^{n} \frac{ z^{2n+1} }{ (2n+1)! }.
    \end{align*}
\end{definition}

Our main goal for this lecture is to show that \( \exp(z+w) = \exp(z) \exp(w) \).

\subsection{Cauchy Multiplication Theorem}

\begin{theorem}[Cauhcy Multiplication Theorem]
    Assume that \( \sum_{ n = 0  }^{  \infty  } {z}_{n} \) and \( \sum_{ n=0  }^{  \infty  } {w}_{n}  \) converges absolutely. Then 
    \[  \sum_{ n=0 }^{ \infty  } = \Big(  \sum_{  n = 0  }^{ \infty  } {z}_{n} \Big) \Big(  \sum_{ n= 0  }^{ \infty  } {w}_{n} \Big) \tag{1} \]
where the series in the left converges absolutely.
\end{theorem}
\begin{proof}
We will show that the sequence of partial sums of (1) satisfy the conditions of the monotone convergence theorem so that it converges. \textbf{Why is (1) monotone?} First, we show that (1) is bounded. Observe that
\begin{align*}
    \sum_{ n = 0    }^{ N  }  \Big|  \sum_{ i + j = n  }^{  } {z}_{i} {w}_{j} \Big|  &\leq \sum_{ n = 0  }^{ N  } \sum_{ i + j = n  }^{   } | {z}_{i} {w}_{j} |  \\
                                                                                     &= \sum_{ 0 \leq i + j \leq N   }^{  }| {z}_{i} {w}_{j} | \\ 
                                                                                     &\leq \sum_{ 0 \leq i, j \leq N  }^{  } | {z}_{i} {w}_{j} | \\
                                                                                     &= \sum_{i = 0  }^{ N  } | {z}_{i}  |  \sum_{ j=0  }^{ N  } | {w}_{j} |. 
\end{align*}
By our assumption, we can see that the sequence of partial sums of \( \sum_{ n= 0  }^{  \infty  }    \) and \( \sum_{ n = 0  }^{  \infty  } {w}_{n} \) are bounded, and thus the left side of the equation above is bounded. Thus, we see that 
\[  \sum_{ n = 0  }^{  N  } \Big|  \sum_{  i + j = n  }^{   } {z}_{i} {w}_{j} \Big|  \] 
converges by the monotone convergence theorem. Next, we will show that 
\[  {\alpha}_{n} = \Big|  \sum_{ n = 0  }^{ 2N  } \sum_{  i + j = n  } {z}_{i} {w}_{j} - \sum_{ i = 0  }^{ N  } {z}_{i} \sum_{ j = 0  } {w}_{j}  \Big| \to 0  \]
as \( N \to \infty  \). Let us define the following sets
\begin{align*}
    {T}_{N} &= \{ (i,j) \in \Z \times \Z : i \geq 0 , j \geq 0, 0 \leq i + j \leq N  \}  \\
    {t}_{n} &= \{ (i,j) \in \Z \times \Z : 0 \leq i \leq N , 0 \leq j \leq N  \}.
\end{align*}
Observe that \( {T}_{N} \subseteq  {t}_{n} \) and \( {t}_{N} \subseteq  {T}_{2N } \subseteq  {t}_{2N} \). Thus, 
\begin{align*}
    {\alpha}_{N} &= \Big|  \sum_{ (i,j) \in {T}_{2N} \setminus  {t}_{N} }^{  } {z}_{i} {w}_{j} \Big| \leq \sum_{ (i,j) \in {T}_{2N} \setminus {t}_{N} }^{  } | {z}_{i} {w}_{j} |  \\
                 &\leq \sum_{ (i,j) \in {t}_{2N} \setminus  {t}_{N} }^{  } | {z}_{i} {w}_{j} |  \\
                 &= \sum_{ i=0  }^{ 2N  } \sum_{ j = 0  }^{ 2N  } | {z}_{i} {w}_{j} |  - \sum_{ i = 0  }^{  N  } \sum_{ j = 0  }^{  N  } | {z}_{i} {w}_{j} |  \\
                 &= \sum_{ i = 0  }^{  2 N  } | {z}_{i} | \sum_{ j = 0  }^{  2N  } | {w}_{j} |  - \sum_{ i = 0  }^{  N  } | {z}_{i} |  \sum_{ j = 0  }^{  N  } | {w}_{j} | \to 0  \ \text{as} \ N \to \infty.
\end{align*}


\end{proof}

\begin{corollary}
   For any \( z,w \in \C  \), we have \( \exp(z) \cdot \exp(w) = \exp(z + w) \). 
\end{corollary}
\begin{proof}
Let
\[  {C}_{n} = \sum_{ k = 0  }^{ n  } \frac{ z^{k } }{  k!  }  \cdot \frac{ w^{n-k} }{ (n-k)! }. \]
Then multiplying by \( n!  \) on both sides of the equation above, we see that
\[  n! {C}_{n} = \sum_{ k = 0  }^{ n  } \begin{pmatrix} n \\ k  \end{pmatrix} z^{k } w^{n - k } = (z + w)^{n} \]
by the binomial formula where
\[  \begin{pmatrix} n \\ k  \end{pmatrix}  = \frac{ n!  }{ k! (n- k)! }. \]
Now, observe that
\[  \sum_{ n = 0  }^{  N  } {C}_{n} = \sum_{ n = 0  }^{  N  } \frac{ n! {C}_{n}  }{  n!  }  = \sum_{ n = 0  }^{  N  } \frac{ (z + w )^{n} }{ n!  }. \]
Let \( n \to \infty  \). Then we have
\[  \sum_{ n = 0  }^{  \infty  } {C}_{n} = \sum_{ n = 0  }^{  \infty  } \frac{ (z+w)^{n} }{ n!  }  = \exp(z+w). \]

\end{proof}


\section{Lecture 5}





\end{document}
