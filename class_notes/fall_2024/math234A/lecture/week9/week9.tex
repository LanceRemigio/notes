\documentclass[a4paper]{article}
\usepackage[utf8]{inputenc}
\usepackage[T1]{fontenc}
% \usepackage{fourier}
\usepackage{textcomp}
\usepackage{hyperref}
\usepackage[english]{babel}
\usepackage{url}
% \usepackage{hyperref}
% \hypersetup{
%     colorlinks,
%     linkcolor={black},
%     citecolor={black},
%     urlcolor={blue!80!black}
% }
\usepackage{graphicx} \usepackage{float}
\usepackage{booktabs}
\usepackage{enumitem}
% \usepackage{parskip}
% \usepackage{parskip}
\usepackage{emptypage}
\usepackage{subcaption}
\usepackage{multicol}
\usepackage[usenames,dvipsnames]{xcolor}
\usepackage{ocgx}
% \usepackage{cmbright}


\usepackage[margin=1in]{geometry}
\usepackage{amsmath, amsfonts, mathtools, amsthm, amssymb}
\usepackage{thmtools}
\usepackage{mathrsfs}
\usepackage{cancel}
\usepackage{bm}
\newcommand\N{\ensuremath{\mathbb{N}}}
\newcommand\R{\ensuremath{\mathbb{R}}}
\newcommand\Z{\ensuremath{\mathbb{Z}}}
\renewcommand\O{\ensuremath{\emptyset}}
\newcommand\Q{\ensuremath{\mathbb{Q}}}
\newcommand\C{\ensuremath{\mathbb{C}}}
\newcommand\F{\ensuremath{\mathbb{F}}}
\DeclareMathOperator{\sgn}{sgn}
\DeclareMathOperator{\diam}{diam}
\DeclareMathOperator{\LO}{LO}
\DeclareMathOperator{\UP}{UP}
\DeclareMathOperator{\card}{card}
\DeclareMathOperator{\Arg}{Arg}
\DeclareMathOperator{\Dom}{Dom}
\DeclareMathOperator{\Log}{Log}
\DeclareMathOperator{\dist}{dist}
% \DeclareMathOperator{\span}{span}
\usepackage{systeme}
\let\svlim\lim\def\lim{\svlim\limits}
\renewcommand\implies\Longrightarrow
\let\impliedby\Longleftarrow
\let\iff\Longleftrightarrow
\let\epsilon\varepsilon
\usepackage{stmaryrd} % for \lightning
\newcommand\contra{\scalebox{1.1}{$\lightning$}}
% \let\phi\varphi
\renewcommand\qedsymbol{$\blacksquare$}

% correct
\definecolor{correct}{HTML}{009900}
\newcommand\correct[2]{\ensuremath{\:}{\color{red}{#1}}\ensuremath{\to }{\color{correct}{#2}}\ensuremath{\:}}
\newcommand\green[1]{{\color{correct}{#1}}}

% horizontal rule
\newcommand\hr{
    \noindent\rule[0.5ex]{\linewidth}{0.5pt}
}

% hide parts
\newcommand\hide[1]{}

% si unitx
\usepackage{siunitx}
\sisetup{locale = FR}
% \renewcommand\vec[1]{\mathbf{#1}}
\newcommand\mat[1]{\mathbf{#1}}

% tikz
\usepackage{tikz}
\usepackage{tikz-cd}
\usetikzlibrary{intersections, angles, quotes, calc, positioning}
\usetikzlibrary{arrows.meta}
\usepackage{pgfplots}
\pgfplotsset{compat=1.13}

\tikzset{
    force/.style={thick, {Circle[length=2pt]}-stealth, shorten <=-1pt}
}

% theorems
\makeatother
\usepackage{thmtools}
\usepackage[framemethod=TikZ]{mdframed}
\mdfsetup{skipabove=1em,skipbelow=1em}

\theoremstyle{definition}

\declaretheoremstyle[
    headfont=\bfseries\sffamily\color{ForestGreen!70!black}, bodyfont=\normalfont,
    mdframed={
        linewidth=1pt,
        rightline=false, topline=false, bottomline=false,
        linecolor=ForestGreen, backgroundcolor=ForestGreen!5,
    }
]{thmgreenbox}

\declaretheoremstyle[
    headfont=\bfseries\sffamily\color{NavyBlue!70!black}, bodyfont=\normalfont,
    mdframed={
        linewidth=1pt,
        rightline=false, topline=false, bottomline=false,
        linecolor=NavyBlue, backgroundcolor=NavyBlue!5,
    }
]{thmbluebox}

\declaretheoremstyle[
    headfont=\bfseries\sffamily\color{NavyBlue!70!black}, bodyfont=\normalfont,
    mdframed={
        linewidth=1pt,
        rightline=false, topline=false, bottomline=false,
        linecolor=NavyBlue
    }
]{thmblueline}

\declaretheoremstyle[
    headfont=\bfseries\sffamily, bodyfont=\normalfont,
    numbered = no,
    mdframed={
        rightline=true, topline=true, bottomline=true,
    }
]{thmbox}

\declaretheoremstyle[
    headfont=\bfseries\sffamily, bodyfont=\normalfont,
    numbered=no,
    % mdframed={
    %     rightline=true, topline=false, bottomline=true,
    % },
    qed=\qedsymbol
]{thmproofbox}

\declaretheoremstyle[
    headfont=\bfseries\sffamily\color{NavyBlue!70!black}, bodyfont=\normalfont,
    numbered=no,
    mdframed={
        rightline=false, topline=false, bottomline=false,
        linecolor=NavyBlue, backgroundcolor=NavyBlue!1,
    },
]{thmexplanationbox}

\declaretheorem[
    style=thmbox, 
    % numberwithin = section,
    numbered = no,
    name=Definition
    ]{definition}

\declaretheorem[
    style=thmbox, 
    name=Example,
    ]{eg}

\declaretheorem[
    style=thmbox, 
    % numberwithin = section,
    name=Proposition]{prop}

\declaretheorem[
    style = thmbox,
    numbered=yes,
    name =Problem
    ]{problem}

\declaretheorem[style=thmbox, name=Theorem]{theorem}
\declaretheorem[style=thmbox, name=Lemma]{lemma}
\declaretheorem[style=thmbox, name=Corollary]{corollary}

\declaretheorem[style=thmproofbox, name=Proof]{replacementproof}

\declaretheorem[style=thmproofbox, 
                name = Solution
                ]{replacementsolution}

\renewenvironment{proof}[1][\proofname]{\vspace{-1pt}\begin{replacementproof}}{\end{replacementproof}}

\newenvironment{solution}
    {
        \vspace{-1pt}\begin{replacementsolution}
    }
    { 
            \end{replacementsolution}
    }

\declaretheorem[style=thmexplanationbox, name=Proof]{tmpexplanation}
\newenvironment{explanation}[1][]{\vspace{-10pt}\begin{tmpexplanation}}{\end{tmpexplanation}}

\declaretheorem[style=thmbox, numbered=no, name=Remark]{remark}
\declaretheorem[style=thmbox, numbered=no, name=Note]{note}

\newtheorem*{uovt}{UOVT}
\newtheorem*{notation}{Notation}
\newtheorem*{previouslyseen}{As previously seen}
% \newtheorem*{problem}{Problem}
\newtheorem*{observe}{Observe}
\newtheorem*{property}{Property}
\newtheorem*{intuition}{Intuition}

\usepackage{etoolbox}
\AtEndEnvironment{vb}{\null\hfill$\diamond$}%
\AtEndEnvironment{intermezzo}{\null\hfill$\diamond$}%
% \AtEndEnvironment{opmerking}{\null\hfill$\diamond$}%

% http://tex.stackexchange.com/questions/22119/how-can-i-change-the-spacing-before-theorems-with-amsthm
\makeatletter
% \def\thm@space@setup{%
%   \thm@preskip=\parskip \thm@postskip=0pt
% }
\newcommand{\oefening}[1]{%
    \def\@oefening{#1}%
    \subsection*{Oefening #1}
}

\newcommand{\suboefening}[1]{%
    \subsubsection*{Oefening \@oefening.#1}
}

\newcommand{\exercise}[1]{%
    \def\@exercise{#1}%
    \subsection*{Exercise #1}
}

\newcommand{\subexercise}[1]{%
    \subsubsection*{Exercise \@exercise.#1}
}


\usepackage{xifthen}

\def\testdateparts#1{\dateparts#1\relax}
\def\dateparts#1 #2 #3 #4 #5\relax{
    \marginpar{\small\textsf{\mbox{#1 #2 #3 #5}}}
}

\def\@lesson{}%
\newcommand{\lesson}[3]{
    \ifthenelse{\isempty{#3}}{%
        \def\@lesson{Lecture #1}%
    }{%
        \def\@lesson{Lecture #1: #3}%
    }%
    \subsection*{\@lesson}
    \testdateparts{#2}
}

% \renewcommand\date[1]{\marginpar{#1}}


% fancy headers
\usepackage{fancyhdr}
\pagestyle{fancy}

\makeatother

% notes
\usepackage{todonotes}
\usepackage{tcolorbox}

\tcbuselibrary{breakable}
\newenvironment{verbetering}{\begin{tcolorbox}[
    arc=0mm,
    colback=white,
    colframe=green!60!black,
    title=Opmerking,
    fonttitle=\sffamily,
    breakable
]}{\end{tcolorbox}}

\newenvironment{noot}[1]{\begin{tcolorbox}[
    arc=0mm,
    colback=white,
    colframe=white!60!black,
    title=#1,
    fonttitle=\sffamily,
    breakable
]}{\end{tcolorbox}}

% figure support
\usepackage{import}
\usepackage{xifthen}
\pdfminorversion=7
\usepackage{pdfpages}
\usepackage{transparent}
\newcommand{\incfig}[1]{%
    \def\svgwidth{\columnwidth}
    \import{./figures/}{#1.pdf_tex}
}

% %http://tex.stackexchange.com/questions/76273/multiple-pdfs-with-page-group-included-in-a-single-page-warning
\pdfsuppresswarningpagegroup=1



\begin{document}

\section{Handout-13}

\subsection{Topics}

\begin{itemize}
    \item Discuss more consequences of Cauchy-Riemann equations.
    \item Introduce holomorphic functions
\end{itemize}

\subsection{Recap}

Let \( D \subseteq  \C  \), let \( f: D \to \C  \) defined by \( f(z) = u(z) + i v(z) \). Assume that the partial derivatives of \( u  \) and \( v  \) are continuous. Then the following statements are equivalent.
\begin{enumerate}
    \item[(i)] \( f  \) is complex differentiable on \( D  \).
    \item[(ii)] The Cauchy-Riemann equations hold:
        \[  \frac{\partial u }{\partial x } = \frac{\partial v }{\partial y }  \ \ , \ \ \frac{\partial u }{\partial y }= - \frac{\partial v }{\partial x } \ \ \text{on \( D  \)}. \]
\end{enumerate}

\subsection{Laplace's Equation and Harmonic Functions}

\begin{definition}[ ]
    Let \( D \subseteq  \C   \) be open and let \( f: D \to \C  \). If \( f  \) is complex differentiable on \( D  \), then we say \( f  \) is holomorphic on \( D  \). Let \( a \in D  \). We say \( f  \) is holomorphic at \( a  \) if we can find an open set \( D' \subseteq D  \) such that \( a \in D' \) and \( f  \) is holomorphic on \( D' \).
\end{definition}

\begin{lemma}
    Let \( D \subseteq \C   \) be open, let \( f: D \to \C  \). Then the following statements are equivalent 
    \begin{enumerate}
        \item[(i)] \( f  \) is holomorphic on \( D  \)
        \item[(ii)] \( f  \) is holomorphic at \( a \in D  \) for all \( a \in D  \).
    \end{enumerate}
\end{lemma}

Let \( D \subseteq  \C  \), let \( D  \) be an open set, and let \( f: D \to \C  \) be holomorphic. Let \( f = u + iv \). In addition, assume that \( u \) and \( v  \) have second order continuous partial derivatives. By Cauchy-Riemann equation, we have
\[  \frac{\partial u }{\partial x }  = \frac{\partial v }{\partial y } , \ \ \frac{\partial u }{\partial y }  = - \frac{\partial v }{\partial x }  \ \ \text{on \( D  \)}. \]
Therefore, we have 
\[  \frac{\partial ^{2} u  }{\partial x^{2} }  = \frac{\partial ^{2} v  }{\partial x \partial y }  \ \ \text{and} \ \ \frac{\partial ^{2} u  }{\partial  y^{2} }  = - \frac{\partial ^{2}v  }{\partial y \partial x }. \]

Since we assumed, the second partial derivatives are continuous, we have 
\[  \frac{\partial ^{2} u  }{\partial x^{2} }  = - \frac{\partial ^{2} u  }{\partial  y^{2} }  \implies \frac{\partial ^{2} u  }{\partial  x^{2} }  + \frac{\partial ^{2} u  }{\partial  y^{2} }  = 0.  \]

Thus, we have proved the following proposition.

\begin{prop}[Laplace's Equation]
    Let \( D \subseteq  \C  \) be an open set, let \( f: D \to \C  \) be a holomorphic function and let \( f = u + i v \). In addition, assume that \( u  \) and \( v  \) have second order continuous partial derivatives. Then
    \begin{align*}
        \frac{\partial ^{2} u   }{\partial x^{2}  }  + \frac{\partial ^{2} u  }{\partial  y^{2} }  = 0, \\
        \frac{\partial ^{2} v  }{\partial x^{2}  }  + \frac{\partial ^{2} v  }{\partial  y^{2} }  = 0. 
    \end{align*}
\end{prop}

\begin{definition}[Harmonic Function]
    Let \( D \subseteq  \R^{2} \) be open. A function \( u: D \to \R  \) is called \textbf{harmonic} if \[ \frac{\partial ^{2} u  }{\partial x^{2} }  + \frac{\partial ^{2} u  }{\partial y^{2} }   = 0  \] on \( D  \).
\end{definition}

Thus, we learned that if \( f: D \to \C  \) with \( D  \) being an open set, \( f  \) being holomorphic, \( \Re(f) \) and \( \Im(f) \) have continuous partial order partial derivatives, then \( \Re(f) \) and \( \Im(f) \) are harmonic functions on \( D  \). From the Cauchy-Riemann equations, we see that \( u  \) and \( v  \) are heavily dependent on each other. This begs the following two questions:
\begin{enumerate}
    \item[(1)] Can we determine \( v  \) from \( u \)?
    \item[(2)] Suppose that \( u : D \to \R  \) that is harmonic. Is it possible to find a holomorphic \( f  \) such that \( u = \Re(f) \)?
\end{enumerate}

It turns out that the answers to these questions depends on the topology of \( D  \). But first we recall some basic topological facts in order to answer these questions.

\subsection{Basic Topological Facts}

\begin{definition}[ ]
    Let \( D \subseteq  \C   \) be an open set. Let \( z ,w \in D  \). A path in \( D  \) joining \( z  \) to \( w  \) is a continuous map \(  \gamma : [a,b] \to D  \) such that \( \gamma(a) = z  \) and \( \gamma(b) = w  \), where \( [a,b] \) is a closed interval in \( \R  \).
\end{definition}

\begin{itemize}
    \item Let \( D \subseteq  \C   \) be an open set. Let \( z,w \in D  \). Then \( z  \) and \( w  \) can be joined by a line segment if \( \gamma:[0,1] \to D  \) is given by \( \gamma(t) = (1-t)z + tw \).
    \item An \textbf{open set} \( D \subseteq  \C   \) is connected if any two points \( z,w \in D  \) can be joined by a sequence of line segments; that is, we can find points \( {z}_{1}, \dots, {z}_{k} \) such that \( z  \) and \( {z}_{1} \) can be joined by a line segment, \( {z}_{i} \) and \( {z}_{i+1} \) can be joined by a line segment for \( i = 1,2, \dots , k - 1  \) and \( {z}_{k} \) and \( w  \) can be joined by a line segment.
\end{itemize}

\begin{eg}
    \begin{itemize}
        \item \(  B (a,R) \) is connected.
        \item The annulus \( \{ z \in \C : {r}_{1} < | z | < {r}_{2} \}  \) where \( {r}_{1}, {r}_{2} > 0  \) such that \( {r}_{1} < {r}_{2} \) is connected. 
        \item \( D = B(0,1) \cup B(5,2) \) is not connected as they are disjoint.
    \end{itemize}
\end{eg}

\begin{remark}
    In topology, one uses a more general version of connectedness. Our definition of connectedness is specific to open subsets of \( \C  \) (or for any set in \( \R^{2}  \) rather).
\end{remark}

One consequence of connectedness is outlined in the proposition below:

\begin{prop}
   Let \( D \subseteq \C   \) be an open set and let \( f: D \to \C  \) be a holomorphic function. Suppose that \( f  \) is locally constant on \( D  \). If \( D  \) is connected, then \( f  \) is constant.
\end{prop}

\begin{theorem}[ ]
    Let \( D \subseteq  \C  \) be an open rectangle, whose sides parallel to the real and imaginary axes. Let \( u : D \to \R \) be a harmonic function. Then, we can find \( v: D \to \R  \) such that \( f: u + iv \) is holomorphic on \( D  \).
\end{theorem}








\end{document}

