\documentclass[a4paper]{report}
\usepackage{standalone}
\usepackage{import}

\usepackage[utf8]{inputenc}
\usepackage[T1]{fontenc}
\usepackage{textcomp}
\usepackage{hyperref}
% \usepackage{fourier}
% \usepackage[dutch]{babel}
\usepackage{url}
% \usepackage{hyperref}
% \hypersetup{
%     colorlinks,
%     linkcolor={black},
%     citecolor={black},
%     urlcolor={blue!80!black}
% }
\usepackage{graphicx}
\usepackage{float}
\usepackage{booktabs}
\usepackage{enumitem}
% \usepackage{parskip}
\usepackage{emptypage}
\usepackage{subcaption}
\usepackage{multicol}
\usepackage[usenames,dvipsnames]{xcolor}

% \usepackage{cmbright}


\usepackage[margin=1in]{geometry}
\usepackage{amsmath, amsfonts, mathtools, amsthm, amssymb}
\usepackage{mathrsfs}
\usepackage{cancel}
\usepackage{bm}
\newcommand\N{\ensuremath{\mathbb{N}}}
\newcommand\R{\ensuremath{\mathbb{R}}}
\newcommand\Z{\ensuremath{\mathbb{Z}}}
\renewcommand\O{\ensuremath{\emptyset}}
\newcommand\Q{\ensuremath{\mathbb{Q}}}
\newcommand\C{\ensuremath{\mathbb{C}}}
\DeclareMathOperator{\sgn}{sgn}
\usepackage{systeme}
\let\svlim\lim\def\lim{\svlim\limits}
\let\implies\Rightarrow
\let\impliedby\Leftarrow
\let\iff\Leftrightarrow
\let\epsilon\varepsilon
\usepackage{stmaryrd} % for \lightning
\newcommand\contra{\scalebox{1.1}{$\lightning$}}
% \let\phi\varphi
\renewcommand\qedsymbol{$\blacksquare$}




% correct
\definecolor{correct}{HTML}{009900}
\newcommand\correct[2]{\ensuremath{\:}{\color{red}{#1}}\ensuremath{\to }{\color{correct}{#2}}\ensuremath{\:}}
\newcommand\green[1]{{\color{correct}{#1}}}



% horizontal rule
\newcommand\hr{
    \noindent\rule[0.5ex]{\linewidth}{0.5pt}
}


% hide parts
\newcommand\hide[1]{}



% si unitx
\usepackage{siunitx}
\sisetup{locale = FR}
% \renewcommand\vec[1]{\mathbf{#1}}
\newcommand\mat[1]{\mathbf{#1}}


% tikz
\usepackage{tikz}
\usepackage{tikz-cd}
\usetikzlibrary{intersections, angles, quotes, calc, positioning}
\usetikzlibrary{arrows.meta}
\usepackage{pgfplots}
\pgfplotsset{compat=1.13}


\tikzset{
    force/.style={thick, {Circle[length=2pt]}-stealth, shorten <=-1pt}
}

% theorems
\makeatother
\usepackage{thmtools}
\usepackage[framemethod=TikZ]{mdframed}
\mdfsetup{skipabove=1em,skipbelow=0em}


\theoremstyle{definition}

\declaretheoremstyle[
    headfont=\bfseries\sffamily\color{ForestGreen!70!black}, bodyfont=\normalfont,
    mdframed={
        linewidth=2pt,
        rightline=false, topline=false, bottomline=false,
        linecolor=ForestGreen, backgroundcolor=ForestGreen!5,
    }
]{thmgreenbox}

\declaretheoremstyle[
    headfont=\bfseries\sffamily\color{NavyBlue!70!black}, bodyfont=\normalfont,
    mdframed={
        linewidth=2pt,
        rightline=false, topline=false, bottomline=false,
        linecolor=NavyBlue, backgroundcolor=NavyBlue!5,
    }
]{thmbluebox}

\declaretheoremstyle[
    headfont=\bfseries\sffamily\color{NavyBlue!70!black}, bodyfont=\normalfont,
    mdframed={
        linewidth=2pt,
        rightline=false, topline=false, bottomline=false,
        linecolor=NavyBlue
    }
]{thmblueline}

\declaretheoremstyle[
    headfont=\bfseries\sffamily\color{RawSienna!70!black}, bodyfont=\normalfont,
    mdframed={
        linewidth=2pt,
        rightline=false, topline=false, bottomline=false,
        linecolor=RawSienna, backgroundcolor=RawSienna!5,
    }
]{thmredbox}

\declaretheoremstyle[
    headfont=\bfseries\sffamily\color{RawSienna!70!black}, bodyfont=\normalfont,
    numbered=no,
    mdframed={
        linewidth=2pt,
        rightline=false, topline=false, bottomline=false,
        linecolor=RawSienna, backgroundcolor=RawSienna!1,
    },
    qed=\qedsymbol
]{thmproofbox}

\declaretheoremstyle[
    headfont=\bfseries\sffamily\color{NavyBlue!70!black}, bodyfont=\normalfont,
    numbered=no,
    mdframed={
        linewidth=2pt,
        rightline=false, topline=false, bottomline=false,
        linecolor=NavyBlue, backgroundcolor=NavyBlue!1,
    },
]{thmexplanationbox}

\declaretheorem[style=thmgreenbox, numberwithin = section, name=Definition]{definition}
\declaretheorem[style=thmbluebox, name=Example]{eg}
\declaretheorem[style=thmredbox, numberwithin = section, name=Proposition]{prop}
\declaretheorem[style=thmredbox, numberwithin = section, name=Theorem]{theorem}
\declaretheorem[style=thmredbox, numberwithin = section,  name=Lemma]{lemma}
\declaretheorem[style=thmredbox, numberwithin = section,  numbered=no, name=Corollary]{corollary}


\declaretheorem[style=thmproofbox, name=Proof]{replacementproof}
\renewenvironment{proof}[1][\proofname]{\vspace{-10pt}\begin{replacementproof}}{\end{replacementproof}}


\declaretheorem[style=thmexplanationbox, name=Proof]{tmpexplanation}
\newenvironment{explanation}[1][]{\vspace{-10pt}\begin{tmpexplanation}}{\end{tmpexplanation}}


\declaretheorem[style=thmblueline, numbered=no, name=Remark]{remark}
\declaretheorem[style=thmblueline, numbered=no, name=Note]{note}

\newtheorem*{uovt}{UOVT}
\newtheorem*{notation}{Notation}
\newtheorem*{previouslyseen}{As previously seen}
\newtheorem*{problem}{Problem}
\newtheorem*{observe}{Observe}
\newtheorem*{property}{Property}
\newtheorem*{intuition}{Intuition}


\usepackage{etoolbox}
\AtEndEnvironment{vb}{\null\hfill$\diamond$}%
\AtEndEnvironment{intermezzo}{\null\hfill$\diamond$}%
% \AtEndEnvironment{opmerking}{\null\hfill$\diamond$}%

% http://tex.stackexchange.com/questions/22119/how-can-i-change-the-spacing-before-theorems-with-amsthm
\makeatletter
% \def\thm@space@setup{%
%   \thm@preskip=\parskip \thm@postskip=0pt
% }
\newcommand{\oefening}[1]{%
    \def\@oefening{#1}%
    \subsection*{Oefening #1}
}

\newcommand{\suboefening}[1]{%
    \subsubsection*{Oefening \@oefening.#1}
}

\newcommand{\exercise}[1]{%
    \def\@exercise{#1}%
    \subsection*{Exercise #1}
}

\newcommand{\subexercise}[1]{%
    \subsubsection*{Exercise \@exercise.#1}
}


\usepackage{xifthen}

\def\testdateparts#1{\dateparts#1\relax}
\def\dateparts#1 #2 #3 #4 #5\relax{
    \marginpar{\small\textsf{\mbox{#1 #2 #3 #5}}}
}

\def\@lesson{}%
\newcommand{\lesson}[3]{
    \ifthenelse{\isempty{#3}}{%
        \def\@lesson{Lecture #1}%
    }{%
        \def\@lesson{Lecture #1: #3}%
    }%
    \subsection*{\@lesson}
    \testdateparts{#2}
}

% \renewcommand\date[1]{\marginpar{#1}}


% fancy headers
\usepackage{fancyhdr}
\pagestyle{fancy}

\fancyhead[LE,RO]{Lance Remigio}
\fancyhead[RO,LE]{\@lesson}
\fancyhead[RE,LO]{}
\fancyfoot[LE,RO]{\thepage}
\fancyfoot[C]{\leftmark}

\makeatother




% notes
\usepackage{todonotes}
\usepackage{tcolorbox}

\tcbuselibrary{breakable}
\newenvironment{verbetering}{\begin{tcolorbox}[
    arc=0mm,
    colback=white,
    colframe=green!60!black,
    title=Opmerking,
    fonttitle=\sffamily,
    breakable
]}{\end{tcolorbox}}

\newenvironment{noot}[1]{\begin{tcolorbox}[
    arc=0mm,
    colback=white,
    colframe=white!60!black,
    title=#1,
    fonttitle=\sffamily,
    breakable
]}{\end{tcolorbox}}




% figure support
\usepackage{import}
\usepackage{xifthen}
\pdfminorversion=7
\usepackage{pdfpages}
\usepackage{transparent}
\newcommand{\incfig}[1]{%
    \def\svgwidth{\columnwidth}
    \import{./figures/}{#1.pdf_tex}
}

% %http://tex.stackexchange.com/questions/76273/multiple-pdfs-with-page-group-included-in-a-single-page-warning
\pdfsuppresswarningpagegroup=1





\begin{document}

\section{Lecture 3}

\subsection{Topics}

\begin{itemize}
    \item Polar Representation of complex numbers.
    \item Convergence of sequences in \( \C  \).
\end{itemize}

\subsection{Polar Representation of Complex Numbers}

If \( (x,y) \in \R^{2} \setminus \{ (0,0) \}  \), we can represent this as  
\[  (x,y) = (\gamma \cos \varphi, \gamma \sin \varphi) \]
where \( \gamma =\sqrt{ x^{2} + y^{2} }   \) and \( \tan(\varphi) = \frac{ y }{ x }  \)
is a polar representation of \( (x,y) \).
\begin{remark}
    This representation may not be unique!
\end{remark}

If we insist, we can make \( \varphi  \) unique by restricting the domain to \( -\pi < \varphi \leq \pi \) where \( \varphi  \) is denoted as the \textbf{argument of \( z  \)}.

\begin{definition}[Principle Argument]
    If \( - \pi < \varphi \leq \pi \), we call this angle the \textbf{principle argument} which we denote as 
    \[  \varphi = \Arg((x,y)). \]
\end{definition}
\begin{remark}
    For any other domain, we denote the argument by \( \varphi = \arg((x,y)) \).
\end{remark}

\begin{lemma}
    Let \( z = \gamma(\cos \varphi + i \sin \varphi) \) and \( w = \gamma' (\cos(\varphi') + \sin(\varphi') \) in \( \C \setminus  \{ 0  \}  \). Then
    \[  zw = \varphi \varphi' [\cos(\varphi + \varphi') + i \sin(\varphi + \varphi')]. \]
\end{lemma}
\begin{proof}
Using the addition formula, we can write
\begin{align*}
    zw &= \gamma \gamma'  (\cos \varphi + i \sin \varphi)(\cos \varphi' + i \sin \varphi')  \\
       &= \gamma \gamma' [ (\cos \varphi cos \varphi' + \sin \varphi \sin \varphi') + i(\sin \varphi \cos \varphi' + \sin \varphi \cos \varphi') ] \\
       &= \gamma \gamma' (\cos(\varphi + \varphi') + i \sin(\varphi + \varphi')).
\end{align*}
\end{proof}

\begin{corollary}[De Moivre's Theorem]
   Let \( z = \gamma (\cos \varphi + i \sin \varphi) \in \C \setminus  \{ 0 \}  \) and let \( n \in \Z   \). Then
   \[  z^{n} = \gamma^{n} (\cos n \varphi + i \sin n \varphi). \]
\end{corollary}

\begin{remark}
 If \( n  \) is a negative integer, then \( z^{n} = (z^{-1})^{-n} \). 
\end{remark}

The corollary above allows us to compute the \( n \)th roots of a non-zero complex number.

\begin{eg}[An example of De Moivre's Theorem]
   Suppose we had the complex number  
   \[  z = \frac{ 1 }{ 2 }  + i \frac{ \sqrt{ 3 }  }{ 2 }. \]
  Suppose we want to find \( z^{10} \). First, we need to find the angle that makes this complex number. Since the \( x  \) and \( y  \) coordinates are both positive this means that the angle must lie in the first quadrant. Thus, we have 
  \[  \varphi = \arg(z) = \frac{ \pi }{ 3  }. \]
  Using De Moivre's Theorem, we can write
  \begin{align*}
      z^{10 } &= \cos \Big(  10 \cdot \frac{ \pi  }{ 3 }  \Big) + i \sin \Big(  10 \cdot \frac{ \pi }{ 3 }  \Big) \\
              &= -\frac{1 }{ 2 }  - i \frac{ \sqrt{ 3 }  }{ 2 }. 
  \end{align*}
\end{eg}

Some notations we would like to establish are the following:
\begin{enumerate}
    \item[(i)] \textbf{The set of all positive real numbers} \( {\R}_{+} = \{ r \in \R : r > 0  \}  \)
    \item[(ii)] \textbf{The set of all complex numbers excluding zero} \( \C^{\cdot} =  \C \setminus \{ 0 \}  \).
\end{enumerate}

\begin{prop}
    The map \( {\R}_{+} \times \R \longrightarrow \C^{} \) defined by 
    \[  (r, \varphi) \longrightarrow \gamma ( \cos \varphi + i \sin \varphi) \]
    is surjective.
\end{prop}

\begin{remark}
    This gives us the tool we need to show that every non-zero \( z \in \C  \) has a polar representation.
\end{remark}

\subsection{Convergence of Sequences in \( \C \)}

\begin{definition}[Convergence in \( \C  \)]
    Let \( {\{ {z}_{n} \} }_{n=1}^{\infty }     \) be a sequence in \( \C  \). We say that \( \{ {z}_{n} \}   \) converges to \( z \in \C  \) if for all \(  \epsilon > 0  \), we can find \( {N}_{\epsilon} \in \N  \) such that  
    \[  | {z}_{n} - z  |  < \epsilon \]
    for all \( n \geq {N}_{\epsilon} \).
\end{definition}

If \( ({z}_{n}) \) converges to \( z  \), then we write \( {z}_{n} \to z  \).

\begin{prop}[Properties of Convergent Sequences]
   Assume \( ({z}_{n}) \to z  \) and \( ({w}_{n}) \to w   \). 
   \begin{enumerate}
       \item[(i)] Let \( \alpha, \beta \in \C  \), then \( \alpha {z}_{n} + \beta {w}_{n} \to \alpha z + \beta w  \).
        \item[(ii)] \( {z}_{n} {w}_{n} \to zw  \).
        \item[(iii)] \( {z}_{n}^{-1} \to z^{-1} \).
        \item[(iv)] \( ({z}_{n}) \to z  \) if and only if \( \Re({z}_{n}) \to \Re(z) \) and \( \Im({z}_{n}) \to \Im(z) \) as a sequences in \( \R  \).
   \end{enumerate}
\end{prop}
\begin{proof}

\end{proof}

\end{document}
