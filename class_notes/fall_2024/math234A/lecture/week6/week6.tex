\documentclass[a4paper]{report}
\usepackage{standalone}
\usepackage{import}

\usepackage[utf8]{inputenc}
\usepackage[T1]{fontenc}
\usepackage{textcomp}
\usepackage{hyperref}
% \usepackage{fourier}
% \usepackage[dutch]{babel}
\usepackage{url}
% \usepackage{hyperref}
% \hypersetup{
%     colorlinks,
%     linkcolor={black},
%     citecolor={black},
%     urlcolor={blue!80!black}
% }
\usepackage{graphicx}
\usepackage{float}
\usepackage{booktabs}
\usepackage{enumitem}
% \usepackage{parskip}
\usepackage{emptypage}
\usepackage{subcaption}
\usepackage{multicol}
\usepackage[usenames,dvipsnames]{xcolor}

% \usepackage{cmbright}


\usepackage[margin=1in]{geometry}
\usepackage{amsmath, amsfonts, mathtools, amsthm, amssymb}
\usepackage{mathrsfs}
\usepackage{cancel}
\usepackage{bm}
\newcommand\N{\ensuremath{\mathbb{N}}}
\newcommand\R{\ensuremath{\mathbb{R}}}
\newcommand\Z{\ensuremath{\mathbb{Z}}}
\renewcommand\O{\ensuremath{\emptyset}}
\newcommand\Q{\ensuremath{\mathbb{Q}}}
\newcommand\C{\ensuremath{\mathbb{C}}}
\DeclareMathOperator{\sgn}{sgn}
\usepackage{systeme}
\let\svlim\lim\def\lim{\svlim\limits}
\let\implies\Rightarrow
\let\impliedby\Leftarrow
\let\iff\Leftrightarrow
\let\epsilon\varepsilon
\usepackage{stmaryrd} % for \lightning
\newcommand\contra{\scalebox{1.1}{$\lightning$}}
% \let\phi\varphi
\renewcommand\qedsymbol{$\blacksquare$}




% correct
\definecolor{correct}{HTML}{009900}
\newcommand\correct[2]{\ensuremath{\:}{\color{red}{#1}}\ensuremath{\to }{\color{correct}{#2}}\ensuremath{\:}}
\newcommand\green[1]{{\color{correct}{#1}}}



% horizontal rule
\newcommand\hr{
    \noindent\rule[0.5ex]{\linewidth}{0.5pt}
}


% hide parts
\newcommand\hide[1]{}



% si unitx
\usepackage{siunitx}
\sisetup{locale = FR}
% \renewcommand\vec[1]{\mathbf{#1}}
\newcommand\mat[1]{\mathbf{#1}}


% tikz
\usepackage{tikz}
\usepackage{tikz-cd}
\usetikzlibrary{intersections, angles, quotes, calc, positioning}
\usetikzlibrary{arrows.meta}
\usepackage{pgfplots}
\pgfplotsset{compat=1.13}


\tikzset{
    force/.style={thick, {Circle[length=2pt]}-stealth, shorten <=-1pt}
}

% theorems
\makeatother
\usepackage{thmtools}
\usepackage[framemethod=TikZ]{mdframed}
\mdfsetup{skipabove=1em,skipbelow=0em}


\theoremstyle{definition}

\declaretheoremstyle[
    headfont=\bfseries\sffamily\color{ForestGreen!70!black}, bodyfont=\normalfont,
    mdframed={
        linewidth=2pt,
        rightline=false, topline=false, bottomline=false,
        linecolor=ForestGreen, backgroundcolor=ForestGreen!5,
    }
]{thmgreenbox}

\declaretheoremstyle[
    headfont=\bfseries\sffamily\color{NavyBlue!70!black}, bodyfont=\normalfont,
    mdframed={
        linewidth=2pt,
        rightline=false, topline=false, bottomline=false,
        linecolor=NavyBlue, backgroundcolor=NavyBlue!5,
    }
]{thmbluebox}

\declaretheoremstyle[
    headfont=\bfseries\sffamily\color{NavyBlue!70!black}, bodyfont=\normalfont,
    mdframed={
        linewidth=2pt,
        rightline=false, topline=false, bottomline=false,
        linecolor=NavyBlue
    }
]{thmblueline}

\declaretheoremstyle[
    headfont=\bfseries\sffamily\color{RawSienna!70!black}, bodyfont=\normalfont,
    mdframed={
        linewidth=2pt,
        rightline=false, topline=false, bottomline=false,
        linecolor=RawSienna, backgroundcolor=RawSienna!5,
    }
]{thmredbox}

\declaretheoremstyle[
    headfont=\bfseries\sffamily\color{RawSienna!70!black}, bodyfont=\normalfont,
    numbered=no,
    mdframed={
        linewidth=2pt,
        rightline=false, topline=false, bottomline=false,
        linecolor=RawSienna, backgroundcolor=RawSienna!1,
    },
    qed=\qedsymbol
]{thmproofbox}

\declaretheoremstyle[
    headfont=\bfseries\sffamily\color{NavyBlue!70!black}, bodyfont=\normalfont,
    numbered=no,
    mdframed={
        linewidth=2pt,
        rightline=false, topline=false, bottomline=false,
        linecolor=NavyBlue, backgroundcolor=NavyBlue!1,
    },
]{thmexplanationbox}

\declaretheorem[style=thmgreenbox, numberwithin = section, name=Definition]{definition}
\declaretheorem[style=thmbluebox, name=Example]{eg}
\declaretheorem[style=thmredbox, numberwithin = section, name=Proposition]{prop}
\declaretheorem[style=thmredbox, numberwithin = section, name=Theorem]{theorem}
\declaretheorem[style=thmredbox, numberwithin = section,  name=Lemma]{lemma}
\declaretheorem[style=thmredbox, numberwithin = section,  numbered=no, name=Corollary]{corollary}


\declaretheorem[style=thmproofbox, name=Proof]{replacementproof}
\renewenvironment{proof}[1][\proofname]{\vspace{-10pt}\begin{replacementproof}}{\end{replacementproof}}


\declaretheorem[style=thmexplanationbox, name=Proof]{tmpexplanation}
\newenvironment{explanation}[1][]{\vspace{-10pt}\begin{tmpexplanation}}{\end{tmpexplanation}}


\declaretheorem[style=thmblueline, numbered=no, name=Remark]{remark}
\declaretheorem[style=thmblueline, numbered=no, name=Note]{note}

\newtheorem*{uovt}{UOVT}
\newtheorem*{notation}{Notation}
\newtheorem*{previouslyseen}{As previously seen}
\newtheorem*{problem}{Problem}
\newtheorem*{observe}{Observe}
\newtheorem*{property}{Property}
\newtheorem*{intuition}{Intuition}


\usepackage{etoolbox}
\AtEndEnvironment{vb}{\null\hfill$\diamond$}%
\AtEndEnvironment{intermezzo}{\null\hfill$\diamond$}%
% \AtEndEnvironment{opmerking}{\null\hfill$\diamond$}%

% http://tex.stackexchange.com/questions/22119/how-can-i-change-the-spacing-before-theorems-with-amsthm
\makeatletter
% \def\thm@space@setup{%
%   \thm@preskip=\parskip \thm@postskip=0pt
% }
\newcommand{\oefening}[1]{%
    \def\@oefening{#1}%
    \subsection*{Oefening #1}
}

\newcommand{\suboefening}[1]{%
    \subsubsection*{Oefening \@oefening.#1}
}

\newcommand{\exercise}[1]{%
    \def\@exercise{#1}%
    \subsection*{Exercise #1}
}

\newcommand{\subexercise}[1]{%
    \subsubsection*{Exercise \@exercise.#1}
}


\usepackage{xifthen}

\def\testdateparts#1{\dateparts#1\relax}
\def\dateparts#1 #2 #3 #4 #5\relax{
    \marginpar{\small\textsf{\mbox{#1 #2 #3 #5}}}
}

\def\@lesson{}%
\newcommand{\lesson}[3]{
    \ifthenelse{\isempty{#3}}{%
        \def\@lesson{Lecture #1}%
    }{%
        \def\@lesson{Lecture #1: #3}%
    }%
    \subsection*{\@lesson}
    \testdateparts{#2}
}

% \renewcommand\date[1]{\marginpar{#1}}


% fancy headers
\usepackage{fancyhdr}
\pagestyle{fancy}

\fancyhead[LE,RO]{Lance Remigio}
\fancyhead[RO,LE]{\@lesson}
\fancyhead[RE,LO]{}
\fancyfoot[LE,RO]{\thepage}
\fancyfoot[C]{\leftmark}

\makeatother




% notes
\usepackage{todonotes}
\usepackage{tcolorbox}

\tcbuselibrary{breakable}
\newenvironment{verbetering}{\begin{tcolorbox}[
    arc=0mm,
    colback=white,
    colframe=green!60!black,
    title=Opmerking,
    fonttitle=\sffamily,
    breakable
]}{\end{tcolorbox}}

\newenvironment{noot}[1]{\begin{tcolorbox}[
    arc=0mm,
    colback=white,
    colframe=white!60!black,
    title=#1,
    fonttitle=\sffamily,
    breakable
]}{\end{tcolorbox}}




% figure support
\usepackage{import}
\usepackage{xifthen}
\pdfminorversion=7
\usepackage{pdfpages}
\usepackage{transparent}
\newcommand{\incfig}[1]{%
    \def\svgwidth{\columnwidth}
    \import{./figures/}{#1.pdf_tex}
}

% %http://tex.stackexchange.com/questions/76273/multiple-pdfs-with-page-group-included-in-a-single-page-warning
\pdfsuppresswarningpagegroup=1




\begin{document}

\section{Lecture 7}

\subsection{Topics}

Our goal for this lecture is to discuss complex powers of a non-zero complex number.

\subsection{Complex Powers}

Let \( z \in \C^{\bullet} \). We want to define \( z^{w} \) for \( w \in \C  \). Recall from real number system that, for \( x \in {\R}_{+} \), \( x^{a} = e^{a \ln x } \). We want our complex powers to extend this idea.

\begin{definition}[ ]
   We define \( z^{w} = \exp(w \log z) \). Note that  
   \begin{align*}
       \log z &= \Log z + 2 \pi i k  \\
              &= \ln | z  |  + i (\Arg(z) + 2\pi k), \ k \in \Z 
   \end{align*}
   is a multi-valued function.
\end{definition}

For this reason, the complex power of a non-zero complex number will also be multi-valued.
Hence, by choosing a fixed \( k  \), we can find a specific value of \( z^{w} \).

\begin{eg}
    \begin{enumerate}
        \item[(i)] Compute \( i^{i} \). Recall that \( \log i = i \Big(  \frac{ \pi  }{  2  }  + 2 \pi k  \Big) \). Hence, we have 
            \[  i^{i} = \exp(i \log i) = \{ e^{\frac{ -\pi }{ 2 }  + 2 \pi k }: k \in \Z  \}. \]
            If we are just referring to the principal branch, then 
            \[ i^{i} = e^{\frac{ -\pi }{ 2 } }.  \]
        \item[(ii)] Compute: \( (-1)^{1 + i} \).
            Note that 
            \[  \log(-1) = i \pi + 2 \pi i k  =  i (2k+1) \pi. \]
            Then we see that for all \( k \in \Z  \), we have
            \begin{align*}
                (-1)^{1 + i} &= \exp((2k+1)\pi (i-1)) \\
                             &= e^{-(2k+1)\pi} \exp(e^{(2k+1) \pi i }) \\
                             &= e^{-(2k+1)\pi} \Big[\cos (2k+1) \pi + i \sin (2k+1) \pi \Big] \\
                             &= - e^{-(2k+1)\pi}.
            \end{align*}
    \end{enumerate}
\end{eg}

Let us summarize some facts about \( \exp(z) \).

\begin{itemize}
    \item For all \( x \in \R  \), we have \( \exp(z) = e^{x} \). Thus, from now on, we write \( e^{z} \) in place of \( \exp(z) \).
    \item For \( z \in \C^{\bullet} \), we have \( \log z  \) is a multi-valued function. That is, we have 
        \[  \log z = \ln | z  |  + i (\Arg z + 2 \pi k ), \ k \in \Z.  \]
    \item For \( z \in \C^{\bullet} \), we have \( \Log z = \ln | z  |  + i \Arg z  \) (this is the principal branch log).
    \item For \( z \in \C^{\bullet} \), we have \( z^{w } = e^{w \log z} \) is also a multi-valued function.
\end{itemize}

\begin{eg}
    \begin{itemize}
        \item Find all solutions of \( \sin z = 1  \).
        \item Find all possible solutions of \( \cos z = i  \).
    \end{itemize}
\end{eg}

\begin{problem}
     Find all solutions of \( e^{z} = i  \).
\end{problem}
\begin{solution}

\end{solution}

\section{Lecture 8}

\begin{itemize}
    \item Limits of a function
    \item Continuity
    \item Derivative
\end{itemize}

\subsection{Functional Limit}\label{Functional Limit}

\begin{definition}[Functional Limit]
    Let \( D \subseteq \C   \) and \( f: D \to \C  \) be a function. Let \( a \) be an accumulation point of \( D  \). Let \( \ell \in \C  \). We say that \( \ell  \) is \textbf{the limit of \( f(z) \)} as \( z  \) approaches to \( a  \) if for all \( \epsilon > 0 \), there is \( \delta> 0  \) such that for all \( z \in D  \) with \( 0 < |  z - a  |  < \delta \), we have
    \[  | f(z) - \ell | < \epsilon; \]
    that is, \( z \in D \cap (B(a,\delta) \setminus  \{ a \} ) \) implies \( f(z) \in B(\ell, \epsilon) \).
\end{definition}

\begin{eg}
    Let \( f: B(0,1) \to \C  \) defined by \( f(z) = \frac{ iz }{ 2  }  \) then \( \lim_{ z \to i }  f(z) = - \frac{ 1 }{ 2 }  \).
\end{eg}
\begin{proof}
Observe that 
\[  | f(z) - \ell | = \Big| \frac{ i }{ 2 } z - \Big(  - \frac{ 1 }{ 2 }  \Big) \Big| = \Big| \frac{ i }{ 2 }  \Big|  | z - i  |  = \frac{ 1 }{ 2 }  | z - i  |.  \]
So, given any \( \epsilon > 0  \), we can take any \( 0 < \delta< 2 \epsilon  \) such that 
\[  0 < | z - i  | < \delta  \] implies 
\[  | f(z) - \ell | < \epsilon. \]
\end{proof}

Some facts about limits are:

\begin{theorem}[ ]
    Let \( D \subseteq \C  \) with \( f: D \to \C  \), \( g: D \to \C  \) be two functions let us write \( f = u + i v \).
    \begin{enumerate}
        \item[(i)] Let \( a = \alpha + i \beta \). Then \( \lim_{ z \to a  } f(x) = \ell \) if and only if 
        \begin{center}
            \( \lim_{ (x,y) \to (\alpha, \beta) } u(x,y) = \Re(\ell)  \) and \( \lim_{ (x,y) \to (\alpha, \beta)  } v(x,y) = \Im(\ell) \); that is,
            \[  \lim_{ z \to a }  f(z) = \lim_{ z \to a }  u(z) + i \lim_{ z \to a } v(z). \] 
        \end{center}
    \item[(ii)] Let \( \lim_{ z \to  a  }  f(z) = \ell  \) and \( \lim_{ z \to a } g(z) = \ell' \) and \( c,d \in \C  \). Then
        \[  \lim_{ z \to a }  (cf(z) + dg(z)) = c \ell + d \ell'.  \]
    \item[(iii)] \( \lim_{ z \to a } (f(z)g(z)) = \lim_{ z \to a } f(z) \cdot \lim_{ z \to a }  g(z) \).
    \item[(iv)] We have
        \[  \lim_{ z \to a }  \frac{ f(z) }{ g(z) }  = \frac{ \lim_{ z \to a }  f(z) }{ \lim_{ z \to a } g(z) }  \]
        whenever \( \lim_{ z \to a } g(z) \neq 0  \).
    \end{enumerate}
\end{theorem}

\begin{problem}
    Let \( \lim_{ z \to i } f(z) = 2 + i    \) and \( \lim_{ z \to i }  g(z) = 1 - i  \). Compute the following:
    \[ \lim_{ z \to i }  \Big[ (f(z))^{3} + \frac{ (1+i) g(z) }{ z^{2} } \Big]. \]
\end{problem} 
\begin{solution}
Observe that 
\begin{align*}
  \lim_{ z \to i }  \Big[(f(z))^{3} + \frac{ (1+i) g(z) }{ z^{2} } \Big]  &= \lim_{ z \to i }  (f(z))^{3} + \lim_{ z \to i }  \frac{ (1+i) g(z)  }{ z^{2} }  \\
                                                                          &= (2+i)^{3} + \frac{ (1+i) (1-i) }{ i^{2} } \\
                                                                          &= (2+i)^{3} -2. 
\end{align*}
\end{solution}

\begin{definition}[Limits at Infinity]
    \begin{enumerate}
        \item[(i)] Let \( f: D \to \C  \) be a function and \( a \in \C  \) be an accumulation point of \( D  \). Then we say \( \lim_{ z \to a } f(z) = \infty   \) if for all \( M > 0  \), there exists \( \delta > 0  \) such that
            \[  z \in D \cap (B(a,\delta) \setminus  \{ a \} ) \]
            implies \( | f(z) | \geq  M \); that is, \( f  \) is unbounded as \( z  \) approaches to \( a  \).
        \item[(ii)] Let \( f  \) be a complex function defined on the complement of a ball in \( \C  \). We say \( \lim_{ z \to \infty  } f(z) = \ell  \) if for all \( \epsilon > 0  \), there exists \( R > 0  \) such that \( | z  |  > R  \) implies 
            \[  | f(z) - \ell  | < \epsilon. \]
    \end{enumerate}
\end{definition}

\begin{eg}
    \begin{enumerate}
        \item[(i)] Consider \( \lim_{ z \to 0 }  \frac{ 1 }{ z }  = \infty   \). Let \( M > 0  \). Let \( \delta = \frac{ 1 }{ M }  \). Then \( 0 < | z  |  < \delta \) implies that 
            \[  \Big| \frac{ 1 }{ z }  \Big| > \frac{ 1 }{ \delta } = M. \]
        \item[(ii)] Consider \( \lim_{ z \to \infty  }   \frac{ 1 }{ z }  = 0.  \)
            Let \( \epsilon > 0  \). Choose \( R = \frac{ 1 }{ \epsilon } > 0  \). Then 
            \(  | z  |  > R  \) implies 
            \[  \Big| \frac{ 1 }{ z }  \Big| < \frac{ 1 }{ R } = \epsilon. \]
    \end{enumerate}
\end{eg}

\begin{problem}
    \begin{enumerate}
        \item[(i)] Show that \( \lim_{ z \to \infty  } f(z) = \ell  \) if and only if
            \[  \lim_{ z \to \infty  } f \Big(  \frac{ 1 }{ z }  \Big) = \ell.  \]
        \item[(ii)] \( \lim_{ z \to a } f(z) = \infty   \) if and only if \( \lim_{ z \to a }  \frac{ 1 }{ f(z) } = 0  \).
        \item[(iii)] Give a definition of \( \lim_{ z \to \infty  } f(z) = \infty   \). Show that \( \lim_{ z \to \infty  } f(z) = \infty  \) if and only if \( \lim_{ z \to 0 } f \Big(  \frac{ 1 }{ z }  \Big) = \infty  \).
        \item[(iv)] Compute \( \lim_{ z \to \infty  }  \frac{ z - 1  }{ z + i  }  \).
        \item[(v)] Compute \( \lim_{ z \to \infty    }  \frac{ gz + i  }{ z^{2} + z + 1  }  \). 
        \item[(vi)] Can you compute \( \lim_{ z \to \infty   }  e^{-z}  \)?
    \end{enumerate}
\end{problem}

\subsection{Continuity}

\begin{definition}[Continuity]
    Let \( D \subseteq \C   \) and \( f: D \to \C  \) be a function. We say \textbf{\( f  \) is continuous at \( a \in D  \)} if for all \( \epsilon > 0  \), there exists \( \delta > 0  \) such that for all \( z \in B(a,\delta) \cap D \), we have \( f(z) \in B(f(a), \epsilon) \).
\end{definition}


\end{document}
