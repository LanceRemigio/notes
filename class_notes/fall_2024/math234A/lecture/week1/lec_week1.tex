\documentclass[a4paper]{report}
\usepackage[utf8]{inputenc}
\usepackage[T1]{fontenc}
\usepackage{textcomp}
\usepackage{hyperref}
% \usepackage{fourier}
% \usepackage[dutch]{babel}
\usepackage{url}
% \usepackage{hyperref}
% \hypersetup{
%     colorlinks,
%     linkcolor={black},
%     citecolor={black},
%     urlcolor={blue!80!black}
% }
\usepackage{graphicx}
\usepackage{float}
\usepackage{booktabs}
\usepackage{enumitem}
% \usepackage{parskip}
\usepackage{emptypage}
\usepackage{subcaption}
\usepackage{multicol}
\usepackage[usenames,dvipsnames]{xcolor}

% \usepackage{cmbright}


\usepackage[margin=1in]{geometry}
\usepackage{amsmath, amsfonts, mathtools, amsthm, amssymb}
\usepackage{mathrsfs}
\usepackage{cancel}
\usepackage{bm}
\newcommand\N{\ensuremath{\mathbb{N}}}
\newcommand\R{\ensuremath{\mathbb{R}}}
\newcommand\Z{\ensuremath{\mathbb{Z}}}
\renewcommand\O{\ensuremath{\emptyset}}
\newcommand\Q{\ensuremath{\mathbb{Q}}}
\newcommand\C{\ensuremath{\mathbb{C}}}
\DeclareMathOperator{\sgn}{sgn}
\usepackage{systeme}
\let\svlim\lim\def\lim{\svlim\limits}
\let\implies\Rightarrow
\let\impliedby\Leftarrow
\let\iff\Leftrightarrow
\let\epsilon\varepsilon
\usepackage{stmaryrd} % for \lightning
\newcommand\contra{\scalebox{1.1}{$\lightning$}}
% \let\phi\varphi
\renewcommand\qedsymbol{$\blacksquare$}




% correct
\definecolor{correct}{HTML}{009900}
\newcommand\correct[2]{\ensuremath{\:}{\color{red}{#1}}\ensuremath{\to }{\color{correct}{#2}}\ensuremath{\:}}
\newcommand\green[1]{{\color{correct}{#1}}}



% horizontal rule
\newcommand\hr{
    \noindent\rule[0.5ex]{\linewidth}{0.5pt}
}


% hide parts
\newcommand\hide[1]{}



% si unitx
\usepackage{siunitx}
\sisetup{locale = FR}
% \renewcommand\vec[1]{\mathbf{#1}}
\newcommand\mat[1]{\mathbf{#1}}


% tikz
\usepackage{tikz}
\usepackage{tikz-cd}
\usetikzlibrary{intersections, angles, quotes, calc, positioning}
\usetikzlibrary{arrows.meta}
\usepackage{pgfplots}
\pgfplotsset{compat=1.13}


\tikzset{
    force/.style={thick, {Circle[length=2pt]}-stealth, shorten <=-1pt}
}

% theorems
\makeatother
\usepackage{thmtools}
\usepackage[framemethod=TikZ]{mdframed}
\mdfsetup{skipabove=1em,skipbelow=0em}


\theoremstyle{definition}

\declaretheoremstyle[
    headfont=\bfseries\sffamily\color{ForestGreen!70!black}, bodyfont=\normalfont,
    mdframed={
        linewidth=2pt,
        rightline=false, topline=false, bottomline=false,
        linecolor=ForestGreen, backgroundcolor=ForestGreen!5,
    }
]{thmgreenbox}

\declaretheoremstyle[
    headfont=\bfseries\sffamily\color{NavyBlue!70!black}, bodyfont=\normalfont,
    mdframed={
        linewidth=2pt,
        rightline=false, topline=false, bottomline=false,
        linecolor=NavyBlue, backgroundcolor=NavyBlue!5,
    }
]{thmbluebox}

\declaretheoremstyle[
    headfont=\bfseries\sffamily\color{NavyBlue!70!black}, bodyfont=\normalfont,
    mdframed={
        linewidth=2pt,
        rightline=false, topline=false, bottomline=false,
        linecolor=NavyBlue
    }
]{thmblueline}

\declaretheoremstyle[
    headfont=\bfseries\sffamily\color{RawSienna!70!black}, bodyfont=\normalfont,
    mdframed={
        linewidth=2pt,
        rightline=false, topline=false, bottomline=false,
        linecolor=RawSienna, backgroundcolor=RawSienna!5,
    }
]{thmredbox}

\declaretheoremstyle[
    headfont=\bfseries\sffamily\color{RawSienna!70!black}, bodyfont=\normalfont,
    numbered=no,
    mdframed={
        linewidth=2pt,
        rightline=false, topline=false, bottomline=false,
        linecolor=RawSienna, backgroundcolor=RawSienna!1,
    },
    qed=\qedsymbol
]{thmproofbox}

\declaretheoremstyle[
    headfont=\bfseries\sffamily\color{NavyBlue!70!black}, bodyfont=\normalfont,
    numbered=no,
    mdframed={
        linewidth=2pt,
        rightline=false, topline=false, bottomline=false,
        linecolor=NavyBlue, backgroundcolor=NavyBlue!1,
    },
]{thmexplanationbox}

\declaretheorem[style=thmgreenbox, numberwithin = section, name=Definition]{definition}
\declaretheorem[style=thmbluebox, name=Example]{eg}
\declaretheorem[style=thmredbox, numberwithin = section, name=Proposition]{prop}
\declaretheorem[style=thmredbox, numberwithin = section, name=Theorem]{theorem}
\declaretheorem[style=thmredbox, numberwithin = section,  name=Lemma]{lemma}
\declaretheorem[style=thmredbox, numberwithin = section,  numbered=no, name=Corollary]{corollary}


\declaretheorem[style=thmproofbox, name=Proof]{replacementproof}
\renewenvironment{proof}[1][\proofname]{\vspace{-10pt}\begin{replacementproof}}{\end{replacementproof}}


\declaretheorem[style=thmexplanationbox, name=Proof]{tmpexplanation}
\newenvironment{explanation}[1][]{\vspace{-10pt}\begin{tmpexplanation}}{\end{tmpexplanation}}


\declaretheorem[style=thmblueline, numbered=no, name=Remark]{remark}
\declaretheorem[style=thmblueline, numbered=no, name=Note]{note}

\newtheorem*{uovt}{UOVT}
\newtheorem*{notation}{Notation}
\newtheorem*{previouslyseen}{As previously seen}
\newtheorem*{problem}{Problem}
\newtheorem*{observe}{Observe}
\newtheorem*{property}{Property}
\newtheorem*{intuition}{Intuition}


\usepackage{etoolbox}
\AtEndEnvironment{vb}{\null\hfill$\diamond$}%
\AtEndEnvironment{intermezzo}{\null\hfill$\diamond$}%
% \AtEndEnvironment{opmerking}{\null\hfill$\diamond$}%

% http://tex.stackexchange.com/questions/22119/how-can-i-change-the-spacing-before-theorems-with-amsthm
\makeatletter
% \def\thm@space@setup{%
%   \thm@preskip=\parskip \thm@postskip=0pt
% }
\newcommand{\oefening}[1]{%
    \def\@oefening{#1}%
    \subsection*{Oefening #1}
}

\newcommand{\suboefening}[1]{%
    \subsubsection*{Oefening \@oefening.#1}
}

\newcommand{\exercise}[1]{%
    \def\@exercise{#1}%
    \subsection*{Exercise #1}
}

\newcommand{\subexercise}[1]{%
    \subsubsection*{Exercise \@exercise.#1}
}


\usepackage{xifthen}

\def\testdateparts#1{\dateparts#1\relax}
\def\dateparts#1 #2 #3 #4 #5\relax{
    \marginpar{\small\textsf{\mbox{#1 #2 #3 #5}}}
}

\def\@lesson{}%
\newcommand{\lesson}[3]{
    \ifthenelse{\isempty{#3}}{%
        \def\@lesson{Lecture #1}%
    }{%
        \def\@lesson{Lecture #1: #3}%
    }%
    \subsection*{\@lesson}
    \testdateparts{#2}
}

% \renewcommand\date[1]{\marginpar{#1}}


% fancy headers
\usepackage{fancyhdr}
\pagestyle{fancy}

\fancyhead[LE,RO]{Lance Remigio}
\fancyhead[RO,LE]{\@lesson}
\fancyhead[RE,LO]{}
\fancyfoot[LE,RO]{\thepage}
\fancyfoot[C]{\leftmark}

\makeatother




% notes
\usepackage{todonotes}
\usepackage{tcolorbox}

\tcbuselibrary{breakable}
\newenvironment{verbetering}{\begin{tcolorbox}[
    arc=0mm,
    colback=white,
    colframe=green!60!black,
    title=Opmerking,
    fonttitle=\sffamily,
    breakable
]}{\end{tcolorbox}}

\newenvironment{noot}[1]{\begin{tcolorbox}[
    arc=0mm,
    colback=white,
    colframe=white!60!black,
    title=#1,
    fonttitle=\sffamily,
    breakable
]}{\end{tcolorbox}}




% figure support
\usepackage{import}
\usepackage{xifthen}
\pdfminorversion=7
\usepackage{pdfpages}
\usepackage{transparent}
\newcommand{\incfig}[1]{%
    \def\svgwidth{\columnwidth}
    \import{./figures/}{#1.pdf_tex}
}

% %http://tex.stackexchange.com/questions/76273/multiple-pdfs-with-page-group-included-in-a-single-page-warning
\pdfsuppresswarningpagegroup=1



\usepackage{standalone}
\usepackage{import}
\title{Week 1: Lecture Notes}
\author{Lance Remigio}

\begin{document}
\maketitle   

\section{Lecture 1}

\subsection{Why do we need complex analysis?}

We need it for:
\begin{itemize}
    \item Solving polynomials with either coefficients in either \( \R  \) or \( \C \). For example, the polynomial \( x^{2} + 1  \) does not have a solution in \( \R  \), but it does have a solution \( \C  \).
    \item Solving real integrals that may be difficult to deal with using standard techniques developed in \( \R  \); that is, something like
    \[ \int_{ 0 }^{ \infty  }  \frac{ \sin x  }{x}   \ dx.   \]
    \item Solving problems in physics, particularly, in the Quantum Field Theory.
\end{itemize}

\subsection{What is the goal?}

Let's recall some facts about the real number system \( \R  \).
\begin{enumerate}
    \item[(i)] \( (\R , + , \cdot) \) is a field.
    \item[(ii)] We have an order relation on \( \R  \).
    \item[(iii)] For all \( x \in \R  \), \( x^{2} + 1 > 0  \). Hence, the polynomial equation \( x^{2} + 1 = 0  \) does not have any solutions in \( \R  \).
\end{enumerate}

Our goal is to find the "smallest" field \( \C  \) such that 
\begin{enumerate}
    \item[(i)] \( \R  \) is "contained" in \( \C  \).
    \item[(ii)] For any polynomial \(  f \in \C  \), there exists a solution for \( f \) in \( \C  \). 
\end{enumerate}

Let's assume for a moment that we CAN solve the equation \( x^{2} + 1 = 0   \). Then we define the following set
\[  \zeta = \{ \alpha + i \beta : \alpha, \beta \in \R  \}.   \]
Note that in this set, we are using the properties of \( \R  \) as a vector space, and using the operations defined on that vector space to define the operations of \( \C  \). Recall from Linear Algebra that \( \zeta  \) is just the span of the basis vectors \( 1  \) and \( i  \). In other words, we have
\[ \zeta = \{ \alpha + i \beta : \alpha , \beta \in \R  \} = \text{span}_{\R} \{ 1, i \}.     \]

Now, let's define the two operations, addition and multiplication, on \( \C  \).

\begin{definition}[Operations on \( \C  \)]
    Let \( z,w \in \C  \) and set \( z = {\alpha}_{1} + i {\beta}_{1} \) and \( w = {\alpha}_{2} + i {\beta}_{2} \) for any \( {\alpha}_{1}, {\alpha}_{2}, {\beta}_{1}, {\beta}_{2} \in \R  \). We define the two operations, addition \( +  \) and multiplication \( \cdot \), in the following way:    
    \begin{itemize}
        \item Addition:  
            \begin{align*}
                z + w &= ({\alpha}_{1} + i {\beta}_{1}) + ({\alpha}_{2} + i {\beta}_{2}) \\
                      &= ({\alpha}_{1} + {\alpha}_{2}) + i ({\beta}_{1} + {\beta}_{2}).
            \end{align*}
        \item Multiplication:
            \begin{align*}
                z \cdot w &= ({\alpha}_{1} + i {\beta}_{1}) \cdot ({\alpha}_{2} + i {\beta}_{2})  \\
                          &= ({\alpha}_{1} {\alpha}_{2} - {\beta}_{1} {\beta}_{2}) + i ({\alpha}_{1} {\beta}_{2} + {\beta}_{1} {\alpha}_{2}).
            \end{align*}
    \end{itemize}
\end{definition}

With these operations, we can say that \( \C  \) forms a field.

\begin{prop}
   The defined operations of \( \C  \) form a field.
\end{prop}
\begin{proof}
\textbf{To do.}
\end{proof}

\begin{lemma}[Existence of a Square Root]
    Let \(\alpha + i \beta \in \zeta \). Then there exists \( \gamma + i \delta \in \zeta  \) such that \( (\gamma + i \delta)^{2} = \alpha + i \beta \).
\end{lemma}
\begin{proof}
    \textbf{To do.}
\end{proof}

\section{Lecture 2}

\subsection{Topics}

\begin{itemize}
    \item Discuss the complex plane \( \C  \) as a working model for complex numbers.
    \item Discuss the Euclidean Topology on \( \C  \).
    \item Discuss polar representation of a complex number.
\end{itemize}

\section{Complex Plane \( \C  \) as a working model for Complex Numbers}
\begin{lemma}
  Let \( F  \) be a field containing \( \R  \) and the equation \( x^{2} + 1  =0 \) contains a solution \( i \in F  \).
\begin{enumerate}
    \item[(i)] Let   
  \[  \C = \{ \alpha + i \beta : \alpha, \beta \in \R  \}. \]
  Then \( \C  \) is a \textbf{subfield} of \( F  \).
    \item[(ii)] Let \( F  \) be another field containing \( \R  \) and containing a solution \( i' \) of \( x^{2} + 1 = 0   \) and
        \[  \C' = \{ \alpha + \beta i' : \alpha, \beta \in \R  \}.  \]
        Then \( \C  \) and \( \C' \) are isomorphic as fields.
\end{enumerate}
\end{lemma}

The second part of this lemma is simply saying that \( \C  \) is a unique subfield of \( F  \).

\subsection{A model for \( \C  \)}

Let \( \C = \R^{2} \). Then define addition \( +  \) and multiplication \( \cdot \) as 
\[  ({\alpha}_{1}, {\beta}_{1}) + ({\alpha}_{2}, {\beta}_{2}) = ({\alpha}_{1} + {\alpha}_{2}, {\beta}_{1} + {\beta}_{2}) \]
and 
\[  ({\alpha}_{1}, {\beta}_{1}) \cdot ({\alpha}_{2}, {\beta}_{2}) = ({\alpha}_{1} {\alpha}_{2} - {\beta}_{1} {\beta}_{2} , {\alpha}_{1} {\beta}_{2} + {\alpha}_{2} {\beta}_{1}), \]
respectively.


\begin{lemma} The complex numbers \( \C  \) have the following properties
    \begin{enumerate}
        \item[(i)] \( (\C, +, \cdot) \) is a field.
        \item[(ii)] Let \( \mathcal{J}: \R \to \C  \) by \( x \to (x,0) \). Then \( \mathcal{J}  \) is a subfield and forms an isomorphism between \( \R  \) and \( \mathcal{J}(\R ) \).
    \end{enumerate}
\end{lemma}

\begin{proof}
Rough outline of proof:
\begin{enumerate}
    \item[(i)] Prove that \( \mathcal{J}(\R ) \) is a subfield.
    \item[(ii)] Prove that \( \mathcal{J}(\R) \) is a field homorphism.
    \item[(iii)] Prove that \( \mathcal{J}(\R) \) is an isomorphism; that is, show that \( \mathcal{J} \) is a bijective map.
\end{enumerate}
\end{proof}

\begin{definition}[Constructing a solution for \( x^2 + 1 = 0  \)]
   Define \( i = (0,1) \) as our imaginary number in \( \C  \) and let \( i^{2} = (-1,0) \). 
\end{definition}


\begin{prop}
    Given \( \alpha, \beta \in \R  \), show that \( (\alpha, 0 ) + (0,1)(\beta, 0) = (\alpha, \beta) \). 
\end{prop}
\begin{proof}
    Let \( \alpha, \beta \in \R  \). Then
    \[ (\alpha, 0 ) + (0,1)(\beta, 0) = (\alpha, 0) + (\beta, 0) = (\alpha, \beta).  \]
\end{proof}

\begin{definition}[Real and Imaginary part of Complex Number]
   Let \( z = \alpha + i \beta  \) for \( \alpha, \beta \in \R  \). Then  
   \begin{center}
       \( \Re(z) = \alpha  \) and \( \Im(z) = \beta   \),
   \end{center}
   are the \textbf{real and imaginary of \( z  \)}, respectively.
   If \( \Im(z) = 0   \), \( z  \) is a real number, and if \( \Re(z) = 0  \), then we call \( z  \) \textbf{purely imaginary}.
\end{definition}

\begin{definition}[Complex Conjugate]
    Let \( z = \alpha + i \beta  \) be a complex number. Its complex conjugate is defined as \( \overline{z} = \alpha - i \beta  \).
\end{definition}

Geometrically, this is viewed as a reflection of the ordered pair \( z = (\alpha, \beta)  \) over the \( x- \)axis.

\begin{prop}
    For any \( z,w \in \C  \), we have the following properties:
   \begin{enumerate}
       \item[(i)] \( \overline{\overline{z}} = z  \).
        \item[(ii)] \( \overline{z \pm w} = \overline{z} \pm \overline{w} \). 
        \item[(iii)] \( \overline{zw} = \overline{z} \cdot \overline{w} \).
        \item[(iv)] \( \Re(z) = \frac{ 1 }{ 2 }  ( z + \overline{z}) \).
        \item[(v)] \( \Im(z) = \frac{ 1 }{ 2i } (z - \overline{z}) \). 
        \item[(vi)] \( z \overline{z} = \alpha^{2} + \beta^{2} \geq 0 \).
   \end{enumerate} 
\end{prop}
\begin{proof}
    \begin{enumerate}
        \item[(i)] 
    \end{enumerate}
\end{proof}

\begin{definition}[Modulus of a Complex Number]
   Let \( z \in \C  \). We define the \textbf{modulus} \( | z  |  = \sqrt{ z \overline{z} }  \).
\end{definition}


\begin{prop}[More Properties of Complex Numbers]
   Given \( z,w \in \C  \), we have the following properties:
   \begin{enumerate}
       \item[(i)] \( | z  |  = 0  \) if and only if \( z = 0  \).
        \item[(ii)] \( | zw  |  = | z  | | w |  \).
        \item[(iii)] \( | \Re(z) |  \leq | z  |  \).
        \item[(iv)] \( | \Im(z) | \leq | z |  \).
        \item[(v)] \( | z + w  | \leq | z  |  + | w |  \).
        \item[(vi)] For any \( z \in \C  \) and \( z \neq 0  \), then
            \(  \frac{ 1 }{ z }  = \frac{ \overline{z} }{ | z |^{2} }. \)
   \end{enumerate}
\end{prop}

\subsection{Viewing \( \C  \) as a Metric Space}

In \( \C  \), the metric we will be using is \( d(z,w) =  | z - w  |  \).

\begin{definition}[Metric Space]
   For \( z,w \in \C  \), we call \( \C  \) a metric space if it satisfies the following properties: 
   \begin{enumerate}
       \item[(i)] \( d(z,w) = d(w,z) \).
        \item[(ii)] \( d(z,w) = 0  \) if and only if \( z = w  \) and \( d(z,w) > 0  \) if and only if \( z \neq w  \).
   \end{enumerate}
\end{definition}

Now we have \( \C  \) is a metric space endowed with the metric \( d(z,w) = | z - w  |  \).








\end{document}
