\documentclass[a4paper]{article}
\usepackage[utf8]{inputenc}
\usepackage[T1]{fontenc}
% \usepackage{fourier}
\usepackage{textcomp}
\usepackage{hyperref}
\usepackage[english]{babel}
\usepackage{url}
% \usepackage{hyperref}
% \hypersetup{
%     colorlinks,
%     linkcolor={black},
%     citecolor={black},
%     urlcolor={blue!80!black}
% }
\usepackage{graphicx} \usepackage{float}
\usepackage{booktabs}
\usepackage{enumitem}
% \usepackage{parskip}
% \usepackage{parskip}
\usepackage{emptypage}
\usepackage{subcaption}
\usepackage{multicol}
\usepackage[usenames,dvipsnames]{xcolor}
\usepackage{ocgx}
% \usepackage{cmbright}


\usepackage[margin=1in]{geometry}
\usepackage{amsmath, amsfonts, mathtools, amsthm, amssymb}
\usepackage{thmtools}
\usepackage{mathrsfs}
\usepackage{cancel}
\usepackage{bm}
\newcommand\N{\ensuremath{\mathbb{N}}}
\newcommand\R{\ensuremath{\mathbb{R}}}
\newcommand\Z{\ensuremath{\mathbb{Z}}}
\renewcommand\O{\ensuremath{\emptyset}}
\newcommand\Q{\ensuremath{\mathbb{Q}}}
\newcommand\C{\ensuremath{\mathbb{C}}}
\newcommand\F{\ensuremath{\mathbb{F}}}
\DeclareMathOperator{\sgn}{sgn}
\DeclareMathOperator{\diam}{diam}
\DeclareMathOperator{\LO}{LO}
\DeclareMathOperator{\UP}{UP}
\DeclareMathOperator{\card}{card}
\DeclareMathOperator{\Arg}{Arg}
\DeclareMathOperator{\Dom}{Dom}
\DeclareMathOperator{\Log}{Log}
\DeclareMathOperator{\dist}{dist}
% \DeclareMathOperator{\span}{span}
\usepackage{systeme}
\let\svlim\lim\def\lim{\svlim\limits}
\renewcommand\implies\Longrightarrow
\let\impliedby\Longleftarrow
\let\iff\Longleftrightarrow
\let\epsilon\varepsilon
\usepackage{stmaryrd} % for \lightning
\newcommand\contra{\scalebox{1.1}{$\lightning$}}
% \let\phi\varphi
\renewcommand\qedsymbol{$\blacksquare$}

% correct
\definecolor{correct}{HTML}{009900}
\newcommand\correct[2]{\ensuremath{\:}{\color{red}{#1}}\ensuremath{\to }{\color{correct}{#2}}\ensuremath{\:}}
\newcommand\green[1]{{\color{correct}{#1}}}

% horizontal rule
\newcommand\hr{
    \noindent\rule[0.5ex]{\linewidth}{0.5pt}
}

% hide parts
\newcommand\hide[1]{}

% si unitx
\usepackage{siunitx}
\sisetup{locale = FR}
% \renewcommand\vec[1]{\mathbf{#1}}
\newcommand\mat[1]{\mathbf{#1}}

% tikz
\usepackage{tikz}
\usepackage{tikz-cd}
\usetikzlibrary{intersections, angles, quotes, calc, positioning}
\usetikzlibrary{arrows.meta}
\usepackage{pgfplots}
\pgfplotsset{compat=1.13}

\tikzset{
    force/.style={thick, {Circle[length=2pt]}-stealth, shorten <=-1pt}
}

% theorems
\makeatother
\usepackage{thmtools}
\usepackage[framemethod=TikZ]{mdframed}
\mdfsetup{skipabove=1em,skipbelow=1em}

\theoremstyle{definition}

\declaretheoremstyle[
    headfont=\bfseries\sffamily\color{ForestGreen!70!black}, bodyfont=\normalfont,
    mdframed={
        linewidth=1pt,
        rightline=false, topline=false, bottomline=false,
        linecolor=ForestGreen, backgroundcolor=ForestGreen!5,
    }
]{thmgreenbox}

\declaretheoremstyle[
    headfont=\bfseries\sffamily\color{NavyBlue!70!black}, bodyfont=\normalfont,
    mdframed={
        linewidth=1pt,
        rightline=false, topline=false, bottomline=false,
        linecolor=NavyBlue, backgroundcolor=NavyBlue!5,
    }
]{thmbluebox}

\declaretheoremstyle[
    headfont=\bfseries\sffamily\color{NavyBlue!70!black}, bodyfont=\normalfont,
    mdframed={
        linewidth=1pt,
        rightline=false, topline=false, bottomline=false,
        linecolor=NavyBlue
    }
]{thmblueline}

\declaretheoremstyle[
    headfont=\bfseries\sffamily, bodyfont=\normalfont,
    numbered = no,
    mdframed={
        rightline=true, topline=true, bottomline=true,
    }
]{thmbox}

\declaretheoremstyle[
    headfont=\bfseries\sffamily, bodyfont=\normalfont,
    numbered=no,
    % mdframed={
    %     rightline=true, topline=false, bottomline=true,
    % },
    qed=\qedsymbol
]{thmproofbox}

\declaretheoremstyle[
    headfont=\bfseries\sffamily\color{NavyBlue!70!black}, bodyfont=\normalfont,
    numbered=no,
    mdframed={
        rightline=false, topline=false, bottomline=false,
        linecolor=NavyBlue, backgroundcolor=NavyBlue!1,
    },
]{thmexplanationbox}

\declaretheorem[
    style=thmbox, 
    % numberwithin = section,
    numbered = no,
    name=Definition
    ]{definition}

\declaretheorem[
    style=thmbox, 
    name=Example,
    ]{eg}

\declaretheorem[
    style=thmbox, 
    % numberwithin = section,
    name=Proposition]{prop}

\declaretheorem[
    style = thmbox,
    numbered=yes,
    name =Problem
    ]{problem}

\declaretheorem[style=thmbox, name=Theorem]{theorem}
\declaretheorem[style=thmbox, name=Lemma]{lemma}
\declaretheorem[style=thmbox, name=Corollary]{corollary}

\declaretheorem[style=thmproofbox, name=Proof]{replacementproof}

\declaretheorem[style=thmproofbox, 
                name = Solution
                ]{replacementsolution}

\renewenvironment{proof}[1][\proofname]{\vspace{-1pt}\begin{replacementproof}}{\end{replacementproof}}

\newenvironment{solution}
    {
        \vspace{-1pt}\begin{replacementsolution}
    }
    { 
            \end{replacementsolution}
    }

\declaretheorem[style=thmexplanationbox, name=Proof]{tmpexplanation}
\newenvironment{explanation}[1][]{\vspace{-10pt}\begin{tmpexplanation}}{\end{tmpexplanation}}

\declaretheorem[style=thmbox, numbered=no, name=Remark]{remark}
\declaretheorem[style=thmbox, numbered=no, name=Note]{note}

\newtheorem*{uovt}{UOVT}
\newtheorem*{notation}{Notation}
\newtheorem*{previouslyseen}{As previously seen}
% \newtheorem*{problem}{Problem}
\newtheorem*{observe}{Observe}
\newtheorem*{property}{Property}
\newtheorem*{intuition}{Intuition}

\usepackage{etoolbox}
\AtEndEnvironment{vb}{\null\hfill$\diamond$}%
\AtEndEnvironment{intermezzo}{\null\hfill$\diamond$}%
% \AtEndEnvironment{opmerking}{\null\hfill$\diamond$}%

% http://tex.stackexchange.com/questions/22119/how-can-i-change-the-spacing-before-theorems-with-amsthm
\makeatletter
% \def\thm@space@setup{%
%   \thm@preskip=\parskip \thm@postskip=0pt
% }
\newcommand{\oefening}[1]{%
    \def\@oefening{#1}%
    \subsection*{Oefening #1}
}

\newcommand{\suboefening}[1]{%
    \subsubsection*{Oefening \@oefening.#1}
}

\newcommand{\exercise}[1]{%
    \def\@exercise{#1}%
    \subsection*{Exercise #1}
}

\newcommand{\subexercise}[1]{%
    \subsubsection*{Exercise \@exercise.#1}
}


\usepackage{xifthen}

\def\testdateparts#1{\dateparts#1\relax}
\def\dateparts#1 #2 #3 #4 #5\relax{
    \marginpar{\small\textsf{\mbox{#1 #2 #3 #5}}}
}

\def\@lesson{}%
\newcommand{\lesson}[3]{
    \ifthenelse{\isempty{#3}}{%
        \def\@lesson{Lecture #1}%
    }{%
        \def\@lesson{Lecture #1: #3}%
    }%
    \subsection*{\@lesson}
    \testdateparts{#2}
}

% \renewcommand\date[1]{\marginpar{#1}}


% fancy headers
\usepackage{fancyhdr}
\pagestyle{fancy}

\makeatother

% notes
\usepackage{todonotes}
\usepackage{tcolorbox}

\tcbuselibrary{breakable}
\newenvironment{verbetering}{\begin{tcolorbox}[
    arc=0mm,
    colback=white,
    colframe=green!60!black,
    title=Opmerking,
    fonttitle=\sffamily,
    breakable
]}{\end{tcolorbox}}

\newenvironment{noot}[1]{\begin{tcolorbox}[
    arc=0mm,
    colback=white,
    colframe=white!60!black,
    title=#1,
    fonttitle=\sffamily,
    breakable
]}{\end{tcolorbox}}

% figure support
\usepackage{import}
\usepackage{xifthen}
\pdfminorversion=7
\usepackage{pdfpages}
\usepackage{transparent}
\newcommand{\incfig}[1]{%
    \def\svgwidth{\columnwidth}
    \import{./figures/}{#1.pdf_tex}
}

% %http://tex.stackexchange.com/questions/76273/multiple-pdfs-with-page-group-included-in-a-single-page-warning
\pdfsuppresswarningpagegroup=1



\begin{document}

\section{Handout-14}

\subsection{Plan}

Discuss vector valued integrals and complex line integrals.

\subsection{Recap}

\begin{itemize}
    \item Let's recall how we find anti-derivative or primitive of a function in calculus of one-variable.

\textbf{Question:} Let \( f: [a,b] \to \R  \). Is it possible to find \( F : [a,b] \to \R  \) such that \( F'(x) = f(x)  \) for all \( x \in [a,b] \)?

\textbf{Answer:} Yes, it is possible. A sufficient condition is to have \( f  \) be Riemann integrable on \( [a,b] \) such that 
\[  F(x) = \int_{ a }^{ x }  f(t) \ dt. \]

    \textbf{Question:} If \( f: D \to \C  \) where \( D \subseteq \C  \), can we find \( F: D \to \C  \) such that \( F'(z) = f(z)  \) for all \( z  \)?

    \textbf{Answer:} Not obvious. First note that \( f = u + iv \). Thus, if we want to mimick the real case, then we need to be able to discuss on integral of a "vector" valued function over a line segment. But a careful set up will allows us to use some of the ideas from calculus and real analysis.

\item \textbf{Digression:} Vector valued integrals: 
    Let \( f: [a,b] \to \R^{2} \), \( f(t) = \begin{pmatrix} x(t) \\ y(t) \end{pmatrix}  \). We say that \( f \) is integrable on \( [a,b] \) if both \( u  \) and \( v  \) are integrable in the sense of real analysis; that is, 
    \[  \int_{ a }^{ b } | x(t) |  \ dt \ \ \text{and} \ \ \int_{ a }^{ b } | y(t) |  \ dt \ \ \text{exist}. \]
    We define 
    \[  \int_{ a }^{ b } f(t) \ dt = \begin{pmatrix} \int_{ a }^{ b }  x(t)  \ dt \\ \int_{ a }^{ b }  y(t)  \ dt  \end{pmatrix}. \]
    We can show 
    \begin{enumerate}
        \item[(i)] Linearity of the integral.
        \item[(ii)] Let \( F:[a,b] \to \R^{2} \) such that \( F'(t) = f(t) \) for all \( t \in [a,b] \). Then 
            \[  \int_{ a }^{ b } f(t) \ dt = F(b) - F(a) \in \R^{2} \]
            assuming all quantities exist.
    \end{enumerate}
\item Integration of functions of the form \( f: [a,b] \to \C  \) with \( f(t) = x(t) + i y(t) \) are integrable and define
    \[  \int_{ a }^{ b }  f(t) \ dt = \int_{ a }^{ b } x(t) \ dt + i \int_{ a }^{ b }  y(t) \ dt. \]
\end{itemize}

\begin{eg}
    Let \( f: [0,1] \to \C  \) be defined by \( f(t) = 3 t^{2} + 2i t \). Then one can easily check that  
    \[  \int_{ 0 }^{ 1 }  f(t) \ dt = 1 + i. \]
\end{eg}

Let \( \mathcal{F} = \{ f: [a,b] \to \C \ \text{such that \( f  \) is integrable} \}  \).
\begin{enumerate}
    \item[(i)] Then the map \( I: \mathcal{F} \to \C  \) defined by
        \[  I(f) = \int_{ a }^{ b } f(t) \ dt \ \text{is \( \C-\)linear}. \]
        Indeed, it follows immediately that for any \( {f}_{1}, {f}_{2} \in \mathcal{F} \), we have 
        \[  I({f}_{1} + {f}_{2}) = I({f}_{1}) + I({f}_{2}). \]
        For any \( \alpha \in \R  \) and \( f \in \mathcal{F} \), we have 
        \begin{align*}
            I(\alpha f ) &= \int_{ a }^{ b }  [\Re(\alpha) x(t) - \Im(\alpha) y(t) + i (\Re(\alpha) y(t) + \Im(\alpha) x(t))]  \ dt \\
                         &= \Re(\alpha) \int_{ a }^{ b }  x(t) \ dt - \Im(\alpha) \int_{ a }^{ b } y(t) \ dt \\
                         &+ i [ \Re(\alpha) \int_{ a }^{ b } y(t) \ dt + \Im(\alpha) \int_{ a }^{ b }  x(t) \ dt] \\
                         &= \alpha \int_{ a }^{ b } f(t) \ dt.
        \end{align*}
    \item[(ii)] Let \( F: [a,b] \to \C  \) such that \( F'(t = f(t)) \) for all \( t \in [a,b] \). Then \( \int_{ a }^{ b } f(t) \ dt = F(b) - F(a) \).
    \item[(iii)] Let \( {I}_{1}, {I}_{2} \) be two intervals in \( \R  \), \( \varphi : {I}_{1} \to {I}_{2} \) continuous differentiable, \( f: {I}_{2} \to \C  \) integrable. Let \( a,b \in {I}_{1} \), \( a < b \). Then
        \[  \int_{ \varphi(a) }^{ \varphi(b) }  f(s)  \ ds  = \int_{ a }^{ b }  f(\varphi(t)) \varphi'(t) \ dt. \]
    \item[(iv)] Let \( f,g : [a,b] \to \C  \), \( f \) and \( g  \) are continuously differentiable (i.e \( \Re(f), \Im(f), \Re(g), \Im(g) \) are continuously differentiable). Then
        \[  \int_{ a }^{ b } f(t) g'(t) \ dt = f(b)g(b) - f(a)g(a) - \int_{ a }^{ b } f'(t) g(t) \ dt. \]
\end{enumerate}

The key point here is that if we have a complex valued function defined on \( I \subseteq \R \), then the integral enjoys properties similar to that of a real valued function defined on \( I  \).

\subsection{Complex Line Integrals}

We would like to define the integral of a complex-valuedf function along a curve in \( \C  \) in the same way we defined a line integral of a function in multivariable calculus.

We will start by introducing some basic terminologies.

\begin{definition}[Curve]
    A curve in \( \C  \) is a continuous map \( \alpha : [a,b] \to \C  \), where \( [a,b] \) is an interval in \( \R  \). We call \( \alpha(a) \) the starting point and \( \alpha(b) \) the end point of \( \alpha \).
\end{definition}

\begin{definition}[Trace of a Curve]
    The  set 
    \[  \text{tr}(\alpha) = \{ \alpha(t) : t \in [a,b] \} \subseteq  \C  \]
    is called the trace or the image of \( \alpha \).
\end{definition}

\begin{eg}
    \begin{enumerate}
        \item[(i)] Let \( z,w \in \C  \) and define \( \alpha: [0,1] \to \C  \) by 
            \[  \alpha(t) = (1-t)z + tw , \ \ \alpha(0) = z , \ \alpha(1) = w. \]
        \item[(ii)] Let \( \alpha: [0,1] \to \C \) be defined by \( \alpha(t) = e^{2\pi i t} \) with \( \alpha(0) = 1  \) and \( \alpha(1) = 1 \).
    \end{enumerate}
\end{eg}

\begin{definition}[Smooth Curve]
    A curve \( \alpha \) is smooth if it is continuously differentiable.
\end{definition}

\begin{definition}[Piecewise Smooth Curve]
    A curve \( \alpha : [a,b] \to \C  \) is \textbf{piecewise smooth} if there is a partition \( a = {a}_{0} < {a}_{1} < \cdots < {a}_{n} = b \) such that \( \alpha |_{[{a}_{i-1}, {a}_{i}]} \) is smooth for \( i = 1 ,2, \dots , n \).
\end{definition}

\begin{definition}[Integral of Curve in terms of its Parametrization]
    Let \( \alpha: [a,b] \to \C  \) be a smooth curve, \( f: D \to \C  \) continuous, \( \alpha([a,b]) \subseteq  D \). Then, we define 
    \[  \int_{ \alpha }^{  } f = \int_{ \alpha }^{  } f(z) \ dz = \int_{ a }^{ b } f(\alpha(t)) \alpha'(t) \ dt\tag{*} \]
    where the integrand of the above equation is a complex valued function defined on \( [a,b] \). We call (*) the \textbf{line integral of \( f  \) along \( \alpha \).}
\end{definition}

\begin{definition}[Integral of a Piecewise Smooth Curve]
    Assume \( \alpha: [a,b] \to \C  \) is a piecewise smooth curve with partition
    \[  a = {a}_{0} < {a}_{1} < \cdots < {a}_{n} = b \]
    such that \( \alpha |_{[{a}_{i-1}, {a}_{i}]} \) smooth for \( i = 1,2,\dots, n \). Let \( f: D \to \C  \) be a continuous function and \( \alpha([a,b]) \subseteq D \). Then we define 
    \[  \int_{ \alpha }^{  } f(z) \ dz = \sum_{ i=1  }^{ n } \int_{ \alpha |_{[{a}_{i-1}, {a}_{i}]} }^{  }  f(z) \ dz. \tag{**} \]
    We call (**) the integral along the piecewise curve \( \alpha \).
\end{definition}

\begin{lemma}
    Let \( \alpha: [a,b] \to \C  \) be a piecewise smooth curve. Let \( a = {a}_{0} < \cdots < {a}_{n} = b \) be a partition of \( [a,b] \) such that \( \alpha |_{[{a}_{i-1}, {a}_{i}]} \) is smooth for \( i = 1,2,\dots, n \). Let \( a = {a}_{0}' < {a}_{1}' < \dots < {a}_{m}' = b \) be another partition of \( [a,b] \) such that \( \alpha  |_{[{a}_{j-1}, {a}_{j}]} \) is smooth. Assume that \( f: D \to \C  \) continuous and \( \alpha([a,b]) \subseteq  \C  \). Then
    \[  \sum_{ j=1  }^{ m } \int_{ \alpha |_{[{a}_{j-1}', {a}_{j}']} }^{  } f(z) \ dz = \sum_{ i=1  }^{ n } \int_{ \alpha |_{[{a}_{i-1}, {a}_{i}]} }^{  } f(z) \ dz.  \]
\end{lemma}

\begin{proof}
    Our goal is to show that for any smooth curve \( \gamma: [c,d] \to \C  \) and \( c < f < d  \), we have 
    \[  \int_{ \gamma }^{  } f(z)  \ dz = \int_{ \gamma |_{[c,f]} }^{  } f(z)  \ dz + \int_{ \gamma |_{[f,d]} }^{  } f(z) \ dz. \tag{\( \dagger \)} \]
    Note that (*) follows immediately from the properties of integrals of complex valued functions defined on an interval. Indeed, without loss of generality, assume that \( m = n + 1 \) and \[ {a}_{0} = {a}_{0}' < {a}_{1}' < {a}_{1} = {a}_{2}' < {a}_{2} = {a}_{3}' < \cdots < {a}_{n-1} = {a}_{n}' < {a}_{n} = {a}_{n+1}'.  \]
    By (*), we have 
    \[  \int_{ \alpha|_{[{a}_{0}, {a}_{1}]} }^{  }  f  = \int_{ \alpha |_{[{a}_{0}, {a}_{1}']} }^{  } f + \int_{ \alpha |_{[{a}_{1}', {a}_{2}']} }^{  } f. \]
    Thus, 
    \[  \sum_{ i=1  }^{ n } \int_{ \alpha |_{[{a}_{i-1}, {a}_{i}]} }^{  }  f  = \sum_{ j=1  }^{ n } \int_{ \alpha |_{[{a}_{j-1}', {a}_{j}']} }^{  }  f. \]
\end{proof}

The lemma above shows us that (*) is well-defined.

\begin{definition}[Arc Length]
    Let \( \alpha: [a,b] \to \C  \) be a smooth curve. Then we define 
    \[  \ell(\alpha) = \int_{ a }^{ b } | \alpha'(t) |  \ dt.  \]
    If \( \alpha \) is piecewise smooth then we define \[ \ell(\alpha) = \text{sum of arc lengths of smooth arcs of \( \alpha \)}.   \]
\end{definition}



\end{document}

