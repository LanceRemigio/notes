\section{Properties of the Integral}

For our first property of integration, integrating over a closed interval \( [a,b]  \) is equivalent to taking the sum of the closed intervals, say, \( [a,c]  \) and \( [c,b]  \) that make up \( [a,b]  \).

\begin{tcolorbox}
\begin{thm}
	Assume \( f: [a,b] \to \R   \) is bounded, and let \( c \in (a,b)  \). Then, \( f  \) is integrable on \( [a,b]  \) if and only if \( f  \) integrable on \( [a,c]  \) and \( [c,b]  \). In this case, we have 
	\[  \int_{ a }^{ b } f = \int_{ a }^{ c }  f  + \int_{ b }^{ c }  f. \]
\end{thm}
\end{tcolorbox}

\begin{proof}
	Suppose \( f \) is integrable on \( [a,b]  \). Then for every \( \epsilon > 0   \), there exists a partition \( P  \) such that 
	\[  U(f,P) - L(f,P) < \epsilon. \]
	Since the refinement of a paritition might cause the upper and lower sums to be closet together, all we need to do is simply add the point \( c  \) to \( P  \) if it does not already exists. Then, letting \( P_{1} = P \cap [a,c]  \) be a partition of \( [a,c]  \), and \( P_{2} = P \cap [c,b]  \) be a partition of \( [c,b]  \), it follows that 
	\[  U(f,P_{1}) - L(f, P_{1}) < \epsilon \ \text{ and } \ U(f, P_{2}) - L(f, P_{2}) < \epsilon. \] This implies that \( f  \) is integrable on \( [a,c]  \) and \( [c,b]  \).

	Conversely, if we are given that \( f  \) is integrable on the two smaller intervals \( [a,c]  \) and \( [c,b] \), then given an \( \epsilon > 0  \), we can create partitions \( P_{1}   \) and \( P_{2} \) of \( [a,c] \) and \( [c,b] \), respectively, such that 
	\[  U(f, P_{1}) - L(f, P_{1}) < \frac{ \epsilon  }{ 2  } \ \text{ and } U(f, P_{2}) - L(f,P_{2}) < \frac{ \epsilon  }{ 2 }.  \]
	Since \( P = P_{1} \cup P_{2}  \) produces a partition of \( [a,b]  \), we must have 
	\[ U(f, P) - L(f,P) < \epsilon.  \] Hence, \( f  \) is integrable on \( [a,b]  \).
	Now let \( P = P_{1} \cup P_{2} \) as before, we have 
	\begin{align*}
		\int_{ a }^{ b } f  \leq U(f, P) &< L(f, P)  + \epsilon  \\
										 &= L(f, P_{1}) + L(f, P_{2}) + \epsilon \\
										 &leq \int_{ a }^{ c } f  + \int_{ c }^{ b } f + \epsilon,
	\end{align*}
	This implies that \( \int_{ a }^{ b } f \leq \int_{ a }^{ c }f + \int_{ c }^{ b }  f \). To get the other inequality, observe that 
	\begin{align*}
	    \int_{ a }^{ c } f + \int_{ c }^{ b } f   &\leq U(f, P_{1}) + U(f, P_{2})  \\
												  &< L(f, P_{1}) + L(f, P_{2}) + \epsilon \\
												  &= L(f, P) + \epsilon \\
												  &\leq \int_{ a }^{ b } f + \epsilon.
	\end{align*}
	Since \( \epsilon > 0  \) is arbitrary, we must have that 
	\[  \int_{ a }^{ c } f + \int_{ c }^{ b } f \leq \int_{ a }^{ b } f, \]
	so hence, we have 
	\[  \int_{ a }^{ c } f  + \int_{ c }^{ b } f = \int_{ a }^{ b } f, \]
	as desired.
\end{proof}

Some more properties of integration is outlined in the next theorem.

\begin{tcolorbox}
\begin{thm}
	Assume \( f  \) and \( g  \)  are integrable functions on the interval \( [a,b]  \).
	\begin{enumerate}
		\item[(i)] The function \( f + g  \) is integrable on \( [a,b]  \) with \( \int_{ a }^{ b } (f + g) = \int_{ a }^{ b }  f + \int_{ a }^{ b } g  \).
		\item[(ii)] For \( k \in \R  \), the function \( kf  \) is integrable with \( \int_{ a }^{ b } kf  = k \int_{ a }^{ b }  f. \)
		\item[(iii)] If \( m \leq f(x) \leq M  \) on \( [a,b]  \), then \( m(b-a) \leq \int_{ a }^{ b } f \leq M(b-a) \).
		\item[(iv)] If \( f(x) \leq g(x)  \) on \( [a,b]  \), then \( \int_{ a }^{ b } f  \leq \int_{ a }^{ b } g  \).
		\item[(v)] The function \( | f |   \) is integrable and \( | \int_{ a }^{ b }  f | \leq \int_{ a }^{ b } | f |. \) 
	\end{enumerate}
\end{thm}
\end{tcolorbox}

\begin{proof}\label{proof:theproof}
	Let \( \epsilon > 0 \). Since \( f \) and \( g  \) are integrable on \( [a,b]  \), there exists a partition \( P_{\epsilon } \) such that 
	\[  U(f, P_{\epsilon }) - L(f, P_{\epsilon }) < \frac{ \epsilon  }{ 2 }   \]
	and 
	\[  U(g, P_{\epsilon }) - L(g, P_{\epsilon }) < \frac{ \epsilon  }{ 2 } . \]
	To show that \( (f+g)  \) is integrable on \( [a,b] \), we must show that there exists a partition \( P_{\epsilon } \)
	\[  U(f+g, P_{\epsilon }) - L(f +g, P_{\epsilon }) < \epsilon. \]
	First, observe that by the properties of the supremum and infimum, we have 
	that 
	\[  U(f+g, P_{\epsilon }) \leq U(f, P_{\epsilon }) + U(g, P_{\epsilon }) \]
	and 
	\[  L(f+g, P_{\epsilon }) \geq L(f, P_{\epsilon }) + L(g, P_{\epsilon }). \]
	Now observe that
	\begin{align*}
		U(f+g, P_{\epsilon }) - L(f+g, P_{\epsilon }) &\leq [U(f,P_{\epsilon }) + U(g, P_{\epsilon })] - [L(f,P_{\epsilon }) + L(g, P_{\epsilon })] \\
													  &= [U(f,P_{\epsilon }) - L(f,P_{\epsilon })] + [U(g, P_{\epsilon }) - L(g, P_{\epsilon })] \\
													  &< \frac{ \epsilon  }{ 2 }  + \frac{ \epsilon  }{ 2 } \\
													  &= \epsilon.
	\end{align*}
	Hence, \( (f+g)  \) integrable on \( [a,b]  \) by Theorem 7.2.8.  

	Now to show 
	\[  \int_{ a }^{ b } (f+g)  = \int_{ a }^{ b } f + \int_{ a }^{ b } g \]
	we must show 
	\[  \int_{ a }^{ b } (f+g) \leq \int_{ a }^{ b } f + \int_{ a }^{ b } g \ \text{ and } \int_{ a }^{ b } (f+g) \geq \int_{ a }^{ b } f + \int_{ a }^{ b } g  \]
	Since \( (f+g)  \) integrable on \( [a,b] \), we know that \( U(f+g) = L(f+g) = \int_{ a }^{ b } (f+g)  \). Then for any partition \( P  \), we can use the properties of the lower and upper sum derived back in section 7.2 to get 
	\begin{align*}
	    \int_{ a }^{ b } (f+g)  &\leq U(f+g, P) \\
								&< L(f+g, P) + \epsilon \\
								&\leq L(f) + L(g) + \epsilon \\
								&= \int_{ a }^{ b } f  + \int_{ a }^{ b } g + \epsilon.
	\end{align*}
	Since \( \epsilon > 0   \) is arbitrary, we have \( \int_{ a }^{ b } (f+g)  \leq \int_{ a }^{ b } f + \int_{ a }^{ b } g  \). To get the other inequality, we employ a similar process as above. Observe that, 
	\begin{align*}
	    \int_{ a }^{ b } f + \int_{ a }^{ b } g &\leq U(f,P) + U(g, P)\\
												&< L(f,P) + L(g,P) + \epsilon \\
												&\leq L(f) + L(g) + \epsilon \\
												&= L(f+g) + \epsilon \\
												&= \int_{ a }^{ b } (f+g) + \epsilon.
	\end{align*}
	Again, \( \epsilon > 0    \) implies \( \int_{ a }^{ b } f + \int_{ a }^{ b }  g \leq \int_{ a }^{ b } (f+g) \). Hence, we conclude
	\[  \int_{ a }^{ b } (f+g) = \int_{ a }^{ b } f + \int_{ a }^{ b } g. \]
	
\end{proof}

\begin{proof}[Proof of (ii)]
	Let \( k \in \R  \). By our supremum and infimum properties derived back in chapter 1, we know that 
	\[  \sup \{ k f(x) : x \in [x_{k-1}, x_{k }]  \} = k \sup \{ f(x) : x \in \{ x_{k-1}, x_{k} \}  \}   \]
	and likewise, 
	\[  \inf \{ kf(x) : x \in [x_{k-1}, x_{k}] \} = k \inf \{ f(x) : x  \in [x_{k-1}, x_{k }]  \}.  \]
	Hence, we have
	\[  U(kf, P_{n}) = k U(f, P_{n}) \ \text{ and } \ L(kf, P_{n}) = k L(f, P_{n}).  \]
	Since \( f  \) is integrable on \( [a,b]  \), there exists a partition \( P_{n} \) such that 
	\begin{align*}
	    | U(kf, P_{n}) - L(kf, P_{n}) | &= k | U(f, P_{n}) - L(f, P_{n}) | \to 0.  \\
	\end{align*}
	Hence, \( kf  \) is integrable on \( [a,b] \).

\end{proof}

\begin{proof}[Proof of (iii)]
	Suppose \( m \leq f(x) \leq M  \) on \( [a,b]  \). Since \( f  \) is integrable on \( [a,b]  \), we know that \( U(f) = L(f) = \int_{ a }^{ b } f  \). Let \( P  \) be a parition of \( [a,b] \). By using the properties of the upper sums and lower sums derived back in 7.2, we know that 
	\begin{align*}
	    \int_{ a }^{ b } f = U(f) &\leq U(f,P) \\
			 &= \sum_{ k=1 }^{ n } M_{k } \Delta x_{k } \\
			 &\leq M \sum_{ k=1 }^{ n } \Delta x_{k } \\ 
			 &= M(b-a).
	\end{align*}
	Likewise, we have 
	\begin{align*}
	    \int_{ a }^{ b } f = L(f) &\geq L(f, P) \\
								  &= \sum_{ k=1 }^{ n } m_{k } \Delta x_{k } \\ 
								  &\geq m \sum_{ k=1 }^{ n } \Delta x_{k } \\
								  &= m (b-a).
	\end{align*}
	We conclude 
	\[  m(b-a) \leq \int_{ a }^{ b } f \leq M(b-a) \]
\end{proof}


\begin{proof}[Proof of (iv)]
	Suppose \( f(x) \leq g(x) \) on \( [a,b]  \). Since \( f  \) and \( g \) are integrable, we know that \( U(f) = L(f) = \int_{ a }^{ b } f  \) and \( U(g) = L(g) = \int_{ a }^{ b } g  \). Let \( \epsilon > 0  \) and let \( P  \) be a partition of \( [a,b] \). Then observe that 
	\begin{align*}
		\int_{ a }^{ b } f \leq U(f,P) &\leq U(g, P) \\
									 &< L(g,P) + \epsilon \\ 
									 &\leq \int_{ a }^{ b } g  + \epsilon \\
	\end{align*}
	Since \( \epsilon > 0 \) is arbitrary, we can conclude 
	\[  \int_{ a }^{ b } f \leq \int_{ a }^{ b } g. \]
\end{proof}

\begin{proof}[Proof of (v)]
	Suppose \( | f |  \) is integrable on \( [a,b]  \) and let \( P \) be an arbitrary partition of \( [a,b] \). Let \( \epsilon > 0 \). Then using the properties of the upper sums, we have 
	\begin{align*}
	    \Big| \int_{ a }^{ b } f  \Big| \leq | U(f,P) |
										&= \Big| \sum_{ k=1 }^{ n } M_{k } \Delta x_{k }  \Big| \\
										&\leq \sum_{ k=1 }^{ n } | M_{k } \Delta x_{k } | \\
										&= U(| f |, P) \\
										&\leq \int_{ a }^{ b } | f |   + \epsilon \\
	\end{align*}
	Since \( \epsilon > 0  \) is arbitrary, we conclude that 
	\[  \Big| \int_{ a }^{ b } f  \Big| \leq \int_{ a }^{ b } | f |. \]
\end{proof}

\begin{tcolorbox}
\begin{defn}
	If \( f  \) is integrable on the interval \( [a,b]  \), define 
	\[  \int_{ a }^{ b } f = - \int_{ a }^{b  } f. \]
	Also, for \( c \in [a,b]  \) define 
	\[  \int_{ c }^{ c } f = 0. \]
\end{defn}
\end{tcolorbox}

\subsection{Uniform Convergence and Integration}

An interesting question we can ask is that when we have a sequence of function \( (f_{n}) \) on \( [a,b]  \) where \( f_{n} \to f  \), then does 
\[  \int_{ a }^{ b } f_{n} \to \int_{ a }^{ b } f  \]
hold? Suppose \( f_{n} \to f  \) pointwise, then consider 
\[  f_{n}(x) = 
\begin{cases}
	n \ &\text{ if } 0 < x < \frac{ 1 }{ n } \\
	0 \ &\text{ if } x = 0 \ \text{ or } x \geq \frac{ 1 }{ n }
\end{cases} \]
as a counter-example. Notice that each \( f_{n} \) contains two discontinuities on \( [0,1] \) and is integrable with \( \int_{ 0 }^{ 1 } f_{n} = 1  \). For every \( x \in [0,1] \), note that \( \lim f_{n}(x) = 0  \) pointwise.  Then observe that the limit function \( 0  \) clearly integrates to \( 0 \). But this means that 
\[  \lim_{ n \to \infty  }  \int_{ a }^{ b } f_{n} \neq 0. \]
To fix this problem caused by pointwise convergence, we require the assumption of uniform convergence. 


\begin{tcolorbox}
	\begin{thm}[Integrable Limit Theorem]
	Assume that \( f_{n} \to f  \) uniformly on \( [a,b]  \) and that each \( f_{n}  \) integrable. Then, \( f \) is integrable and 
	\[  \lim_{ n \to \infty  }  \int_{ a }^{ b } f_{n} = \int_{ a }^{ b } f. \]
	\end{thm}
\end{tcolorbox}
\begin{proof} From exercise 7.2.5, we have proven that \( f  \) is integrable on \( [a,b] \). 
	Using part (v) of Theorem 7.4.2, we can make the following statement:
	\[ \Big| \int_{ a }^{ b } f_{n} - \int_{ a }^{ b } f  \Big| = \Big| \int_{ a }^{ b }  (f_{n} - f )  \Big| \leq \int_{ a }^{ b }  | f_{n} -f  |.   \]
	Since \( f_{n} \to f  \) uniformly on \( [a,b] \), we can let \( \epsilon > 0  \) such that there exists \( N \in \N \) implies 
	\[  | f_{n}(x) - f(x)  | < \frac{ \epsilon  }{ b -a  }  \ \text{ for all } n \geq N \ \text{and} \ x \in [a,b].\] Then observe that 
	\begin{align*}
	    \Big| \int_{ a }^{ b } f_{n} - \int_{ a }^{ b } f  \Big| &\leq \int_{ a }^{ b } | f_{n} - f  |   \\
																 &< \int_{ a }^{ b }  \frac{ \epsilon  }{ b -a  } \\
																 &= \epsilon.
	\end{align*}
	We conclude that 
	\[  \lim_{ n \to \infty  }  \int_{ a }^{ b } f_{n} = \int_{ a }^{ b } f. \]
\end{proof}


\subsection{Definitions and Theorems}


\begin{tcolorbox}
\begin{thm}
	Assume \( f: [a,b] \to \R   \) is bounded, and let \( c \in (a,b)  \). Then, \( f  \) is integrable on \( [a,b]  \) if and only if \( f  \) integrable on \( [a,c]  \) and \( [c,b]  \). In this case, we have 
	\[  \int_{ a }^{ b } f = \int_{ a }^{ c }  f  + \int_{ b }^{ c }  f. \]
\end{thm}
\end{tcolorbox}


\begin{tcolorbox}
\begin{thm}
	Assume \( f  \) and \( g  \)  are integrable functions on the interval \( [a,b]  \).
	\begin{enumerate}
		\item[(i)] The function \( f + g  \) is integrable on \( [a,b]  \) with \( \int_{ a }^{ b } (f + g) = \int_{ a }^{ b }  f + \int_{ a }^{ b } g  \).
		\item[(ii)] For \( k \in \R  \), the function \( kf  \) is integrable with \( \int_{ a }^{ b } kf  = k \int_{ a }^{ b }  f. \)
		\item[(iii)] If \( m \leq f(x) \leq M  \) on \( [a,b]  \), then \( m(b-a) \leq \int_{ a }^{ b } f \leq M(b-a) \).
		\item[(iv)] If \( f(x) \leq g(x)  \) on \( [a,b]  \), then \( \int_{ a }^{ b } f  \leq \int_{ a }^{ b } g  \).
		\item[(v)] The function \( | f |   \) is integrable and \( | \int_{ a }^{ b }  f | \leq \int_{ a }^{ b } | f |. \) 
	\end{enumerate}
\end{thm}
\end{tcolorbox}



\begin{tcolorbox}
\begin{defn}
	If \( f  \) is integrable on the interval \( [a,b]  \), define 
	\[  \int_{ a }^{ b } f = - \int_{ a }^{b  } f. \]
	Also, for \( c \in [a,b]  \) define 
	\[  \int_{ c }^{ c } f = 0. \]
\end{defn}
\end{tcolorbox}


\begin{tcolorbox}
	\begin{thm}[Integrable Limit Theorem]
	Assume that \( f_{n} \to f  \) uniformly on \( [a,b]  \) and that each \( f_{n}  \) integrable. Then, \( f \) is integrable and 
	\[  \lim_{ n \to \infty  }  \int_{ a }^{ b } f_{n} = \int_{ a }^{ b } f. \]
	\end{thm}
\end{tcolorbox}



