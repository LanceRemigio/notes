\section{Properties of the Integral}

For our first property of integration, integrating over a closed interval \( [a,b]  \) is equivalent to taking the sum of the closed intervals, say, \( [a,c]  \) and \( [c,b]  \) that make up \( [a,b]  \).

\begin{tcolorbox}
\begin{thm}
	Assume \( f: [a,b] \to \R   \) is bounded, and let \( c \in (a,b)  \). Then, \( f  \) is integrable on \( [a,b]  \) if and only if \( f  \) integrable on \( [a,c]  \) and \( [c,b]  \). In this case, we have 
	\[  \int_{ a }^{ b } f = \int_{ a }^{ c }  f  + \int_{ b }^{ c }  f. \]
\end{thm}
\end{tcolorbox}

\begin{proof}
	Suppose \( f \) is integrable on \( [a,b]  \). Then for every \( \epsilon > 0   \), there exists a partition \( P  \) such that 
	\[  U(f,P) - L(f,P) < \epsilon. \]
	Since the refinement of a paritition might cause the upper and lower sums to be closet together, all we need to do is simply add the point \( c  \) to \( P  \) if it does not already exists. Then, letting \( P_{1} = P \cap [a,c]  \) be a partition of \( [a,c]  \), and \( P_{2} = P \cap [c,b]  \) be a partition of \( [c,b]  \), it follows that 
	\[  U(f,P_{1}) - L(f, P_{1}) < \epsilon \ \text{ and } \ U(f, P_{2}) - L(f, P_{2}) < \epsilon. \] This implies that \( f  \) is integrable on \( [a,c]  \) and \( [c,b]  \).

	Conversely, if we are given that \( f  \) is integrable on the two smaller intervals \( [a,c]  \) and \( [c,b] \), then given an \( \epsilon > 0  \), we can create partitions \( P_{1}   \) and \( P_{2} \) of \( [a,c] \) and \( [c,b] \), respectively, such that 
	\[  U(f, P_{1}) - L(f, P_{1}) < \frac{ \epsilon  }{ 2  } \ \text{ and } U(f, P_{2}) - L(f,P_{2}) < \frac{ \epsilon  }{ 2 }.  \]
	Since \( P = P_{1} \cup P_{2}  \) produces a partition of \( [a,b]  \), we must have 
	\[ U(f, P) - L(f,P) < \epsilon.  \] Hence, \( f  \) is integrable on \( [a,b]  \).
	Now let \( P = P_{1} \cup P_{2} \) as before, we have 
	\begin{align*}
		\int_{ a }^{ b } f  \leq U(f, P) &< L(f, P)  + \epsilon  \\
										 &= L(f, P_{1}) + L(f, P_{2}) + \epsilon \\
										 &leq \int_{ a }^{ c } f  + \int_{ c }^{ b } f + \epsilon,
	\end{align*}
	This implies that \( \int_{ a }^{ b } f \leq \int_{ a }^{ c }f + \int_{ c }^{ b }  f \). To get the other inequality, observe that 
	\begin{align*}
	    \int_{ a }^{ c } f + \int_{ c }^{ b } f   &\leq U(f, P_{1}) + U(f, P_{2})  \\
												  &< L(f, P_{1}) + L(f, P_{2}) + \epsilon \\
												  &= L(f, P) + \epsilon \\
												  &\leq \int_{ a }^{ b } f + \epsilon.
	\end{align*}
	Since \( \epsilon > 0  \) is arbitrary, we must have that 
	\[  \int_{ a }^{ c } f + \int_{ c }^{ b } f \leq \int_{ a }^{ b } f, \]
	so hence, we have 
	\[  \int_{ a }^{ c } f  + \int_{ c }^{ b } f = \int_{ a }^{ b } f, \]
	as desired.
\end{proof}

Some more properties of integration is outlined in the next theorem.

\begin{tcolorbox}
\begin{thm}
	Assume \( f  \) and \( g  \)  are integrable functions on the interval \( [a,b]  \).
	\begin{enumerate}
		\item[(i)] The function \( f + g  \) is integrable on \( [a,b]  \) with \( \int_{ a }^{ b } (f + g) = \int_{ a }^{ b }  f + \int_{ a }^{ b } g  \).
		\item[(ii)] For \( k \in \R  \), the function \( kf  \) is integrable with \( \int_{ a }^{ b } kf  = k \int_{ a }^{ b }  f. \)
		\item[(iii)] If \( m \leq f(x) \leq M  \) on \( [a,b]  \), then \( m(b-a) \leq \int_{ a }^{ b } f \leq M(b-a) \).
		\item[(iv)] If \( f(x) \leq g(x)  \) on \( [a,b]  \), then \( \int_{ a }^{ b } f  \leq \int_{ a }^{ b } g  \).
		\item[(v)] The function \( | f |   \) is integrable and \( | \int_{ a }^{ b }  f | \leq \int_{ a }^{ b } | f |. \) 
	\end{enumerate}
\end{thm}
\end{tcolorbox}

\begin{proof}\label{proof:theproof}
	Let \( \epsilon > 0 \). Since \( f \) and \( g  \) are integrable on \( [a,b]  \), there exists a partition \( P_{\epsilon } \) such that 
	\[  U(f, P_{\epsilon }) - L(f, P_{\epsilon }) < \frac{ \epsilon  }{ 2 }   \]
	and 
	\[  U(g, P_{\epsilon }) - L(g, P_{\epsilon }) < \frac{ \epsilon  }{ 2 } . \]
	To show that \( (f+g)  \) is integrable on \( [a,b] \), we must show that there exists a partition \( P_{\epsilon } \)
	\[  U(f+g, P_{\epsilon }) - L(f +g, P_{\epsilon }) < \epsilon. \]
	First, observe that by the properties of the supremum and infimum, we have 
	that 
	\[  U(f+g, P_{\epsilon }) \leq U(f, P_{\epsilon }) + U(g, P_{\epsilon }) \]
	and 
	\[  L(f+g, P_{\epsilon }) \geq L(f, P_{\epsilon }) + L(g, P_{\epsilon }). \]
	Now observe that
	\begin{align*}
		U(f+g, P_{\epsilon }) - L(f+g, P_{\epsilon }) &\leq [U(f,P_{\epsilon }) + U(g, P_{\epsilon })] - [L(f,P_{\epsilon }) + L(g, P_{\epsilon })] \\
													  &= [U(f,P_{\epsilon }) - L(f,P_{\epsilon })] + [U(g, P_{\epsilon }) - L(g, P_{\epsilon })] \\
													  &< \frac{ \epsilon  }{ 2 }  + \frac{ \epsilon  }{ 2 } \\
													  &= \epsilon.
	\end{align*}
	Hence, \( (f+g)  \) integrable on \( [a,b]  \) by Theorem 7.2.8.  

	Now to show 
	\[  \int_{ a }^{ b } (f+g)  = \int_{ a }^{ b } f + \int_{ a }^{ b } g \]
	we must show 
	\[  \int_{ a }^{ b } (f+g) \leq \int_{ a }^{ b } f + \int_{ a }^{ b } g \ \text{ and } \int_{ a }^{ b } (f+g) \geq \int_{ a }^{ b } f + \int_{ a }^{ b } g  \]
	Since \( (f+g)  \) integrable on \( [a,b] \), we know that \( U(f+g) = L(f+g) = \int_{ a }^{ b } (f+g)  \). Then for any partition \( P  \), we can use the properties of the lower and upper sum derived back in section 7.2 to get 
	\begin{align*}
	    \int_{ a }^{ b } (f+g)  &\leq U(f+g, P) \\
								&< L(f+g, P) + \epsilon \\
								&\leq L(f) + L(g) + \epsilon \\
								&= \int_{ a }^{ b } f  + \int_{ a }^{ b } g + \epsilon.
	\end{align*}
	Since \( \epsilon > 0   \) is arbitrary, we have \( \int_{ a }^{ b } (f+g)  \leq \int_{ a }^{ b } f + \int_{ a }^{ b } g  \). To get the other inequality, we employ a similar process as above. Observe that, 
	\begin{align*}
	    \int_{ a }^{ b } f + \int_{ a }^{ b } g &\leq U(f,P) + U(g, P)\\
												&< L(f,P) + L(g,P) + \epsilon \\
												&\leq L(f) + L(g) + \epsilon \\
												&= L(f+g) + \epsilon \\
												&= \int_{ a }^{ b } (f+g) + \epsilon.
	\end{align*}
	Again, \( \epsilon > 0    \) implies \( \int_{ a }^{ b } f + \int_{ a }^{ b }  g \leq \int_{ a }^{ b } (f+g) \). Hence, we conclude
	\[  \int_{ a }^{ b } (f+g) = \int_{ a }^{ b } f + \int_{ a }^{ b } g. \]
	
\end{proof}

\begin{proof}[Proof of (ii)]
	Let \( k \in \R  \). By our supremum and infimum properties derived back in chapter 1, we know that 
	\[  \sup \{ k f(x) : x \in [x_{k-1}, x_{k }]  \} = k \sup \{ f(x) : x \in \{ x_{k-1}, x_{k} \}  \}   \]
	and likewise, 
	\[  \inf \{ kf(x) : x \in [x_{k-1}, x_{k}] \} = k \inf \{ f(x) : x  \in [x_{k-1}, x_{k }]  \}.  \]
	Hence, we have
	\[  U(kf, P_{n}) = k U(f, P_{n}) \ \text{ and } \ L(kf, P_{n}) = k L(f, P_{n}).  \]
	Since \( f  \) is integrable on \( [a,b]  \), there exists a partition \( P_{n} \) such that 
	\begin{align*}
	    | U(kf, P_{n}) - L(kf, P_{n}) | &= k | U(f, P_{n}) - L(f, P_{n}) | \to 0.  \\
	\end{align*}
	Hence, \( kf  \) is integrable on \( [a,b] \).

\end{proof}

\begin{proof}[Proof of (iii)]
	Suppose \( m \leq f(x) \leq M  \) on \( [a,b]  \). Since \( f  \) is integrable on \( [a,b]  \), we know that \( U(f) = L(f) = \int_{ a }^{ b } f  \). Let \( P  \) be a parition of \( [a,b] \). By using the properties of the upper sums and lower sums derived back in 7.2, we know that 
	\begin{align*}
	    \int_{ a }^{ b } f = U(f) &\leq U(f,P) \\
			 &= \sum_{ k=1 }^{ n } M_{k } \Delta x_{k } \\
			 &\leq M \sum_{ k=1 }^{ n } \Delta x_{k } \\ 
			 &= M(b-a).
	\end{align*}
	Likewise, we have 
	\begin{align*}
	    \int_{ a }^{ b } f = L(f) &\geq L(f, P) \\
								  &= \sum_{ k=1 }^{ n } m_{k } \Delta x_{k } \\ 
								  &\geq m \sum_{ k=1 }^{ n } \Delta x_{k } \\
								  &= m (b-a).
	\end{align*}
	We conclude 
	\[  m(b-a) \leq \int_{ a }^{ b } f \leq M(b-a) \]
\end{proof}


\begin{proof}[Proof of (iv)]
	Suppose \( f(x) \leq g(x) \) on \( [a,b]  \). Since \( f  \) and \( g \) are integrable, we know that \( U(f) = L(f) = \int_{ a }^{ b } f  \) and \( U(g) = L(g) = \int_{ a }^{ b } g  \). Let \( \epsilon > 0  \) and let \( P  \) be a partition of \( [a,b] \). Then observe that 
	\begin{align*}
		\int_{ a }^{ b } f \leq U(f,P) &\leq U(g, P) \\
									 &< L(g,P) + \epsilon \\ 
									 &\leq \int_{ a }^{ b } g  + \epsilon \\
	\end{align*}
	Since \( \epsilon > 0 \) is arbitrary, we can conclude 
	\[  \int_{ a }^{ b } f \leq \int_{ a }^{ b } g. \]
\end{proof}

\begin{proof}[Proof of (v)]
	Suppose \( | f |  \) is integrable on \( [a,b]  \) and let \( P \) be an arbitrary partition of \( [a,b] \). Let \( \epsilon > 0 \). Then using the properties of the upper sums, we have 
	\begin{align*}
	    \Big| \int_{ a }^{ b } f  \Big| \leq | U(f,P) |
										&= \Big| \sum_{ k=1 }^{ n } M_{k } \Delta x_{k }  \Big| \\
										&\leq \sum_{ k=1 }^{ n } | M_{k } \Delta x_{k } | \\
										&= U(| f |, P) \\
										&\leq \int_{ a }^{ b } | f |   + \epsilon \\
	\end{align*}
	Since \( \epsilon > 0  \) is arbitrary, we conclude that 
	\[  \Big| \int_{ a }^{ b } f  \Big| \leq \int_{ a }^{ b } | f |. \]
\end{proof}

\begin{tcolorbox}
\begin{defn}
	If \( f  \) is integrable on the interval \( [a,b]  \), define 
	\[  \int_{ a }^{ b } f = - \int_{ a }^{b  } f. \]
	Also, for \( c \in [a,b]  \) define 
	\[  \int_{ c }^{ c } f = 0. \]
\end{defn}
\end{tcolorbox}

\subsection{Uniform Convergence and Integration}

An interesting question we can ask is that when we have a sequence of function \( (f_{n}) \) on \( [a,b]  \) where \( f_{n} \to f  \), then does 
\[  \int_{ a }^{ b } f_{n} \to \int_{ a }^{ b } f  \]
hold? Suppose \( f_{n} \to f  \) pointwise, then consider 
\[  f_{n}(x) = 
\begin{cases}
	n \ &\text{ if } 0 < x < \frac{ 1 }{ n } \\
	0 \ &\text{ if } x = 0 \ \text{ or } x \geq \frac{ 1 }{ n }
\end{cases} \]
as a counter-example. Notice that each \( f_{n} \) contains two discontinuities on \( [0,1] \) and is integrable with \( \int_{ 0 }^{ 1 } f_{n} = 1  \). For every \( x \in [0,1] \), note that \( \lim f_{n}(x) = 0  \) pointwise.  Then observe that the limit function \( 0  \) clearly integrates to \( 0 \). But this means that 
\[  \lim_{ n \to \infty  }  \int_{ a }^{ b } f_{n} \neq 0. \]
To fix this problem caused by pointwise convergence, we require the assumption of uniform convergence. 


\begin{tcolorbox}
	\begin{thm}[Integrable Limit Theorem]
	Assume that \( f_{n} \to f  \) uniformly on \( [a,b]  \) and that each \( f_{n}  \) integrable. Then, \( f \) is integrable and 
	\[  \lim_{ n \to \infty  }  \int_{ a }^{ b } f_{n} = \int_{ a }^{ b } f. \]
	\end{thm}
\end{tcolorbox}
\begin{proof} From exercise 7.2.5, we have proven that \( f  \) is integrable on \( [a,b] \). 
	Using part (v) of Theorem 7.4.2, we can make the following statement:
	\[ \Big| \int_{ a }^{ b } f_{n} - \int_{ a }^{ b } f  \Big| = \Big| \int_{ a }^{ b }  (f_{n} - f )  \Big| \leq \int_{ a }^{ b }  | f_{n} -f  |.   \]
	Since \( f_{n} \to f  \) uniformly on \( [a,b] \), we can let \( \epsilon > 0  \) such that there exists \( N \in \N \) implies 
	\[  | f_{n}(x) - f(x)  | < \frac{ \epsilon  }{ b -a  }  \ \text{ for all } n \geq N \ \text{and} \ x \in [a,b].\] Then observe that 
	\begin{align*}
	    \Big| \int_{ a }^{ b } f_{n} - \int_{ a }^{ b } f  \Big| &\leq \int_{ a }^{ b } | f_{n} - f  |   \\
																 &< \int_{ a }^{ b }  \frac{ \epsilon  }{ b -a  } \\
																 &= \epsilon.
	\end{align*}
	We conclude that 
	\[  \lim_{ n \to \infty  }  \int_{ a }^{ b } f_{n} = \int_{ a }^{ b } f. \]
\end{proof}


\subsection{Definitions and Theorems}


\begin{tcolorbox}
\begin{thm}
	Assume \( f: [a,b] \to \R   \) is bounded, and let \( c \in (a,b)  \). Then, \( f  \) is integrable on \( [a,b]  \) if and only if \( f  \) integrable on \( [a,c]  \) and \( [c,b]  \). In this case, we have 
	\[  \int_{ a }^{ b } f = \int_{ a }^{ c }  f  + \int_{ b }^{ c }  f. \]
\end{thm}
\end{tcolorbox}


\begin{tcolorbox}
\begin{thm}
	Assume \( f  \) and \( g  \)  are integrable functions on the interval \( [a,b]  \).
	\begin{enumerate}
		\item[(i)] The function \( f + g  \) is integrable on \( [a,b]  \) with \( \int_{ a }^{ b } (f + g) = \int_{ a }^{ b }  f + \int_{ a }^{ b } g  \).
		\item[(ii)] For \( k \in \R  \), the function \( kf  \) is integrable with \( \int_{ a }^{ b } kf  = k \int_{ a }^{ b }  f. \)
		\item[(iii)] If \( m \leq f(x) \leq M  \) on \( [a,b]  \), then \( m(b-a) \leq \int_{ a }^{ b } f \leq M(b-a) \).
		\item[(iv)] If \( f(x) \leq g(x)  \) on \( [a,b]  \), then \( \int_{ a }^{ b } f  \leq \int_{ a }^{ b } g  \).
		\item[(v)] The function \( | f |   \) is integrable and \( | \int_{ a }^{ b }  f | \leq \int_{ a }^{ b } | f |. \) 
	\end{enumerate}
\end{thm}
\end{tcolorbox}



\begin{tcolorbox}
\begin{defn}
	If \( f  \) is integrable on the interval \( [a,b]  \), define 
	\[  \int_{ a }^{ b } f = - \int_{ a }^{b  } f. \]
	Also, for \( c \in [a,b]  \) define 
	\[  \int_{ c }^{ c } f = 0. \]
\end{defn}
\end{tcolorbox}


\begin{tcolorbox}
	\begin{thm}[Integrable Limit Theorem]
	Assume that \( f_{n} \to f  \) uniformly on \( [a,b]  \) and that each \( f_{n}  \) integrable. Then, \( f \) is integrable and 
	\[  \lim_{ n \to \infty  }  \int_{ a }^{ b } f_{n} = \int_{ a }^{ b } f. \]
	\end{thm}
\end{tcolorbox}



\subsection{Exercises}





\subsubsection{Exercise 7.4.1} Let \( f  \) be a bounded function on a set \( A  \), and set 
\[  M = \sup \{ f(x) : x \in A  \},  \ m = \inf \{ f(x) : x \in A \}, \]
\[  M' = \sup \{ | f(x)  | : x \in A  \} \ \text{ and } \ m' = \inf \{ | f(x)  | : x \in A \}. \]

\begin{enumerate}
    \item[(a)] Show that \( M - m \geq M' - m' \).
		\begin{proof}
		Observe that \( f(x) \leq M' \leq M \) and likewise \( m \leq m'  \). Then we have  
		\[  -m \geq -m' \iff M' - m \geq M' - m'. \tag{1}\]
		Since \( M' \leq M \), (1) implies 
		\[  M - m \geq M' - m'. \]
		\end{proof}
	\item[(b)] Show that if \( f  \) is integrable on the interval \( [a,b]  \), then \( | f |  \) is also integrable on this interval.
		\begin{proof}
			Suppose \( f \) is integrable on \( [a,b] \). Let \( \epsilon > 0  \). Then there exists a partition \(  P_{\epsilon } \) such that 
			\[  U(f,P_{\epsilon } ) - L(f, P_{\epsilon }) = \sum_{ k=1 }^{ n } [M_{k } - m_{k }] \Delta x_{k} < \epsilon. \]
			By using part (a), we know that 
			\[  \sum_{ k=1 }^{ n } [M'_{k } - m'_{k }] \Delta x_{k } \leq \sum_{ k=1 }^{ n } [ M_{k } - m_{k } ] \Delta x_{k }. \]
			Then using the same partition \( P_{\epsilon } \) that we found, we have 
			\[ U(| f | , P_{\epsilon } ) - L(| f | , P_{\epsilon } ) \leq U(f , P_{\epsilon }) - L( f  , P_{\epsilon }) < \epsilon. \] Hence, we conclude that \( | f |  \) is an integrable function on \( [a,b]  \). 
		\end{proof}
	\item[(b)] Provide the details for the argument that in this case we have \( | \int_{ a }^{ b } f  | \leq \int_{ a }^{ b } | f | \).
		\begin{proof}
			Since \( | f |  \) integrable, we know that \( U(| f | ) = L(| f | ) = \int_{ a }^{ b } | f |    \). Likewise \( f \) being integrable implies \( U( f ) = L(f) = \int_{ a }^{ b } f   \). Let \( \epsilon > 0 \) and let \( P \) be a partition of \( [a,b]  \). Using the properties of the upper and lower integral, we must have 
		\begin{align*}
		    \Big| \int_{ a }^{ b } f  \Big| = | U(f)  |   
											&\leq | U(f,P)  | \\
											&< \Big| L(f,P) + \epsilon  \Big| \\  
											&\leq | L(f,P)  | + \epsilon \\
											&\leq L(| f | , P) + \epsilon \\
											&\leq \int_{ a }^{ b } | f | + \epsilon. 
		\end{align*}
		Since \( \epsilon > 0  \) is arbitrary, we must have \( | \int_{ a }^{ b } f  |  \leq \int_{ a }^{ b } | f |   \). 
		\end{proof}
\end{enumerate}


\subsubsection{Exercies 7.4.2}  
\begin{enumerate}
    \item[(a)] Let \( g(x) = x^{3}  \), and classify each of the following as positive, negative, or zero.
		\[  \text{ (i) } \int_{ 0 }^{ -1 } g + \int_{ 0 }^{ 1 } g \ \ \   \text{ (ii) } \int_{ 1 }^{ 0 } g + \int_{ 0 }^{ 1 }  g \ \ \ \text{ (iii) } \int_{ 1 }^{ -2 } g + \int_{ 0 }^{ 1 } g.  \]
		\begin{proof}[Solution]
			(i) zero, (ii) zero, (iii) positive 
		\end{proof}
	\item[(b)] Show that if \( b \leq a \leq c  \) and \( f  \) is integrable on the interval \( [b,c]  \), then it is still the case that \( \int_{ a }^{ b } f = \int_{ a }^{ c } f + \int_{ b }^{ c } f  \).
		\begin{proof}
			Since \( f  \) is integrable on the interval \( [b,c]  \), we have 
			\[  \int_{ b }^{ c } f   = \int_{ b }^{ a } f  + \int_{ a }^{ c } f. \tag{1} \]
			Rearranging (1), we have 
			\[  - \int_{ b }^{ a } f = \int_{ a }^{ c } f  - \int_{ b }^{ c } f.\] By using Definition 7.4.1, we know that 
			\[  - \int_{ b }^{ a } f = \int_{ a }^{ b } f  \ \text{ and }  -\int_{ b }^{ c } f = \int_{ c }^{ b } f.  \]
			Hence, we conclude that 
			\[ \int_{ a }^{ b } f = \int_{ a }^{ c } f + \int_{ b }^{ c } f  \]

		\end{proof}
\end{enumerate}

\subsubsection{Exercise 7.4.3} Decide which of the following conjectures is true and supply a short proof. For those that are not true, give a counter-example.
\begin{enumerate}
	\item[(a)] If \( | f |  \) is integrable on \( [a,b]  \), then \( f  \) is also integrable on this set. 
		\begin{proof}[Solution]
		Define 
		\[  f(x) = 
		\begin{cases}
			1  \ \text{ for } x \in \Q \\
			-1 \ \text{ for } x \notin \Q.
		\end{cases} \]
		Notice that \( | f | \) is integrable, but not \( f \). 
		\end{proof}
	\item[(b)] Assume \( g  \) is integrable and \( g(x) \geq 0  \) on \( [a,b]  \). If \( g(x) > 0  \) for an infinite number of points \( x \in [a,b]  \), then \( \int_{ a }^{ b } g > 0  \).
		\begin{proof}[Solution]
		We can use Thomae's function in the last section 
		\[  g(x) = 
		\begin{cases}
			1 \ &\text{ if } x = 0 \\ 
			1 / n \ &\text{ if } x = m/n \in \Q \setminus \{ 0 \} \ \text{ is in lowest terms with } n > 0 \\
			0 \ &\text{ if } x \notin \Q.
		\end{cases} \]
		We see that \( g(x) \geq 0  \)  and \( g(x) > 0  \) for an infinite number of points \( x \in [a,b]  \), but \( \int_{ a }^{ b } g = 0  \). 
		\end{proof}
	\item[(c)] If \( g  \) is continuous on \( [a,b]  \) and \( g(x) \geq 0  \) with \( g(y_{0}) > 0  \) for at least one point \( y_{0} \in [a,b]  \), then \( \int_{ a }^{ b } g > 0. \)
		\begin{proof}
			Since \( g  \) is continuous on \( [a,b]  \) and \( g(x) \geq 0  \), we know that \( g  \) must be integrable on \( [a,b]  \). Furthermore, \( g  \) reaches its maximum and minimum on \( [a,b]  \) since \( [a,b]  \) is a compact interval. Hence, there exists at least one point \( y_{0} \) such that \( g(y_{0}) \) is the minimum of \( g  \) on \( [a,b] \). Since \( g  \) is integrable on \( [a,b]  \), we have that 
			\begin{align*}
				\int_{ a }^{ b } g  = L(g) &\geq L(g,P) \\
										   &= \sum_{ k=1 }^{ n } m_{k } \Delta x_{k } \\
											&\geq \sum_{ k=1 }^{ n } g(y_{0}) \Delta x_{k } \\ 
											&> 0. 
			\end{align*}
			Hence, we conclude that \( \int_{ a }^{ b } g > 0 \).
		\end{proof}
\end{enumerate}



\subsubsection{Exercise 7.4.4} Show that if \( f(x) > 0  \) for all \( x \in [a,b]  \) and \( f  \) is integrable, then \( \int_{ a }^{ b } f > 0 . \)
\begin{proof}
	Let \( x \in [a,b]  \). Since \( f \) is integrable on \( [a,b]  \), we have \( U(f) = L(f) = \int_{ a }^{ b } f   \). Let \( P \) be an arbitrary partition of \( [a,b]  \). Since \( f(x) > 0  \), we have
	\begin{align*}
		\int_{ a }^{ b } f = L(f) \geq L(f, P)  
								  = \sum_{ k=1 }^{ n } m_{k } \Delta x_{k }  > 0 \\
	\end{align*}
	Hence, \( \int_{ a }^{ b } f > 0  \).
\end{proof}

\subsubsection{Exercise 7.4.5} Let \( f  \) and \( g  \) be integrable functions on \( [a,b] \). 
\begin{enumerate}
	\item[(a)] Show that if \( P  \) is any partition of \( [a,b]  \), then 
		\[  U(f+g, P ) \leq U(f,P) + U(g,P) \tag{1}. \]
		Provide a specific example where the inequality is strict. What does the corresponding inequality for lower sums look like? 

		\begin{proof}
		In exercise 1.3.6, we proved for any two sets \( A,B \neq \emptyset  \), we have 
		\[  \sup(A + B) \leq \sup A + \sup B. \]
		In this context, we have 
		\[ A =  \{ f(x) : x \in [x_{k-1}, x_{k}] \} \ \text{and} \ B =  \{ g(x) : x \in [x_{k-1}, x_{k }] \} \]
		with 
		\[  A + B = \{ f(x) + g(x) : x \in [x_{k-1} , x_{k }] \} . \]
		Let \( P  \) be any partition of \( [a,b]  \). Then observe that 
		\begin{align*}
			U(f+g, P) &=\sum_{ k=1 }^{ n } \sup_{x \in [x_{k-1}, x_{k }]}(f+g)(x) \Delta x_{k } \\
					  &\leq \sum_{ k=1 }^{ n } \sup_{x \in [x_{k-1}, x_{k }]}f(x) \Delta x_{k}  + \sum_{ k=1 }^{ n } \sup_{x \in [x_{k-1}, x_{k}]} g(x) \Delta x_{k } \\  
					  &= U(f,P) + U(g,P).
		\end{align*}
		We have strict inequality whenever \( f(x) > g(x)  \) and the corresponding inequality to (1) is 
		\[  L(f+g,P) \geq L(f, P) + L(g, P). \]
		\end{proof}
	\item[(b)] Review the proof of Theorem 7.4.2 (ii), and provide an argument for part (i) of this theorem.
		\begin{proof}
			See part (i) of Theorem 7.4.2 in the section notes.
		\end{proof}
\end{enumerate}
\subsubsection{Exercise 7.4.6} Although not part of Theorem 7.4.2, it is true that the porduct of integrable functions is integrable. Provide the details for each step in the following proof of this fact:
\begin{enumerate}
	\item[(a)] If \( f  \) satisfies \( | f(x)  | \leq M  \) on \( [a,b]  \), show 
		\[  | (f(x))^{2} + (f(y))^{2} |  \leq 2M | f(x) - f(y) |. \]
		\begin{proof}
			Let \( x,y \in [a,b]  \). Since \( f  \) satisfies \( | f(x)  | \leq M  \) and \( | f(y) | \leq M  \), we have 
			\[  | f(x) + f(y)  | \leq | f(x)  | + | f(y)  | \leq 2M. \]
	Multiplying the inequality above by \( | f(x) - f(y)  |  \) yields 
	\[  | (f(x))^{2} - (f(y))^{2} | \leq 2M | f(x) - f(y) |. \]
		\end{proof}
	\item[(b)] Prove that if \( f  \) is integrable on \( [a,b]  \), then so is \( f^{2} \). 
		\begin{proof}
			Let \( \epsilon > 0 \). Since \( f  \) is integrable on \( [a,b]  \), there exists a partition \( P_{\epsilon } \) of \( [a,b] \) such that 
			\[ U(f, P_{\epsilon }) - L(f, P_{\epsilon }) < \frac{ \epsilon  }{ 2M }. \]
			Then using the same partition \( P_{\epsilon } \) and using part (a), we have that  
			\begin{align*}
				U(f^{2}, P_{\epsilon }) - L(f^{2}, P_{\epsilon }) &= \sum_{ k=1 }^{ n } [ (M'_{k})^{2} - (m'_{k })^{2}] \Delta x_{k } \\
																  &\leq 2M \sum_{ k=1 }^{ n } [M'_{k } - m'_{k}] \Delta x_{k } \\
																  &< 2M \cdot \frac{ \epsilon  }{ 2M } = \epsilon. 
			\end{align*}
			Hence, \( f^{2} \) is also integrable on \( [a,b]  \).
		\end{proof}
	\item[(c)] Now show that if \( f  \) and \( g \) are integrable, then \( fg \) is integrable. (Consider \( (f+g)^{2} \)).
		\begin{proof}
		Observe that 
		\[  (f+g)^{2} = f^{2} + 2fg + g^{2} \]
		and solving for \( fg \) yields 
		\[  fg = \frac{ 1 }{ 2 }  [ (f+g)^{2} - (f^2 + g^{2})]. \]
		Since \( f  \) and \( g  \) are integrable, we know by part (i) of Theorem 7.4.2 that \( f + g  \) is integrable.  Let \( h = f +g   \). Then by part (b), we have that \( h^{2} \) is integrable as well as \( f^{2} \) and \( g^{2} \). By using part (i) and (ii) of Theorem 7.4.2, we find that \( fg \) is integrable. 
		\end{proof}
\end{enumerate}

\subsubsection{Exercise 7.4.8} For each \( n \in \N  \), let 
\[  h_{n}(x) = 
\begin{cases}
	1 / 2^{n} \ &\text{ if } 1/2^{n} < x \leq 1 \\
	0 \ &\text{ if } 0 \leq x \leq 1 / 2^{n},
\end{cases} \]
and set \( H(x) = \sum_{ n=1 }^{ \infty  } h_{n}(x)  \). Show that \( H  \) is integrable and compute \( \int_{ 0 }^{ 1 } H  \).
\begin{proof}
	Notice that \( h_{n}  \) is a sequence of continuous functions on the set \( [0,1] \setminus \{  1 / 2^{n}\}  \) and that \( \sum_{ n=1 }^{ \infty  } h_{n} = 0  \) uniformly.  By the Integrability Limit Theorem, we conclude that \( H  \) must be integrale and that \( \int_{ 0 }^{ 1 } H(x) = 0  \). 
\end{proof}

\subsubsection{Exercise 7.4.9} Let \( g_{n}  \) and \( g  \) be uniformly bounded on \( [0,1]  \), meaning that there exists a single \( M > 0  \) satisfying \( | g(x) | \leq M  \) and \( | g_{n}(x)  | \leq M  \) for all \( n \in \N  \) and \( x \in [0,1] \). Assume \( g_{n} \to g  \) pointwise on \( [0,1]  \) and uniformly on any set of the form \( [0, \alpha]  \), where \( 0 < \alpha < 1  \). 

If all the functions are integrable, show that \( \lim_{ n \to \infty  }  \int_{ 0 }^{ 1 } g_{n} = \int_{ 0 }^{ 1 } g  \).
\begin{proof}
Let \( \epsilon > 0  \). Our goal is to show  
\[  \lim_{ n \to \infty  } \int_{ 0 }^{ 1 } g_{n} = \int_{ 0 }^{ 1 }  g \]
which can be done by finding an \( N \in \N  \) such that for any \(  n \geq N  \), we have 
\[  \Big| \int_{ 0 }^{ 1 } g_{n} - \int_{ 0 }^{ 1 }  g \Big| < \epsilon. \]
Suppose \( g_{n} \to g  \) pointwise on \( [0,1]  \) and uniformly on any set of the form \( [0, \alpha] \), where \( 0 < \alpha < 1  \). Looking at \( g  \) on \( [0, \alpha] \), the uniform convergence \( g_{n} \to g  \) imply that there exists an \(  N_{1} \in \N  \) such that for any \( n \geq N_{1}  \) and \( x \in [0, \alpha ] \), we have that 
\[  | g_{n} - g  | < \frac{ \epsilon  }{  2 \alpha }.  \] Utilizing the integrability of both \( g_{n} \) and \( g  \) on \( [0,1] \), we can also state that \( g_{n} - g  \) is integrable by Exercise 7.4.1 and hence, \( | g_{n} -g  |  \). We can use the triangle inequality and Theorem 7.4.1, to write the following:
\begin{align*}
    \Big| \int_{ 0 }^{ 1 } g_{n} - \int_{ 0 }^{ 1 } g  \Big| &= \Big| \int_{ 0 }^{ 1 } g_{n} - g   \Big|  \\
															 &\leq \int_{ 0 }^{ 1 } | g_{n} - g  |  \\
															 &= \int_{ 0 }^{ \alpha } | g_{n} - g |  + \int_{ \alpha  }^{ 1 }  | g_{n} - g  |. \\ 
\end{align*}
We can easily make the first term small by finding \( N_{1} \in \N  \) such that for any \( n \geq N_1 \), we have 
\[  \int_{ 0 }^{ \alpha  } | g_{n} -g  |  < \frac{ \epsilon  }{ 2 \alpha  } \cdot \alpha = \frac{ \epsilon  }{ 2 } . \] To make the second term small, we can utilize the integrability of \( | g_{n} -g  |  \) on \( [0,1] \) to state that for any partition \( P \) of \( [\alpha,1]  \), we have 
\begin{align*}
    \int_{ \alpha  }^{ 1 } | g_{n} -g  |  = U(| g_{n} - g  | ) 
										  &\leq U( | g_{n} -g  |, P) \\
										  &= \sum_{ k=1 }^{ n } \sup_{x \in [x_{k-1}, x_k]} | g_{n} - g  | \cdot \Delta x_{k }.\\
\end{align*}
Since \( g_{n}  \) and \( g  \) are uniformly bounded by a single \( M > 0  \) satisying \( | g(x)  |  \leq M  \) and \( | g_{n}(x) | \leq M  \), we know that 
\[  \sup_{x \in [x_{k-1}, x_{k }]} | g_{n} - g  | \leq 2M.\] Furthermore, utilizing pointwise convergence of \( g_{n} \to g  \) on \( [\alpha, 1 ] \), we can say there exists \( N_{2} \in \N  \) such that for any \( n \geq N_{2} \)
\[ \sup_{x \in [\alpha, x_{1}]} | g_{1} - g  | < \frac{ \epsilon  }{ 4 (x_{1} - \alpha) } .  \]
Then we have 
\begin{align*}
	\int_{ \alpha }^{ 1 }  | g_{n} -g  |  &\leq \sup_{x \in [\alpha, x_{1}]} | g_{1} - g  | (x_{1} - \alpha) + \sum_{ k=1 }^{ n } \sup_{x\in [x_{k-1},x_{k }]} | g_{n} -g  | \Delta x_{k }  \\
										  &< \frac{ \epsilon  }{ 4 ( x_{1} - \alpha) } \cdot (x_{1} - \alpha) + 2M \cdot \frac{ \epsilon  }{ 8M  } \\ 
										  &= \frac{ \epsilon  }{ 2 }.
\end{align*}
Letting \( N = \max \{ N_{1}, N_{2} \}  \), assuming that \( n \geq N   \), we have that 
\begin{align*}
    \Big| \int_{ 0 }^{ 1 } g_{n} - \int_{ 0 }^{ 1 } g  \Big| &= \Big| \int_{ 0 }^{ 1 } g_{n} - g   \Big|  \\
															 &\leq \int_{ 0 }^{ 1 } | g_{n} - g  |  \\
															 &= \int_{ 0 }^{ \alpha } | g_{n} - g |  + \int_{ \alpha  }^{ 1 }  | g_{n} - g  | \\ 
															 &< \frac{ \epsilon  }{ 2 \alpha } \cdot \alpha + \frac{ \epsilon  }{ 2  } \\
															 &= \epsilon.
\end{align*}
Hence, we conclude that
\[  \lim_{ n \to \infty  }  \int_{ 0 }^{ 1 } g  = \int_{ 0 }^{ 1 } g.  \]
\end{proof}


\subsubsection{Exercise 7.4.10} Assume \( g  \) is integrable on \( [0,1]  \) and continuous at \( 0  \). Show 
\[  \lim_{ n \to \infty  } \int_{ 0 }^{ 1 }  g(x^{n}) \ dx = g(0). \]
\begin{proof}[Proof. (Check later)]
Let \( \epsilon > 0 \). Since \( g \) is continuous at \( 0 \), we know that \( g  \) is also integrable at \( 0  \) which means that \( \int_{ 0 }^{ 1 } g(0) = g(0). \) Furthermore, the continuity of \( g  \) at \( 0 \) implies that there exists \( \delta > 0  \) and \( N \in \N  \) such that whenever \( | x^{n} | < \delta  \) and \( n \geq N  \), we have 
\[  | g(x^{n}) - g(0)  | < \epsilon. \] Then observe that 
\begin{align*}
    \Big| \int_{ 0 }^{ 1 }  g(x^{n}) - g(0)  \Big| &= \Big| \int_{ 0 }^{ 1 } g(x^{n}) - \int_{ 0 }^{ 1 }  g(0) \Big|  \\
												   &=\Big| \int_{ 0 }^{ 1 }  g(x^{n}) - g(0) \Big|  \\
												   &\leq \int_{ 0 }^{ 1 }  | g(x^{n}) - g(0)  | \\
												   &< \epsilon \int_{ 0 }^{ 1 }  = \epsilon.
\end{align*}
\end{proof}



