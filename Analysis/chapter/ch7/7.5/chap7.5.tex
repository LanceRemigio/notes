\section{The Fundamental Theorem of Calculus}

A quick review of the topics we have learned so far tells us that there two main themes:
\begin{enumerate}
    \item[(i)] The derivative's purpose is to find the slopes of tangent lines at a single point which can be calculated using the functional limits of quotients.
	\item[(ii)] The integra's purpose is to calculate areas under nonconstant functions using supremums and infimums of finite sums.
\end{enumerate}

The Fundamental Theorem of Calculus links these two concepts together via an inverse relationship between the two operations which comes in two statements:
\begin{enumerate}
    \item[(i)] An antiderivative can be used to evaluate an integral over some closed interval.
	\item[(ii)] A continuous function is the derivative of its indefinite integral.
\end{enumerate}

\begin{tcolorbox}
	\begin{thm}[Fundamental Theorem of Calculus]
	\begin{enumerate}
		\item[(i)] If \( f: [a,b] \to \R  \) is integrable, and \( F : [a,b] \to \R  \) satisfies \( F'(x) = f(x)  \) for all \( x \in [a,b]  \), then 
			\[  \int_{ a }^{ b } f = F(b) - F(a). \]
		\item[(ii)] Let \( g: [a,b] \to \R  \) be integrable, and for \( x \in [a,b]  \), define 
			\[  G(x) = \int_{ a }^{ x } g. \]
			Then \( G  \) is continuous on \( [a,b]  \). If \( g  \) is continuous at some point \( c \in [a,b]  \), then \( G  \) is differentiable at \( c  \) and \( G'(c) = g(c)  \).
	\end{enumerate}
	\end{thm}
\end{tcolorbox}
\begin{proof}
\begin{enumerate}
	\item[(i)] Let \( P  \) be a partition of \( [a,b]  \). Since \( f(x) = F'(x)  \), we can apply the Mean Value Theorem. Then there exists a \( t_{k } \in [x_{k-1}, x_{k }] \) such that 
		\[  F'(t_{k } ) =  \frac{ F(x_{k } ) - F(x_{k-1})}{  x_{k} - x_{k-1} }\] and thus 
		\begin{align*}
		    F(x_{k } ) - F(x_{k-1}) &= F'(t_{ k }) ( x_{k } - x_{k-1})  \\
									&= f(t_{k }) (x_{k} - x_{k-1}). \\
		\end{align*}
		Now consider the upper and lower sums \( U(f,P)  \) and \( L(f,P)  \). Since \( m_{k } \leq f(t_{k }) \leq M_{k }   \) (where \( m_{k }  \) is the infimum on \( [ x_{k-1}, x_{k }]  \) and \( M_{k }  \) is the supremum), it follows that 
		\[  L(f,P) \leq \sum_{ k=1 }^{ n } [F(x_{k }) - F(x_{k-1})] \leq U(f,P).\]
		Since the term in the middle telescopes, we have that 
		\[  \sum_{ k=1 }^{ n } [F(x_{k}) - F(x_{k-1})] = F(b) - F(a),  \]
		which is \textit{independent} of the partition \( P  \), and hence it follows that 
		\[  L(f) \leq F(b) - F(a) \leq U(f). \]
		Hence, we have 
		\[  \int_{ a }^{ b } f = F(b) - F(a). \]
	\item[(ii)] Let \( \epsilon > 0 \). Then choose \( \delta = \frac{ \epsilon  }{ M  }   \) such that whenever \( | x - c  | < \delta \) By the way \( G  \) is defined above, and the fact that \( g  \) is a Riemann Integrable function,  we know that
		\begin{align*}
		    \Big| G(x) - G(c)  \Big| &= \Big| \int_{ x }^{ a } g - \int_{ c }^{ a } g  \Big|  \\
									 &= \Big| \int_{ c }^{ x } g  \Big| \\
									 &\leq \int_{ c }^{ x } | g |  \\
									 &\leq M (x - c )  \\
									 &< M \cdot \frac{ \epsilon  }{ M  } = \epsilon.
		\end{align*}
		Hence, \( G  \) is a continuous function on \( [a,b]  \). Now we would like to show that \( G  \) is a differentiable function and that \( G'(x) = g(x)  \). Since \( g  \) is continuous at some point \( c \in [a,b]  \), we know that 
		\[  | g(t) - g(c)  | < \epsilon. \] Letting \( \epsilon > 0   \) once again, our goal is to show that there exists a \( \delta > 0   \) such that whenever \(0  < | x  -c  |  < \delta \), we have
		\[  \Big| \frac{ G(x) - G(c)  }{ x  -c  } - g(c)  \Big| = \Big| \frac{ 1 }{ x - c  } \int_{ c }^{ x } g(t) \ dt - g(c) \Big|  < \epsilon. \]
	Observe that we can cleverly write \( g(c)  \) as follows	
	\[  g(c) = \frac{ 1 }{ x -c  } \int_{ c }^{ x } g(c) \  dt. \] Then observe that 
	\begin{align*}
	    \Big| \frac{ 1 }{ x -c  } \int_{ c }^{ x } g(t) \  dt - g(c)  \Big| &= \Big| \frac{ 1 }{ x -c  } \int_{ c }^{ x } g(t)  \ dt - \frac{ 1 }{ x -c  } \int_{ c }^{ x } g(c) \   dt \Big|  \\
																			&= \Big| \frac{ 1 }{ x -c  } \int_{ c }^{ x } [ g(t) - g(c) ] \ dt \Big| \\
																			&\leq \frac{ 1 }{ x -c  } \int_{ c }^{ x } | g(t) - g(c) | \  dt \\
																			&< \frac{ \epsilon  }{ x- c  } \cdot \int_{ c }^{ x } dt  \\ 
																			&= \epsilon.
	\end{align*}
	This concludes that, indeed, 
	\[  \lim_{ x \to c } \frac{ 1 }{x -c  } \int_{ c }^{ x } g(t) \ dt = G'(c) = g(c) \]
\end{enumerate}
\end{proof}

\subsection{Exercises}


\subsubsection{Exercise 7.5.4} Show that if \( f: [a,b] \to \R  \) is continuous and \( \int_{ a }^{ x } f = 0   \) for all \( x \in [a,b]  \), then \( f(x) = 0  \) everywhere on \( [a,b]  \). Provide an example to show that this conclusion does not follow if \( f  \) is not continuous.
\begin{proof}
	Suppose for sake of contradiction that \( f(x) > 0  \) (the proof is similar for \( f(x) < 0  \)). Since \( f \) is continuous, we know that \( f  \) must also be integrable. Hence, we know that \( \int_{ a }^{ b } f   > 0  \) by Exercise 7.4.4. But this contradicts our assumption that \( \int_{ a }^{ b } f   = 0  \). Hence, \( f(x) = 0  \) for all \( x \in [a,b] \).
\end{proof}




\subsubsection{Exercise 7.5.5} The Fundamental Theorem of Calculus can be used to supply a shorter argument for Theorem 6.3.1 under the additional assumption that the sequence of derivatives is continuous.

Assume \( f_{n} \to f  \) pointwise and \( f'_{n} \to g  \) uniformly on \( [a,b]  \). Assuming each \( f'_{n}  \) is continuous, we can apply Theorem 7.5.1 (i) to get 
\[  \int_{ a }^{ x } f'_{n}   = f_{n}(x) - f_{n}(a)  \]
for all \( x \in [a,b]  \). Show that \( g(x) = f'(x) \).


\begin{proof}
Our goal is to show that 
\[  f'(c) = \lim_{ x \to c }  \frac{ f(x) - f(c) }{ x -c  } = g(c). \] Let \( \epsilon > 0  \). Since \( f'_{n} \to g  \) uniformly and each \( f'_{n}  \) continuous, we must have 
\[  \lim_{ n \to \infty  }  \int_{ x }^{ c } f'_{n}   = \int_{ x }^{ c } g = f(x) - f(c)  \]
Then by the same setup as in the original proof of Theorem 6.3.1, we have 
\begin{align*}
	\Big| \frac{ f(x) - f(c) }{ x - c   } - g(c) \Big| &\leq \Big| \frac{ f(x) - f(c)  }{ x -c  } - \frac{ f_{n}(x) - f_{n}(c)  }{ x - c  }  \Big| \\  &+ \Big|  \frac{ f_{n}(x) - f_{n}(c)  }{ x -c  } - f_n'(c)  \Big| + | f'_{n}(c) - g'(c)  |   \\
\end{align*}
\end{proof}





\subsubsection{Exercise 7.5.6 (Integration-by-parts).} 
\begin{enumerate}
	\item[(a)] Assume \( h(x)  \) and \( k(x)  \) have continuous derivatives on \( [a,b]  \) and derive the familiar integration-by-parts formula 
		\[  \int_{ a }^{ b } h(t)k'(t) \   dt = h(b)k(b) - h(a)k(a) - \int_{ a }^{ b } h'(t)k(t) \    dt. \tag{1} \]
		\begin{proof}[Solution]
		Let us rewrite (1) by adding the second term on the right side of (1) to both sides so that we may show that 
		\[  \int_{ a }^{ b } [h(t)k'(t) + h'(t)k'(t)] \    dt = h(b)k(b) - h(a)k(a) \tag{2}. \]
	Since \( h(x)  \) and \( k(x)  \) have continuous derivatives, it follows that \( h'(x)  \) and \( k'(x)  \) are both integrable. Furthermore, the differentiability of both \( h  \) and \( k  \) imply that they are continuous and thus \( h \) and \( k  \) are integrable. Then by exercise 7.4.6, we now that \( h k'  \) and \( h'k  \) are integrable and thus their sum \( hk' + h'k  \) is also integrable. But observe that 
	\[  (hk)' = hk' + h'k  \]
	by the product rule which means that \( (hk)' \) is integrable. By part (i) of theorem 7.5.1, we have 
	\[  \int_{ a }^{ b } (hk)'(t) \ dt = (hk)(b) - (hk)(a) = h(b)k(b) - h(a)k(a) \tag{3}. \]
	Hence, we can rearrange (3) to 
	\[  \int_{ a }^{ b } h(t)k'(t) \   dt = h(b)k(b) - h(a)k(a) - \int_{ a }^{ b } h'(t)k(t) \    dt. \]
		\end{proof}
	\item[(b)] Explain how the result in Exercise 7.4.6 can be used to slightly weaken the hypothesis in part (a).
		\begin{proof}[Solution]
		
		\end{proof}
\end{enumerate}
 
\subsubsection{Exercise 7.5.7} Use part (ii) of Theorem 7.5.1 to construct another proof of part (i) of Theorem 7.5.1 under the stronger hypothesis that \( f  \) is continuous. (To get started, set \( G(x) = \int_{ a }^{ x }  f   \).)
\begin{proof}
\end{proof}



Hello
