\section{The Fundamental Theorem of Calculus}

A quick review of the topics we have learned so far tells us that there two main themes:
\begin{enumerate}
    \item[(i)] The derivative's purpose is to find the slopes of tangent lines at a single point which can be calculated using the functional limits of quotients.
	\item[(ii)] The integra's purpose is to calculate areas under nonconstant functions using supremums and infimums of finite sums.
\end{enumerate}

The Fundamental Theorem of Calculus links these two concepts together via an inverse relationship between the two operations which comes in two statements:
\begin{enumerate}
    \item[(i)] An antiderivative can be used to evaluate an integral over some closed interval.
	\item[(ii)] A continuous function is the derivative of its indefinite integral.
\end{enumerate}

\begin{tcolorbox}
	\begin{thm}[Fundamental Theorem of Calculus]
	\begin{enumerate}
		\item[(i)] If \( f: [a,b] \to \R  \) is integrable, and \( F : [a,b] \to \R  \) satisfies \( F'(x) = f(x)  \) for all \( x \in [a,b]  \), then 
			\[  \int_{ a }^{ b } f = F(b) - F(a). \]
		\item[(ii)] Let \( g: [a,b] \to \R  \) be integrable, and for \( x \in [a,b]  \), define 
			\[  G(x) = \int_{ a }^{ x } g. \]
			Then \( G  \) is continuous on \( [a,b]  \). If \( g  \) is continuous at some point \( c \in [a,b]  \), then \( G  \) is differentiable at \( c  \) and \( G'(c) = g(c)  \).
	\end{enumerate}
	\end{thm}
\end{tcolorbox}
\begin{proof}

\end{proof}



