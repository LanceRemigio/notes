\section{The Definition of the Riemann Integral}

Before we have \textit{Riemann sums}, we need to construct \textit{upper sums}  and \textit{lower sums} using the notion of the supremum and infimum. In this section, let us assume that \( f  \) is defined on a closed interval \( [a,b]  \) where \( f  \) is bounded my some \( M > 0  \) on this interval; that is, \( | f(x)  | \leq M  \) for all \( x \in [a,b] \). 

\subsection{Partitions, Upper Sums, and Lower Sums}

\begin{tcolorbox}
\begin{defn}
	A \textit{partition} \( P  \) of \( [a,b]  \) is a finite set of points from \( [a,b]  \) that includes both \( a  \) and \( b  \). The notational convention is to always list the points of a partition \( P = \{ x_{0}, x_{1}, x_{2}, \dots, x_{n} \}  \) in increasing order; thus, \[  a = x_{0} < x_{1} < x_{2} < \dotsb < x_{n} = b. \]
	For each subinterval \( [x_{k-1}, x_{k} ] \) of \( P  \), let 
	\[  m_{k } = \inf \{ f(x) : x \in [x_{k-1} , x_{k } ] \} \ \text{ and } \  M_{k } = \sup \{ f(x) : x \in [x_{k-1}, x_{k }] \}. \]
	The \textit{lower sum} of \( f  \) respect to \( P  \) is given by 
	\[  L(f, P ) = \sum_{ k=1 }^{ n } m_{ k } ( x_{k } -  x_{ k -1 }). \]
	Likewise, we define the \textit{upper sum} of \( f  \) with respect to \( P  \) by 
	\[  U(f, P ) = \sum_{ k=1 }^{ n } M_k ( x_{k } - x_{ k -1 }). \]
\end{defn}
\end{tcolorbox}

It is clear from this definition that 
\[  U(f, P ) \leq L(f, P ).\]  This inequality holds even with respect to different partitions.

\begin{tcolorbox}
\begin{defn}
A partition \( Q  \) is a \textit{refinement} of a partition \( P  \) if \( Q  \) contains all of the points of \( P  \); that is, if \( P \subseteq Q  \).



\end{defn}
\end{tcolorbox}

\begin{tcolorbox}
\begin{lem}
If \( P \subseteq Q  \), then \( L(f,P) \leq L(f,Q)  \), and \( U(f,P ) \geq U(f,Q) \).
\end{lem}
\end{tcolorbox}

\begin{proof}
	First let us prove the inequality for the lower sums. Let \( P \subseteq Q  \). Suppose we refine \( P  \) by a adding a point \( z  \) to some subinterval \(  [x_{k-1}, x_{k }] \) of P. Then we have that 
	\begin{align*}
	    m_{k } ( x_{k } - x_{ k -1 })&= m_{  k } (x_{k } - z ) + m_{ k } (z - z _{k -1 })  \\
									 &\leq m'_{k }( x_{k } - z ) + m"_{k} (z - x_{k -1 }),
	\end{align*}
	where 
	\[  m'_{k} = \inf \{ f(x) : x \in [z, x_{k }] \} \  \ \text{ and } \  \ m"_{k} = \inf \{ f(x) : x \in [x_{k -1 }, z ] \}  \]
	are each necessarily as large or larget than \( m_k  \). We can use an induction argument to show that \(  L(f, P ) \leq L(f,Q) \). The same can be done for the upper sums.

	That is, take some point \( \ell \in [x_{k-1}, x_{k }] \) such that 
	\begin{align*}
	    M_{k } (x_{k} - x_{k -1 }) &= M_{k } (x_{ k } - \ell ) + M_{k } ( \ell - x_{k -1 }) \\
								   &\geq M_k' (x_{k } - \ell ) + M"_{k } (\ell - x_{k -1 }) \\
	\end{align*}
	\[  M'_{k} = \sup \{ f(x) : x \in [\ell, x_{k }] \} \  \ \text{ and } \  \ M"_{k}= \sup \{ f(x) : x \in [x_{k -1 }, \ell ] \}.  \]
	Using induction again to repeat the argument above, we can show that \( U(f, P) \geq U(f,Q) \).
\end{proof}

\begin{tcolorbox}
\begin{lem}
	If \( P_1  \) and \( P_2  \) are any two partitions of \( [a,b]  \), then \( L(f, P_{1}) \leq U(f, P_2) \).
\end{lem}
\end{tcolorbox}

\begin{proof}
Let \( Q = P_{1} \cup P_{2}  \) be the so-called \textit{common refinement} of \( P_{1} \) and \( P_{2} \). Because \(  P_{1} \subseteq Q  \) and \( P_{2} \subseteq Q  \), it follows that 
\[ L(f, P_{1}) \leq L(f,Q) \leq U(f,Q) \leq U(f, P_{2}).  \]
\end{proof}


\subsection{Integrability}

Another way of thinking Integrability is to think of the upper sums as an overestimate of the value of the integral and lower sums as an underestimate of the value of the integral. We can see that as the we continually refine our partitions, the upper sums become smaller and smaller and the lower sums become larger and larger until they meet at some common point in the middle. 
	Rather than thinking of this whole process as the limit of these sums, we will instead make use of the Axiom of Completeness and consider the \textit{infimum} of the upper sums and the \textit{supremum} of the lower sums.

\begin{tcolorbox}
\begin{defn}
	Let \( \mathcal{P}  \) be the collection of all possible partitions of the interval \( [a,b] \). The \textit{upper integral} of \( f \) is defined to be 
	\[  U(f) = \inf \{ U(f,P) : P \in \mathcal{P} \}.  \]
	In a similar way, define the \textit{lower integral} of \( f  \) by 
	\[  L(f) = \sup \{ L(f,P) : P \in \mathcal{P}. \}  \]
\end{defn}
\end{tcolorbox}
 The following fact is not surprising. 
\begin{tcolorbox}
\begin{lem}
	For any bounded function \( f  \) on \( [a,b]  \), it is always the case that \( U(f) \leq L(f) \).
\end{lem}
\end{tcolorbox}
Why is this not surprising you ask? It is because \( U(f) \) is the exact definition for what it means to be the least upper bound and \( L(f)  \) is the exact definition for what it means to be the greatest upper bound.

\begin{tcolorbox}
	\begin{defn}[Riemann Integrability]
	A bounded function \( f  \) defined on the interval \( [a,b] \) is \textit{Riemann-integrable} if \( U(f) = L(f) \). In this case, we define \( \int_{ a }^{ b } f   \) or \( \int_{ a }^{ b } f(x) \ dx \) to be this common value; namely, 
	\[  \int_{ a }^{ b } f  = U(f) = L(f). \]
	\end{defn}
\end{tcolorbox}

\subsection{Criteria for Integrability} 













