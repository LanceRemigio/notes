\section{Uniform Convergence and Differentiation}

We shall start this section by asking what is the effect of having a pointwise converging sequence of functions that are differentiable? It turns out that if we require the sequence of derivatives of some function to be uniformly convergent, then the limit of the sequence of derivative is the derivative of the original function.

\begin{theorem}{Differentiable Limit Theorem}{}
        Let \( f_n \to f  \) pointwise on the closed interval \( [a,b] \), and assume that each \( f_n  \) differentiable. If \( (f'_n) \) converges uniformly on \( [a,b] \) to a function \( g  \), then the function \( f  \) is differentiable and \( f' = g  \).
    \end{theorem}%

\begin{proof}
    Fix \( c \in [a,b]  \) and let \( \epsilon > 0  \). We want to show that \( f'(c)  \) exists and equals \( g(c)  \); that is, we want to show that for all \( \epsilon > 0  \), there exists \( \delta > 0  \) such that 
    \[ \Big| \frac{ f(x) - f(c) }{ x - c  } - g(c)  \Big| < \epsilon  \]
    whenever \( 0 < | x -c  | < \delta  \). We can do this by observing that for all \( n \geq N  \) and \( x \neq c  \), we can use the triangle inequality to say that 
    \begin{align*}
        \Big| \frac{ f(x) - f(c)  }{ x - c  } - g(c)  \Big| &\leq \Big| \frac{ f(x) - f(c)  }{  x- c  } - \frac{ f_n(x) - f_(c)  }{ x - c  }  \Big| \\ &+ \Big| \frac{ f_n(x) - f_n(c)  }{ x - c  } - f_n'(c)  \Big|  + | f_n'(c) - g(c)   |. \\
    \end{align*}
    We can make the last two terms "small" by having them be both less than \( \epsilon / 3  \). Since \( (f'_n) \to g  \) uniformly, we can choose an \( N_1 \in \N  \) such that for any \( n \geq N  \) that 
    \[ | f'_n(c) - g(c)  | < \frac{ \epsilon  }{ 3 }. \tag{1} \]
We can also invoke the uniform convergence of \( f'_n  \) to state that for any \( m,n \geq N_2  \) for some \( N_2 \in \N  \) that 
\[ | f_n(x) - f_m(x)  | < \frac{ \epsilon  }{ 3 }  \]
for all \( x \in [a,b] \).
    Furthermore, for \( x \neq c  \), we can make 
    \[  \Big| \frac{ f_n(x) - f_n(c)  }{ x -c  } - f_n'(c)    \Big| < \frac{ \epsilon  }{ 3  }. \tag{2}  \]
    whenever \( 0 < | x - c  | < \delta  \).
    
    \[   \]
    The first term can be made small by using the Mean Value Theorem. Fix an \( x  \) satisfying \( 0 < | x - c  | < \delta  \) and let \( m \geq N  \), and apply the Mean Value Theorem to \( f_m - f_N  \) on the interval \( [c,x] \). Suppose \( x > c  \), then there exists an \( \alpha \in (c,x) \) such that 
    \[  f'_m(\alpha) - f'_N(\alpha)  = \frac{ (f_m(x) - f_N(x)) - (f_m(c) - f_N(c) ) }{ x - c  }. \]
    Since we have \( m \geq N  \) for some \( N \in \N  \), we can have 
    \[  | f'_m(\alpha) - f'_N(\alpha)  | < \frac{ \epsilon  }{ 3 }, \]
    which means that 
    \[  \Bigg| \frac{ f_m(x) - f_m(c)  }{ x -c  } - \frac{ f_N(x) - f_N(c)  }{  x- c  }  \Bigg| < \frac{ \epsilon  }{ 3 }. \tag{3} \] 
    Since \( f_m \to f  \) pointwise, we can use the Order Limit Theorem to imply that 
    \[  \Big| \frac{ f(x) - f(c)   }{ x- c  } - \frac{ f_n(x) - f_n(c)  }{ x - c  }  \Big| \leq \frac{ \epsilon  }{ 3 }. \]
    Using (1), (2), and (3), we can now conclude that  
    \begin{align*}
        \Big| \frac{ f(x) - f(c)  }{ x - c  } - g(c)  \Big| &\leq \Big| \frac{ f(x) - f(c)  }{  x- c  } - \frac{ f_n(x) - f_(c)  }{ x - c  }  \Big| \\ &+ \Big| \frac{ f_n(x) - f_n(c)  }{ x - c  } - f_n'(c)  \Big|  + | f_n'(c) - g(c)   | \\
                                                            &< \frac{ \epsilon  }{ 3  }  + \frac{ \epsilon  }{ 3  } + \frac{ \epsilon  }{ 3  } \\
                                                            &=\epsilon.
    \end{align*}
    Hence, \( (f'_n) \to g  \) and \( f' = g  \).
\end{proof}

As it turns out, we don't really need to assume that \( f_n(x) \to f(x)  \) for the conclusion above to be true. We only need uniform convergence of \( (f'_n)  \) for the theorem above to work. Two functions with the same derivative may differ by a constant, so we must suppose that there is at least one point \( x_0  \) where \( f_n(x_0) \to f(x_0)  \).

\begin{theorem}{}{}
    Let \( (f_n)  \) be a sequence of differentiable functions defined on the closed interval \( [a,b]  \), and assume \( (f'_n)  \) converges uniformly on \( [a,b]  \). If there exists a point \( x_0 \in [a,b]  \) where \( f_n(x_0)  \) is convergent, then \( (f_n)  \) converges uniformly on \( [a,b]  \).
\end{theorem}

\begin{proof}
    Let \( (f_n)  \) be a sequence of differentiable functions defined on the closed interval \( [a,b]  \), and assume \( (f'_n) \) converges uniformly on \( [a,b]  \). Furthermore, assume that there exists a point \( x_0 \in [a,b] \) where \( f_n(x_0)  \) is convergent. Since \( (f'_n)  \) converges uniformly on \( [a,b] \), let \( \epsilon = 1  \) such that  there exists \( N \in \N \) such that for any \( m,n \geq N \) and \( x \in [a,b]  \), we have that 
    \[  | f_n'(x) - f_m'(x)   | < \epsilon. \tag{1}\] Since \( f_n  \) is differentiable on \( [a,b]  \), we can use the Mean Value Theorem to state that there exists an \( \alpha \in (x_0, x)   \) such that 
    \[  f'_n(\alpha) = \frac{ f_n(x) - f_n(x_0)  }{ x - x_0  }  \]
    and 
    \[  f'_m(\alpha) = \frac{ f_m(x) - f_m(x_0)  }{ x - x_0 }. \]
   Using the fact that \( (f'_n)  \) converges uniformly, we know that 
   \[ | f_n'(\alpha) - f_m'(\alpha) | < 1. \] This implies that
   \[  \Big| \frac{ f_n(x) - f_n(x_0)  }{ x - x_0  } - \frac{ f_m(x) - f_m(x_0) }{ x - x_0 }  \Big| < 1. \] This implies that 
   \[  | f_n(x) - f_n(x_0) - (f_m(x) - f_m(x_0) ) | < | x - x_0  |  \]
   for which we can assume \( 0 < | x -x_0 | < \delta  \) since \( f_n  \) is differentiable. Using the reverse triangle inequality, we can take the left side of the above inequality and state that
   \[   | f_n(x) - f_m(x)  |  - | f_m(x_0) - f_n(x_0) | \leq | f_n(x) - f_m(x) - (f_m(x_0) - f_n(x_0) ) | < | x - x_0  |  \]
   which manipulating even further implies 
   \begin{align*}  | f_n(x) - f_m(x)  | &\leq | f_n(x) - f_m(x) - (f_m(x_0) - f_n(x_0) ) |  \\
       &+ | f_m(x_0) - f_n(x_0)  | \\
       &< | x - x_0 | + | f_m(x_0) - f_n(x_0) |  \tag{2}. 
   \end{align*}
   Using the triangle inequality of the left side of (2), using the fact that \( f_n(x_0) \to f(x_0)  \) and setting \( \delta = \epsilon / 3  \), we can say that for some \( N \in \N  \) where, we have that for any \( m,n \geq N  \) 
   \begin{align*}  | f_n(x) - f_m(x)  | &\leq | f_n(x) - f_m(x) - (f_m(x_0) - f_n(x_0) ) |  \\
       &+ | f_m(x_0) - f_n(x_0)  | \\
       &< | x - x_0 | + | f_m(x_0) - f_n(x_0) |  \\ 
       &= | x -x_0  | + | f_m(x_0) - f(x_0)  | + | f(x_0) - f_n(x_0) | \\
       &< \frac{ \epsilon  }{ 3 } + \frac{ \epsilon  }{ 3 } + \frac{ \epsilon  }{ 3  } = \epsilon.
   \end{align*}
   Hence, this means that \( (f_n)  \) is uniformly convergent.
\end{proof}%

Now we have a stronger version of the first theorem of this section.

\begin{theorem}{}{}
    Let \( (f_n)  \) be a sequence of differentiable functions defined on the closed interval \( [a,b]  \), and assume \( (f'_n)  \) converges uniformly to a function \( g  \) on \( [a,b]  \). If there exists a point \( x_0 \in [a,b]  \) for which \( f_n(x_0)  \) is convergent, then \( (f_n)  \) converges uniformly. Moreover, the limit function \( f = \lim f_n  \) is differentiable and satisfies \( f' = g  \).
\end{theorem}

