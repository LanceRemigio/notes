
\section{Power Series}

We can express functions in the form of power series where it takes the form of 
\[  f(x) = \sum_{ n=0  }^{  \infty  } a_n x^n = a_0 + a_1 x + a_2 x^2 + \dots . \]
We want to be able to find all possible \( x \in \R \) such that the above series converges. 

\begin{tcolorbox}
\begin{thm}
If a power series \( \sum_{ n=0  }^{ \infty  } a_n x^n  \) converges at some point \( x_0 \in \R  \), then it converges absolutely for any \( x  \) satisfying \( | x  |  < | x_0  |  \).
\end{thm}
\end{tcolorbox}

\begin{proof}
Assume the power series \( \sum_{ n=0  }^{ \infty  } a_n x^n  \) converges at some point \( x_0 \in \R  \). Since the sequence of terms \( (a_n x_0^n) \) converges to zero, we know that they must be bounded. Hence, there exists \( M > 0  \) such that \( | a_n x_0^n |  \leq M  \) for all \( n \in \N  \). If \( x \in \R  \) satisfies the property that \( | x  |  < | x_0  |  \), then we have that 
\[  | a_n x^n  |  = | a_n x_0^n  | \Big| \frac{ x  }{  x_0  }  \Big|^n \leq M \Big| \frac{ x  }{  x_0  }  \Big|^n.\] This tells us that the series 
\[  \sum_{ n=0  }^{  \infty   } M \Big| \frac{ x  }{  x_0  }  \Big|^n  \] is a geometric series with \( | x / x_0  |  <  1  \) which converges. Hence, we can use the comparison test, to state that the original series \( \sum_{ n = 0  }^{  \infty  } a_n x^n  \) converges absolutely.
\end{proof}

This theorem tells us a few things:
\begin{enumerate}
    \item[(a)] The set of points for which a given power series converges must be either \( \{ 0  \}  \), \( \R  \), or some bounded interval that is centered at \( x = 0  \). 
    \item[(b)] The strict inequality in the condition tells us that the intervals may come in the following forms; either, \( (-R , R ) \), \( [-R, R ] \), \( (-R, R ] \), or \( [-R , R ] \).
    \item[(c)] We denote the value \( R  \) in the intervals above as the \textit{radius of convergence} of a power series which can be either \( 0  \) or \( \infty  \) to represent \( \{ 0  \}  \) or \( \R   \) respectively.
\end{enumerate} 

Below are questions to be answered about the properties of power series:
\begin{enumerate}
    \item[(a)] Continuity
    \item[(b)] Differentiability
    \item[(c)] Term-by-term differentiability
    \item[(d)] Behavior of endpoints.
\end{enumerate}
\subsection{Establishing Uniform Convergence}

\begin{tcolorbox}
\begin{thm}
    If a power series \( \sum_{ n= 0 }^{  \infty   } a_n x^n \) converges absolutely at a point \( x_0  \), then it converges uniformly on the closed interval \( [-c , c ] \), where \( c  = | x_0  |  \).
\end{thm}
\end{tcolorbox}

\begin{proof}
    Suppose a power series \( \sum_{ n=0  }^{  \infty   } a_n x^n  \) converges absolutely at a point \( x_0  \). Then the series \(  \sum_{ n=0  }^{  \infty  } | a_n x_0^n |  \) converges. Let \( x \in [-c , c ] \) where \( c = | x_0  |  \). We proceed via the Weierstrass M-test to show that \( \sum_{ n=0  }^{ \infty  } a_n x^n  \) converges uniformly. We observe that 
    \[  | a_n x^n  |  \leq a_n c^n = a_n | x_0  |^n = a_n | x_0^n | .  \]
    This tells us that 
    \[  \sum_{ n=0 }^{  \infty   } | a_n x^n | \leq \sum_{ n=0  }^{ \infty  } a_n | x_0^n  |.   \]
    Since the right side of the above inequality converges, we know that \( \sum_{ n=0  }^{ \infty  } a_n x^n  \) must converge uniformly on any \( x \in [-c ,c ]  \).
\end{proof}
A few remarks about this result:
\begin{enumerate}
    \item[(a)] Any \( x \in (-R, R ) \) is contained in the interior of a closed interval \( [-c , c ]  \subseteq (-R ,R )\).
    \item[(b)] If the interval above was open instead of closed, then the limit of the series above is necessarily continuous on this interval. 
\end{enumerate}

Some questions we can ask about this result are:
\begin{enumerate}
    \item[(a)] Can a power series converge at an endpoint of the interval of convergence? 
    \item[(b)] Does the behavior of the power series on an open interval necessarily imply that it will be convergent at \( x = R  \)? 
    \item[(c)] What happens when we \textit{conditionally} convergent power series?
\end{enumerate}

\subsection{Abel's Theorem}

\begin{tcolorbox}
    \begin{lem}[Abel's Lemma]
    Let \( b_n  \) satisfy \( b_1 \geq b_2 \geq b_3 \geq \dots \geq 0,  \) and  let \( \sum_{ n=1 }^{ \infty  } a_n  \) be a series for which the partial sums are bounded. In other words, assume there exists \( A > 0  \) such that 
    \[  | a_1 + a_2 + \dots + a_n  |  \leq A  \] for all \( n \in \N  \). Then, for all \( n \in \N  \), 
    \[  \Big| \sum_{ n=1 }^{ \infty  } a_n b_n  \Big|  \leq Ab_1. \]
    \end{lem}
\end{tcolorbox}

\begin{proof}
Let \( s_n = \sum_{ k=1 }^{ n  } a_k  \) be a bounded sequence of partial sums for the series 
\[  \sum_{ n=1  }^{  \infty  } a_n.  \] Hence, there exists some \( A > 0  \) such that \( | x_n  | \leq A  \). Using the Summation-by-parts formula, we have that 
\begin{align*}
    \Big| \sum_{ k=1 }^{ n } a_k b_k  \Big| &= \Big| s_n y_{n+1} + \sum_{ k=1 }^{ n } s_k (y_k - y_{k+1}) \Big|  \\
                                            &\leq | s_n | | y_{n+1} | + \Big| \sum_{ k=1 }^{ n } s_k (y_k - y_{k+1}) \Big| \\
                                            &\leq Ay_1 + \sum_{ k=1 }^{ n } A (y_k - y_{k+1}) \\
                                            &= Ay_1 + A(y_1 - y_{n+1}) \\
                                            &\leq Ay_1.
\end{align*}
Hence, for all \( n \in \N  \), we have that 
\[  \Big| \sum_{ n=1 }^{ \infty  } a_n b_n  \Big|  \leq Ay_1. \]
\end{proof}

We can use this bound to prove the next theorem about proving convergence at one of the endpoints of an interval.

\begin{tcolorbox}
    \begin{thm}[Abel's Theorem]
        Let \( g(x) = \sum_{ n=0  }^{  \infty  } a_n x^n  \) be a power series that converges at the point \( x = R > 0  \). Then the series converges uniformly on the interval \( [0,R] \). A similar result holds if the series converges at \( x = -R  \).
    \end{thm}
\end{tcolorbox}

\begin{proof}

\end{proof}






