\section{Uniform Convergence of a Sequence of Functions}


Just like our studies demonstrated in Chapter 2, we will first study the behaviors and properties of converging \textit{sequences} of functions. The results that we have gathered about sequences and series so fat will be immediately applicable to our study of sequences of functions.

\subsection{Pointwise Convergence} 

\begin{tcolorbox}
\begin{defn}
For each \( n \in \N  \), let \( f_n \) be a function defined on a set \( A \subseteq \R  \). The sequence \( (f_n)  \) of functions \textit{converges pointwise }  on \( A  \) to a function \( f  \) if, for all \( x \in A  \), the sequence of real numbers \( f_n(x)  \) converges to \( f(x)  \). 

In this case, the following notations are all equivalent to each other
\begin{enumerate}
    \item[(i)] \( f_n \to f  \) 
    \item[(ii)] \( \lim f_n = f  \)
    \item[(iii)] \( \lim_{ n \to \infty  } f_n(x) = f(x)  \).
\end{enumerate}
\end{defn}
\end{tcolorbox}

(iii) of the definition above is especially useful if there are any confusions the may arise as to whether or not \( x  \) or \( n  \) is the limiting variable.

\begin{ex}
\begin{enumerate}
    \item[(i)] Consider the sequence of functions \( f_n  \) defined by 
        \[  f_n(x) = \frac{ x^2 + nx  }{ n } \]
        on all of \( \R  \). We can compute the limit of \( f_n  \)
        \[  \lim_{ n \to \infty  } f_n(x) = \lim_{ n \to \infty  } \frac{ x^2 + nx  }{ n  } = \lim_{ n \to \infty  } \frac{ x^2  }{ n } + x  = x. \] Thus, we have that \( (f_n) \) converges \textit{pointwise} to \( f(x) = x  \) on \( \R  \).
    \item[(ii)] Let \( g_n(x) = x^n  \) on the set \( [0,1]  \) where we consider the situation as \( n \to \infty  \). If \( 0 \leq x < 1  \), then we know that \( x^n \to 0  \). On the other hand, suppose \( x = 1  \), then we have that \( x^n \to 1  \). It follows that \( g_n \to g  \) converges pointwise on \( [0,1] \), where 
        \[  g(x) = 
        \begin{cases}
            0 &\text{for } 0 \leq x < 1 \\
            1 &\text{for } x = 1.
        \end{cases} \]
        We have a problem when considering continuity at \( x = 1  \).
    \item[(iii)] Consider \( h_n(x) = x^{1+ \frac{ 1 }{ 2n - 1  } } \) on the set \( [-1,1]  \). For a fixed \( x \in [-1,1]  \), we have 
        \[  \lim_{ n \to \infty  } h_n(x) = x \lim_{ n \to \infty  } x^{\frac{ 1 }{ 2n-1 } }  = | x  |.\]
        Note that this function is not differentiable at \( x = 0  \).
\end{enumerate}
\end{ex}

\subsection{Continuity of the Limit Function}

We will begin this section by failing to prove that the pointwise limit of continuous functions is continuous. We will then find the holes of the subsequent argument so that we may understand why we need a stronger footing on the meaning of convergence for a sequence of functions.

Let \( (f_n)  \) be a sequence of continuous functions on a set \( A \subseteq \R  \) and let us assume that \( (f_n)  \) converges to a pointwise limit \( f  \). We will try to argue that the limit \( f  \) is continuous. Let us fix \( c \in A  \), and let \( \epsilon > 0  \). Our objective is to find  \( \delta > 0 \) such that whenever \( | x - c  | < \delta  \), we have 
\[  | f(x) - f(c) | < \epsilon. \]
We may use the Triangle Inequality to write
\begin{align*}
    | f(x) - f(c)  | &= | f(x) - f_n(x) + f_n(x) - f_n(c) + f_n(c) - f(c)   |  \\
                     &\leq | f(x) - f_n(x)  | + | f_n(x) - f_n(c)  | + | f_n(c) - f(c)  |.
\end{align*}
Our impression is to make each term of the right hand side of this inequality small by using the fact that \( f_n \to f  \) and the continuity of \( f_n  \).  Since \( c \in A  \) is fixed, let us choose \( N \in \N  \) such that 
\[  | f_N(c) - f(c)  | < \frac{ \epsilon  }{ 3 }. \]
Since \( N  \) is chosen, the continuity of our particular choice \( f_N  \) implies that there exists a \(  \delta > 0  \) such that 
\[  | f_N(x) - f_N(c)  | < \frac{ \epsilon  }{ 3 } \]
for all \( x  \) whenever \( | x - c  | < \delta  \). But here lies the problem of using the continuity of \( f_n  \); that is, we also need the following to hold:
\[  | f_N(x) - f(x)  | < \frac{ \epsilon  }{ 3 }  \] for all \( x  \) satisfying \( | x - c  | < \delta  \). A few problems with this argument include 
\begin{enumerate}
    \item[(i)] Our choice of \( x  \) depends on \( \delta  \) which also depends on our choice of \( N  \). This means for every choice of \( x  \) along \( (c - \delta, c + \delta ) \), we will get a different \( N  \). We want our choice of \( \delta  \) to be uniform for any \( x  \).
    \item[(ii)] The choice of \( x  \) is not fixed the way \( c  \) is on the interval \( (c - \delta, c + \delta ) \). This means that our choice \( x  \) has to work along the interval.
\end{enumerate}
This problem is apparent in our second example at the beginning of this section where the inequality
\[  | g_n(1/2 ) - g(1/2)  | < \frac{ 1 }{ 3 } \]
for \( n \geq 2  \) whereas 
\[  |g_n(9/10) - g(9/10)   | < \frac{ 1 }{ 3 }  \]
is true only after \( n \geq 11 \).

\subsection{Uniform Convergence} 

To solve our the problems of pointwise convergence of functions, we introduce a stronger notion for convergence of functions.

\begin{tcolorbox}
    \begin{defn}[Uniform Convergence]
    Let \( (f_n)  \) be a sequence of functions defined on a set \( A \subseteq \R  \). Then, \( (f_n)  \) \textit{converges uniformly} on \( A  \) to a limit function \( f  \) defined on \( A  \) if, for every \( \epsilon  > 0 \), there exists an \( N \in \N  \) such that 
    \[  | f_n(x) - f(x)  | < \epsilon   \]
    whenever \( n \geq N  \) and \( x \in A  \).
    \end{defn}
\end{tcolorbox}

Let us restate the definition of Pointwise convergence so that we are able to distinguish the key differences between the two.

\begin{tcolorbox}
    \begin{defn}[Pointwise Convergence]
    Let \( (f_n)  \) be a sequence of functions defined on a set \( A \subseteq \R  \). Then, \( (f_n)  \) \textit{converges pointwise on} A to a limit \( f  \) defined on \( A  \) if, for every \( \epsilon > 0   \) and \( x \in A  \), there exists an \( N \in \N  \) (may depend on x ) such that 
    \[  | f_n(x) - f(x)  | < \epsilon  \] whenever \( n \geq N  \).
    \end{defn}
\end{tcolorbox}

Key Differences:
\begin{enumerate}
    \item[(i)] In uniform convergence, notice that we only need 
        \[ | f_n(x) - f(x)  | < \epsilon  \] to hold for all \( \epsilon > 0  \); that is, our choice of \( x    \) will not affect our choice of \( N  \). Another way to state this is \( N \neq N(\epsilon, x ) \)
    \item[(ii)] In pointwise convergence, not only do we need convergence to hold for all \( \epsilon > 0   \), we also need it to hold for all \( x  \).
\end{enumerate}

\begin{ex}
\begin{enumerate}
    \item[(i)] Let 
        \[  g_n(x) = \frac{ 1 }{ n(1 + x^2 ) }.  \]
        For any fixed \( x \in \R  \), it is apparent that \( \lim_{ n \to \infty  } g_n(x) = 0  \) so that \( g(x) = 0   \) is the pointwise limit of the sequence \( (g_n)  \) on \( \R  \). We want to know if \( (g_n)  \) uniformly convergent. Since \( 1 / (1 + x^2 )  \leq 1  \) for all \( x \in \R  \) implies that 
        \[  | g_n(x) - g(x)  | = \Big| \frac{ 1 }{ n(1+x^2 )  } - 0  \Big| \leq \frac{ 1 }{ n }.  \]
Hence, any given \( \epsilon > 0  \), we can choose \( N > 1 / \epsilon  \) (which does not depend on \( x  \)), we have that
\begin{center}
    \( n \geq N  \) implies \( | g_n(x) - g(x)  | < \epsilon  \)
\end{center}
for all \( x \in \R  \). Hence, \( g_n \to 0 \) uniformly on \( \R  \).
\item[(ii)] What about our first example from the very beginning of this section? Does it converge uniformly as well?  Let \( f_n(x) = (x^2 + n x ) / n  \). Since \( (f_n) \to f  \) pointwise where \( f(x) =x  \). It turns our that \( f_n  \) is not uniformly convergent. To see why this is the case, we write that
    \[  | f_n(x) - f(x)  | = \Big| \frac{ x^2 + nx  }{ n  } - x  \Big| = \frac{ x^2  }{ n }. \]
    For \( | f_n(x) - f(x)  | < \epsilon  \) to hold, we would need to create a choice of \( N  \) such that 
    \[  N  > \frac{ x^2  }{ \epsilon  }. \]
    While we certainly have convergence for every \( x \in \R  \), we still have our choice of \( N  \) not uniform. Although not uniformly convergent on all of \( \R  \), we do end up having uniform convergence when we consider \( f_n  \) over a closed interval \( [-b,b] \). Hence, we have that 
    \[  \frac{ x^2  }{ n }  \leq \frac{ b^2  }{ n  }. \]
    Given any \( \epsilon > 0  \), we can choose \( N > b^2 / \epsilon  \) that is not dependent on any \( x \in [-b,b]  \).
\end{enumerate}
\end{ex}

Graphically speaking, the uniform convergence of \( f_n  \) to a limit \( f  \) on a set \( A  \) can be visualized by constructing an \( \epsilon -  \)neighborhood around the limit \( f  \) for which all of \( f_n  \) is completely contained within the neighborhood for all \(  n \geq N  \) for some point \( N \in \N  \).

\subsection{Cauchy Criterion}

Recall that the Cauchy Criterion states an equivalence between convergent sequences and Cauchy sequences without stating the limit of the sequence. The usefulness of such a theorem creates an opportunity for an analogous  characterization of uniformly convergent sequences of functions.

\begin{tcolorbox}
    \begin{thm}[Cauchy Criterion for Uniform Convergence]
    A sequence of functions \( (f_n)  \) defined on a set \( A \subseteq \R  \) converges uniformly on \( A  \) if and only if for every \( \epsilon > 0  \), there exists an \( N \in \N  \) such that 
    \[  | f_n(x) - f_m(x)  | < \epsilon  \]
    whenever \( m,n \geq N  \) and \( x \in A  \).
    \end{thm}
\end{tcolorbox}
\begin{proof}
Exercise 6.2.5.
\end{proof}

\subsection{Continuity Revisited}

Let us now prove that the limit function of a sequence of continuous functions is continuous.

\begin{tcolorbox}
    \begin{thm}[Continuous Limit Theorem]
    Let \( (f_n)  \) be a sequence of functions defined on \( A \subseteq \R  \) that converges uniformly on \( A  \) to a function \( f  \), If each \( f_n   \) is continuous at \( c \in A  \), then \( f  \) is continuous at \( c  \).
    \end{thm}
\end{tcolorbox}

\begin{proof}
Fix \( c \in A  \) and let \( \epsilon > 0  \). Since \( (f_n) \to f   \) is uniformly convergent on \( \R  \), we can choose an \( N \in \N   \) (that does not depend on \( x  \)) such that 
\[  | f_N(x) - f(x)  | < \frac{ \epsilon  }{ 3  }  \]
for all \( x \in A  \). Since \( f_N  \) is continuous, there exists \( \delta > 0  \) for which 
\[  | f_N(x) - f_N(c)  | < \frac{ \epsilon  }{ 3 }  \]
is true whenever \( | x - c  | < \delta  \). Just like our argument at the beginning of this section, we have that
\begin{align*}
    | f(x) - f(c)  | &= | f(x) - f_N(x) + f_N(x) - f_N(c) + f_N(c) - f(c)  |  \\
                     &\leq | f(x) - f_N(x)  | + | f_N(x) - f_N(c)  | + | f_N(c) - f(c)  | \\
                     &< \frac{ \epsilon  }{ 3  } + \frac{ \epsilon  }{ 3 } + \frac{ \epsilon  }{ 3  } \\
                     &= \epsilon.
\end{align*}
Hence, \( f  \) is continuous at \( c \in  A  \).
\end{proof}




\subsection{Definitions and Theorems}


\begin{tcolorbox}
\begin{defn}
For each \( n \in \N  \), let \( f_n \) be a function defined on a set \( A \subseteq \R  \). The sequence \( (f_n)  \) of functions \textit{converges pointwise }  on \( A  \) to a function \( f  \) if, for all \( x \in A  \), the sequence of real numbers \( f_n(x)  \) converges to \( f(x)  \). 

In this case, the following notations are all equivalent to each other
\begin{enumerate}
    \item[(i)] \( f_n \to f  \) 
    \item[(ii)] \( \lim f_n = f  \)
    \item[(iii)] \( \lim_{ n \to \infty  } f_n(x) = f(x)  \).
\end{enumerate}
\end{defn}
\end{tcolorbox}


\begin{tcolorbox}
    \begin{defn}[Uniform Convergence]
    Let \( (f_n)  \) be a sequence of functions defined on a set \( A \subseteq \R  \). Then, \( (f_n)  \) \textit{converges uniformly} on \( A  \) to a limit function \( f  \) defined on \( A  \) if, for every \( \epsilon  > 0 \), there exists an \( N \in \N  \) such that 
    \[  | f_n(x) - f(x)  | < \epsilon   \]
    whenever \( n \geq N  \) and \( x \in A  \).
    \end{defn}
\end{tcolorbox}


\begin{tcolorbox}
    \begin{defn}[Pointwise Convergence]
    Let \( (f_n)  \) be a sequence of functions defined on a set \( A \subseteq \R  \). Then, \( (f_n)  \) \textit{converges pointwise on} A to a limit \( f  \) defined on \( A  \) if, for every \( \epsilon > 0   \) and \( x \in A  \), there exists an \( N \in \N  \) (may depend on x ) such that 
    \[  | f_n(x) - f(x)  | < \epsilon  \] whenever \( n \geq N  \).
    \end{defn}
\end{tcolorbox}


\begin{tcolorbox}
    \begin{thm}[Cauchy Criterion for Uniform Convergence]
    A sequence of functions \( (f_n)  \) defined on a set \( A \subseteq \R  \) converges uniformly on \( A  \) if and only if for every \( \epsilon > 0  \), there exists an \( N \in \N  \) such that 
    \[  | f_n(x) - f_m(x)  | < \epsilon  \]
    whenever \( m,n \geq N  \) and \( x \in A  \).
    \end{thm}
\end{tcolorbox}


\begin{tcolorbox}
    \begin{thm}[Continuous Limit Theorem]
    Let \( (f_n)  \) be a sequence of functions defined on \( A \subseteq \R  \) that converges uniformly on \( A  \) to a function \( f  \), If each \( f_n   \) is continuous at \( c \in A  \), then \( f  \) is continuous at \( c  \).
    \end{thm}
\end{tcolorbox}



\subsection{Exercises}

\subsubsection{Exercise 6.2.1} Let 
\[  f_n(x) = \frac{ nx }{ 1 + nx^2  }. \]
\begin{enumerate}
    \item[(a)] Find the pointwise limit of \( (f_n)  \) for all \( x \in (0, \infty ) \).
        \begin{proof}[Solution]
        Let us take the limit of \( (f_n)  \). A well know trick we can use is by dividing by \( (1/n) / (1/n)  \); that is, we have that
        \begin{align*}
            \lim_{ n \to \infty  } f_n &= \lim_{ n \to \infty  } \frac{ nx  }{  1 + nx^2  }  \\
                                       &= \lim_{ n \to \infty  } \frac{ x  }{ \frac{ 1 }{ n } + x^2  } \\
                                       &= \frac{ 1 }{ x  }.
        \end{align*}
        \end{proof}
    \item[(b)] Is the convergence uniform on \( (0,\infty)  \)?
        \begin{proof}[Solution]
        We claim that \( (f_n)  \) is not uniform on \( (0, \infty ) \). Let \( \epsilon > 0  \). Then 
        \begin{align*}
            \Big| \frac{ nx }{ 1 + nx^2  } -  \frac{ 1 }{ x }  \Big| = \frac{ 1  }{ x(1+nx^2 ) }.  &\\
        \end{align*}
        The right side of this equation implies 
        \[  \frac{ 1 }{ x(1+nx^2)  } < \epsilon. \]
        After a few algebraic manipulations, we can choose \( N \in \N  \) such that 
        \[  N > \Big( \frac{ 1 }{ \epsilon  } - x  \Big) \frac{ 1 }{ x^3  }  \]
        which is clearly dependent on \( x  \). 
        \end{proof}
    \item[(c)] Is the convergence uniform on \( (0,1)  \)? 
        \begin{proof}[Solution]
            We cannot have a lower bound that gets rid of the dependency of our choice of \( N   \) along \( (0,1)  \). Hence, \( (f_n)  \) does not converge uniformly along \( (0,1)  \).
        \end{proof}
    \item[(d)] Is the convergence uniform on \( (1,\infty ) \)?
        \begin{proof}[Solution]
        We claim that \( (f_n) \to f  \) is uniform on \( (1,\infty )  \). Since \( (f_n)  \) is defined on \( (0,1)  \), we can lower bound 
        \[   \frac{ 1 }{ x (1 +nx^2 ) } < \frac{ 1 }{ n+ 1 }.  \]
        along the interval \( (1,\infty )  \) which produces the following 
        \[  \frac{ 1 }{ \epsilon  }  < n + 1.\]
        This implies that for some \( N \in \N  \), we have 
        \[  N > \frac{ 1 }{ \epsilon  }  - 1. \]
        We can see that our choice of \( N  \) does not depend on \( x  \). Hence, we have that \( (f_n) \to f  \) uniformly along \( (1,\infty ) \).
        \end{proof}
\end{enumerate}

\subsubsection{Exercise 6.2.2} 
\begin{enumerate}
    \item[(a)] Define a sequence of functions on \( \R  \) by 
        \[  f_n(x) = 
        \begin{cases}
            1 &\text{~if~} x = 1, \frac{ 1 }{ 2 }, ..., \frac{ 1 }{ n } \\ 
            0 &\text{~otherwise} 
        \end{cases} \]
and let \( f  \) be the pointwise limit of \( f_n \). Is each \( f_n  \) continuous at zero? Does \( f_n \to f  \) uniformly on \( \R  \)? Is \( f  \) continuous at zero? 
\begin{proof}[Solution]
Each \( f_n  \) is continuous at \( x = 0  \), but \( f  \) is not continuous at \( x = 0  \) since \( 1/ n \to 0  \) as \( n \to \infty  \). Finally, we have that \( f_n  \) does not converge uniformly on \( \R  \).
\end{proof}

\item[(b)] Repeat this exercise using the sequence of functions
    \[  g_n(x) = 
    \begin{cases}
        x &\text{~if~} x = 1, \frac{ 1 }{ 2 }, \frac{ 1 }{ 3 }, \dots, \frac{ 1 }{ n } \\
        0 &\text{~otherwise}
    \end{cases} \]
    \begin{proof}[Solution]
    Each \( g_n  \) as well as its limit \( g  \) are continuous at \( x = 0  \). Furthermore, \( g  \) converges uniformly on \( \R  \).
    \end{proof}
\item[(c)] Repeat the exercise once more with the sequence
    \[  h_n(x) = 
    \begin{cases}
        1 &\text{~if~} x = \frac{ 1 }{ n } \\
        x &\text{~if~} x = 1, \frac{ 1 }{ 2 }, \frac{ 1 }{ 3 }, \dots \frac{ 1 }{ n-1  } \\
        0 &\text{~otherwise}.
    \end{cases} \]
    \begin{proof}[Solution]
    Each \( h_n  \) continuous at \( x = 0  \) but its limit \( h  \) is not continuous at \( x = 0  \). We have \( h_n   \) does not converge uniformly on \( \R  \).
    \end{proof} 
\end{enumerate}

\subsubsection{Exercise 6.2.3} For each \( n \in \N  \) and \( x \in [0,\infty ) \), let 
\[  g_n(x) = \frac{ x  }{ 1 + x^n } \text{~ and ~} h_n(x) = 
\begin{cases}
    1 &\text{~if~} x \geq 1 /n \\
    nx &\text{~if~} 0 \leq x < 1 / n.
\end{cases}  \]

\begin{enumerate}
    \item[(a)] Find the pointwise limit on \( [0,\infty ] \).
        \begin{proof}[Solution]
        Let us find the pointwise limit of \( g_n(x)  \). If we take the limit of \( (g_n) \), we have
        \[  \lim_{ n \to \infty  } \frac{ x  }{ 1 + x^n  } = 0. \] This limit cannot be uniform on \( [0,\infty ) \) since our choice of \( N  \) depends on \( x  \). 

        Now let us find the limit of \( (h_n)  \). If \( x \geq 1 / n  \), we have \( \lim_{  n \to \infty  } h_n(x) = 1  \). If \( 0 \leq x < 1 / n \), then we have 
        \[  \lim_{ n \to \infty  } h_n(x) = 0. \]
        Just like \( g_n(x)  \), \( h_n(x)  \) cannot converge on \( [0,\infty )  \) since our choice of \( N  \) depends on \( x  \).
        \end{proof}
\end{enumerate}



\subsubsection{Exercise 6.2.5}  Using the Cauchy Criterion for convergent sequences of real numbers, supply a proof for Theorem 6.2.5. (First, define a candidate for \( f(x)  \), and then argue that \( f_n \to f  \) uniformly.)

\begin{proof}
    (\( \Rightarrow \)) Let \( (f_n)  \) be defined on \( A \subseteq \R   \). Define \( f_n(x) = y_n  \), \( f_m(x) = y_m  \), and \( f(x) = L  \). We want to show that \( (f_n)  \) is Cauchy. Let \( \epsilon > 0  \). Since \( (f_n) \to f  \) uniformly, there exists \( N \in \N  \) such that for any \( n,n \geq N  \), we have 
\begin{align*}
    |y_n - L   | &< \epsilon,  \\
    | y_m - L  | &< \epsilon.
\end{align*}
But  since \( (f_n)  \) is just a sequence of real numbers that converges, we have that for any \( m,n \geq N  \), 
\[ | y_n - y_m  | < \epsilon. \]

(\( \Leftarrow \)) Suppose for any \( \epsilon > 0  \), there exists an \( N \in \N  \) such that whenever \( m,n \geq N  \) and \( x \in A \), we have 
\[  | f_n(x) - f_m(x)  | < \epsilon. \]
This means that \( (f_n)  \) is a Cauchy sequence. Since \( (f_n) \) is a sequence of real numbers on \( A  \), we know that \( (f_n)  \) is also convergent. Since \( (f_n) \to f  \) whenever \( n,m \geq N  \) and \( x \in A  \), we have that \( (f_n) \to f  \) uniformly.
\end{proof}

\subsubsection{Exercise 6.2.6} Assume \( f_n \to f  \) on a set \( A  \). Theorem 6.2.6 is an example of a typical type of question which asks whether a trait possessed by each \( f_n  \) is inherited by the limit function. Provide an example to show that all of the following propositions are false if the convergence is only assumed to be pointwise on \( A  \). Then go back and decide which are true under the stronger hypothesis of uniform convergence.
\begin{enumerate}
    \item[(a)] If each \( f_n  \) is uniformly continuous, then \( f  \) is uniformly continuous.
        \begin{proof}
        We want to show that this is true when \( f_n \to f   \) uniformly. Let \( \epsilon > 0  \). Suppose \( f_n  \) is uniformly continuous. Then for any \( n \geq N  \) for some \( N \in \N  \), we have that 
        \[  | f_n(x) - f_n(y)  | < \epsilon. \]we know that Choose \( N = \max \{  N_1, N_2, N_3  \}  \) such that for any \( n \geq N  \) and \( x,y \in A   \), we have that
        \begin{align*}
            | f(x) - f(y)  | &= | f(x) - f_n(x) + f_n(x) - f_n(y) + f_n(y) - f(y)  |  \\
                             &\leq | f(x) - f_n(x)  | + | f_n(x) - f_n(y)  | + | f_n(y) - f(y)  | \\
                             &< \frac{ \epsilon  }{ 3 } + \frac{ \epsilon  }{ 3 } + \frac{ \epsilon  }{ 3 } = \epsilon.
        \end{align*}
    Hence, \( (f_n) \to f  \) uniformly.
        \end{proof}
    \item[(b)] If each \( f_n  \) is bounded, then \( f  \) is bounded.
        \begin{proof}
        Suppose \( f_n  \) is bounded, then there exists \( M > 0  \) such that \( | f_n(x)  | \leq  M \) for any \( n \in \N  \). Assume \( f_n \to f  \) uniformly. Then let \( \epsilon = 1   \) and let \( x \in A  \) be arbitrary, there exists \( N \in \N  \) such that for any \( n \geq N  \), we have 
        \[  | f(x) - f_n(x)  | < 1 \iff | f(x) | < 1 + | f_n(x)     | \leq M + 1 = M'.  \]
        Hence, \( f  \) is bounded.
        \end{proof}
\end{enumerate}

\subsubsection{Exercise 6.2.7} Let \( f  \) be uniformly continuous on all of \( \R  \), and define a sequence of functions \( f_n(x) = f(x+ \frac{ 1 }{ n } ) \). Show that \( f_n \to f  \) uniformly. Give an example to show that this proposition fails if \( f  \) is only assumed to be continuous and not uniformly continuous on \( \R  \).

\begin{proof}
Let \( f  \) be uniformly continuous on all of \( \R  \), and define a sequence of functions \( f_n(x) = f(x+ \frac{ 1 }{ n } )   \). Let \( (x_n), (y_n) \subseteq K  \). Define \( y_n = x + \frac{ 1 }{ n }  \) and \( x_n = x  \). Since \( f  \) is uniformly continuous, we know that \( | x_n - y_n | \to 0   \) implies that \( | f(x_n) - f(y_n)  | \to 0  \). Let \( \epsilon > 0  \). Choose \( N \in \N  \) such that for any \( n \geq N  \), we have 
\begin{align*}
    | f_n(x) - f(x)  | &= | f(x + \frac{ 1 }{ n } ) - f(x) |  \\
                       &= | f(y_n) - f(x_n)   | \\
                       &< \epsilon.
\end{align*}
Hence, \( (f_n) \to f  \) uniformly.
\end{proof}


\subsubsection{Exercise 6.2.8} Let \( (g_n)  \) be a sequence of continuous functions that converges uniformly to \( g  \) on a compact set \( K  \). If \( g(x) \neq 0  \) on \( K  \), show \( (1/g_n) \) converges uniformly on \( K  \) to \( 1/g \).

\begin{proof}
Let \( \epsilon > 0  \). We want to show that there exists \( N \in \N  \) such that for any \( n \in \N  \), we have 
\[  \Big| \frac{ 1 }{ g_n(x)  } - \frac{ 1 }{ g(x)  }  \Big| < \epsilon.  \]
Since \( (g_n) \to g  \) uniformly on \( K  \), we know that whenever \( n \geq N  \) and \( x \in K  \), for some \( N \in \N  \), we have that 
\[  | g_n(x) - g(x)  | < \epsilon. \]
Furthermore, \( (g_n)  \) continuous on compact set \( K  \) for each \( n \in \N  \). This means that for each \( n \in \N  \), we have that \( g_n(x)  \) has a bounded range. Hence, there exists \( M > 0  \) such that \( | g_n(x)  | \geq M   \). Similarly, \( | g(x)  | \geq L  \) for some \( L > 0 \). Then choose \( N = ML \cdot \epsilon   \) such that for any \( n \geq N  \), we have that 
\begin{align*}
    \Big| \frac{ 1 }{ g_n(x)  } - \frac{ 1 }{ g(x)  }  \Big| &= \Big| \frac{ g(x) - g_n(x)  }{ g_n(x) \cdot g(x) }  \Big|   \\
                                                             &= \frac{ | g_n(x) - g(x)  |  }{ | g_n(x)  | | g(x)  |  }  \\
                                                             &< \frac{ ML \cdot \epsilon  }{ ML  } = \epsilon. \\ 
\end{align*}
Hence, \( g_n  \) converges to \( g  \) uniformly on \( K  \).
\end{proof}

        There is another way to prove this using the Cauchy Criterion.
        \begin{proof}
        Let \( \epsilon > 0  \). It suffices to show that \( 1/g_n  \to 1/g \) uniformly by showing \( (1/g_n)  \) satisfies the Cauchy Criterion. Since \( g_n  \) is a bounded sequence of functions, there exists \( M > 0  \) such that \( | g_n | \geq M    \). Choose \( N \in \N  \) such that for any \( n,m \geq N  \), we have 
        \begin{align*}
            \Big| \frac{ 1 }{ g_n(x) } - \frac{ 1 }{ g_m(x) }  \Big| &= \Big| \frac{ g_m(x) - g_n(x)  }{ g_n(x) g_m(x) }  \Big|  \\
                                                               &= \frac{ | g_m(x) - g_n(x)  |  }{ | g_n(x)  | | g_m(x) |  } \\
                                                               &< \frac{ M^2 \epsilon  }{ M^2  } \\
                                                               &= \epsilon.
        \end{align*}
        Since \( (1/g_n)  \) is a Cauchy sequence of functions, we know that \( (1/g_n)  \) must converge uniformly.
        \end{proof}
\subsubsection{Exercise 6.2.9} Assume \( (f_n)  \) and \( (g_n)  \) are uniformly convergent sequences of functions.
\begin{enumerate}
    \item[(a)] Show that \( (f_n + g_n)  \) is uniformly convergent sequence of functions.
        \begin{proof}
        Let \( \epsilon > 0  \) and let \( x \in \R  \). Since \( (f_n)  \) and \( (g_n)  \) are uniformly convergent sequences of functions, there exists \( N_1, N_2 \in \N  \) such that for any \( n \geq N  \) where \( N = \max \{ N_1, N_2  \}  \), we have that 
        \begin{align*}
            | f_n(x) + g_n(x) - (f(x) + g(x))  | &= | f_n(x) - f(x) + g_n(x) - g(x)   |  \\
                                                 &\leq | f_n(x) - f(x)  | + | g_n(x) + g(x)  | \\
                                                 &< \frac{ \epsilon  }{ 2 } + \frac{ \epsilon  }{ 2 } = \epsilon.
        \end{align*}
        \end{proof}
    \item[(b)] Give an example to show that the product \( (f_ng_n)  \) may not converge uniformly.
        \begin{proof}[Solution]
        Let either \( (g_n)  \) or \( (f_n)  \) be unbounded. Take, for example, \( (f_n)  \) defined as 
        \[  f_n = x^2 + \frac{ 1 }{ n }  \]
        on \( \R  \).
        We can see that \( f_n \to f  \) where \( f  \) is also unbounded. Hence, \( (f_ng_n) \) cannot be uniformly convergent. 
        \end{proof}
    \item[(c)] Prove that if there exists an \( M > 0  \) such that \( | f_n  | \leq M  \) and \( | g_n  | \leq M  \) for all \( n \in \N \), then \( (f_ng_n) \) does converge uniformly.
        \begin{proof}
        Suppose there exists \( M > 0  \) such that \( | f_n | \leq M  \) and \( | g_n | \leq M  \) for all \( n \in \N  \). Let \( \epsilon > 0  \) and let \( x \in \R  \). It suffices to show that \( (f_ng_n) \) is Cauchy. Since \( (f_n) \to f  \) and \( (g_n) \to g  \) uniformly, choose \( N = \max \{ N_1, N_2  \}   \) such that for any \( n \geq N  \), we have that 
        \begin{align*}
           | f_n(x) g_n(x) - f_m(x) g_m(x)  |  &= | f_n(x) g_n(x) -f_m(x)g_n(x) + f_m(x)g_n(x) - f_m(x)g_m(x) |  \\
                                           &\leq | g_n(x)  | | f_n(x) - f(x)  | + | f_m(x)  | | g_n(x) - g(x)  | \\
                                           &< M \cdot \frac{ \epsilon  }{ 2 M  }  + M \cdot \frac{ \epsilon  }{ 2M  } = \epsilon. 
        \end{align*}
        Hence, \( (f_ng_n)  \) converges uniformly by the Cauchy Criterion.
        \end{proof}
\end{enumerate}

\subsubsection{Exercise 6.2.10} This exercise and the next explore the partial converse of the Continuous Limit Theorem. Assume \( f_n \to f  \) pointwise on \( [a,b]  \) and the limit function \( f  \) is continuous on \( [a,b]  \). If each \( f_n  \) is increasing (but not necessarily continuous), show \( f_n \to f  \) uniformly.

\begin{proof}
    Suppose \( f  \) is continuous on \( [a,b]  \), \( f_n  \) increasing, and \( (f_n) \to f  \) pointwise on \( [a,b]  \). Since \( f  \) continuous on \( [a,b]  \), we will have a mapping from the closed interval \( [a,b]  \) to another closed interval \( [c,d]  \). Our intention is to split \( [c,d]  \) into small intervals so that we may use the fact that \( f_n  \) is increasing to create a bound that is at most a size of \( \epsilon_1 \). By the Order Limit Theorem, we know that \( f_n  \) increasing also implies that \( f  \) is also increasing. 

    Let \( \epsilon > 0  \). Define that for each \( x_i  \), we have \( f(x_i) = y_i   \). Let \( y_1 = c  \) and \( y_n = d  \), where our small intervals our defined as everything in between; that is, for every \( k \in \N  \), we have \( [y_{k_2 }, y_{k+1} ] \) implies \( | y_{k+1} - y_k  | < \epsilon_1 \).  Since \( (f_n) \to f  \), we can find \( M_k  \) such that \( m_k > M_k  \) implies \( | y_k - f_{m_k} (x_k)  | < \epsilon_1 \). Let \( M = \max \{ M_1, M_2, \dots, M_n  \}  \). Let \( m > M  \) be arbitrary. Since \( f_m  \) increasing, we can bound \( f_m(x_{i+1} - f_m(x_i) ) \) by 
    \begin{align*}
        | f_m(x_{i+1}) - f_m(x_i)  | &= | f_m(x_{i+1}) - y_{i+1} + y_{i+1} - y_i + y_i - f_m(x_i)  |  \\
                                     &\leq | f_m(x_{i+1}) - y_{i+1}  | + | y_{i+1} - y_i  | + | y_i - f_m(x_i)  | \\
                                     &< \epsilon_1.
    \end{align*}
    Consider \( | f(x) - f_m(x)  |  \) and let \( x \in [x_i, x_{i+1} ] \) with \( i  \) arbitrary. Since \( f  \) is increasing and \( f  \) is continuous, we know that \( y_i \leq f(x) \leq y_{i+1}  \) and hence \( f(x) -  y_i  < \epsilon_1 / 3 \). Similarly, we have that \( f_m(x) - f_m(x_i)  \leq f_m(x) - f_m(x_{i+1} ) < \epsilon_1 / 3 \). Hence, observe that 
    \begin{align*}
        | f(x) - f_m(x)  | &= | f(x) - y_i + y_i - f_m(x_i) + f_m(x) - f_m(x_i) |  \\
                           &\leq | f(x) - y_i | + | y_i - f_m(x_i)  | + | f_m(x) - f_m(x_i)  | \\
                           &< \frac{ \epsilon_1 }{ 3 } +  \frac{ \epsilon_1  }{ 3 } + \frac{ \epsilon_1  }{ 3  } \\
                           &= \epsilon_1.
    \end{align*}
    Hence, \( f_n \to f  \) uniformly.
\end{proof}




