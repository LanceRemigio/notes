\section{Series of Functions}

\begin{tcolorbox}
\begin{defn}
    For each \( n \in \N  \), let \( f_n  \) and \( f  \) be functions defined on a set \( A \subseteq \R  \). The infinite series 
    \[  \sum_{ n=1 }^{ \infty  } f_n(x) = f_1(x) + f_2(x) + f_3(x) + \dots \]
    \textit{converges pointwise}  on \( A  \) to \( f(x)  \) if the sequence \( s_k(x)  \) of partial sums defined by 
    \[  s_k(x) = f_1(x) + f_2(x) + \dots + f_k(x) \]
    converges pointwise to \( f(x)  \). The series \textit{converges uniformly} on \( A  \) to \( f  \) if the sequence \( s_k(x)  \) converges uniformly on \( A  \) to \( f(x)  \). In either case, we write 
    \[  f = \sum_{ n=1 }^{ \infty  } f_n  \] or 
    \[  f(x) = \sum_{ n=1 }^{ \infty  } f_n(x)  \] always being explicit about the type of convergence involved.
\end{defn}
\end{tcolorbox}
