\section{Series of Functions}

\subsubsection{Exercise 6.4.1} Supply the details for the proof of the Weierstrass M-test (Corollary 6.4.5).

\begin{proof}
Let \( M_n > 0  \) be a real number satisfying \( | f_n(x)  | \leq M_n  \) for all \( x \in A  \).Let \( \epsilon > 0 \). Our goal is to use the Cauchy Criterion for Uniform Convergence of Series to prove that \( \sum_{ n=1 }^{ \infty  } f_n  \) converges uniformly. Suppose \( \sum_{ n=1 }^{ \infty  } M_n  \) converges, then for some \( N \in \N  \), we have for any \( n > m \geq N  \) implies 
\[ \Big| \sum_{ k=m+1 }^{ n } M_k \Big| < \epsilon.   \]
Since \( M_n > 0  \), this result can be restated as follows
\[  \sum_{ k=m+1 }^{ n } M_k < \epsilon. \]
By letting \( n > m \geq N  \) and \( x \in A  \) arbitrary as before, we can use the triangle inequality to state that 
\begin{align*}
    \Big| \sum_{ k=m+1 }^{ n } f_k(x)  \Big| &\leq \sum_{ k=m+1 }^{ n }| f_k(x)  |  \\
                                             &\leq  \sum_{ k=m+1 }^{ n } M_k \\
                                             &< \epsilon.
\end{align*}
By the Cauchy Criterion for Uniform Convergence, we know that \( \sum_{ n=1 }^{ \infty  } f_m(x)  \) must converge uniformly on \( A  \),
\end{proof}

\subsubsection{Exercise 6.4.2} Decide whether each proposition is true or false, providing a short justification or counterexample as appropriate.  

\begin{enumerate}
    \item[(a)] If \( \sum_{ n=1 }^{ \infty  } g_n  \) converges uniformly, then \( (g_n)  \) converges uniformly to zero.
        \begin{proof}[Solution]
         Since \( \sum_{ n=1 }^{ \infty  } g_n  \) converges uniformly, we know that 
        \[  \lim_{ n \to \infty  } \sum_{ k=1 }^{ n } g_k(x)  \]
        converges uniformly. If we consider the case when \( n = m - 1  \), then we have that 
        \[  \lim_{ n \to \infty  } \sum_{ k=1 }^{ n } g_k(x) = \lim_{ n \to \infty  } \Big( g_n(x) + \sum_{ k=1 }^{ n - 1  } g_n(x)  \Big)  \tag{1}. \]
        Furthermore, (1) implies that 
        \[  \lim_{ n \to \infty  }  \sum_{ k=1 }^{  n } g_k(x) = \lim_{ n \to \infty  }  g_n(x) + \lim_{ n \to \infty  } \sum_{ k=1 }^{ n-1 } g_n(x). \tag{2} \]
        Since the sequence of partial sums for \( \sum_{ n=1 }^{ \infty  } g_n(x)  \), we know that subtracting the second term on the right side of (2) results in 
        \[  \lim_{ n \to \infty  }  g_n(x) = 0. \]
        Hence, \( g_n \to 0  \).
        \end{proof}
        
    \item[(b)] If \( 0 \leq f_n(x) \leq g_n(x)  \) and \( \sum_{ n=1 }^{ \infty  } g_n  \) converges uniformly, then \( \sum_{ n=1 }^{ \infty  } f_n  \) converges uniformly.
        \begin{proof}
        Let \( x \in A  \) be arbitrary. Since \( 0 \leq f_n(x) \leq g_n(x)  \) and \( \sum_{ n=1 }^{ \infty  } g_n  \) satisfies the Cauchy Criterion for uniform convergence, we know that for any \( n > m \geq N  \) and \( x \in A  \), we have that
        \begin{align*}
            \Big| \sum_{ k=m+1 }^{ n } f_n(x)  \Big|  &\leq \Big| \sum_{ k=m+1 }^{ n } g_n(x) \Big| \\ 
                                                      &< \epsilon.
        \end{align*}
        Hence, \( \sum_{ n=1 }^{ \infty  } f_n(x)  \) converges uniformly.
        \end{proof}
    \item[(c)] If \( \sum_{ n=1 }^{ \infty  } f_n(x)  \) converges uniformly on \( A  \), then there exists constants \( M_n  \) such that \( | f_n(x)  | \leq M_n  \) for all \( x \in  A \) and \( \sum_{ n=1 }^{ \infty  } M_n  \) converges.
        \begin{proof}[Solution]
        Suppose our series is defined by 
        \[  \sum_{ n=1 }^{ \infty  } \frac{ (-1)^{n+1}  }{  n } \tag{1}\]
        where \( f_n(x) = (-1)^{n+1} / n  \). We can see by the Alternating Series test that (1) converges uniformly. But note that \( | f_n(x)  | \leq 1 / n = M_n   \) which produces the harmonic series 
        \[  \sum_{ n=1 }^{ \infty  } \frac{ 1 }{ n }  \]
        which diverges.
        
        \end{proof}
\end{enumerate}

\subsubsection{Exercise 6.4.3} 
\begin{enumerate}
    \item[(a)] Show that 
        \[  g(x) = \sum_{ n=0 }^{ \infty  } \frac{ \cos(2^n x) }{ 2^n }  \]
        is continuous on all of \( \R  \).
        \begin{proof}
            Our goal is to use the Term-by-term Continuity Theorem to show that \( g  \) is continuous on \( \R  \); that is, we want to show that \( g_n  \) is continuous and \( \sum_{ n=0 }^{ \infty  }  g_n  \) converges uniformly. Since \( \cos(x)  \) is a continuous function, we know that for each \( n \in \N  \), \( g_n  \) must be a sequence of continuous functions. All that is left is to show that \( \sum_{ n=0 }^{ \infty  } g_n(x)  \) converges uniformly. We shall do this by using the Weierstrass M-test. Since \( | \cos(2^n x ) | \leq 1  \) for all \( x \in \R  \), we know that \[  \sum_{ n=0 }^{ \infty  } \Big| \frac{ \cos(2^n x ) }{  2^n  }  \Big|  \leq \sum_{ n=0  }^{ \infty  } \Big( \frac{ 1 }{ 2 }  \Big)^n \tag{1}.  \]
            Since the right side of (1) is a geometric series that converges, we know that the series \( \sum_{ n=0 }^{ \infty  } g_n(x)  \) must converge uniformly to \( g(x)  \). Hence, \( g(x) \) is continuous on all of \( \R  \).
        \end{proof}
    \item[(b)] The function \( g  \) was cited in Section 5.4 as an example of a nowhere differentiable function. What happens if we try to use Theorem 6.4.3 to explore whether \( g  \) is differentiable?
    \item[(b)] We can define the series above via the sawtooth function from section 5.4 where 
        \[  g(x) = \sum_{ n=0 }^{  \infty  } h_n(x) = \sum_{ n=0  }^{  \infty  } 2^n h(2^n x ) \tag{1}.\]
        If we look at the terms \( h_n(x)  \), we can see that 
        \[ | h_n(x)  |  \leq \frac{ 1 }{ 2^n  } \tag{2} \] which forms a geometric series on the right side of (2). Hence, we must have (1) converge uniformly via the Weierstrass M-test.
\end{enumerate}

\subsubsection{Exercise 6.4.4} Define 
\[  g(x) = \sum_{ n=0  }^{ \infty  } \frac{ x^{2n} }{ (1+x^{2n}) }. \]
Find the values of \( x  \) where the series converges and show that we get a continuous function on this set.
\begin{proof}
    Let \( h_n(x) = \frac{ x^{2n} }{ (1 + x^{2n}) }    \) be the terms of the series 
    \[  \sum_{ n=0  }^{ \infty  } h_n(x).\] When \( | x  |  \geq 1  \), we observe that the terms of \( (h_n)  \) does not converge to \( 0  \) as \(  n \to \infty  \). If \( | x  |  < 1  \), then we have that 
    \[  | h_n(x)  |  \leq x^{2n}. \]
    Since \( | x  |  < 1  \), \( x^{2n} \) forms a geometric series that converges. Hence, we must have that the series of \( h_n(x)  \) must be uniformly convergent. Furthermore, for any \( 0 \leq a < 1  \), we will find that the infinite series of \( h_n(x)  \) converges uniformly where 
    \[  \sum_{ n=0  }^{ \infty  } \frac{ a^{2n} }{  1 + a^{2n} }. \]
\end{proof}

\subsubsection{Exercise 6.4.5} 
\begin{enumerate}
    \item[(a)] Prove that 
        \[  h(x) = \sum_{ n=1 }^{ \infty  } \frac{ x^n  }{ n^2  } = x + \frac{ x^2  }{ 4  } + \frac{ x^3  }{ 9  } + \frac{ x^4  }{ 16 } + \dots\]
        is continuous on \( [-1,1] \).
        \begin{proof}
        Using the Weierstrass M-test, we have that
        \[  \sum_{ n=1 }^{ \infty  } \Big| \frac{ x^n  }{ n^2  }  \Big|   \leq \sum_{ n=1 }^{ \infty  } \frac{ 1 }{ n^2  } \tag{1}.\]
        Since the series on the right side of (1) is a P-series, we know that it must converge which means that 
        \[  \sum_{ n=1 }^{ \infty  } \frac{ x^n  }{ n^2  }   \] must converge. Note the sequence of functions
        \[  h_n(x) = \frac{ x^n  }{ n^2  }  \] is continuous for each \( n \in \N  \) because for each \( n \in \N  \) \( h_n(x)  \) is just a polynomial which is continuous. Hence, we must have that 
        \[  h(x) = \sum_{ n=1  }^{ \infty  } \frac{ x^n  }{ n^2  }  \]
        is continuous on \( [-1,1] \).
        \end{proof}
    \item[(b)] The series 
        \[  f(x) = \sum_{ n=1 }^{ \infty  } \frac{ x^n  }{  n  }  = x + \frac{ x^2  }{ n  }  + \frac{ x^3  }{ 3  }  + \frac{ x^4  }{  4  }  + \dots \]
        converges for every \( x  \) in the half-open interval \( [-1,1) \) but does not converge when \( x = 1  \). For a fixed \( x_0 \in (-1,1)  \), explain how we can still use the Weierstrass M-test to prove that \( f  \) is continuous at \( x_0  \).
        \begin{proof}[Solution]
            Fix \( x_0  \), then consider any \( | x_0  |  < 1  \). Using the Weierstrass M-test, we will find that the terms 
            \[  \Big| \frac{ x_0^n  }{ n }  \Big|   \leq | x_0^2 |^n    \]
            forms a geometric series for which the right side will converge implying that the series of the left side will converge.
        \end{proof}
\end{enumerate}

\subsubsection{Exercise 6.4.7} Let 
\[ f(x) = \sum_{ k=1 }^{ \infty  } \frac{ \sin (kx)  }{  k^{3} }.  \]
\begin{enumerate}
    \item[(a)] Show that \( f(x)  \) is differentiable and that the derivative \( f'(x)  \) is continuous. 
        \begin{proof}
        We will use the term-by-term differentiability theorem to show that 
        \[  f(x) = \sum_{ k=1 }^{ \infty  } \frac{ \sin(kx) }{ k^3  }  \]
        is differentiable. We need to show that \( (f_k)  \) is differentiable for every \( k \in \N  \), \( \sum_{ k=1 }^{ \infty  } f'_n(x)  \) converges uniformly to some \( g(x)  \) in \( \R  \), and that for some \( x_0 \in [a,b] \subseteq \R  \) that \( \sum_{ k=1 }^{ \infty  } f_k(x_0)  \) converges to \( f(x_0)  \). 

        Note that for any \(k \in \N  \), we know that \( f_k(x)  \) is differentiable since \( \sin(kx)  \) is a differentiable function for all \( x \in \R  \). Now we compute \( f'_n(x)  \) which results in 
        \[  f'_n(x) = \frac{ \cos(kx)  }{  k^2  }. \]
        Since \( | \cos(kx)  |  \leq 1  \), we can state that 
        \[  \sum_{ k=1 }^{ \infty  } \Big| \frac{ \cos(kx)  }{ k^2  }  \Big|  \leq \sum_{ k=1 }^{ \infty  }  \frac{ 1 }{ k^2  } \tag{1}. \]
        Since the p-series on the right of (1) converges, we know that 
        \[ \sum_{ k=1 }^{ \infty  } f'_k(x)  \] 
        must converge uniformly to some \( g(x)  \) on \( \R  \). By the same reasoning, we can show that \( \sum_{ k=1 }^{ \infty  } f_n(x)  \) converges for all \( x \in \R  \); that is, since \( | \sin(kx)   |  \leq 1  \), we know that
        \[  \sum_{ k=1 }^{ \infty  } \Big| \frac{ \sin(kx)  }{ k^3  }  \Big|  \leq \sum_{ k=1 }^{ \infty  } \frac{ 1 }{ k^3  }  \]
        where the p-series on the right converges which implies that \( \sum_{ k=1 }^{ \infty  } f_n(x)  \) converges. By the Term-by-term differentiability theorem, we know that 
        \[  \sum_{ k=1 }^{ \infty  } \frac{ \sin(kx)  }{ k^3  }   \]
        must converge uniformly to a differentiable function \( f(x) \). Furthermore, we know that \( f'(x)  \) is continuous since \( f'_n  \) for all \( n \in \N  \) is continuous and 
        \[  \sum_{ k=1 }^{ \infty  } \frac{ \cos(kx) }{ k^2  }  \]
        converges uniformly on \( \R  \).
        \end{proof}
    \item[(b)] Can we determine if \( f  \) is twice-differentiable?
        We claim that \( f  \) is not twice-differentiable. If we compute \( f"_n(x)  \) and end up with the series
        \[  \sum_{ k=1 }^{ \infty  } \frac{ \sin(kx)  }{ k }  \]
        , then the series above fails the Weierstrass M-test since the constant series 
        \[ \sum_{ k=1 }^{ \infty  } \frac{ 1 }{ k}  \]
        diverges (this is a harmonic series). 

\end{enumerate}

\subsubsection{Exercise 6.4.8} Consider the function 
\[  f(x) = \sum_{ k=1 }^{  \infty  } \frac{ \sin(x / k) }{ k }.  \]
Where is \( f  \) defined? Continuous? Differentiable? Twice-differentiable?
\begin{proof}[Solution]
The function \( f  \) is defined on every \( x \in \R  \). We first claim that \( f  \) is continuous. We do this by showing that 
\[  \sum_{ k=1 }^{ \infty  } \frac{ \sin(x/k)  }{ k  }  \] converges uniformly and that each \( f_n  \) is continuous. Since \( | \sin(k/x)  |  \leq | kx  |  \), we know that 
\[ \sum_{ k=1 }^{ \infty  } \Big| \frac{ \sin(k/x)  }{  k  }   \Big|  \leq \sum_{ k=1 }^{ \infty  } \frac{ | x/k |   }{ k  } \leq \sum_{ k=1 }^{ \infty  } \frac{ 1 }{ k^2 }.    \]
Since \( \sum_{ k=1 }^{ \infty  } 1 / k^2  \) converges, we know that 
\[  \sum_{ k=1 }^{ \infty  } f_n(x)  \] must converge uniformly. Since \( \sin(x)  \) is continuous (trig functions are continuous), we know that \( f(x)   \) must be a continuous function. 

We clam that \( f  \) is also differentiable. To do this we show that 
\[  \sum_{ k=1 }^{ \infty  } f_n(x_0)  \]
converges to some function \( f(x_0) \) and then show that 
\[  \sum_{ k=1 }^{ \infty  } \frac{ \cos(x / k ) }{ k^2 }  \] converges uniformly where \( f'_n(x) = \cos(x / k ) / k^2  \). Using the same process to show that \( \sum_{ k=1  }^{  \infty  } f_n \) converges but only for some \( x_0 \in (a,b) \), we know that 
\[  \sum_{ k=1 }^{ \infty  } f_n(x_0) = f(x_0). \]
Now to show that \( \sum_{ k=1 }^{ \infty  } f_n'(x)  \) converges uniformly to some \( \ell(x)  \) in \( \R  \), we can see that \( | \cos(x/k)  |  \leq 1  \) such that 
\[  \sum_{ k=1 }^{ \infty  } \Big| \frac{ \cos(k/x) }{ k^2  }   \Big| \leq \sum_{ k=1 }^{ \infty  } \frac{ 1 }{ k^2  }. \]
Since the series on the right of the above converges (p-series), we have that 
\( \sum_{ k=1 }^{ \infty  } f_n'(x)  \) must converge uniformly using the Weierstrass M-test. Since \( (f_n)  \) differentiable and \( \sum_{ k=1 }^{ \infty  } f_n'(x)  \) converges uniformly to some \( \alpha (x)  \) on \( \R  \), we know that \( f(x)  \) must be differentiable. 

We can use the same process above to show that \( f   \) is twice-differentiable.
\end{proof}

\subsubsection{Exercise 6.4.9} Let 
\[  h(x) = \sum_{ n=1 }^{ \infty  } \frac{ 1 }{ x^2 +n^2  }. \]
\begin{enumerate}
    \item[(a)] Show that \( h  \) is a continuous function defined on all of \( \R  \).
        \begin{proof}
        Note that \( (h_n)  \) is a sequence of rational functions which are continuous. Our goal is to show that \( \sum_{ n=1 }^{ \infty  } h_n(x)  \) converges uniformly where 
        \[  h_n(x) = \frac{ 1 }{ x^2 + n^2  }\] differentiable for all \(  n \in \N  \).  Let \( x \in \R  \) be arbitrary. Since \( h_n(x)  \) reaches a maximum at \( x = 0  \), we know that \( h_n(x) \leq 1 / n^2  \). Thus,
        \[  \sum_{ n=1 }^{ \infty  } \Big| \frac{ 1 }{ x^2 + n^2 } \Big|  \leq \sum_{ n=1 }^{ \infty  } \frac{ 1 }{ n^2  },   \]
        and the fact that \( \sum_{ n=1 }^{ \infty  } 1 / n^{2} \) is a p-series which converges, we know that \( \sum_{ n=1 }^{ \infty  } h_n(x)   \) must converge uniformly to \( h(x)  \). This means that \( h  \) must be continuous on \( \R  \) by the Term-by-term Continuity Theorem.
        \end{proof}
    \item[(b)] Is \( h  \) differentiable? If so, is the derivative function \( h' \) continuous? 
        \begin{proof}[Solution]
        We claim that \( h  \) is a differentiable function. First we compute \( h_n'(x)  \). Using our differentiation rules, we arrive at
        \[  h_n'(x) = \frac{ -2x  }{ (x^2 + n^2 )^2 }.  \]
        Our goal is to show that 
        \[  \sum_{ n=1 }^{ \infty  } h_n'(x)   \]
        converges uniformly to some \( \ell(x)  \) in \( \R  \). Since \( h_n(x)  \) is also secondly differentiable, we can use the Interior Extremum Theorem to find the points on \( \R  \) such that \( h"(x) = 0   \) such that \( h'_n(x)  \) is at its maximum. Differentiating \( h_n'(x)  \) again, we arrive arrive at 
        \[  h_n"(x) = \frac{ 6x^2 - 2n^2  }{ (x^2 + n^2 )^3 }.  \]
        Setting \( h_n''(x_0) = 0  \) produces the following values where \( h'(x)  \) reaches its extrema:
        \[  x_0 = \pm \frac{ n }{ \sqrt{ 3 }  }. \] Plugging this point into \( h_n'(x)  \) where
        \[  h_n'(x_0) = \frac{ 9 }{ 8\sqrt{ 3 }  n^3  }.  \]
        Hence, we can bound \( h_n'(x)  \) by this value. So we have that 
        \[ \sum_{ n=1 }^{ \infty  } | h_n"(x) |  \leq \sum_{ n=1 }^{ \infty  } \frac{ 9 }{ 8 \sqrt{ 3 } n^3  }.  \]
        Since the series on the right side of the above inequality converges, we know that by the  Weierstrass M-test that 
        \[  \sum_{ n=1 }^{ \infty  } h_n'(x) = \ell(x).  \]
        Furthermore, since \( \sum_{ n = 1 }^{  \infty } h_n(x)  \) converges uniformly for all \( x \in \R  \), we know that \( h(x)  \) must be twice-differentiable.
        
        \end{proof}
\end{enumerate}




