
\section{The Axiom of Completeness}

\begin{tcolorbox}
    \begin{thm}[Axiom of Completeness]
        Every nonempty set of real numbers that is bounded above has a least upper bound.
    \end{thm}
\end{tcolorbox}

\begin{tcolorbox}
    \begin{defn}
        We call a set \( A \subseteq \R \) is \textit{bounded above} if there exists a number \( b \in \R \) such that \( a \leq b \) for all \( a \in A \).
        Otherwise, a set is \textit{bounded below} if there exists a \( \ell \in \R \) satisfying \( \ell \leq a \) for every \( a \in A \) .
    \end{defn}

\end{tcolorbox}

\begin{tcolorbox}

    \begin{defn}
        A real number \( s \) is the \textit{least upper bound} for a set \( A \subseteq \R \) if it satisfies the following criteria:

        \begin{enumerate}
            \item[(i)] \( s \) is an upper bound for \( A \);
            \item[(ii)] if \( b \) is any upper bound for \( A \), then \( s \leq b \). 
        \end{enumerate}


        
    \end{defn}


\end{tcolorbox}

We denote the least upper bound of a set \( A \) by calling it the \textit{supremum} of \( A \) i.e \( \sup(A) \). Similarly, we denote the greatest upper bound of set \( A \) by calling it the \textit{infimum} of \( A \) i.e \( \inf(A) \).

Note that a set can have many upper/lower bounds. But there can only exist one supremum and one infimum. In other words, these bounds are unique. Furthemore, the infimum and supremum need not be in the set.

Consider the following set
\[ A = \bigg\{ \frac{1}{n} : n \in \N \bigg\} = \bigg\{ 1,\frac{1}{2}, \frac{1}{3},... \bigg\}\]

This set is bounded above and below. In addition, we can see that \( \sup(A) = 1 \) and \( \inf(A) = 0 \) (this is because each subsequent number in the sequence gets smaller and smaller). 


\begin{tcolorbox}
    \begin{defn}
        We say that \( a_0 \in \R \) is a \textit{maximum} of the set \( A \) if \( a_0 \in A \) and \( a \leq a_0 \) for all \( a \in A \). Likewise, we say that \(a_1 \in \R \) is a \textit{minimum} of \( A \) if \( a_1 \in A \) and \( a \geq a_1 \) for every \( a \in A \).  
    \end{defn}
\end{tcolorbox}

If we have an open set \( (0,2) \) then the end points of this set are the infimum and supremum of the set respectively. Note that the maximum and the minimum do not exists because the infimum and the supremum are not in the set. If this set were to be closed, then the supremum and infimum would be in the set which implies that the max and min exists. 

Now consider the Example

\[ S = \{ r \in \Q: r^2 < 2 \} \]

Notice that when we try and search for the supremum for this set, we cannot find one since we can always find a smaller number for an upper bound. One might say that \( r = \sqrt{2} \) is the supremum of \( S \) but this is false since \( r \not\in \Q \) and is irrational. 


\begin{ex}
    Let \( A \subseteq \R \) such that \( A \neq \emptyset \) and is bounded above. Let \( c \in \R \). Define the set 
    \( c + A \) by 
    \[ c + A = \{ c + a: a \in A  \}  \] 
    Prove that \( \sup(c+A) = c + \sup(A) \) 

    \begin{proof}
        We use defintion 0.2 to prove this proposition. First, we need to prove that this \( \sup(c + A) \) is an upper bound. We have \( \sup(A) = s \) for some \( s \in A \) if \( s \geq a \) for all \( a \in A \). We find that adding \( c \in \R \) gives us
        \[  c+s \geq c + a.\]
        Hence, we have that \( c + s \) is an upper bound for the set \( c + A \). 

        Next, we prove that \( \sup(c+ A) = c + s \) is the \textit{least upper bound}. We know that \( c + s \geq c + a \) for all \( a \in A \). Suppose we have another upper bound \( b \in A \) such that \( c + a \leq b \) for all \( a \in A \). Another manipulation gives us \( a \leq b - c \) for all \( a \in A \). Since \( \sup(A) = s \) is the least upper bound for \( A \), it follows that 
        \( s \leq b - c \). Hence, we have 
        \[ c+s \leq b \implies \sup(c+A) = c + \sup(A). \]
    \end{proof}
\end{ex}

There is another way to restate part (ii) of defintion 0.2 i.e 
\begin{tcolorbox}
\begin{lem}
Assume \( s \in \R \) is an upper bound for a set \( A \subseteq \R \). Then, \( s = \sup A \) if and only if for every \( \epsilon > 0 \), there exists \( a \in A \) such that \( s - \epsilon < a \).
\end{lem}
\end{tcolorbox}

\begin{proof}
For the forward direction, suppose that \( s = \sup A \) and consider \( s - \epsilon \). Since \( s \) is an upper bound, we have that \( s -\epsilon < s \). This means that \( s - \epsilon \) is not an upper bound. Hence, we can find an element \( a \in A \) such that \( s - \epsilon < a \) because otherwise \( s - \epsilon \) would be an upper bound. This concludes the forward direction.

For the backwards direction, assume \( s \) is an upper bound. We must satisfy part (ii) of defintion 0.2. Let \( \epsilon > 0 \), then \( \epsilon = s - b \). But since any number smaller than \( s \) is not an upper bound, we have that \( s \leq b \) if \( b \) is any other upper bound for \( S \).
Hence, \( s = \sup A \). 
\end{proof}




\subsection{Exercises}





\subsubsection{Exercise 1.3.3} \begin{enumerate} \item[(a)] Let \( A \neq \emptyset \) and bounded below, and define \[ B = \{ b \in \R : b \text{ is a lower bound for   } A  \},\] Show that \( \inf A = \sup B \). \begin{proof} Our goal is to show that both \( \inf A \leq \sup B \) and \( \inf A \geq \sup B \). Since \( B \neq \emptyset\) and bounded above, we have that the \( \sup B \) exists. First we want to show that \( \inf A \leq \sup B \). By definition of \( \sup B \), it is the greatest lower bound of \( B \). Since \( A \neq \emptyset \) and bounded below, we have that the \( \sup B \) is greater than any lower bound of \( A \). Hence, we have that \( \inf A \leq \sup B \). Now we want to show that \( \inf A \geq \sup B \). Suppose for sake of contradiction that \( \inf A < \sup B \). Since \( A \neq \emptyset\) and bounded below, we have that \[ a - \epsilon \geq \inf A \tag{1}\] for some \( a \in A \). Our goal is to show that there exists that some \( a \in A \) is less than \( \sup B \). Hence, choose \( \epsilon = \sup B + a \) such that (1) and \( \inf A < \sup B \) implies that \begin{align*} a - \epsilon &< \sup B \\ a - (\sup B + 3a) &< \sup B \\ a &< \sup B. \end{align*}  But this is a contradiction since every element in \(A  \) has to be bigger than \( B \) i.e \( b > a \) for all \( b \in B \). Hence, it must be the case that \(\inf A \geq \sup B  \). Hence, we have that \( \inf A = \sup B \). \end{proof} \item[(b)] Use (a) to explain why there is no need to assert that greatest lower bounds exist as part of the Axiom of Completeness. \begin{proof}[Solution] There is no need to assert that greatest lower bounds exist as part of the axiom because we can always seperate a set \( A \) that is bounded below into a set \( B \) that just consists of lower bounds from \( A \). Since the infimum is just the greatest lower bound, it is equivalent to taking the supremum of a set of lower bounds. We can do this because every element in \( B \) is bounded above by every element in \( A \) which is permitted by the Axiom of Completeness.  \end{proof}
\end{enumerate}

\subsubsection{Exercise 1.3.4}
Let \( A_1, A_2, A_3,... \) be a collection of nonempty sets, each of which is bounded above. 

\begin{enumerate}
    \item Find a formula for \( \sup (A_1 \cup A_2 )\). Extend this to \( \sup \big( \cup_{k=1 }^{n} A_k \big)\). 
    \begin{proof}[Solution]
        For \( \sup (A_1 \cup A_2 )\), we have \[ \sup (A_1 \cup A_2 ) = \sup\{ A_1, A_2 \}\] and for  
        \( \sup \big( \cup_{k=1 }^{n} A_k \big)\), we have 
        \[ \sup \big( \cup_{k=1 }^{n} A_k\big) = \sup\{ A_k \} \]
        for \( k \in \N \). 
    \end{proof}
    \item Consider \( \sup \big( \cup_{k=1 }^{\infty} A_k\big) \). Does the formula in (a)  extend to the infinite case? 
    \begin{proof}[Solution]
        No, because then \( \cup_{k=1 }^{\infty} A_k  \) would be an unbounded set. 
    \end{proof}
\end{enumerate}



\subsubsection{Exercise 1.3.5}
As in Example 1.3.7, let \( A \subseteq \R \) be nonempty and bounded above, and let \( c \in \R \). This time define the set 
\[ cA = \{ca: a \in A \}. \]
\begin{enumerate}
    \item[(a)] If \( c \geq 0 \), show that \( \sup A (cA ) = c\sup A \).
    \begin{proof}
        Suppose \( c \geq 0 \). Since \( A \neq \emptyset \) and bounded above, we have that \( \sup A \) exists. Denote \( \alpha = \sup A \). By definition, we have that \( \alpha \geq a \) for all \( a \in A \). Multiplying by \( c \geq 0 \), we have that 
        \begin{align*}
            c\alpha &\geq ca \\ 
            c\sup A &\geq ca
        \end{align*}
        for all \( a \in A \). This shows that \( c \sup A \) is an upper bound for \( cA \).
        
        Now we want to show that this upper bound is the least upper bound in \( cA \). Hence, take any upper bound in \( b \in A \) such that \( ca \leq b \). This implies that \( a \leq b/c \). Since \( \alpha = \sup A \) is the least upper bound for \( A \), we have that \( \sup A \leq b/c \) which further implies that \( c \sup A \leq b \) showing that it is the least upper bound in \( cA \). Hence, we have that 
        \( \sup A (cA ) = c\sup A \).
    \end{proof}
    \item[(b)] Postulate a similar type of statement for \( \sup (cA) \) for the cases \( c < 0 \).
    \begin{proof}[Postulate]
        For cases \( c < 0 \), we have \( \sup (cA ) = c\inf A \).
    \end{proof} 
\end{enumerate}




\subsubsection{Exercise 1.3.6}
Suppose that \( A,B \neq \emptyset \) and bounded above. Prove that 
\[ 
\sup(A+B) = \sup A + \sup B     
\]
We prove this proposition using two methods. One deals with direct application of the definition and the other deals with using lemma 1.3.8. 


\begin{proof}
Our goal is to show that 
\[ 
\sup(A+B) = \sup A + \sup B     
\]
We know that since \( A,B \neq \emptyset \) and bounded above, we have that \( \sup A, \sup B \) exists. we denote the supremums by the following 
\begin{align*}
    \sup A &= \alpha, \\ 
    \sup B &= \beta.
\end{align*}
It suffices to show that following 
\begin{align}
    \sup(A+B) \leq \sup A + \sup B     
\end{align}
and 
\begin{align}
    \sup(A+B) \geq \sup A + \sup B
\end{align}
We first show (2) first then we will show (1) next. Suppose we have abirtrary \( x \in A \) and \( y \in B \). Because \( A, B \neq \emptyset \) and bounded above, we know that the set \(A + B  \) is also non-empty and bounded above which means its supremum \( \sup(A + B) \) also exists. Hence, we know that 
\[
    x + y \leq \sup(A+B)    
\] 
Subtracting \( y \in B \) to the other side of this inequality will yield
\[ 
    x \leq \sup(A + B) - y     
\]
But we know that since \( x \in A \) and \( \sup A \geq a \) for all \( a \in A \), we have that 
\[ 
  \sup A \leq \sup(A+B) - y.   
\]
Likewise, we isolate \( y \in B \) to the other side and note that \( b \leq \sup B \) for all \( b \in B \). Then we get the following:
\begin{align*}
    y &\leq \sup(A+B) - \sup A  \\
    \sup B &\leq \sup(A+B) - \sup A  \\
\end{align*}
But this implies that 
\[ 
    \sup A + \sup B \leq \sup(A+B)     
\]
Now we show (1). By lemma 1.3.8, we know that for all \( \epsilon > 0 \), we have that 
\begin{align*}
    \sup A - \frac{\epsilon }{2} &< a \\
    \sup B - \frac{\epsilon }{2} &< b
\end{align*}
for some \( a \in A \) and \( b \in B \). Adding these two together we have that 
\begin{align*}
    \sup A + \sup B - \epsilon < a + b 
\end{align*}
But we also know that \( a \) and \( b \) are bounded above by their respective supremums so 
\begin{align*}
    \sup A + \sup B - \epsilon < a + b \leq \sup A + \sup B
\end{align*}
Setting \( \epsilon = \sup A+\sup B-\sup(A +B) \). Hence, we have that 
\[
    \sup(A+B) \leq \sup A + \sup B.    
\]

Since we have (1) and (2), we see that 
\[ 
    \sup(A+B) = \sup A + \sup B
\]

\end{proof}



\subsubsection{Exercise 1.3.7}

Prove that if \( a \) is an upper bound for \( A \), and \( a \in A \), then \( \sup A = a \).
    

\begin{proof}
    We want to show that \( a \leq \sup A \) and \( a \geq \sup A \). We start with the former. Since \( A \neq \emptyset \) and bounded above, we have that the \( \sup A \) exists. Label this supremum as \( \sup A = \beta \). For every \( \epsilon > 0 \), we have that there exists \( b \in A \) such that \( \sup A - \epsilon \leq b \). Choose \( \epsilon = 2 \sup A  - a - b \) such that 
    \begin{align*}
        \sup A - 2 \sup A + a + b &\leq b \\ 
        -\sup A +  a &\leq  0 \\
       \implies a \leq \sup A    
    \end{align*}
    Now for the latter case, since \( \sup A = \beta \) is the least upper bound of \( A \) and 
    \( a \in A \), it follows immediately that \( a \leq \sup A \) for all \( a \in A\). Hence, \( \sup A = a \)


\end{proof}




\subsubsection{Exercise 1.3.8}

\begin{enumerate}
    
    \item[(a)] If \( \sup A < \sup B \), show that there exists an element \( b \in B \) that is an upper bound for \( A \). 
    \begin{proof}
        Suppose \( \sup A < \sup B \). Since we have \( \sup B \), by lemma 1.3.8 we can say that for every \( \epsilon > 0 \), there exists \( b \in B \) such that 
        \[ \sup B - \epsilon < b \tag{1} \] 
        Choose \( \epsilon = \sup B - \sup A \). We can do this because \( \sup A < \sup B \). Hence, (1) implies 
        \begin{align*} 
            \sup B - \epsilon &< b \\
            \sup B - (\sup B - \sup A ) &< b \\ 
            \sup A &< b.
        \end{align*}  
        By definition, \( \sup A \) is the least upper bound for \(A \). Since \( \sup A \geq a\) for all \( a \in A\), it follows that 
        from (1) that \( a < b \) for all \( a \in A\). Hence, for some \( b \in B \), \( b \) is an upper bound for \( A \).    




    \end{proof}
    
    \item[(b)] Give an example to show that this is not always the case if we only assume \( \sup A \leq \sup B \). 


\end{enumerate}

    \subsubsection{Exercise 1.3.10 (Cut Property)}
    If \( A \) and \( B \) are nonempty, disjoint sets with \( A \cup B = \R \) and \( a < b \) for all \( a \in A \) and \( b \in B \), then there exists \( c \in \R \) such that \( x \leq c \) whenever \( x \in A \) and \( x \geq c \) whenever \( x \in B \). 
   
    \begin{enumerate}
    \item[(a)] Use the Axiom of Completeness to prove the Cut Property.
    \begin{proof}
        Suppose \( A \) and \( B \) are nonempty, disjoint sets with \( A \cup B = \R \) and \( a < b \) for all \( a \in A \) and \( b \in B \). By Axiom of Completeness, \( A \) and \( B \) are bounded above and below respectively. This implies that their supremum and infimums exists. 
        
        Firstly, we want to show that there exists \( c \in \R \) such that \( x \leq c \) whenever \( x \in A \). Since \( a < b \) for all \( a \in A\) and \( b \in B \), every \( b \in B \) is an upper bound for \( A \). Denote \( B \) as the set of upper bounds for \( A \). Hence, there must exist \( c \in B \) such that \( c \) is the \textbf{least upper bound} for \( A \) due to the Axiom of Completeness. Furthermore, note that \( \sup A \in B  \) and not in \( A \) since \( A \cap B = \emptyset \) which means \( \sup A \in \R \). Hence, \( \sup A \leq b \). But \( x \in A \) so \( x \leq \sup A \).  
        
        Now we want to show there exists \( c \in \R \) such that \( x \geq c \). Since every \( a \in A \) is a lower bound for \( B \) and that \( B \neq \emptyset \), there must exist an element in \( A  \) such that it is the \textbf{greatest lower bound } for \( B \). Denote this element as \( c = \inf B \). Hence, \( \inf B \geq a \) for all \( a \in A \). Furthermore, \( \inf B \in A \) and not in \( B \) since \( A \cap B = \emptyset\) so \( \inf B \in \R \) when we union \( A \) and \( B \) together. Since \( x \in B \), we have that \( \inf B \leq x  \). 
        
        Furthermore, \( B \) is nonempty and bounded below and \( A \) is the set of lower bounds for \( B \), we have that \( \inf B = \sup A = c \in \R \). 
        \end{proof}
    
        \item[(b)] Show that the implication goes the other way; that is, assume \( \R \) possesses the Cut Property and let \( E \) be a nonempty set that is bounded above. Prove that \( \sup E \) exists.

        \begin{proof}
            Assume \( \R \) possesses the Cut Property and let \( E \neq \emptyset \) that is bounded above. Suppose we have that \( E \subseteq \R \). Since \( \R \) possesses the cut property, we can find \( c \in \R \) such that \( x \leq c \) if \( x \in E \). Since \( A \cap B = \emptyset \), \( c \in A \cup B = \R \). Hence, either \( c \in A \) or \( c \in B \). If \(c \in A \), then \( c \) is not an upper bound for \( E \) since every \(a \in A  \) is less than every \( b \in B \). Furthermore, if \( c \in A \) and \( A \) is the set of lower bounds for \( B \), then it would contradict that \( c \) is an upper bound for \( E \). Thus, we must have \( c \in B \). Since \( c \in B \), \( B \) is the set of upper bounds for \( E \), and \( E \neq \emptyset \) and bounded above, \( c \in B \) is the smallest element in \( B \) which makes it the \textbf{least upper bound} for \( E \). Hence, \( c = \sup E \) exists.
        \end{proof} 
        

    
\end{enumerate}














% \end{document}



