
\section{The Axiom of Completeness}

\begin{tcolorbox}
    \begin{thm}[Axiom of Completeness]
        Every nonempty set of real numbers that is bounded above has a least upper bound.
    \end{thm}
\end{tcolorbox}

\begin{tcolorbox}
    \begin{defn}
        We call a set \( A \subseteq \R \) is \textit{bounded above} if there exists a number \( b \in \R \) such that \( a \leq b \) for all \( a \in A \).
        Otherwise, a set is \textit{bounded below} if there exists a \( \ell \in \R \) satisfying \( \ell \leq a \) for every \( a \in A \) .
    \end{defn}

\end{tcolorbox}

\begin{tcolorbox}

    \begin{defn}
        A real number \( s \) is the \textit{least upper bound} for a set \( A \subseteq \R \) if it satisfies the following criteria:

        \begin{enumerate}
            \item[(i)] \( s \) is an upper bound for \( A \);
            \item[(ii)] if \( b \) is any upper bound for \( A \), then \( s \leq b \). 
        \end{enumerate}


        
    \end{defn}


\end{tcolorbox}

We denote the least upper bound of a set \( A \) by calling it the \textit{supremum} of \( A \) i.e \( \sup(A) \). Similarly, we denote the greatest upper bound of set \( A \) by calling it the \textit{infimum} of \( A \) i.e \( \inf(A) \).

Note that a set can have many upper/lower bounds. But there can only exist one supremum and one infimum. In other words, these bounds are unique. Furthemore, the infimum and supremum need not be in the set.

Consider the following set
\[ A = \bigg\{ \frac{1}{n} : n \in \N \bigg\} = \bigg\{ 1,\frac{1}{2}, \frac{1}{3},... \bigg\}\]

This set is bounded above and below. In addition, we can see that \( \sup(A) = 1 \) and \( \inf(A) = 0 \) (this is because each subsequent number in the sequence gets smaller and smaller). 


\begin{tcolorbox}
    \begin{defn}
        We say that \( a_0 \in \R \) is a \textit{maximum} of the set \( A \) if \( a_0 \in A \) and \( a \leq a_0 \) for all \( a \in A \). Likewise, we say that \(a_1 \in \R \) is a \textit{minimum} of \( A \) if \( a_1 \in A \) and \( a \geq a_1 \) for every \( a \in A \).  
    \end{defn}
\end{tcolorbox}

If we have an open set \( (0,2) \) then the end points of this set are the infimum and supremum of the set respectively. Note that the maximum and the minimum do not exists because the infimum and the supremum are not in the set. If this set were to be closed, then the supremum and infimum would be in the set which implies that the max and min exists. 

Now consider the Example

\[ S = \{ r \in \Q: r^2 < 2 \} \]

Notice that when we try and search for the supremum for this set, we cannot find one since we can always find a smaller number for an upper bound. One might say that \( r = \sqrt{2} \) is the supremum of \( S \) but this is false since \( r \not\in \Q \) and is irrational. 


\begin{ex}
    Let \( A \subseteq \R \) such that \( A \neq \emptyset \) and is bounded above. Let \( c \in \R \). Define the set 
    \( c + A \) by 
    \[ c + A = \{ c + a: a \in A  \}  \] 
    Prove that \( \sup(c+A) = c + \sup(A) \) 

    \begin{proof}
        We use defintion 0.2 to prove this proposition. First, we need to prove that this \( \sup(c + A) \) is an upper bound. We have \( \sup(A) = s \) for some \( s \in A \) if \( s \geq a \) for all \( a \in A \). We find that adding \( c \in \R \) gives us
        \[  c+s \geq c + a.\]
        Hence, we have that \( c + s \) is an upper bound for the set \( c + A \). 

        Next, we prove that \( \sup(c+ A) = c + s \) is the \textit{least upper bound}. We know that \( c + s \geq c + a \) for all \( a \in A \). Suppose we have another upper bound \( b \in A \) such that \( c + a \leq b \) for all \( a \in A \). Another manipulation gives us \( a \leq b - c \) for all \( a \in A \). Since \( \sup(A) = s \) is the least upper bound for \( A \), it follows that 
        \( s \leq b - c \). Hence, we have 
        \[ c+s \leq b \implies \sup(c+A) = c + \sup(A). \]
    \end{proof}
\end{ex}

There is another way to restate part (ii) of defintion 0.2 i.e 
\begin{tcolorbox}
\begin{lem}
Assume \( s \in \R \) is an upper bound for a set \( A \subseteq \R \). Then, \( s = \sup A \) if and only if for every \( \epsilon > 0 \), there exists \( a \in A \) such that \( s - \epsilon < a \).
\end{lem}
\end{tcolorbox}

\begin{proof}
For the forward direction, suppose that \( s = \sup A \) and consider \( s - \epsilon \). Since \( s \) is an upper bound, we have that \( s -\epsilon < s \). This means that \( s - \epsilon \) is not an upper bound. Hence, we can find an element \( a \in A \) such that \( s - \epsilon < a \) because otherwise \( s - \epsilon \) would be an upper bound. This concludes the forward direction.

For the backwards direction, assume \( s \) is an upper bound. We must satisfy part (ii) of defintion 0.2. Let \( \epsilon > 0 \), then \( \epsilon = s - b \). But since any number smaller than \( s \) is not an upper bound, we have that \( s \leq b \) if \( b \) is any other upper bound for \( S \).
Hence, \( s = \sup A \). 
\end{proof}




