% \begin{document}

\section{Consequences of Completeness}

The first application of the Axiom of Completeness is a result that says that the real line contains no gaps. 

\begin{theorem}{}{}
        For each \( n \in \N \), assume we are given a closed interval \( I_n = [a_n, b_n ] = \{ x \in \R : a_n \leq x \leq b_n \} \). Assume also that each \( I_n \) contains \( I_{n+1}\). Then, the resulting nested sequence of closed intervals 
        \[ I_1 \supseteq I_2 \supseteq I_3 \supseteq I_4 \supseteq ... \]
        has a nonempty intersection; that is, \( \cap_{n=1}^{\infty} I_n \neq \emptyset \).
    \end{theorem}

\begin{proof}
    Our goal is to produce a real number \( x \) such that this element is in every closed interval \( I_n \) for every \( n \in \N \). Using the Axiom of Completeness, we can denote the following sets 
    \begin{align*} A &= \{  a_n : n \in \N  \} \\ 
                   B &= \{  b_n : n \in \N  \}
     \end{align*} 
    where \( A \) and \( B \) consists of the left-hand and right-hand endpoints respectively. Since every closed interval are nested, we know that every \( b_n \) serves as an upper bound for \( A \). By the Axiom of completeness, we can say that a supremum exists for \( A \) and we can label this supremum as \( x = \sup A \). By definition, this is an upper bound for \( A \). Hence, we have that \( a_n \leq x \). But since \( x \) is the least upper bound and every \( b_n \in B \) is an upper bound for every \( a_n \in A \), we have that \( x \leq b_n \). Hence, we have that \( a_n \leq x \leq b_n \) which means that \( x \in I_n \) for all \( n \in \N \). This precisely means that \( \cap_{n=1}^{\infty} I_n \neq \emptyset \).
\end{proof}

\subsection{The Density of the Rationals }

\begin{theorem}{Archimedean Property}{}
    \begin{itemize}
        
        \item Given any number \( x \in \R \), there exists an \( n \in \N \) satisfying an \( n \in \N \) satisfying \( n > x \)
        \item Given any real number \( y > 0 \), there exists an \( n \in \N \) satisfying \( 1/n < y \)

    \end{itemize}
    
\end{theorem}

Before we head on to the proof, it is important to notice that \( \N \) is not bounded above and we shall not prove this fact since we are taking this property of the set to be a given just like all the properties that are contained in \( \N, \Z,\)  and \( \Q \). 

\begin{proof}
    Assume for sake of contradiction that \( \N \) is bounded above. Using the Axiom of Completeness, \( \N \) contains a supremum, say, \( \sup \N = \alpha \). Using lemma 1.3.8, we know that there exists \( n \in \N \) such that 
    \[  \alpha - 1 < n \tag{\( \epsilon = 1 \)}.\]
    This impplies that 
    \[ \alpha < n + 1 \]
    but this shows that \( n+1 \in \N \) which is a contradiction because we assumed that \( \alpha \geq n \) for all \( n \in \N \) thereby rendering \( \alpha \) to no longer be an upper bound for \( \N \). Hence, we have that 
    there exists an \( n \in \N \) satisfying an \( n \in \N \) satisfying \( n > x \).
    The second part of this theorem follows immediately by setting \( x = 1/y \).
\end{proof}

\begin{theorem}{Density of \( \Q \) in \( \R \)}{}
        For every two \( a,b \in \R \) with \( a < b \), there exists \( r \in \Q \) such that \( a < r < b \).
    \end{theorem}


\begin{proof}
    Our goal is to choose \( m \in \Z \) and \( n \in \N \) such that 
    \[  a < \frac{ m}{n} < b \tag{1}\]
    The idea is to choose a denominator large enough so that when we increment by size \( \frac{1}{n}\) that it will be too big to increment over the open interval \( (a,b)\). Using the (2) of the Archimedean Property, we choose \( n \in \N \) such that 
    \[ \frac{1}{n} < b - a \tag{2}.\]
    We now need to choose an \( m \in \Z \) such that \( na \) is smaller than this chosen number. A diagram for choosing such a number is helpful. Hence, 

    Judging from our diagram, we can see that 
    \[ m-1 \leq na < m.\]
    Focusing on the left part of the inequality, we can solve (2) for \( a \) and say that 
    \begin{align*}
        m &\leq na + 1  \\ 
            &< n(b - 1/n) + 1 \\ 
            &= nb
    \end{align*} 
    This implies that \( m < nb \) and consequently \( na < m < nb \) which is equivalent to (1). 
    
    
\end{proof}



\section{The Existence of Square Roots}
\begin{theorem}{The Existence of \( \sqrt{ 2 }  \)}{}
    There exists \( \alpha \in \R  \) satisfying \( \alpha^2 = 2 \).
\end{theorem}
\begin{proof}
    Consider the set 
    \[ T = \{ t \in \R : t^2 < 2 \} \]
    and set \( \alpha = \sup T \). We need to show that \( \alpha^2 = 2\). Hence, we need to show cases where \(  \alpha^2 < 2 \) and \( \alpha^2 > 2 \). The idea behind these cases is to produce a contradiction that will show that having either one of these cases will violate the fact that \( \alpha \)is an upper bound for \( T \) and \( \alpha\) is the least upper bound respectively. 

    Assume the first case, \( \alpha^2 < 2 \). We know that \( \alpha\) is an upper bound for \( T \). We need to construct an element that is larger than \( \alpha \). Hence, we construct 
    \[ \alpha + \frac{1}{n} \in T \tag{1}\]
    Squaring (1) we have that 
    \begin{align*}
        \bigg(\alpha + \frac{1}{n}\bigg)^2 &= \alpha^2 + \frac{2\alpha}{n} + \frac{1}{n^2} \\ 
                &< \alpha^2 + \frac{2\alpha}{n} + \frac{1}{n} \\
                &= \alpha^2 + \frac{2\alpha + 1}{n}.
    \end{align*}

    We can use the fact that \( \Q \) is dense in \( \R \) to choose an \( n_0 \in \N \) such that 
    \[ \frac{1}{n_0} < \frac{ 2 - \alpha^2}{2 \alpha + 1 }.\]
    Rearranging we get that 
    \[ \frac{2 \alpha + 1}{n_0} <  2 - \alpha^2 \] 
    and consequently 
    \[\bigg(\alpha + \frac{1}{n_0}\bigg)^2 < \alpha^2 + (2 - \alpha^2) = 2 \]
    But this means that \( \alpha + 1/n_0 \in T \) showing that \( \alpha \) is not an upper bound for \( T \) contradicting our assumption. 

    Now we want to show the other case that \( \alpha^2 < 2 \) cannot happen. Now we need to produce an element in \( T \) such that it is less than \( \alpha \), thereby showing that \( \alpha \) is not the least upper bound of \( T \). Hence, we construct the following element 
    \[ \bigg(\alpha - \frac{1}{n}\bigg)\in T. \]
    Squaring this quantity will give us the following
    \begin{align*}
        \bigg(\alpha - \frac{1}{n}\bigg)^2 &= \alpha^2 -\frac{2\alpha}{n} + \frac{1}{n^2} \\
        &> \alpha^2 - \frac{2 \alpha}{n}.
    \end{align*}
    Like we did before, we get to choose an \( n_0 \in \N \) such that 
    \[ \frac{1}{n_0} > \frac{\alpha^2 - 2}{2 \alpha} \]
    to make 
    \[\bigg(\alpha - \frac{1}{n_0}\bigg)^2 < \alpha^2 - (\alpha^2 - 2) = 2.\]
    But this shows that \( \alpha - \frac{1}{n_0} < \alpha \) showing that \( \alpha \) and that our constructed element contradicts that fact that \( \alpha\) is the least upper bound. 

\end{proof}



