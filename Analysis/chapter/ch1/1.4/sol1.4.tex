\section{Consequences of Completeness}

\subsubsection{Exercise 1.4.1} Recall that \( \mathbb{I}\) stands for the set of irrational numbers. 

\begin{enumerate}
    \item[(a)] Show that if \( a,b \in \Q \), then \( ab \) and \( a + b \) are elements of \( \Q \) as well. 
        
    
    \begin{proof}
        Suppose \( a,b \in \Q \). Then \( p,q,m,n \in \Z \) such that \( n, q \neq 0 \). Hence, \( a = \frac{p}{q} \) and \( b = \frac{m}{n}\). Adding \( a + b \) will give us 
        \begin{align*}
            a + b &= \frac{p}{q} + \frac{m}{n} \\ 
                  &= \frac{pn + mq}{qn}. 
        \end{align*}
        Since \( pq + mn, qn \in \Z \) with \( q,n \neq 0 \), we have that \( a + b \in \Q \). Now we multiply \( a \) and \( b \) together. Then we have 
        \begin{align*}
            ab &= \frac{p}{q} \cdot \frac{m}{n} \\ 
               &= \frac{pm}{qn}.
        \end{align*}
        Since \( pm, qn \in \Z \) and \( q,n \neq 0 \), we have that \( ab \in \Q \).
    \end{proof}

    \item[(b)] Show that if \( a \in \Q  \) and \( t \in \mathbb{I} \) and \( at \in \mathbb{I} \) as long as \( a \neq 0 \). 
    \begin{proof}
       Suppose for sake of contradiction that \( at = r \) where \( r \in \Q \). Solving for \( t \), we have that \( t = \frac{r}{a} \). But this tells us that \( t \in \Q \) since \( r, a \in \Q \) which is a contradics our assumption that \( t \in \mathbb{I} \).
    
    \end{proof}
     
    \item[(c)] Part (a) can be summarised by saying that \( \Q \) is closed under addition and multiplication. Is \( \mathbb{I} \) closed under addition and multiplication? Given two irrational numbers \( s \) and \( t \), what can we say about \( s + t \) and \( st\)?
        
    
    \begin{proof}[Solution]
        We can say that \( s + t \) is an irrational number while \( st \) can either be rational or irrational depending if \( s = t \) or \( s \neq t \). If \( s = t \), then \( st \) is rational and if \( s \neq t \), then \( st \) is irrational. 
    \end{proof}
    
\end{enumerate}

\subsubsection{Exercise 1.4.2} 
    Let \( A \subseteq \R \) be nonempty and bounded above, and let \( s \in \R \) have the property that for all \( n \in \N \), \( s + \frac{1}{n} \) is an upper bound for \( A \) and \( s - \frac{1}{n} \) is not an upper bound for \( A \). \textbf{Show that } \( s = \sup A \).
        
    
    \begin{proof}
        Since \( A \neq \emptyset \) and bounded above, we have that \( \sup A \) exists. Since \( s + \frac{1}{n} \) for all \( n \in \N \) is an upper bound for \( A \), we have that 
        \[ \sup A \leq s + \frac{1}{n} \tag{1} \]
        for all \( n \in \N \). On the other hand, \( s - \frac{1}{n} \) is a lower bound for \( A \). Hence, 
        \[ \sup A > s - \frac{1}{n} \tag{2} \]
        for all \( n \in \N \). We have \( (1) \) and \( (2) \) imply 
        \[ s - \frac{1}{n} < \sup A \leq s + \frac{1}{n}. \tag{3} \]
        This means that either \( \sup A < s, \sup A > s, \) or \( \sup A = s, \). 
        If \( \sup A < s  \), then \( s - \sup A > 0 \). Using the Archimedean Property, we can find an \( n \in \N \) such that 
        \[ s - \sup A > \frac{1}{n}\]
        but this means that \( \sup  A < s - \frac{ 1}{n}\) which contradicts \( (3) \). On the other hand, if \( \sup A > s \), then \( \sup A - s > 0 \). Using the Archimedean property again, we can find an \( n \in \N \) such that 
        \[ \sup A - s > \frac{1}{n} \]
        but this means that \( \sup A > s + \frac{1}{n} \) which is a contradiction since \( \sup A < s + \frac{1}{n} \) from (3). Hence, it must be that \( \sup A = s \). 
    \end{proof}
    
\subsubsection{Exercise 1.4.3}
    Prove that \( \cap_{n=1}^{\infty} (0,1/n) = \emptyset \). Notice that this demonstrates that the intervals in the Nested Interval Property must be closed for the conclusion for the theorem to hold. 
    \begin{proof}
        Suppose \( x \in (0,\frac{1}{n}) \), then \( x > 0 \). By the Archimedean Property, we can find an \( N \in \N \) that is sufficiently large such that \( x > \frac{1}{N} \). But this means that \( x \in (0, 1/n )\) for all \( n \in \N  \). Hence, \( x \not\in \cap_{n=1}^{\infty} (0,\frac{1}{n})\) and then 
        \[ \cap_{n=1}^{\infty} (0,\frac{1}{n}) = \emptyset.\]
    \end{proof}
    
    \subsubsection{Exercise 1.4.4}
    Let \( a < b \) be real numbers and consider the set \( T = \Q \cap [a,b]\). Show that \( \sup T = b \). 

        \begin{proof}
            Let \( a < b \) where \( a,b \in \R \). Consider the following set \( T = \Q \cap [a,b] \). We want to show that \( \sup T = b \). By definition, \( b \) is an upper bound for \( T \) since \( a < b \). All we need to show is that \( b \) is the least upper bound. Hence, we use lemma 1.3.8 and the fact that \( \Q \) is dense in \( \R  \) to state that for every \( \epsilon > 0 \), there exists \( r \in \Q \) such that \( b - \epsilon < r < b \). But this means that \( r \in T \) and \( b - \epsilon \) is not an upper bound for \( T \). Hence, \( \sup T = b \).


        \end{proof}
    

    Another proof for this: 

        \begin{proof}
            Let \( a < b \) where \( a,b \in \R \). Consider the following set \( T = \Q \cap [a,b] \). We want to show that \( \sup T = b \). By definition, \( b \) is an upper bound for \( T \) since \( a < b \). All we need to show is that \( b \) is the least upper bound. Since \( a < b \) where \( a,b \in \R \), we can find \(x \in \Q \) such that \( a < x < b \). Since \( x = \frac{m}{n}\) where \( m,n \in \Z \) with \( n \neq 0 \), we have that \( na < m < nb\). But note that \( nb \) is another upper bound for \( T \) for \( n \) sufficiently large and \( nb > b \) implying that \( b \) is the least upper bound of \( T \). Hence, \( \sup T = b \).
        \end{proof}
    

    \subsubsection{Exercise 1.4.5}

    Using Exercise 1.4.1, supply a proof for Corollary 1.4.4 by considering the real numbers \( a - \sqrt{2} \) and \( b - \sqrt{2}\). 

        \begin{proof}
            Consider the real numbers \( a - \sqrt{p}\) and \( b - \sqrt{p}\) where \( p \) is any prime number. Using the fact that \( \Q \) is dense in \( \R \), we have that 
            \[ a - \sqrt{p} < r < b - \sqrt{p} \] 
            for some \( r \in \Q \). Adding \( \sqrt{p} \) to both sides, we have that 
            \[ a < r + \sqrt{p} < b .\]
            But know that \( r + \sqrt{p} \in \mathbb{I} \) by (c) of Exercise 1.4.1. Hence, \( t = r + \sqrt{p} \). We can follow the same procedure for trancendental numbers and make this conclusion. 
        \end{proof}
    


    \subsubsection{Exercise 1.4.7}

    Finish the proof of Theorem 1.4.5 by showing that the assumption \( \alpha ^2 > 2 \) leads to a contradiction of the fact that \( \alpha = \sup T \). 
        
        
    
    \begin{proof}
        Now we want to show the other case that \( \alpha^2 < 2 \) cannot happen. Now we need to produce an element in \( T \) such that it is less than \( \alpha \), thereby showing that \( \alpha \) is not the least upper bound of \( T \). Hence, we construct the following element 
        \[ \bigg(\alpha - \frac{1}{n}\bigg)\in T. \]
        Squaring this quantity will give us the following
        \begin{align*}
            \bigg(\alpha - \frac{1}{n}\bigg)^2 &= \alpha^2 -\frac{2\alpha}{n} + \frac{1}{n^2} \\
            &> \alpha^2 - \frac{2 \alpha}{n}.
        \end{align*}
        Like we did before, we get to choose an \( n_0 \in \N \) such that 
        \[ \frac{1}{n_0} > \frac{\alpha^2 - 2}{2 \alpha} \]
        to make 
        \[\bigg(\alpha - \frac{1}{n_0}\bigg)^2 < \alpha^2 - (\alpha^2 - 2) = 2.\]
        But this shows that \( \alpha - \frac{1}{n_0} < \alpha \) showing that \( \alpha \) and that our constructed element contradicts that fact that \( \alpha\) is the least upper bound. 
    \end{proof}


    \subsubsection{Exercise 1.4.6}
    Recall that a set \( B \) is dense in \( \R \) if an element of \( B \) can be found between any two real numbers \( a < b \). Which of the following sets are dense in \( \R \)? Take \( p \in \Z \) and \( q \in \N \) in every case. 
    \begin{enumerate}
        \item[(a)]
        The set \( \{ r \in \Q : q \leq 10  \} \)
        \begin{proof}[Solution]
            Yes, since \( a < \frac{p}{10} < \frac{p}{q} < b \). 
            
        \end{proof}
        
        \item[(b)]
        The set of all rationals \( p/q \) such that \( q \) is a power of 2.
        \begin{proof}
            Yes since \( a < \frac{p}{2^n} < b \) for \( n \in \N \). 
        \end{proof}
        
        \item[(c)] 
        The set of all rationals \( p/q \) with \( 10|p| \geq q \)
            \begin{proof}
                
            \end{proof}
        
    \end{enumerate}







