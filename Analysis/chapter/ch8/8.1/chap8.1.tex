\section{The Generalized Riemann Integral}

\subsection{The Riemann Integral as a Limit}

Let \( P = \{ x_{0}, x_{1}, x_{2}, \dots, x_{n} \}  \) be a partition of \( [a,b]  \). A \textit{tagged partition} is one where in addition to the partition \( P \), we choose a sampling point \( c_{k }  \) in each of the subintervals \( [x_{k-1}, x_{k }]  \). We can now define the \textit{Riemann Sum} where given a function \( f: [a,b] \to \R  \), and a tagged partition \( (P , \{ c_{k }  \} _{k=1}^{n}  ) \), the \textit{Riemann sum} generated by this partition is defined by 
\[  R(f,P ) = \sum_{ k=1 }^{ n } f(c_{k }) (x_{k } - x_{k-1}). \]
By definition of the upper sum and the lower sum given in section 7.2, it follows immediately that 
\[  L(f,P) \leq R(f,P) \leq U(f,P)   \] for any bounded function \( f  \). From section 7.2.7, integrability is guaranteed when the supremum of the lower sums and infimum of the upper sums go to the same value. By the inequality above, it is quite clear that we expect \( R(f,P ) \) to also have the same value. We can characterize this by using an \( \epsilon - \delta   \) definition applied to \( R(f,P)  \).
\begin{tcolorbox}
\begin{defn}
    Let \( \delta > 0   \). A partition \( P  \) is \( \delta- \)\textit{fine} if every subinterval \( [x_{k }, x_{k-1} ]  \) satisfies \( x_{k } - x_{k-1} <  \delta  \). In other words, every subinterval has width less than \( \delta  \).
\end{defn}
\end{tcolorbox}

\begin{tcolorbox}
    \begin{thm}[Limit Criterion for Riemann Integrability]
    A bounded function \( f: [a,b] \to \R  \) is Riemann-integrable with 
    \[  \int_{ a }^{ b } f  = A  \] if and only if, for every \( \epsilon > 0 \), there exists \( \delta > 0  \) such that, for any tagged partition \( (P, \{ c_{k } \} ) \) that is \( \delta- \)fine, it follow that 
    \[  | R(f,P) - A  | < \epsilon. \]
    \end{thm}
\end{tcolorbox}

The idea is that partitions become finer with the effect that the approximations get closer to the value of the integral. To rephrase the forwards direction of the theorem above, the integrability of some function implies that the approximations converge to the value of the integral independent of the tags chosen. In the backwards direction, the Riemann sum approximations accumulate around some value \( A  \) which implies that a function is integrable and integrates to \( A  \). 

\begin{proof}
    \( (\Rightarrow)  \) Let us assume that \( f  \) is integrable on \( [a,b]  \). Given \( \epsilon > 0  \), we must show that there exists a \( \delta > 0  \) such that if \( ( P, \{ c_{k } \} ) \) is any tagged partition that is \( \delta- \)fine, then 
    \[  \Big| R(f,P) - \int_{ a }^{ b } f  \Big| < \epsilon. \]
    Since \( f  \) is integrable, we can find a partition \( P_{\epsilon } \) such that 
    \[  U(f, P_{\epsilon }) - L(f, P_{\epsilon })  < \epsilon. \] Let \( M > 0  \) be a bound on \( | f |  \), and let \( n  \) be the number of subintervals of \( P_{\epsilon } \) (this so \( P_{\epsilon }\) ) really consists of \( n + 1  \) points in \( [a,b]  \). We claim that choosing 
    \[  \delta = \frac{ \epsilon  }{ 9nM }  \] has this desired property. 

    Let \( (P, \{ c_{k}    \} ) \) be an arbitrary tagged partition of \( [a,b]  \) that is \( \delta- \)fine, and let \( P' = P \cup P_{\epsilon } \). The key is to establish the string of inequalities 
    \[  L(f,P') - \frac{ \epsilon  }{ 3 }  < L(f,P ) \leq U(f,P) < U(f,P') + \frac{ \epsilon  }{ 3 }.\]
\end{proof}

\subsubsection{Exercise 8.1.1} 
\begin{enumerate}
    \item[(a)] Explain why both the Riemann sum \( R(f,P)  \) and \( \int_{ a }^{ b } f  \) fall between \( L(f,P)  \) and \( U(f,P)  \). 
        \begin{proof}[Solution]
            Let \( (P, \{ c_{k }  \} ) \) be a tagged partition of \( [a,b]  \). Then by definition of by the definition of \( M_{k }  \)  and \( m_{k }   \) (the supremums and infimums  of each subinterval \( [x_{k-1} , x_{k }] \) ), we know that 
            \[  m_{ k } \leq f(c_{ k }  )  \leq M_{k } \] which imply that 
            \[ L(f,P) \leq R(f,P) \leq U(f,P). \]
            If \( f  \) is integrable then \( L(f) = U(f) = \int_{ a }^{ b } f   \). This means 
            \[ L(f,P) \leq \int_{ a }^{ b } f \leq U(f,P).\]
        \end{proof}
        \item[(b)] Explain why \( U(f,P') - L(f, P') < \epsilon / 3.  \)
            \begin{proof}[Solution]
           Let \( \epsilon > 0  \). By the properties of the supremum and infimum, we have 
            \[ U(f,P') < U(f) + \frac{ \epsilon  }{ 6 } \] and 
            \[  L(f,P') > L(f) - \frac{ \epsilon  }{ 6 }. \] Subtracting these two inequalities and assuming \( f  \) is integrable (\( U(f) = L(f) \)), we end up with 
            \[  U(f,P') - L(f,P') < (U(f) - L(f)) + \frac{ \epsilon  }{ 3 } = \frac{ \epsilon  }{ 3 }. \]
            \end{proof}
\end{enumerate}

If we can show \( U(f,P) < U(f,P') + \epsilon / 3   \) (and similarly \( L(f,P') - \epsilon / 3 > L(f,P)  \)), then it will follow that 
\[  \Big| R(f,P) - \int_{ a }^{ b } f  \Big| < \epsilon \]
and the proof will be done. To do this, we can try to estimate the distance between \( U(f,P)  \) and \( U(f, P')  \).
\subsubsection{Exercise 8.1.2} Explain why \( U(f,P) - U(f,P') \geq 0  \).
\begin{proof}[Solution]
If \( P = P' \cup P_{\epsilon }  \), then it follows from lemma 7.2.4 that \( U(f,P) \geq U(f,P')   \) which implies that \( U(f,P) - U(f,P') \geq 0  \). 
\end{proof}

Observe that for any partition, the upper sum takes on the form 
\[  U(f,P) = \sum_{ k=1 }^{ n } M_{k } \Delta x_{k } \] which contains a good number of the \( M_{k }  \) terms cancel out.
\subsubsection{Exercise 8.1.3 } 
\begin{enumerate}
    \item[(a)] In terms of n, what is the largest number of terms of the form \( M_{k} ( x_{k } - x_{k-1} ) \) that could appear in one of \( U(f,P)  \) or \( U(f,P') \) but not the other?
        \begin{proof}[Solution]
            Since \( P_{\epsilon } \) consists of \( n+1  \) points in \( [a,b]  \) and there are three points, that being the two endpoints and our sampling point \( c_{k }  \), we must have at most \( 3(n+1)  \) points.
        \end{proof}
    \item[(b)] Finish the proof in this direction by arguing that 
        \[  U(f,P) - U(f,P') < \frac{ \epsilon  }{ 3 }. \]
        \begin{proof}
        Observe that for all \( k \in \N  \) that \( M_{ k } \leq 3(n+1)M   \) for some \( M > 0  \) from part (a). Since \( P  \) is \( \delta- \)fine, we must have \( \Delta x_{k } < \epsilon / 9nM \). Hence, we must have   
        \begin{align*}
            U(f,P) - U(f, P') &= \sum_{ k=1  }^{ n  } M_{k } \Delta x_{ k }  \\
                              &\leq (3n+3)M \sum_{ k=1 }^{ n } \Delta x_{k } \\  
                              &< (3n+3)M \cdot \frac{ \epsilon  }{ 9nM  } \sum_{ k=1 }^{ n } \\
                              &= (3n+3) \cdot \frac{ \epsilon  }{ 9  } \\ 
                              &< \frac{ \epsilon  }{ 3 }.
        \end{align*}
        The same argument can be applied to the lower sums \( L(f,P)  \) and \( L(f,P') \). Then observe that 
        \[   L(f,P') - \frac{ \epsilon  }{ 3 } <  L(f,P) \leq R(f,P) \leq U(f,P) < U(f,P') + \frac{ \epsilon  }{ 3 }  \] now holds which completes our proof that 
        \[ \Big| R(f,P) - \int_{ a }^{ b } f  \Big| < \epsilon. \]
        \end{proof}
\end{enumerate}





\( (\Leftarrow) \) For the backwards direction, we can assume that \( \epsilon -\delta \)criterion in Theorem 8.1.2 holds and show that \( f  \) is integrable. To show this, we must have the upper sums are close to the lower sums. We now know that it is always the case that 
\[  L(f,P) \leq R(f,P) \leq U(f,P) \] independent of the tags chosen to compute \( R(f,P)  \). 
\subsubsection{Exercise 8.1.4} 
\begin{enumerate}
    \item[(a)] Show that if \( f  \) is continuous, then it is possible to pick tags \( \{ c_{k } \}_{k=1}^n  \) so that 
        \[  R(f,P) = U(f,P). \] Similarly, there are tags for which \( R(f,P) = L(f,P)  \) as well.
        \begin{proof}
            Let \( \{ c_{k }  \}_{k=1}^n  \) be an arbitrary tag on a partition \( P \). Since \( f \) is continuous on the compact set \( [a,b] \), we know that \( f  \) must also be uniformly continuous. Denote the supremums of each subinterval \(  [x_{k-1}, x_{k }] \) by \( M_{k } = f(z_{k })  \) for all \( k  \).  Let \( \epsilon >0  \). Then there exists some \( \delta > 0  \) such that whenever \( | c_{k } - z_{k } | < \delta  \), we have 
            \[   f(c_{k}) - M_{k }     < \frac{ \epsilon  }{ b -a  }. \] Then observe that for any partition \( P  \) of \( [a,b]  \), we have
        \begin{align*}
            R(f,P) - U(f, P) &= \sum_{ k=1 }^{ n } [f(c_{k }) - M_{k } ] \Delta x_{k } \\
                             &< \frac{ \epsilon  }{ b -a  } \sum_{ k=1 }^{ n } \Delta x_{k } \\
                             &= \frac{ \epsilon  }{ b -a  } \cdot b-a = \epsilon. \\
        \end{align*}
        Since \( \epsilon > 0 \) is arbitrary, we must have \( R(f,P) = U(f,P) \). A similar argument can be used to show \( R(f,P) = L(f,P) \).
        \end{proof}
    \item[(b)] If \( f  \) is not continuous, it may not be possible to find tags for which \( R(f,P) = U(f,P) \). Show, however, that given an arbitrary \( \epsilon >0  \), it is possible to pick tags for \( P  \) so that 
        \[  U(f,P) - R(f,P) < \epsilon. \] The analogous statement holds for lower sums.
        \begin{proof}
            Let \( \epsilon > 0  \). Let \( \{ c_{k } \}_P{k=1}^{n}  \) be an arbitrary tag for \( P  \). Since \( | f |  \) is bounded by some \( M > 0  \), we know that the distance between the supremums of each subinterval \( M_{k} \) and each tag \( f(c_{k }) \) can be bounded by \( M  \); that is, we have 
            \[   M_{k } - f(c_{k }) \leq  2Mn. \] Since the partition \( P  \) is \( \delta- \)fine, we know that we can choose \( \delta = \frac{ \epsilon  }{ 2Mn }  \) such that every subinterval \( [x_{k-1}, x_{k}] \) satisfies 
            \[ \Delta x_{k } < \frac{ \epsilon  }{ 2Mn }. \] Then observe that 
        \begin{align*}
            U(f,P) - R(f,P)   &= \sum_{ k=1 }^{ n } [ M_{k } - f(c_{k })]  \Delta x_{k }\\
                              &\leq 2Mn \sum_{ k=1 }^{ \Delta x_{k } } \\ 
                              &< 2Mn \cdot \frac{ \epsilon  }{ 2Mn } \sum_{ k=1 }^{ n } \\
                              &= \epsilon.
        \end{align*}
        The same argument can be applied to show 
        \[  R(f,P) - L(f,P) < \epsilon. \]
        \end{proof}
\end{enumerate}

\subsubsection{Exercise 8.1.5} Use the results of the previous exercise to finish the proof of Theorem 8.1.2.

\begin{proof}
Let \( \epsilon > 0  \). Then let \( (P, \{ c_{k } \}) \) be a tagged partition. Let \( P = P_{1} \cup P_{2} \) be a common refinement. By assumption, we can have 
\begin{align*}
    R(f,P_{1}) - R(f, P_{2}) &= \big[ R(f, P_{1}) - A  \big] + \big[ A - R(f, P_{2})\big]  \\
                             &< \frac{ \epsilon  }{ 4 }  + \frac{ \epsilon  }{ 4 }.
\end{align*} By using the results of part (a) and part(b), we have 
\begin{align*}
    U(f,P) - L(f,P) &= \big[ U(f,P) - R(f,P_{1}) \big]  + \big[ R(f,P_{1}) - R(f,P_{2} \big] \\ 
                    &+ \big[ R(f,P_{2} - L(f,P) \big]  \\
                    &< \frac{ \epsilon  }{ 4 }  + \frac{ \epsilon  }{ 4 }  + \frac{ \epsilon  }{ 4 }  + \frac{ \epsilon  }{ 4  } \\
                    &= \epsilon.
\end{align*}
Hence, \( f  \) is integrable and \( A = \int_{ a }^{ b } f  \).
\end{proof}

\subsection{Gauges and \( \delta(x)- \)fine Partitions}

The main component of the generalized Riemann Integral above is to have \( \delta  \) to be a function of \( x  \). 

\begin{tcolorbox}
\begin{defn}
    A function \( \delta: [a,b] \to \R  \) is called a \textit{gauge} on \( [a,b]  \) if \( \delta(x) > 0  \) for all \( x \in [a,b]  \).
\end{defn}
\end{tcolorbox}

\begin{tcolorbox}
\begin{defn}
    Given a particular gauge \( \delta (x)  \), a tagged partition \( (P, \{ c_{k }  \}_{k=1}^{n} ) \) is \( \delta(x)- \)fine if every subinterval \( [x_{k-1}, x_{k }] \) satisfies \( x_{k-1} - x_{k } < \delta(c_{k})  \). In other words, each subinterval \( [x_{k-1}, x_{k} ] \) has width less than \( \delta(c_{k }) \). 
\end{defn}
\end{tcolorbox}

It's important to note that \( \delta(x)  \) is normally a constant function. The definition above is a more generalized version of what was stated earlier in definition 8.1.4. 

\subsubsection{Exercise 8.1.6} Consider the interval \( [0,1]  \). 
\begin{enumerate}
    \item[(a)] If \( \delta(x) = 1/ 9 \), find a \( \delta(x)-\)fine tagged partition of \( [0,1]  \). Does the choice of tags matter in this case? 
        \begin{proof}[Solution]
        Since \( \delta(x) \) is just a constant, the choice of tags does not matter in this case.
        \end{proof}
    \item[(b)] Let 
        \[ \delta(x) = 
        \begin{cases}
            1/4 \ &\text{ if } x = 0 \\
            x/3  &\text{ if  } 0  < x \leq 1.
        \end{cases}  \]
        Construct a \( \delta(x)- \)fine tagged partition of \( [0,1] \).  
    \begin{proof}[Solution]
        Let \( P = \{ ([0, 1/7], 1/2), (\{ 1/2, 2/3 \}, 0 ) \}, (\{ 2/3, 1 \}, 1) \) is a \( \delta(x)- \)fine partition, then observe that 
        \begin{align*}
            x_{1} - x_{0} &< \delta(c_{1}) \implies \frac{ 1 }{ 7 } < \frac{ 1 }{ 6 }. \\
        \end{align*}
        and 
        \begin{align*}
            x_{2} - x_{1} < \delta(c_{2}) \implies \frac{ 11 }{ 21 } < \frac{ 1 }{ 4 } \\
        \end{align*}
        and then finally, 
        \begin{align*}
            x_{3} - x_{2} &< \delta(c_{3}) \implies \frac{ 1 }{ 3 } < \frac{ 1 }{ 2 }. \\
        \end{align*}

    \end{proof}
\end{enumerate}

\begin{tcolorbox}
\begin{thm}
    Given a gauge \( \delta(x)  \) on an interval \( [a,b]  \), there exists a tagged partition \( (P, \{ c_{k }  \}_{k=1}^{n}) \) that is \( \delta(x)- \)fine.
\end{thm}
\end{tcolorbox}
\begin{proof}
    Let \( I_{0} = [a,b]  \). It may be possible to find a tag that the trivial partition \( P = \{ a,b  \}  \) works. Specifically, if \( b-a < \delta(x)  \) for some \( x \in [a,b]  \), then we can set \( c_{1} \) equal to such an \( x  \) and notice that \( (P, \{ c_{1} \} ) \) is \( \delta(x)-\)fine. If no such \( x  \) exists, then bisect \( [a,b] \) into two equal halves.
\end{proof}

\subsubsection{Exercise 8.1.7} Finish the proof of Theorem 8.1.5.
\begin{proof}
    \textbf{To do}.
\end{proof}

\subsection{Generalized Riemann Integrability} 
Using gauges now, we can restate Riemann Integrability in a more generalized way.
\begin{tcolorbox}
\begin{defn}
    A function \( f  \) on \( [a,b]  \) has \textit{generalized Riemann Integral A} if, for every \( \epsilon > 0  \), there exists a gauge \( \delta(x)  \) on \( [a,b]  \) such that for each tagged partition \( (P, \{ c_{k } \}_{k=1}^n) \) that is \( \delta(x)- \)fine, it is true that 
    \[  | R(f,P) - A  | <\epsilon. \] In this case, we write \( A = \int_{ a }^{ b } f \).
\end{defn}
\end{tcolorbox}

\begin{tcolorbox}
\begin{thm}
If a functino has a generalized Riemann integral, then the value of the integral is unique.
\end{thm}
\end{tcolorbox}

\begin{proof}

\end{proof}






