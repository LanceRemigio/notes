\section{The Generalized Riemann Integral}

\subsection{The Riemann Integral as a Limit}

Let \( P = \{ x_{0}, x_{1}, x_{2}, \dots, x_{n} \}  \) be a partition of \( [a,b]  \). A \textit{tagged partition} is one where in addition to the partition \( P \), we choose a sampling point \( c_{k }  \) in each of the subintervals \( [x_{k-1}, x_{k }]  \). We can now define the \textit{Riemann Sum} where given a function \( f: [a,b] \to \R  \), and a tagged partition \( (P , \{ c_{k }  \} _{k=1}^{n}  ) \), the \textit{Riemann sum} generated by this partition is defined by 
\[  R(f,P ) = \sum_{ k=1 }^{ n } f(c_{k }) (x_{k } - x_{k-1}). \]
By definition of the upper sum and the lower sum given in section 7.2, it follows immediately that 
\[  L(f,P) \leq R(f,P) \leq U(f,P)   \] for any bounded function \( f  \). From section 7.2.7, integrability is guaranteed when the supremum of the lower sums and infimum of the upper sums go to the same value. By the inequality above, it is quite clear that we expect \( R(f,P ) \) to also have the same value. We can characterize this by using an \( \epsilon - \delta   \) definition applied to \( R(f,P)  \).
\begin{tcolorbox}
\begin{defn}
    Let \( \delta > 0   \). A partition \( P  \) is \( \delta- \)\textit{fine} if every subinterval \( [x_{k }, x_{k-1} ]  \) satisfies \( x_{k } - x_{k-1} <  \delta  \). In other words, every subinterval has width less than \( \delta  \).
\end{defn}
\end{tcolorbox}

\begin{tcolorbox}
    \begin{thm}[Limit Criterion for Riemann Integrability]
    A bounded function \( f: [a,b] \to \R  \) is Riemann-integrable with 
    \[  \int_{ a }^{ b } f  = A  \] if and only if, for every \( \epsilon > 0 \), there exists \( \delta > 0  \) such that, for any tagged partition \( (P, \{ c_{k } \} ) \) that is \( \delta- \)fine, it follow that 
    \[  | R(f,P) - A  | < \epsilon. \]
    \end{thm}
\end{tcolorbox}

The idea is that partitions become finer with the hope that the approximations get closer to the value of the integral. To rephrase the forwards direction of the theorem above, the integrability of some function implies that the approximations converge to the value of the integral independent of the tags chosen. In the backwards direction, the Riemann sum approximations accumulate around some value \( A  \) which implies that a function is integrable and integrates to \( A  \). 




