\section{Euler's Sum}

Recall Euler's famous series derivation
\[  1 + \frac{ 1 }{ 4 }  + \frac{ 1 }{ 9  }  + \frac{ 1 }{ 16 } + \frac{ 1 }{ 25 } + \dotsb = \frac{ \pi^{2} }{ 6 } \] which used the Taylor series representation 
\[  \sin(x) = x - \frac{ x^{3} }{ 3! } + \frac{ x^{5} }{ 5! } - \frac{ x^{7} }{ 7! } + \dotsb. \tag{1} \] There is also the infinite product representation 
\[  \sin(x) = x \Big( 1 - \frac{ x }{ \pi }  \Big) \Big( 1 + \frac{ x }{ \pi }  \Big) \Big(  1 - \frac{ x }{ 2 \pi }  \Big) \Big( 1 + \frac{ x  }{ 2 \pi }  \Big) \dotsb. \tag{2} \] We have developed the sufficient theory to show why (1) is true, but not (2). There have been many derivations for (2) using multi-variable calculus, Fourier series, and even complex integration. However, we will try to show (2) by using the properties of uniformly convergent series and Taylor series expansions.

\subsection{Walli's Product}

We currently don't have enough machinery at our disposal to be able to prove the infinite product representation of \( \sin(x)  \) in (2), but we can prove the special case when 
\[  \frac{ \pi }{ 2 }  = \lim_{ n \to \infty  }  \prod_{n=1}^{n} \Big(  \frac{ 2n \cdot 2n  }{ (2n-1)(2n+1) }  \Big) \tag{3} \] where (3) is the partial products of (2) but with \( x = \pi / 2  \).

\subsubsection{Exercise 8.3.1} Supply the details to show (3) above.
\begin{proof}
Plugging in \( x = \pi / 2  \) into (2), we get that 
\[  1 = \frac{ \pi }{ 2 }  \prod_{n=1}^{\infty } \Big( 1 - \frac{ 1 }{ 2n }  \Big)\Big( 1 + \frac{ 1 }{ 2n }  \Big) =  \frac{ \pi }{ 2 }  \prod_{n=1}^{\infty } \frac{ (2n-1)(2n+1) }{ (2n)^{2} }.  \]
Taking the reciprocal of the infinite product above, we end up with
\[  \prod_{n=1}^{\infty } \frac{ (2n)^{2} }{  (2n-1) (2n+1) } = \frac{ \pi }{ 2 }.  \]
\end{proof}

Now we will prove why (3) holds. Set 
\[  b_{n} = \int_{ 0 }^{ \frac{ \pi }{ 2 }  } \sin^{n}(x)  \ dx , \text{ for } n = 0,1,2, \ \dots \] If we look at the \( n=0  \) and \( n = 1  \) case, we can easily obtain the following equations 
\[  b_{0 } = \int_{ 0 }^{ \frac{ \pi }{ 2 }  }  dx = \frac{ \pi }{ 2 } \ \text{and} \ b_{1} = \int_{ 0 }^{ \frac{ \pi }{ 2 }  } \sin(x)  \ dx = 1. \]
\subsubsection{Exercise 8.3.2} Assume \( h(x)  \) and \( k(x)  \) have continuous derivatives on \( [a,b]  \), and derive the integration-by-parts formula
\[  \int_{ a }^{ b } h(t) k'(t) \  dt = h(b)k(b) - h(a)k(a) - \int_{ a }^{ b } h'(t) k(t)  \  dt. \]
\begin{proof}[Solution]
Refer to the solution in part (a) of Exercise 7.5.6.
\end{proof}

\subsubsection{Exercise 8.3.3} \begin{enumerate}
    \item[(a)] Using the simple identity \( \sin^{n}(x) = \sin^{n-1}(x) \sin(x)  \) and the previous exercise, derive the recurrence relation 
        \[  b_{n} = \frac{ n-1 }{ n }  b_{n-2} \ \text{for all } n \geq 2. \]
        \begin{proof}
        
        \end{proof}
    \item[(b)] Use this relation to generate the first three even terms and the first three odd terms of the sequence \( (b_{n}) \).
        \begin{proof}[Solution]
        The first three even terms are 
		\begin{align*}
		    b_{2} &= \frac{ 1 }{ 2 }  b_{0} = \frac{ 1 }{ 2 } \cdot \frac{ \pi }{ 2 } = \frac{ \pi }{ 4 },  \\
			b_{4} &= \frac{ 3 }{ 4 } b_{2} = \frac{ 3 }{ 4 }  \cdot \frac{ \pi }{ 4 }  = \frac{ 3 \pi   }{ 16 }, \\
			b_{6} &= \frac{ 5 }{ 6 } b_{4} = \frac{ 5 }{ 6 } \cdot \frac{ 3 \pi }{ 16 } = \frac{ 5 \pi }{ 32 }. 
		\end{align*}
		The first odd terms are 
		\begin{align*}
		    b_{3} &= \frac{ 2 }{ 3 }  b_{1} = \frac{ 2 }{ 3 } \cdot 1 = \frac{ 2 }{ 3 }    \\
			b_{5} &= \frac{ 4 }{ 5 } b_{3} = \frac{ 4 }{ 5 }  \cdot \frac{ 2 }{ 3 } = \frac{ 8 }{ 15 }  \\
			b_{7} &= \frac{ 6 }{ 7 }  b_{5} = \frac{ 6 }{ 7 }  \cdot \frac{ 8 }{ 15 }  = \frac{ 16 }{ 35 }.
		\end{align*}
        \end{proof}
    \item[(c)] Write a general expression for \( b_{2n} \) and \( b_{2n+1}  \).
        \begin{proof}[Solution]
        
        \end{proof}
\end{enumerate}


