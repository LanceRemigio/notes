\section{Euler's Sum}

Recall Euler's famous series derivation
\[  1 + \frac{ 1 }{ 4 }  + \frac{ 1 }{ 9  }  + \frac{ 1 }{ 16 } + \frac{ 1 }{ 25 } + \dotsb = \frac{ \pi^{2} }{ 6 } \] which used the Taylor series representation 
\[  \sin(x) = x - \frac{ x^{3} }{ 3! } + \frac{ x^{5} }{ 5! } - \frac{ x^{7} }{ 7! } + \dotsb. \tag{1} \] There is also the infinite product representation 
\[  \sin(x) = x \Big( 1 - \frac{ x }{ \pi }  \Big) \Big( 1 + \frac{ x }{ \pi }  \Big) \Big(  1 - \frac{ x }{ 2 \pi }  \Big) \Big( 1 + \frac{ x  }{ 2 \pi }  \Big) \dotsb. \tag{2} \] We have developed the sufficient theory to show why (1) is true, but not (2). There have been many derivations for (2) using multi-variable calculus, Fourier series, and even complex integration. However, we will try to show (2) by using the properties of uniformly convergent series and Taylor series expansions.

\subsection{Walli's Product}

We currently don't have enough machinery at our disposal to be able to prove the infinite product representation of \( \sin(x)  \) in (2), but we can prove the special case when 
\[  \frac{ \pi }{ 2 }  = \lim_{ n \to \infty  }  \prod_{n=1}^{n} \Big(  \frac{ 2n \cdot 2n  }{ (2n-1)(2n+1) }  \Big) \tag{3} \] where (3) is the partial products of (2) but with \( x = \pi / 2  \).

\subsubsection{Exercise 8.3.1} Supply the details to show (3) above.
\begin{proof}
Plugging in \( x = \pi / 2  \) into (2), we get that 
\[  1 = \frac{ \pi }{ 2 }  \prod_{n=1}^{\infty } \Big( 1 - \frac{ 1 }{ 2n }  \Big)\Big( 1 + \frac{ 1 }{ 2n }  \Big) =  \frac{ \pi }{ 2 }  \prod_{n=1}^{\infty } \frac{ (2n-1)(2n+1) }{ (2n)^{2} }.  \]
Taking the reciprocal of the infinite product above, we end up with
\[  \prod_{n=1}^{\infty } \frac{ (2n)^{2} }{  (2n-1) (2n+1) } = \frac{ \pi }{ 2 }.  \]
\end{proof}

Now we will prove why (3) holds. Set 
\[  b_{n} = \int_{ 0 }^{ \frac{ \pi }{ 2 }  } \sin^{n}(x)  \ dx , \text{ for } n = 0,1,2, \ \dots \] If we look at the \( n=0  \) and \( n = 1  \) case, we can easily obtain the following equations 
\[  b_{0 } = \int_{ 0 }^{ \frac{ \pi }{ 2 }  }  dx = \frac{ \pi }{ 2 } \ \text{and} \ b_{1} = \int_{ 0 }^{ \frac{ \pi }{ 2 }  } \sin(x)  \ dx = 1. \]
\subsubsection{Exercise 8.3.2} Assume \( h(x)  \) and \( k(x)  \) have continuous derivatives on \( [a,b]  \), and derive the integration-by-parts formula
\[  \int_{ a }^{ b } h(t) k'(t) \  dt = h(b)k(b) - h(a)k(a) - \int_{ a }^{ b } h'(t) k(t)  \  dt. \]
\begin{proof}[Solution]
Refer to the solution in part (a) of Exercise 7.5.6.
\end{proof}

\subsubsection{Exercise 8.3.3} \begin{enumerate}
    \item[(a)] Using the simple identity \( \sin^{n}(x) = \sin^{n-1}(x) \sin(x)  \) and the previous exercise, derive the recurrence relation 
        \[  b_{n} = \frac{ n-1 }{ n }  b_{n-2} \ \text{for all } n \geq 2. \]
        \begin{proof}
            Let \( h(x) = \sin^{n}(x)  \) and \( k'(x) = \sin(x)  \). Let \( n \geq 2  \). Then by the integration-by-parts formula and using the trigonometric identity \( \sin^{2}(x) + \cos^{2}(x) = 1  \), we must have 
            \begin{align*}
                \int_{ 0 }^{ \frac{ \pi  }{ 2  }   }  \sin^{n}(x) \ dx &= \int_{ 0 }^{ \frac{ \pi }{ 2 }  } \sin^{n}(x) \cdot \sin(x) \  dx  \\
                                                                       &= \Big[ - \sin^{n-1}(x) \cdot \cos(x)  \Big]_{0}^{\frac{ \pi }{ 2 }} + \int_{ 0 }^{ \frac{ \pi }{ 2 }  } (n-1)\sin^{n-2}(x) \cdot \cos^{2}(x)  \  dx \\
                                                                       &=  \Big[ - \sin^{n-1}(x) \cdot \cos(x)  \Big]_{0}^{\frac{ \pi }{ 2 }} +  \int_{ 0 }^{ \frac{ \pi }{ 2 }  } (n-1)\sin^{n-2}(x) \cdot  [ 1 - \sin^{2}(x)  ]  \  dx
            \end{align*}
        The first term on the last equality cancels out and the second term can be expanded into 
        \begin{align*}
            \int_{ 0 }^{ \frac{ \pi }{ 2 }  }  (n-1 ) \sin^{n-2}(x) \cdot [ 1 - \sin^{2}(x) ] \  dx 
                                                                                                    &= \int_{ 0 }^{ \frac{ \pi }{ 2 }   } (n-1) \sin^{n-2}(x)  \   dx \\ &+ \int_{ 0 }^{ \frac{ \pi }{ 2 }  } (n-1) \sin^{n}(x) \  dx. \\
        \end{align*}
        Hence, we end up with 
        \begin{align*}  \int_{ 0 }^{ \frac{ \pi }{ 2 }  }  \sin^{n}(x) \  dx &=  \int_{ 0 }^{ \frac{ \pi }{ 2 }   } (n-1) \sin^{n-2}(x)  \   dx + \int_{ 0 }^{ \frac{ \pi }{ 2 }  } (n-1) \sin^{n}(x) \  dx. \tag{1}  \\
        \end{align*}
        Finally, subtracting the second term on the right side of (1), simplifying, and dividing by \( n  \) on both sides gives us our desired result
        \begin{align*}
            b_{n}  &= \int_{ 0 }^{ \frac{ \pi }{ 2 }  }  \sin^{n}(x) \ dx \\ &= \frac{ n-1 }{ n } \int_{ 0 }^{ \frac{ \pi }{ 2 }  } \sin^{n-2}(x) \ dx \\
                                                                &= \frac{ n-1 }{ n } b_{n-2}.
        \end{align*}
        \end{proof}
    \item[(b)] Use this relation to generate the first three even terms and the first three odd terms of the sequence \( (b_{n}) \).
        \begin{proof}[Solution]
        The first three even terms are 
		\begin{align*}
		    b_{2} &= \frac{ 1 }{ 2 }  b_{0} = \frac{ 1 }{ 2 } \cdot \frac{ \pi }{ 2 } = \frac{ \pi }{ 4 },  \\
			b_{4} &= \frac{ 3 }{ 4 } b_{2} = \frac{ 3 }{ 4 }  \cdot \frac{ \pi }{ 4 }  = \frac{ 3 \pi   }{ 16 }, \\
			b_{6} &= \frac{ 5 }{ 6 } b_{4} = \frac{ 5 }{ 6 } \cdot \frac{ 3 \pi }{ 16 } = \frac{ 5 \pi }{ 32 }. 
		\end{align*}
		The first odd terms are 
		\begin{align*}
		    b_{3} &= \frac{ 2 }{ 3 }  b_{1} = \frac{ 2 }{ 3 } \cdot 1 = \frac{ 2 }{ 3 }    \\
			b_{5} &= \frac{ 4 }{ 5 } b_{3} = \frac{ 4 }{ 5 }  \cdot \frac{ 2 }{ 3 } = \frac{ 8 }{ 15 }  \\
			b_{7} &= \frac{ 6 }{ 7 }  b_{5} = \frac{ 6 }{ 7 }  \cdot \frac{ 8 }{ 15 }  = \frac{ 16 }{ 35 }.
		\end{align*}
        \end{proof}
    \item[(c)] Write a general expression for \( b_{2n} \) and \( b_{2n+1}  \).
        \begin{proof}[Solution]
        Using the formula we derived in part (a), plugging in the desired cases gives us
        \[  b_{2n} = \frac{ 2n -1  }{ 2n  }  b_{2(n-1)} \ \text{ and } \  b_{2n+1} = \frac{ 2n }{ 2n+1 } b_{2n-1}.  \]

        \end{proof}
\end{enumerate}

For the \( (n+1) \)th term, we have the following bound \(  0 \leq \sin^{n+1}(x) \leq \sin^{n}(x)  \) on \( [0, \pi / 2]  \). But this tells us that \( (b_{n})  \) is a decreasing sequence of functions.  Since \( (b_{n}) \) is bounded and decreasing, we know that it must converge. It turns out that \( (b_{n}) \to 0  \) but this isn't the limit that we want to concern ourselves at the moment. 

\subsubsection{Exercise 8.3.4} Show 
\[  \lim_{ n \to \infty  }  \frac{ b_{2n} }{  b_{2n+1} }  = 1,  \] and use this fact to finish the proof of Walli's product formula in (3).
\begin{proof}
For \( k \geq 1  \), observe that
\begin{align*}
  \frac{ b_{2n} }{  b_{2n+1} }   &= \frac{ (2n-1) (2n+1)  }{ (2n)(2n) } \cdot \frac{ b_{2n-2}  }{ b_{2n-1} }   \\
                                 &= \frac{ (2n-1) (2n+1)  }{ (2n)(2n)  } \cdot \frac{ (2n-3) (2n-1)  }{ (2n-2) (2n-2)  } \cdot \frac{ b_{2n-4}  }{ b_{2n-3} } . \\ 
\end{align*}
Notice when expanding the terms on the numerator and the denominator of \( b_{2n} / b_{2n+1} \), we will always have the same coefficient. Hence, the limit of \( b_{2n} / b_{2n+1} \) gives us our result that 
\[  \lim_{ n \to \infty  }  \frac{ b_{2n}  }{  b_{2n+1} } = 1. \]
\end{proof}

Some techniques to dealing with the notation in (3) is to use the following equations 
\[  2 \cdot 4 \cdot 6 \dotsb (2n) = 2^{n} n! \] and
\[  1 \cdot 3 \cdot 5 \cdot \dotsb (2n+1) = \frac{ (2n+1)! }{  2 \cdot 4 \cdot 6 \dotsb (2n)  } = \frac{ (2n+1)! }{ 2^{n} n! }. \]


\subsubsection{Exercise 8.3.5} Derive the following alternative form of Walli's product formula: 
\[  \sqrt{ \pi }  = \lim_{ n \to \infty    }  \frac{ 2^{2n} (n!)^2  }{ (2n)! \sqrt{ n }  }. \]
\begin{proof}
\end{proof}

\subsection{Taylor Series} 

To prove (2), we need to somehow generate the Taylor series for \( \arcsin(x)  \). This can't be done directly from Taylor's Formula for the coefficients. We need to first find the expansion for \( 1 / \sqrt{ 1 - x  }  \) by dealing with  

\[  (\arcsin(x))' = \frac{ 1 }{ \sqrt{ 1 - x^{2} }  } \] first. 
\subsubsection{Exercise 8.3.6} Show that \( 1 / \sqrt{ 1 - x  }  \) has Taylor expansion \( \sum_{ n=0 }^{ \infty  } c_{n} x^{n}  \), where \( c_{0} = 1  \) and 
\[  c_{n} = \frac{ (2n)!  }{  2^{2n} (n!)^{2}  } = \frac{ 1 \cdot 3 \cdot 5 \dotsb (2n-1) }{ 2 \cdot 4 \cdot 6 \dotsb 2n }  \] for \( n \geq 1  \).
\begin{proof}
Let \( f(x) = 1 / \sqrt{ 1 - x  }  \). Using Taylor's coefficient formula, we have the first three derivatives of \( f  \)
\begin{align*}
    f^{(1)}(x) &= \frac{ 1 }{ 2 } \cdot (1 - x )^{-3/2}, \\
    f^{(2)}(x) &= \frac{ 1  }{ 2 }  \cdot \frac{ 3 }{ 2 } \cdot  (1-x)^{-5/2}, \\
    f^{(3)}(x) &= \frac{ 1 }{ 2 }  \cdot \frac{ 3 }{ 2 }  \cdot \frac{ 5 }{ 2 } \cdot (1- x)^{-5/2}.
\end{align*}
For \( n  \geq 1 \), we can use induction to show  
\[  f^{(n)}(x) =  \Big[ \prod_{k=1}^{n} \frac{ 2k-1 }{ 2k }   \Big] (1 -x )^{-(2n+1)/2}. \] Plugging in \( x = 0  \) and using the techniques given to us above, we now have the desired formula
\[  c_{n} = \prod_{k=1}^{n} \frac{ 2k-1 }{ 2k } = \frac{ (2n)!  }{ 2^{2n}  (n!)^2   }  \] 
where 
\[  \frac{ 1 }{ \sqrt{ 1-x  }  }  = \sum_{ n=0 }^{ \infty  } c_{n} x^{n}. \]

\end{proof}

Observe that the coefficients above should look familiar to the formulas produced from Walli's product.

\subsubsection{Exercise 8.3.7} Show that \( \lim c_{n} =  0  \) but \( \sum_{ n=0  }^{ \infty  } c_{n}  \) diverges.
\begin{proof}
The first statement is shown in Exercise 2.7.10. Observe that 
\[  c_{n} \geq \frac{ 1 }{ 2^{2n} } \geq \frac{ 1 }{ n }. \] Since \( \sum 1 / n  \) diverges, we must also have \( \sum c_{n}   \) diverge by the Comparison test. 
\end{proof}

Now our goal is to establish at which particular points in the domain of \( f  \) where 
\[  \frac{ 1 }{ \sqrt{ 1- x  }  }  = \sum_{ n=0 }^{ \infty  } c_{n} x^{n}  \] is valid. This can be done by using Lagrange's Remainder Theorem. 

To properly show that 
\[  \frac{ 1 }{ \sqrt{ 1-x  }  }  = \sum_{ n=0  }^{ \infty  } c_{n} x^{n}  \]
holds for all \( x \in (-1,1)  \), we need to show that the error function 
\[  E_{N}(x) = \frac{ 1 }{ \sqrt{ 1- x  }  }  - \sum_{ n=0  }^{ N  } c_{n } x^{ n} \]
approaches zero as \( N \to \infty  \). This can be done using Lagrange's Remainder Theorem (Theorem 6.6.3).


\subsubsection{Exercise 8.3.8} Using the expression for \( E_{N}(x)  \) from Lagrange's Remainder Theorem, show that equation (4) is valid for all \( | x  |  < 1 / 2  \). What goes wrong when we try try to use this method to prove (4) for \( x \in (1/2, 1 ) \)?
\begin{proof}
Since \( f  \) is \( N+1  \) times differentiable on \( (-1/2 , 1/ 2 ) \), there exists a \( c  \) such that \( |  c  |  < | x  |  \) where the error function \( E_{N}(x)  \) satisfies 
\[  E_{N}(x) = \frac{ f^{(N+1)}(c)  x^{n }  }{ (N+1)! } \] by Lagrange's Remainder Theorem. 
Observe that 
\[  f^{(N+1)}(c) = (1 -c )^{-(2N+3)/2} < \Big( \frac{ 2 }{ 3 }  \Big)^{ (2N+3)/ 2}.\]
Since \( | x  |  < 1/ 2  \) and \( |  c  | < | x  |  \), we can now write
\[ E_{N}(x) < \frac{  2^{3/2}}{  3^{(2N+3)/2} (N+1)! } \xrightarrow{N\rightarrow\infty} 0.   \]
Hence, (4) holds for all \( x \in (-1,1) \).
\end{proof}

\subsection{The Integral Form of the Remainder}

The goal of the previous exercise is to recognize a different method is needed to estimate the error function \( E_{N}(x)   \). The following theorem is one such way to do this.

\begin{tcolorbox}
    \begin{thm}[Integral Remainder Theorem]
    Let \( f  \) be differentiable \( N+1  \) times on \( (-R,R ) \) and assume \( f^{(N+1)}  \) is continuous. Define \( a_{n} = f^{(n)} (0) / n !   \) for \( n = 0,1 , \dots , N  \), and let 
    \[  S_{N}(x) = \sum_{ k=0 }^{ N   } a_{k } x^{k }.  \] For all \( x \in (-R ,R ) \), the error function \( E_{N}(x) = f(x) - S_{N}(x)   \) satisfies 
    \[  E_{N}(x) = \frac{ 1 }{ N! } \int_{ 0 }^{ x  }  f^{(N+1) }(t) (x-t)^{N} \  dt.  \]
    \end{thm}
\end{tcolorbox}

\begin{proof}

\end{proof}
