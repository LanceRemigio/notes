\section{Metric Spaces and the Baire Category Theorem}

In this section, we aim to give a more generalized view of what it means to have a "distance" over sets other than \( \R  \). Do the theorems and properties we have proved about sequences, series, and functions carry over to sets like \( \R^{2} \) or even in higher dimensions like \( \R^n \)? We will be mainly testing our notions that we have developed throughout the book on sets such as \( \R^{2} \) and \( C[0,1] \), the space of continuous functions on \( [0,1] \).


\begin{tcolorbox}
\begin{defn}
Given a set \( X  \), a function \( d: X \times X \to \R  \) is a \textit{metric} on \( X  \) if for all \( x,y \in X  \):
\begin{enumerate}
    \item[(i)]\( d(x,y) \geq 0  \) with \( d(x,y) =0  \) if and only if \( x = y  \),
    \item[(ii)] \( d(x,y) = d(y,x)  \), and 
    \item[(iii)] for all \( z \in X \), \( d(x,y) \leq d(x,z) + d(z,y) \).
\end{enumerate}
A \textit{metric space} is a set \( X  \) together with a metric \( d  \).
\end{defn}
\end{tcolorbox}

\begin{itemize}
    \item Property (iii) in the definition above is just the triangle inequality.
    \item The set \( X  \) can have different metrics on it.
    \item Whenever a metric space is mentioned, we usually specify what metric are using.
\end{itemize}

\subsubsection{Exercise 8.2.1} Decide which of the following are metrics on \( X = \R^2  \). For each, we let \( x = (x_{1}, x_{2}) \) and \( y = (y_{1}, y_{2}) \) be points in the plane.
\begin{enumerate}
    \item[(a)] \( d(x,y) = \sqrt{ (x_{1}- y_{1})^2 + (x_{2} - y_{2})^{2} }. \)
    \item[(b)] \( d(x,y) = \max \{ | x_{1} - y_{1} |, | x_{2} - y_{2} |  \}. \)
    \item[(c)] \( d(x,y) = | x_{1} x_{2} + y_{1} y_{2} |  \).
\end{enumerate}

\begin{proof}[Solution]
    \begin{enumerate}
        \item[(a)]  We claim that \( d(x,y)  \) is a metric on \( X = \R^{2} \). Let \( x', y' \in \R^{2} \) where \( x' = (x_{1}, y_{1}) \) and \( y' = (x_{2}, y_{2}) \). For part (i), suppose \( x' \neq y' \). Then observe that by property of the square root, we know that \( d(x,y) >  0  \). Otherwise, \( d(x,y) = 0  \). 

            For part (ii), observe that 
            \begin{align*}
                d(x,y) &= \sqrt{ (x_{1} - y_{1})^2 + (x_{2} - y_{2})^{2} }  \\
                       &= \sqrt{ (y_{1} - x_{1})^2 + (y_{2} - x_{2})^{2} } \\
                       &= d(y,x).
            \end{align*} 
        For part(iii), let \( x',y', z' \in \R^2  \). Then observe that 
        \begin{align*}
            d(x,y) &= \sqrt{ (x_{1} - y_{1})^2 + (x_{2} - y_{2})^2  }  \\
                   &= \sqrt{ (x_{1} - z_{1})^2 + (x_{2} - z_{2})^{2} + (z_{1} - y_{1})^{2} + (z_{2} - y_{2})^{2}  } \\ 
                   &\leq \sqrt{ (x_{1} - z_{1})^2 + (x_{2}  - z_{2})^2   }  + \sqrt{ (z_{1} - y_{1})^2 + (z_{2} - y_{2})^{2} } \\
                   &= d(x,z) + d(z,y).
        \end{align*}
        Hence, we conclude that \( d(x,y)  \) is a metric on \( \R^{2}  \).
    \item[(b)] We claim that \( d(x,y) = \max \{ | x_{1} - y_{1}  |, | x_{2} - y_{2} |  \}  \) is a metric on \( \R^2  \). 
    \end{enumerate}
\end{proof}










