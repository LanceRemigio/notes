\section{Metric Spaces and the Baire Category Theorem}

In this section, we aim to give a more generalized view of what it means to have a "distance" over sets other than \( \R  \). Do the theorems and properties we have proved about sequences, series, and functions carry over to sets like \( \R^{2} \) or even in higher dimensions like \( \R^n \)? We will be mainly testing our notions that we have developed throughout the book on sets such as \( \R^{2} \) and \( C[0,1] \), the space of continuous functions on \( [0,1] \).


\begin{tcolorbox}
\begin{defn}
Given a set \( X  \), a function \( d: X \times X \to \R  \) is a \textit{metric} on \( X  \) if for all \( x,y \in X  \):
\begin{enumerate}
    \item[(i)]\( d(x,y) \geq 0  \) with \( d(x,y) =0  \) if and only if \( x = y  \),
    \item[(ii)] \( d(x,y) = d(y,x)  \), and 
    \item[(iii)] for all \( z \in X \), \( d(x,y) \leq d(x,z) + d(z,y) \).
\end{enumerate}
A \textit{metric space} is a set \( X  \) together with a metric \( d  \).
\end{defn}
\end{tcolorbox}

\begin{itemize}
    \item Property (iii) in the definition above is just the triangle inequality.
    \item The set \( X  \) can have different metrics on it.
    \item Whenever a metric space is mentioned, we usually specify what metric are using.
\end{itemize}

\subsubsection{Exercise 8.2.1} Decide which of the following are metrics on \( X = \R^2  \). For each, we let \( x = (x_{1}, x_{2}) \) and \( y = (y_{1}, y_{2}) \) be points in the plane.
\begin{enumerate}
    \item[(a)] \( d(x,y) = \sqrt{ (x_{1}- y_{1})^2 + (x_{2} - y_{2})^{2} }. \)
    \item[(b)] \( d(x,y) = \max \{ | x_{1} - y_{1} |, | x_{2} - y_{2} |  \}. \)
    \item[(c)] \( d(x,y) = | x_{1} x_{2} + y_{1} y_{2} |  \).
\end{enumerate}

\begin{proof}[Solution]
    \begin{enumerate}
        \item[(a)]  We claim that \( d(x,y)  \) is a metric on \( X = \R^{2} \). Let \( x', y' \in \R^{2} \) where \( x = (x_{1}, x_{2}) \) and \( y = (y_{1}, y_{2}) \). For part (i), suppose \( x' \neq y' \). Then observe that by property of the square root, we know that \( d(x,y) >  0  \). Otherwise, \( d(x,y) = 0  \). 

            For part (ii), observe that 
            \begin{align*}
                d(x,y) &= \sqrt{ (x_{1} - y_{1})^2 + (x_{2} - y_{2})^{2} }  \\
                       &= \sqrt{ (y_{1} - x_{1})^2 + (y_{2} - x_{2})^{2} } \\
                       &= d(y,x).
            \end{align*} 
        For part(iii), let \( x,y, z \in \R^2  \). Then observe that 
        \begin{align*}
            d(x,y) &= \sqrt{ (x_{1} - y_{1})^2 + (x_{2} - y_{2})^2  }  \\
                   &= \sqrt{ (x_{1} - z_{1})^2 + (x_{2} - z_{2})^{2} + (z_{1} - y_{1})^{2} + (z_{2} - y_{2})^{2}  } \\ 
                   &\leq \sqrt{ (x_{1} - z_{1})^2 + (x_{2}  - z_{2})^2   }  + \sqrt{ (z_{1} - y_{1})^2 + (z_{2} - y_{2})^{2} } \\
                   &= d(x,z) + d(z,y).
        \end{align*}
        Hence, we conclude that \( d(x,y)  \) is a metric on \( \R^{2}  \).
    \item[(b)] We have \( d(x,y) = \max \{ | x_{1} - y_{1}  |, | x_{2} - y_{2} |  \}  \) is a metric on \( \R^2  \). 
        For property (i), observe that \( d(x,y) > 0  \) if either \( x \geq y  \) or \( x < y  \). This holds because \( |  \cdot | > 0  \). If \( x = y  \), then it follows immediately that \( d(x,y) = 0   \). To show the triangle inequality, we will use the formula 
        \[ \max \{ a,b \}  = a + b + | | a | - | b |  |.  \]
        Then observe that for any \( x,y,z \in \R^2 \), we have 
        \begin{align*}
            d(x,y) &= \max \{ | x_{1} - y_{1} | , | x_{2} - y_{2} |  \}  \\
                   &= \frac{ 1 }{ 2 } \Big[ | x_{1} - y_{1} |  + | x_{2} - y_{2} |  + | | x_{1} -  x_{1}  | + | x_{2} - y_{2} |   |  \Big] \\
                   &\leq \frac{ 1 }{ 2 } \Big[ | x_{1} - z_{1} | + | z_{1} - y_{1}   |   + | x_{2} - z_{2} | + | z_{2} - y_{2} |  \\ 
                   &+ \Big| | x_{1} -  z_{1}  | + | z_{1} - y_{1} |  - | x_{2} - z_{2} | + | z_{2} - y_{2} |    \Big|  \Big] \\
                   &= \frac{ 1 }{ 2 }  \Big[  | x_{1} - z_{1}  |  + | x_{2} - z_{2} | + \Big| | x_{1} - z_{1} |  - | x_{2} - z_{2} |  \Big|  \Big] \\ 
                   &+ \frac{ 1 }{ 2 }  \Big[ | z_{1} - y_{1} | + | z_{2} - y_{2} | + \Big| | z_{1} - y_{1}  |  - | z_{2} - y_{2} |  \Big|   \Big] \\
                   &= \max \{ | x_{1} -  z_{1} |, | x_{2} - z_{2} |  \} + \max \{ | z_{1} - y_{1} |, | z_{2} - y_{2} |  \} \\ 
                   &= d(x,z) +  d(z,y). 
        \end{align*}
    \item[(c)] \( d(x,y) = | x_{1}x_{2} + y_{1}y_{2} |  \) cannot be a metric since \( d(x,y) \neq 0  \) for all \( x,y \in \R^{2} \).
    \end{enumerate}
\end{proof}

\begin{itemize}
    \item The metric in part (a) is the Euclidean distance between two points in a plane.
    \item \( d(x,y) = |   x - y  |   \) is a metric over \( \R  \) (the main metric we have been working with throughout the book).
\end{itemize}

\subsubsection{Exercise 8.2.2} Let \( C[0,1]  \) be the collection of continuous functions on the closed interval \( [0,1]  \). Decide which of the following are metrics on \( C[0,1] \).
\begin{enumerate}
    \item[(a)] \( d(f,g) = \sup \{ | f(x) - g(x)  | : x \in [0,1] \}  \).
    \item[(b)] \( d(f,g) = | f(1) - g(1) |  \).
    \item[(c)] \( d(f,g) = \int_{ 0 }^{ 1 }  | f - g |   \).
\end{enumerate}

\begin{proof}[Solution]
\begin{enumerate}
    \item[(a)] Observe that for any two functions \( f, g \in C[0,1] \) that are distinct, we know that 
        \[  d(f,g) = \sup \{ | f(x) - g(x)  |: x \in [0,1]  \} \geq | f(x) - f(x)  | > 0. \]
        If \( f = g  \), then it immediately follows that \( d(f,g) = 0  \). Hence, property (i) is satisfied.  

        Observe that part (ii) is satisfied by taking 
        \begin{align*}  d(f,g) &= \sup \{ | f(x) - g(x)  | : x \in [0,1] \} \\ &= \sup \{ | g(x) - f(x)  | : x \in [0,1] \} \\  &= d(g,f). \end{align*}

        For part (iii), let \( f,g,h \in C[0,1] \), then we must have 
        \begin{align*}
            d(f,g) &= \sup | f(x) - g(x) |  \\
                   &= \sup | f(x) - h(x) + h(x) - g(x) | \\
                   &\leq \sup | f(x) - h(x)  | + \sup | h(x) - g(x) | \\
                   &= d(f,h) + d(h,g).
        \end{align*}
    \item[(b)] The first property fails (take \( f(1) = 1  \) and \( g(x) = x  \)).
    \item[(c)] We claim that \( d(f,g) = \int_{ 0 }^{ 1 } | f - g  |     \) is a metric on \( \R^{2} \).    Note that for any two distinct functions \( f,g \in C[0,1]  \), we must have \( | f -g  | > 0  \). By exercise 7.4.4, we must have \( \int_{ 0 }^{ 1 }  | f - g  |   > 0  \). Otherwise, \( f = g  \) implies \( \int_{ 0 }^{ 1 }  | f - g  |  = 0  \). If \( \int_{ 0 }^{ 1 }  | f -g  |  =0  \), then we must have \( | f -g  | = 0  \) and hence, \( f = g  \). To show the triangle inequality, let \( f,g,h \in C[0,1] \) be integrable (since they are part of a set of continuous functions that are bounded). Hence, observe that 
        \begin{align*}
            d(f,g) &= \int_{ 0 }^{ 1 } | f -g  |    \\
                   &\leq \int_{ 0 }^{ 1 }  | f- h  |  + | h - g  |  \\
                   &= \int_{ 0 }^{ 1 } | f -h  | + \int_{ 0 }^{ 1 } | h - g  |  \\
                   &= d(f,h)  + d(h,g).
        \end{align*}
\end{enumerate}
\end{proof}

Define the \textit{discrete metric} on any set \( X  \) where for any \( x, y \in X  \), let 
\[  \rho (x,y) = 
\begin{cases}
    1 \  &\text{ if } x \neq y \\
    0 \ &\text{ if } x = y.
\end{cases}  \]

\subsubsection{Exercise 8.2.3} Verify that the discrete metric is actually a metric.
\begin{proof}[Solution]
Observe that if \( x \neq y  \), then by definition we must have \( \rho(x,y) > 0  \). Otherwise, \( \rho(x,y) = 0  \) by definition. It is clear that \( \rho(x,y) = \rho(y,x) \). To show the triangle inequality, let \( x,y,z \in X  \), then we must have 
\begin{align*}
    \rho(x,y) &= 1 + 0   \\
              &\leq 1 + 1 \\
              &= \rho(x,z) + \rho(z,y). 
\end{align*}
Hence, \( \rho(x,y)  \) is a metric on any arbitrary set \(  X \).
\end{proof}

\subsection{Basic Definitions}

\begin{tcolorbox}
\begin{defn}
Let \( (X,d)  \) be a metric space. A sequence \( (x_{n}) \subseteq  X  \) \textit{converges} to an element \( x \in X  \) if for all \( \epsilon > 0  \) there exists an \( N \in \N  \) such that \( d(x_{n} ,x )  < \epsilon \) whenever \( n \geq N  \).
\end{defn}
\end{tcolorbox}


\begin{tcolorbox}
\begin{defn}
A sequence \( (x_{n}) \) in a metric space is a \textit{Cauchy sequence} if for all \( \epsilon > 0  \), there exists an \( N \in \N  \) such that \( d(x_{m}, x_{n}) < \epsilon \) whenever \( m, n \geq N  \).
\end{defn} 
\end{tcolorbox}

\subsubsection{Exercise 8.2.4} Show that a convergent sequence is Cauchy.
\begin{proof}
    Since \( (x_{n}) \subseteq \) is a Cauchy sequence, we can pick an \( N \in \N  \) such that for any \( n,m \geq N  \), we must have 
    \[  d(x_{n}, x ) < \frac{ \epsilon  }{ 2  }  \ \text{ and } \ d(x, x_{m}) < \frac{ \epsilon  }{ 2 } .\] Using the same choice of \( N \in \N  \) so that \( n,m \geq N  \), we must have that
    \begin{align*}
        d(x_{n}, x_{m}) &\leq d(x_{n}, x )  + d(x, x_{m})\\ 
                        &< \frac{ \epsilon  }{ 2 }  + \frac{ \epsilon  }{ 2 }  \\
                        &= \epsilon.
    \end{align*}
    Hence, \( (x_{n}) \) is Cauchy.
\end{proof}

\begin{itemize}
    \item Notice that this is only the forwards direction of the Cauchy Criterion we studied under \( \R  \). 
    \item For metric spaces other than \( \R  \), the converse of the Cauchy Criterion does not necessarily hold.
    \item We need to develop an ordering of our space similar to how the Axiom of Completeness is used in \( \R  \) ( This is called \textit{completeness}  ). 
    \item The convergence of Cauchy sequences is taken to be the definition of completeness.
\end{itemize}


\begin{tcolorbox}
\begin{defn}
A metric space \( (X, d ) \) is \textit{complete} if every Cauchy sequence in \( X  \) converges to an element of \( X  \).
\end{defn}
\end{tcolorbox}


\subsubsection{Exercise 8.2.5} 
\begin{enumerate}
    \item[(a)] Consider \( \R^{2} \) with the discrete metric \( \rho(x,y)  \) examined in Exercise 8.2.3. What do Cauchy sequences look like in this pace? Is \( \R^{2} \) complete with respect to this metric? 
        \begin{proof}[Solution]
        
        \end{proof}
    \item[(b)] Show that \( C[0,1]  \) is complete with respect to the metric in Exercise 8.2.2(a).
        \begin{proof}
        The metric from Exercise 8.2.2 (a) is 
        \[  d(f,g) = \sup_{x \in [0,1]} | f(x) - g(x) |.    \]
        Let \( f_{n}, f_{m} \in C[0,1] \). The Cauchy sequence under sup norm metric will be 
        \[  d(f_{n}, f_{m}) = \sup_{x \in [0,1]} | f_{n}(x) - f_{m}(x) | \]
        We want to show that the Cauchy sequence of functions \( (f_{n}) \) converges under \( C[0,1] \). Since \( (f_{n}) \) is a Cauchy sequence under \( \R  \), we know that is satisfies the Cauchy Criterion. Hence, \( (f_{n}) \) must converge. By choosing \( N \in \N  \), we can let \( m, n \geq N  \) such that 
        \begin{align*}
            d(f_{n}, f) &= \sup_{x \in [0,1]} | f_{n}(x) - f(x)  |   \\
                        &\leq \sup_{x \in [0,1] } | f_{n}(x) - f_{m}(x)  | + \sup_{x \in [0,1]} | f_{m}(x) - f(x)  | \\
                        &< \frac{ \epsilon  }{ 2  }  + \frac{ \epsilon  }{ 2 } = \epsilon.
        \end{align*}
        Hence, the sup norm metric is complete under \( C[0,1] \).
        \end{proof}
    \item[(c)] Define \( C^1[0,1] \) to be the collection of differentiable functions on \( [0,1]  \) whose derivatives are also continuous. Is \( C^{1}[0,1] \) complete with respect to the metric defined in Exercise 8.2.2(a)?
        \begin{proof}
        
        \end{proof}
\end{enumerate}

The sup metric is usually written as 
\[  \lVert f -g  \rVert_{\infty }  = d(f,g) = \sup \{ | f(x) - g(x)  | : x \in [0,1] \}  \] and setting \( g = 0  \) gives us the "sup norm" 
\[  \lVert f \rVert_{\infty } = d(f,0 ) = \sup \{ | f(x)  | : x \in [0,1] \} . \]
From now on, we will assume that the space \( C[0,1]  \) is paired with the metric above unless otherwise specified.

\begin{tcolorbox}
\begin{defn}
Let \( (X, d_{1}) \) and \( (Y, d_{2}) \) be metric spaces. A function \( f: X \to Y  \) is \textit{continuous} at \( x \in  X  \) if for all \(\epsilon > 0  \), there exists a \( \delta > 0  \) such that \( d_{2}(f(x) , f(y)) < \epsilon  \) whenever \( d_{1} (x,y) < \delta \).  
\end{defn}
\end{tcolorbox}

\subsubsection{Exercise 8.2.6} Which of these functions from \( C[0,1] \) to \( \R  \) (with the usual )

