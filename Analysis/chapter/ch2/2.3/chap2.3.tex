
% !TEX root =  ../../../main.tex 

\section{The Algebraic and Order Limit Theorems}

The goal of having a rigorous definition of convergence in Analysis is to prove statements about sequences in general like the notion of "boundedness" which we will define below.

\begin{tcolorbox}
\begin{defn}
A sequence \( (x_n) \) is \textit{bounded} if there exists a number \( M > 0 \) such that \( | x_n | \leq M \) for all \( n \in \N \).
\end{defn}
\end{tcolorbox}

Geometrically, this means that we can find an interval \( [-M, M]\) that contains every term in the sequence \( (x_n)\).
This naturally leads us to the point that all convergent sequences are bounded i.e 

\begin{tcolorbox}
\begin{thm}
Every convergent sequence is bounded.
\end{thm}
\end{tcolorbox}

\begin{proof}
Assume \( (x_n) \) converges to a limit \( \ell\). This means that given \( \epsilon =  1\), we can find an 
\( N \in \N\) such that for every \( n \geq N \), we can say that 
\begin{align*}
    \implies&| x_n - \ell |  <  1  \\
    \iff &-1 < x_n - \ell < 1 \\
    \iff &\ell  - 1 < x_n < \ell + 1.
\end{align*}
Note the terms of the sequence \( (x_n)\) can be found in the open interval \( (\ell - 1, \ell + 1)\). Since \( \ell \in \R \) can either be positive or negative, we can conclude that  
\[ | x_n | < | \ell | + 1 \]
for all \( n \geq N \) where
\[ M = \max \{ | x_1 |, | x_2 |, ..., | \ell | + 1 \}.\]
Hence, it follows that \( | x_n  | \leq M \) for all \( n \in \N \) as desired.
\end{proof}

\begin{tcolorbox}
    \begin{thm}[Algebraic Limit Theorem]
        Let \( \lim a_n = a \), and \( \lim b_n = b \). Then, 
        \begin{enumerate}
            \item[(i)] \( \lim(ca_n) = ca \) for all \( c \in \R \);
            \item[(ii)] \( \lim (a_n + b_n) = a + b \);
            \item[(iii)] \( \lim (a_nb_n) = ab\);
            \item[(iv)] \( \lim (a_n / b_n) = a / b \) provided that \( a \neq 0\).
        \end{enumerate}
\end{thm}
\end{tcolorbox}

\begin{proof}[Proof of (i)]
We begin by proving part \( (i)\). Suppose \( a_n \to a \). Then for every \( \epsilon  > 0 \), there exists 
\( N \in \N \) such that for every \( n \geq N \), we have 
\[ | a_n  - a  | < \epsilon / | c |. \tag{1}\]
In order to show \( (i)\), we need to show that 
\[ | ca_n - ca | < \epsilon.\]
Hence, observe that 
\begin{align*}
| ca_n - ca |&< | c(a_n - a) | \\
&< | c | | a_n - a  | \\  
&< | c | \frac{ \epsilon  }{ | c |} \\ 
&= \epsilon.   
\end{align*}
If \( c = 0 \), then our sequence \( (ca_n)\) reduces to the sequence \( \{0,0,0,...,0 \}\) which is clearly converging to \( ca = 0 \).
Hence, we have attained our desired property that \( \lim (ca_n) = ca\).
The parts are left to you to prove.
\end{proof}
\begin{proof}[Proof of (ii)]
    To show part \( (ii)\), it suffices to show that for every \( \epsilon> 0 \), there exists \( N \in \N \) such that for every \( n \geq N \), we have 
    \[ | a_n + b_n - (a+b) | < \epsilon.\] Hence, we start with the left side of (ii). Since \( a_n \to a \) and \( b_n \to b\), there exists \( N_1, N_2 \in \N \). We can choose \( N = \max \{ N_1, N_2 \}\) such that for every \( n \geq N \), we can say that 
    \begin{align*}
     | a_n + b_n - (a + b) | &< | (a_n-a) + (b_n - b) |  \\
                             &< | a_n - a  |  + | b_n - b | \\ 
                             &< \frac{ \epsilon }{2} + \frac{ \epsilon }{ 2} \\  
                             &= \epsilon. 
    \end{align*}
Hence, it follows that \( \lim (a_n + b_n) = a + b \) as required.

\end{proof}

\begin{proof}[proof of (iii)]
    To show part \( (iii)\), it suffices to show for every \( \epsilon  > 0 \), there exists \( N \in \N \) such that for every \( n \geq N \), we have 
    \[ | a_nb_n - ab | < \epsilon.\]
Since \( a_n \to a \) and \( b_n \to b\), there exists \( N_1 , N_2 \in \N \). We can choose \( N = \max \{ N_1, N_2  \}\) such that for every \( n \geq N \), we can say that 
\begin{align*}
 | a_nb_n - ab |&< | a_nb_n -a_nb + a_nb - ab |  \\
                &< | a_n (b_n - b ) + b (a_n - a)| \\ 
                &< | a_n (b_n - b) |  + | b (a_n - a) | \\ 
                &< | a_n | | b_n - b  |  + | b | | a_n - a |  \\
                &< M \frac{ \epsilon }{2 M }  + | b | \frac{ \epsilon }{2 | b |} \tag{ \( a_n \) is bounded } \\ 
                &< \epsilon  
\end{align*}
Hence, it follows that \( \lim (a_nb_n) = ab\).
\end{proof}

\begin{proof}[Proof of (iv)]
To show part (iv), it suffices to show for every \( \epsilon  > 0 \), there exists an \( N \in \N\) such that for every \( n \geq N \), we have 
\[ \Big| \frac{a_n}{b_n} - \frac{a}{b} \Big| < \epsilon.\]
Since \( a_n \to a \) and \( b_n \to b\) with \( b \neq 0 \), there exists an \( N_1, N_2 \in \N   \) such that whenever \( n \geq N_1, N_2\), we can have
\begin{align*}
 | a_n - a  |&<  M \epsilon / 2,  \\
 | b_n - b | &<  \frac{ | b |}{ | a |} \cdot \frac{ M \epsilon }{2}.
\end{align*}



we can choose \( N = \max \{ N_1, N_2 \}\) so that 
\begin{align*}
   \Big| \frac{a_n}{b_n} - \frac{a}{b} \Big| &=  \Big| \frac{a_nb - b_n a}{b_nb} \Big|   \\
                                     &=  \Big| \frac{a_nb - b_n a}{b_nb} \Big| \\
                                     &= \Big| \frac{a_nb - ab + ab- b_n a}{b_nb} \Big| \\
                                     &=  \Big| \frac{b(a_n - a) + (b- b_n)a}{b_nb} \Big| \\
                                     &<  \frac{|a_n - a|}{|b_n|} + \frac{ | a |}{ | b |} \cdot \frac{|b_n - b|}{|b_n|} \\
                                     &< \frac{ M \epsilon }{ 2M} + \frac{ | a |}{ | b |} \cdot \frac{ | b | M  \epsilon}{  | a | 2 M} \tag{ \( b_n\) bounded} \\
                                     &= \epsilon. 
\end{align*}
Hence, it follows that \( \lim ( \frac{a_n}{b_n} ) = \frac{a}{b} \) provided that \( b \neq 0\).



\end{proof}

\begin{tcolorbox}
    \begin{thm}[Order Limit Theorem] 
    Assume \( \lim a_n  = a\) and \( \lim b_n =  b\).
    \begin{enumerate}
        \item[(i)] If \( a_n \geq 0 \) for all \( n \in \N \), then \( a \geq 0\).
        \item[(ii)] If \( a_n \leq b_n\) for all \( n \in \N \), then \( a \leq b \).
        \item[(iv)] If there exists \( c \in \R \) for which \( c \leq b_n\), for all \( n \in \N \), then 
        \( c \leq b \). Similarly, if \( a_n \leq c \) for all \( n \in \N \), then \( a \leq c\).
    \end{enumerate}
    \end{thm}
\end{tcolorbox}

\begin{enumerate}


    \item[(i)] \begin{proof}
We proceed by contradiction by assuming that \( a < 0 \). Suppose \( a_n \geq 0 \) and \( a_n \to a \). Let \( \epsilon  = | a |\) and suppose \( n \geq N \). Then
\[ | a_n - a | < | a | = -a.\]
But this means that \( a_N < 0\) which is a contradiction since \( a_N \geq 0\).
\end{proof}
    \item[(ii)]
        \begin{proof}
        We can ensure that the sequence \( b_n - a_n\) converges to \( b - a\) by the Algebraic Limit Theorem. Since \( b_n - a_n \geq 0\), we can use (i) to write \( b - a \geq 0\). Hence, \( a \leq b\).
        \end{proof}
    \item[(iii)]
        \begin{proof}
            Suppose there exists \( c \in \R  \) for which \( c \leq b_n\) for all \( n \in \N \). Suppose \( a_n  = c \) then using (ii) yields \( c \leq b\). Suppose \( a_n \leq c\) for all \( n \in \N \) then setting \( b_n = c \) and using (ii) again yields \( a \leq c\).
        \end{proof}
\end{enumerate}







