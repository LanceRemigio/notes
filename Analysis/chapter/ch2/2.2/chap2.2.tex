
\section{The Limit of a Sequence}

Understanding infinite series depends on understanding sequences that make up sequences of partial sums.

\begin{tcolorbox}
\begin{defn}
A sequence is a function whose domain is \( \N \).
\end{defn}
\end{tcolorbox}

A way we describe sequences is to assign each \( n \in \N \), use a mapping rule, and then have an output for the \( n \)th term. Mathematically we can describe it as a map \( f: \N \to \R \).

\begin{ex}
    Each of the following are common ways to describe a sequence. 
    \begin{enumerate}
        \item \( (1, \frac{1}{2},  \frac{1}{3}, \frac{1}{4},...  )\)
        \item \( \{  \frac{1+n}{n}  \}_{n=1}^{ \infty} = ( \frac{2}{1}, \frac{3}{2}, \frac{4}{3}, ...)\)
        \item \( (a_n) \), where \( a_n = 2^n \) for each \( n \in \N \),
        \item \( (x_n)\), where \( x_1 = 2 \) and \( x_{n+1} = \frac{x_n + 1 }{2}\).
    \end{enumerate}
\end{ex}
It should not be confused that in some instances, the index \( n \) will start at \( n = 0 \) or \( n = n_0 \) for some other \( n_0 > 1 \). It is important to keep in mind that sequences are just infinite lists of real numbers. The main point of our analysis deals with what happens at the "tail" end of a given sequence. 

\begin{tcolorbox}
\begin{defn}[Convergence of a Sequence]
A sequence \( (a_n) \) \textit{converges} to a real number \( a \) if, for every \( \epsilon > 0 \), there exists an \( N \in \N \) such that whenever \( n \geq N \) it follows that \( |a_n - a | < \epsilon \).
\end{defn}
\end{tcolorbox}
Furthermore, the convergence of a sequence \( (a_n) \) to \( a \) is denoted by 
\[ \lim_{n \to \infty} a_n = a.\]

To understand the last part of this definition, namely, \( |a_n - a| < \epsilon \), we can think of it as a neighborhood where a given value will be located in. 
\begin{tcolorbox}
\begin{defn}
Given \( a \in \R \) and \( \epsilon  > 0 \), the set 
\[ V_{  \epsilon }(a) = \{ x \in \R : |x-a| < \epsilon  \}\]
is called the \textit{\( \epsilon \)-neighborhood of \( a \)}. 
\end{defn}
\end{tcolorbox}
We can think of \( V_{ \epsilon }(a)\) as an interval where 
\[ a - \epsilon < a < a + \epsilon.\]
Another way is to think of it as a ball with radius \( \epsilon > 0\) centered at \( a \). 
we can also think about the convergence of a sequence to a point with the following definition.
\begin{tcolorbox}
\begin{defn}
    A sequence \( (a_n) \) converges to \( a \) if, given any \( \epsilon-\)neighborhood \( V_{ \epsilon } (a)\) of \( a \), there exists a point in the sequence after which all of the terms are in \( V_{ \epsilon } (a) \). In other words, every \( \epsilon - \)neighborhood contains all but a finite number of the terms of \( (a_n) \). 
\end{defn}
\end{tcolorbox}

The main idea here is that for some \( n \in \N \) along a sequence \( (a_n) \), all the points of the sequence converge to some point within a certain \( \epsilon -\)neighborhood. Note that when increase the value of \( n \in \N \), the smaller this \( \epsilon-\)neighborhood has to be and vice versa.

\begin{ex}
Consider the sequence \( (a_n) \), where \( a_n = \frac{1}{ \sqrt{n} }\). From our regular understanding of calculus, one can see that the limit of this sequence goes to zero. 

\begin{proof}
Let \( \epsilon  > 0 \). Choose \( N \in \N \) such that 
\[ N > \frac{1}{e^2}.\]
We now proceed by verifying that this choice \( N \in \N \) has the desired property that \( a_n \to 0 \). Let \( n \geq N \) such that \( n > \frac{1}{ \epsilon^2} \). Hence, we have 
\[ \frac{1}{ \sqrt{n}} < \epsilon. \]
But this implies that \( |a_n - 0| < \epsilon \) and hence our sequence contains the desired property. 
\end{proof}
\end{ex}
The main idea of these convergence proofs is to find an \( N \in \N \) such that the value we want can be "hit" within some range that we specify with any number \( \epsilon > 0   \).


\subsubsection{Quantifiers}

The phrase 

\begin{center}
"For all \( \epsilon> 0 \)", there exists \( N \in \N \) such that ..."
\end{center}

means that for every positive integer I give you, there exists some index or natural number that contains some property that allows the sequence to converge to some value that we desire and as long as we satisfy this rule, then we can say that the sequence converges to our desired value. The template for our subsequent covergence proof will follow the steps below:

\begin{itemize}
    \item "Let \( \epsilon> 0 \)" be arbitrary."
    \item Demonstrate that a specific choice of \( N \in \N \) leads to the desired property. Note that finding this \( N \) often involves working backwards from \( |a_n - a | < \epsilon \). 
    \item Show that this \( N \) actually works.
    \item Now assume \( n \geq N \). 
    \item With this choice of \( \N \), you can work towards the property that \( |a_n - a | < \epsilon \)
\end{itemize}

\begin{ex}
Show 
\[ \lim \Big( \frac{n+1}{n}\Big) = 1.\]
In other words, show that for every \( \epsilon  > 0 \), there exists some \( N \in \N \) such that 
\[ |a_n - 1| < \epsilon \] where 
\[ a_n = \frac{n+1}{n}. \]
To obtain our choice of \( N \in \N \), we must work backwards from our conclusion. Hence, we have 
\begin{align*}
a_n - 1 &< \epsilon  \\
\frac{n+1}{n} - \frac{n}{n} &< \epsilon \\ 
\iff \frac{1}{n} &<  \epsilon \\
\iff \frac{1}{ \epsilon } &< n.
\end{align*}
Hence, our choice of \( N \in \N \) is \( N = 1/ \epsilon \). Now for the actual proof. 

\begin{proof}
Let \( \epsilon  > 0 \) be arbitrary. Choose \( N = 1 / \epsilon  \) such that 
\[ N > \frac{1}{ \epsilon }.\]
Let \( n \geq N \). Then we proceed by showing that this choice of \( N \in \N \) leads to the desired property. 
Hence, 
\begin{align*}
n &> \frac{1}{ \epsilon } \\
\epsilon &> \frac{1}{ n } \\
\epsilon  &> \frac{ n+1 }{n} - \frac{n}{n} \\ 
\epsilon &> \frac{n+1}{n} - 1 \\
\epsilon  &> |a_n - 1|.
\end{align*}
Hence, our choice of \( N \in \N \) leads to \( a_n \to 1 \). We can now conclude that 
\[ \lim_{n \to \infty} a_n = 1.\]
\end{proof}
\end{ex}

\begin{tcolorbox}
    \begin{thm}[Uniqueness of Limits]
The limit of a sequence, when it exists, must be unique.
\end{thm}
\end{tcolorbox}

\begin{proof}
Suppose we have \( (a_n) \subseteq \R \). Suppose \( a_n \to a \) and \( a_n \to a' \). We want to show that 
\[ a = a' .  \]
By definition, we have that 
\begin{align*}
    |a_n - a |&< \epsilon/2   \text{ for some } n_1 \in \N \\
    |a_n - a'| &< \epsilon/2 \text{ for some } n_2 \in \N .
\end{align*}
We can show that \( a = a' \) by showing that \( |a - a'| < \epsilon\). Hence, choose \( N = \min \{ n_1, n_2 \}\) such that 
\begin{align*}
 |a - a'|&< |a - a_n + a_n - a' |  \\
         &< |a - a_n | + |a_n - a'| \\
         &< \epsilon/2 + \epsilon/2 \\
         &= \epsilon.
\end{align*}
Hence, we have that \( a = a' \) showing that our limit is unique. 
\end{proof}



\subsection{Divergence}

We can study the divergence of sequences by negating the definition we have above. 
\begin{ex}
Consider the sequence 
\[ \Big(1, -\frac{1}{2}, \frac{1}{3}, -\frac{1}{4}, \frac{1}{5}, -\frac{1}{5}, \frac{1}{5}...  \Big)\]
We can prove that this sequence does not converge to zero. Why? When we choose an \( \epsilon  = 1/10 \), there is none of the term of the sequence converge within the neighborhood \( (-1/10, 1/10 )\) since the sequence oscillates between \(-1 / 5 \)  and \( 1 / 5\). There is no \( N \in \N \), that satisfies \( a_n \to 0 \). We can also give a counter-example in which we disprove the claim that \( (a_n) \) converges to \( 1 / 5 \). Choose \( \epsilon = 1 / 10 \). This produces the neighborhood \( (1/10, 3/10 ) \). We can see that the sequence does in fact converge to \( 1 / 5 \), but it does so in an oscillating fashion. Furthermore, the sequence does not stay within the neighbor we specified where we expect all the terms of the sequence to converge towards the value. Hence, there is no such \( N \in \N \) where the property can be satisfied.  
\end{ex}


\begin{tcolorbox} 
\begin{defn}
A sequence that does not converge is said to diverge.
\end{defn}
\end{tcolorbox}


