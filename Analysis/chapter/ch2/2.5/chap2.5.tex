\section{Subsequences and Bolzano-Weierstrass}

In the last section, we observed that the convergence of partial sums of a particular series can be determined by the behavior of a subsequence of the partial sums. 

\begin{tcolorbox}
\begin{defn}
Let \( (a_n) \subseteq \R \), and let \( n_1 < n_2 < n_3 < n_4 < \dots\) be an increasing sequence of natural numbers. Then the sequence 
\[ (a_{n_1}, a_{n_2}, a_{n_3}, a_{n_4}, a_{n_5} \dots)\]
is called a \textit{subsequence} of \( (a_n)\) and is denoted by \( (a_{n_k})\), where \( k \in \N \) indexes the subsequence.
\end{defn}
\end{tcolorbox} 

A few remarks about subsequences: 

\begin{enumerate}
    \item[(a)] The order of the subsequence is the same as in the original sequence.
        \begin{ex}
        If we have the sequence  
        \[ (a_n) = \Big( 1, \frac{1}{2}, \frac{1}{3}, \frac{1}{4}, ...\Big)\]
        then the subsequences 
        \[ \Big( \frac{1}{2}, \frac{1}{4}, \frac{1}{6}, \frac{1}{8},... \Big)\]
        and 
        \[ \Big( \frac{1}{10}, \frac{1}{100}, \frac{1}{1000}, \frac{1}{10000}, ...\Big)\]
        are permitted.
        \end{ex}
        
    \item[(b)] Repetitions and swapping are not allowed.
        \begin{ex}
        Like 
        \[ \Big( \frac{1}{10}, \frac{1}{5}, \frac{1}{100}, \frac{1}{50}, \frac{1}{1000}, \frac{1}{500},...\Big)\] 
        and 
        \[ \Big( 1,1, \frac{1}{3}, \frac{1}{5}, \frac{1}{5}, ...\Big)\]
        \end{ex}
        
\end{enumerate}

Since subsequences have the same ordering as the original sequence, one can conjecture about them converging to the same limit. 

\begin{tcolorbox}
\begin{thm}
Subsequences of a convergent sequence converge to the same limit as the original sequence.
\end{thm}
\end{tcolorbox}

\begin{proof}
    Let \( (a_n) \to a \) and let \( (a_{n_k})\) be a subsequence for \( (a_n)\). We want to show \( (a_{n_k})\) converges to \( a \) as well. Since \( (a_n) \to a\), there exists an \( N \) such that for any \( n \geq N \), we have \( |a_n - a| < \epsilon \). 

    We claim that \( n_{k} \geq k \) for any \( k \in \N\). Let us proceed by inducting on \( k \). Let the base case be \( k = 1 \). Since \(n_k\) is an \textit{increasing} sequence of natural numbers, we see that \( n_1 \geq 1\). Now let us assume \(n_{k-1} \geq k - 1\). Since \( (a_{n_k})\) in \textit{increasing}, we have \( a_k \geq a_{k-1} \geq k - 1\) which implies that \( n_k \geq k \). 

    Since any choice of \( n \geq N \), we can say that \( n_k \geq k \geq N \). Hence, we have 
    \[ |a_{n_k} - a| < \epsilon \]
which is what we desired.
\end{proof}

\begin{ex}
Let \(0 < b < 1\). Because 
\[ b > b^2 > b^3 > b^4 > ... > 0,\]
the sequence \((b^n)\) is \textit{decreasing} and \textit{bounded} below. The Monotone Convergence Theorem allows us to conclude that \((b^n)\) converges to some \(\ell\) satisfying \( 0 \leq \ell < b\). To compute \(\ell\), notice that \((b^{2n})\) is a subsequence, so \(b^{2n} \to \ell\) by Theorem 2.5.2. But \( b^{2n} = b^n \cdot b^n\), so by the Algebraic Limit Theorem, \(b^{2n} \to \ell \cdot \ell = \ell^2 \). Because limits are unique (Theorem 2.2.7), \( \ell^2 = \ell\), and thus \( \ell = 0\). 
\end{ex}










\begin{ex}
Suppose we have an oscillating sequence of numbers 
\[ \Big( 1, -\frac{1}{2}, \frac{1}{3}, - \frac{1}{4}, \frac{1}{5}, -\frac{1}{5}, \frac{1}{5}, -\frac{1}{5},... \Big)\]
Note that this sequence does not converge to any proposed limit yet if we take a subsequence of it, weget a sequence that converges! Observe, that the subsequence 
\[ \Big(\frac{1}{5}, \frac{1}{5}, \frac{1}{5}, \frac{1}{5},... \Big)\]
and
\[ \Big(-\frac{1}{5}, -\frac{1}{5}, -\frac{1}{5}, -\frac{1}{5},... \Big)\]
converge to \( 1 / 5 \) and \(- 1 / 5\) respectively. Since we have two subsequences that converge to two different limits, we immediately conclude that the original sequence diverges.
\end{ex}

This leads us to our next theorem that states that
% add definitions at the end of each chapter or put references to them
\begin{tcolorbox}
    \begin{thm}[Bolzano-Weierstrass Theorem]
        Every bounded sequence contains a convergent subsequence.
\end{thm}
\end{tcolorbox}

\begin{proof}
    Let \( (a_n)\) be a \textit{bounded} sequence. Then there exists \( M > 0 \) such that \( a_n \in [-M,M ]\). Suppose we divide this interval in half for \( k \) times: that is, let the length of the intervals be defined by the sequence \( M (1/2)^{k-1}\). We claim that a subsequence \( (a_{n_k})\) lies in either one of these intervals: that is, let \( n_{k} > n_{k-1}\) for all \( k \in \N \) such that \( a_{n_k} \in I_k\). 

    Let us induct on \( k\). Then let our base case be \( k = 1 \). Since we have an increasing sequence of natural numbers \( n_k\), we have that \( n_2 > n_1 \) which means that \( a_{n_2} \in I_2\) as well as \( a_{n_1} \in I_1\). Now let us assume that this holds for all \( k \leq \ell - 1 \). We want to show that this holds for \( k < \ell \). By the monotonicity of \( n_k \), we have that \( n_{\ell} > n_{\ell - 1} > n_k > n_1\) which implies that \( a_{n_\ell} \in I_\ell\) for all \( \ell \in \N \). Furthermore, the sets 
    \[ I_1 \subseteq I_2 \subseteq I_3 ...\]
form a nested sequence of closed intervals.

    By the \textit{Nested Interval Property}, we can conclude that there exists an \( x \in I_k\) for all \( k \in \N \)  such that \( \bigcup_{k=1}^{ \infty} I_k \neq \emptyset\). Let \( \epsilon > 0  \). Since \( a_{n_k}, x \in I_k \) for all \( k \in \N \) and \( M (1/2)^{k-1} \to 0\) by the Algebraic Limit Theorem, we can choose an \( N \in \N \) such that for any \( k \geq n_k \geq N \), we have
    \[ |a_{n_k} - x | < \epsilon.\] 
    Hence, \( (a_{n_k}) \to x \).
\end{proof}



\subsubsection{Definitions}


\begin{tcolorbox}
\begin{defn}
Let \( (a_n) \subseteq \R \), and let \( n_1 < n_2 < n_3 < n_4 < \dots\) be an increasing sequence of natural numbers. Then the sequence 
\[ (a_{n_1}, a_{n_2}, a_{n_3}, a_{n_4}, a_{n_5} \dots)\]
is called a \textit{subsequence} of \( (a_n)\) and is denoted by \( (a_{n_k})\), where \( k \in \N \) indexes the subsequence.
\end{defn}
\end{tcolorbox} 


\begin{tcolorbox}
\begin{thm}
Subsequences of a convergent sequence converge to the same limit as the original sequence.
\end{thm}
\end{tcolorbox}


\begin{tcolorbox}
    \begin{thm}[Bolzano-Weierstrass Theorem]
        Every bounded sequence contains a convergent subsequence.
\end{thm}
\end{tcolorbox}

\subsection{Exercises}


\subsubsection{Exercise 2.5.1} Give an example of each of the following, or argue that such a request is impossible. 

\begin{enumerate}
    \item[(a)] A sequence that has a subsequence that is bounded but contains no subsequence that converges.
        \begin{proof}[Solution]
        The subsubsequence of the bounded subsequence must converge by the \textit{Bolzano-Weierstrass} theorem.
        \end{proof}
    \item[(b)] A sequence that does not contain \( 0 \) or \( 1\) as a term but contains subsequences converging to each of these values.
        \begin{proof}[Solution]
        Let \( (a_n)\) be a sequence defined by 
        \[ 
           a_n =  
            \begin{cases}
                \frac{1}{2n} \text{ if } n = 2k \\ 
                \frac{1}{2n} + 1 \text{ if } n = 2k + 1.
            \end{cases}
        \]
        Note that \( 1,0 \notin (a_n)\) but if we take the subsequences \( (a_{2k}) = 1 / 4k\) and \( (a_{2k+1}) = 1/(4k+2) + 1\), and take their limit, then we end up with the former converging to \( 0 \) and the latter converging to \( 1 \).

        \end{proof} 
    \item[(c)] A sequence that contains subsequences converging to every point in the infinite set 
        \[ \Big\{ 1 , \frac{1}{2} , \frac{1}{3}, \frac{1}{4}, \frac{1}{5},...\Big\}\]
        \begin{proof}[Solution]
         Let's define the infinite set  
         \[ A_n = \Big\{ 1, \frac{1}{2}, \frac{1}{3}, \frac{1}{4}, \frac{1}{5},... \Big\}\]
         and define a subsequence such that we can make a subsequence for each \( n \in \N \) where \( (a_n)\) hits every value of \( A_n\).
        \end{proof}
    \item[(d)] A sequence that contains subsequences converging to every point in the infinite set 
        \[ \Big\{1, \frac{1}{2}, \frac{1}{3}, \frac{1}{4}, \frac{1}{5},...\Big\},\]
        and no subsequences converging to points outside of this set.
        \begin{proof}[Solution]
          This is not possible. There exists such a subsequence that does go to \( 0 \) but it is not within the infinite set.
        \end{proof}
\end{enumerate}

\subsubsection{Exercise 2.5.2}
Decide whether the following propositions are true or false, providing a short justification for each conclusion.
\begin{enumerate}
    \item[(a)] If every proper subsequence of \( (x_n)\) converges, then \((x_n)\) converges as well.
        \begin{proof}[Solution]
            If every proper subsequence of \((x_n)\) converges to \(x\), then \((x_2, x_3, x_4, ...)\) also converges to \(x\). Hence, \((x_n) \to x\) by the \textit{uniqueness of limits}.
        \end{proof}
    \item[(b)] If \( (x_n)\) contains a divergent subsequence, then \( (x_n)\) diverges.
        \begin{proof}[Solution]
        This is just the contrapositve of the statement: 
        \begin{center}
            "If \( (x_n)\) converges then every subsequence of \( (x_n)\) converges as well."
        \end{center}
        \end{proof}
    \item[(c)] If \( (x_n)\) contains a divergent subsequence, then there exists two subsequences of \( (x_n)\) that converge to different limits. 
        \begin{proof}[Solution]
        This is false. we can find an \((x_n)\) that is not bounded such that we cannot find a subsequence that converges to a limit. 
        \end{proof}
    \item[(d)] If \((x_n)\) is \textit{monotone} and contains a convergent subsequence, then \((x_n)\) converges.
        \begin{proof}
            Assume \( (x_n)\) is \textit{monotone} and contains a convergent \textit{subsequence} \((x_{n_k})\). It suffices to show that \( (x_n)\) is \textit{bounded}. Since \((x_{n_k})\) is convergent, it is also \textit{bounded}. Hence, there exists \(M > 0 \) such that for all \(n_k \in \N \), we have \(|x_{n_k}| \leq M \). Since \((x_n)\) \textit{monotone} then either \(n \leq n_k \) or \( n \geq n_k\) for all \( n \in \N \). Hence, we can write either \(-M \leq x_n \) or \(x_{n_k} \leq M \). But this means that \((x_n)\) is also \textit{bounded}. Since \((x_n)\) both \textit{bounded} and \textit{monotone}, \((x_n)\) is convergent by the \textit{Monotone Convergence Theorem}. Also, \((x_n)\) and \((x_{n_k})\) converge to the same limit by the \textit{Uniqueness of Limits}.
        \end{proof}
\end{enumerate}

\subsubsection{Exercise 2.5.3}
\begin{enumerate}
    \item[(a)] Prove that if an infinite series converges, then the associative propeprty holds. Assume \(a_1 + a_2 + a_3 + a_4 + a_5 + ...\) converges to a limit \(L \) (i.e., the sequence of partial sums \((s_n) \to L \)). Show that any regrouping of the terms 
        \[ (a_1 + a_2 + ... + a_{n_1}) + (a_1 + a_2 + ... + a_{n_2}) + (a_1 + a_2 + ... + a_{n_3}) + ...\]
        leads to a series that also converges to \(L\).
\begin{proof}
Our goal is to show that the associative property for a converging infinite series holds. Let us define the terms of the subsequence 
\begin{align*}
    b_1 &= a_1 + a_2 + a_3 + ... + a_{n_1} \\
    b_2 &= a_{n_1 + 1}  + a_{n_1 + 2} + a_{n_1 + 3} + \dots + a_{n_2} \\
        &\vdots  \\
    b_m &= a_{n_{m-1}+1} + \dots + a_{n_m}.
\end{align*}
Our goal is to show that the subsequence \((b_m)\) converges to \(L \) as well. Suppose \( \lim s_n = L \). Let the partial sums \( (t_m)\) be regrouped in terms of the subsequence above
\begin{align*}
    t_m&= b_1 + b_2 + \dots + b_m \\
       &= (a_1 + a_2 + a_3 + ... + a_{n_1}) + (a_{n_1 + 1} + \dots + a_{n_2}) + \dots \\
       &+ ( a_{n_{m-1}+1} + \dots + a_{n_m}).
\end{align*}
Since \( \lim s_n = L \), its sequence of partial sums also converge to \(L\). But this means every subsequence of \( (t_k)\) also converges to \(L \). Hence, \((b_m)\) converges to \(L \) as well.
\end{proof}
\item[(b)] Compare this result to the example discussed at the end of Section 2.1 where infinite addition was shown not to associative. Why doesn't our proof in (a) apply to this example?
    \begin{proof}[Solution]
    We cannot have infinite series be associative if the sequence of partial sums diverges. This means we cannot regroup the terms of our partial sums into a subsequence that converges, since there are divergent subsequences. 
    \end{proof}
\end{enumerate}

\subsubsection{Exercise 2.5.5} Assume \((a_n)\) is \textit{bounded} sequence with the property that every convergent subsequence of \((a_n)\) converges to the same limit \(a \in \R\). Show that \((a_n)\) must converge to \(a\).
\begin{proof}
    Suppose for sake of contradiction that \((a_n) \not\to a\). Then there exists \(\epsilon_0 > 0\) such that \(|a_n - a| \geq \epsilon_0\) for all \(N \in \N \). Since \((a_n)\) is \textit{bounded}, we can find a subsequence \((a_{n_k})\) that converges to some \( \ell \in \R \). Since \((a_n) \not\to a\), then \( (a_{n_k}) \to \ell \) where \( \ell \neq a\). Yet we assumed every convergent subsequence of \((a_n)\) converges to the same limit \( a \) but \( a \neq \ell \) which is a contradiction. Hence, it must be the case that \( \lim a_n = a \).
\end{proof}
\subsubsection{Exercise 2.5.6}
Use a similar strategy to the one in Example 2.5.3 to show \(\lim b^{1/n} \) exists for all \( b \geq 0\) and find the value of the limit. (The results in Exercise 2.3.1 may be assumed)
\begin{proof}
    Let \(b \geq 0 \). Our goal is to show that \( \lim b^{1/n}\) exists. We observe that 
    \[b > b^{1/2} > b^{1/3} > b^{1/4} > ... >  b^{1/n} \geq 0\]
    and conclude by induction that \(b^{1/n}\) is a \textit{decreasing}
    sequence. Since \( 0 \leq b^{1/n} < b\), we can also conclude that \( (b^{1/n})\) is a \textit{bounded} sequence. Hence, \((b^{1/n})\) is a convergent sequence. But note that \((b^{1/n}) \to 0\) for all \(b^{1/n} \geq 0\) by exercise 2.3.1. Hence, \(\lim b^{1/n} = 0\).
\end{proof}

\subsubsection{Exercise 2.5.7}
Extend the result proved in Example 2.5.3 to the case \(|b|<1\); that is, show \(\lim(b^n) = 0\) if and only if \(-1 < b < 1\). 
\begin{proof}
Suppose \(\lim b^n = 0 \). Let \( \epsilon = 1\), then there exists \( N \in \N \) such that for any \( n \geq N \), we have \( -1 < b^n < 1 \). Then we have \( -1 < b < 1 \). Hence, \( |b| < 1\). 

Now let us show the converse. Assume \( |b | < 1\); that is, \( -1 < b < 1 \). Since \( 0 \leq b < 1 \) holds by Example 2.5.3, we can write \( \lim b^n = 0  \). Suppose \( -1 < b < 0\). We observe that 
\[ b < b^2 < b^3 < b^4 < ... <  0\]
impying that \(b^n\) is an \textit{increasing} sequence for all \( n \in \N \) for \(b^n \in (-1, 0)\). Furthermore, \((b^n)\) is \textit{bounded} since \(1 < b^n < 0 \). Hence, \((b^n)\) is a convergent sequence by the \textit{Monotone Convergence Theorem}. Hence, \( (b^{2n}) \to \ell \) satisfying \( b < \ell \leq 0 \). Suppose \( \lim (b^n) = \ell\). Let \((b^{2n})\) be a subsequence, then \( (b^{2n})\) also converges to the same limit \(b \). Hence, we have 
\begin{align*}
    \ell = \lim b^n &= \lim b^{2n} \\
                 &= \lim (b^{n} \cdot b^{n}) \\
                 &= \lim b^n \cdot \lim b^n \\ 
                 &= \ell^2
\end{align*}
Then by the same process in Example 2.5.3, we have \( \lim (b^n) = 0 \).

\end{proof}

\subsubsection{Exercise 2.5.6}
Let \((a_n)\) be a \textit{bounded} sequence, and define the set 
\[ S = \{ x \in \R : x < a_n \text{ for infinitely many terms } a_n \}.\]
Show that there exists a subsequence \((a_{n_k})\) converging to \( s = \sup S\). (This is a direct proof of the Bolzano-Weierstrass Theorem using the Axiom of Completeness.) 

\begin{proof}
    Let \((a_n)\) be a \textit{bounded} sequence. We observe that \( S \neq \emptyset \) and \(S \) \textit{bounded above} since there exists \( M > 0\) such that \(|a_n| \leq M \) where \( x < a_n \leq M \). By the \textit{Axiom of Completeness}, \( s = \sup S \) exists. Then by lemma 1.3.8, let \( \epsilon = 1 / n_k \) such that for some \( a_{n_k} \in S \), we have 
    \[ s - \frac{1}{n_k} \leq a_{n_k} \leq s  \]
    Note that we can write \( \lim (s - 1 / n_k) = s \) by the \textit{Algebraic Limit Theorem}. By the \textit{Squeeze Theorem}, it follows that \( a_{n_k} \to s = \sup S\). 
\end{proof}

