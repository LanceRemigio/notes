\section{The Monotone Convergence Theorem}

As we have seen in the last section, convergent sequences are bounded while the converse is not true. But if a sequence is monotone then surely it is convergent. 

\begin{tcolorbox}
\begin{defn}
    A sequence \( (a_n) \) is \textit{increasing } if \( a_n \leq a_{n+1}\) for all \( n \in \N \) and \textit{decreasing} if \( a_n \geq a_{n+1}\) for all \( n \in \N \). A sequence is \textit{monotone} if it is either increasing or decreasing.
\end{defn}
\end{tcolorbox}

\begin{tcolorbox}
    \begin{thm}[Monotone Convergence Theorem]
        If a sequence is monotone and bounded, then it converges.
    \end{thm}
\end{tcolorbox}

\begin{proof}
Let \( (a_n)\) be \textit{monotone} and \textit{bounded}. We need to show that \( (a_n)\) converges to some value \( s \). Let our set of points \( a_n\) be defined as 
\[ A = \{ a_n : \text{ for all } n \in \N  \}\] 
and because we have a bounded sequence, we must have an upper bound \( s \) which can be defined as out supremum i.e
\[ s = \sup \{ a_n : \text{ for all } n \in \N  \}.\]
Let \( \epsilon > 0 \). We need to show that 
\[ |a_n - s| < \epsilon \] 
Since \( s - \epsilon  \) is not an upper bound of \( A \), there exists \( N \in \N \) such that 
\[ s - \epsilon < a_N.\]
Let's assume that \( (a_n)\) is an increasing sequence. By assuming \(n \geq N \), we can say that \( a_n \geq a_N\). Since \( s + \epsilon \) is an upper bound and \( s \) is the least upper bound, then we can say that  
\[ s - \epsilon < a_N \leq a_n < s \leq s + \epsilon \]
which imply that 
\begin{align*}
    &s - \epsilon  <  a_n < s + \epsilon  \\
    &\implies |a_n - s| < \epsilon.  
\end{align*}
Hence, it follows that any \textit{monotone} and \textit{bounded} sequence converges.
\end{proof}

The key takeaway from this theorem is that we don't actually need to specify a value for a limit in order to show that it converges. As long as we have a monotone sequence and that we know it is bounded then we know for sure that the sequence converges.

\begin{tcolorbox}
    \begin{defn}[Convergence of a Series]
        Let \( (b_n)\) be a sequence. An \textit{infinite series} is a formal expression of the form 
        \[ \sum_{n=1}^{ \infty} b_n = b_1 + b_2 + b_3 + \dots { }. \]
        We define the corresponding \textit{sequence of partial sums} \( (s_m)\) by 
        \[ s_m = b_1 + b_2 + b_3 + ... + b_m = \sum_{i=1}^{m} s_i,\]
        and say that the series \( \sum_{n=1}^{ \infty} b_n \) \textit{converges} to \( B \) if the sequence \( (s_m)\) converges to \( B \). In this case, we write 
        \[ \sum_{n=1}^{ \infty} b_n = B.\]
\end{defn}
\end{tcolorbox} 

\begin{ex}
Consider 
\[ \sum_{n=1}^{ \infty} \frac{1}{n^2}.\]
Because the terms in the sum are all positive, the sequence of partial sums are given by 
\[ s_m = \sum_{k = 1}^{ m} \frac{1}{k^2}\]
is increasing. Our goal is to show that this sequence is convergent so that the series converges. We proceed by using the Monotone Convergence Theorem to do this. Since we already have a monotone sequence of partial sums, only we need to do now find an upper bound for \( s_m \). Observe that  
\begin{align*}
s_m &= 1 = \frac{1}{2 \cdot 2} + \frac{1}{3 \cdot 3} + \frac{1}{4 \cdot 4} + ... + \frac{1}{m^2} \\
    &< 1 + \frac{1}{2 \cdot 1} + \frac{1}{3 \cdot 2} + \frac{1}{4 \cdot 3} + ... + \frac{1}{m(m-1)} \\
    &= 1 + \Big(  1 - \frac{1}{2}\Big) + \Big( \frac{1}{2} - \frac{1}{3} \Big) + \Big( \frac{1}{3} - \frac{1}{4} \Big) + ... + \Big( \frac{1}{(m-1)} - \frac{1}{m} \Big) \\
    &= 1 + 1 - \frac{1}{m} \\
    &< 2. 
\end{align*}
The third second equality is found by taking the partial fractions of the line before it.
Thus, we find that \( 2 \) is an upper bound for the sequence of partial sums, so we can conclude that the infinite series 
\[ \sum_{n=1}^{ \infty} \frac{1}{n^2} \]
is convergent.
\end{ex}

\begin{ex}[Harmonic Series]
    Let's consider the Harmonic Series 
    \[ \sum_{n=1}^{ \infty} \frac{1}{n}. \]
    The sequence of partial sums is defined as follows
    \[ s_m = \sum_{k=1}^{m} \frac{1}{k}.\]
    Like our last example, we expect these sequence of terms to be bounded by 2 but upon further inspection, we have 
    \[ s_4 = 1  + \frac{1}{2} + \Big( \frac{1}{3} + \frac{1}{4} \Big) > 1 + \frac{1}{2} + \Big( \frac{1}{4}+ \frac{1}{4} \Big) = 2\]
    which is not true. Similarly, we find that \( s_8 > 2 \frac{1}{2}\), and we can see that in general we have that 
    \begin{align*}
        s_{2^k}&= 1 + \frac{1}{2} + \Big( \frac{1}{3} + \frac{1}{4} \Big) + \Big( \frac{1}{5} + ... + \frac{1}{8} \Big) + ... + \Big( \frac{1}{2^{k-1}} + ... + \frac{1}{2^k} \Big) \\
               &> 1 + \frac{1}{2} + \Big( \frac{1}{4} + \frac{1}{4}\Big) + \Big( \frac{1}{8} + ... + \frac{1}{8}\Big) + ... + \Big( \frac{1}{2^k} + ... + \frac{2}{2^k}\Big) \\
               &= 1 +... + \Big( 2^{k-1} \frac{1}{2^k}\Big) \\ 
               &= 1 + \frac{1}{2} + \frac{1}{2} + ... + \frac{1}{2} \\
               &= 1 + k\frac{1}{2}.
    \end{align*}
    This shows that our sequence is unbounded because we found \( M = 1 + k \Big( \frac{1}{2}\Big) > 0\) such that \( s_k > M \). Despite how slow the sequence of partial of sums may be at reaching this point, it does end up surpassing every number on the postive real line. Since we have an unbounded sequence of partial sums, we conclude that the Harmonic series as divergent.

\end{ex}


\begin{tcolorbox}
    \begin{thm}[Cauchy Condensation Test]
    Suppose \( (b_n)\) is decreasing and satisfies \( b_n \geq 0  \) for all \( n \in \N \). Then, the series 
    \[ \sum_{n=0}^{ \infty} b_n \]
    converges if and only if 
    \[ \sum_{n=0}^{ \infty} 2^n b_{2^n}\]
converges.
\end{thm}
\end{tcolorbox} 

\begin{proof}
    For the forwards direction, assume that \( \sum_{n=0}^{ \infty} 2^n b_{2^n}\) converges. This means that the sequence of partial sums 
    \[ t_k = b_1 + 2b_2 + ... + 2^k b_{2k}\] are bounded. Hence, there exists \( M > 0 \) such that \( t_k \leq M \) for all \( k \in \N \). Our goal is to show that the sequence of partial sums for the series 
    \[ \sum_{n=0}^{ \infty} b_n.\] 
     Since \( b_n \geq 0 \) and that for all \( n \in \N \) \( b_n \) decreasing, we have that the partial sums \( t_k \) is monotone. Our goal is to show that 
     \[ s_m = \sum_{k=0}^{m} b_k  \]
     is bounded. Hence, fix \( m\) and let \( k \) be large enough to ensure \( m \leq 2^{k+1} - 1\) and hence \( s_m \leq s_{2^{k+1} - 1}\) which imply that 
     \begin{align*}
         s_{2^{k+1}-1}&= b_1 + (b_2 + b_3) + (b_4 + b_5 + b_6 + b_7 ) + ... + (b_{2^k} + ... + b_{2^{k+1}-1})\\ 
                      &\leq b_1 + (b_2 + b_2) + (b_4 + b_4 + b_4 + b_4) + ... + (b_{2k} + ... + b_{2k}) \\
                      &= b_1 + 2b_2 + 4b_4 + ... + 2^k b_{2k} \\
                      &= t_k 
     \end{align*}
     Hence, we have \( s_m \leq s_{2^{k+1} - 1} < t_k \leq M \) which means that \( (s_m)\) is bounded. By the Monotone Convergence Theorem, it follows that the series \( \sum_{n=1}^{ \infty} b_n \) converges. 
     For the forwards direction, we proceed with contrapostive. Hence, assume for sake of contradiction that the series 
     \[ \sum_{n=0}^{ \infty} 2^n b_{2^n}\]
     is a divergent series. We want to show that the series 
     \[ \sum_{n=0}^{ \infty} b_n \]
     is also a divergent series.
\end{proof}

