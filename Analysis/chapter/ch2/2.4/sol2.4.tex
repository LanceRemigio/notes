\section{The Monotone Convergence Theorem and a First Look at Infinite Series}

\subsubsection{Exercise 2.4.1}
\begin{enumerate}
    \item[(a)] Prove that the sequence defined by \( x_1 = 3\) and 
        \[ x_{n+1} = \frac{1}{4 - x_n}\]
        for all \( n \in \N \) converges.
        \begin{proof}
            Let \( (x_n)\) be the sequence defined by \( x_1 = 3\) and 
            \[ x_{n+1} = \frac{1}{4-x_n}\]
            for all \( n \in \N\). Our goal is to show that \( (x_n)\) is convergent. It is sufficient to show that \( (x_n)\) is both \textit{monotone} and \textit{bounded}. We first show that \( (x_n)\) is \textit{monotone}. We claim that \( (x_n)\) is a \textit{decreasing} sequence. Hence, we will show that for all \( n \in \N \), we have \( x_n > x_{n+1}\). We proceed by inducting on \( n \). Let the base case be \( n = 1 \). Then we have that 
            \begin{align*}
                x_1 = 3 &> x_2 = \frac{1}{4 - 3} = 1. \\ 
            \end{align*}
            Hence, we have \( x_1 > x_2 \). Now we assume that \( (x_n)\) is decreasing for all \( 1 < n \leq k-1 \). We want to show that \( x_n > x_k \) for all \( n < k \). Since \( n \leq k - 1\), we have \( x_{k-1} \leq x_n \) by inductive hypothesis. Consider \( x_k \). By definition, we have that \( x_k = 1 /(4 - x_{k-1})\). Since \( x_{k-1} \leq x_n \), then for all \( n \in \N \) we have 
            \[
                x_k = \frac{1}{4 - x_{k-1}} < \frac{1}{4-x_n}.
            \]
            Hence, \( x_k < x_n \) for all \( n \in \N \). This is equivalent to showing \( x_n > x_{n+1}\) for all \( n \in \N \). Therefore, \( (x_n)\) is a \textit{monotone} sequence. 
            Now we show that \( (x_n)\) is \textit{bounded}. Since \( 3 = x_1 \geq x_{n+1}\) for all \( n \in \N \) and \( x_{n+1} = 1 / (4 - x_n) > 0 \), we have that 
            \[ 0 < x_n \leq 3.\]
            Hence, \( (x_n)\) is bounded. Since \( (x_n)\) is \textit{monotone} and \textit{bounded}, we have that 
            \( (x_n)\) is a convergent sequence by the Monotone Convergence theorem. 
        \end{proof}
    \item[(b)] Now that we know \( \lim x_n \) exists, explain why \( \lim x_{n+1}\) must also exist and equal the same value.
        \begin{proof}[Solution]
            Since \( (x_n) \) is \textit{monotone} and \textit{bounded}, then \( (x_{n+1})\) is also \textit{monotone} and \textit{bounded}. By the Monotone Convergence Theorem, we have that \( (x_{n+1})\) is also convergent. Hence, \( \lim x_{n+1}\) also exists.    
        \end{proof}
    \item[(c)] Take the limit of each side of the recursive equation in part (a) to explicitly compute \( \lim x_n \). 
        \begin{proof}[Solution]
            Since \( \lim x_n = \lim x_{n+1}\), we have
            \begin{align*}
                x = \lim x_{n+1} &= \lim \frac{1}{4-x_n}  \\ 
                                 &= \frac{ \lim 1 }{ \lim (4 - x_n ) } \\ 
                                 &= \frac{1}{ \lim (4) - \lim x_n} \\
                                 &= \frac{1}{4 - x} \tag{ \( \lim x_n = x \)}.
            \end{align*}
            Then we have 
            \begin{align*}
                x &= \frac{1}{4-x} \\
            \end{align*}
            and then 
            \begin{align*}
                x^2 - 4x + 1 = 0 \\ 
            \end{align*}
            which we can solve via the quadratic formula. Hence, we have \( x = 2 + \sqrt{3}\). 
        \end{proof}

\end{enumerate}

\subsubsection{Exercise 2.4.3}
Following the model of Exercise 2.4.2, show that the sequence defined by \( y_1 = 1 \) and \( y_{n+1} = 2 - \frac{1}{y_n}\) converges and find the limit.

\begin{proof}
Let \( (y_n)\) be the sequence defined by \( y_1 = 1 \) and 
\[
    y_{n+1} = 4 - \frac{1}{y_n}.
\] for all \( n \in \N \). 
We want to show that \( (y_n)\) converges. Hence, our goal is to show that \( (y_n)\) is \textit{monotone} and \textit{bounded}. We claim that \( (y_n)\) is increasing. Hence, we show this by inducting on \( n \in \N \). Our goal is to show that \( y_n \leq y_{n+1}\) for all \( n \in \N \). Let the base case be \( n = 1 \). Then observe that 
\begin{align*}
 y_1 &= 1 < y_2 = 4 - \frac{1}{1} = 3  \\ 
\end{align*}
Hence, we have \( y_1 < y_2\). 

Now assume that \( (y_n)\) is increasing for all \( 1 \leq n \leq k - 1\). Hence, \( y_n \leq y_{k-1}\). Our goal now is to show that \( y_n \leq y_k\) for all \( n \in \N \). Let's consider \( y_k\). Then by definition of \( (y_n)\), we have 
\[
    y_k = 4 - \frac{1}{y_{k-1}}.
\]
Since \( y_n \leq y_{k-1}\), we have 
\[
y_k = 4 - \frac{1}{y_{k-1}} \geq 4 - \frac{1}{y_{n}}
\]
This shows that \( y_{k} \geq y_n \) for any \( n \in \N \). Hence, it follows that \( y_{n}\) is an increasing sequence and, therefore, \textit{monotone}. Now

Now we want to show that \( (y_n)\) is \textit{bounded}. Observe that \( 1 < y_n \) for all \(n \in \N \) which means \( (y_n)\) contains a lower bound. Furthermore, for each \(n \in \N\) we also have that \( y_{n+1} = 4 - 1/y_n < 4\) which means that \( (y_n)\) also contains an uppper bound. Hence, it follows that 
\[
1 < y_n < 4
\]
for all \( n \in \N \). Hence, we have \( (y_n)\) is \textit{bounded}. By the Monotone Convergence Theorem, it follows that \( (y_n)\) is a convergent sequence. 

By last exercise, we know that \( \lim y_n = \lim y_{n+1}  \). Let's assume \( (y_n) \to y \). Our goal is to compute \( \lim y_n \). By the Algebraic Limit Theorem, we have 
\begin{align*}
    y = \lim y_n &= \lim \Big( 4 - \frac{1}{y_n}\Big)  \\ 
                 &= \lim (4) - \lim \Big( \frac{1}{y_n}\Big) \\
                 &= 4 - \frac{ \lim(1) }{ \lim y_n } \\
                 &= 4 - \frac{1}{y}.
\end{align*}
Hence, we have 
\begin{align*}
y &= 4 - \frac{1}{y} \\ 
\end{align*}
which yields the following quadratic equation set to zero
\[
y^2 -4y + 1 = 0. 
\]
Solving for \( y \) using the quadratic formula yields \( y = 2 + \sqrt{3}\)
\end{proof}

\subsubsection{Exercise 2.4.5 (Calculating Square Roots)}
Let \(x_1 = 2\), and define 
\[
    x_{n+1} = \frac{1}{2} \Big( x_n + \frac{2}{x_n}\Big).
\]

\begin{enumerate}
    \item[(a)] Show that \( x_n^2\) is always greater than 2, and then use this to prove that \( x_n - x_{n+1} \geq 0\). Conclude that \( \lim x_n = \sqrt{2} \).
        \begin{proof}
         Our first goal is to show that \( x_n^2 > 2 \) for all \( n \in \N \). We proceed by inducting on \( n \in \N\). Let our base case be \( n = 1\). Then 
         \begin{align*}
         x_1 = 2 &< x_1^2   \\
                 &= 4 \\ 
                 &< \frac{9}{4} \\
                 &= \frac{1}{4} \Big( x_1^2 + \frac{4}{x_1^2} + 4 \Big) \\
                 &= x_2^2
         \end{align*}
         which implies that \( 2 < x_1^2 < x_2^2\). Now suppose \( x_{k-1}^2 > 2\) for all \( n \leq k - 1\). We want to show that \( x_k^2 > 2 \) for all \( n \in [1, k)\). Consider \( x_k \) and then by definition, we have 
         \begin{align*}
             x_{k}^2 &= \frac{1}{4} \Big( x_{k-1}^2 + \frac{4}{x_{k-1}^2} + 4 \Big) \\
                   &> \frac{1}{4} (2 + 2 + 4) \\ 
                   &= \frac{8}{4} \\
                   &= 2.\\
        \end{align*}
        Hence, \( x_k^2 > 2\) for all \( n \in \N\).  Now we want to show that \( x_n - x_{n+1} \geq 0  \) for all \( n \in \N \). Consider \( x_n - x_{n+1}\) then observe that since \( x_n^2 > 2 \) for all \( n \in \N \), we have 
        \begin{align*}
            x_n - x_{n+1}&= x_n - \frac{1}{2} \Big( x_n + \frac{2}{x_n}\Big) \\ 
                        &= \frac{x_n^2 - 2 }{2x_n} \\
                        &> \frac{2 - 2 }{2 \sqrt{2}} \\
                        &= 0. 
        \end{align*}
        Furthermore, when \( x^2 = 2 \) we get that \( x_n - x_{n+1} = 0 \). Hence, we have \( x_n - x_{n+1} \geq 0\) for all \( n \in \N \). By the Monotone Convergence Theorem, we get that \( (x_n)\) is a convergent sequence. Since \( \lim x_n = \lim x_{n+1}  \), we can show that \( \lim x_n = \sqrt{2}\). By the Algebraic Limit Theorem, we have 
        \begin{align*}
            x = \lim x_{n+1} &= \lim \Big( \frac{1}{2} \Big( x_n + \frac{2}{x_n}\Big)\Big)  \\ 
                            &= \frac{1}{2} \lim \Big( x_n + \frac{2}{x_n}\Big) \\ 
                            &= \frac{1}{2} \Big( \lim x_n + \lim  \frac{2}{x_n}  \Big) \\
                            &= \frac{1}{2} \Big(x + \frac{2}{x}\Big) \\
                            &= \frac{1}{2}x + \frac{1}{x} \\
                            &= \frac{x^2 + 2}{2x} \\
        \end{align*}
        which implies that 
        \begin{align*}
        x^2 &= 2 \\ 
        \implies x &= \sqrt{2}.
        \end{align*}
        Hence, we have \( \lim x_n = \sqrt{2} \)
    \end{proof} 
    \item[(b)] Modify the sequence \( (x_n)\) so that it converges to \( \sqrt{c}\).
        \begin{proof}[Solution]
         Let the sequence \( (x_n)\) be defined recursively as \( x_1  =c \) and 
        \[
            x_{n+1} = \frac{1}{c} \Big( x_n + \frac{c}{x_n}\Big). 
        \] 
        Assume \( x_n^2 > c \) for all \( n \in \N \) and \( x_n - x_{n+1} \geq 0\), then we have \( \lim x_n = \sqrt{c} \).
        \end{proof}
        
\end{enumerate}

\subsubsection{Exercise 2.4.6 (Limit Superior.) }
Let \( (a_n)\) be a \textit{bounded} sequence. 
\begin{enumerate}
    \item[(a)] Prove that the sequence defined by \( y_n = \sup \{ a_k : k \geq n \}\) converges. 
        \begin{proof}
        Let \( (a_n)\) be a \textit{bounded} sequence. Let \( (y_n)\) be defined as 
        \[ y_n = \sup \{ a_k : k \geq n \}\] and further denote
        \[ A_n = \{ a_k : k \geq n\}.\]
        Our goal is to show that \( (y_n)\) converges. Thus, our goal is to show that \( (y_n)\) is both \textit{monotone} and \textit{bounded}.

        We first show the former. We claim that \( (y_n)\) is a \textit{decreasing} sequence. Hence, we need to show that \( y_n \geq y_{n+1}\) for all \( n \in \N \). We know that for each term \( y_n \), by definition, is the \textit{least upper bound} of the set \( \{a_k: k \geq n\}\). By exercise 1.3.4, we know that since \( A_{n+1} \subseteq A_n \), we have that \( y_{n+1} \leq y_n\) for all \( n \in \N \). Hence, we have that \( y_n\) is a \textit{decreasing} sequence.

        Now we want to show that \( (y_n)\) is \textit{bounded}. Since \( (a_n)\) is a \textit{bounded} sequence, there exists an \( M > 0 \) such that \( M \leq a_n \leq y_n  \) for all \( n \in \N \). Hence, \( (y_n)\) is a \textit{bounded} sequence. In this case, it is enough to have \( (y_n)\) be bounded below.

        Since \( (y_n)\) is \textit{monotone} and \textit{bounded}, we have that \( (y_n)\) is a convergent sequence.

        \end{proof}
    \item[(b)] The \textit{limit superior} of \( (a_n)\), or \( \lim \sup a_n   \) is defined by 
        \[ \lim  \sup a_n = \lim y_n   \] where \( y_n \) is the sequence from part (a) of this exercise. Provide a reasonable definition for \( \lim \inf a_n  \) and briefly explain why it always exists for any bounded sequence.
      \begin{proof}[Solution]
       Let \( a_n \) be a \textit{bounded} sequence. Then define   
       \[ w_n = \inf \{ a_k : k \geq n  \}\]
       so we can have 
       \[ \lim w_n = \lim \inf\{a_k : k \geq n\}. \]
      This limit exists because the terms of \( (w_n) \) are \textit{increasing} and that \( (w_n) \) is \textit{bounded} since there exists an \( L > 0 \) such that \( w_n \leq a_n \leq L \).


       \end{proof} 
    \item[(c)] Prove that \( \lim \inf a_n \leq \lim \sup a_n  \) for every \textit{bounded} sequence, and give an example of a sequence for which the inequality is strict.
        \begin{proof}
        Suppose that \( (a_n)\) is a \textit{bounded} sequence. Since \( \lim \inf a_n \) and \( \lim \sup a_n \) exists, we have that \( \lim \inf a_n \leq a_n \leq \lim \sup a_n  \). Hence, 
        \[ \lim \inf a_n \leq \lim \sup a_n.  \]
        \end{proof}
    \item[(d)] Show that \( \lim \inf a_n = \lim \sup a_n\) if and only if \( \lim a_n \) exists. In this case, all three share the same value.
        \begin{proof}
        We start by assuming that \( \lim \inf a_n = \lim \sup a_n \). We want to show that the \( \lim a_n  \) exists. Define the following:
        \begin{align*}
        w &= \lim \inf a_n = \lim w_n ,  \\ 
        y &= \lim \sup a_n = \lim y_n . 
        \end{align*} 
        Since \( w_n \) and \( y_n\) both bound \( a_n \), it follows that \( w_n \leq a_n \leq y_n\).
        Since the \( \lim \inf a_n \) and \( \lim \sup a_n \) both exists such that \( \lim  w_n = \lim y_n = \ell \), it follows that \( \lim a_n = \ell \) by the \textit{Squeeze Theorem}. Hence, \( \lim a_n \) exists.

        Now assume the converse. Hence, our goal now is to show that \( \lim \inf a_n = \lim \sup a_n\). Since \( (a_n)\) is a convergent sequence, let \( \epsilon > 0  \) such that there exists an \( N \in \N\) where for every \( n \geq N \), we have
        \[ |a_n - \ell| < \epsilon. \]  
        This is equivalent to saying 
        \[ \ell - \epsilon \leq a_n \leq \ell + \epsilon. \]
        Since \( \epsilon > 0  \) is abitrary, we can conclude that \( \lim \inf a_n = \lim \sup a_n  \).

        \end{proof}
        
\end{enumerate}


\subsubsection{Exercise 2.4.9} Complete the proof of Theorem 2.4.6 by showing that if the series \( \sum_{n=0}^{ \infty} 2^n b_{2n}  \) diverges, then so does \( \sum_{n=1}^{ \infty} b_n \). Example 2.4.5 may be a useful reference.

\begin{proof}
    Our goal is to show that \( \sum_{n=1}^{ \infty} b_n\) diverges. Since \( \sum_{n=0}^{ \infty} b_{{2^n}} \) diverges, the sequence of partial sums 
    \[t_k = b_1 + 2b_2 + 4b_4 + ... + 2^k b_{2^k}\]
    diverges. This implies that \( t_k \) is \textit{unbounded}: that is, there exists \( M > 0 \) such that for some \( K \in \N \), we have for all \( k \geq K \) we have \( t_k > M \). Since \( b_n \geq 0\), it suffices to show that the partial sums of \( \sum_{n=1}^{ \infty} b_n\) are \textit{unbounded}. Let 
    \[ s_m = \sum_{k=1}^{m} b_k = b_1 + b_2 + b_2 + ... + b_m.  \]
    Let us fix \( m \) such that we choose \( k \) sufficiently large so that \( m \geq 2^{k+1} + 1 \). Hence, we have that \( s_m \geq s_{2^{k-1} + 1} \). Observe that 
    \begin{align*}
        s_{2^{k+1}+1} &= b_1 + (b_2 + b_3) + (b_4 + b_5 + b_6 + b_7) + ... + (b_{2^k} + ... + b_{2^{k+1}+1})  \\ 
                      &\geq b_1 + (b_2 + b_2) + (b_4 + b_4 + b_4 + b_4) + ... + (b_{2k} + ... + b_{2k}) \\
                      &= b_1 + 2b_2 + 4b_4 + ... + 2^k b_{2k} \\
                      &= t_k.
    \end{align*}
    This implies that \( s_m \geq s_{2^{k+1}+1} \geq t_k > M \) for all \( k \geq K\). Hence, we conclude \( t_k\) is \textit{unbounded}. Thus, the series 
    \[ \sum_{n=1}^{ \infty} b_n.\]
    diverges.
\end{proof}

















\subsubsection{Exercise 2.4.10 (Infinite Products)}
A close relative of infinite series is the \textit{infinite product} 
\[ \prod_{n=1}^{ \infty } b_n = b_1 b_2 b_3 \dots \]
which is understood in terms of its sequence of \textit{partial products}
\[ p_m = \prod_{n=1}^m b_n = b_1b_2b_3 \dots b_m.\]
Consider the special class of infinite products of the form 
\[ \prod_{n=1}^{ \infty} (1 + a_n) = (1+a_1)(1+a_2)(1+a_3) \dots ~ \text{where } a_n \geq 0 \]
\begin{enumerate}
    \item[(a)] Find an explicit formula for the sequence of partial products in the case where \( a_n = 1 / n \) and decide whether the sequence converges. Write out the first few terms in the sequence of partial products in the case where \( a_n = 1 / n^2\) and make a conjecture about the convergence of this sequence.
    \item[(b)] Show, in general, that the sequence of partial products converges if and only if \( \sum_{n=1}^{ \infty} a_n \) converges. (The inequality \( 1 + x \leq 3^x\)) for positive \( x \) will be useful in one direction.)
\end{enumerate}

\subsubsection{Exercise 2.4.6 (Limit Superior.) }
Let \( (a_n)\) be a \textit{bounded} sequence. 
\begin{enumerate}
    \item[(a)] Prove that the sequence defined by \( y_n = \sup \{ a_k : k \geq n \}\) converges. 
        \begin{proof}
        Let \( (a_n)\) be a \textit{bounded} sequence. Let \( (y_n)\) be defined as 
        \[ y_n = \sup \{ a_k : k \geq n \}\] and further denote
        \[ A_n = \{ a_k : k \geq n\}.\]
        Our goal is to show that \( (y_n)\) converges. Thus, our goal is to show that \( (y_n)\) is both \textit{monotone} and \textit{bounded}.

        We first show the former. We claim that \( (y_n)\) is a \textit{decreasing} sequence. Hence, we need to show that \( y_n \geq y_{n+1}\) for all \( n \in \N \). We know that for each term \( y_n \), by definition, is the \textit{least upper bound} of the set \( \{a_k: k \geq n\}\). By exercise 1.3.4, we know that since \( A_{n+1} \subseteq A_n \), we have that \( y_{n+1} \leq y_n\) for all \( n \in \N \). Hence, we have that \( y_n\) is a \textit{decreasing} sequence.

        Now we want to show that \( (y_n)\) is \textit{bounded}. Since \( (a_n)\) is a \textit{bounded} sequence, there exists an \( M > 0 \) such that \( M \leq a_n \leq y_n  \) for all \( n \in \N \). Hence, \( (y_n)\) is a \textit{bounded} sequence. In this case, it is enough to have \( (y_n)\) be bounded below.

        Since \( (y_n)\) is \textit{monotone} and \textit{bounded}, we have that \( (y_n)\) is a convergent sequence.

        \end{proof}
    \item[(b)] The \textit{limit superior} of \( (a_n)\), or \( \lim \sup a_n   \) is defined by 
        \[ \lim  \sup a_n = \lim y_n   \] where \( y_n \) is the sequence from part (a) of this exercise. Provide a reasonable definition for \( \lim \inf a_n  \) and briefly explain why it always exists for any bounded sequence.
      \begin{proof}[Solution]
       Let \( a_n \) be a \textit{bounded} sequence. Then define   
       \[ w_n = \inf \{ a_k : k \geq n  \}\]
       so we can have 
       \[ \lim w_n = \lim \inf\{a_k : k \geq n\}. \]
      This limit exists because the terms of \( (w_n) \) are \textit{increasing} and that \( (w_n) \) is \textit{bounded} since there exists an \( L > 0 \) such that \( w_n \leq a_n \leq L \).


       \end{proof} 
    \item[(c)] Prove that \( \lim \inf a_n \leq \lim \sup a_n  \) for every \textit{bounded} sequence, and give an example of a sequence for which the inequality is strict.
        \begin{proof}
        Suppose that \( (a_n)\) is a \textit{bounded} sequence. Since \( \lim \inf a_n \) and \( \lim \sup a_n \) exists, we have that \( \lim \inf a_n \leq a_n \leq \lim \sup a_n  \). Hence, 
        \[ \lim \inf a_n \leq \lim \sup a_n.  \]
        \end{proof}
    \item[(d)] Show that \( \lim \inf a_n = \lim \sup a_n\) if and only if \( \lim a_n \) exists. In this case, all three share the same value.
        \begin{proof}
        We start by assuming that \( \lim \inf a_n = \lim \sup a_n \). We want to show that the \( \lim a_n  \) exists. Define the following:
        \begin{align*}
        w &= \lim \inf a_n = \lim w_n ,  \\ 
        y &= \lim \sup a_n = \lim y_n . 
        \end{align*} 
        Since \( w_n \) and \( y_n\) both bound \( a_n \), it follows that \( w_n \leq a_n \leq y_n\).
        Since the \( \lim \inf a_n \) and \( \lim \sup a_n \) both exists such that \( \lim  w_n = \lim y_n = \ell \), it follows that \( \lim a_n = \ell \) by the \textit{Squeeze Theorem}. Hence, \( \lim a_n \) exists.

        Now assume the converse. Hence, our goal now is to show that \( \lim \inf a_n = \lim \sup a_n\). Since \( (a_n)\) is a convergent sequence, let \( \epsilon > 0  \) such that there exists an \( N \in \N\) where for every \( n \geq N \), we have
        \[ |a_n - \ell| < \epsilon. \]  
        This is equivalent to saying 
        \[ \ell - \epsilon \leq a_n \leq \ell + \epsilon. \]
        Since \( \epsilon > 0  \) is abitrary, we can conclude that \( \lim \inf a_n = \lim \sup a_n  \).

        \end{proof}
        
\end{enumerate}


\subsubsection{Exercise 2.4.9} Complete the proof of Theorem 2.4.6 by showing that if the series \( \sum_{n=0}^{ \infty} 2^n b_{2n}  \) diverges, then so does \( \sum_{n=1}^{ \infty} b_n \). Example 2.4.5 may be a useful reference.

\begin{proof}
    Our goal is to show that \( \sum_{n=1}^{ \infty} b_n\) diverges. Since \( \sum_{n=0}^{ \infty} b_{{2^n}} \) diverges, the sequence of partial sums 
    \[t_k = b_1 + 2b_2 + 4b_4 + ... + 2^k b_{2^k}\]
    diverges. This implies that \( t_k \) is \textit{unbounded}: that is, there exists \( M > 0 \) such that for some \( K \in \N \), we have for all \( k \geq K \) we have \( t_k > M \). Since \( b_n \geq 0\), it suffices to show that the partial sums of \( \sum_{n=1}^{ \infty} b_n\) are \textit{unbounded}. Let 
    \[ s_m = \sum_{k=1}^{m} b_k = b_1 + b_2 + b_2 + ... + b_m.  \]
    Let us fix \( m \) such that we choose \( k \) sufficiently large so that \( m \geq 2^{k+1} + 1 \). Hence, we have that \( s_m \geq s_{2^{k-1} + 1} \). Observe that 
    \begin{align*}
        s_{2^{k+1}+1} &= b_1 + (b_2 + b_3) + (b_4 + b_5 + b_6 + b_7) + ... + (b_{2^k} + ... + b_{2^{k+1}+1})  \\ 
                      &\geq b_1 + (b_2 + b_2) + (b_4 + b_4 + b_4 + b_4) + ... + (b_{2k} + ... + b_{2k}) \\
                      &= b_1 + 2b_2 + 4b_4 + ... + 2^k b_{2k} \\
                      &= t_k.
    \end{align*}
    This implies that \( s_m \geq s_{2^{k+1}+1} \geq t_k > M \) for all \( k \geq K\). Hence, we conclude \( t_k\) is \textit{unbounded}. Thus, the series 
    \[ \sum_{n=1}^{ \infty} b_n.\]
    diverges.
\end{proof}


\subsubsection{Exercise 2.4.10 (Infinite Products)}
A close relative of infinite series is the \textit{infinite product} 
\[ \prod_{n=1}^{ \infty } b_n = b_1 b_2 b_3 \dots \]
which is understood in terms of its sequence of \textit{partial products}
\[ p_m = \prod_{n=1}^m b_n = b_1b_2b_3 \dots b_m.\]
Consider the special class of infinite products of the form 
\[ \prod_{n=1}^{ \infty} (1 + a_n) = (1+a_1)(1+a_2)(1+a_3) \dots ~ \text{where } a_n \geq 0 \]
\begin{enumerate}
    \item[(a)] Find an explicit formula for the sequence of partial products in the case where \( a_n = 1 / n \) and decide whether the sequence converges. Write out the first few terms in the sequence of partial products in the case where \( a_n = 1 / n^2\) and make a conjecture about the convergence of this sequence.
        \begin{proof}
        Let \( a_{n} = 1 / n  \). Then observe that 
        \[  \prod_{k=1}^{n} \Big(  1 + \frac{ 1 }{ k }  \Big) = \prod_{k=1}^{n} \frac{ k+ 1  }{ k }. \tag{1} \] As we write out the first few terms of the finite product above, we find that the \(  n \)th term of the denominators of each term is cancelled out. Hence, the explicit formula for (1) above must be 
        \[  \prod_{k=1}^{n} \frac{ k+1  }{ k }  = n + 1\] which clearly diverges as \( n \to \infty  \). If \( a_{n} = 1/ n^{2} \), then we find that 
        \[ \gamma_n = \prod_{k=1}^{n} \Big( 1 + \frac{ 1 }{ k^{2} }   \Big) = \prod_{k=1}^{n} \frac{ k^{2} + 1  }{ k^{2}  } = \frac{ 2 }{ 1 } \cdot \frac{ 5 }{ 4 } \cdot \frac{ 10 }{ 9 } \cdot \dotsb \cdot \frac{ n^{2} + 1  }{ n^{2} }. \] I postulate that \( \gamma_n  \) converges similarly to how the sequence of partial sums \( \sum_{ k=1 }^{ n } 1 / k^{2} \) also converges.
        \end{proof}
    \item[(b)] Show, in general, that the sequence of partial products converges if and only if \( \sum_{n=1}^{ \infty} a_n \) converges. (The inequality \( 1 + x \leq 3^x\)) for positive \( x \) will be useful in one direction.)
        \begin{proof}
        \( (\Rightarrow)  \) Suppose the sequence of partial products 
        \[  \gamma_n = \prod_{k=1}^{n} ( 1 + a_{n} )\] converges. Then we must have \( (\gamma_{n}) \) is bounded by some \( M > 0  \). Furthermore, expanding \( (\gamma_{n}) \), we find that \(  \gamma_{n} > \sum_{ k=1 }^{ n } a_{k}    \) for all \(  n \in \N  \). This means \( \sum_{ k=1 }^{ n } a_{k }  \) is bounded. Since \( a_{n} \geq 0  \) for all \( n  \), we also have that \( \sum_{ k=1 }^{ n } a_{n}   \) is a monotone sequence. By the Monotone Convergence Theorem, \( \sum_{ k=1 }^{ n } a_{k } \) converges.

        \( (\Leftarrow)  \) Now suppose the converse. Using the inequality \( 1 + x \leq 3^{x} \) and the fact that \( \sum_{ k=1 }^{ n  } a_{k}  \) is a bounded by some \( M > 0  \), observe that 
        \[  \gamma_{n} = \prod_{k=1}^{n} ( 1 + a_{n}) \leq \prod_{k=1}^{n} 3^{a_{n}} = 3^{\sum_{ k=1 }^{ n } a_{k}} \leq 3^{M}.  \] Hence, \( (\gamma_{n})   \) is a bounded sequence. Now observe that \( a_{n} \geq 0  \) implies \( (\gamma_{n})  \) is increasing. By the Monotone Convergence Theorem, \( \gamma_{n} \) converges.
        \end{proof}
\end{enumerate}





