\documentclass[12pt]{book}

% 
\usepackage{amsmath}
\usepackage{amsthm}
\usepackage{amssymb}
\usepackage{hyperref,enumitem,tcolorbox}
\usepackage{fancyhdr, bookmark, parskip}
\usepackage[hmarginratio=1:1]{geometry}
\setlength{\headheight}{28pt}
\newcommand{\R}{\mathbb{R}}
\newcommand{\C}{\mathbb{C}}
\newcommand{\Q}{\mathbb{Q}}
\newcommand{\Z}{\mathbb{Z}}
\newcommand{\F}{\mathbb{F}}
\newcommand{\N}{\mathbb{N}}
\usepackage{tcolorbox}
\usepackage{graphicx}
\newtheorem{thm}{Theorem}[section]
\newtheorem{prop}{Proposition}[section]
\newtheorem{cor}{Corollary}[thm]
\newtheorem{lem}{Lemma}[section]
\theoremstyle{definition}
\newtheorem{defn}{Definition}[section]
\newtheorem*{rem}{Remark}
\newtheorem*{ex}{Example}
\renewcommand\qedsymbol{$\blacksquare$}





\usepackage{amsmath}
\usepackage{amsthm}
\usepackage{amssymb}
\usepackage{hyperref,enumitem,tcolorbox}
\usepackage{fancyhdr, bookmark, parskip}
\usepackage[hmarginratio=1:1]{geometry}
\setlength{\headheight}{28pt}
\newcommand{\R}{\mathbb{R}}
\newcommand{\C}{\mathbb{C}}
\newcommand{\Q}{\mathbb{Q}}
\newcommand{\Z}{\mathbb{Z}}
\newcommand{\F}{\mathbb{F}}
\newcommand{\N}{\mathbb{N}}
\usepackage{tcolorbox}
\usepackage{graphicx}
\newtheorem{thm}{Theorem}[section]
\newtheorem{prop}{Proposition}[section]
\newtheorem{cor}{Corollary}[thm]
\newtheorem{lem}{Lemma}[section]
\theoremstyle{definition}
\newtheorem{defn}{Definition}[section]
\newtheorem*{rem}{Remark}
\newtheorem*{ex}{Example}
\renewcommand\qedsymbol{$\blacksquare$}


\title{Understanding Analysis Notes}
\author{Lance Remigio}

\begin{document}

\maketitle
\tableofcontents

\chapter{The Real Numbers}


\section{The Axiom of Completeness}

\begin{tcolorbox}
    \begin{thm}[Axiom of Completeness]
        Every nonempty set of real numbers that is bounded above has a least upper bound.
    \end{thm}
\end{tcolorbox}

\begin{tcolorbox}
    \begin{defn}
        We call a set \( A \subseteq \R \) is \textit{bounded above} if there exists a number \( b \in \R \) such that \( a \leq b \) for all \( a \in A \).
        Otherwise, a set is \textit{bounded below} if there exists a \( \ell \in \R \) satisfying \( \ell \leq a \) for every \( a \in A \) .
    \end{defn}

\end{tcolorbox}

\begin{tcolorbox}

    \begin{defn}
        A real number \( s \) is the \textit{least upper bound} for a set \( A \subseteq \R \) if it satisfies the following criteria:

        \begin{enumerate}
            \item[(i)] \( s \) is an upper bound for \( A \);
            \item[(ii)] if \( b \) is any upper bound for \( A \), then \( s \leq b \). 
        \end{enumerate}


        
    \end{defn}


\end{tcolorbox}

We denote the least upper bound of a set \( A \) by calling it the \textit{supremum} of \( A \) i.e \( \sup(A) \). Similarly, we denote the greatest upper bound of set \( A \) by calling it the \textit{infimum} of \( A \) i.e \( \inf(A) \).

Note that a set can have many upper/lower bounds. But there can only exist one supremum and one infimum. In other words, these bounds are unique. Furthemore, the infimum and supremum need not be in the set.

Consider the following set
\[ A = \bigg\{ \frac{1}{n} : n \in \N \bigg\} = \bigg\{ 1,\frac{1}{2}, \frac{1}{3},... \bigg\}\]

This set is bounded above and below. In addition, we can see that \( \sup(A) = 1 \) and \( \inf(A) = 0 \) (this is because each subsequent number in the sequence gets smaller and smaller). 


\begin{tcolorbox}
    \begin{defn}
        We say that \( a_0 \in \R \) is a \textit{maximum} of the set \( A \) if \( a_0 \in A \) and \( a \leq a_0 \) for all \( a \in A \). Likewise, we say that \(a_1 \in \R \) is a \textit{minimum} of \( A \) if \( a_1 \in A \) and \( a \geq a_1 \) for every \( a \in A \).  
    \end{defn}
\end{tcolorbox}

If we have an open set \( (0,2) \) then the end points of this set are the infimum and supremum of the set respectively. Note that the maximum and the minimum do not exists because the infimum and the supremum are not in the set. If this set were to be closed, then the supremum and infimum would be in the set which implies that the max and min exists. 

Now consider the Example

\[ S = \{ r \in \Q: r^2 < 2 \} \]

Notice that when we try and search for the supremum for this set, we cannot find one since we can always find a smaller number for an upper bound. One might say that \( r = \sqrt{2} \) is the supremum of \( S \) but this is false since \( r \not\in \Q \) and is irrational. 


\begin{ex}
    Let \( A \subseteq \R \) such that \( A \neq \emptyset \) and is bounded above. Let \( c \in \R \). Define the set 
    \( c + A \) by 
    \[ c + A = \{ c + a: a \in A  \}  \] 
    Prove that \( \sup(c+A) = c + \sup(A) \) 

    \begin{proof}
        We use defintion 0.2 to prove this proposition. First, we need to prove that this \( \sup(c + A) \) is an upper bound. We have \( \sup(A) = s \) for some \( s \in A \) if \( s \geq a \) for all \( a \in A \). We find that adding \( c \in \R \) gives us
        \[  c+s \geq c + a.\]
        Hence, we have that \( c + s \) is an upper bound for the set \( c + A \). 

        Next, we prove that \( \sup(c+ A) = c + s \) is the \textit{least upper bound}. We know that \( c + s \geq c + a \) for all \( a \in A \). Suppose we have another upper bound \( b \in A \) such that \( c + a \leq b \) for all \( a \in A \). Another manipulation gives us \( a \leq b - c \) for all \( a \in A \). Since \( \sup(A) = s \) is the least upper bound for \( A \), it follows that 
        \( s \leq b - c \). Hence, we have 
        \[ c+s \leq b \implies \sup(c+A) = c + \sup(A). \]
    \end{proof}
\end{ex}

There is another way to restate part (ii) of defintion 0.2 i.e 
\begin{tcolorbox}
\begin{lem}
Assume \( s \in \R \) is an upper bound for a set \( A \subseteq \R \). Then, \( s = \sup A \) if and only if for every \( \epsilon > 0 \), there exists \( a \in A \) such that \( s - \epsilon < a \).
\end{lem}
\end{tcolorbox}

\begin{proof}
For the forward direction, suppose that \( s = \sup A \) and consider \( s - \epsilon \). Since \( s \) is an upper bound, we have that \( s -\epsilon < s \). This means that \( s - \epsilon \) is not an upper bound. Hence, we can find an element \( a \in A \) such that \( s - \epsilon < a \) because otherwise \( s - \epsilon \) would be an upper bound. This concludes the forward direction.

For the backwards direction, assume \( s \) is an upper bound. We must satisfy part (ii) of defintion 0.2. Let \( \epsilon > 0 \), then \( \epsilon = s - b \). But since any number smaller than \( s \) is not an upper bound, we have that \( s \leq b \) if \( b \) is any other upper bound for \( S \).
Hence, \( s = \sup A \). 
\end{proof}




\subsection{Exercises}





\subsubsection{Exercise 1.3.3} \begin{enumerate} \item[(a)] Let \( A \neq \emptyset \) and bounded below, and define \[ B = \{ b \in \R : b \text{ is a lower bound for   } A  \},\] Show that \( \inf A = \sup B \). \begin{proof} Our goal is to show that both \( \inf A \leq \sup B \) and \( \inf A \geq \sup B \). Since \( B \neq \emptyset\) and bounded above, we have that the \( \sup B \) exists. First we want to show that \( \inf A \leq \sup B \). By definition of \( \sup B \), it is the greatest lower bound of \( B \). Since \( A \neq \emptyset \) and bounded below, we have that the \( \sup B \) is greater than any lower bound of \( A \). Hence, we have that \( \inf A \leq \sup B \). Now we want to show that \( \inf A \geq \sup B \). Suppose for sake of contradiction that \( \inf A < \sup B \). Since \( A \neq \emptyset\) and bounded below, we have that \[ a - \epsilon \geq \inf A \tag{1}\] for some \( a \in A \). Our goal is to show that there exists that some \( a \in A \) is less than \( \sup B \). Hence, choose \( \epsilon = \sup B + a \) such that (1) and \( \inf A < \sup B \) implies that \begin{align*} a - \epsilon &< \sup B \\ a - (\sup B + 3a) &< \sup B \\ a &< \sup B. \end{align*}  But this is a contradiction since every element in \(A  \) has to be bigger than \( B \) i.e \( b > a \) for all \( b \in B \). Hence, it must be the case that \(\inf A \geq \sup B  \). Hence, we have that \( \inf A = \sup B \). \end{proof} \item[(b)] Use (a) to explain why there is no need to assert that greatest lower bounds exist as part of the Axiom of Completeness. \begin{proof}[Solution] There is no need to assert that greatest lower bounds exist as part of the axiom because we can always seperate a set \( A \) that is bounded below into a set \( B \) that just consists of lower bounds from \( A \). Since the infimum is just the greatest lower bound, it is equivalent to taking the supremum of a set of lower bounds. We can do this because every element in \( B \) is bounded above by every element in \( A \) which is permitted by the Axiom of Completeness.  \end{proof}
\end{enumerate}

\subsubsection{Exercise 1.3.4}
Let \( A_1, A_2, A_3,... \) be a collection of nonempty sets, each of which is bounded above. 

\begin{enumerate}
    \item Find a formula for \( \sup (A_1 \cup A_2 )\). Extend this to \( \sup \big( \cup_{k=1 }^{n} A_k \big)\). 
    \begin{proof}[Solution]
        For \( \sup (A_1 \cup A_2 )\), we have \[ \sup (A_1 \cup A_2 ) = \sup\{ A_1, A_2 \}\] and for  
        \( \sup \big( \cup_{k=1 }^{n} A_k \big)\), we have 
        \[ \sup \big( \cup_{k=1 }^{n} A_k\big) = \sup\{ A_k \} \]
        for \( k \in \N \). 
    \end{proof}
    \item Consider \( \sup \big( \cup_{k=1 }^{\infty} A_k\big) \). Does the formula in (a)  extend to the infinite case? 
    \begin{proof}[Solution]
        No, because then \( \cup_{k=1 }^{\infty} A_k  \) would be an unbounded set. 
    \end{proof}
\end{enumerate}



\subsubsection{Exercise 1.3.5}
As in Example 1.3.7, let \( A \subseteq \R \) be nonempty and bounded above, and let \( c \in \R \). This time define the set 
\[ cA = \{ca: a \in A \}. \]
\begin{enumerate}
    \item[(a)] If \( c \geq 0 \), show that \( \sup A (cA ) = c\sup A \).
    \begin{proof}
        Suppose \( c \geq 0 \). Since \( A \neq \emptyset \) and bounded above, we have that \( \sup A \) exists. Denote \( \alpha = \sup A \). By definition, we have that \( \alpha \geq a \) for all \( a \in A \). Multiplying by \( c \geq 0 \), we have that 
        \begin{align*}
            c\alpha &\geq ca \\ 
            c\sup A &\geq ca
        \end{align*}
        for all \( a \in A \). This shows that \( c \sup A \) is an upper bound for \( cA \).
        
        Now we want to show that this upper bound is the least upper bound in \( cA \). Hence, take any upper bound in \( b \in A \) such that \( ca \leq b \). This implies that \( a \leq b/c \). Since \( \alpha = \sup A \) is the least upper bound for \( A \), we have that \( \sup A \leq b/c \) which further implies that \( c \sup A \leq b \) showing that it is the least upper bound in \( cA \). Hence, we have that 
        \( \sup A (cA ) = c\sup A \).
    \end{proof}
    \item[(b)] Postulate a similar type of statement for \( \sup (cA) \) for the cases \( c < 0 \).
    \begin{proof}[Postulate]
        For cases \( c < 0 \), we have \( \sup (cA ) = c\inf A \).
    \end{proof} 
\end{enumerate}




\subsubsection{Exercise 1.3.6}
Suppose that \( A,B \neq \emptyset \) and bounded above. Prove that 
\[ 
\sup(A+B) = \sup A + \sup B     
\]
We prove this proposition using two methods. One deals with direct application of the definition and the other deals with using lemma 1.3.8. 


\begin{proof}
Our goal is to show that 
\[ 
\sup(A+B) = \sup A + \sup B     
\]
We know that since \( A,B \neq \emptyset \) and bounded above, we have that \( \sup A, \sup B \) exists. we denote the supremums by the following 
\begin{align*}
    \sup A &= \alpha, \\ 
    \sup B &= \beta.
\end{align*}
It suffices to show that following 
\begin{align}
    \sup(A+B) \leq \sup A + \sup B     
\end{align}
and 
\begin{align}
    \sup(A+B) \geq \sup A + \sup B
\end{align}
We first show (2) first then we will show (1) next. Suppose we have abirtrary \( x \in A \) and \( y \in B \). Because \( A, B \neq \emptyset \) and bounded above, we know that the set \(A + B  \) is also non-empty and bounded above which means its supremum \( \sup(A + B) \) also exists. Hence, we know that 
\[
    x + y \leq \sup(A+B)    
\] 
Subtracting \( y \in B \) to the other side of this inequality will yield
\[ 
    x \leq \sup(A + B) - y     
\]
But we know that since \( x \in A \) and \( \sup A \geq a \) for all \( a \in A \), we have that 
\[ 
  \sup A \leq \sup(A+B) - y.   
\]
Likewise, we isolate \( y \in B \) to the other side and note that \( b \leq \sup B \) for all \( b \in B \). Then we get the following:
\begin{align*}
    y &\leq \sup(A+B) - \sup A  \\
    \sup B &\leq \sup(A+B) - \sup A  \\
\end{align*}
But this implies that 
\[ 
    \sup A + \sup B \leq \sup(A+B)     
\]
Now we show (1). By lemma 1.3.8, we know that for all \( \epsilon > 0 \), we have that 
\begin{align*}
    \sup A - \frac{\epsilon }{2} &< a \\
    \sup B - \frac{\epsilon }{2} &< b
\end{align*}
for some \( a \in A \) and \( b \in B \). Adding these two together we have that 
\begin{align*}
    \sup A + \sup B - \epsilon < a + b 
\end{align*}
But we also know that \( a \) and \( b \) are bounded above by their respective supremums so 
\begin{align*}
    \sup A + \sup B - \epsilon < a + b \leq \sup A + \sup B
\end{align*}
Setting \( \epsilon = \sup A+\sup B-\sup(A +B) \). Hence, we have that 
\[
    \sup(A+B) \leq \sup A + \sup B.    
\]

Since we have (1) and (2), we see that 
\[ 
    \sup(A+B) = \sup A + \sup B
\]

\end{proof}



\subsubsection{Exercise 1.3.7}

Prove that if \( a \) is an upper bound for \( A \), and \( a \in A \), then \( \sup A = a \).
    

\begin{proof}
    We want to show that \( a \leq \sup A \) and \( a \geq \sup A \). We start with the former. Since \( A \neq \emptyset \) and bounded above, we have that the \( \sup A \) exists. Label this supremum as \( \sup A = \beta \). For every \( \epsilon > 0 \), we have that there exists \( b \in A \) such that \( \sup A - \epsilon \leq b \). Choose \( \epsilon = 2 \sup A  - a - b \) such that 
    \begin{align*}
        \sup A - 2 \sup A + a + b &\leq b \\ 
        -\sup A +  a &\leq  0 \\
       \implies a \leq \sup A    
    \end{align*}
    Now for the latter case, since \( \sup A = \beta \) is the least upper bound of \( A \) and 
    \( a \in A \), it follows immediately that \( a \leq \sup A \) for all \( a \in A\). Hence, \( \sup A = a \)


\end{proof}




\subsubsection{Exercise 1.3.8}

\begin{enumerate}
    
    \item[(a)] If \( \sup A < \sup B \), show that there exists an element \( b \in B \) that is an upper bound for \( A \). 
    \begin{proof}
        Suppose \( \sup A < \sup B \). Since we have \( \sup B \), by lemma 1.3.8 we can say that for every \( \epsilon > 0 \), there exists \( b \in B \) such that 
        \[ \sup B - \epsilon < b \tag{1} \] 
        Choose \( \epsilon = \sup B - \sup A \). We can do this because \( \sup A < \sup B \). Hence, (1) implies 
        \begin{align*} 
            \sup B - \epsilon &< b \\
            \sup B - (\sup B - \sup A ) &< b \\ 
            \sup A &< b.
        \end{align*}  
        By definition, \( \sup A \) is the least upper bound for \(A \). Since \( \sup A \geq a\) for all \( a \in A\), it follows that 
        from (1) that \( a < b \) for all \( a \in A\). Hence, for some \( b \in B \), \( b \) is an upper bound for \( A \).    




    \end{proof}
    
    \item[(b)] Give an example to show that this is not always the case if we only assume \( \sup A \leq \sup B \). 


\end{enumerate}

    \subsubsection{Exercise 1.3.10 (Cut Property)}
    If \( A \) and \( B \) are nonempty, disjoint sets with \( A \cup B = \R \) and \( a < b \) for all \( a \in A \) and \( b \in B \), then there exists \( c \in \R \) such that \( x \leq c \) whenever \( x \in A \) and \( x \geq c \) whenever \( x \in B \). 
   
    \begin{enumerate}
    \item[(a)] Use the Axiom of Completeness to prove the Cut Property.
    \begin{proof}
        Suppose \( A \) and \( B \) are nonempty, disjoint sets with \( A \cup B = \R \) and \( a < b \) for all \( a \in A \) and \( b \in B \). By Axiom of Completeness, \( A \) and \( B \) are bounded above and below respectively. This implies that their supremum and infimums exists. 
        
        Firstly, we want to show that there exists \( c \in \R \) such that \( x \leq c \) whenever \( x \in A \). Since \( a < b \) for all \( a \in A\) and \( b \in B \), every \( b \in B \) is an upper bound for \( A \). Denote \( B \) as the set of upper bounds for \( A \). Hence, there must exist \( c \in B \) such that \( c \) is the \textbf{least upper bound} for \( A \) due to the Axiom of Completeness. Furthermore, note that \( \sup A \in B  \) and not in \( A \) since \( A \cap B = \emptyset \) which means \( \sup A \in \R \). Hence, \( \sup A \leq b \). But \( x \in A \) so \( x \leq \sup A \).  
        
        Now we want to show there exists \( c \in \R \) such that \( x \geq c \). Since every \( a \in A \) is a lower bound for \( B \) and that \( B \neq \emptyset \), there must exist an element in \( A  \) such that it is the \textbf{greatest lower bound } for \( B \). Denote this element as \( c = \inf B \). Hence, \( \inf B \geq a \) for all \( a \in A \). Furthermore, \( \inf B \in A \) and not in \( B \) since \( A \cap B = \emptyset\) so \( \inf B \in \R \) when we union \( A \) and \( B \) together. Since \( x \in B \), we have that \( \inf B \leq x  \). 
        
        Furthermore, \( B \) is nonempty and bounded below and \( A \) is the set of lower bounds for \( B \), we have that \( \inf B = \sup A = c \in \R \). 
        \end{proof}
    
        \item[(b)] Show that the implication goes the other way; that is, assume \( \R \) possesses the Cut Property and let \( E \) be a nonempty set that is bounded above. Prove that \( \sup E \) exists.

        \begin{proof}
            Assume \( \R \) possesses the Cut Property and let \( E \neq \emptyset \) that is bounded above. Suppose we have that \( E \subseteq \R \). Since \( \R \) possesses the cut property, we can find \( c \in \R \) such that \( x \leq c \) if \( x \in E \). Since \( A \cap B = \emptyset \), \( c \in A \cup B = \R \). Hence, either \( c \in A \) or \( c \in B \). If \(c \in A \), then \( c \) is not an upper bound for \( E \) since every \(a \in A  \) is less than every \( b \in B \). Furthermore, if \( c \in A \) and \( A \) is the set of lower bounds for \( B \), then it would contradict that \( c \) is an upper bound for \( E \). Thus, we must have \( c \in B \). Since \( c \in B \), \( B \) is the set of upper bounds for \( E \), and \( E \neq \emptyset \) and bounded above, \( c \in B \) is the smallest element in \( B \) which makes it the \textbf{least upper bound} for \( E \). Hence, \( c = \sup E \) exists.
        \end{proof} 
        

    
\end{enumerate}














% \end{document}





% \begin{document}

\section{Consequences of Completeness}

The first application of the Axiom of Completeness is a result that says that the real line contains no gaps. 

\begin{tcolorbox}
    \begin{thm}
        For each \( n \in \N \), assume we are given a closed interval \( I_n = [a_n, b_n ] = \{ x \in \R : a_n \leq x \leq b_n \} \). Assume also that each \( I_n \) contains \( I_{n+1}\). Then, the resulting nested sequence of closed intervals 
        \[ I_1 \supseteq I_2 \supseteq I_3 \supseteq I_4 \supseteq ... \]
        has a nonempty intersection; that is, \( \cap_{n=1}^{\infty} I_n \neq \emptyset \).
    \end{thm}
\end{tcolorbox}

\begin{proof}
    Our goal is to produce a real number \( x \) such that this element is in every closed interval \( I_n \) for every \( n \in \N \). Using the Axiom of Completeness, we can denote the following sets 
    \begin{align*} A &= \{  a_n : n \in \N  \} \\ 
                   B &= \{  b_n : n \in \N  \}
     \end{align*} 
    where \( A \) and \( B \) consists of the left-hand and right-hand endpoints respectively. Since every closed interval are nested, we know that every \( b_n \) serves as an upper bound for \( A \). By the Axiom of completeness, we can say that a supremum exists for \( A \) and we can label this supremum as \( x = \sup A \). By definition, this is an upper bound for \( A \). Hence, we have that \( a_n \leq x \). But since \( x \) is the least upper bound and every \( b_n \in B \) is an upper bound for every \( a_n \in A \), we have that \( x \leq b_n \). Hence, we have that \( a_n \leq x \leq b_n \) which means that \( x \in I_n \) for all \( n \in \N \). This precisely means that \( \cap_{n=1}^{\infty} I_n \neq \emptyset \).
\end{proof}

\subsection{The Density of the Rationals }

\begin{tcolorbox}
\begin{thm}(Archimedean Property)
    \begin{itemize}
        
        \item Given any number \( x \in \R \), there exists an \( n \in \N \) satisfying an \( n \in \N \) satisfying \( n > x \)
        \item Given any real number \( y > 0 \), there exists an \( n \in \N \) satisfying \( 1/n < y \)

    \end{itemize}
    
\end{thm}
\end{tcolorbox}

Before we head on to the proof, it is important to notice that \( \N \) is not bounded above and we shall not prove this fact since we are taking this property of the set to be a given just like all the properties that are contained in \( \N, \Z,\)  and \( \Q \). 

\begin{proof}
    Assume for sake of contradiction that \( \N \) is bounded above. Using the Axiom of Completeness, \( \N \) contains a supremum, say, \( \sup \N = \alpha \). Using lemma 1.3.8, we know that there exists \( n \in \N \) such that 
    \[  \alpha - 1 < n \tag{\( \epsilon = 1 \)}.\]
    This impplies that 
    \[ \alpha < n + 1 \]
    but this shows that \( n+1 \in \N \) which is a contradiction because we assumed that \( \alpha \geq n \) for all \( n \in \N \) thereby rendering \( \alpha \) to no longer be an upper bound for \( \N \). Hence, we have that 
    there exists an \( n \in \N \) satisfying an \( n \in \N \) satisfying \( n > x \).
    The second part of this theorem follows immediately by setting \( x = 1/y \).
\end{proof}

\begin{tcolorbox}
    \begin{thm}(Density of \( \Q \) in \( \R \))
        For every two \( a,b \in \R \) with \( a < b \), there exists \( r \in \Q \) such that \( a < r < b \).
    \end{thm}
\end{tcolorbox}


\begin{proof}
    Our goal is to choose \( m \in \Z \) and \( n \in \N \) such that 
    \[  a < \frac{ m}{n} < b \tag{1}\]
    The idea is to choose a denominator large enough so that when we increment by size \( \frac{1}{n}\) that it will be too big to increment over the open interval \( (a,b)\). Using the (2) of the Archimedean Property, we choose \( n \in \N \) such that 
    \[ \frac{1}{n} < b - a \tag{2}.\]
    We now need to choose an \( m \in \Z \) such that \( na \) is smaller than this chosen number. A diagram for choosing such a number is helpful. Hence, 

    Judging from our diagram, we can see that 
    \[ m-1 \leq na < m.\]
    Focusing on the left part of the inequality, we can solve (2) for \( a \) and say that 
    \begin{align*}
        m &\leq na + 1  \\ 
            &< n(b - 1/n) + 1 \\ 
            &= nb
    \end{align*} 
    This implies that \( m < nb \) and consequently \( na < m < nb \) which is equivalent to (1). 
    
    
\end{proof}



\section{The Existence of Square Roots}
\begin{tcolorbox}
    \begin{thm}
        There exists \( \alpha \in \R  \) satisfying \( \alpha^2 = 2 \).
    \end{thm}
\end{tcolorbox}

\begin{proof}
    Consider the set 
    \[ T = \{ t \in \R : t^2 < 2 \} \]
    and set \( \alpha = \sup T \). We need to show that \( \alpha^2 = 2\). Hence, we need to show cases where \(  \alpha^2 < 2 \) and \( \alpha^2 > 2 \). The idea behind these cases is to produce a contradiction that will show that having either one of these cases will violate the fact that \( \alpha \)is an upper bound for \( T \) and \( \alpha\) is the least upper bound respectively. 

    Assume the first case, \( \alpha^2 < 2 \). We know that \( \alpha\) is an upper bound for \( T \). We need to construct an element that is larger than \( \alpha \). Hence, we construct 
    \[ \alpha + \frac{1}{n} \in T \tag{1}\]
    Squaring (1) we have that 
    \begin{align*}
        \bigg(\alpha + \frac{1}{n}\bigg)^2 &= \alpha^2 + \frac{2\alpha}{n} + \frac{1}{n^2} \\ 
                &< \alpha^2 + \frac{2\alpha}{n} + \frac{1}{n} \\
                &= \alpha^2 + \frac{2\alpha + 1}{n}.
    \end{align*}

    We can use the fact that \( \Q \) is dense in \( \R \) to choose an \( n_0 \in \N \) such that 
    \[ \frac{1}{n_0} < \frac{ 2 - \alpha^2}{2 \alpha + 1 }.\]
    Rearranging we get that 
    \[ \frac{2 \alpha + 1}{n_0} <  2 - \alpha^2 \] 
    and consequently 
    \[\bigg(\alpha + \frac{1}{n_0}\bigg)^2 < \alpha^2 + (2 - \alpha^2) = 2 \]
    But this means that \( \alpha + 1/n_0 \in T \) showing that \( \alpha \) is not an upper bound for \( T \) contradicting our assumption. 

    Now we want to show the other case that \( \alpha^2 < 2 \) cannot happen. Now we need to produce an element in \( T \) such that it is less than \( \alpha \), thereby showing that \( \alpha \) is not the least upper bound of \( T \). Hence, we construct the following element 
    \[ \bigg(\alpha - \frac{1}{n}\bigg)\in T. \]
    Squaring this quantity will give us the following
    \begin{align*}
        \bigg(\alpha - \frac{1}{n}\bigg)^2 &= \alpha^2 -\frac{2\alpha}{n} + \frac{1}{n^2} \\
        &> \alpha^2 - \frac{2 \alpha}{n}.
    \end{align*}
    Like we did before, we get to choose an \( n_0 \in \N \) such that 
    \[ \frac{1}{n_0} > \frac{\alpha^2 - 2}{2 \alpha} \]
    to make 
    \[\bigg(\alpha - \frac{1}{n_0}\bigg)^2 < \alpha^2 - (\alpha^2 - 2) = 2.\]
    But this shows that \( \alpha - \frac{1}{n_0} < \alpha \) showing that \( \alpha \) and that our constructed element contradicts that fact that \( \alpha\) is the least upper bound. 

\end{proof}


\subsection{Exercises}

\subsubsection{Exercise 1.4.1} Recall that \( \mathbb{I}\) stands for the set of irrational numbers. 

\begin{enumerate}
    \item[(a)] Show that if \( a,b \in \Q \), then \( ab \) and \( a + b \) are elements of \( \Q \) as well. 
        
    
    \begin{proof}
        Suppose \( a,b \in \Q \). Then \( p,q,m,n \in \Z \) such that \( n, q \neq 0 \). Hence, \( a = \frac{p}{q} \) and \( b = \frac{m}{n}\). Adding \( a + b \) will give us 
        \begin{align*}
            a + b &= \frac{p}{q} + \frac{m}{n} \\ 
                  &= \frac{pn + mq}{qn}. 
        \end{align*}
        Since \( pq + mn, qn \in \Z \) with \( q,n \neq 0 \), we have that \( a + b \in \Q \). Now we multiply \( a \) and \( b \) together. Then we have 
        \begin{align*}
            ab &= \frac{p}{q} \cdot \frac{m}{n} \\ 
               &= \frac{pm}{qn}.
        \end{align*}
        Since \( pm, qn \in \Z \) and \( q,n \neq 0 \), we have that \( ab \in \Q \).
    \end{proof}

    \item[(b)] Show that if \( a \in \Q  \) and \( t \in \mathbb{I} \) and \( at \in \mathbb{I} \) as long as \( a \neq 0 \). 
    \begin{proof}
       Suppose for sake of contradiction that \( at = r \) where \( r \in \Q \). Solving for \( t \), we have that \( t = \frac{r}{a} \). But this tells us that \( t \in \Q \) since \( r, a \in \Q \) which is a contradics our assumption that \( t \in \mathbb{I} \).
    
    \end{proof}
     
    \item[(c)] Part (a) can be summarised by saying that \( \Q \) is closed under addition and multiplication. Is \( \mathbb{I} \) closed under addition and multiplication? Given two irrational numbers \( s \) and \( t \), what can we say about \( s + t \) and \( st\)?
        
    
    \begin{proof}[Solution]
        We can say that \( s + t \) is an irrational number while \( st \) can either be rational or irrational depending if \( s = t \) or \( s \neq t \). If \( s = t \), then \( st \) is rational and if \( s \neq t \), then \( st \) is irrational. 
    \end{proof}
    
\end{enumerate}

\subsubsection{Exercise 1.4.2} 
    Let \( A \subseteq \R \) be nonempty and bounded above, and let \( s \in \R \) have the property that for all \( n \in \N \), \( s + \frac{1}{n} \) is an upper bound for \( A \) and \( s - \frac{1}{n} \) is not an upper bound for \( A \). \textbf{Show that } \( s = \sup A \).
        
    
    \begin{proof}
        Since \( A \neq \emptyset \) and bounded above, we have that \( \sup A \) exists. Since \( s + \frac{1}{n} \) for all \( n \in \N \) is an upper bound for \( A \), we have that 
        \[ \sup A \leq s + \frac{1}{n} \tag{1} \]
        for all \( n \in \N \). On the other hand, \( s - \frac{1}{n} \) is a lower bound for \( A \). Hence, 
        \[ \sup A > s - \frac{1}{n} \tag{2} \]
        for all \( n \in \N \). We have \( (1) \) and \( (2) \) imply 
        \[ s - \frac{1}{n} < \sup A \leq s + \frac{1}{n}. \tag{3} \]
        This means that either \( \sup A < s, \sup A > s, \) or \( \sup A = s, \). 
        If \( \sup A < s  \), then \( s - \sup A > 0 \). Using the Archimedean Property, we can find an \( n \in \N \) such that 
        \[ s - \sup A > \frac{1}{n}\]
        but this means that \( \sup  A < s - \frac{ 1}{n}\) which contradicts \( (3) \). On the other hand, if \( \sup A > s \), then \( \sup A - s > 0 \). Using the Archimedean property again, we can find an \( n \in \N \) such that 
        \[ \sup A - s > \frac{1}{n} \]
        but this means that \( \sup A > s + \frac{1}{n} \) which is a contradiction since \( \sup A < s + \frac{1}{n} \) from (3). Hence, it must be that \( \sup A = s \). 
    \end{proof}
    
\subsubsection{Exercise 1.4.3}
    Prove that \( \cap_{n=1}^{\infty} (0,1/n) = \emptyset \). Notice that this demonstrates that the intervals in the Nested Interval Property must be closed for the conclusion for the theorem to hold. 
    \begin{proof}
        Suppose \( x \in (0,\frac{1}{n}) \), then \( x > 0 \). By the Archimedean Property, we can find an \( N \in \N \) that is sufficiently large such that \( x > \frac{1}{N} \). But this means that \( x \in (0, 1/n )\) for all \( n \in \N  \). Hence, \( x \not\in \cap_{n=1}^{\infty} (0,\frac{1}{n})\) and then 
        \[ \cap_{n=1}^{\infty} (0,\frac{1}{n}) = \emptyset.\]
    \end{proof}
    
    \subsubsection{Exercise 1.4.4}
    Let \( a < b \) be real numbers and consider the set \( T = \Q \cap [a,b]\). Show that \( \sup T = b \). 

        \begin{proof}
            Let \( a < b \) where \( a,b \in \R \). Consider the following set \( T = \Q \cap [a,b] \). We want to show that \( \sup T = b \). By definition, \( b \) is an upper bound for \( T \) since \( a < b \). All we need to show is that \( b \) is the least upper bound. Hence, we use lemma 1.3.8 and the fact that \( \Q \) is dense in \( \R  \) to state that for every \( \epsilon > 0 \), there exists \( r \in \Q \) such that \( b - \epsilon < r < b \). But this means that \( r \in T \) and \( b - \epsilon \) is not an upper bound for \( T \). Hence, \( \sup T = b \).


        \end{proof}
    

    Another proof for this: 

        \begin{proof}
            Let \( a < b \) where \( a,b \in \R \). Consider the following set \( T = \Q \cap [a,b] \). We want to show that \( \sup T = b \). By definition, \( b \) is an upper bound for \( T \) since \( a < b \). All we need to show is that \( b \) is the least upper bound. Since \( a < b \) where \( a,b \in \R \), we can find \(x \in \Q \) such that \( a < x < b \). Since \( x = \frac{m}{n}\) where \( m,n \in \Z \) with \( n \neq 0 \), we have that \( na < m < nb\). But note that \( nb \) is another upper bound for \( T \) for \( n \) sufficiently large and \( nb > b \) implying that \( b \) is the least upper bound of \( T \). Hence, \( \sup T = b \).
        \end{proof}
    

    \subsubsection{Exercise 1.4.5}

    Using Exercise 1.4.1, supply a proof for Corollary 1.4.4 by considering the real numbers \( a - \sqrt{2} \) and \( b - \sqrt{2}\). 

        \begin{proof}
            Consider the real numbers \( a - \sqrt{p}\) and \( b - \sqrt{p}\) where \( p \) is any prime number. Using the fact that \( \Q \) is dense in \( \R \), we have that 
            \[ a - \sqrt{p} < r < b - \sqrt{p} \] 
            for some \( r \in \Q \). Adding \( \sqrt{p} \) to both sides, we have that 
            \[ a < r + \sqrt{p} < b .\]
            But know that \( r + \sqrt{p} \in \mathbb{I} \) by (c) of Exercise 1.4.1. Hence, \( t = r + \sqrt{p} \). We can follow the same procedure for trancendental numbers and make this conclusion. 
        \end{proof}
    


    \subsubsection{Exercise 1.4.7}

    Finish the proof of Theorem 1.4.5 by showing that the assumption \( \alpha ^2 > 2 \) leads to a contradiction of the fact that \( \alpha = \sup T \). 
        
        
    
    \begin{proof}
        Now we want to show the other case that \( \alpha^2 < 2 \) cannot happen. Now we need to produce an element in \( T \) such that it is less than \( \alpha \), thereby showing that \( \alpha \) is not the least upper bound of \( T \). Hence, we construct the following element 
        \[ \bigg(\alpha - \frac{1}{n}\bigg)\in T. \]
        Squaring this quantity will give us the following
        \begin{align*}
            \bigg(\alpha - \frac{1}{n}\bigg)^2 &= \alpha^2 -\frac{2\alpha}{n} + \frac{1}{n^2} \\
            &> \alpha^2 - \frac{2 \alpha}{n}.
        \end{align*}
        Like we did before, we get to choose an \( n_0 \in \N \) such that 
        \[ \frac{1}{n_0} > \frac{\alpha^2 - 2}{2 \alpha} \]
        to make 
        \[\bigg(\alpha - \frac{1}{n_0}\bigg)^2 < \alpha^2 - (\alpha^2 - 2) = 2.\]
        But this shows that \( \alpha - \frac{1}{n_0} < \alpha \) showing that \( \alpha \) and that our constructed element contradicts that fact that \( \alpha\) is the least upper bound. 
    \end{proof}


    \subsubsection{Exercise 1.4.6}
    Recall that a set \( B \) is dense in \( \R \) if an element of \( B \) can be found between any two real numbers \( a < b \). Which of the following sets are dense in \( \R \)? Take \( p \in \Z \) and \( q \in \N \) in every case. 
    \begin{enumerate}
        \item[(a)]
        The set \( \{ r \in \Q : q \leq 10  \} \)
        \begin{proof}[Solution]
            Yes, since \( a < \frac{p}{10} < \frac{p}{q} < b \). 
            
        \end{proof}
        
        \item[(b)]
        The set of all rationals \( p/q \) such that \( q \) is a power of 2.
        \begin{proof}
            Yes since \( a < \frac{p}{2^n} < b \) for \( n \in \N \). 
        \end{proof}
        
        \item[(c)] 
        The set of all rationals \( p/q \) with \( 10|p| \geq q \)
            \begin{proof}
                
            \end{proof}
        
    \end{enumerate}












\section{Cardinality}

\subsection{Correspondence}

\begin{definition}
    A function \( f: A \to B \) is \textit{one-to-one} if \( a_1 \neq a_2 \) in \( A \) implies that 
    \( f(a_1) \neq f(a_2) \) in \( B \). The function \( f \) is \textit{onto} if, given any \( b \in B \), there exists an element \( a \in A \) for which \( f(a) = b\).
\end{definition}

An equivalent defintion for a function to be one-to-one is the following:

\begin{definition}
    A function \( f: A \to B \) is \textit{one-to-one} if \( f(a_1) \neq f(a_2) \) implies that \( a_1 = a_2 \).
\end{definition}
    
A function that is both one-to-one and onto is said to be bijective. Meaning that we have a one-to-one correspodence between the sets \( A \) and \( B \). Another way to explain a function being injective is to say that no two elements from \( A \) can map to the same element in \( B \) ( think of the function \( x^2\)). And a function being onto can be explained as every element in \( A \) has to be mapped to an element in \( B \).

From an algebraic perspective, we can denote a function being bijective to mean the same thing as two sets having the same cardinality i.e we can say that 

\begin{definition}
Two sets \( A \) and \( B \) have the same cardinality if there exists \( f: A \to B \) that is both one-to-one and onto. We can denote this symbolically as \( A \sim B\)
\end{definition}

Some examples of bijective maps are
\begin{enumerate}
    \item[(i)] Let the following map \(f: \N \to \mathbf{E} \) be defined as \( f(n) = 2n \). We can see that \( \N \sim \mathbf{E}\). It's true that \( \mathbf{E}\) is indeed a subset of \( \N \), but do not conclude that it is a smaller set than \( \N \) since they have the same cardinality or isomorphic to each other.
    \item[(ii)] We can show this again. This time let us have a map \( f: \N \to \Z \) such that 
        \[ f(n) = \begin{cases}  (n-1)/2 \text{ if } n \text{ is odd.}       \\
                                 -n/2 \text{ if } n \text{ is even.}
                                                \end{cases}\]
We have that \( \N \sim \Z \) indeed.
\end{enumerate}


\subsection{Countable Sets}

\begin{definition}
A set \( A \) is \textit{countable} if \( \N \sim A \). An infinite set that is not countable is called an \textit{uncountable set}.
\end{definition}

\begin{theorem}
Let \( \Q, \R \). Then 
\begin{enumerate}
    \item[(i)] The set \( \Q \) is countable. 
    \item[(ii)] The set \( \R \) is uncountable.
\end{enumerate}


\end{theorem} 


\begin{proof}
\begin{enumerate}
\item Suppose we define \( A_n \) to be split into two sets. When \( n = 1 \), define \( A_n \) to be 
\[ A_1 = \{  0 \}\] and define \( A_n \) when \( n \geq 2 \) as 
\[ A_n =   \Big\{ \pm \frac{p}{q} : \text{ where } p,q \in \N \text{ are in lowest terms with } p + q = n   \Big\}\]
We can observe here that for every \( n \in \N \) we can find every element of \( \Q \) exactly once in the sets we have defined. So we can conclude that our map is onto. Since we designed our sets so that each rational numer appears once and the fact that for \( n =1 \) and \( n \geq 2 \) produces two disjoint sets, we can see that our map is also one-to-one. 
\item We can prove that second statement of theorem by contradiction. Assume for the sake of contradiction that there exists a \textit{one-to-one} and \textit{onto} function where \( f: \N \to \R \). Letting \( x_1 = f(1)\) and \( x_2 = f(2)\) and so on, then we can enumerate each element of \( \R \) i.e 
    \[ \R = \{ x_1, x_2, x_3, ... \}.\]
    Using the Nested Interval Property, we will now produce a real number that is not in this set. Let \( I_n \) be a closed interval which does not contain  \( x_n \) but contains \( x_{n+1}\). Furthermore, \( I_{n+1}\) is contained within \( I_n \). Note that within \( I_n \) there are two sets which are disjoint and \( x_{n+1}\) can be in either one of these sets. Now consider the following intersection \( \cap_{n=1}^{\infty} I_n \). Using our construction that every \(x_n \not\in I_n \), then we can say that
    \begin{align*} \bigcap_{n=1 }^{\infty} I_n = \emptyset. \end{align*}
But this is a contradiction because the nested interval property asserts that this intersection is nonempty meaning that every \( x \in \R \) is contained in the above set. Hence, we cannot emumerate every single element \( x_n  \) of \( \R \). Therefore, \( \R \) is an \textit{uncountable} set.
\end{enumerate}
\end{proof}

This gives us three insights: 

\begin{enumerate}
    \item The smallest type of infinite set is the countable set.
    \item We can create another set by deleting or inserting elements into it. 
    \item Anything smaller than a countable set is either finite or countable. 
\end{enumerate}

We can create \( \R \) by taking the union of \( \Q \) and \( \mathbb{I} \). Since \( \R \) is not countable and \( \Q \) is, this would mean that the set of irrational numbers \( \mathbb{I}\) would be uncountable. This tells us that \( \mathbb{I}\) is a bigger subset of \( \R \) than \( \Q \). 

We can summarize these results in the follow two theorems: 


\begin{theorem}
    If \( A \subseteq B \) and \( B \) is \textit{countable}, then \( A \) is either countable or finite. 
\end{theorem}

\begin{theorem}
\begin{enumerate}
    \item[(i)] If \( A_1, A_2,... A_n\) are each countable sets, then the union of 
        \[ A_1 \cup A_2 \cup ... \cup A_m \] is countable.
    \item[(ii)] If \( A_n \) is a countable set for each \( n \in \N \), the \( \bigcup_{n=1}^{\infty}A_n \) is countable. 
\end{enumerate}
\end{theorem}










\chapter{Sequences and Series}


\section{The Limit of a Sequence}

Understanding infinite series depends on understanding sequences that make up sequences of partial sums.

\begin{tcolorbox}
\begin{defn}
A sequence is a function whose domain is \( \N \).
\end{defn}
\end{tcolorbox}

A way we describe sequences is to assign each \( n \in \N \), use a mapping rule, and then have an output for the \( n \)th term. Mathematically we can describe it as a map \( f: \N \to \R \).

\begin{ex}
    Each of the following are common ways to describe a sequence. 
    \begin{enumerate}
        \item \( (1, \frac{1}{2},  \frac{1}{3}, \frac{1}{4},...  )\)
        \item \( \{  \frac{1+n}{n}  \}_{n=1}^{ \infty} = ( \frac{2}{1}, \frac{3}{2}, \frac{4}{3}, ...)\)
        \item \( (a_n) \), where \( a_n = 2^n \) for each \( n \in \N \),
        \item \( (x_n)\), where \( x_1 = 2 \) and \( x_{n+1} = \frac{x_n + 1 }{2}\).
    \end{enumerate}
\end{ex}
It should not be confused that in some instances, the index \( n \) will start at \( n = 0 \) or \( n = n_0 \) for some other \( n_0 > 1 \). It is important to keep in mind that sequences are just infinite lists of real numbers. The main point of our analysis deals with what happens at the "tail" end of a given sequence. 

\begin{tcolorbox}
\begin{defn}[Convergence of a Sequence]
A sequence \( (a_n) \) \textit{converges} to a real number \( a \) if, for every \( \epsilon > 0 \), there exists an \( N \in \N \) such that whenever \( n \geq N \) it follows that \( |a_n - a | < \epsilon \).
\end{defn}
\end{tcolorbox}
Furthermore, the convergence of a sequence \( (a_n) \) to \( a \) is denoted by 
\[ \lim_{n \to \infty} a_n = a.\]

To understand the last part of this definition, namely, \( |a_n - a| < \epsilon \), we can think of it as a neighborhood where a given value will be located in. 
\begin{tcolorbox}
\begin{defn}
Given \( a \in \R \) and \( \epsilon  > 0 \), the set 
\[ V_{  \epsilon }(a) = \{ x \in \R : |x-a| < \epsilon  \}\]
is called the \textit{\( \epsilon \)-neighborhood of \( a \)}. 
\end{defn}
\end{tcolorbox}
We can think of \( V_{ \epsilon }(a)\) as an interval where 
\[ a - \epsilon < a < a + \epsilon.\]
Another way is to think of it as a ball with radius \( \epsilon > 0\) centered at \( a \). 
we can also think about the convergence of a sequence to a point with the following definition.
\begin{tcolorbox}
\begin{defn}
    A sequence \( (a_n) \) converges to \( a \) if, given any \( \epsilon-\)neighborhood \( V_{ \epsilon } (a)\) of \( a \), there exists a point in the sequence after which all of the terms are in \( V_{ \epsilon } (a) \). In other words, every \( \epsilon - \)neighborhood contains all but a finite number of the terms of \( (a_n) \). 
\end{defn}
\end{tcolorbox}

The main idea here is that for some \( n \in \N \) along a sequence \( (a_n) \), all the points of the sequence converge to some point within a certain \( \epsilon -\)neighborhood. Note that when increase the value of \( n \in \N \), the smaller this \( \epsilon-\)neighborhood has to be and vice versa.

\begin{ex}
Consider the sequence \( (a_n) \), where \( a_n = \frac{1}{ \sqrt{n} }\). From our regular understanding of calculus, one can see that the limit of this sequence goes to zero. 

\begin{proof}
Let \( \epsilon  > 0 \). Choose \( N \in \N \) such that 
\[ N > \frac{1}{e^2}.\]
We now proceed by verifying that this choice \( N \in \N \) has the desired property that \( a_n \to 0 \). Let \( n \geq N \) such that \( n > \frac{1}{ \epsilon^2} \). Hence, we have 
\[ \frac{1}{ \sqrt{n}} < \epsilon. \]
But this implies that \( |a_n - 0| < \epsilon \) and hence our sequence contains the desired property. 
\end{proof}
\end{ex}
The main idea of these convergence proofs is to find an \( N \in \N \) such that the value we want can be "hit" within some range that we specify with any number \( \epsilon > 0   \).


\subsubsection{Quantifiers}

The phrase 

\begin{center}
"For all \( \epsilon> 0 \)", there exists \( N \in \N \) such that ..."
\end{center}

means that for every positive integer I give you, there exists some index or natural number that contains some property that allows the sequence to converge to some value that we desire and as long as we satisfy this rule, then we can say that the sequence converges to our desired value. The template for our subsequent covergence proof will follow the steps below:

\begin{itemize}
    \item "Let \( \epsilon> 0 \)" be arbitrary."
    \item Demonstrate that a specific choice of \( N \in \N \) leads to the desired property. Note that finding this \( N \) often involves working backwards from \( |a_n - a | < \epsilon \). 
    \item Show that this \( N \) actually works.
    \item Now assume \( n \geq N \). 
    \item With this choice of \( \N \), you can work towards the property that \( |a_n - a | < \epsilon \)
\end{itemize}

\begin{ex}
Show 
\[ \lim \Big( \frac{n+1}{n}\Big) = 1.\]
In other words, show that for every \( \epsilon  > 0 \), there exists some \( N \in \N \) such that 
\[ |a_n - 1| < \epsilon \] where 
\[ a_n = \frac{n+1}{n}. \]
To obtain our choice of \( N \in \N \), we must work backwards from our conclusion. Hence, we have 
\begin{align*}
a_n - 1 &< \epsilon  \\
\frac{n+1}{n} - \frac{n}{n} &< \epsilon \\ 
\iff \frac{1}{n} &<  \epsilon \\
\iff \frac{1}{ \epsilon } &< n.
\end{align*}
Hence, our choice of \( N \in \N \) is \( N = 1/ \epsilon \). Now for the actual proof. 

\begin{proof}
Let \( \epsilon  > 0 \) be arbitrary. Choose \( N = 1 / \epsilon  \) such that 
\[ N > \frac{1}{ \epsilon }.\]
Let \( n \geq N \). Then we proceed by showing that this choice of \( N \in \N \) leads to the desired property. 
Hence, 
\begin{align*}
n &> \frac{1}{ \epsilon } \\
\epsilon &> \frac{1}{ n } \\
\epsilon  &> \frac{ n+1 }{n} - \frac{n}{n} \\ 
\epsilon &> \frac{n+1}{n} - 1 \\
\epsilon  &> |a_n - 1|.
\end{align*}
Hence, our choice of \( N \in \N \) leads to \( a_n \to 1 \). We can now conclude that 
\[ \lim_{n \to \infty} a_n = 1.\]
\end{proof}
\end{ex}

\begin{tcolorbox}
    \begin{thm}[Uniqueness of Limits]
The limit of a sequence, when it exists, must be unique.
\end{thm}
\end{tcolorbox}

\begin{proof}
Suppose we have \( (a_n) \subseteq \R \). Suppose \( a_n \to a \) and \( a_n \to a' \). We want to show that 
\[ a = a' .  \]
By definition, we have that 
\begin{align*}
    |a_n - a |&< \epsilon/2   \text{ for some } n_1 \in \N \\
    |a_n - a'| &< \epsilon/2 \text{ for some } n_2 \in \N .
\end{align*}
We can show that \( a = a' \) by showing that \( |a - a'| < \epsilon\). Hence, choose \( N = \min \{ n_1, n_2 \}\) such that 
\begin{align*}
 |a - a'|&< |a - a_n + a_n - a' |  \\
         &< |a - a_n | + |a_n - a'| \\
         &< \epsilon/2 + \epsilon/2 \\
         &= \epsilon.
\end{align*}
Hence, we have that \( a = a' \) showing that our limit is unique. 
\end{proof}



\subsection{Divergence}

We can study the divergence of sequences by negating the definition we have above. 
\begin{ex}
Consider the sequence 
\[ \Big(1, -\frac{1}{2}, \frac{1}{3}, -\frac{1}{4}, \frac{1}{5}, -\frac{1}{5}, \frac{1}{5}...  \Big)\]
We can prove that this sequence does not converge to zero. Why? When we choose an \( \epsilon  = 1/10 \), there is none of the term of the sequence converge within the neighborhood \( (-1/10, 1/10 )\) since the sequence oscillates between \(-1 / 5 \)  and \( 1 / 5\). There is no \( N \in \N \), that satisfies \( a_n \to 0 \). We can also give a counter-example in which we disprove the claim that \( (a_n) \) converges to \( 1 / 5 \). Choose \( \epsilon = 1 / 10 \). This produces the neighborhood \( (1/10, 3/10 ) \). We can see that the sequence does in fact converge to \( 1 / 5 \), but it does so in an oscillating fashion. Furthermore, the sequence does not stay within the neighbor we specified where we expect all the terms of the sequence to converge towards the value. Hence, there is no such \( N \in \N \) where the property can be satisfied.  
\end{ex}


\begin{tcolorbox} 
\begin{defn}
A sequence that does not converge is said to diverge.
\end{defn}
\end{tcolorbox}


\subsection{Exercises}

\subsubsection{Exercise 2.2.1} What happens if we reverse the order of the quantifiers in our convergence definition? 
\begin{tcolorbox}
    \begin{defn}[Reversed]
A sequence \( x_n \) converges to \( x \) if there exists an \( \epsilon > 0 \) such that for all \( N \in \N \) we have for \( n \geq N \) such that 
\[ |x_n  - x| < \epsilon.\]
\end{defn}
\end{tcolorbox}
Give an example of a convergent sequence. Is there an example of a convergent sequence that is divergent? Can a sequence converge to two different values? What exactly is being described in this strange definition.

\begin{enumerate}
    \item [(a)] When we reverse the quantifiers, the definition now requires us to construct such an \( \epsilon  \) such that any choice of \( N \in \N \) will satisfy the property. 
    \item [(b)] An example of a convergent sequence is \( x_n = 1/n \). It can be easily shown that \( x_n \to 0 \).  
    \item [(c)] Based on our definition and the fact that we can choose any \( N \in \N \) suggest that we can have two different values for which the sequence can converge to.
    \item [(d)] There is a specific contruction of an \( \epsilon\) such that all \( x_n \) clusters converges towards a point determined by any choice of \( N \in \N \). 
\end{enumerate}

\subsubsection{Exercise 2.2.2}
Verify, using the definition of convergence of a sequence, that the following sequences converge to the 
proposed limit. 

\begin{enumerate}
    \item \( \lim \frac{2n+1}{5n+4} = \frac{2}{5}\)
            Let \( x_n = \frac{2n+1}{5n+4} \). We want to work backwards from our conclusion 
            \[ |x_n - \frac{2}{5}| < \epsilon \] to find our choice of \( N \in \N \). Hence,
            \begin{align*}
                \Big| \frac{2n+1}{5n+4} - \frac{2}{5} \Big| &< \epsilon  \\
                \frac{3}{5(5n+4)} &< \epsilon.  \\
            \end{align*}
    Solving for \( n \), we get that 
    \[ n > \frac{ 3/ \epsilon - 20}{25}.\]
    This only holds for all \( 0 < \epsilon < 3/20\). Hence, our choice of \( N \in \N \) is 
    \[ N = \frac{ 3/ \epsilon - 20}{25}.\]
        \begin{proof}
        Let \( 0 < \epsilon < 3/20 \). Choose \( N = \frac{ 3/ \epsilon - 20}{25}\) such that 
        \( N > \frac{ 3/ \epsilon - 20}{25}\). Suppose \( n \geq N \). We want to show that 
        \[ \Big| \frac{2n+1}{5n+4} - \frac{2}{5} \Big| < \epsilon.\]
        So we have the folowing manipulations
        \begin{align*} 
        n &> \frac{ 3/ \epsilon - 20}{25} \\
        25n \epsilon &> 3 - 20 \epsilon
        \end{align*}
        so we have 
        \begin{align*}
          \epsilon (25n + 20)&> 3.\\
        \end{align*}
        Hence, we have 
        \[  \epsilon > \frac{3}{25n + 20}\] which satisfies our given property that 
        \[ \lim x_n = 2/5.\]

        \end{proof}
    \item \( \lim \frac{2n^2}{n^3 + 3} = 0 \)
        Let \( x_n = \frac{2n^2}{n^3 + 3}\). We want to produce an \( N \in \N \) from 
        \[ |x_n - 0| < \epsilon. \]
        Observe that 
        \begin{align*}
        \frac{2n^2}{n^3 + 3}&< \epsilon  \\
        \end{align*}
       Notice that it is somewhat difficult to solve for \( n \) so we need to upper bound and lower bound the numerator and the denominator separately. Furthermore, we notice that \( (x_n)\) is bounded by \( \frac{2n^2}{n^3} = \frac{2}{n} \).  Then we lower bound the denominator. Observe that \( n^3 + 3 \geq n^3  \). Hence, we can estimate \( x_n \) to have the following form:
       \[ \frac{2n^2}{n^3 + 3 } \leq \frac{2}{n} < \epsilon \]
       which implies that 
       \[ n > \frac{2}{ \epsilon }\]
       for \( n > 2 \). 

        \begin{proof}
            Let \( \epsilon  > 0 \). Choose \( N = \min \{2,\frac{2}{ \epsilon }\} \) and suppose \( n \geq N \). Then observe that 
            \begin{align*}
               \epsilon &> \frac{2}{n} \geq \frac{2n^2}{n^3+3}.\
            \end{align*}
            Hence, we have 
            \[ \frac{2n^2}{n^3 + 3 } < \epsilon\] and our property is satisfied. 
        \end{proof}
    \item \( \lim \frac{ \sin (n^2)}{ n^{1/3} } = 0 \)

        \begin{proof}
        Let \( \epsilon > 0  \) be arbitrary. Choose \( N = 1 / \epsilon^3 \in \N \) and assume \( n > N \). Then observe that 
        \begin{align*}
            \frac{\sin (n^2)}{n^{1/3}} \leq \frac{1}{n^{1/3}} < \epsilon \\
        \end{align*}
        since \( \sin (n^2) \leq 1 \). Hence, we have that 
        \[ 
            \Big| \frac{\sin(n^2)}{n^{1/3}} - 0 \Big| < \epsilon.
        \] Hence, the property is satisfied.
        \end{proof}
\end{enumerate}

\begin{tcolorbox}
    \begin{defn}[Greatest Integer]
        For all \( x \in \R \), if for all \( k \in \Z \), \( r \in \Z \) where \( k > r \)  such that \( k \leq x < k + 1 \) and \( r \leq x < r+1 \) then we say that \( \max(k,r) \) is the greatest integer less than or equal to \( x \) and denote it as 
        \[ k = [[x ]].\]
\end{defn}
\end{tcolorbox}

\subsubsection{Exercise 2.2.5} 
Let \( [[x]] \) be the greatest integer less than or equal to \( x \). For example, \( [[\pi]] = 3  \) and \( [[3]] = 3 \). Find \( \lim a_n \) and supply proofs for each conlusion if  
\begin{enumerate}
    \item[(a)] \( a_n = [[ 1/n ]]\),
    \begin{proof}
        We claim that the limit of \( a_n = [[ 1 / n ]]\) is equal to zero. We want to show that 
        for all \( \epsilon  > 0 \), there exists an \( n \in \N \) such that for every \( n \geq N \) \[ | a_n - 0 | < \epsilon. \]
        We proceed by choosing \( N > 1 \). Suppose \( n \geq N \). Our goal is to show that following property above. Since for every \( N > 1 \) such that \( a_n = 0 \), we have \( n \geq N \) \[ | a_n - 0 | = | 0 - 0 | = 0 < \epsilon  .\] 
        Hence, our \( N \in \N \) shows that \( \lim a_n =0\).
    \end{proof}
    \item [(b)] \(a_n =  [[ (10+n) / 2n]]\).
        \begin{proof}
        We claim that \( \lim a_n = 0 \). Our goal is to show that for every \( \epsilon  > 0 \), there exists \( N \in \N \) such that for every \( n \geq N \), we have  
        \[ |a_n - 0 | < \epsilon \]
        Choose \( N > 10\). Suppose \( n \geq N \) then we have 
    \[ | a_n - 0 | = |0 - 0| < \epsilon.\]
        Hence, we have \( \lim a_n = 0 \).
        \end{proof}
\end{enumerate}
Reflecting on these examples, comment on the statement following Definition 2.2.3 that "the smaller the \(\epsilon-\)neighborhood, the larger \( N \) may have to be."

\subsubsection{Exercise 2.2.6}
Prove the uniqueness of limits theorem. To get started, assume \( (a_n) \to a \) and \( (a_n) \to b \). Now argue \( a = b \).
\begin{proof}
Suppose \( a_n \to a \) and \( a_n \to b \). Then for every \( \epsilon  > 0 \), there exists \( N_1, N_2 \in \N \) such that for every \( n \geq N_1 \) and \( n \geq N_2\)
\begin{align*}
    | a_n - a |  &< \epsilon /2, \\
    | a_n - b  | &< \epsilon / 2. 
\end{align*}
Choose \( N = \min \{ N_1, N_2 \}\) and assume \( n \geq N \). We want to show that \( a = b \) by showing that 
\[ | a - b  | < \epsilon. \]
Hence, we have 

\begin{align*}
   |a - b | &< | a - a_n + a_n - b | \\
            &< |a - a_n| + |a_n - n | \tag{Triangle Inequality}\\  
            &< \epsilon /2 + \epsilon /2 \tag{ \( a_n \to a \), \( a_n \to b\)} \\ 
            &= \epsilon.
\end{align*}
Therefore, \( | a - b  | < \epsilon  \) and thus \( a = b \).

\end{proof}
\subsubsection{Exercise 2.2.7}

Here are two useful definitions 
\begin{tcolorbox}
\begin{defn}
A sequence \( (a_n)\) is \textit{eventually} in a set \( A \subseteq \R \) if there exists an \( N \in \N \) such that \( a_n \in A \) for all \( n \geq N \).
\end{defn}
\end{tcolorbox}

and

\begin{tcolorbox}
\begin{defn}
A sequence \( (a_n)\) is \textit{frequently} in a set \( A \subseteq \R \) if, for every \( N \in \N \), there exists an \( n \geq N \) such that \( a_n \in A \).
\end{defn}
\end{tcolorbox}

\begin{enumerate}
    \item[(a)] Is the sequence \( (-1)^n \) eventually or frequently in the set \( \{ 1 \}\)? 
        \begin{proof}[Solution]
        The sequence \(  (-1)^n\) is frequently in the set \( \{ 1 \}\) since for every \( n > 0 \), the sequence oscillates between two values in the set \( \{  -1, 1 \}\).
        \end{proof}
    \item[(b)] Which definition is stronger? Does frequently imply eventually or does eventually imply frequently?
        \begin{proof}[Solution]
        The first definition is stronger because it implies that any sequence \( (x_n)\) will eventually converge to a point in some set \( A \subseteq \R \) whereas the second definition explains how a point is constantly being "hit" but not letting all the terms of \( x_n\) settle within \( A \subseteq \R \) past some \( N \in \N \).
        \end{proof}
    \item[(c)] Give an alternate rephrasing of Definition 2.2.3B using either frequently or eventually. Which is the term we want? 
        \begin{proof}[Solution]
        We can rephrase definition 2.2.3B (Convergence of a Sequence: Topological Version) 
by replacing every instance of the word \textit{converge} with the phrase "eventually settling into" and rephrasing the \( \epsilon-\)neighborhood as a set \( A \subseteq \R \) that a sequence  \( x_n \) "eventually settles into to".




        \end{proof}
    \item[(d)] Suppose an infinite number of terms of a sequence \( (x_n)\) are equal to \( 2 \). Is \( (x_n)\) necessarily eventually in the interval \( (1.9, 2.1)\)? Is it frequently in \( (1.9,2.1)\)?
        \begin{proof}[Solution]
        Since \( (x_n) = 2 \) for all \( n \in \N \), \( x_n \) is frequently in the interval \( (1.9,2.1)\).
        \end{proof}
\end{enumerate}



























% !TEX root =  ../../../main.tex 

\section{The Algebraic and Order Limit Theorems}

The goal of having a rigorous definition of convergence in Analysis is to prove statements about sequences in general like the notion of "boundedness" which we will define below.

\begin{tcolorbox}
\begin{defn}
A sequence \( (x_n) \) is \textit{bounded} if there exists a number \( M > 0 \) such that \( | x_n | \leq M \) for all \( n \in \N \).
\end{defn}
\end{tcolorbox}

Geometrically, this means that we can find an interval \( [-M, M]\) that contains every term in the sequence \( (x_n)\).
This naturally leads us to the point that all convergent sequences are bounded i.e 

\begin{tcolorbox}
\begin{thm}
Every convergent sequence is bounded.
\end{thm}
\end{tcolorbox}

\begin{proof}
Assume \( (x_n) \) converges to a limit \( \ell\). This means that given \( \epsilon =  1\), we can find an 
\( N \in \N\) such that for every \( n \geq N \), we can say that 
\begin{align*}
    \implies&| x_n - \ell |  <  1  \\
    \iff &-1 < x_n - \ell < 1 \\
    \iff &\ell  - 1 < x_n < \ell + 1.
\end{align*}
Note the terms of the sequence \( (x_n)\) can be found in the open interval \( (\ell - 1, \ell + 1)\). Since \( \ell \in \R \) can either be positive or negative, we can conclude that  
\[ | x_n | < | \ell | + 1 \]
for all \( n \geq N \) where
\[ M = \max \{ | x_1 |, | x_2 |, ..., | \ell | + 1 \}.\]
Hence, it follows that \( | x_n  | \leq M \) for all \( n \in \N \) as desired.
\end{proof}

\begin{tcolorbox}
    \begin{thm}[Algebraic Limit Theorem]
        Let \( \lim a_n = a \), and \( \lim b_n = b \). Then, 
        \begin{enumerate}
            \item[(i)] \( \lim(ca_n) = ca \) for all \( c \in \R \);
            \item[(ii)] \( \lim (a_n + b_n) = a + b \);
            \item[(iii)] \( \lim (a_nb_n) = ab\);
            \item[(iv)] \( \lim (a_n / b_n) = a / b \) provided that \( a \neq 0\).
        \end{enumerate}
\end{thm}
\end{tcolorbox}

\begin{proof}[Proof of (i)]
We begin by proving part \( (i)\). Suppose \( a_n \to a \). Then for every \( \epsilon  > 0 \), there exists 
\( N \in \N \) such that for every \( n \geq N \), we have 
\[ | a_n  - a  | < \epsilon / | c |. \tag{1}\]
In order to show \( (i)\), we need to show that 
\[ | ca_n - ca | < \epsilon.\]
Hence, observe that 
\begin{align*}
| ca_n - ca |&< | c(a_n - a) | \\
&< | c | | a_n - a  | \\  
&< | c | \frac{ \epsilon  }{ | c |} \\ 
&= \epsilon.   
\end{align*}
If \( c = 0 \), then our sequence \( (ca_n)\) reduces to the sequence \( \{0,0,0,...,0 \}\) which is clearly converging to \( ca = 0 \).
Hence, we have attained our desired property that \( \lim (ca_n) = ca\).
The parts are left to you to prove.
\end{proof}
\begin{proof}[Proof of (ii)]
    To show part \( (ii)\), it suffices to show that for every \( \epsilon> 0 \), there exists \( N \in \N \) such that for every \( n \geq N \), we have 
    \[ | a_n + b_n - (a+b) | < \epsilon.\] Hence, we start with the left side of (ii). Since \( a_n \to a \) and \( b_n \to b\), there exists \( N_1, N_2 \in \N \). We can choose \( N = \max \{ N_1, N_2 \}\) such that for every \( n \geq N \), we can say that 
    \begin{align*}
     | a_n + b_n - (a + b) | &< | (a_n-a) + (b_n - b) |  \\
                             &< | a_n - a  |  + | b_n - b | \\ 
                             &< \frac{ \epsilon }{2} + \frac{ \epsilon }{ 2} \\  
                             &= \epsilon. 
    \end{align*}
Hence, it follows that \( \lim (a_n + b_n) = a + b \) as required.

\end{proof}

\begin{proof}[proof of (iii)]
    To show part \( (iii)\), it suffices to show for every \( \epsilon  > 0 \), there exists \( N \in \N \) such that for every \( n \geq N \), we have 
    \[ | a_nb_n - ab | < \epsilon.\]
Since \( a_n \to a \) and \( b_n \to b\), there exists \( N_1 , N_2 \in \N \). We can choose \( N = \max \{ N_1, N_2  \}\) such that for every \( n \geq N \), we can say that 
\begin{align*}
 | a_nb_n - ab |&< | a_nb_n -a_nb + a_nb - ab |  \\
                &< | a_n (b_n - b ) + b (a_n - a)| \\ 
                &< | a_n (b_n - b) |  + | b (a_n - a) | \\ 
                &< | a_n | | b_n - b  |  + | b | | a_n - a |  \\
                &< M \frac{ \epsilon }{2 M }  + | b | \frac{ \epsilon }{2 | b |} \tag{ \( a_n \) is bounded } \\ 
                &< \epsilon  
\end{align*}
Hence, it follows that \( \lim (a_nb_n) = ab\).
\end{proof}

\begin{proof}[Proof of (iv)]
To show part (iv), it suffices to show for every \( \epsilon  > 0 \), there exists an \( N \in \N\) such that for every \( n \geq N \), we have 
\[ \Big| \frac{a_n}{b_n} - \frac{a}{b} \Big| < \epsilon.\]
Since \( a_n \to a \) and \( b_n \to b\) with \( b \neq 0 \), there exists an \( N_1, N_2 \in \N   \) such that whenever \( n \geq N_1, N_2\), we can have
\begin{align*}
 | a_n - a  |&<  M \epsilon / 2,  \\
 | b_n - b | &<  \frac{ | b |}{ | a |} \cdot \frac{ M \epsilon }{2}.
\end{align*}



we can choose \( N = \max \{ N_1, N_2 \}\) so that 
\begin{align*}
   \Big| \frac{a_n}{b_n} - \frac{a}{b} \Big| &=  \Big| \frac{a_nb - b_n a}{b_nb} \Big|   \\
                                     &=  \Big| \frac{a_nb - b_n a}{b_nb} \Big| \\
                                     &= \Big| \frac{a_nb - ab + ab- b_n a}{b_nb} \Big| \\
                                     &=  \Big| \frac{b(a_n - a) + (b- b_n)a}{b_nb} \Big| \\
                                     &<  \frac{|a_n - a|}{|b_n|} + \frac{ | a |}{ | b |} \cdot \frac{|b_n - b|}{|b_n|} \\
                                     &< \frac{ M \epsilon }{ 2M} + \frac{ | a |}{ | b |} \cdot \frac{ | b | M  \epsilon}{  | a | 2 M} \tag{ \( b_n\) bounded} \\
                                     &= \epsilon. 
\end{align*}
Hence, it follows that \( \lim ( \frac{a_n}{b_n} ) = \frac{a}{b} \) provided that \( b \neq 0\).



\end{proof}

\begin{tcolorbox}
    \begin{thm}[Order Limit Theorem] 
    Assume \( \lim a_n  = a\) and \( \lim b_n =  b\).
    \begin{enumerate}
        \item[(i)] If \( a_n \geq 0 \) for all \( n \in \N \), then \( a \geq 0\).
        \item[(ii)] If \( a_n \leq b_n\) for all \( n \in \N \), then \( a \leq b \).
        \item[(iv)] If there exists \( c \in \R \) for which \( c \leq b_n\), for all \( n \in \N \), then 
        \( c \leq b \). Similarly, if \( a_n \leq c \) for all \( n \in \N \), then \( a \leq c\).
    \end{enumerate}
    \end{thm}
\end{tcolorbox}

\begin{enumerate}


    \item[(i)] \begin{proof}
We proceed by contradiction by assuming that \( a < 0 \). Suppose \( a_n \geq 0 \) and \( a_n \to a \). Let \( \epsilon  = | a |\) and suppose \( n \geq N \). Then
\[ | a_n - a | < | a | = -a.\]
But this means that \( a_N < 0\) which is a contradiction since \( a_N \geq 0\).
\end{proof}
    \item[(ii)]
        \begin{proof}
        We can ensure that the sequence \( b_n - a_n\) converges to \( b - a\) by the Algebraic Limit Theorem. Since \( b_n - a_n \geq 0\), we can use (i) to write \( b - a \geq 0\). Hence, \( a \leq b\).
        \end{proof}
    \item[(iii)]
        \begin{proof}
            Suppose there exists \( c \in \R  \) for which \( c \leq b_n\) for all \( n \in \N \). Suppose \( a_n  = c \) then using (ii) yields \( c \leq b\). Suppose \( a_n \leq c\) for all \( n \in \N \) then setting \( b_n = c \) and using (ii) again yields \( a \leq c\).
        \end{proof}
\end{enumerate}


\subsection{Exercises}


\subsubsection{Exercise 2.3.1} Let \( x_n \geq 0\) for all \( n \in \N \).

\begin{enumerate}
    \item[(a)] If \( (x_n ) \to 0 \), show that \( \sqrt{x_n} \to 0\).
        \begin{proof}
            Suppose \( x_n \geq 0 \) and \( x_n \to 0\). In order to show that \( \sqrt{x_n} \to 0\), it suffices to show that for every \( \epsilon> 0\), there exists an \( N \in \N \) such that for every \( n \geq N \) we have 
            \[ | \sqrt{x_n} - 0  | < \epsilon.\]
            Choose \( N \in \N\). Suppose \( x_n = 0 \) for all \( n \in \N \), then \( ( \sqrt{x_n}) = 0 \) for all \( n \geq N \) which means that \( ( \sqrt{x_n}) \to 0\). Suppose \( x_n > 0 \) for all \( n \in \N\), then observe that since \( (x_n) \to 0\)  and \( (x_n)\) bounded, we have 
            \begin{align*}
             | \sqrt{x_n} - 0 | &= | \sqrt{x_n} | \\
                            &= \Big| \frac{x_n}{ \sqrt{x_n}} \Big| \\
                            &=  \Big| \frac{x_n - 0}{ \sqrt{x_n}} \Big| \\ 
                            &=  \frac{|x_n - 0|}{ \sqrt{x_n}}  \\
                            &< \sqrt{M} \frac{ \epsilon }{ \sqrt{M} }  \\ 
                            &= \epsilon 
            \end{align*}
            Hence, it follows that \( ( \sqrt{x_n}) \to 0\).
        \end{proof}
    \item[(b)] If \( (x_n) \to x\), show that \( (\sqrt{x_n}) \to \sqrt{x}\).
        \begin{proof}
            Suppose that \( x_n \geq 0\) for all \( n \in \N \). Suppose \( (x_n) \to x \). We want to show that \( ( \sqrt{x_n}) \to x\). Suppose \( x_n = 0 \) and suppose \( N \in \N \) such that for every \( n \geq N \), then we have the first case above where \( x=0\) and \( ( \sqrt{x_n}) \to 0\). Now suppose \( x_n > 0 \) and choose \( N \in \N \) such that for every \( n \geq N \), then observe that since \( (x_n ) \to x \) and \( (x_n)\) is bounded by an integer \( M > 0 \), we have that 

        \begin{align*}
        |  \sqrt{x_n } - \sqrt{x}  | &= \Big| \frac{ x_n - x}{ \sqrt{x_n} + \sqrt{x}} \Big| \\
                          &= \frac{ | x_n - x |}{ |\sqrt{x_n} + \sqrt{x} | } \\
                          &< ( \sqrt{M} + \sqrt{x}) \frac{ \epsilon }{ (\sqrt{M} + \sqrt{x})} \\ 
                          &= \epsilon. 
        \end{align*}
        Hence, it follows that \( ( \sqrt{x_n}) \to \sqrt{x}\).
        \end{proof}
\end{enumerate}

\subsubsection{Exercise 2.3.2}
Using only Definition 2.2.3, prove that if \( (x_n) \to 2\), then 
\begin{enumerate}
    \item[(a)] \(  \Big(\frac{2x_n - 1}{ 3} \Big) \to 1\);
        \begin{proof}
            Suppose \( (x_n) \to 2\). Our goal is to show that property above. It suffices to show that for every \( \epsilon  > 0\), there exists \( N \in \N  \) such that for every  \( n \geq N  \), we have 
            \[ \Big| \frac{2x_n - 1}{3} - 1 \Big| < \epsilon. \]
        Choose \( N \in \N \) and suppose \( n \geq N \)
        \begin{align*}
        \Big| \frac{2x_n  - 1}{3} - 1 \Big| &= \Big| \frac{2x_n - 4}{3}  \Big| \\
                                    &= \Big| \frac{2}{3} (x_n - 2)\Big| \\
                                    &= \Big| \frac{2}{3} \Big| | x_n - 2 | \\ 
                                    &< \frac{2}{3} \cdot \frac{3 \epsilon }{2} \\
                                    &= \epsilon.
        \end{align*}
        Hence, it follows that 
        \[ \Big(  \frac{2x_n  - 1}{3}\Big) \to 1.\]
        \end{proof}
    \item[(b)] \( \Big(   \frac{1}{x_n}\Big) \to \frac{1}{2}\).
        \begin{proof}
        We want to show that for every \( \epsilon  > 0 \), there exists \( N \in \N \) such that for every \( n \geq N \), we have 
        \[ \Big| \frac{1}{x_n} - \frac{1}{2}\Big| < \epsilon.\]
        Choose \( N \in \N \) and assume \( n \geq N \). Since \( (x_n) \to 2\), we can write 
        \begin{align*}
        \Big| \frac{1}{x_n} - \frac{1}{2} \Big| &= \Big| \frac{2 - x_n}{2x_n} \Big| \\
                                                &= \frac{ | x_n - 2 |}{ 2|x_n |}. \tag{1} \\
        \end{align*}
        Since \( (x_n) \to 2\), we can set \( \epsilon  = 1\) so that we can lower bound the denominator of (1) using 
        \[ 2 - \epsilon < | x_n | \implies 1 < | x_n |. \]
        Then we can set \( N = \max \{ 1, \epsilon / 2 \}\) so that 
        \[ \frac{ | x_n - 2  |}{ 2| x_n |} < \frac{2 \epsilon }{2} = \epsilon \]
        which satisifes our desired property.
        \end{proof}

\end{enumerate}

\subsubsection{Exercise 2.3.3}
Show 
\begin{tcolorbox}
    \begin{thm}[Squeeze Theorem]
        If \( x_n \leq y_n \leq z_n\) for all \( n \in \N \), and if \( \lim x_n = \lim z_n = \ell\), then \( \lim y_n = \ell\).
    \end{thm}
\end{tcolorbox}
\begin{proof}
    Suppose \( x_n \leq y_n \leq z_n\) for all \( n \in \N \) and suppose \( \lim x_n = \lim z_n = \ell\). We want to show that \( \lim y_n = \ell\). By the Order Limit Theorem, we have \( x_n \leq y_n \leq z_n \) for all \( n \in \N \) implies that \( \ell \leq y_n \leq \ell\) for all \( n \in \N \). But this means that \( y_n = \ell\) for all \( n \in \N \). Hence, for every \( \epsilon  > 0\), there exists an \( N \in \N\) such that for every \( n \geq N\) 
    \[ | y_n - \ell | = | \ell - \ell | = 0 < \epsilon.\]
    Hence, it follows that \( \lim y_n = \ell\).
\end{proof}

\subsubsection{Exercise 2.3.4}
Let \( (a_n) \to 0\), and use the Algebraic Limit Theorem to compute each of the following limits (assuming the fractions are always defined).
\begin{enumerate}
    \item[(a)] \( \lim \Big( \frac{1 + 2a_n}{ 1 + 3a_n - 4a_n^2}   \Big) \)
        \begin{proof}[Solution]
        Let \( (a_n) \to 0\). Then
        \begin{align*}
            \lim \Big( \frac{ 1 + 2a_n}{ 1 + 3a_n - 4a_n^2 } \Big) &= \frac{ \lim (1 + 2a_n)}{ \lim (1 + 3a_n - 4a_n^2)} \\
                                 &= \frac{ \lim 1 + \lim (2a_n)}{ \lim 1 + \lim (3a_n) - \lim (4a_n^2)} \\
                                 &= \frac{ 1 + 2 \cdot 0}{ 1 + 3 \cdot 0 + 4 \cdot 0^2} \\
                                 &= 1.
        \end{align*}
        \end{proof}
    \item[(b)] \( \lim \Big( \frac{ (a_n + 2)^2 - 4}{a_n} \Big)\)
        \begin{proof}[Solution]
        Let \( (a_n) \to 0\). Then 
        \begin{align*}
          \lim  \Big( \frac{ (a_n + 2)^2 - 4}{ a_n} \Big)&= \lim \Big( \frac{a_n^2 + 4a_n}{a_n} \Big) \\
                                                 &= \lim \Big( a_n + 4 \Big) \\  
                                                 &= \lim a_n + \lim 4 \\ 
                                                 &= 0 + 4 \\ 
                                                 &= 4.
        \end{align*}
        \end{proof}
    \item[(c)] \( \lim \Big( \frac{ \frac{2}{a_n} + 3}{ \frac{1}{a_n} + 5} \Big)\).
        \begin{proof}[Solution]
        Let \( (a_n) \to 0\). Then 
        \begin{align*}
        \lim \Big( \frac{ \frac{2}{a_n} + 3}{ \frac{1}{a_n} + 5} \Big) &= \lim \Big( \frac{2 + 3a_n}{1 + 5a_n} \Big)  \\
                                                               &= \frac{ \lim 2 + \lim (3a_n)}{ \lim 1 + \lim (5a_n)} \\ 
                                                               &= \frac{2 + 3 \cdot 0}{ 1 + 5 \cdot 0}\\
                                                               &=2. 
        \end{align*}
        \end{proof}
\end{enumerate}

\subsubsection{Exercise 2.3.5}
Let \( (x_n)\) and \( (y_n)\) be given, and define \( (z_n)\) to be the "shuffled" sequence \[ (x_1, y_1, x_2, y_2 ,...,x_n, y_n).\] 
For the forwards direction, assume \( (z_n) \) is a convergent sequence. We want to show that \( \lim x_n = \lim y_n \). It suffices to show that given any \( \epsilon > 0\), there exists an \( N \in \N \) such that for any \( n \geq N \), we have 
\[ | x_n - y_n | < \epsilon.\]
Suppose \( (x_n) \to x\) and \( (y_n) \to y\), then we can write 
\begin{align*}
   | x_n - y_n  | &= | x_n - z_n + z_n - y_n |  \\
                  &< | x_n - z_n | + | z_n - y_n | \\  
                  &= | x_n - z + z - z_n | + | z_n - z + z - y_n | \\
                  &< | x_n - x | + | x - z_n | + | z_n - y | + | y - y_n |. \tag{1} 
\end{align*}
By definition, \( (z_n)\) is a shuffled sequence and convergent. Hence, \( z_n \to x \) and \( z_n \to y\). But by the uniqueness of limits, \( x = y\) so we have that 
\[ | x_n - y_n | < \frac{ \epsilon }{4} + \frac{ \epsilon }{4} +  \frac{ \epsilon }{4} + \frac{ \epsilon }{4} = \epsilon.\]
which means \( \lim (x_n - y_n) = \lim x_n - \lim y_n = 0\). 

Now for the backwards direction, assume \( \lim x_n = \lim y_n\). We want to show \( (z_n)\) converges i.e for every \( \epsilon> 0\), there exists \( N \in \N \) such that for every \( n \geq N \), we have 
\[ | z_n - z | < \epsilon.\]

\subsubsection{Exercise 2.3.6} 
Consider the sequence given by \( b_n = n - \sqrt{n^2 + 2n}\). Taking \( (1 / n ) \to 0\) as given, and using both the Algebraic Limit Theorem and the result in Exercise 2.3.1, show \( \lim b_n \) exists and find the value of the limit.

\begin{proof}
Consider the sequence given by \( b_n = n - \sqrt{n^2 + 2n}\). Assume \( (1 / n ) \to 0\) and \( \sqrt{x_n} \to \sqrt{x} \). Then taking the limit of \( b_n\), we have 
\begin{align*}
    \lim b_n &= \lim (n - \sqrt{n^2 + 2n}) \\
             &= \lim   \frac{-2n}{n + \sqrt{n^2 + 2n}} \\  
             &= \lim \frac{ -2}{ 1 + \sqrt{1 + 2 / n}} \\
             &= \frac{ \lim (-2) }{ \lim (1 + \sqrt{1 + 2 / n})} \\
             &= \frac{\lim(-2)}{\lim (1) + \lim (\sqrt{1 + 2 / n})} \\
             &= \frac{-2}{1 + 1 + 0} \tag{\((1/n) \to 0\), \( (\sqrt{x_n}) \to \sqrt{x} \) } \\
             &= -1.
\end{align*}
Hence, we have \( \lim b_n = -1 \). Now we can show that \( b_n\) does reach this limit. 

Let \( \epsilon > 0\). Then choose 
\[ N = \frac{2}{ \sqrt{ \frac{1 + \epsilon }{1 - \epsilon }} - 1}.\]
Then assumme \( n \geq N \). Our goal is to show that 
\[ | b_n + 1 | < \epsilon.\]
Then 
\begin{align*}
    &n >  \frac{2}{ \sqrt{ \frac{1 + \epsilon }{1 - \epsilon }} - 1} \\
    \implies &\sqrt{ \frac{1 + \epsilon }{1 - \epsilon }} -1  > \frac{2}{n}  \\  
\end{align*} Then we have 
\begin{align*}
  \sqrt{1+2/n}  &< \frac{1 + \epsilon }{1 - \epsilon } \\
  (1 - \epsilon )\sqrt{1 - 2/n} &< 1 + \epsilon \\   
  (1 - \epsilon )\sqrt{1 - 2/n} - 1 &< \epsilon .
\end{align*}
Then we get 
\begin{align*}
    - 1 + \sqrt{1 + 2 / n} < \epsilon ( 1 + \sqrt{ 1 + 2 / n})     \\
\end{align*}
and then 
\begin{align*}
    \frac{ - 1 + \sqrt{1 + 2/n}}{ 1 + \sqrt{ 1 + 2/n}} &< \epsilon \\ 
    \frac{ -2n}{ n + \sqrt{ n^2 + 2n}} + \frac{n + \sqrt{n^2 + 2n}}{ n + \sqrt{ n^2 + 2n}} &< \epsilon \\ 
    n - \sqrt{n^2 + 2n} + 1 &< \epsilon.
\end{align*}
Hence, it follows that \( | b_n + 1  | < \epsilon\).

\end{proof}

\subsubsection{Exercise 2.3.8}
Let \( (x_n) \to x \) and let \( p(x)\) be a polynomial.
\begin{enumerate}
    \item[(a)] Show \( p(x_n) \to p(x)\).
        \begin{proof}
            Let \( (x_n) \to x \) and let \( p(x)\) be a polynomial. Let 
            \[ p(x) = \sum_{i=0}^{m} a_i x^i\]
            and 
            \[ p(x_n) = \sum_{i=0}^{m} a_i x_n^{i}. \]
            Our goal is to show that for every \( \epsilon > 0\), there exists \( N \in \N \) such that for every \( n \geq N \)            we have 
            \[ | p(x_n) - p(x)| < \epsilon.\]
           Then by part (i) of the Algebraic Limit Theorem, we have
           \begin{align*}
            | p(x_n) - p(x) |  &= \Big| \sum_{i=0}^{m} a_i x_n^{i} - \sum_{i=0}^{ m} a_i x^{i}  \Big| \\
                               &= \Big| \sum_{i=0}^{m} a_i (x^i_n - x^i)  \Big| \\
                               &<  \sum_{ i = 0}^{ m} | a_ix^i_n - a_ix^i | \tag{T.I}\\     
                               &<  \sum_{i=0}^{m} \frac{ \epsilon }{ m}  \tag{\(x_n \to x\)} \\
                               &= \frac{ \epsilon }{m }  \cdot m \\ 
                               &= \epsilon. 
           \end{align*}
           Hence, we have \( p(x_n) \to p(x)\).

        \end{proof}
    \item[(b)] Find an example of a function \( f(x)\) and a convergent sequence \( (x_n) \to x\) where the sequence \( f(x_n)\) converges, but not to \( f(x)\).
        
\end{enumerate}

\subsubsection{Exercise 2.3.9}
\begin{enumerate}
    \item[(a)] Let \( (a_n)\) be a bounded (not necessarily convergent) sequence, and assume \( \lim b_n  = 0\). Show that \( \lim (a_n b_n) = 0\). Why are we not allowed to use the Algebraic Limit Theorem to prove this? 
        \begin{proof}
        Let \( (a_n)\) be a bounded but not necessarily convergent sequence, and assume \( \lim b_n = 0 \). We want to show that \( \lim (a_n b_n ) = 0\). It suffices to show that for every \( \epsilon  > 0\), there exists \( N \in \N\) such that for every \( n \geq N \), we have 
        \[ | a_nb_n - 0 | < \epsilon. \tag{1}\]
        Since \( (a_n)\) bounded, there exists an \( M > 0\) such that \( | a_n  | < M \). Starting with the left side of (1), choose \( N \in \N \) such that for every \( n \geq N \) 
        \begin{align*}
            | a_n b_n - 0 |&= | a_n | | b_n |\\
                           &< M \cdot \frac{ \epsilon }{M} \tag{\(b_n \to 0\)} \\ 
                           &= \epsilon. 
        \end{align*}
        Hence, it follows that \( \lim (a_n b_n) \to 0\). We cannot use the Algebraic Limit Theorem here because \( (a_n)\) does not necessarily have a defined limit even though it is bounded.
        \end{proof}
    \item[(b)] Can we conclude anything about the convergence of \( (a_nb_n)\) if we assume that \( (b_n)\) converges to some nonzero limit \( b \)?
        \begin{proof}[Solution]
        It would simply not converge.
        \end{proof}
    \item[(c)] Use (a) to prove Theorem 2.3.3, part (iii), for the case when \( a = 0\). 
        \begin{proof}
            Suppose \( a_n \to a\) where \( a = 0\) and \( b_n \to b\). Our goal is to show that \( \lim(a_n b_n) = 0\). Let \( \epsilon > 0 \), then choose \( N \in \N \) such that for every \( n \geq N \),  
        \begin{align*}
         | a_nb_n - 0| &< | a_n | | b_n |   \\
                       &< \frac{ \epsilon }{ M } \cdot M  \tag{\( a_n \to 0, b_n \to b\)} \\ 
                       &< \epsilon   
        \end{align*}
        Hence, it follows that \( \lim (a_nb_n) = 0\).
        \end{proof}
\end{enumerate}

\subsubsection{Exercise 2.3.10} Consider the following list of conjectures. Provide a short proof that are true and a counterexample for any that are false.
\begin{enumerate}
    \item[(a)] If \( \lim (a_n - b_n) = 0 \), then \( \lim a_n = \lim b_n\). 
        \begin{proof}[Counterexample]
        Suppose \( a_n = \frac{n}{2n+1}\) and \( b_n = \frac{n}{2n+5}\). We have \( \lim a_n = \lim b_n \) but \( \lim (a_n - b_n ) \neq 0\).
        \end{proof}
    \item[(b)] If \( (b_n) \to b\), then \( | b_n | \to | b |\).
        \begin{proof}
        Let \( \epsilon > 0\). Consider \( | |b_n| - |b|| \). Assume \( n \geq N \) then 
        \begin{align*}
           | |b_n| - |b|| &< | b_n - b| < \epsilon   \
        \end{align*}
        by reverse triangle inequality and \( (b_n) \to b\).
        \end{proof}
    \item[(c)] If \( (a_n) \to a\) and \( (b_n - a_n) \to 0\), then \( (b_n) \to a\).
        \begin{proof}
        Assume \( (a_n) \to a\) and \( (b_n - a_n) \to 0\). Let \( \epsilon  > 0\). By assumption, 
        \begin{align*}
            | a_n - a | &< \frac{ \epsilon }{2}  \text{,   \space  }  n \geq N_1\\
            | b_n - a_n | &< \frac{ \epsilon }{2} \text{, \space } n \geq N_2.
        \end{align*}
        Hence, choose \(  N = \max \{ N_1, N_2 \} \) such that 
        \begin{align*}
        | b_n - a |&= | b_n - a_n + a_n - a  | \\
                   &< | b_n - a_n | + | a_n - a | \\
                   &< \frac{ \epsilon }{2} + \frac{ \epsilon }{2} \\ 
                   &= \epsilon. 
        \end{align*}
        Hence, \( (b_n) \to a\).
        \end{proof}
    \item[(d)]If \( (a_n) \to a\) and \( | b_n - b | \leq a_n \) for all \( n \in \N \), then \( (b_n) \to b\).
        \begin{proof}
        Let \( \epsilon > 0\). Choose \( N \) so that \( a_n \to 0 \). Then consider \( |b_n - b |\) and observe that
        \begin{align*}
        | b_n - b| &\leq a_n < \epsilon, 
        \end{align*}
        Hence, it follows that \( (b_n ) \to b\).
        \end{proof}
\end{enumerate}

\subsubsection{Exercise 2.3.13(Iterated Limits).} Given a doubly indexed array \( a_{mn}\) where \( m, n \in \N \), what should \(\lim_{m,n \to \infty} a_{mn}\) represent?  

\begin{enumerate}
    \item[(a)] Let \( a_{mn} = m / (m+n)\)  and compute the \textit{iterated} Limits
        \[ \lim_{n \to \infty } \Big( \lim_{m \to \infty} a_{mn} \Big) \text{ and } \lim_{m \to \infty } \Big( \lim_{n \to \infty} a_{mn} \Big).\]
        Define \(\lim_{m,n \to \infty} a_{mn} = a \) to mean that for all \( \epsilon > 0 \) there exists an \( N \in \N \) such that if both \( m,n \geq N \), then \( |a_{mn} - a | < \epsilon \).
        \begin{proof}
        Let \( a_{mn} = m / (m+n)\). We can compute the \textit{iterated} limits 
        \[  \lim_{n \to \infty} \Big( \lim_{m \to \infty} \frac{m}{m+n}\Big) \tag{1}  \]
        and
        \[ \lim_{m \to \infty} \Big( \lim_{n \to \infty} \frac{m}{m+n}\Big). \tag{2}\]
        We start with (1). Hence, we have 
        \begin{align*}
            \lim_{n \to \infty } \Big( \lim_{ m \to \infty} \frac{m}{m+n}\Big)  &= \lim_{ n \to \infty} (1)  \\
                                                                                &= 1.
        \end{align*}
        With (2), we have 
        \begin{align*}
            \lim_{m \to \infty} \Big( \lim_{ n \to \infty} \frac{m}{m+n}\Big)&= \lim_{m \to \infty} (0) \\
                                                                             &= 0. 
        \end{align*}

        \end{proof}
    \item[(b)] Let \( a_{mn} = 1/(m+n)\). Does \( \lim_{m,n \to \infty} a_{mn} \) exist in this case? Do the two iterated limits exists? How do these three values compare? Answer these same questions for \( a_{mn} = mn / (m^2 + n^2 )\). 
        \begin{proof}
            Let us try and compute the limit of \( a_{mn} \) the same way we did above. Hence, we have 
            \[
                \lim_{n \to \infty} \Big( \lim_{m \to \infty} \frac{1}{m+n} \Big)= \lim_{n \to \infty} \Big( \lim_{ m \to \infty} \frac{1}{1 + n/m}\Big)= \lim_{ n \to \infty} (1) = 1 \tag{1}\\
            \]
            and 
            \[ \lim_{m \to \infty} \Big( \lim_{n \to \infty} \frac{1}{m+n} \Big) = 1. \tag{2}\]
            We claim that for \( \lim_{m,n \to  \infty} a_{mn}\) to exists, we have to have 
            \[ \lim_{n \to \infty} \Big( \lim_{m \to \infty} a_{mn}\Big) = \lim_{m \to \infty} \Big( \lim_{n \to \infty} a_{mn}\Big) \]
            hold. Since (1) and (2) are the equal to each other, we claim that \( \lim_{m,n \to  \infty} a_{mn}\) exists. 

            We can try to do the same process for the sequence \( a_{mn} = mn / (m^2 + n^2 )\). Hence, we have 
            \[ \lim_{n \to \infty} \Big( \lim_{m \to \infty} \frac{mn}{m^2 + n^2 } \Big) = \lim_{n \to \infty} (0) = 0 \tag{3} \]
            and 
            \[ \lim_{m \to \infty} \Big( \lim_{ n \to \infty} \frac{mn}{m^2 + n^2 } \Big) = \lim_{m \to \infty} (0)  = 0 \tag{4}\]
            Since (2) and (3) are equal, we have that \( \lim_{m,n \to  \infty} a_{mn}\) exists.  
        \end{proof}
    \item[(c)] Produce an example where \( \lim_{m,n \to \infty} a_{mn}\) exists but neither iterated limit can be computed. 
        \begin{proof}[Solution]
        Consider 
        \[ a_{m,n} = \frac{(-1)^n}{m} + \frac{(-1)^m}{n}\]
        where the iterated limits cannot be computed.
        \end{proof}
    \item Assume \( \lim_{m,n \to \infty} a_{mn} = a \), and assume that for each fixed \( m \in \N \), \( \lim_{n \to \infty} (a_{mn}) \to b_m \). Show \( \lim_{m \to \infty} b_m = a .\)
        \begin{proof}
            Suppose \( \lim_{m,n \to \infty} a_{mn} = a \), and assume that for each fixed \( m \in \N \), \( \lim_{n \to \infty } (a_{mn}) \to b_m \). We want to show that \( \lim_{m \to \infty } b_m = a \). Consider \( | b_m - a | \). Then fix \( m \in \N \) such that for any \( m,n \geq N \), we have that  
            \begin{align*}
                |b_m - a |&= |b_m - a_{mn} + a_{mn} - a | \\
                          &\leq |b_m - a_{mn}| + |a_{mn} - a | \\
                          &< \frac{\epsilon}{2} + \frac{\epsilon}{2} \\
                          &= \epsilon.
            \end{align*}
        Hence, \( (b_{m}) \to a \). 
        \end{proof}
\end{enumerate}
 









\section{The Monotone Convergence Theorem}

As we have seen in the last section, convergent sequences are bounded while the converse is not true. But if a sequence is monotone then surely it is convergent. 

\begin{tcolorbox}
\begin{defn}
    A sequence \( (a_n) \) is \textit{increasing } if \( a_n \leq a_{n+1}\) for all \( n \in \N \) and \textit{decreasing} if \( a_n \geq a_{n+1}\) for all \( n \in \N \). A sequence is \textit{monotone} if it is either increasing or decreasing.
\end{defn}
\end{tcolorbox}

\begin{tcolorbox}
    \begin{thm}[Monotone Convergence Theorem]
        If a sequence is monotone and bounded, then it converges.
    \end{thm}
\end{tcolorbox}

\begin{proof}
Let \( (a_n)\) be \textit{monotone} and \textit{bounded}. We need to show that \( (a_n)\) converges to some value \( s \). Let our set of points \( a_n\) be defined as 
\[ A = \{ a_n : \text{ for all } n \in \N  \}\] 
and because we have a bounded sequence, we must have an upper bound \( s \) which can be defined as out supremum i.e
\[ s = \sup \{ a_n : \text{ for all } n \in \N  \}.\]
Let \( \epsilon > 0 \). We need to show that 
\[ |a_n - s| < \epsilon \] 
Since \( s - \epsilon  \) is not an upper bound of \( A \), there exists \( N \in \N \) such that 
\[ s - \epsilon < a_N.\]
Let's assume that \( (a_n)\) is an increasing sequence. By assuming \(n \geq N \), we can say that \( a_n \geq a_N\). Since \( s + \epsilon \) is an upper bound and \( s \) is the least upper bound, then we can say that  
\[ s - \epsilon < a_N \leq a_n < s \leq s + \epsilon \]
which imply that 
\begin{align*}
    &s - \epsilon  <  a_n < s + \epsilon  \\
    &\implies |a_n - s| < \epsilon.  
\end{align*}
Hence, it follows that any \textit{monotone} and \textit{bounded} sequence converges.
\end{proof}

The key takeaway from this theorem is that we don't actually need to specify a value for a limit in order to show that it converges. As long as we have a monotone sequence and that we know it is bounded then we know for sure that the sequence converges.

\begin{tcolorbox}
    \begin{defn}[Convergence of a Series]
        Let \( (b_n)\) be a sequence. An \textit{infinite series} is a formal expression of the form 
        \[ \sum_{n=1}^{ \infty} b_n = b_1 + b_2 + b_3 + \dots { }. \]
        We define the corresponding \textit{sequence of partial sums} \( (s_m)\) by 
        \[ s_m = b_1 + b_2 + b_3 + ... + b_m = \sum_{i=1}^{m} s_i,\]
        and say that the series \( \sum_{n=1}^{ \infty} b_n \) \textit{converges} to \( B \) if the sequence \( (s_m)\) converges to \( B \). In this case, we write 
        \[ \sum_{n=1}^{ \infty} b_n = B.\]
\end{defn}
\end{tcolorbox} 

\begin{ex}
Consider 
\[ \sum_{n=1}^{ \infty} \frac{1}{n^2}.\]
Because the terms in the sum are all positive, the sequence of partial sums are given by 
\[ s_m = \sum_{k = 1}^{ m} \frac{1}{k^2}\]
is increasing. Our goal is to show that this sequence is convergent so that the series converges. We proceed by using the Monotone Convergence Theorem to do this. Since we already have a monotone sequence of partial sums, only we need to do now find an upper bound for \( s_m \). Observe that  
\begin{align*}
s_m &= 1 = \frac{1}{2 \cdot 2} + \frac{1}{3 \cdot 3} + \frac{1}{4 \cdot 4} + ... + \frac{1}{m^2} \\
    &< 1 + \frac{1}{2 \cdot 1} + \frac{1}{3 \cdot 2} + \frac{1}{4 \cdot 3} + ... + \frac{1}{m(m-1)} \\
    &= 1 + \Big(  1 - \frac{1}{2}\Big) + \Big( \frac{1}{2} - \frac{1}{3} \Big) + \Big( \frac{1}{3} - \frac{1}{4} \Big) + ... + \Big( \frac{1}{(m-1)} - \frac{1}{m} \Big) \\
    &= 1 + 1 - \frac{1}{m} \\
    &< 2. 
\end{align*}
The third second equality is found by taking the partial fractions of the line before it.
Thus, we find that \( 2 \) is an upper bound for the sequence of partial sums, so we can conclude that the infinite series 
\[ \sum_{n=1}^{ \infty} \frac{1}{n^2} \]
is convergent.
\end{ex}

\begin{ex}[Harmonic Series]
    Let's consider the Harmonic Series 
    \[ \sum_{n=1}^{ \infty} \frac{1}{n}. \]
    The sequence of partial sums is defined as follows
    \[ s_m = \sum_{k=1}^{m} \frac{1}{k}.\]
    Like our last example, we expect these sequence of terms to be bounded by 2 but upon further inspection, we have 
    \[ s_4 = 1  + \frac{1}{2} + \Big( \frac{1}{3} + \frac{1}{4} \Big) > 1 + \frac{1}{2} + \Big( \frac{1}{4}+ \frac{1}{4} \Big) = 2\]
    which is not true. Similarly, we find that \( s_8 > 2 \frac{1}{2}\), and we can see that in general we have that 
    \begin{align*}
        s_{2^k}&= 1 + \frac{1}{2} + \Big( \frac{1}{3} + \frac{1}{4} \Big) + \Big( \frac{1}{5} + ... + \frac{1}{8} \Big) + ... + \Big( \frac{1}{2^{k-1}} + ... + \frac{1}{2^k} \Big) \\
               &> 1 + \frac{1}{2} + \Big( \frac{1}{4} + \frac{1}{4}\Big) + \Big( \frac{1}{8} + ... + \frac{1}{8}\Big) + ... + \Big( \frac{1}{2^k} + ... + \frac{2}{2^k}\Big) \\
               &= 1 +... + \Big( 2^{k-1} \frac{1}{2^k}\Big) \\ 
               &= 1 + \frac{1}{2} + \frac{1}{2} + ... + \frac{1}{2} \\
               &= 1 + k\frac{1}{2}.
    \end{align*}
    This shows that our sequence is unbounded because we found \( M = 1 + k \Big( \frac{1}{2}\Big) > 0\) such that \( s_k > M \). Despite how slow the sequence of partial of sums may be at reaching this point, it does end up surpassing every number on the postive real line. Since we have an unbounded sequence of partial sums, we conclude that the Harmonic series as divergent.

\end{ex}


\begin{tcolorbox}
    \begin{thm}[Cauchy Condensation Test]
    Suppose \( (b_n)\) is decreasing and satisfies \( b_n \geq 0  \) for all \( n \in \N \). Then, the series 
    \[ \sum_{n=0}^{ \infty} b_n \]
    converges if and only if 
    \[ \sum_{n=0}^{ \infty} 2^n b_{2^n}\]
converges.
\end{thm}
\end{tcolorbox} 

\begin{proof}
    For the forwards direction, assume that \( \sum_{n=0}^{ \infty} 2^n b_{2^n}\) converges. This means that the sequence of partial sums 
    \[ t_k = b_1 + 2b_2 + ... + 2^k b_{2k}\] are bounded. Hence, there exists \( M > 0 \) such that \( t_k \leq M \) for all \( k \in \N \). Our goal is to show that the sequence of partial sums for the series 
    \[ \sum_{n=0}^{ \infty} b_n.\] 
     Since \( b_n \geq 0 \) and that for all \( n \in \N \) \( b_n \) decreasing, we have that the partial sums \( t_k \) is monotone. Our goal is to show that 
     \[ s_m = \sum_{k=0}^{m} b_k  \]
     is bounded. Hence, fix \( m\) and let \( k \) be large enough to ensure \( m \leq 2^{k+1} - 1\) and hence \( s_m \leq s_{2^{k+1} - 1}\) which imply that 
     \begin{align*}
         s_{2^{k+1}-1}&= b_1 + (b_2 + b_3) + (b_4 + b_5 + b_6 + b_7 ) + ... + (b_{2^k} + ... + b_{2^{k+1}-1})\\ 
                      &\leq b_1 + (b_2 + b_2) + (b_4 + b_4 + b_4 + b_4) + ... + (b_{2k} + ... + b_{2k}) \\
                      &= b_1 + 2b_2 + 4b_4 + ... + 2^k b_{2k} \\
                      &= t_k 
     \end{align*}
     Hence, we have \( s_m \leq s_{2^{k+1} - 1} < t_k \leq M \) which means that \( (s_m)\) is bounded. By the Monotone Convergence Theorem, it follows that the series \( \sum_{n=1}^{ \infty} b_n \) converges. 
     For the forwards direction, we proceed with contrapostive. Hence, assume for sake of contradiction that the series 
     \[ \sum_{n=0}^{ \infty} 2^n b_{2^n}\]
     is a divergent series. We want to show that the series 
     \[ \sum_{n=0}^{ \infty} b_n \]
     is also a divergent series.
\end{proof}

\subsection{Exercises}

\subsubsection{Exercise 2.4.1}
\begin{enumerate}
    \item[(a)] Prove that the sequence defined by \( x_1 = 3\) and 
        \[ x_{n+1} = \frac{1}{4 - x_n}\]
        for all \( n \in \N \) converges.
        \begin{proof}
            Let \( (x_n)\) be the sequence defined by \( x_1 = 3\) and 
            \[ x_{n+1} = \frac{1}{4-x_n}\]
            for all \( n \in \N\). Our goal is to show that \( (x_n)\) is convergent. It is sufficient to show that \( (x_n)\) is both \textit{monotone} and \textit{bounded}. We first show that \( (x_n)\) is \textit{monotone}. We claim that \( (x_n)\) is a \textit{decreasing} sequence. Hence, we will show that for all \( n \in \N \), we have \( x_n > x_{n+1}\). We proceed by inducting on \( n \). Let the base case be \( n = 1 \). Then we have that 
            \begin{align*}
                x_1 = 3 &> x_2 = \frac{1}{4 - 3} = 1. \\ 
            \end{align*}
            Hence, we have \( x_1 > x_2 \). Now we assume that \( (x_n)\) is decreasing for all \( 1 < n \leq k-1 \). We want to show that \( x_n > x_k \) for all \( n < k \). Since \( n \leq k - 1\), we have \( x_{k-1} \leq x_n \) by inductive hypothesis. Consider \( x_k \). By definition, we have that \( x_k = 1 /(4 - x_{k-1})\). Since \( x_{k-1} \leq x_n \), then for all \( n \in \N \) we have 
            \[
                x_k = \frac{1}{4 - x_{k-1}} < \frac{1}{4-x_n}.
            \]
            Hence, \( x_k < x_n \) for all \( n \in \N \). This is equivalent to showing \( x_n > x_{n+1}\) for all \( n \in \N \). Therefore, \( (x_n)\) is a \textit{monotone} sequence. 
            Now we show that \( (x_n)\) is \textit{bounded}. Since \( 3 = x_1 \geq x_{n+1}\) for all \( n \in \N \) and \( x_{n+1} = 1 / (4 - x_n) > 0 \), we have that 
            \[ 0 < x_n \leq 3.\]
            Hence, \( (x_n)\) is bounded. Since \( (x_n)\) is \textit{monotone} and \textit{bounded}, we have that 
            \( (x_n)\) is a convergent sequence by the Monotone Convergence theorem. 
        \end{proof}
    \item[(b)] Now that we know \( \lim x_n \) exists, explain why \( \lim x_{n+1}\) must also exist and equal the same value.
        \begin{proof}[Solution]
            Since \( (x_n) \) is \textit{monotone} and \textit{bounded}, then \( (x_{n+1})\) is also \textit{monotone} and \textit{bounded}. By the Monotone Convergence Theorem, we have that \( (x_{n+1})\) is also convergent. Hence, \( \lim x_{n+1}\) also exists.    
        \end{proof}
    \item[(c)] Take the limit of each side of the recursive equation in part (a) to explicitly compute \( \lim x_n \). 
        \begin{proof}[Solution]
            Since \( \lim x_n = \lim x_{n+1}\), we have
            \begin{align*}
                x = \lim x_{n+1} &= \lim \frac{1}{4-x_n}  \\ 
                                 &= \frac{ \lim 1 }{ \lim (4 - x_n ) } \\ 
                                 &= \frac{1}{ \lim (4) - \lim x_n} \\
                                 &= \frac{1}{4 - x} \tag{ \( \lim x_n = x \)}.
            \end{align*}
            Then we have 
            \begin{align*}
                x &= \frac{1}{4-x} \\
            \end{align*}
            and then 
            \begin{align*}
                x^2 - 4x + 1 = 0 \\ 
            \end{align*}
            which we can solve via the quadratic formula. Hence, we have \( x = 2 + \sqrt{3}\). 
        \end{proof}

\end{enumerate}

\subsubsection{Exercise 2.4.3}
Following the model of Exercise 2.4.2, show that the sequence defined by \( y_1 = 1 \) and \( y_{n+1} = 2 - \frac{1}{y_n}\) converges and find the limit.

\begin{proof}
Let \( (y_n)\) be the sequence defined by \( y_1 = 1 \) and 
\[
    y_{n+1} = 4 - \frac{1}{y_n}.
\] for all \( n \in \N \). 
We want to show that \( (y_n)\) converges. Hence, our goal is to show that \( (y_n)\) is \textit{monotone} and \textit{bounded}. We claim that \( (y_n)\) is increasing. Hence, we show this by inducting on \( n \in \N \). Our goal is to show that \( y_n \leq y_{n+1}\) for all \( n \in \N \). Let the base case be \( n = 1 \). Then observe that 
\begin{align*}
 y_1 &= 1 < y_2 = 4 - \frac{1}{1} = 3  \\ 
\end{align*}
Hence, we have \( y_1 < y_2\). 

Now assume that \( (y_n)\) is increasing for all \( 1 \leq n \leq k - 1\). Hence, \( y_n \leq y_{k-1}\). Our goal now is to show that \( y_n \leq y_k\) for all \( n \in \N \). Let's consider \( y_k\). Then by definition of \( (y_n)\), we have 
\[
    y_k = 4 - \frac{1}{y_{k-1}}.
\]
Since \( y_n \leq y_{k-1}\), we have 
\[
y_k = 4 - \frac{1}{y_{k-1}} \geq 4 - \frac{1}{y_{n}}
\]
This shows that \( y_{k} \geq y_n \) for any \( n \in \N \). Hence, it follows that \( y_{n}\) is an increasing sequence and, therefore, \textit{monotone}. Now

Now we want to show that \( (y_n)\) is \textit{bounded}. Observe that \( 1 < y_n \) for all \(n \in \N \) which means \( (y_n)\) contains a lower bound. Furthermore, for each \(n \in \N\) we also have that \( y_{n+1} = 4 - 1/y_n < 4\) which means that \( (y_n)\) also contains an uppper bound. Hence, it follows that 
\[
1 < y_n < 4
\]
for all \( n \in \N \). Hence, we have \( (y_n)\) is \textit{bounded}. By the Monotone Convergence Theorem, it follows that \( (y_n)\) is a convergent sequence. 

By last exercise, we know that \( \lim y_n = \lim y_{n+1}  \). Let's assume \( (y_n) \to y \). Our goal is to compute \( \lim y_n \). By the Algebraic Limit Theorem, we have 
\begin{align*}
    y = \lim y_n &= \lim \Big( 4 - \frac{1}{y_n}\Big)  \\ 
                 &= \lim (4) - \lim \Big( \frac{1}{y_n}\Big) \\
                 &= 4 - \frac{ \lim(1) }{ \lim y_n } \\
                 &= 4 - \frac{1}{y}.
\end{align*}
Hence, we have 
\begin{align*}
y &= 4 - \frac{1}{y} \\ 
\end{align*}
which yields the following quadratic equation set to zero
\[
y^2 -4y + 1 = 0. 
\]
Solving for \( y \) using the quadratic formula yields \( y = 2 + \sqrt{3}\)
\end{proof}

\subsubsection{Exercise 2.4.5 (Calculating Square Roots)}
Let \(x_1 = 2\), and define 
\[
    x_{n+1} = \frac{1}{2} \Big( x_n + \frac{2}{x_n}\Big).
\]

\begin{enumerate}
    \item[(a)] Show that \( x_n^2\) is always greater than 2, and then use this to prove that \( x_n - x_{n+1} \geq 0\). Conclude that \( \lim x_n = \sqrt{2} \).
        \begin{proof}
         Our first goal is to show that \( x_n^2 > 2 \) for all \( n \in \N \). We proceed by inducting on \( n \in \N\). Let our base case be \( n = 1\). Then 
         \begin{align*}
         x_1 = 2 &< x_1^2   \\
                 &= 4 \\ 
                 &< \frac{9}{4} \\
                 &= \frac{1}{4} \Big( x_1^2 + \frac{4}{x_1^2} + 4 \Big) \\
                 &= x_2^2
         \end{align*}
         which implies that \( 2 < x_1^2 < x_2^2\). Now suppose \( x_{k-1}^2 > 2\) for all \( n \leq k - 1\). We want to show that \( x_k^2 > 2 \) for all \( n \in [1, k)\). Consider \( x_k \) and then by definition, we have 
         \begin{align*}
             x_{k}^2 &= \frac{1}{4} \Big( x_{k-1}^2 + \frac{4}{x_{k-1}^2} + 4 \Big) \\
                   &> \frac{1}{4} (2 + 2 + 4) \\ 
                   &= \frac{8}{4} \\
                   &= 2.\\
        \end{align*}
        Hence, \( x_k^2 > 2\) for all \( n \in \N\).  Now we want to show that \( x_n - x_{n+1} \geq 0  \) for all \( n \in \N \). Consider \( x_n - x_{n+1}\) then observe that since \( x_n^2 > 2 \) for all \( n \in \N \), we have 
        \begin{align*}
            x_n - x_{n+1}&= x_n - \frac{1}{2} \Big( x_n + \frac{2}{x_n}\Big) \\ 
                        &= \frac{x_n^2 - 2 }{2x_n} \\
                        &> \frac{2 - 2 }{2 \sqrt{2}} \\
                        &= 0. 
        \end{align*}
        Furthermore, when \( x^2 = 2 \) we get that \( x_n - x_{n+1} = 0 \). Hence, we have \( x_n - x_{n+1} \geq 0\) for all \( n \in \N \). By the Monotone Convergence Theorem, we get that \( (x_n)\) is a convergent sequence. Since \( \lim x_n = \lim x_{n+1}  \), we can show that \( \lim x_n = \sqrt{2}\). By the Algebraic Limit Theorem, we have 
        \begin{align*}
            x = \lim x_{n+1} &= \lim \Big( \frac{1}{2} \Big( x_n + \frac{2}{x_n}\Big)\Big)  \\ 
                            &= \frac{1}{2} \lim \Big( x_n + \frac{2}{x_n}\Big) \\ 
                            &= \frac{1}{2} \Big( \lim x_n + \lim  \frac{2}{x_n}  \Big) \\
                            &= \frac{1}{2} \Big(x + \frac{2}{x}\Big) \\
                            &= \frac{1}{2}x + \frac{1}{x} \\
                            &= \frac{x^2 + 2}{2x} \\
        \end{align*}
        which implies that 
        \begin{align*}
        x^2 &= 2 \\ 
        \implies x &= \sqrt{2}.
        \end{align*}
        Hence, we have \( \lim x_n = \sqrt{2} \)
    \end{proof} 
    \item[(b)] Modify the sequence \( (x_n)\) so that it converges to \( \sqrt{c}\).
        \begin{proof}[Solution]
         Let the sequence \( (x_n)\) be defined recursively as \( x_1  =c \) and 
        \[
            x_{n+1} = \frac{1}{c} \Big( x_n + \frac{c}{x_n}\Big). 
        \] 
        Assume \( x_n^2 > c \) for all \( n \in \N \) and \( x_n - x_{n+1} \geq 0\), then we have \( \lim x_n = \sqrt{c} \).
        \end{proof}
        
\end{enumerate}

\subsubsection{Exercise 2.4.6 (Limit Superior.) }
Let \( (a_n)\) be a \textit{bounded} sequence. 
\begin{enumerate}
    \item[(a)] Prove that the sequence defined by \( y_n = \sup \{ a_k : k \geq n \}\) converges. 
        \begin{proof}
        Let \( (a_n)\) be a \textit{bounded} sequence. Let \( (y_n)\) be defined as 
        \[ y_n = \sup \{ a_k : k \geq n \}\] and further denote
        \[ A_n = \{ a_k : k \geq n\}.\]
        Our goal is to show that \( (y_n)\) converges. Thus, our goal is to show that \( (y_n)\) is both \textit{monotone} and \textit{bounded}.

        We first show the former. We claim that \( (y_n)\) is a \textit{decreasing} sequence. Hence, we need to show that \( y_n \geq y_{n+1}\) for all \( n \in \N \). We know that for each term \( y_n \), by definition, is the \textit{least upper bound} of the set \( \{a_k: k \geq n\}\). By exercise 1.3.4, we know that since \( A_{n+1} \subseteq A_n \), we have that \( y_{n+1} \leq y_n\) for all \( n \in \N \). Hence, we have that \( y_n\) is a \textit{decreasing} sequence.

        Now we want to show that \( (y_n)\) is \textit{bounded}. Since \( (a_n)\) is a \textit{bounded} sequence, there exists an \( M > 0 \) such that \( M \leq a_n \leq y_n  \) for all \( n \in \N \). Hence, \( (y_n)\) is a \textit{bounded} sequence. In this case, it is enough to have \( (y_n)\) be bounded below.

        Since \( (y_n)\) is \textit{monotone} and \textit{bounded}, we have that \( (y_n)\) is a convergent sequence.

        \end{proof}
    \item[(b)] The \textit{limit superior} of \( (a_n)\), or \( \lim \sup a_n   \) is defined by 
        \[ \lim  \sup a_n = \lim y_n   \] where \( y_n \) is the sequence from part (a) of this exercise. Provide a reasonable definition for \( \lim \inf a_n  \) and briefly explain why it always exists for any bounded sequence.
      \begin{proof}[Solution]
       Let \( a_n \) be a \textit{bounded} sequence. Then define   
       \[ w_n = \inf \{ a_k : k \geq n  \}\]
       so we can have 
       \[ \lim w_n = \lim \inf\{a_k : k \geq n\}. \]
      This limit exists because the terms of \( (w_n) \) are \textit{increasing} and that \( (w_n) \) is \textit{bounded} since there exists an \( L > 0 \) such that \( w_n \leq a_n \leq L \).


       \end{proof} 
    \item[(c)] Prove that \( \lim \inf a_n \leq \lim \sup a_n  \) for every \textit{bounded} sequence, and give an example of a sequence for which the inequality is strict.
        \begin{proof}
        Suppose that \( (a_n)\) is a \textit{bounded} sequence. Since \( \lim \inf a_n \) and \( \lim \sup a_n \) exists, we have that \( \lim \inf a_n \leq a_n \leq \lim \sup a_n  \). Hence, 
        \[ \lim \inf a_n \leq \lim \sup a_n.  \]
        \end{proof}
    \item[(d)] Show that \( \lim \inf a_n = \lim \sup a_n\) if and only if \( \lim a_n \) exists. In this case, all three share the same value.
        \begin{proof}
        We start by assuming that \( \lim \inf a_n = \lim \sup a_n \). We want to show that the \( \lim a_n  \) exists. Define the following:
        \begin{align*}
        w &= \lim \inf a_n = \lim w_n ,  \\ 
        y &= \lim \sup a_n = \lim y_n . 
        \end{align*} 
        Since \( w_n \) and \( y_n\) both bound \( a_n \), it follows that \( w_n \leq a_n \leq y_n\).
        Since the \( \lim \inf a_n \) and \( \lim \sup a_n \) both exists such that \( \lim  w_n = \lim y_n = \ell \), it follows that \( \lim a_n = \ell \) by the \textit{Squeeze Theorem}. Hence, \( \lim a_n \) exists.

        Now assume the converse. Hence, our goal now is to show that \( \lim \inf a_n = \lim \sup a_n\). Since \( (a_n)\) is a convergent sequence, let \( \epsilon > 0  \) such that there exists an \( N \in \N\) where for every \( n \geq N \), we have
        \[ |a_n - \ell| < \epsilon. \]  
        This is equivalent to saying 
        \[ \ell - \epsilon \leq a_n \leq \ell + \epsilon. \]
        Since \( \epsilon > 0  \) is abitrary, we can conclude that \( \lim \inf a_n = \lim \sup a_n  \).

        \end{proof}
        
\end{enumerate}


\subsubsection{Exercise 2.4.9} Complete the proof of Theorem 2.4.6 by showing that if the series \( \sum_{n=0}^{ \infty} 2^n b_{2n}  \) diverges, then so does \( \sum_{n=1}^{ \infty} b_n \). Example 2.4.5 may be a useful reference.

\begin{proof}
    Our goal is to show that \( \sum_{n=1}^{ \infty} b_n\) diverges. Since \( \sum_{n=0}^{ \infty} b_{{2^n}} \) diverges, the sequence of partial sums 
    \[t_k = b_1 + 2b_2 + 4b_4 + ... + 2^k b_{2^k}\]
    diverges. This implies that \( t_k \) is \textit{unbounded}: that is, there exists \( M > 0 \) such that for some \( K \in \N \), we have for all \( k \geq K \) we have \( t_k > M \). Since \( b_n \geq 0\), it suffices to show that the partial sums of \( \sum_{n=1}^{ \infty} b_n\) are \textit{unbounded}. Let 
    \[ s_m = \sum_{k=1}^{m} b_k = b_1 + b_2 + b_2 + ... + b_m.  \]
    Let us fix \( m \) such that we choose \( k \) sufficiently large so that \( m \geq 2^{k+1} + 1 \). Hence, we have that \( s_m \geq s_{2^{k-1} + 1} \). Observe that 
    \begin{align*}
        s_{2^{k+1}+1} &= b_1 + (b_2 + b_3) + (b_4 + b_5 + b_6 + b_7) + ... + (b_{2^k} + ... + b_{2^{k+1}+1})  \\ 
                      &\geq b_1 + (b_2 + b_2) + (b_4 + b_4 + b_4 + b_4) + ... + (b_{2k} + ... + b_{2k}) \\
                      &= b_1 + 2b_2 + 4b_4 + ... + 2^k b_{2k} \\
                      &= t_k.
    \end{align*}
    This implies that \( s_m \geq s_{2^{k+1}+1} \geq t_k > M \) for all \( k \geq K\). Hence, we conclude \( t_k\) is \textit{unbounded}. Thus, the series 
    \[ \sum_{n=1}^{ \infty} b_n.\]
    diverges.
\end{proof}

















\subsubsection{Exercise 2.4.10 (Infinite Products)}
A close relative of infinite series is the \textit{infinite product} 
\[ \prod_{n=1}^{ \infty } b_n = b_1 b_2 b_3 \dots \]
which is understood in terms of its sequence of \textit{partial products}
\[ p_m = \prod_{n=1}^m b_n = b_1b_2b_3 \dots b_m.\]
Consider the special class of infinite products of the form 
\[ \prod_{n=1}^{ \infty} (1 + a_n) = (1+a_1)(1+a_2)(1+a_3) \dots ~ \text{where } a_n \geq 0 \]
\begin{enumerate}
    \item[(a)] Find an explicit formula for the sequence of partial products in the case where \( a_n = 1 / n \) and decide whether the sequence converges. Write out the first few terms in the sequence of partial products in the case where \( a_n = 1 / n^2\) and make a conjecture about the convergence of this sequence.
    \item[(b)] Show, in general, that the sequence of partial products converges if and only if \( \sum_{n=1}^{ \infty} a_n \) converges. (The inequality \( 1 + x \leq 3^x\)) for positive \( x \) will be useful in one direction.)
\end{enumerate}

\subsubsection{Exercise 2.4.6 (Limit Superior.) }
Let \( (a_n)\) be a \textit{bounded} sequence. 
\begin{enumerate}
    \item[(a)] Prove that the sequence defined by \( y_n = \sup \{ a_k : k \geq n \}\) converges. 
        \begin{proof}
        Let \( (a_n)\) be a \textit{bounded} sequence. Let \( (y_n)\) be defined as 
        \[ y_n = \sup \{ a_k : k \geq n \}\] and further denote
        \[ A_n = \{ a_k : k \geq n\}.\]
        Our goal is to show that \( (y_n)\) converges. Thus, our goal is to show that \( (y_n)\) is both \textit{monotone} and \textit{bounded}.

        We first show the former. We claim that \( (y_n)\) is a \textit{decreasing} sequence. Hence, we need to show that \( y_n \geq y_{n+1}\) for all \( n \in \N \). We know that for each term \( y_n \), by definition, is the \textit{least upper bound} of the set \( \{a_k: k \geq n\}\). By exercise 1.3.4, we know that since \( A_{n+1} \subseteq A_n \), we have that \( y_{n+1} \leq y_n\) for all \( n \in \N \). Hence, we have that \( y_n\) is a \textit{decreasing} sequence.

        Now we want to show that \( (y_n)\) is \textit{bounded}. Since \( (a_n)\) is a \textit{bounded} sequence, there exists an \( M > 0 \) such that \( M \leq a_n \leq y_n  \) for all \( n \in \N \). Hence, \( (y_n)\) is a \textit{bounded} sequence. In this case, it is enough to have \( (y_n)\) be bounded below.

        Since \( (y_n)\) is \textit{monotone} and \textit{bounded}, we have that \( (y_n)\) is a convergent sequence.

        \end{proof}
    \item[(b)] The \textit{limit superior} of \( (a_n)\), or \( \lim \sup a_n   \) is defined by 
        \[ \lim  \sup a_n = \lim y_n   \] where \( y_n \) is the sequence from part (a) of this exercise. Provide a reasonable definition for \( \lim \inf a_n  \) and briefly explain why it always exists for any bounded sequence.
      \begin{proof}[Solution]
       Let \( a_n \) be a \textit{bounded} sequence. Then define   
       \[ w_n = \inf \{ a_k : k \geq n  \}\]
       so we can have 
       \[ \lim w_n = \lim \inf\{a_k : k \geq n\}. \]
      This limit exists because the terms of \( (w_n) \) are \textit{increasing} and that \( (w_n) \) is \textit{bounded} since there exists an \( L > 0 \) such that \( w_n \leq a_n \leq L \).


       \end{proof} 
    \item[(c)] Prove that \( \lim \inf a_n \leq \lim \sup a_n  \) for every \textit{bounded} sequence, and give an example of a sequence for which the inequality is strict.
        \begin{proof}
        Suppose that \( (a_n)\) is a \textit{bounded} sequence. Since \( \lim \inf a_n \) and \( \lim \sup a_n \) exists, we have that \( \lim \inf a_n \leq a_n \leq \lim \sup a_n  \). Hence, 
        \[ \lim \inf a_n \leq \lim \sup a_n.  \]
        \end{proof}
    \item[(d)] Show that \( \lim \inf a_n = \lim \sup a_n\) if and only if \( \lim a_n \) exists. In this case, all three share the same value.
        \begin{proof}
        We start by assuming that \( \lim \inf a_n = \lim \sup a_n \). We want to show that the \( \lim a_n  \) exists. Define the following:
        \begin{align*}
        w &= \lim \inf a_n = \lim w_n ,  \\ 
        y &= \lim \sup a_n = \lim y_n . 
        \end{align*} 
        Since \( w_n \) and \( y_n\) both bound \( a_n \), it follows that \( w_n \leq a_n \leq y_n\).
        Since the \( \lim \inf a_n \) and \( \lim \sup a_n \) both exists such that \( \lim  w_n = \lim y_n = \ell \), it follows that \( \lim a_n = \ell \) by the \textit{Squeeze Theorem}. Hence, \( \lim a_n \) exists.

        Now assume the converse. Hence, our goal now is to show that \( \lim \inf a_n = \lim \sup a_n\). Since \( (a_n)\) is a convergent sequence, let \( \epsilon > 0  \) such that there exists an \( N \in \N\) where for every \( n \geq N \), we have
        \[ |a_n - \ell| < \epsilon. \]  
        This is equivalent to saying 
        \[ \ell - \epsilon \leq a_n \leq \ell + \epsilon. \]
        Since \( \epsilon > 0  \) is abitrary, we can conclude that \( \lim \inf a_n = \lim \sup a_n  \).

        \end{proof}
        
\end{enumerate}


\subsubsection{Exercise 2.4.9} Complete the proof of Theorem 2.4.6 by showing that if the series \( \sum_{n=0}^{ \infty} 2^n b_{2n}  \) diverges, then so does \( \sum_{n=1}^{ \infty} b_n \). Example 2.4.5 may be a useful reference.

\begin{proof}
    Our goal is to show that \( \sum_{n=1}^{ \infty} b_n\) diverges. Since \( \sum_{n=0}^{ \infty} b_{{2^n}} \) diverges, the sequence of partial sums 
    \[t_k = b_1 + 2b_2 + 4b_4 + ... + 2^k b_{2^k}\]
    diverges. This implies that \( t_k \) is \textit{unbounded}: that is, there exists \( M > 0 \) such that for some \( K \in \N \), we have for all \( k \geq K \) we have \( t_k > M \). Since \( b_n \geq 0\), it suffices to show that the partial sums of \( \sum_{n=1}^{ \infty} b_n\) are \textit{unbounded}. Let 
    \[ s_m = \sum_{k=1}^{m} b_k = b_1 + b_2 + b_2 + ... + b_m.  \]
    Let us fix \( m \) such that we choose \( k \) sufficiently large so that \( m \geq 2^{k+1} + 1 \). Hence, we have that \( s_m \geq s_{2^{k-1} + 1} \). Observe that 
    \begin{align*}
        s_{2^{k+1}+1} &= b_1 + (b_2 + b_3) + (b_4 + b_5 + b_6 + b_7) + ... + (b_{2^k} + ... + b_{2^{k+1}+1})  \\ 
                      &\geq b_1 + (b_2 + b_2) + (b_4 + b_4 + b_4 + b_4) + ... + (b_{2k} + ... + b_{2k}) \\
                      &= b_1 + 2b_2 + 4b_4 + ... + 2^k b_{2k} \\
                      &= t_k.
    \end{align*}
    This implies that \( s_m \geq s_{2^{k+1}+1} \geq t_k > M \) for all \( k \geq K\). Hence, we conclude \( t_k\) is \textit{unbounded}. Thus, the series 
    \[ \sum_{n=1}^{ \infty} b_n.\]
    diverges.
\end{proof}


\subsubsection{Exercise 2.4.10 (Infinite Products)}
A close relative of infinite series is the \textit{infinite product} 
\[ \prod_{n=1}^{ \infty } b_n = b_1 b_2 b_3 \dots \]
which is understood in terms of its sequence of \textit{partial products}
\[ p_m = \prod_{n=1}^m b_n = b_1b_2b_3 \dots b_m.\]
Consider the special class of infinite products of the form 
\[ \prod_{n=1}^{ \infty} (1 + a_n) = (1+a_1)(1+a_2)(1+a_3) \dots ~ \text{where } a_n \geq 0 \]
\begin{enumerate}
    \item[(a)] Find an explicit formula for the sequence of partial products in the case where \( a_n = 1 / n \) and decide whether the sequence converges. Write out the first few terms in the sequence of partial products in the case where \( a_n = 1 / n^2\) and make a conjecture about the convergence of this sequence.
    \item[(b)] Show, in general, that the sequence of partial products converges if and only if \( \sum_{n=1}^{ \infty} a_n \) converges. (The inequality \( 1 + x \leq 3^x\)) for positive \( x \) will be useful in one direction.)
\end{enumerate}







































\section{Subsequences and Bolzano-Weierstrass}

In the last section, we observed that the convergence of partial sums of a particular series can be determined by the behavior of a subsequence of the partial sums. 

\begin{tcolorbox}
\begin{defn}
Let \( (a_n) \subseteq \R \), and let \( n_1 < n_2 < n_3 < n_4 < \dots\) be an increasing sequence of natural numbers. Then the sequence 
\[ (a_{n_1}, a_{n_2}, a_{n_3}, a_{n_4}, a_{n_5} \dots)\]
is called a \textit{subsequence} of \( (a_n)\) and is denoted by \( (a_{n_k})\), where \( k \in \N \) indexes the subsequence.
\end{defn}
\end{tcolorbox} 

A few remarks about subsequences: 

\begin{enumerate}
    \item[(a)] The order of the subsequence is the same as in the original sequence.
        \begin{ex}
        If we have the sequence  
        \[ (a_n) = \Big( 1, \frac{1}{2}, \frac{1}{3}, \frac{1}{4}, ...\Big)\]
        then the subsequences 
        \[ \Big( \frac{1}{2}, \frac{1}{4}, \frac{1}{6}, \frac{1}{8},... \Big)\]
        and 
        \[ \Big( \frac{1}{10}, \frac{1}{100}, \frac{1}{1000}, \frac{1}{10000}, ...\Big)\]
        are permitted.
        \end{ex}
        
    \item[(b)] Repetitions and swapping are not allowed.
        \begin{ex}
        Like 
        \[ \Big( \frac{1}{10}, \frac{1}{5}, \frac{1}{100}, \frac{1}{50}, \frac{1}{1000}, \frac{1}{500},...\Big)\] 
        and 
        \[ \Big( 1,1, \frac{1}{3}, \frac{1}{5}, \frac{1}{5}, ...\Big)\]
        \end{ex}
        
\end{enumerate}

Since subsequences have the same ordering as the original sequence, one can conjecture about them converging to the same limit. 

\begin{tcolorbox}
\begin{thm}
Subsequences of a convergent sequence converge to the same limit as the original sequence.
\end{thm}
\end{tcolorbox}

\begin{proof}
    Let \( (a_n) \to a \) and let \( (a_{n_k})\) be a subsequence for \( (a_n)\). We want to show \( (a_{n_k})\) converges to \( a \) as well. Since \( (a_n) \to a\), there exists an \( N \) such that for any \( n \geq N \), we have \( |a_n - a| < \epsilon \). 

    We claim that \( n_{k} \geq k \) for any \( k \in \N\). Let us proceed by inducting on \( k \). Let the base case be \( k = 1 \). Since \(n_k\) is an \textit{increasing} sequence of natural numbers, we see that \( n_1 \geq 1\). Now let us assume \(n_{k-1} \geq k - 1\). Since \( (a_{n_k})\) in \textit{increasing}, we have \( a_k \geq a_{k-1} \geq k - 1\) which implies that \( n_k \geq k \). 

    Since any choice of \( n \geq N \), we can say that \( n_k \geq k \geq N \). Hence, we have 
    \[ |a_{n_k} - a| < \epsilon \]
which is what we desired.
\end{proof}

\begin{ex}
Let \(0 < b < 1\). Because 
\[ b > b^2 > b^3 > b^4 > ... > 0,\]
the sequence \((b^n)\) is \textit{decreasing} and \textit{bounded} below. The Monotone Convergence Theorem allows us to conclude that \((b^n)\) converges to some \(\ell\) satisfying \( 0 \leq \ell < b\). To compute \(\ell\), notice that \((b^{2n})\) is a subsequence, so \(b^{2n} \to \ell\) by Theorem 2.5.2. But \( b^{2n} = b^n \cdot b^n\), so by the Algebraic Limit Theorem, \(b^{2n} \to \ell \cdot \ell = \ell^2 \). Because limits are unique (Theorem 2.2.7), \( \ell^2 = \ell\), and thus \( \ell = 0\). 
\end{ex}










\begin{ex}
Suppose we have an oscillating sequence of numbers 
\[ \Big( 1, -\frac{1}{2}, \frac{1}{3}, - \frac{1}{4}, \frac{1}{5}, -\frac{1}{5}, \frac{1}{5}, -\frac{1}{5},... \Big)\]
Note that this sequence does not converge to any proposed limit yet if we take a subsequence of it, weget a sequence that converges! Observe, that the subsequence 
\[ \Big(\frac{1}{5}, \frac{1}{5}, \frac{1}{5}, \frac{1}{5},... \Big)\]
and
\[ \Big(-\frac{1}{5}, -\frac{1}{5}, -\frac{1}{5}, -\frac{1}{5},... \Big)\]
converge to \( 1 / 5 \) and \(- 1 / 5\) respectively. Since we have two subsequences that converge to two different limits, we immediately conclude that the original sequence diverges.
\end{ex}

This leads us to our next theorem that states that
% add definitions at the end of each chapter or put references to them
\begin{tcolorbox}
    \begin{thm}[Bolzano-Weierstrass Theorem]
        Every bounded sequence contains a convergent subsequence.
\end{thm}
\end{tcolorbox}

\begin{proof}
    Let \( (a_n)\) be a \textit{bounded} sequence. Then there exists \( M > 0 \) such that \( a_n \in [-M,M ]\). Suppose we divide this interval in half for \( k \) times: that is, let the length of the intervals be defined by the sequence \( M (1/2)^{k-1}\). We claim that a subsequence \( (a_{n_k})\) lies in either one of these intervals: that is, let \( n_{k} > n_{k-1}\) for all \( k \in \N \) such that \( a_{n_k} \in I_k\). 

    Let us induct on \( k\). Then let our base case be \( k = 1 \). Since we have an increasing sequence of natural numbers \( n_k\), we have that \( n_2 > n_1 \) which means that \( a_{n_2} \in I_2\) as well as \( a_{n_1} \in I_1\). Now let us assume that this holds for all \( k \leq \ell - 1 \). We want to show that this holds for \( k < \ell \). By the monotonicity of \( n_k \), we have that \( n_{\ell} > n_{\ell - 1} > n_k > n_1\) which implies that \( a_{n_\ell} \in I_\ell\) for all \( \ell \in \N \). Furthermore, the sets 
    \[ I_1 \subseteq I_2 \subseteq I_3 ...\]
form a nested sequence of closed intervals.

    By the \textit{Nested Interval Property}, we can conclude that there exists an \( x \in I_k\) for all \( k \in \N \)  such that \( \bigcup_{k=1}^{ \infty} I_k \neq \emptyset\). Let \( \epsilon > 0  \). Since \( a_{n_k}, x \in I_k \) for all \( k \in \N \) and \( M (1/2)^{k-1} \to 0\) by the Algebraic Limit Theorem, we can choose an \( N \in \N \) such that for any \( k \geq n_k \geq N \), we have
    \[ |a_{n_k} - x | < \epsilon.\] 
    Hence, \( (a_{n_k}) \to x \).
\end{proof}



\subsubsection{Definitions}


\begin{tcolorbox}
\begin{defn}
Let \( (a_n) \subseteq \R \), and let \( n_1 < n_2 < n_3 < n_4 < \dots\) be an increasing sequence of natural numbers. Then the sequence 
\[ (a_{n_1}, a_{n_2}, a_{n_3}, a_{n_4}, a_{n_5} \dots)\]
is called a \textit{subsequence} of \( (a_n)\) and is denoted by \( (a_{n_k})\), where \( k \in \N \) indexes the subsequence.
\end{defn}
\end{tcolorbox} 


\begin{tcolorbox}
\begin{thm}
Subsequences of a convergent sequence converge to the same limit as the original sequence.
\end{thm}
\end{tcolorbox}


\begin{tcolorbox}
    \begin{thm}[Bolzano-Weierstrass Theorem]
        Every bounded sequence contains a convergent subsequence.
\end{thm}
\end{tcolorbox}



\section{The Cauchy Criterion}


\begin{tcolorbox}
\begin{defn}
A sequence \((a_n)\) is called a \textit{Cauchy Sequence} if, for every \( \epsilon > 0 \), there exists an \( N \in \N \) such that whenever \( m,n \geq N \) it follows that 
\[ | a_n - a_m | < \epsilon.\]
\end{defn}
\end{tcolorbox}

In the regular convergence defintion, we are given any \( \epsilon > 0 \) where there is a point in the sequence \(N \in \N \) such that past this point, all of our terms fall within an \(\epsilon\) range around some limit point. In the Cauchy Criterion defintion, we begin with the same conditions but this time, all the terms of the sequence are all tightly packed together wthin the \(\epsilon > 0 \) range we were given. It turns out, that these two definitions are equivalent: that is, \textit{Cauchy sequences} are convergent sequences and convergent sequences are \textit{Cauchy sequences}. 

\begin{tcolorbox}
\begin{thm}
Every convergent sequence is a Cauchy sequence. 
\end{thm}
\end{tcolorbox}

\begin{proof}
Assume \((x_n)\) converges to \(x\). To show that \((x_n)\) is \textit{Cauchy}, there must existsa point \( N \in \N \) after which we can conclude that 
\[ |x_n - x_m| < \epsilon. \]
Let \( \epsilon > 0 \). Since \( (x_n) \to x \), we can choose \( N \in \N \) such that for any \( n,m \geq N \), we have 
\begin{align*}
    |x_n - x|&< \frac{\epsilon}{2}, \tag{1} \\
    |x_m - x|&< \frac{\epsilon}{2}. \tag{2}
\end{align*}
Consider \( |x_n - x_m| \). Then (1) and (2) imply that 
\begin{align*}
    |x_n - x_m|&= |x_n - x + x - x_m| \\
               &< | x_n - x | + | x - x_n | \tag{\text{Triangle Inequality}}\\  
               &< \frac{\epsilon}{2} + \frac{\epsilon}{2} \\
               &= \epsilon.
\end{align*}
Hence, \((x_n)\) is a \textit{Cauchy Sequence}.
\end{proof}

We can prove the other direction, by using either the \textit{Bolzano Weierstrass Theorem} or the \textit{Monotone Convergence Theorem}. This is a little bit more difficult since we need to have a proposed limit for the sequence to converge to.  
\begin{tcolorbox}
\begin{lem}
Cauchy sequences are bounded.
\end{lem}
\end{tcolorbox}

\begin{proof}
    Given \( \epsilon = 1\), there exists an \( N \in \N \) such that \( |x_m - x_n | < 1 \) for amm \( m,n \geq N \). Thus, we must have \( |x_n| < |x_{N}| + 1 \) for all \( n \geq N \) (just substituted \(m = N \) here). Hence, define 
    \[ M = \max \{ |x_1|, |x_1|, |x_1|,..., |x_{N-1}|, |x_{N}| + 1 \}.\]
   Therefore, \( |x_n| < M \) for all \( n \in \N \) Hence, the \textit{Cauchy sequence} \((x_n)\) is \textit{bounded}.
\end{proof}

\begin{tcolorbox}
\begin{thm}
A sequence converges if and only if it is a Cauchy sequence.
\end{thm}
\end{tcolorbox}

\begin{proof}
    (\(\Rightarrow\)) This direction is just Theorem 2.6.2 which we have proved above. 

    (\( \Leftarrow\)) Suppose \((x_n)\) is a \textit{Cauchy sequence}. Let \( \epsilon > 0 \). Since \( (x_n)\) is a \textit{bounded} sequence, there exists a subsequence \((x_{n_k})\) such that \((x_{n_k}) \to x \) by the \textit{Bolzano Weierstrass Theorem}. Let \( \epsilon > 0 \). Then for some \( N \in \N \), every \( n_k \geq N \) has the property
    \[ |x_{n_k} - x| < \epsilon.\]
\end{proof}
Our goal now is to show that \( (x_n) \to x\). Hence, consider \( |x_n - x| \). Then for every \(n, n_{k} \geq N \), we have 
\begin{align*}
    |x_n - x | &= |x_n - x_{n_k} + x_{n_k} - x | \\
     &< |x_n - x_{n_k}| + |x_{n_k} - x| \\
     &< \frac{\epsilon}{2} + \frac{\epsilon}{2} \\
     &= \epsilon.
\end{align*}
Hence, \((x_n) \to x \).  

\subsection{Completeness Revisited}

We can summarize all of our results thus far in the following way 
\[ \text{AOC} 
\begin{cases}
    \text{NIP} \implies \text{BW} \implies \text{CC} \\ 
    \text{MCT} 
\end{cases}   \]
where AOC is our defining axiom to base all our reults on and giving us the notion that an ordered field contains no holes. We could also take the MCT to be our defining axiom and gives us the notion of least upper bounds by proving NIP. In addition, we could also take NIP to be our starting point but we need to have an extra hypothesis; that is, the Archimedean Property to prove all our results above (This is unavoidable).

It could be possible to assume the Arcimedean property holds, suppose one of the results we have proven is true, and derive the others yet this is sort of limited since \( \Q \) contains a set that is not complete. 

Below is the least of implications we can prove based on which theorem we would like to select asour defining axiom. Hence, we have
\[ \text{NIP} + \text{Archimedean Property} \implies \text{AOC} \] and 
\[ \text{BW} \implies \text{MCT} \implies \text{Archimedean Property}\]



\section{Properties of Infinite Series}

We have learned the convergence of the series \( \sum_{k=1}^{\infty} a_k \) is defined in terms of the sequence \( (s_n)\) where 
\[ \sum_{k=1}^{\infty} a_k = A \text{ means that } \lim s_n = A.\]
We called \( (s_n)\) the \textit{sequence of partial sums} of the series \( \sum_{k=1}^{\infty} a_k\). Just like the \textit{Algebraic Limit Theorem} for sequences, we can also do the same thing for series. 

\begin{tcolorbox}
    \begin{thm}[Algebraic Limit Theorem for Series] 
If \( \sum_{k=1}^{ \infty} a_k = A \) and \( \sum_{k=1}^{ \infty} b_k = B \), then 
\begin{enumerate}
    \item[(i)] \( \sum_{k=1}^{ \infty } ca_k = cA \) for all \( c \in \R \),
    \item[(ii)] \( \sum_{k=1}^{ \infty } (a_k + b_k) = A + B\)
\end{enumerate}
\end{thm}
\end{tcolorbox}

\begin{proof}
Suppose \( \sum_{k=1}^{\infty} a_k = A \) and let \( c \in \R \). Define the sequence of partial sums of \( \sum_{k=1}^{ \infty} ca_k \) as 
\[ t_k = cs_n =  ca_1 + ca_2 + ca_3 + ... + ca_n.\]
By the \textit{Algebraic Limit Theorem}, we know that \( \lim cs_n = cA \). Hence, 
\[ \sum_{k=1}^{\infty} ca_k = cA.\]
To prove the addition rule, suppose \( \sum_{k=1}^{ \infty} b_k = B \). We want to show that 
\[ \sum_{k=1}^{\infty} (a_k + b_k) = A + B.\]
Define the sequence of partial sums for the two series as the following:
\begin{align*}
    t_k &= a_1 + a_2 + ... + a_n, \\
    u_k &= b_1 + b_2 + ... + b_n
\end{align*}
Since \( \sum_{k=1}^{\infty} a_k = A \) and \( \sum_{k=1}^{\infty} b_k = B \), their sequence of partial sums also converges to the same value. Hence, let \( \lim t_k = A \) and \( \lim u_k = B \). By the \textit{Algebraic Limit Theorem}, the sum of these two limits also converges i.e  
\[ \lim ( t_k + u_k ) = \lim t_k + \lim u_k = A + B.\] 
Hence, 
\[ \sum_{k=1}^{\infty} (a_k + b_k) = A + B \]
\end{proof}

We can summarize this theorem by keeping in mind that we can perform distribution over infinite addition and that we can add two infinite series together. 

\begin{tcolorbox}
    \begin{thm}[Cauchy Criterion for Series]
     The series \( \sum_{k=1}^{ \infty} a_k \) converges if and only if, given \( \epsilon > 0\), there exists \( N \in \N \) such that whenever \( n > m \geq N \) it follows that 
     \[ |a_{m+1} + a_{m+2} + ... + a_n| < \epsilon.\]
    \end{thm}
\end{tcolorbox}

\begin{proof}
    Let \(\epsilon > 0 \). We want to show that there exists \( N \in \N \) such that whenever \( n > m \geq N \) it follows that 
    \[|a_{m+1} + a_{m+2} + ... + a_n| < \epsilon.\]
    Suppose \( \sum_{k=1}^{\infty} a_k \) converges. This is true if and only if the sequence of partial sums \( (t_k)\) converges. This is true if and only if \( (s_k)\) is \textit{Cauchy} by the \textit{Cauchy Criterion}. Hence, there exists \( N \in \N \) such that whenever \( n > m \geq N \) 
    \[ |s_n - s_m | < \epsilon.\]
Note that 
\begin{align*}
    |s_n - s_m|&= | \sum_{k=m+1}^{\infty} a_k - \sum_{k=m}^{m} a_k|  \\
               &= |\sum_{k=m+1}^{n}a_k|\\
               &= |a_{m+1} + ... + a_n| < \epsilon
\end{align*}
\end{proof}

This gives us the opportunity to prove some basic facts about series.

\begin{tcolorbox}
\begin{thm}
If the series \( \sum_{k=1}^{\infty} a_k \) converges, then \( (a_k) \to 0\).
\end{thm}
\end{tcolorbox}

\begin{proof}
From the last theorem, we note that for every \( \epsilon > 0 \) such that whenever \(n  \geq m \geq N \), we have  
\[ | s_n - s_m| = \Big| \sum_{k=m+1}^{ \infty} a_k - 0 \Big| < \epsilon\]
implies that \( (a_n) \to 0 \).
\end{proof}

Keep in mind that the converse of this statement is not true! Just because \( (a_k)\) tends to \( 0 \) does not immediately imply that the series converges! 

\begin{tcolorbox}
    \begin{thm}[Comparison Test]
Assume \((a_k)\) and \((b_k)\) are sequences satisfying \( 0 \leq a_k \leq b_k \) for all \( k \in \N \). Then we have 

\begin{enumerate}
    \item[(i)] If \( \sum_{k=1}^{\infty}b_k\) converges, then \( \sum_{k=1}^{\infty} a_k\) converges.
    \item[(ii)] If \( \sum_{k=1}^{\infty} a_k\) diverges, then \( \sum_{k=1}^{\infty} b_k\) diverges.
\end{enumerate}
\end{thm}
\end{tcolorbox}

\begin{proof}
Let us show part (i). Suppose \( \sum_{k=1}^{\infty} b_k \) converges. We want to show that \( \sum_{k=1}^{\infty} a_k\) converges. Let \( \epsilon > 0 \). There exists \( N \in \N \) such that for every \( n > m \geq N \) and the fact that \( a_k \leq b_k\) for all \(k \in \N \) 
\begin{align*}
    \Big|\sum_{k=m+1}^{n} a_k \Big| &\leq \Big|\sum_{k=m+1}^{n}b_k\Big| \\ 
                                    &< \epsilon.
\end{align*}
Hence, \(a_k\) converges as well. 

Note that part (ii) is just the contrapositive of part (i) which is also true. 
\end{proof}

Note that the convergence of sequences and series are relatively immutable when it comes to changes in some finite number of initial terms: that is, the behavior of sequences and series can be found past some choice of \( N \in \N \). In order for the above test to be of any use to us, it is important to have a few examples under our belt i.e any \( p > 1 \) implies that 
\begin{center}
    \( \sum_{n=1}^{\infty} 1/ n^p\) converges if and only if \( p > 1\).    
\end{center}

\begin{ex}
A series is called \textit{geometric} if it is of the form 
\[ \sum_{k=0}^{\infty} ar^k = a + ar + ar^2 + ar^3 + ... ~ .\]
If \( r = 1 \) and \( a \neq  0\), the series diverges. We can use the following algebraic identity, for \( r \neq 1 \), to write the following: 
\[ (1-r)(1 + r + r^2 ... + r^{m-1}) = 1 - r^m\]
which allows us to rewrite the partial sum \( (s_m)\) of the above series to say that 
\[ s_m = a+ ar + ar^2 + ar^3 + ... + a r^{m-1} = \frac{a(1-r^m)}{1-r}\]
where \( s_m = at_{m}\) where 
\[ t_m = 1 + r + r^2 + ... + r^{m-1}\]
is a convergent sequence.
Using the \textit{Algebraic Limit Theorem}, therefore, allows us to say that 
\[ \sum_{k=0}^{\infty} ar^k = \frac{a}{1-r}\]
if and only if \( |r| < 1\).
\end{ex}

The next theorem is a modification of the \textit{Comparison Test} to handle series that contain negative terms.

\begin{tcolorbox}
    \begin{thm}[Absolute Convergence Test]
    If the series \( \sum_{n=1}^{\infty} |a_n| \) converges, then \( \sum_{n=1}^{\infty} a_n\) converges as well.
\end{thm}
\end{tcolorbox}

\begin{proof}
    Suppose \( \sum_{n=1}^{\infty}|a_n|\) converges. We want to show that \( \sum_{n=1}^{\infty} a_n\) converges as well. Let \(\epsilon > 0 \). By the \textit{Cauchy Criterion} for series, there exists \( N \in \N \) such that whenever \( n > m \geq N \), we have 
\begin{align*}
    \Big|\sum_{k=m+1}^{n} a_k \Big|&\leq \sum_{k=m+1}^{n} |a_k| \\
                     &< \epsilon.
\end{align*}
Hence, \( \sum_{n=1}^{\infty} a_n \) converges.
\end{proof}

Note that the converse of the above statement is false as taking the absolute value of the alternating harmonic series 
\[ 1 - \frac{1}{2} + \frac{1}{3} - \frac{1}{4} + \frac{1}{5} - \frac{1}{6} + ... ~  \]
produces the regular harmonic series which \textit{diverges}.

\begin{tcolorbox}
    \begin{thm}[Alternating Series Test]
    Let \((a_n)\) be a sequence satisfying, 
    \begin{enumerate}
        \item[(i)] \( a_1 \geq a_2 \geq a_3 ... \geq a_n \geq a_{n+1} \geq ... \) and
        \item[(ii)] \((a_n) \to 0\).
    \end{enumerate}
    Then, the alternating series \( \sum_{n=1}^{\infty} (-1)^{n+1} a_n\) converges.
\end{thm}
\end{tcolorbox}
% correct this proof
\begin{proof}
See exercise 2.7.1 for proof
\end{proof}

\begin{tcolorbox}
\begin{defn}
If \( \sum_{n=1}^{\infty} |a_n| \) converges, then we say that the original series \( \sum_{n=1}^{\infty} a_n \) \textit{converges absolutely}. If, on the other hand, the series \( \sum_{n=1}^{\infty} a_n \) converges but the series of absolute values \( \sum_{n=1}^{\infty} |a_n|\) does not converges, then we say that the original series \(\sum_{n=1}^{\infty} a_n \) \textit{converges conditionally}.
\end{defn}
\end{tcolorbox}

We can chart a few examples of some \textit{conditionally convergent } series and \textit{absolutely convergent} series.

\begin{itemize}
    \item \(\sum_{n=1}^{\infty} \frac{(-1)^{n+1}}{n} \implies \) \textit{conditionally convergent}
    \item \(\sum_{n=1}^{\infty} \frac{(-1)^{n+1}}{n^2}, \sum_{n=1}^{\infty} \frac{1}{2^n},  \) and \( \sum_{n=1}^{\infty} \frac{(-1)^{n+1}}{2^n} \implies \) \textit{converges absolutely} 
\end{itemize}

This tells us that any convergent series with positive terms must converge absolutely. 

\subsection{Rearrangements}

We can obtain a rearrangement of an infinite series by permuting terms in the sum in some other order. In order for a sum to be a valid rearrangement, all the terms must appear and there should be no repeats.

\begin{tcolorbox}
\begin{defn}
    Let \( \sum_{k=1}^{\infty} a_k\) be a series. A series \( \sum_{k=1}^{\infty} b_k\) is called a \textit{rearrangement} of \(\sum_{k=1}^{\infty} a_k \) if there exists a \textit{bijective} function \(f: \N \to \N \) such that \( b_{f(k)} = a_k \) for all \( k \in \N \).
\end{defn}
\end{tcolorbox}

We can now explain the weird behavior for why the \textit{harmonic series} converges to a different limit when rearranging the terms; that is, it is because the \textit{harmonic series} is a \textit{conditionally convergent} series which leads us to the next theorem. 

\begin{tcolorbox}
\begin{thm}
If a series converges absolutely, then any rearrangement of this series converges to the same limit.
\end{thm}
\end{tcolorbox}

\begin{proof}
Assume \(\sum_{k=1}^{\infty} a_k \) \textit{converges absolutely} to \(A\), and let \( \sum_{k=1}^{\infty} b_k \) be a rearrangement of \( \sum_{k=1}^{\infty} a_k\). 
Let us define the sequence of partial sums of \( \sum_{k=1}^{\infty} a_k\) as 
\[ s_n = \sum_{k=1}^{n}a_k\] and the sequence of partial sums for the rearranged series \( \sum_{n=1}^{\infty}b_n\) as 
\[ t_m = \sum_{k=1}^{m} b_k.\] Since \( \sum_{n=1}^{\infty}a_n\) \textit{converges absolutely}, let \(\epsilon > 0 \) such that there exists \( N_1 \in \N \) such that whenever \( n \geq N \), we have 
\[ |s_n - A | < \frac{\epsilon}{2}\]
as well some \( N_2 \in \N \) such that whenever \( n > m \geq N_2\), we have 
\[ \sum_{k=m+1}^{n} |a_k| < \frac{\epsilon}{2}.\]
All that is left to do is to set a point in the sequence of the rearranged series where our ultimate goal is to have \( |t_m - A | < \epsilon.\) Hence, define 
\[ M = \max \{ f(k): 1 \leq k \leq N \}.\]
Let \( m \geq M \) such that, when using the \textit{triangle inequality}, we get 
\begin{align*}
    |t_m - A | &= |t_m - s_N + s_N - A |  \\
               &\leq |t_m - s_N | + |s_N - A | \\
               &< \frac{\epsilon}{2} + \frac{\epsilon}{2} \\
               &= \epsilon.
\end{align*}
Hence, we have that \( \sum_{n=1}^{\infty}b_n \) converges to \(A\).
\end{proof}

\subsection{Definitions}


\begin{tcolorbox}
    \begin{thm}[Algebraic Limit Theorem for Series] 
If \( \sum_{k=1}^{ \infty} a_k = A \) and \( \sum_{k=1}^{ \infty} b_k = B \), then 
\begin{enumerate}
    \item[(i)] \( \sum_{k=1}^{ \infty } ca_k = cA \) for all \( c \in \R \),
    \item[(ii)] \( \sum_{k=1}^{ \infty } (a_k + b_k) = A + B\)
\end{enumerate}
\end{thm}
\end{tcolorbox}


\begin{tcolorbox}
    \begin{thm}[Cauchy Criterion for Series]
     The series \( \sum_{k=1}^{ \infty} a_k \) converges if and only if, given \( \epsilon > 0\), there exists \( N \in \N \) such that whenever \( n > m \geq N \) it follows that 
     \[ |a_{m+1} + a_{m+2} + ... + a_n| < \epsilon.\]
    \end{thm}
\end{tcolorbox}


\begin{tcolorbox}
\begin{thm}
If the series \(\sum_{k=1}^{\infty} a_k  \) converges, then \( (a_k) \to 0 \).
\end{thm}
\end{tcolorbox}

\begin{tcolorbox}
    \begin{thm}[Comparison Test]
Assume \((a_k)\) and \((b_k)\) are sequences satisfying \( 0 \leq a_k \leq b_k \) for all \( k \in \N \). Then we have 

\begin{enumerate}
    \item[(i)] If \( \sum_{k=1}^{\infty}b_k\) converges, then \( \sum_{k=1}^{\infty} a_k\) converges.
    \item[(ii)] If \( \sum_{k=1}^{\infty} a_k\) diverges, then \( \sum_{k=1}^{\infty} b_k\) diverges.
\end{enumerate}
\end{thm}
\end{tcolorbox}

\begin{tcolorbox}
    \begin{thm}[Absolute Convergence Test]
    If the series \( \sum_{n=1}^{\infty} |a_n| \) converges, then \( \sum_{n=1}^{\infty} a_n\) converges as well.
\end{thm}
\end{tcolorbox}

\begin{tcolorbox}
\begin{defn}
If \( \sum_{n=1}^{\infty} |a_n| \) converges, then we say that the original series \( \sum_{n=1}^{\infty} a_n \) \textit{converges absolutely}. If, on the other hand, the series \( \sum_{n=1}^{\infty} a_n \) converges but the series of absolute values \( \sum_{n=1}^{\infty} |a_n|\) does not converges, then we say that the original series \(\sum_{n=1}^{\infty} a_n \) \textit{converges conditionally}.
\end{defn}
\end{tcolorbox}

\begin{tcolorbox}
\begin{defn}
    Let \( \sum_{k=1}^{\infty} a_k\) be a series. A series \( \sum_{k=1}^{\infty} b_k\) is called a \textit{rearrangement} of \(\sum_{k=1}^{\infty} a_k \) if there exists a \textit{bijective} function \(f: \N \to \N \) such that \( b_{f(k)} = a_k \) for all \( k \in \N \).
\end{defn}
\end{tcolorbox}

\begin{tcolorbox}
\begin{thm}
If a series converges absolutely, then any rearrangement of this series converges to the same limit.
\end{thm}
\end{tcolorbox}

\begin{tcolorbox}
    \begin{thm}[Alternating Series Test]
    Let \((a_n)\) be a sequence satisfying, 
    \begin{enumerate}
        \item[(i)] \( a_1 \geq a_2 \geq a_3 ... \geq a_n \geq a_{n+1} \geq ... \) and
        \item[(ii)] \((a_n) \to 0\).
    \end{enumerate}
    Then, the alternating series \( \sum_{n=1}^{\infty} (-1)^{n+1} a_n\) converges.
\end{thm}
\end{tcolorbox}





\subsection{Exercises}


\subsubsection{Exercise 2.7.1} Proving the \textit{Alternating Series Test} amounts to showing that the sequence of partial sums 
\[ s_n = a_1 - a_2 + a_3 - ... \pm a_n\] converges. (The opening example in Section 2.1 includes a typical illustration of \((s_n)\). Different characterizations of completeness lead to different proofs. 
\begin{enumerate}
    \item[(a)] Prove the \textit{Alternating Series Test} by showing that \( (s_n)\) is a \textit{Cauchy Sequence}. 
        \begin{proof}
            Let \( (a_n)\) be a \textit{decreasing sequence} and suppose \( (a_n) \to 0 \). We want to show that the \textit{Alternating series} \( \sum_{n=1}^{\infty} (-1)^{n+1} a_n\) meets the \textit{Cauchy Criterion}. 

            We first need to show that for every \( n > m\), we have the property
            \[ 0 \leq |a_{m+1} - a_{m+2} + a_{m+3} - ... \pm a_n| \leq |a_{m+1}|\]
            Hence, we proceed by induction on \( k\). Note that 

            \[ \sum_{k=m+1}^{n} (-1)^{k+1}a_k = a_{m+1} - a_{m+2} + a_{m+3} - ... \pm a_n .\]
            Let our base case be \( P(1)\). Then \( a_{m+1} \geq 0\). For \( P(2)\), we have \( a_{m+1} \geq a_{m+2}\) for all \( m \) since \( (a_n)\) is a \textit{decreasing sequence}. Suppose this holds for all \( m \leq k-1   \). We want to show that this holds for \( P(k)\). Since \((a_n)\) is \textit{decreasing}, we have that \( a_{k-1} \geq a_{k}\). Hence, \( a_{k-1} - a_{k} \geq 0 \). Since \( P(k-1)\) holds where 
            \[ 0 \leq a_{m+1} - a_{m+2} + a_{m+3} - ... \pm a_{k-1} \leq a_{m+1}.\]
            But this means that every term leading up to \( a_k \) is bounded by \( a_{m+1}\). Hence, 
            \[ 0 \leq a_{m+1} - a_{m+2} + a_{m+3} - ... \pm a_k \leq a_{m+1}.\]
        
        Let \( \epsilon > 0 \). All is left to show is that 
        \[ \Big|\sum_{k=1}^{n}(-1)^{k+1} a_k \Big| < \epsilon.\]
        Hence, for some \( N \in \N \), let \( n > m \geq N \) and \( (a_n) \to 0 \) such that 
        \begin{align*}
            \Big| \sum_{k=1}^{\infty} (-1)^{k+1} a_k \Big|&\leq |a_{m+1}| \\
                                        &< \epsilon.            
        \end{align*}
        Hence, the series \( \sum_{k=1}^{\infty}(-1)^{k+1} a_k \) meets the \textit{Cauchy Criterion}.
        \end{proof}
    \item[(b)] Supply another proof for this result using the Nested Interval Property. 
        \begin{proof}
            Suppose \( (a_n)\) is \textit{decreasing} sequence and \( (a_n) \to 0 \). Our goal is to show the series \( \sum_{n=1}^{\infty} (-1)^{n+1}a_n\) converges. Since \( (a_n)\) is \textit{decreasing}, we can use the \textit{Nested Interval Property} to construct closed intervals \( I_n = [s_n, s_{n+1}] \) such that the length of these intervals is \( |s_{n} - s_{n+1}| \leq a_n\). The \textit{Nested Interval Property} gurantees the following property that 
            \[ I_1 \subseteq I_2 \subseteq I_3 \subseteq ...  \]
            where \( \bigcap_{n=1}^{\infty} I_n \neq \emptyset\). Hence, \( S \in \R \) can be our candidate limit since \( S \in I_n\) for all \( n \). Let \( \epsilon > 0 \). Since \( (a_n) \to 0 \), there exists \( N \in \N \) such that \( n \geq N \)
            \[ |s_n - S | \leq a_n < \epsilon.\]
        \end{proof}
        Hence, \( (s_n) \to S \). 
    \item[(c)] Consider the subsequences \( (s_{2n})\) and \( (s_{2n+1})\), and show how the \textit{Monotone Convergence Theorem} leads to a third proof for the \textit{Alternating Series Test}. 
        \begin{proof}
        Define the subsequence of partial sums \( (s_{2n})\) as 
        \[ \sum_{k=1}^{n} (-1)^{2k} a_{2k}.\]
        Since \( (a_n)\) is a \textit{decreasing sequence}, we have that \( a_n \geq a_{n+1}\) for all \( n \in \N \). Observe that 
        \begin{align*}
            s_1 &= a_2 \geq 0  \\
            s_2  &= a_2 + a_4  \geq s_1 \\
            s_3 &= a_2 + a_4 + a_6 \geq s_2 \\
                &\vdots \\ 
            s_n &= a_2 + a_4 + a_6 + ... + a_{2n}.
        \end{align*}
        We can see that \( s_{2n}\) is an \textit{increasing sequence}. Also, \(|s_{2n}| < M \) since \( (a_n)\) is a \textit{bounded sequence}. Hence, we can conclude that the subsequence of partial sums \( (s_{2n})\) is converges to some \( S \in \R \). 

        We can show that \( (s_{2n+1})\) converges to \( S \) as well. Since \( s_{2n+1} = s_{2n} + a_{2n+1} \), we can use the \textit{Algebraic Limit Theorem} to say that 
        \begin{align*}
            \lim(s_{2n+1})&= \lim(s_{2n} + a_{2n+1}) \\
                          &= \lim(s_{2n}) + \lim (a_{2n+1}) \\
                          &= S + 0 \\
                          &= S.
        \end{align*}
        Since \( (s_{2n}) \to S \) and \( (s_{2n+1}) \to S \), we have \( (s_n) \to S \) as well. 
        \end{proof}
\end{enumerate}

\subsubsection{Exercise 2.7.4}

\begin{enumerate}
    \item[(a)] Provide the details for the proof of the Comparison Test (Theorem 2.7.4) using the \textit{Cauchy Criterion} for Series.
        \begin{proof}
        Suppose \( (a_k)\) and \( (b_k)\) are sequences such that \( 0 \leq a_n \leq b_n\) for all \( n \in \N \). Assume \( \sum_{n=1}^{\infty} b_n \) converges. Our goal is to show that \( \sum_{k=1}^{\infty} a_k \) converges. Define the sequence of partial sums for \( \sum_{n=1}^{\infty}a_n\) as 
        \[ t_n = \sum_{k=1}^{n}a_k.\]
        Let \(\epsilon > 0 \). Since \( a_k \leq b_k \) and \( \sum_{n=1}^{\infty}b_n\) converges, there exists \( N \in \N \) such that for all \( n > m \geq N \), we have  
        \begin{align*}
             |t_n - t_m |&= \Big|\sum_{k=m+1}^{n} a_k\Big| \\
                         &\leq \Big|\sum_{k=m+1}^{n} b_k \Big| \\
                         &< \epsilon.
        \end{align*}
        Hence, the series \( \sum_{n=1}^{\infty} a_n \) converges. Note that part (ii) is just the contrapositive of part (i). Hence, it is also true.
        \end{proof}
    \item[(b)] Give another proof for the \textit{Comparison Test}, this time using the \textit{Monotone Convergence Theorem}.
        \begin{proof}
        Suppose the series \( \sum_{n=1}^{\infty} b_n \) converges. Our goal is to use the \textit{Monotone Convergence Theorem} to show that \( \sum_{n=1}^{\infty} a_n \) converges i.e our goal is to show that the sequence of partial sums \( t_n = \sum_{k=1}^{n} a_n \) is \textit{bounded} and \textit{monotone}. 

        Since the sequence of partial sums of \( \sum_{n=1}^{\infty} b_n \) are \textit{bounded} and \( 0 \leq a_n \leq b_n \) for all \( n \in \N \), it follows that we have \( |t_n| \leq M  \) as well. 

        Now we want to show that \( (t_n )\) is a \textit{decreasing sequence}. Since \( \sum_{n=1}^{\infty} b_n \) is convergent, we know that \( b_n \to  0 \). Since \( a_n \geq 0 \) and \( (b_n) \to 0 \), the terms \( (t_n)\) must also be \textit{decreasing}. Hence, \( t_{n+1} \leq t_n \) for all \( n \in \N \).
        
        Since \( (t_n)\) is both \textit{decreasing} and \textit{bounded}, it follows that \( \sum_{n=1}^{\infty} a_n \) is a convergent
        \end{proof}
\end{enumerate}

\subsubsection{Exercise 2.7.4} Give an example of each or explain why the request is impossible referencing the proper theorem(s).
\begin{enumerate}
    \item[(a)] Two series \( \sum x_n \) and \( \sum y_n \) that both diverge but where \(\sum x_ny_n \) converges.
        \begin{proof}[Solution]
        Take \(\sum x_n = (-1)^n\) and \( \sum y_n = 1/n\). These two series diverge but \(\sum x_n y_n = (-1)^n / n \) converges.
        \end{proof}
    \item[(b)] A convergent series \( \sum x_n \) and a bounded sequence \( (y_n)\) such that \( \sum x_n y_n \) diverges.
        \begin{proof}[Solution]
        Take the convergent series \( \sum 1 / n^2\) and the bounded sequence \( y_n = \sin(n)\). We have \( \sum x_n y_n = \sum \sin(n)/n^2\) is divergent by the comparison test.
        \end{proof}
    \item[(c)] Two sequences \( (x_n)\) and \((y_n)\) where \(\sum x_n \) and \( \sum (x_n + y_n)\) both converges but \( \sum y_n \) diverges.
        \begin{proof}[Solution]
        This is impossible. By the Algebraic Series Theorem, we cannot have \( \sum (x_n + y_n)\) converge without \( \sum y_n \) converging as well. 
        \end{proof}
    \item[(d)] A sequence \( (x_n)\) satisfying \( 0 \leq x_n \leq 1/n\) where \( \sum (-1)^n x_n\) diverges.
        \begin{proof}[Solution]
        By the comparison test, \( \sum (-1)^n x_n \) diverges.
        \end{proof}
\end{enumerate}

\subsubsection{Exercise 2.7.5} Now that we have proved the basic facts about geometric series, supply a proof for Corollary 2.4.7.
\begin{tcolorbox}
\begin{cor}
The series \( \sum_{n=1}^{\infty} 1/n^p \) converges if and only if \( p > 1 \).
\end{cor}
\end{tcolorbox}
\begin{proof}
    We start with the backwards direction. Suppose \( p > 1 \). Our goal is to show that \( \sum_{n=1}^{\infty} 1/n^p\) converges. Notice that \( b_n = 1/n^p\) where \( b_n \geq 0 \) and \( b_n \)
\textit{decreasing}. By the \textit{Cauchy Condensation Test}, we can prove that 
\[ \sum_{n=0}^{\infty} 2^n b_{2^n} = \sum_{n=0}^{\infty} 2^n \Big( \frac{1}{2^p}\Big)^n.\]
converges. Since \( p > 1 \), we have that 
\[ \sum_{n=0}^{\infty} 2^n \Big( \frac{1}{2^n}\Big)^p = \sum_{n=0}^{\infty} 2^n \Big( \frac{1}{2^n}\Big)^p = \sum_{n=0}^{\infty} 2^n \Big( \frac{1}{2^n}\Big)^{p-1} = \sum_{n=0}^{\infty} \Big(\frac{1}{2^p} \Big)^n.\]
Since \( |r| = |1/2^p| < 1 \), we know that \( \sum_{n=0}^{\infty}2^n b_{2^n}\) is a \textit{Geometric Series}. By the \textit{Cauchy Condensation Test}, we can say that \( \sum_{n=1}^{\infty} b_n \) converges. 

For the forwards direction, since  \(\sum_{n=0}^{\infty} 2^n b_{2^n}\) converges, the only reasonable choice of \( p \) is when \( p > 1 \) or else it is \textit{Harmonic Series} which diverges.
\end{proof}

\subsubsection{Exercise 2.7.6}
Let's say that a series \textit{subverges} if the sequence of partial sums contains a subsequence that converges. Consider this (invented) definition for a  moment, and then decide which of the following statements are valid propositions about \textit{subvergent} series: 
\begin{enumerate}
    \item[(a)] If \( (a_n)\) is \textit{bounded}, then \( \sum a_n \) \textit{subverges}.
        \begin{proof}[Solution]
        This is a valid proposition since the sequence of partial sums for \( \sum_{n=1}^{\infty}a_n\) are bounded which implies that the sequence of partial sums contains a subsequence partial sums that is convergent. Hence, we can say that \( \sum a_n \) is a \textit{subvergent} series. 
        \end{proof}
    \item[(b)] All convergent series are \textit{subvergent}. 
        \begin{proof}[Solution]
        This is valid since the sequence of partial sums for a convergent series converges and hence all of the possible subsequence of partial sums for the series converges to the same limit. 
        \end{proof}
    \item[(c)] If \( \sum |a_n| \) \textit{subverges}, then \(\sum a_n \) \textit{subverges} as well. 
        \begin{proof}[Solution]
        This is not valid. 
        \end{proof}
    \item[(d)] If \( \sum a_n \) \textit{subverges}, then \( (a_n)\) has a convergent subsequence. 
        \begin{proof}[Solution]
        This is not valid.
        \end{proof}
\end{enumerate}

\subsubsection{Exercise 2.7.7}
\begin{enumerate}
    \item[(a)] Show that if \( a_n > 0 \) and \( \lim (na_n) = l  \) with \( l \neq 0 \), then the series \( \sum a_n \) diverges.
        \begin{proof}
        Suppose for sake of contradiction that \( \sum a_n  \) converges. Hence, \( (a_n) \to 0 \). This means that \( \lim (na_n) = 0 \) but this contradicts our assumption that \( \lim (na_n) = l \neq 0 \). Hence, the series \( \sum a_n \) must diverge. 
        \end{proof}
        Another why is to use the limit assumption directly. 
        \begin{proof}
        Suppose \( a_n > 0 \) and \( \lim (na_n) = l \). We want to show that \( \sum a_n  \) diverges. Since \( \lim (na_n) = l \neq 0 \), let \( \epsilon = 1  \) such that there exists \( N \in \N \) such that whenever \( n \geq N \) for all \( n \), we have 
        \[ |na_n - l | < 1 \iff a_n < \frac{1+ l }{n }. \]
        This implies that 
        \[ \sum_{n=1}^{\infty } a_n < \sum_{n=1}^{\infty} \frac{1+l}{n}.\]
        Note that \( \sum \frac{1+l }{n} \) is not a \textit{p-series} since \( n^p\) where \(p=1\). Hence, the series \( \sum \frac{1+l}{n}\) diverges. Hence, we have that \( \sum a_n \) is also a divergent series by the comparison test. 
        \end{proof}
    \item[(b)] Assume \( a_n > 0 \) and \( \lim (n^2 a_n )\) exists. Show that \( \sum a_n \) converges. 
        \begin{proof}
            Suppose \( a_n > 0 \) and \( \lim (n^2 a_n )\) exists. Suppose \( \lim (n^2 a_n  ) = L \) for some \( L \in \R \). Let \( \epsilon = 1 \), there exists \(  N \in \N \) such that whenever \( n \geq N \), we have 
            \[ |n^2a_n - L | < \epsilon.\]
        Hence, we have
        \[ n^2 a_n - L < 1 \iff a_n < \frac{1+ l }{n^2 } \tag{1}\]
        Our goal is to show via \textit{comparison test} that the series \( \sum a_n \) converges.
        From (1), we have 
        \[\sum_{n=1}^{\infty} a_n < \sum_{n=1}^{\infty} \frac{1 + l }{n^2 }.  \]
        Observe that the series \( \sum \frac{1+l}{n^2} \) is a \textit{p-series} test which converges. Hence, the series \( \sum a_n \) converges by the \textit{Comparison test}.
        \end{proof}
\end{enumerate}

\subsubsection{Exercise 2.7.8}

Consider each of the following propositions. Provide short proofs for those that are true and counterexamples for any that are not.

\begin{enumerate}
    \item[(a)] If \( \sum a_n \) \textit{converges absolutely}, then \( \sum a_n^2\) also \textit{converges absolutely}.
        \begin{proof}
        Since \( \sum a_n \) converges absolutely, then we have the series \( \sum |a_n| \) converges. In order for \( \sum a_n^2 \) to converge absolutely, we need to show that \( \sum | a_n^2 |\) converges. Furthermore, \( (a_n) \) is a \textit{bounded} sequence. Hence, there exists \( M > 0 \) such that \( |a_n| \leq M \). Since there exists \( N \in \N \), for any \( n \geq N \), we can write
        \begin{align*}
            \sum |a_n^2|&= \sum | a_n \cdot a_n |  \\
                        &= \sum |a_n | \cdot |a_n | \\
                        &\leq \sum M \cdot |a_n | \\
                        &= M \sum |a_n| \\
        \end{align*}
        We know by the \textit{Algebraic Limit Theorem} for series that \( M \sum |a_n|\) converges. Hence, the series \( \sum a_n^2 \) converges absolutely by the \textit{Comparison Test}.
        \end{proof}
    \item[(b)] If \( \sum a_n \) converges and \( (b_n)\) converges, then \( \sum a_n b_n \) converges. 
        \begin{proof}
        Since \( (b_n)\) converges, we have that \( (b_n)\) is also \textit{bounded}. Hence, there exists \( M > 0 \) such that for all \( n \) we have \( b_n \leq M \). Hence, we have 
        \[ \sum a_n b_n \leq M \sum a_n. \]
        By the \textit{Algebraic Limit Theorem} for series, we have that \( M \sum a_n \) converges. Since \( a_n b_n \leq Ma_n \), we have that the series \( \sum a_n b_n \) also converges by the \textit{Comparison test}.
        \end{proof}
    \item[(c)] If \( \sum a_n \) \textit{converges conditionally}, then \(\sum n^2 a_n \) diverges. \begin{proof}[Solution]
   This is false. Consider the series \( \sum \frac{(-1)^n}{n^2}\) which \textit{converges conditionally} but note that \( \sum n^2 \frac{(-1)^n}{n^2} = \sum (-1)^n\) diverges. 
\end{proof}
\end{enumerate}

\subsubsection{Exercise 2.7.9 (Ratio Test).}  
Given a series \( \sum_{ n=1}^{\infty} a_n \) with \( a_n \neq 0 \), the \textit{Ratio Test} states that if \( (a_n)\) satisfies 
\[ \lim \Big| \frac{a_{n+1}}{a_n}\Big| = r < 1, \]
then the series converges absolutely.

\begin{enumerate}
    \item[(a)] Let \( r' \) satisfy \( r < r' < 1 \). Explain why there exists an \( N \) such that \( n \geq N \) implies \( | a_{n+1}| \leq |a_n|r'\).
        \begin{proof}
            There exists \( N \in \N \) such that \( n \geq N \) because \( \lim  | \frac{a_{n+1}}{a_n}| = r \). This means that \( | \frac{a_{n+1}}{a_n}|\) is \textit{bounded}. Hence, we have that \( | \frac{a_{n+1}}{a_n}| \leq r' \) which means that \( |a_{n+1}| \leq r' |a_n|\).
        \end{proof}
    \item[(b)] Why does \( |a_{N}| \sum (r')^n \) converge?
        \begin{proof}
        The series \( |a_N | \sum (r')^n \) converges because \( | r' | < 1 \) which means that \( |a_N| \sum (r')^n \) is a \textit{geometric series} which converges.  
        \end{proof}
    \item[(c)] Now, show that \( \sum |a_n|\) converges, and conclude that \( \sum a_n \) converges. 
        \begin{proof}
        Consider the series \( \sum |a_n|\) and the fact that
        \[ \sum |a_n| \leq |a_N| \sum (r')^n \]
        for all \( n \geq N \). Since the right hand series is \textit{geometric} which converges, we can conclude that \(\sum | a_n|\) also converges by the comparison test. Hence, the series \( \sum a_n \) converges \textit{absolutely} and thus the series \( \sum a_n \) converges.  



        \end{proof}
\end{enumerate}










% !TEX root =  ../../../main.tex 


\section{Double Summations and Products}

We discovered in an earlier section that given any doubly indexed array of real numbers \( \{  a_{ij} : i,j \in \N  \}\), it can be an ambiguous task to define 
\[ \sum_{i,j = 1}^{\infty} a_{ij}. \tag{1}\]
We also observed that performing \textit{iterated summations}can lead to different summations. Of course, this can be avoided completely if we were to define the partial sum of (1) in the following way 
\[ s_{mn} =  \sum_{i=1}^{m} \sum_{j=1}^{n} a_{ij}\]
for \( m, n \in \N \). In order for the sum of (1) to converge we have to have the following hold:
\[ \sum_{i,j = 1 }^{\infty} a_{ij} = \lim_{n \to \infty} s_{mn}\]

\subsubsection{Exercise 2.8.1} Using the particular array \( (a_{ij})\) from Section 2.1, compute \( \lim_{n \to \infty} s_{mn}\). How does this value compare to the two iterated values for the sum already computed?  

The double summation from section 2.1 is \( a_{ij} = \frac{1}{2^{j-i}}\) where \( \{ a_{ij} : i, j \in \N  \}\) if \( j > i \), \( a_{ij} = -1 \) if \( j = i \), and \( a_{ij} = 0 \) if \( j < i \). 
\begin{proof}
    To find \( \sum_{i,j = 1}^{\infty} a_{ij} = \lim_{n \to \infty} s_{mn}\), we first need to define the sequence of partial sums. We can fix \( j \) (the rows of the matrix) and define the sequence of partial sums for the series \( \sum_{i,j = 1 }^{\infty} a_{ij}\) as 
    \[ s_n = \sum_{k=1}^{n} \Big( \frac{1}{2^{n-1}}\Big) = -2 + \frac{1}{2^{n-1}} \]
    which taking the limit leads to 
    \[ \lim_{n \to \infty} \Big( -2 + \frac{1}{2^{n-1}} \Big) = -2.\]

\end{proof}

The issue of rearrangements to an infinite series arises due to commutativity of addition in an infinite context. It was found that having an absolutely converging infinite series fixes this problem. 

\subsubsection{Exercise 2.8.2}

Show that if the iterated series 
\[ \sum_{i=1}^{\infty} \sum_{j=1}^{\infty} |a_{ij}|\]
converges (meaning that for each fixed \( i \in \N \) the series \( \sum_{j=1}^{\infty} |a_{ij}|\) converges to some \( b_i \in \R \), and the series \( \sum_{i=1}^{\infty} b_i \) converges as well), then the iterated series 
\[ \sum_{i=1}^{\infty} \sum_{j=1}^{\infty} a_{ij}\]
converges.

\begin{proof}
Suppose the iterated series 
\[ \sum_{i=1}^{\infty} \sum_{j=1}^{\infty} |a_{ij}|  \tag{1}\]
converges. This means that the (1) meets the \textit{Cauchy Criterion}. Let \(\epsilon > 0 \). This implies that there exists \( N \in \N \) such that for every \( n > m  \geq N \), we have that 
\[ \sum_{i= p + 1}^{m} \sum_{j= \ell + 1}^{n} |a_{ij}| < \epsilon.\]
Consider \( \Big| \sum_{i=p+1}^{\infty} \sum_{j = \ell + 1 }^{\infty} a_{ij} \Big|\). Using the \textit{Triangle Inequality}, we find that  
j\begin{align*}
    \Big| \sum_{i=p+1}^{\infty} \sum_{j = \ell + 1 }^{\infty} a_{ij} \Big|&\leq \sum_{i=p+1}^{\infty} \Big|\sum_{ j = \ell + 1 }^{\infty} a_{ij} \Big|   \\
                                                                          &\leq \sum_{i = p + 1}^{\infty} \sum_{j = \ell + 1 }^{\infty} |a_{ij}| \\
                                                                          &< \epsilon.
\end{align*}
Since (1) meets the \textit{Cauchy Criterion} for series, we know that (1) converges as well. 
\end{proof}

Another proof using the Comparison Test goes something like this

\begin{proof}
    Suppose the iterated series 
    \[ \sum_{i=1}^{\infty} \sum_{j=1}^{\infty} |a_{ij}| \]
    converges. This means that for each \( i \in \N \) the infinite series 
    \[ \sum_{j=1}^{\infty} a_{ij} = r_i \] for some \( r_i \in \R \). Hence, we have the infinite series 
    \[ \sum_{i=1}^{\infty} r_i. \tag{1} \]
    Our goal is to show that (1) converges. Suppose we look at the terms 
    \[ |r_i| = \Big| \sum_{j=1}^{\infty} a_{ij}\Big|.\]
    Note by the \textit{Triangle Inequality} that 
    \[ \sum_{i=1}^{\infty} |r_i| \leq \sum_{i=1}^{\infty} \sum_{j=1}^{\infty} |a_{ij}|.  \]
    by assumption the infinite series to the right converges. Hence, the series to the left must also converge by the \textit{Comparison Test}. Since \( \sum |r_i|\) converges, then the series 
    \( \sum r_i \) converges by the \textit{ Absolute Convergence Test }. 
\end{proof}

\begin{tcolorbox}
\begin{thm}
    Let \( \{ a_{ij} : i,j \in \N  \}\) be a doubly indexed array of real numbers. If 
    \[ \sum_{i=1}^{\infty} \sum_{j=1}^{\infty} |a_{ij}|\]
    converges, then both \( \sum_{i=1}^{\infty} \sum_{ j=1 }^{ \infty  }  a_{ij}\) and \( \sum_{j=1}^{\infty} \sum_{i=1}^{\infty} a_{ij} \) converge to the same value. Moreover, we have that 
    \[ \lim_{n \to \infty} s_{nn} = \sum_{i=1}^{\infty} \sum_{j=1}^{\infty} a_{ij} = \sum_{i=1}^{\infty} \sum_{j=1}^{\infty} a_{ij},\]
    where \( s_{nn} = \sum_{i=1}^{n} \sum_{j=1}^{n} a_{ij}\). 
\end{thm}
\end{tcolorbox}

\begin{proof}
    In the same way that we defined the rectangular partial sums \( s_{mn}\) above in equation (1)
    , define 
    \[ t_{mn} =  \sum_{ i=1 }^{ m } \sum_{ j=1 }^{ n }  | a_{ij} |.\] 

\end{proof}

\subsubsection{Exercise 2.8.3}

\begin{enumerate}
    \item[(a)] Prove that \( (t_{nn})\) converges. 
        \begin{proof}
            From our definition of \( t_{nn}\) above we have 
            \[ t_{nn} = \sum_{ i=1 }^{ n } \sum_{ j=1 }^{ n } | a_{ij} |.\] 
            We want to show for all \(\epsilon > 0 \), there exists \( N \in \N \) such that for all \( n \geq  N \), we have that  \( |t_{nn} - L  | < \epsilon \). By assumption, we know that 
            \[  \sum_{ i=1 }^{ \infty  } \sum_{ j=1 }^{ \infty  } | a_{ij} |  \tag{1}\] 
            converges absolutely which implies that 
            \[  \sum_{ i=1 }^{ \infty  } \sum_{ j=1 }^{ \infty   } a_{ij} \] 
            converges. Note that \( t_{nn} = \sum_{ i=1 }^{ \infty   } \sum_{ j=1 }^{ \infty  } a_{ij}  \to s_n = \sum_{ i=1 }^{ \infty  } r_i \) for some \( r_i \in \R  \). Furthermore, we have \( s_n \to L  \) since (1) converges. Let \( \epsilon  > 0  \). Then there exists \(  N \in \N  \) such that for any \(  n \geq N  \), we have that 
            \begin{align*}
                | t_{nn} - L  | &= | t_{nn} -s_n + s_n - L |  \\
                                &\leq  | t_{nn} - s_n  | + | s_n - L  | \\
                                &< \frac{ \epsilon  }{ 2 } + \frac{ \epsilon  }{ 2 } \\
                                &= \epsilon.
            \end{align*}
            Hence, the sequence of partial sums \(  (t_{nn}) \) converges. 
        \end{proof}
    \item[(b)] Now, use the fact that \( (t_{nn})\) is a Cauchy sequence to argue that \( (s_{nn})\) converges. In order to prove the theorem, we must show that the two iterated sums converge to this same limit. We will first show that 
        \[ S = \sum_{ i=1 }^{ \infty  } \sum_{ j=1 }^{ \infty  } a_{ij}, \] 
        Because \( \{ t_{mn} : m,n \in \N  \} \) is bounded above, we can let 
        \[ B = \sup \{ t_{mn}: m,n \in \N  \}.\] 
        \begin{proof}
        Suppose \( (t_{nn}) \) is a Cauchy Sequence. Then for some \( N \in \N \) we have that for any \( n \geq m > N  \)
        \[  | t_{nn} - t_{mm} | < \epsilon.  \]
        We can rewrite this in the following way to say that 
        \[  | \sum_{ n,m } t_{ij} | < \epsilon.    \]
        Our goal is to show that 
        \[  | s_{nn} - s_{mm} | < \epsilon.   \]
        Hence, for any \( n \geq  m > N  \), we have that 
        \begin{align*}
            | s_{nn} - s_{mm } | &\leq | t_{nn} - t_{m m }|  \\
                                 &= \Big| \sum_{ n,m } t_{ij} \Big|  \\
                                 &< \epsilon.
        \end{align*}
        Hence, \( (s_{nn}) \) converges.

        \end{proof}

    Now, use the fact that \( (t_{nn})\) is a Cauchy sequence to argue that \( (s_{nn})\) converges. In order to prove the theorem, we must show that the two iterated sums converge to this same limit. We will first show that 
        \[ S = \sum_{ i=1 }^{ \infty  } \sum_{ j=1 }^{ \infty  } a_{ij}, \] 
        Because \( \{ t_{mn} : m,n \in \N  \} \) is bounded above, we can let 
        \[ B = \sup \{ t_{mn}: m,n \in \N  \}.\] 
        \subsubsection{Exercise 2.8.4}     
        \begin{enumerate}
            \item[(a)] Let \( \epsilon > 0  \) be arbitrary and argue that there exists an \(  N_1 \in \N  \) such that \( m,n \geq  N_1 \) implies \( B - \frac{ \epsilon  }{ 2 } < t_{mn} \leq  B.\)
                \begin{proof}
                    Since \( (t_{mn}) \) bounded, we can say that \( t_{mn} \leq B \). Since the set 
                    \[  \{ t_{mn} : m,n \in \N \}  \]
                    is bounded above and non-empty, we also have that 
                    \( B = \sup \{ t_{mn}: m,n \in \N  \}  \) exists. Hence, for any \( \epsilon > 0  \), we have that \( B - \frac{ \epsilon  }{ 2 }  \) is not an upper bound. Hence, there exists some \( t_{n_0 m_0} \) such that \( B - \frac{ \epsilon  }{ 2 } < t_{m_0 n_0} \leq t_{mn}\). Furthermore, there exists \( N_1 \in \N  \) such that for any \(  n \geq m > N_1  \) since \( (t_{mn}) \) converges. Hence,  we must have that 
                    \( B - \frac{ \epsilon  }{ 2 } < t_{mn} \leq B \)
                \end{proof}
            \item[(b)] Now, show that there exists an \( N  \) such that 
            \[ | s_{mn} - S  | < \epsilon \]
            for all \( m,n \geq N \).
            \begin{proof}
                Consider \(  | s_{mn} - S  | < \epsilon \). Since \(  (s_{nn}) \to S  \), let \( \epsilon > 0  \) such that for some \( N_2 \in \N  \) we have \( n \geq m > N_2  \), we have
                \[ | s_{nn} - S  | < \frac{ \epsilon  }{ 2 } . \]
                Since \( (s_{nn}) \) meets the Cauchy Criterion, we have that there exists \( N_2 \in \N  \) such that for any \( n \geq m > N  \), we have 
                \[ | s_{nn} - s_{mn} | < \frac{ \epsilon  }{ 2 }.   \]
                Hence, observe that for any \( n \geq m > N = \max \{ N_1, N_2 \}  \), we have
                \begin{align*}
                    | s_{mn} - S  | &= | s_{mn} - s_{nn} + s_{nn} - S  |  \\
                                    &\leq  | s_{mn} - s_{nn} | + | s_{nn} - S  | \\ 
                                    &< \frac{ \epsilon  }{ 2 } + \frac{ \epsilon  }{ 2 } \\
                                    &= \epsilon.
                \end{align*}
                Hence, we have that \( (s_{mn} ) \to S \). 
            \end{proof}
        \end{enumerate}

        Our hypothesis guarantees that for each fixed row \( i \), the series \( \sum_{ j=1 }^{ \infty  } a_{ij} \) converges absolutely to some real number \( r_i  \). 

        \subsubsection{Exercise 2.8.5}
        \begin{enumerate}
            \item[(a)] Show that for all \( m \geq N \) 
                \[ | (r_1 + r_2 + ... + r_m) - S  | \leq \epsilon. \]
                Conclude that the iterated sum \( \sum_{ i=1 }^{ \infty  } \sum_{ j=1 }^{ \infty  } a_{ij} \) converges to \( S \). 
        \end{enumerate}

\end{enumerate}








\chapter{Basic Topology of The Real Numbers}


\section{Discussion: The Cantor Set}

The following construction demonstrates that \( \R  \) is an uncountable set. Let \( C_0  \) be the closed interval \( [0,1] \), and define \( C_1 \) to be the set that results when we remove an open set in the middle third; that is, 
\[ C_1 = C_0 \setminus \Big( \frac{ 1 }{ 3 } , \frac{ 2 }{ 3 }  \Big) = \Big[ 0, \frac{ 1 }{ 3 } \Big] \cup \Big[\frac{ 2 }{ 3 } , 1 \Big]. \]
We can construct the next iteration \( C_2 \) in a similar way above of each of the two sets unioned above. Hence, we have 
\[ C_2 = \Big(  \Big[ 0, \frac{ 1  }{ 9 } \Big] \cup \Big[ \frac{ 2 }{ 9 } , \frac{ 1 }{ 3 } \Big]   \Big) \cup \Big( \Big[ \frac{ 2 }{ 3 } , \frac{ 7 }{ 9 } \Big] \cup \Big[ \frac{ 8 }{ 9 } , 1\Big]\Big)\]
or
\[ C_n = [0,1] \setminus \Big[ \Big(  \frac{ 1 }{ 3 } , \frac{ 2 }{ 3 } \Big) \cup \Big( \frac{ 1 }{ 9 } , \frac{ 2 }{ 9 }  \Big) \cup \Big( \frac{ 7 }{ 8 } , \frac{ 8 }{ 9 }  \Big) \cup \dots   \Big]  \]
If we continue this process inductively, then for each \(  n \in \N  \), we get sets \( C_n \) consisting of \( 2^n \) closed intervals with each having a length of \( 1/3^n \). The Cantor set \( C \) is just the intersection of an infinite number of \( C_n \); that is, 
\[ C = \bigcup_{ n=0 }^{ \infty  } C_n. \]





\section{Open and Closed Sets}

Recall that given any \( \epsilon >0 \), the \( \epsilon-\)neighborhood of \( a \in \R  \) is the set 
\[ V_{\epsilon } = \{ x \in \R : | x - a | < \epsilon  \}.\]
In other words, we have an open interval \( (a -\epsilon , a + \epsilon ) \) or \( a - \epsilon < x < a + \epsilon  \) centered at \( a \) with radius \( \epsilon  \). 

\begin{tcolorbox}
\begin{defn}
    A set \( A \subseteq \R  \) is \textit{open} if for all points \( a \in A \) there exists an \( \epsilon - \)neighborhood \(  V_{\epsilon }(a) \subseteq A \). 
\end{defn}
\end{tcolorbox}

\begin{ex}
\begin{enumerate}
    \item[(i)] The set \( \R  \) is an \textit{open} set because for any \( a \in \R  \), we can pick a \( \epsilon- \)neighborhood \( V_{\epsilon }(a) \) such that \( V_{\epsilon } (a) \subseteq \R \). 
    \item[(ii)] The empty set \( \emptyset \) is an open subset of the real line. This statement is vacuously true due to the definition of an open \textit{set} i.e this set has no interior points to consider so it is true by default. 
    \item[(iii)] Take any \( c,d \in \R  \) and create an open interval as such where 
        \[  (c,d) = \{ x \in \R : c < x < d \}.\] To see why \( (c,d) \) is an \textit{open} set, let \( x \in (c,d) \) be an arbitrary point. Let \(  \epsilon = \min\{ x - c, d -x  \}  \), then we can construct the following \( \epsilon-\)neighborhood where
        \[  V_{\epsilon} = \{ x' \in \R : | x' - x  | < \epsilon \}.  \]
\end{enumerate}
\end{ex}

\begin{tcolorbox}
\begin{thm}
\begin{enumerate}
    \item[(i)] The union of an arbitrary collection open sets is open.
    \item[(ii)] The intersection of a finite collection of open sets is open. 
\end{enumerate}
\end{thm}
\end{tcolorbox}

\begin{proof}
    To prove (i), define \( \{ O_{\lambda} : \lambda \in A \}  \) be a collection of open sets and let \( O = \bigcup_{ \lambda \in A  } O_{\lambda} \). Let \( a  \) be an arbitrary element of \( O \). In order to show that \( O \) is \textit{open}, we need to show that \( V_{\epsilon }(a) \subseteq O \) where \( V_{\epsilon }(a) \) is the \( \epsilon - \)neighborhood. Let \(  a \in O_{\lambda} \) be an arbitrary element. Since we have a collection of open sets 
    \[ \{ O_{\lambda}  : \lambda \in A \}  \]
    we can create a \( \epsilon - \)neighborhood around \( a \in O_{\lambda}  \) for some \( \lambda \in A  \) such that \( V_{\epsilon }(a) \subseteq O_{\lambda}\). But note that \( O_{\lambda} \subseteq O \). Hence, we have that \( V_{\epsilon } (a) \subseteq O = \bigcup_{ \lambda \in A  } O_{\lambda}\). Hence, \( O \) is an \textit{open} set.
    
    To prove (ii), suppose \( O = \bigcap_{ i =1  }^{ N } O_{i}  \). Suppose \( a \in O_{i} \) for all \( 1 \leq  i \leq N \) where \( O_{i} \) is a collection of open sets. Hence, there exists an \( \epsilon- \)neighborhood for every \( O_i  \). We need only one value of \( \epsilon  \) to make this work so define \( \epsilon = \min \{ \epsilon_1, \epsilon_2, \epsilon_3, ... \epsilon_N \}  \).  This means that \[ V_{\epsilon_i} (a) \subseteq V_{\epsilon}(a) \subseteq O_{i} \subseteq O  \] 
Hence, we have 
\[ V_{\epsilon}(a) \subseteq \bigcap_{ i=1 }^{ N } O_i. \] 
\end{proof}

\subsection{Closed Sets} 

\begin{tcolorbox}
\begin{defn}
    A point \( x \) is a \textit{limit point} of a set \( A  \) if every \( \epsilon- \)neighborhood \( V_{\epsilon }(x) \) of \( x \) intersects the set \( A  \) at some point other than \( x \). 
\end{defn}
\end{tcolorbox}
In other words, we have the following intersection 
\[ x \notin V_{\epsilon }(x) \cap A.  \]
This is another way of saying that a sequence approaches of values approaches the limit point \( x \) where \( V_{\epsilon }(x) \) can be thought of as neighborhoods "clustering" around the point \( x \). 

\begin{tcolorbox}
\begin{thm}
A point \( x \) is a limit point of a set \( A \) if and only of \( x = \lim a_n \) for some sequence \( (a_n) \) contained in A satisfying \( a_n \neq x  \) for all \( n \in \N  \). 
\end{thm}
\end{tcolorbox}

\begin{proof}
    \( (\Rightarrow) \) Let \( V_{\epsilon }(x) \) be an \( \epsilon-\)neighborhood around \( x \). We want to show that \( \lim a_n = x  \) for some sequence \( (a_n) \) contained in \( A  \) satisfying \( a_n \neq x  \) for all \( n \in \N  \). By definition, 
    \[ V_{\epsilon}(x) = \{ x \in \R : | a_n - x  | < \epsilon  \}.  \]
    Let \( \epsilon = \frac{ 1 }{ n }  \). Since \( x  \) is a \textit{limit point}, for each \( n \in \N  \), we can pick any point 
    \[ a_n \in V_{1/n}(x) \cap A. \]Then we have 
    \[  | a_n - x | < \frac{ 1 }{ n }  \]
    which is equivalent to 
    \[ x - \frac{ 1 }{ n } < a_n < x + \frac{ 1 }{ n }. \] 
    By the Algebraic limit theorem and Squeeze Theorem, we have that \( (a_n) \to x  \) where \( a_n \neq x  \) for all \( n \in \N  \). 

    \( (\Leftarrow) \) Suppose \( x = \lim a_n \) for some sequence \( (a_n) \) contained in \( A \) satisfying \( a_n \neq x  \) for all \( n \in \N  \). We want to show the converse. Let \( \epsilon > 0  \). Then By definition of \( \lim a_n = x  \), there exists \( N \in \N  \) such that for any \( n \geq N \), we have  
    \[  | a_n - x  | < \epsilon.  \]
    But this is also the definition of an \( \epsilon-\)neighborhood. Hence, \( a_n \neq x  \) for all \( n \in \N  \) and \( x \in A  \) is a limit point implies 
    \[ V_{\epsilon }(x) \cap A \]
    for all \( \epsilon- \)neighborhoods. 
\end{proof}

Keep in mind that \( a \in A  \) means that there is a sequence in \( A  \) such that \( a_n = {a,a,a, ...} \) which is uninteresting for the most part. We can distinguish \textit{limit points} from \textit{isolated points}.  

\begin{tcolorbox}
\begin{defn}
A point \( a \in A  \) is an \textit{isolated point} of \( A \) if it is not a \textit{limit point} of \( A \).  
\end{defn}
\end{tcolorbox}

Remember that an isolated point is always in the set \( A  \), but a limit point can be sometimes be outside of the set \( A \). An example of this is the endpoint of an open interval. A sequence can approach the endpoint where \( a_n \neq x  \) for all \( n \in \N  \) but \( x  \) is not in the set.  

\begin{tcolorbox}
\begin{defn}
A set \( F \subseteq \R  \) is \textit{closed} if it contains its limit points. 
\end{defn}
\end{tcolorbox}

In other words, can say that a set \( A  \) is closed if sequences contained in \(  A  \) converge to their limits that are within the set \( A  \). 

\begin{tcolorbox}
\begin{thm}
A set \( F \subseteq \R \) is closed if and only if every Cauchy sequence contained in \( F \) has a limit that is also an element of \( F \). 
\end{thm}
\end{tcolorbox}

\begin{proof}
    Suppose \( F \subseteq \R  \) is closed. Let \( x \in F \) be a limit point. Let \( (x_n) \) be a Cauchy sequence contained in \( F  \). By the Cauchy Criterion, \( (x_n) \) converges to \( x \in F  \). 
\end{proof}

\begin{ex}
\begin{enumerate}
    \item[(i)] Consider the set 
        \[ A = \Big\{ \frac{ 1 }{ n }  : n \in \N \Big\}.   \]
        Let's show that each point of \( A  \) is isolated. We can show that each point of \( A  \) is isolated. Given \( \frac{ 1 }{ n }  \in A \). Choose \( \epsilon = \frac{ 1 }{ n }  - \frac{ 1 }{ (n+1) }. \) Then, 
        \[ V_{\epsilon } (1/n) \cap A = \Big\{ \frac{ 1 }{ n } \Big\}. \]
        It follows from Definition 3.2.4 that \( \frac{ 1 }{ n }  \) is not a limit point and so is isolated. Although all of the points of \( A  \) are isolated, the set \( A  \) does have only one limit point 0. The reason for this is can be explained by the very definition of \( A  \) where \( 0 \notin A  \). Since the limit of \( A \) is not contained in \( A \), we can say that \( A  \) is not closed. The set \( F = A \cup \{ 0 \}  \) is an example of a closed set and is called the closure of \( A \).   
    \item[(ii)] Let's prove that a closed interval 
        \[ [c,d] = \{ x \in \R : c \leq c \leq d \}  \]
        is a closed set using Definition 3.2.7. If \( x \) is a limit point of \( [c,d] \), then by Theorem 3.2.5 there exists \( (x_n) \subseteq [c,d] \) with \( (x_n) \to x  \). Since \( (x_n) \to x  \), we can use the Order Limit Theorem to say that 
        \[ c \leq x_n \leq d \iff c \leq x \leq d. \]
        This means \( x \in [c,d] \) which proves that \( [c,d] \) is a closed set. 
    \item[(iii)] Consider the set \( \Q \subseteq \R  \) of rational numbers. An interesting property of \( \Q \) is that all of its limit points is actually all of \( \R  \). To see why this is so, let us have \( y \in \R  \) be arbitrary and construct \( V_{\epsilon }(y) \) such that we have the open set \( (y- \epsilon , y + \epsilon ) \). Since \( \Q  \) is dense in \( \R \), there exists \( x \in \Q \) where \( x \neq y \) such that \(x \in (y-\epsilon , y+\epsilon ) \). Hence, \( y  \) is a limit point of \( \Q \).  
\end{enumerate}
\end{ex}

We can actually restate the Density Property from the first chapter by saying the following:

\begin{tcolorbox}
\begin{thm}
For every \( y \in \R  \), there exists a sequence of rational numbers that converges to \( y \). 
\end{thm}
\end{tcolorbox}

\begin{proof}
    Let \( y \in \R  \) and let \( \epsilon = \frac{ 1 }{ n }  \). Create the following \( \epsilon- \)neighborhood \( (y-\frac{ 1 }{ n }, y + \frac{ 1 }{ n } ) \). Since the end points of this \( \epsilon- \)neighborhood are real numbers, we can find a sequence of rational numbers \( (x_n) \subseteq (y - \epsilon , y + \epsilon  )\) by the Density of \( \Q \) in \( \R \) such that 
    \[ y - \frac{ 1 }{ n } < x_n < y + \frac{ 1 }{ n }. \]
    By the Squeeze Theorem, we can write that \( (x_n) \to y \) where \( x_n \neq y \) for all \( n \in \N  \). 
\end{proof}

\subsection{Closure}

\begin{tcolorbox}
\begin{defn}
    Given a set \( A \subseteq \R  \), let \( L \) be the set of all limit points of \( A \). The closure of \( A \) is defined to be the \( \bar{A} = A \cup L \). 
\end{defn}
\end{tcolorbox}

\begin{ex}
\begin{enumerate}
    \item[(i)] Consider \( A = \{ 1/n : n \in \N  \}  \), then the \textit{closure of} \( A \) is 
    just 
    \[ \bar{A} = A \cup \{ 0 \}. \]
\item[(ii)] In the last example, \( y \notin (y- \epsilon , y + \epsilon ) \) where \( y \in \R  \) gurantees that the closure of \( \Q  \) in \( \R  \); that is, \( \bar{\Q} = \R  \). 
\item[(iii)] If \( A \) is an open interval \( (a,b) \), then the closure is just \( \bar{A} = [a,b] \); that is, \( \bar{A} = A \cup \{ a,b \}  \) where \( a,b \in \R  \) are the endpoints of the set \( (a,b) \). 
\item[(iv)]If \( A \) is a closed interval then the closure is just \( \bar{A} = A  \). The obvious conclusion from this is that closed intervals are always closed sets. 
\end{enumerate}
\end{ex}

\begin{tcolorbox}
\begin{thm}
    For any \( A \subseteq \R  \), the closure of \( \bar{A} \) is a closed set and is the smallest closed set containing \( A \). 
\end{thm}
\end{tcolorbox}

\begin{proof}
    Since \( L \) is the set of limit points of \( A \), it follows immediately that \( \bar{A} \)
    contains its limit points of \( A \). The problem here is that taking the union of \( A \) and \( L \) could produce some new limit points. 
    \begin{center}
        The details are in exercise 3.2.7
    \end{center}
    Hence, any closed set containing \( A \) must contain \( L \) as well. Hence, we have 
    \( \bar{A} = A \cup L  \) is the smallest closed set containing \( A \). 
\end{proof}

\subsection{Complements}

The notions of open and closed imply that they are not antonyms of each other. Just because a set is not open, does not immediately imply that it is closed. We can see this in action by considering the half-open interval 
\[ (c,d] = \{ x \in \R : c \leq x \leq d \}  \]
as being neither open nor closed. Furthermore, \( \R \) and \( \emptyset \) are both simultaneously open and closed at the same time. Luckily, these are the only two sets that exhibit this confusing property. We do have a relationship between open and closed sets however.   

Recall that the complement of a set \( A \subseteq \R  \) is defined to be the set
\[ A^c = \{ x \in \R : x \notin A  \} \]
which describes all of the elements that are not in \( A \). 

\begin{tcolorbox}
\begin{thm}
A set \( A \) is open if and only if \( A^c \) is closed. Likewise, a set \( B \) is closed if and only if \( B^c \) is open. 
\end{thm}
\end{tcolorbox}

\begin{proof}
    Suppose \( A \subseteq \R  \) is an open set. We want to show that \( A^c \) is a closed set. Let \( x  \) be a limit point of \( A^c \). Hence, there exists a sequence \( (x_n) \) such that \( \lim x_n = x \) where \( x_n \neq x  \) for all \( n \in \N  \). By definition of \( \lim x_n = x  \), there is an \( \epsilon- \)neighborhood \( V_{\epsilon } (x) \), but this means that \( x \notin A \) and must be in \( A^c \) since every \( \epsilon - \)neighborhood of \( x \) intersects \( A \) at some point other than \( x \). Hence, we have \( x \in O^c \). 
    
    For the converse statement, we assume \( A^c \) is a closed set. We want to show that \( A \) is open. Hence, let \( x \in A \). Since \( x \in A \), \( x \) is not a limit point of \( A^c \) and \( A^c \) is a closed set, there must exist an \( \epsilon - \)neighborhood such that \( x \notin V_{\epsilon }(x) \cap A^c \). This means \( x \in A \) and so \( V_{\epsilon } \subseteq A \). Hence, \( A \) is an open set.  
    The second statement follows quickly when taking the complement of each going in each direction.
\end{proof}

\begin{tcolorbox}
\begin{thm}
\begin{enumerate}
    \item[(i)] The union of a finite collection of closed sets is closed. 
    \item[(ii)] The intersection of an arbitrary collection of closed sets is closed. 
\end{enumerate}
\end{thm}
\end{tcolorbox}

\begin{proof}
    De Morgan's Laws state that for any collection of sets \( \{ E_{\lambda} : \lambda \in \Lambda \}  \) it is true that 
    \[ \Big( \bigcup_{\lambda \in \Lambda} E_{\lambda}\Big)^c = \bigcap_{\lambda \in \Lambda} E_{\lambda}^c ~~ \text{ and } ~~ \Big( \bigcap_{\lambda \in \Lambda} E_{\lambda}\Big)^c = \bigcup_{\lambda \in \Lambda} E_{\lambda}^c. \]
\end{proof}


\subsection{Definitions}

\begin{tcolorbox}
\begin{defn}
    A set \( A \subseteq \R  \) is \textit{open} if for all points \( a \in A \) there exists an \( \epsilon - \)neighborhood \(  V_{\epsilon }(a) \subseteq A \). 
\end{defn}
\end{tcolorbox}

\begin{tcolorbox}
\begin{thm}
\begin{enumerate}
    \item[(i)] The union of an arbitrary collection open sets is open.
    \item[(ii)] The intersection of a finite collection of open sets is open. 
\end{enumerate}
\end{thm}

\end{tcolorbox}
\begin{tcolorbox}
\begin{defn}
    A point \( x \) is a \textit{limit point} of a set \( A  \) if every \( \epsilon- \)neighborhood \( V_{\epsilon }(x) \) of \( x \) intersects the set \( A  \) at some point other than \( x \). 
\end{defn}
\end{tcolorbox}


\begin{tcolorbox}
\begin{thm}
A point \( x \) is a limit point of a set \( A \) if and only of \( x = \lim a_n \) for some sequence \( (a_n) \) contained in A satisfying \( a_n \neq x  \) for all \( n \in \N  \). 
\end{thm}
\end{tcolorbox}

\begin{tcolorbox}
\begin{defn}
A point \( a \in A  \) is an \textit{isolated point} of \( A \) if it is not a \textit{limit point} of \( A \).  
\end{defn}
\end{tcolorbox}

\begin{tcolorbox}
\begin{defn}
A set \( F \subseteq \R  \) is \textit{closed} if it contains its limit points. 
\end{defn}
\end{tcolorbox}

\begin{tcolorbox}
\begin{thm}
A set \( F \subseteq \R \) is closed if and only if every Cauchy sequence contained in \( F \) has a limit that is also an element of \( F \). 
\end{thm}
\end{tcolorbox}

\begin{tcolorbox}
\begin{thm}
For every \( y \in \R  \), there exists a sequence of rational numbers that converges to \( y \). 
\end{thm}
\end{tcolorbox}

\begin{tcolorbox}
\begin{defn}
    Given a set \( A \subseteq \R  \), let \( L \) be the set of all limit points of \( A \). The closure of \( A \) is defined to be the \( \bar{A} = A \cup L \). 
\end{defn}
\end{tcolorbox}


\begin{tcolorbox}
\begin{thm}
    For any \( A \subseteq \R  \), the closure of \( \bar{A} \) is a closed set and is the smallest closed set containing \( A \). 
\end{thm}
\end{tcolorbox}

\begin{tcolorbox}
\begin{thm}
A set \( A \) is open if and only if \( A^c \) is closed. Likewise, a set \( B \) is closed if and only if \( B^c \) is open. 
\end{thm}
\end{tcolorbox}


\subsection{Exercises}


\subsubsection{Exercise 3.2.2} Let 
\[ A = \Big\{ (-1)^n + \frac{ 2 }{ n } : n = 1,2,3,... \Big\}   \]
and
\[ B = \{ x \in \Q : 0 < x < 1 \}.  \]
Answer the following questions for each set:
\begin{enumerate}
    \item[(a)] What are the limit points? 
        \begin{proof}[Solution]
        The limit points of \( A \) is \( L = \{ -1, 1  \}  \) and the limit points of \( B \) is \( L = \{ 0,1 \}  \).
        \end{proof}
    \item[(b)] Is the set open? Closed? 
        \begin{proof}[Solution]
            The sets \( A  \) and \( B \) are not closed since their limit points are not contained and open since we can create \( V_{\epsilon }(x) \subseteq A \) while \( B \) is not open since \( V_{\epsilon }(x) \not \subseteq B \) for every \( x \in \Q \) however small \( \epsilon  \) is.  
        \end{proof}
    \item[(c)] Does the set contain any isolated points? 
        \begin{proof}[Solution]
        From part (b), since we cannot find any points near each \( x \in \Q  \) in \( B \), we have that all the points of \( B  \) are isolated points. 
        \end{proof}
    \item[(d)] Find the closure of the set.
        \begin{proof}[Solution]
            The closure of sets \( A \) and \( B \) are \( \overline{A} = A \cup \{-1,1\}   \) and \( \overline{B} = B \cup \{ 0,1 \}  \). 
        \end{proof}
\end{enumerate}




\subsubsection{Exercise 1.2.13}
Show De Morgan's Laws where \( \{ A_i : 1 \leq i \leq n \}  \) is a collection of sets such that
\begin{align*}
    \Big( \bigcup_{i = 1}^{n} A_i \Big)^c &= \bigcap_{ i=1 }^{ n } A_i^{c} \tag{1} \\
    \Big( \bigcap_{i = 1}^{n} A_i \Big)^c &= \bigcup_{ i=1 }^{ n } A_i^{c} \tag{2} \\
\end{align*}
for any finite \( n \in \N \). 
\begin{proof}
Our goal is to show that both inclusions hold for (1) and (2). Our first step is to induct on \( n\in \N \) to show that 
\[ \Big( \bigcup_{i=1}^{n} A_i \Big)^c \subseteq \bigcap_{i=1}^{n} A_i^c. \tag{1}\]
Let \( n = 1 \) be the base case. It follows immediately that \( A_1^c \subseteq A_1^c \). Let \( n = 2 \), then it follows that \( (A_1 \cup A_2)^c \subseteq A_1^c \cap A_2^c \) by exercise 1.2.5. For the other inclusion, we also have \( A_1^c \cap A_2^c \subseteq (A_1 \cup A_2)^c \). Now suppose (1) holds for \( 1 \leq n \leq k-1 \). We want to show that (1) holds for \( k \). Let 
\[ A' = \bigcup_{ n=1 }^{ k-1 } A_n  \]
then consider the following 
\[  \Big( \bigcup_{ n=1 }^{ k } A_n \Big)^c = \Big( A_k \cup \Big[ \bigcup_{ n=1 }^{ k-1 } A_n \Big] \Big)^c = ( A_k \cup A')^c \]
Let \( x \in (A_{k} \cup A')^c \), then we know that \( x \notin (A_k \cup A') \). This means that \( x \notin A_k  \) and \( x \notin A'\). Hence, we have \( x \in A_k^c \) and \( x \in (A')^c \); that is, 
\begin{align*}
    (A_k \cup A')^c &\subseteq A_k^c \cap (A')^c  \\
                    &= A_k^c \cap \Big( \bigcup_{n=1}^{k-1} A_n  \Big)^c. \\
                    &\subseteq A_k^c \cap \Big( \bigcap_{n=1}^{k-1} A_n^c \Big) \\
                    &= A_k \cap (A_{k-1} \cap ... \cap A_1) \\
                    &= \bigcap_{n=1}^{k} A_n^c.
\end{align*}
Hence, we have 
\[ \Big( \bigcup_{i=1}^{n} A_i \Big)^c \subseteq \bigcap_{i=1}^{n} A_i^c.\]
For the other inclusion, suppose the containment 
\[ \bigcap_{n=1}^{k-1}A_k^c \subseteq \Big( \bigcup_{n=1}^{k-1} A_k \Big)^c \tag{2} \]
holds for \( 1 \leq n \leq k -1  \). We want to show that (2) holds for \( k  \). Consider the finite intersection
\[ \bigcap_{ n=1 }^{ k } A_n^c = A_k^c \cap \Big( \bigcap_{ n=1 }^{ k-1 } A_{n}^c \Big).  \]
If we know that \( x \notin \bigcap_{ n=1 }^{ k-1 } A_{n} \) and \( x \notin A_k \) then \( x \notin \Big( A_k \cup \Big( \bigcap_{ n=1 }^{ k-1 } A_n \Big) \Big) \). Hence, using our inductive hypothesis, we have
\begin{align*}
    \bigcap_{ n=1 }^{ k } A_n^c &= A_k^c \cap \Big( \bigcap_{ n=1 }^{ k-1 } A_{n}^c \Big)  \\
                                &\subseteq A_k^c \cup \Big( \bigcap_{ n=1 }^{ k-1 } A_n \Big)^c \\ 
                                &\subseteq A_k^c \cup \Big( \bigcup_{ n=1 }^{ k-1 } A_n^c \Big) \\
                                &= \Big( \bigcup_{ n=1 }^{ k } A_n \Big)^c
\end{align*}
Since both containments hold, we must have 
\[ \Big( \bigcup_{ n=1 }^{ k }A_n  \Big)^c = \bigcap_{ n=1 }^{ k } A_n^c.  \]
The proof to the other equation is similar. 
\end{proof}







\subsubsection{Exercise 3.2.4} Let \( A  \) be nonempty and bounded above so that \( s = \sup A \) exists. 
\begin{enumerate}
    \item[(a)] Show that \( s \in \overline{A} \).
        \begin{proof}
            Let \( A \neq \emptyset \) and bounded above. Since \( s = \sup A  \) exists we can let \( \epsilon > 0  \) such that for some \( \alpha \in A  \), we have \( s - \epsilon < \alpha \). Our goal is to show that \( s \in \overline{A} \). Let \( (a_n) \) be a sequence in \( A \) such that \( a_n \neq s \) for all \( n \in \N \). Let \( \epsilon = 1/n \) such that 
            \[ s - \frac{ 1 }{ n } < \alpha \leq a_n \leq s. \]
            By the Squeeze Theorem, we have \( \lim a_n = s = \sup A \). This means \( s = \sup A \) is a limit point where \( L = \{ s \}  \) such that \( \overline{A} = A \cup L  \). Hence, \( s \in \overline{A} \).

        \end{proof}
    \item[(b)] Can an open set contain its supremum? 
        \begin{proof}[Solution]
        An open set \( A \) cannot contain its supremum, which is a limit point in part (a), since otherwise \( A \) would be a closed set.  
        \end{proof}
\end{enumerate}




\subsubsection{Exercise 3.2.5} Prove Theorem 3.2.8:
Show that a set \( F \subseteq \R  \) is closed if and only if if every Cauchy sequence contained in \( F \) has a limit that is also an element of \( F \). 
\begin{proof}
\( (\Rightarrow) \) Let \( F \subseteq \R  \) be a closed set. Let \( x  \) be a limit point and let \( (x_n) \) be a Cauchy sequence be arbitrary. Since \( F \) is a closed set, the limit point \( x \in F \); that is, \( \lim x_n = x \in F  \) where \( x_n \neq x  \) for all \( n \in \N \). 

\( (\Leftarrow) \) Let \( F \subseteq \R  \). We want to show that \( F  \) is closed. Let \( (x_n) \) be a Cauchy sequence contained in \( F \) such that \( \lim x_n = x \in F \). Note that \( x_n \neq x  \) for all \( n \in \N \). Since all the limit points of \( F \) are contained in \( F \), then \( F \) must be a closed set. 
\end{proof}

\subsubsection{Exercise 3.2.7} Given \( A \subseteq \R  \), let \( L  \) be the set of all limit points of \( A \). 
\begin{enumerate}
    \item[(a)] Show that the set \( L  \) is closed. 
        \begin{proof}
            Let \( L \) be the set of limit points of \( A \), and suppose that \( x  \) is a limit point of \( L \). Our goal is to show that \( x  \) is a limit point of \( A \). Let \( V_{\epsilon }(x) \) be arbitrary. Let \( \epsilon > 0  \), then we know that \( V_{\epsilon }(x) \) intersects \( L \) at a point \( \ell \in L \) where \( \ell \neq x  \). Choose \( \epsilon' > 0  \) small enough so that \( V_{\epsilon'}(\ell) \subseteq V_{\epsilon }(x)\) and \( x \notin V_{\epsilon '}(\ell) \). Since \( \ell \in L \), we know that \( \ell  \) is a limit point of \( A \)m and therefore \( x \) is a limit point of \( A \) and thus an element of \( L \). 
        \end{proof}
    \item[(b)] Argue that if \( x  \) is a limit point \( A \cup L  \), then \( x  \) is a limit point of \( A \). Use this observation to furnish a proof for Theorem 3.2.12. 
        \begin{proof}
            Suppose \( x  \) is a limit point of \( \overline{A} = A \cup L  \). By definition, we can construct \( V_{\epsilon}(x) \) such that \( V_{\epsilon }(x) \) intersects \( a \in \overline{A} \) where \( a \neq x  \). This means \( x \in A \) or \( x \in L \). If \( x \in A \), then \( V_{\epsilon }(x) \) intersects every point \( a \in A \) where \( x \neq a \). Hence, \( x  \) is a limit point of \( A \). If \( x \in L \), then we can use the same argument from above to construct an \( \epsilon' > 0  \) small enough so that \( V_{\epsilon'}(\ell) \subseteq V_{\epsilon }(x) \) where \( x \notin V_{\epsilon'}(\ell) \). Since \( \ell \in L  \) is a limit point, this means that that \( V_{\epsilon'}(\ell) \) intersects \( A \). But since \( x \notin V_{\epsilon'}(\ell) \) this means that \( V_{\epsilon }(x) \) intersects \( A \) at every point of \( A \) that is not \( x \). Hence, \( x \) is a limit point of \( A \). 
        \end{proof}
\end{enumerate}


\subsubsection{Exercise 3.2.9}

A proof for De Morgan's Laws in the case of two sets is outlined in Exercise 1.2.5. The general argument is similar. 

Now, provide the details for the proof of Theorem 3.2.14. 
    \begin{proof}
    To prove part (i), suppose we have a finite collection of open sets where 
    \[ \{ E_{i} : 1 \leq  i \leq N \}.   \]
    Since \( E_i  \) closed, their complements \( E_i^c \) is open. Since the finite intersection of open sets is open, we have that 
    \[\Big(  \bigcup_{ i=1 }^{ N } E_{i} \Big)^c = \bigcap_{ i=1 }^{ N  } E_{i}^c  \]
    is open. But this means that 
    \[ \bigcup_{i=1}^{N} E_{i} \]
    is closed. 

    To prove part (ii), suppose we have an arbitrary collection of closed sets 
    \[ \{ E_\lambda : \lambda \in \Lambda \}.  \]
    Since \(E_{\lambda}\) is closed, we have that their complement \( E_{\lambda}^c \) is open. But this means that the union 
    \[ \bigcup_{\lambda \in \Lambda} E_{\lambda}^c = \Big( \bigcap_{ \lambda \in \Lambda } E_{\lambda}  \Big)^c \tag{1} \]
    is also open. But since the complement of the intersection of (1) is open, we have 
    \[ \bigcap_{ \lambda \in \Lambda } E_{\lambda}  \]
    is closed. 
    \end{proof}

\subsubsection{Exercise 3.2.11} 
\begin{enumerate}
    \item[(a)] Prove that \( \overline{A \cup  B} = \overline{A} \cup \overline{B} \)
        \begin{proof}
        We want to show the following containments 
        \begin{align*}
            \overline{A \cup B} &\subseteq \overline{A} \cup \overline{B}, \\
            \overline{A \cup B } &\supseteq \overline{A} \cup \overline{B}
        \end{align*}
        Suppose \( x \in \overline{A \cup B} \). Then \( x \) is a limit point of \( A \cup B \). Hence, either \( x  \in A \) or \( x \in B \). But \( x  \) is a limit point so there exists \( V_{\epsilon}(x) \) that either intersects \(a \neq x \in A \) or \( b \neq x \in B \). But this means that \( x  \) is a limit point of \( A \) or \( B \). Hence, \( x \in \overline{A} \cup \overline{B} \). 
        Suppose \( x \in \overline{A} \cup \overline{B} \). Then either \( x \in \overline{A} \) or \( x \in \overline{B} \). But this means that \( x  \) is a limit point of \( A  \) or \( B \) which imply that \( V_{\epsilon}(x) \) intersects elements of both \( A  \) or \( B \) that is not \( x \). Hence, \( x  \) must be a limit point of either \( A  \) or \( B \). Hence, \( x \in \overline{A \cup B} \).
        Since both containments are true, we have that \( \overline{A \cup B} = \overline{A} \cup \overline{B} \). 
        \end{proof}
    \item[(b)] Does this result about closures extend to infinite unions of sets? 
        \begin{proof}[Solution]
            No this result does not extend to infinite unions. Consider the counter-example where we have a closed set \( H_n = [1/n, 1] \subseteq \R  \) where 
            \begin{align*}
                \bigcup_{ i=1 }^{ \infty  } \overline{H_n} &= (0, 1] \tag{1} \\
                \overline{\bigcup_{ i=1 }^{ \infty  } H_n}  &= [0,1] \tag{2}
            \end{align*}
            It is clear that (1) and (2) are not the same sets. 
        \end{proof}
        
\end{enumerate}


\subsubsection{Exercise 3.2.14} A dual notion to the closure of a set is the interior of a set. The \textit{interior} of \( E \) is denoted \( E^{\circ} \) and is defined as 
\[ E^{\circ} = \{ x \in E: \exists V_{\epsilon}(x) \subseteq E  \}.  \]
Results about closures and interiors posses a useful symmetry. 
\begin{enumerate}
    \item[(a)] Show that \( E \) is closed if and only if \( \overline{E} = E  \). Show that \( E \) is open if and only if \( E^{\circ} = E  \).
        First we show that first statement.
        \begin{proof}
            \( (\Rightarrow) \) Suppose \( E \) is closed. We want to show that \( \overline{E} = E \); that is, we want to show that \( \overline{E} \subseteq E  \) and \( \overline{E} \supseteq E  \). Note that the first containment follows immediately since \( \overline{E} \) is the smallest set containing \( E \). 
            Now we want to show that \( \overline{E} \supseteq E \). Let \( x \in E \) be  a limit point. Since \( x \) is a limit point and \( E \) is a closed set, we know that \( x \) must be contained in \( E \). This means that set of limit points \( L \) of \( E \) must where \( x \in L \) implies that \( x \in \overline{E} \). Hence, \( \overline{E} = E \).

            \( (\Leftarrow) \) It follows that \( E \) is closed since \( \overline{E} \) contains its limit points and that \( \overline{E} = E \). 
        \end{proof}
        Now we show the second statement
        \begin{proof}
           \( (\Rightarrow) \) Suppose \( E \) is an open set. We must show the following two containments: 
            \( E^{\circ} \subseteq E  \) and \( E^{\circ} \supseteq E \). We show the first containment. Let \( x \in E^{\circ} \) be arbitrary. Then there exists \( V_{\epsilon }(x) \) such that \( V_{\epsilon } (x) \subseteq E \). Hence, \( x \in E  \) so we have \( E^{\circ} \subseteq E  \). Now we show that second containment. Since \( E \) is an open set, let \( x \in E \) be arbitrary such that there exists \( V_{\epsilon }(x) \subseteq E \). But this is by definition the interior of \( E \) so we must have \( x \in E^{\circ} \). 
           
            \( (\Leftarrow) \) Suppose \( E = E^{\circ} \). We want to show that \( E \) is an open set. Let \( x \in E  \) be arbitrary. Since \( E = E ^{\circ} \), there exists \( V_{\epsilon }(x) \) such that \( V_{\epsilon }(x) \subseteq E \). But this means \( E \) is an open set by definition.  
        \end{proof}
    \item[(b)] Show that \( \overline{E}^c = (E^c)^{\circ} \), and similarly that \( (E^{\circ})^c = \overline{E^c} \). 

        Show that \( \overline{E}^c = (E^c)^{\circ} \)
        \begin{proof}
            We want to show that first equation; that is, we want to show the following two containments \( \overline{E}^c \subseteq (E^c)^{\circ} \) and \( \overline{E}^c \supseteq (E^c)^{\circ} \). 
            First we show the former containment. Let \( x \in \overline{E}^c \) be arbitrary. If \( x \notin \overline{E} \), then \( x  \) is not a limit point of \( E \) and \( x \notin E \).  But this means that \( x \in (E^c)^{\circ} \) and hence, \( \overline{E}^c \subseteq (E^c)^{\circ} \). 
            Now we show the second containment. Let \( x \in (E^c)^{\circ} \) be arbitrary. There exists \( V_{\epsilon }(x) \subseteq E^c \). We can be sure that \( x\) is not a limit point of \( \overline{E} \) since \( \overline{E} \) contains all its limit points. Hence, we must have \( x \in \overline{E}^c \). Hence, we have \( \overline{E}^c = (E^c)^{\circ} \).   
        \end{proof}
        Now we show \( (E^{\circ})^c = \overline{E^c}\)
        \begin{proof}
            We want to show the following two containments; namely, \( (E^{\circ})^c \subseteq \overline{E^c} \) and \( \overline{E^c} \subseteq (E^{\circ})^c \).

            We start with the first containment. Let \( x \in (E^{\circ})^c \) be arbitrary. This means \( x \notin E^{\circ} \) and hence for all \( \epsilon - \)neighborhoods of \( x \), we have \( V_{\epsilon }(x) \not\subseteq E \). Our goal is to show that \( x \in \overline{E^c} \). If \( x  \) is not a limit point of \( E^c \), then we just have \( x \in E^c \) and hence \( x \in \overline{E^c} \). Otherwise, we can prove \( x  \) is a limit point of \( E^c \). Suppose \( L \) is the set of limit points of \( E^c \). Let \( \epsilon' > 0   \) be as small as possible and \( \ell \in L  \) such that \( V_{\epsilon '}(\ell) \subseteq V_{\epsilon }(x) \) where \( x \notin V_{\epsilon'}(\ell) \). Since \( \ell \) is a limit point of \( E^c \), \( V_{\epsilon '}(\ell) \) intersects \( E^c \). But this also means \( V_{\epsilon }(x) \) intersects points of \( E^c \) that is not \( x  \). Hence, \( x  \) is a limit point of \( E^c \) and thus \( x \in \overline{E^c} \).


            Now let \( x \in \overline{E^c} \) be arbitrary. Then either \( x \in E^c \) or \( x \in L  \) where \( L \) denotes the set of limit points of \( E^c \). If \( x \in E^c \), then surely \( x \notin E^{\circ} \). Hence, \( x \in (E^{\circ})^c \). If \( x \in L \) and \( \overline{E} \) is a closed set, then \( x  \) cannot be in \( E^{\circ} \). Hence, \( x \) must be in \( (E^{\circ})^c \). Hence \( \overline{E^c} \subseteq (E^{\circ})^c \)

        \end{proof}
\end{enumerate}
















\section{Compact Sets}

\subsection{Compactness}
\begin{tcolorbox}
\begin{defn}
A set \( K \subseteq \R \) is compact if every sequence in \( K \) has a subsequence that converges to a limit that is also in \( K \).
\end{defn}
\end{tcolorbox}

\begin{ex}
\begin{enumerate}
    \item[(i)] Closed intervals are compact since all \( (a_n) \) in \( K \) are bounded and so we can always find a subsequence in \( K \) (By Bolzano-Weierstrass) that converges to a limit that is contained within the closed interval. We know the limit is contained in \( K \) since closed intervals are closed sets. 
\end{enumerate}
\end{ex}

In the example above, we used the fact that bounded sequences contain subsequences that converge and the fact that their limits are contained since closed intervals are closed sets. 

\begin{tcolorbox}
\begin{defn}
A set \( A \subseteq \R \) is \textit{bounded} if there exists \( M > 0  \) such that \( | a | \leq M  \) for all \( a \in A \).
\end{defn}
\end{tcolorbox}

\begin{tcolorbox}
    \begin{thm}[Characterization of Compactness in \( \R \)]
    A set \( K \subseteq \R \) is compact if and only if it is closed and bounded.
    \end{thm}
\end{tcolorbox}

\begin{proof}
    Assume \( K \) is a compact set. Suppose for sake of contradiction that \( K \) is not a bounded set. Our goal is to construct a sequence \( (a_n) \) that diverges. Since \( K \) is not bounded, for all \( M > 0  \), there exists \( a_n \in A \) such that \( | a_n | > M  \). But by assumption, \( K \) is compact so \( (a_n) \) must contain a subsequence \( (a_{n_k}) \) that converges to a limit in \( K \). But since \( (a_{n_k}) \) is unbounded, we have a contradiction. Hence, \( K \) must be a bounded set. 

    Now we will show that \( K \) is closed. Since \( K \) has to be bounded, then \( (a_n) \subseteq K  \) must contain a subsequence \( (a_{n_k}) \) that converges to a limit in \( K \). But this is the definition of a closed set. Hence, \( K \) is a closed set.

    Assume that \( K \) is closed and bounded. Let \( (a_n) \) be an arbitrary sequence in \( K \). Since \( K  \) is bounded and hence \( (a_n) \) is bounded, \( (a_n) \) contains a subsequence \( (a_{n_k}) \) such that \( (a_{n_k}) \to a  \). Since \( K  \) is closed, \( a  \) is contained in \( K \). Hence, we have that \( K \) is a compact set.
\end{proof}

It is important remember that closed intervals are not all that is when considering compact sets. The structure is much more intricate and interesting. For example, we can say that the Cantor Set is compact. We can experiment with this new idea of compact sets with the Nested Interval Property from chapter 1.


\begin{tcolorbox}
    \begin{thm}[Nested Compact Set Property]
    If 
    \[ K_1 \supseteq K_2 \supseteq K_3 \supseteq K_4 \supseteq \dots \]
    is a nested sequence of nonempty compact sets, then the intersection \( \bigcap_{ n=1 }^{ \infty  } K_n  \) is not empty.
    \end{thm}
\end{tcolorbox}

\begin{proof}
    Assume \( K_n \neq \emptyset  \) compact for each \( n \in \N \). Then for each \(  n \in \N \), choose \( x_n \in K_n \) where \( x_n \) is a sequence of points. Since we have a nested sequence of nonempty sets, it follows that \( x_n \in K_1 \). By definition of compactness, \( (x_n) \) contains a convergent subsequence \( (x_{n_k}) \) such that \( \lim x_{n_k} = x  \) where \( x \in K_n \) for every \( n \in \N \) and thus \( x \in K_1 \). Given \( n_0 \in \N \), we have that the terms of the sequence \( x_n  \) are contained within \( K_{n_0} \) given all \( n \geq n_0 \). We can ignore the finite number of terms for which \( n_k < n_0  \) so that \( (x_{n_k}) \) can be contained in \( K_{n_0} \). Hence, we have that \( x = \lim x_{n_k} \) is an element of \( K_{n_0} \). Because \( n_0  \) was arbitrary, we have that \( x \in \bigcap_{ n=1 }^{ \infty  } K_n \neq \emptyset\).
\end{proof}

\subsection{Open Covers}
In the last section, we proved that compact sets to be bounded and closed and vice versa. In other cases, we could also have defined compacts in this way and then proved that for every sequences that is bounded, there exists subsequences that converge to limits contained within the set. We can prove compactness in terms of open covers and finite subcovers. 

\begin{tcolorbox}
\begin{defn}
    Let \( A \subseteq \R \). An \textit{open cover} for \( A \) is a (possibly infinite) collection of open sets \( \{ O_{\lambda} : \lambda \in \Lambda \}  \) whose union contains the set \( A \); that is 
    \[ A \subseteq \bigcup_{ \lambda \in \Lambda } O_{\lambda}.  \]
    Given an open cover for \( A \), a \textit{finite subcover} is a finite subcollection of open sets from the original open cover whose union still manages to completely contain \( A \).
\end{defn}
\end{tcolorbox}

Subcollection in this context is just a collection of sets that are subsets of the original collection of open sets.

\begin{ex}
Consider the open interval \( (0,1) \). For each point \( x \in (0,1) \), let \( O_x \) be the open interval \( (x/2, 1) \). Let the infinite collection of \( O_x \) be defined as 
\[ \{ O_{x} : x \in (0,1) \}  \]
forms an open cover for the open interval \( (0,1) \); that is, 
\[  (0,1) \subseteq \bigcup_{ x \in (0,1) } O_x.  \]
Note that it is impossible to find a finite subcover for the open set \( (0,1) \). Given any proposed finite subcollection 
\[ \{ O_{x_1}, O_{x_2}, ..., O_{x_n} \}, \]
let \( x' = \min \{ x_1, x_2, \dots, x_n \}  \) and observe that for any \( y \in \R  \) satisfying \( 0 < y \leq x' /2  \) is not contained in the union \( \bigcup_{ i=1 }^{ n } O_{x_i} \). 
\end{ex}

Now consider a similar cover for the closed interval \( [0,1] \). For \( x \in (0,1) \), the sets \( O_x = (x/2, 1) \) do relatively well to cover \( (0,1) \), but in order to have an open cover for the closed interval \( [0,1] \), we could let \( \epsilon > 0   \) so that we can have epsilon neighborhoods covering both endpoints. That is, we have \( O_o = (-\epsilon , \epsilon ) \) and \( O_1 = (1 - \epsilon, 1 + \epsilon ) \). Then the collection 
\[ \{ O_0, O_1, O_x : x \in (0,1) \}  \]
is an open cover for \( [0,1] \) is a finite subcover for the closed interval \( [0,1] \).

\begin{tcolorbox}
    \begin{thm}[Heine-Borel Theorem]
Let \( K \) be a subset of \( \R \). All of the following statements are equivalent in the sense that any one of them implies the two others. 
\begin{enumerate}
    \item[(i)] \( K \) is compact.
    \item[(ii)] \( K \) is closed and bounded.
    \item[(iii)] Every open cover for \( K \) has a finite subcover.
\end{enumerate}
\end{thm}
\end{tcolorbox}

\begin{proof}
Notice that the proof of the equivalence of (i) and (ii) has already been proven in Theorem 3.3.1. All we need to show now is that (iii) implies (ii) and (iii) implies (i). 

To show (ii), we must show that \( K \) is both bounded and closed. To show that \( K \) is bounded, let us construct an open cover for \( K \) by defining \( O_x  \) to be an open interval of radius \( 1 \) (\( \epsilon  = 1  \)) for each \( x \in K \). This means there exists a \( \epsilon - \)neighborhood for each \( x \in K \); that is, \( O_x = V_1(x) \). Since the open cover \( \{ O_x : x \in K \}  \) contains a finite subcover for \( K \), we have the \( K \) contained in the union of the collection of sets \( \{ O_{x_1}, O_{x_2}, \dots , O_{x_n} \}  \). Hence, \( K \) itself is a bounded set.

Now let us show that \( K \) is closed. Let \( (y_n) \) be a Cauchy sequence contained in \( K \) with \( \lim y_n = y  \). To show that \( K \) is closed, we must show that \( y \in K \). Suppose for sake of contradiction that \( y \notin K  \). By assumption we can construct an open cover by taking \( O_x  \) to be an interval of radius \( | x - y  | / 2  \) around each point \( x \in K \). Also, we are assuming that the open cover \( \{ O_x : x \in K \}  \) for \( K \) contains a finite subcover \( \{ O_{x_1}, O_{x_2}, \dots, O_{x_n} \}  \). If \( y \notin K \), then the distance from \( y \) to each \( x_i \in K \) must be 
\[ \epsilon_0 := \min \Big\{ \frac{ | x_i - y  |  }{ 2  }   : 1 \leq i \leq n \Big\}.\]
Since \( (y_n) \) is a Cauchy sequence, so it must converge. Hence, for some \( N \in \N \), we know that 
\[ | y_N - y  | < \epsilon_0 \] we must have for every \( n \geq N \), 
But note that since \( y \notin K  \), not all of the terms from the sequence \( (y_n) \) for every \( n \geq N \) that is contained in \( K \) are not included in the finite subcover
\[ \bigcup_{ i=1 }^{ n } O_{x_i}. \]
Hence, our finite subcover does not actually cover all of \( K \) which is a contradiction and thus we must have \( y \in K \).




\end{proof}



\subsection{Exercises}

\subsubsection{Exercises 3.3.1} Show that if \( K \) is compact and nonempty, then \( \sup K  \) and \( \inf K \) both exists and are contained in \( K \).
\begin{proof}

\end{proof}




\section{Perfect Sets}



\begin{tcolorbox}
\begin{defn}
A set \( P \subseteq \R \) is \textit{perfect} if it is closed and contains no isolated points. \end{defn} 
\end{tcolorbox}
An straightforward example of perfect sets are closed intervals and singleton sets.

\begin{ex}
It is not too hard to see that the Cantor set from the very beginning of this chapter is perfect. We defined 
\[  C = \bigcap_{ n=0 }^{ \infty  } C_n  \] where each \( C_n \) is a finite union of closed intervals. We know by Theorem 3.2.14 that each \( C_n \) is closed, and as a result of using the same theorem that \( C \) is closed as well. Now all we need to show is that \( C \) contains no isolated points. 

Let \( x \in C \) be arbitrary. Let us construct a sequence \( (x_n) \) of points in \( C \) that are different from \( x  \) such that \( (x_n) \to x  \). We know that \( C \) contains endpoints of each interval that make up each \( C_n \). In exercise 3.4.3, we sketch the argument that these are all that is needed to construct such an \( (x_n) \).
\end{ex}

An argument for uncountability of the Cantor set.

\begin{tcolorbox}
\begin{thm}
A nonempty perfect set is uncountable.
\end{thm}
\end{tcolorbox}

\begin{proof}
Suppose \( P \) is a set that is perfect and nonempty. Hence, it must be the case that \( P \) is an infinite set because otherwise it would only consist of isolated points. Assume for sake of contradiction that \( P \) is countable. Thus, we can define \( P \) as the following:
\[ P = \{ x_1, x_2, x_3 \dots \},  \]
where every element of \( P \) appears on this list. Our goal is to construct a sequence of nested compact sets \( K_n \) that is all contained within \( P \) with the property that 
\( x_1 \notin K_2 \), \( x_2 \notin K_3 \), \( x_3 \notin K_4 \) and so on. Before proceeding with our argument, we must be sure that, in fact, each \( K_n \) is nonempty. Hence, we use the nested
Compact interval theorem to produce 
\[  x \in \bigcap_{ n=1 }^{ \infty  } K_n \subseteq P \]
that cannot be on the list \( \{ x_1, x_2, x_3, \dots \}  \). 

Let \( I_1  \) be a closed interval such that \( x_1 \in (I_0)^{\circ} \); that is, \( x_1  \) is not an endpoint of \( I_1 \). This produces an \( x_1 \) that is not isolated which means there exists some other point, say, \( y_2 \in P \) such that \( y_2 \in (I_1)^{\circ} \). Around \( y_2 \) we can construct a closed interval such that \( I_2 \supseteq I_1 \) with the condition that \( x_1 \notin I_2 \). Let \( \epsilon  > 0  \), then if \( I_1 = [a,b] \) we can define 
\[ \epsilon = \min \{ y_2 - a, b- y_2 , | x_1 - y_2  |  \}.  \]

Then, the interval \( I_2 = \{ [y_2 - \epsilon / 2, y_2 + \epsilon / 2 ] \}  \) has the desired properties. We can continue this process indefinitely. 

Since \( y_2 \in P \) is not isolated, there must exists another point \( y_3 \in P  \) in the interior of \( I_2  \) such that \( y_3 \neq x_2  \). Again, construct a closed interval centered on \( y_3  \) with an \( \epsilon  \) small enough so that \( x_2 \notin I_3  \) and \( I_3 \subseteq I_2  \). Observe that \( I_3 \cap P \neq \emptyset \) because this intersection contains at least \( y_3  \). 

We find that when we carry out this construction inductively, we have a sequence of closed intervals \( I_n \) satisfying the following properties:
\begin{enumerate}
    \item[(i)] \( I_{n+1} \subseteq I_n  \), 
    \item[(ii)] \(x_n \notin I_{n+1} \), and 
    \item[(iii)] \( I_n \cap P \neq \emptyset \).
\end{enumerate}
To finish the proof, let \( K_n = I_n \cap P \). For each \( n \in \N \), we have that \( K_n  \) is closed because it is the intersection of closed sets, and bounded since it is contained in the bounded sets \( I_n \). Hence, \( K_n  \) is compact. We can also see that \( K_n  \) is nonempty and \( K_{n+1} \subseteq K_n\). By employing the Nested Compact Set property, we can conclude that 
\[ \bigcap_{ n=1  }^{ \infty  } K_n \neq \emptyset. \]
But we find that each \( K_n \subseteq P \) where \( x_n \notin i_{n+1} \) leads to the conclusion that \( \bigcap_{ n=1 }^{ \infty  } K_n = \emptyset  \), which is a contradiction. 

\end{proof}

\subsection{Connected Sets}

Consider the two open intervals \( (1,2) \) and \( (2,5) \). Notice that these two intervals have the limit point \( x =2   \) in common. However, there is some space between them in the sense that \( 2 \) isn't contained in the other. Another way to say this is that \( \overline{(1,2)} \cap (2,5) = \emptyset \) and likewise, \( \overline{(2,5)} \cap (1,2) = \emptyset \). Notice that this same observation cannot be extended to the two sets \( (1,2] \) and \( (2,5) \) even though these two sets are disjoint.

\begin{tcolorbox}
\begin{defn}
\begin{enumerate}
    \item[(i)] Two nonempty sets \( A,B \subseteq \R   \) are \textit{separated} if \( \overline{A} \cap B  \) and \( A \cap \overline{B} \) are both empty.
    \item[(ii)] A set \( E \subseteq \R  \) is \textit{disconnected}  if it can be written as \( E = A \cup B \), where \( A  \) and \( B \) are nonempty separated sets.
    \item[(iii)] A set that is not disconnected is called a \textit{connected} set. 
\end{enumerate}
\end{defn}
\end{tcolorbox}






\begin{ex}
\begin{enumerate}
    \item[(i)] If we let \( A = (1,2) \) and \( B = (2,5) \), then it is not difficult to verify that \( E = (1,2) \cup (2,5) \) is disconnected. Notice that the sets \( C = (1, 2] \) and \( D = (2,5) \) are NOT separated because \( C \cap \overline{D}  = \{ 2 \} \) is nonempty. We know that the interval \( (1,5) \) is the union of sets \( C  \) and \( D \), but we cannot say they are disconnected. We will prove later that every interval is a connected subset of \( \R  \) and vice versa. 
    \item[(ii)] Consider the set of rational numbers 
        \[ A = \Q \cap (- \infty , \sqrt{ 2 }  ) ~ \text{and} ~ B = \Q \cap (\sqrt{ 2 }, \infty  ). \]
        It turns out that these two sets are disconnected such that \( \Q = A \cup B \). The fact that \( A \subseteq (\infty , \sqrt{ 2 }  ) \) implies that any limit point of \( A \) will necessarily fall in \( (-\infty , \sqrt{ 2 } ] \) by the Order Limit Theorem. Because this is disjoint from \( B \), we get that \( \overline{A} \cap B = \emptyset \). We can similarly show that \( A \cap \overline{B} = \emptyset \), which implies that \( A  \) and \( B \) are separated.
\end{enumerate}
\end{ex}

The definition of connected is stated as the negation of disconnected, but using the logical negation of the quantifiers in the definition above results in a positive characterization of connectedness. 

A way to show that a set \( E \) is connected is to partition \( E \) into two nonempty disjoint sets where we can show at least one of the sets contains a limit point of the other.

\begin{tcolorbox}
\begin{thm}
A set \( E \subseteq \R \) is connected if and only if, for all nonempty disjoint sets \( A \) and \( B \) satisfying \( E = A \cup B \), there always exists a convergent sequence \( (x_n) \to x  \) with \( (x_n) \) contained in one of \( A \) or \( B \), and \( x  \) an element of the other.
\end{thm}
\end{tcolorbox}

\begin{proof}
Exercise 3.4.6.
\end{proof}

\begin{tcolorbox}
\begin{thm}
A set \( E \subseteq \R  \) is connected if and only if whenever \( a < c < b  \) with \( a,b \in E \), it follows that \( x \in E  \) as well.
\end{thm}
\end{tcolorbox}

\begin{proof}
Assume \( E \) is connected, and let \( a, b \in E  \) and \( a < c < b  \). We can set \( A  \) and \( B  \) such that 
\[  A = (-\infty, c ) \cap E ~ \text{and} B = (c, \infty ) \cap E.  \]
Because \( a \in A  \) and \( b \in B \), neither set is empty and, just as in Example 3.4.5 (ii), neither set contains a limit point of the other. If \( E = A \cup B  \), then we have that \( E  \) is disconnected. If \( E = A \cup B  \), then we would have that \( E  \) is disconnected, which it is not. It must be the case that \( A \cup B  \) is missing some element of \( E  \), and \( c \) is the only possibility. Thus, \( c \in E  \).

Conversely, assume \( E  \) is an interval in the sense that whenever \( a, b \in E  \) satisfy \( a < c < b  \) for some \( c  \), then  \( c \in E  \). Our intent is to use the characterization of connected sets in Theorem 3.4.6, so let \( E = A \cup B \), where \( A  \) and \( B \) are nonempty and disjoint. 

We need to show that one of these sets contains a limit point of the other. Pick \( a_0 \in A  \) and \( b_0 \in B \), and suppose \( a_0 < b_0  \) for sake of argument. Since \( E  \) is an interval, the interval \( I_0 = [a_0, b_0 ] \) is contained in \( E  \). Now, let us bisect \( I_0 \) into two equal halves. The midpoint of \( I_0  \) must either be in \( A \) or \( B \), and so choose \( I_1 = [a_1, b_1 ]  \) to be the half that allows us to have \( a_1 \in A  \) and \( b_1 \in B \). We can continue such a process inductively to get a sequence of nested intervals \( I_n [a_n, b_n] \), where \( a_n \in A \) and \( b_n \in B \), and the length \( (b_n - a_n) \to 0  \). Hence, the following intersection
\[ \bigcap_{ n=0  }^{ \infty  } I_n \neq \emptyset. \]
Since \( (a_n - b_n ) \to 0  \), we have that the sequences of endpoints have the same limit point \( x  \). Since \( x \in E  \), it must be the case that \( x  \) must belong to either \( A  \) or \( B \). Hence, \( E \) is a connected set. 
\end{proof}


\subsection{Definition}


\begin{tcolorbox}
\begin{defn}
A set \( P \subseteq \R \) is \textit{perfect} if it is closed and contains no isolated points. \end{defn} 
\end{tcolorbox}


\begin{tcolorbox}
\begin{thm}
A nonempty perfect set is uncountable.
\end{thm}
\end{tcolorbox}

\begin{tcolorbox}
\begin{defn}
\begin{enumerate}
    \item[(i)] Two nonempty sets \( A,B \subseteq \R   \) are \textit{separated} if \( \overline{A} \cap B  \) and \( A \cap \overline{B} \) are both empty.
    \item[(ii)] A set \( E \subseteq \R  \) is \textit{disconnected}  if it can be written as \( E = A \cup B \), where \( A  \) and \( B \) are nonempty separated sets.
    \item[(iii)] A set that is not disconnected is called a \textit{connected} set. 
\end{enumerate}
\end{defn}
\end{tcolorbox}

\begin{tcolorbox}
\begin{thm}
A set \( E \subseteq \R \) is connected if and only if, for all nonempty disjoint sets \( A \) and \( B \) satisfying \( E = A \cup B \), there always exists a convergent sequence \( (x_n) \to x  \) with \( (x_n) \) contained in one of \( A \) or \( B \), and \( x  \) an element of the other.
\end{thm}
\end{tcolorbox}

\begin{tcolorbox}
\begin{thm}
A set \( E \subseteq \R  \) is connected if and only if whenever \( a < c < b  \) with \( a,b \in E \), it follows that \( c \in E  \) as well.
\end{thm}
\end{tcolorbox}
\subsection{Exercises}


\subsubsection{Exercise 3.4.1} If \( P \) is a perfect set and \( K \) is compact, is the intersection \( P \cap K  \) always compact? Always perfect? 
\begin{proof}[Solution]
\( P \cap K \) always compact but not always perfect. This is because \( P \cap K  \) is always a closed and bounded set.
\end{proof}

\subsubsection{Exercise 3.4.5} 
Let \( A  \) and \( B  \) be nonempty subsets of \( \R  \). Show that if there exists disjoint open sets \( U  \) and \( V  \) with \( A \subseteq U  \) and \( B \subseteq V  \), then \( A  \) and \( B  \) are separated.

\begin{proof}
    Let \( A  \) and \( B  \) be nonempty subsets of \( \R  \). Suppose there exists sets \( U \) and \( V  \) such that \( U \cap V = \emptyset \). Let \( x  \in U \). Since \( U \) is an open set, let \( x \in U^{\circ} \). Hence,  there exists \( V_{\epsilon }(x) \) such that \( V_{\epsilon }(x) \subseteq U  \). Let \( x  \) be a limit point of \( A  \) such that there exists \( (x_n) \to x  \) where \( x_n \neq x  \) for all \( n \in \N  \). Doing the same process for the set \( V \) suppose there exists \( (y_n)  \) is a sequence contained in \( B  \) such that \( (y_n) \to y  \) where \( y \in V  \) is an interior point such that there exist \( V_{\epsilon }(y) \subseteq V  \).

    Since \( U \cap V = \emptyset \) and \( x  \) and \( y \) are interior points of \( U \) and \( V  \) respectively, it follows that \( V_{\epsilon }(x) \cap V_{\epsilon }(y) = \emptyset \). But this means neither limit point of \(A \) nor \( B \) is contained in the other. Hence, \( A  \) and \( B  \) must be separated sets.
\end{proof}





\subsubsection{Exercise 3.4.6} Prove Theorem 3.4.6.
\begin{proof}
    \( (\Rightarrow) \) Suppose \( E \subseteq \R  \) is connected and suppose \( A  \) and \( B \) are disjoint. Since \( E  \) is connected, we have \( \overline{A} \cap B \neq \emptyset \) and \( \overline{B} \cap A \neq \emptyset \). Without loss of generality, let \( x \in \overline{A}  \cap B  \). Since \( A  \) and \( B  \) are disjoint, we must not have \( x \in A   \). Hence, \( x \in B  \) else \( A  \) and \( B  \) would not be disjoint. Hence, \( x  \) is a limit point of \( A  \). Hence, there exists \( (x_n) \subseteq A  \) such that \( (x_n) \to x  \) where \( x \neq x_n \) for all \( n \in \N \).  

\( (\Leftarrow) \) Our goal now is to show the converse; that is, show that \( A  \) and \( B \) are not separated i.e \( \overline{A} \cap B \neq \emptyset \) and \( \overline{B} \cap A \neq \emptyset \). Suppose there exists \( (x_n) \subseteq A \). By assumption \( (x_n) \to x  \) such that \( x \in B  \). Since \( \overline{A}  \) is a closed set, we have that \( x \in \overline{A} \). But this means that \( \overline{A} \cap B \neq \emptyset\). The argument is similar when \( (x_n) \subseteq B  \). Hence, \( \overline{B} \cap A \neq \emptyset \). But this means that \( E = A \cup B  \) is connected.
\end{proof}

\subsubsection{Exercise 3.4.7} A set \( E \) is \textit{totally disconnected} if, given any two distinct points \( x, y \in E  \), there exists separated sets \( A  \) and \( B  \) with \( x \in A  \), \( y \in B  \), and \( E = A \cup B  \).

\begin{enumerate}
    \item[(a)] Show that \( \Q  \) is totally disconnected.
        \begin{proof}
            Since the rational numbers are dense in \( \mathbb{I} \), we can construct the following interval \( x < c < y  \) where \( c \in \mathbb{I}  \). We can set \( A  \) and \( B \) such that 
            \[  A = (-\infty, c ) \cap \Q \text{ ~ and ~ } B = \Q \cap (c, +\infty )   \]
            Let \( x \in A  \) and \( y \in B  \), then neither set is empty and neither set contains a limit  point of the other. Since \( A \cap B = \Q  \), we must have \( \Q  \) as a totally disconnected set unless \( c \in \Q  \) which is not. 
        \end{proof}
    \item[(b)] Is the set of irrational numbers totally disconnected? 
        \begin{proof}
            The set of irrational numbers is totally disconnected because we can always find \( x \in \Q  \) such that for every \( z,y \in \mathbb{I} \), we have \( z < x < y  \). Thus, we can follow the same argument above to produce two sets that are separated.
        \end{proof}
\end{enumerate}




\chapter{Functional Limits and Continuity}

\chapter{Functional Limits and Continuity}



\section{Functional Limits}


\subsection{Defining the Functional Limit}
Consider a function \( f: A \to \R  \). Suppose \( c  \) of \( A  \) is a limit point. From the last chapter, recall that the definition of a limit point is any \( \epsilon - \)neighborhood \( V_{\epsilon }(c) \) intersects \( A \setminus \{ c \}  \). In other words, \( c  \) is a limit point of \(  A \) if and only if \( c = \lim x_n \) for some sequence \( (x_n) \subseteq A  \) with \( x_n \neq c  \) for all \( n \in \N  \). Furthermore, it is important to keep in mind that limit points of \(  A \) do not necessarily belong to \( A  \) unless it is closed. 

If \( c  \) is a limit point of our domain \( A  \), then, we can state that 
\[ \lim_{x \to c }f(x) = L  \] is intended to convey that the values of \( f(x) \) gets arbitrarily close to \( L \) as \( x  \) is chosen arbitrarily close to \( c \). It is important to keep in mind that \( c \) need not be in the domain of \( A  \).

The structure of the definition of functional limits is as follows:
Given a sequence \( (a_n) \), the assertion that \( \lim a_n = L   \) implies that for every \( \epsilon - \)neighborhood \( V_{\epsilon }(L) \) centered at \( L \), we can find a point in a sequence say \( a_N \) after which all the terms of \( a_n \) fall in \( V_{\epsilon }(L) \). This is in response to an arbitrary choice of \( x \) in the domain where we have a \( \delta- \)neighborhood

\begin{tcolorbox}
    \begin{defn}[Functional Limit]
        Let \( f: A \to \R  \), and let \( c \) be a limit point of the domain \( A  \). We say that \( \lim_{ x \to c } f(x) = L   \) provided that, for all \( \epsilon > 0  \), there exists a \( \delta > 0  \) such that whenever 
        \[ 0 <  | x - c  | < \delta  \] (and \( x \in A  \)) it follows that 
        \[ | f(x) - L  | < \epsilon.  \]
    \end{defn}
\end{tcolorbox}

This is often referred to as the epsilon-delta definition of a functional limit. The statement
\[ | f(x) - L  | < \epsilon  \]
is equivalent to saying that \( f(x) \in V_{\epsilon }(L) \). Likewise, the statement 
\[  | x - c  | < \delta \] is true if and only if \( x \in V_{\delta}(c) \). Note that we imposed an additional restriction that \( | x - c  | > 0  \). This is because we don't want \( x = c  \). We can recast the definition above in terms of \( \epsilon- \)neighborhoods to help get a more geometric view of what is happening with these functional limits.

\begin{tcolorbox}
\begin{defn}
    Let \( c  \) be a limit point of the domain \( f: A \to \R  \). We say \( \lim_{ x \to c } f(x) = L  \) provided that for every \( \epsilon - \)neighborhood \( V_{\epsilon }(L)  \) of \( L \), there exists a \( \delta- \)neighborhood \( V_{\epsilon }(c) \) around \( c \) with the property that for all \( x \in V_{\delta}(c) \) different from \( c  \) (with \( x \in A  \)) it follows that \( f(x) \in V_{\epsilon }(L) \).
\end{defn}
\end{tcolorbox}

The reminder that we must have \( x \in A   \) ensures that every possible \( x \in A  \) must be a valid input for the function in question. Note that the appearance of \( f(x) \) in our definitions carries an implicit assumption that \( x  \) is always part of the domain of \( f \). There is no use for considering isolated points outside of \( A  \). Hence, we can always expect that functional limits will have \( x \in A  \) that approach the limit point of \( A  \) or \( \text{dom}(f) \).

\begin{ex}
\begin{enumerate}
    \item[(i)] Suppose we want to show that for \( f(x) = 3x + 1  \), we have 
        \[ \lim_{ x \to 2 } f(x) = 7. \]
        \begin{proof}
        Let \( \epsilon > 0  \). Consider \( | f(x) - 7  |  \). We want to show that whenever \( | x - 2  | < \delta \), that 
        \[  | f(x) - 7 | < \epsilon. \]
        Hence, observe that 
        \begin{align*}
            | f(x) - 7  | &= | (3x+1) - 7  |  \\
                          &= | 3x - 6  |  \\
                          &= 3| x - 2 | \\ 
                          &< 3 \delta. \\
        \end{align*}
        We can choose \( \delta = \epsilon / 3  \) such that 
        \[  | f(x) - 7  | < 3 \delta = 3 \frac{ \epsilon  }{ 3 } = \epsilon. \]
        Hence, we have that \( \lim_{ x \to 2 } f(x) = 7. \)
        \end{proof}
    \item[(ii)] Let's show that 
        \[  \lim_{ x \to 2  } g(x) = 4, \]
        where \( g(x) = x^2  \).
        \begin{proof}
        Let \( \epsilon > 0  \). Suppose \( 0 < | x - 2  | < \delta  \). Then 
        \begin{align*}
            | g(x) - 4  | &= | x^2 - 4  |  \\
                          &= | (x-2)(x+2) | \\
                          &= | x-2 | | x+2 |.\\
        \end{align*}
        Since \( |x + 2 | < \delta + 4   \), observe that for \( \delta = \min \{1, \epsilon / 5\}  \), we have 
        \begin{align*}
           | g(x) - 4  |  &= | x-2 | | x+2 |  \\
                          &< \delta \cdot (\delta + 4 ) \\
                          &= \frac{ \epsilon  }{ 5 } \cdot 5 \\
                          &= \epsilon.
        \end{align*}
        \end{proof}
\end{enumerate}
\end{ex}

\subsection{Sequential Criterion for Functional Limits}

\begin{tcolorbox}
    \begin{thm}[Sequential Criterion for Functional Limits]
    Given a function \( f: A \to \R  \) and a limit point \( c \) of \( A \), the following two statements are equivalent: 
    \begin{enumerate}
        \item[(i)] \( \lim_{ x \to c } f(x) = L. \)
        \item[(ii)] For all sequences \( (x_n) \subseteq A  \) satisfying \( x_n \neq c  \) and \( (x_n) \to c  \), it follows that \( f(x_n) \to L  \).
    \end{enumerate}
    \end{thm}
\end{tcolorbox}

\begin{proof}
    (i) \( \implies \) (ii) Suppose \( \lim_{ x  \to c } f(x) = L  \). Let \( \epsilon > 0  \). By assumption, we have a sequence \( (x_n) \to c  \). It immediately follows that for some \( N \in \N  \) such that for all \( n \geq N  \) that \( x_n \in V_{\delta}(c) \). Hence, \( f(x_n) \in V_{\epsilon }(L) \) by the Topological Definition of functional limits.
 
    (ii) \( \implies \) (i) Let \( (x_n) \subseteq A  \) satisfying \( x_n \neq c  \) and \( (x_n) \to  c \) such that \( f(x_n) \to L  \). Suppose for sake of contradiction that \( \lim_{ x \to c } f(x) \neq L  \). Hence, there exists \( \epsilon_0 \) such that for any \( \delta > 0  \) where \( | x_n -c  | < \delta \) that \( | f(x_n) - L  | \geq \epsilon_0 \). Let \( \delta = 1/n \) and suppose we pick \( x_n \in V_{\delta}(c) \) such that \( f(x_n) \notin V_{\epsilon }(L) \). But this implies that \( f(x_n) \not \to L  \) which contradicts our assumption that it is. Hence, it must be the case that (i) holds.
\end{proof}

\begin{tcolorbox}
    \begin{cor}[Algebraic Limit Theorem for Function Limits]
    Let \( f \) and \( g \) be functions defined on a domain \( A \subseteq \R  \), and assume \( \lim_{ x \to c } f(x) = L  \) and \( \lim_{ x \to c } g(x) = M  \) for some limit point \( c \) of \( A  \). Then, 
    \begin{enumerate}
        \item[(i)] \( \lim_{ x \to c } kf(x) = kL  \) for all \( k \in \R  \),
        \item[(ii)] \( \lim_{ x \to c } [f(x) + g(x)] = L + M, \)
        \item[(iii)] \( \lim_{ x \to c } [f(x)g(x)] = L \cdot M  \), and 
        \item[(iv)] \( \lim_{ x \to c } \frac{ f(x) }{ g(x)  } = \frac{ L }{ M }  \) provided that \( M \neq  0 \).
    \end{enumerate}
    \end{cor}
\end{tcolorbox}

\begin{proof}
Exercise 4.2.1.
\end{proof}


\begin{tcolorbox}
    \begin{cor}[Divergence Criterion for Functional Limits]
    Let \( f \) be a function defined on \( A  \) and let \( c  \) be a limit point of \( A  \). If there exists two sequences \( (x_n) \) and \( (y_n) \) in \( A  \) with \( x_n \neq  c \) and \( y_n \neq c  \) 
    \[  \lim x_n = \lim y_n = c \text{~but~} \lim f(x_n) \neq \lim f(y_n),  \]
    then we conclude that the functional limit \( \lim_{x \to c } f(x)  \) does not exist.
    \end{cor}
\end{tcolorbox}

\begin{ex}
Suppose we wanted to show that \( \lim_{ x \to 0 } \sin(1/x) \) does not exist. Set \( x_n = 1/2n\pi \) and \( y_n = 1 / (2n\pi + \pi/2 ) \), then 
\[  \lim x_n = \lim y_n = 0  \] but
\( \sin(1/x_n) = 0  \) for all \( n \in \N  \) while \( \sin(1/y_n) = 1  \). Thus, we have 
\[  \lim \sin(1/x_n) \neq \lim  \sin (1/y_n), \]
and thus we know that \( \lim_{x \to  0 } \sin(1/x)  \) does not exist.
\end{ex}

\subsection{Exercises}
\subsubsection{Exercise 4.2.1}

\begin{enumerate}
    \item[(a)] Supply the details for how Corollary 4.1.1.1 follows from the Sequential Criterion for Functional Limits in Theorem 4.1.1 and the Algebraic Limit Theorem for sequences proved in Chapter 2.

    \begin{proof}
    Let \( f \) and \( g \) be functions defined on a domain \( A \subseteq \R  \). Assume \( \lim_{ x \to c } f(x) = L  \) and \( \lim_{ x \to c } g(x) = M  \) for some limit point \( c  \) of \( A  \). By the Sequential Criterion for functional limits, let \( (x_n), (y_n) \subseteq A  \) such that \( (x_n) \to c  \) and \( (y_n) \to c  \) where \( x_n, y_n \neq c  \) implying that \(f(x_n) \to L   \) and \( g(y_n) \to M  \). By the Algebraic Limit Theorem, we can state that 
    \[  \lim [f(x_n) + g(y_n)]\lim f(x_n) + \lim g(y_n) = L  + M.\]
    Hence, we have that 
    \[  \lim_{ x \to c } [f(x) + g(x)] = L + M  \]
    by the Sequential Criterion for Functional Limits.
    \end{proof}
    \item[(b)] Now, write another proof of Corollary 4.1.1.1 part (ii) directly from Definition for functional limits without using the sequential criterion in Theorem 4.1.1.
    \begin{proof}
    We can also prove the addition rule for Functional Limits directly from the definition. Suppose \( \lim_{ x \to c } f(x) = L  \) and \( \lim_{ x \to c } g(x) = M  \). Then for some \( \delta > 0  \), suppose \( c  \) is a limit point of \( A  \) such that \( 0 < | x -c  | < \delta \). We want to show that for any arbitrary \( \epsilon > 0  \) that 
    \[  | (f(x) + g(x)) - (L + M) | < \epsilon.  \]
    Hence, choose \( \delta = \min \{ \delta_1, \delta_2  \}  \)
    \begin{align*}
        |(f(x) + g(x)) - (L + M)   | &= | (f(x) - L ) + (g(x) - M ) |  \\
       &\leq  | f(x) - L  |  + | g(x) - M  | \\ 
       &< \frac{ \epsilon  }{ 2  }  + \frac{ \epsilon  }{ 2 } \\
       &= \epsilon.
    \end{align*}
    Hence, we have 
    \[  \lim_{ x \to c } [f(x) + g(x)] = L + M.  \]
    \end{proof}
    \item[(c)] Repeat (a) and (b) for Corollary 4.1.1.1 part (iii).
    \begin{proof}
    Let \( (x_n), (y_n) \subseteq A  \) such that \( (x_n) \to c  \) and \( (y_n) \to c  \) for some limit point \( c  \) of \( A  \) where  we have 
    \( \lim f(x_n) = L   \) and \( \lim  g(y_n) = M  \). By the Algebraic Limit Theorem, we have that 
    \[ \lim [f(x_n)g(y_n)] = \lim f(x_n) \cdot \lim g(y_n) = L \cdot M. \]
   By the Sequential Criterion, this also means that 
   \[  \lim_{ x \to c } [f(x)g(x)] = L \cdot M. \]

    Now we can prove the same fact but this time we use only the Definition of functional limit.
    Let \( f: A \to \R  \). Suppose there exists \( \delta > 0  \) such that \( 0 < | x - c  | < \delta  \) where \( c  \) is a limit point of \( A  \). Let \( \epsilon > 0  \). Our goal is to show that 
    \[ | f(x)g(x) - LM  | < \epsilon. \]
    Since \( \lim_{ x \to c } f(x) = L  \) and \( \lim_{ x  \to c } g(x) = M  \), choose \( \delta = \min \{  \delta_1, \delta_2, \delta_3  \}  \) 
    \begin{align*}
        | f(x)g(x) - LM  | &= | f(x)g(x) - g(x)L + g(x)L - LM |  \\
                           &= | g(x)(f(x) - L ) + L (g(x) - M)  | \\ 
                           &\leq | g(x) | | f(x) - L  | + L | g(x) - M  |   \\
                           &< M + 1 \cdot \frac{ \epsilon  }{ 2(M + 1)  } + L \cdot \frac{ \epsilon  }{ 2L  } \\
                           &= \epsilon.
    \end{align*}
    Hence, we have that 
    \[  \lim_{ x \to c } f(x)g(x) = L M. \]
    \end{proof}
\end{enumerate}


\subsubsection{Exercise 4.2.2} For each stated limit, find the largest possible \( \delta- \)neighborhood that is a proper response to the given \( \epsilon  \) challenge.

\begin{enumerate}
    \item[(a)] \( \lim_{ x \to 3 } (5x - 6 ) = 9  \).
        \begin{proof}[Solution]
        Since \( \epsilon = 1  \), we know that 
        \[ | (5x-6) - 9  | < 1.\]
        To solve for \( \delta \), we do the following
        \begin{align*}
            | (5x - 6) - 9  | &= | 5 (x - 3 ) | < 1    \\
                              &= 5 | x - 3  | < 1 
        \end{align*}
        which implies that 
        \[  | x - 3  | < \frac{ 1 }{ 5 } = \delta. \]
        Hence, the largest possible \( \delta- \)neighborhood that is a proper response to the given \( \epsilon   \) challenge. 
        \end{proof}
    \item[(b)] \( \lim_{ x \to 4 } \sqrt{ x } = 2  \), where \( \epsilon = 1  \).
        \begin{proof}[Solution]
        Since \( \epsilon = 1  \), we know that 
        \[  | \sqrt{ x }  - 2  | < 1. \]
        To get our \( \delta \), we just do the following manipulations
        \begin{align*}
            \sqrt{ x } - 2   &< 1  \\
            \sqrt{ x } &< 3 \\
            x &< 9 \\
            | x - 4  | &< 5 = \delta. \\
        \end{align*}
        Hence, our \( \epsilon  \) response is \( \delta =  5  \).
        \end{proof}
    \item[(c)] \( \lim_{ x  \to \pi } [[x]] = 3 \), where \( \epsilon = 1  \). (The function \( [[x]] \) returns the greatest integer less than or equal to \( x  \).)
        \begin{proof}[Solution]
            Let \( \epsilon = 1  \). We want to generate a \( \delta > 0  \) that satisfies the \( \epsilon  \) challenge. Since \( [[\pi]] = 3  \), our max \( \delta- \)neighborhood can only have \( \delta = \pi - 3  \)
        \end{proof}
    \item[(d)] We have (c) but this time with \( \epsilon = .01 \).  
        \begin{proof}[Solution]
        It would still remain as \( \delta = \pi - 3  \).
        \end{proof}
\end{enumerate}

\subsubsection{Exercise 4.2.5} Use Definition 4.1.1 to supply a proper proof for the following limit statements. 
\begin{enumerate}
    \item[(a)] \( \lim_{ x \to 2 } (3x + 4 ) = 10 \).
        \begin{proof}
        Suppose there exists a \( \delta > 0  \) such that \( 0 < | x - 2  | < \delta \). Let \( \epsilon > 0  \). Then observe that by using definition 4.1.1 that 
        \begin{align*}
            | f(x) - 10 | &= | (3x+4) - 10 |  \\
                          &= 3 | x - 2  | \\
                          &< 3 \delta \\
                          &= 3 \cdot \frac{ \epsilon  }{ 3 }\\
                          &= \epsilon.
        \end{align*}
        \end{proof}
    \item[(b)] \( \lim_{ x \to 0 } x^3 = 0  \).
        \begin{proof}
            Suppose there exists \( \delta > 0  \) such that \( 0 < | x  | < \delta \). By applying the definition of functional limits, choose \( \delta = (\epsilon )^{1/3} \)
        \begin{align*}
            | f(x) - 0  | &= | x^3  |  \\
                          &< \delta^3 \\
                          &= \epsilon.
        \end{align*}
        Hence, we have \( | f(x) - 0  | < \epsilon  \) which implies that 
        \[  \lim_{ x \to 0 } x^3 = 0. \]
        \end{proof}
    \item[(c)] \( \lim_{ x \to 2 } (x^2 + x - 1 ) = 5.  \)
        \begin{proof}
        Let \( \epsilon > 0  \). Choose \( \delta = \min \{ \delta_1, \delta_2  \}  \) such that 
        \begin{align*}
            | f(x)  - 5 | &= | (x^2 + x - 1 ) - 5  |  \\
                          &= | x^2 + x - 6  | \\
                          &= | x + 3  | | x - 2  | \\
                          &< | x+3  | \delta \\
                          &< 3 \cdot \frac{ \epsilon  }{ 3 } .
                          &= \epsilon.
        \end{align*}
        \end{proof}
    \item[(d)] \( \lim_{ x \to 3 } 1/x = 1/3.\)
        \begin{proof}
        Let \( \epsilon > 0  \). Choose \( \delta = \min \{ \delta_1, \delta_2 \}  \) such that 
        \begin{align*}
            | f(x) - \frac{ 1 }{ 3 }  | &= | \frac{ 1 }{ x }  - \frac{ 1 }{ 3 }  |  \\
                                        &= \Big| \frac{ x - 3  }{ 3x }  \Big| \\
    &= \frac{ | x - 3  |  }{ 3| x |  } \\ 
    &< 12 \epsilon \cdot \frac{ 1 }{ 12 } \\
    &= \epsilon.
        \end{align*}
        \end{proof}
\end{enumerate}




\subsubsection{Exercise 4.2.6} Decide if the following claims are true or false, and give short justifications for each conclusion.

\begin{enumerate}
    \item[(a)] If a particular \( \delta \) has been constructed as a suitable response to a particular \( \epsilon  \) challenge, then any smaller positive \( \delta \) will also suffice.
        \begin{proof}[Solution]
        This is true since \( \delta \) that has been constructed is the largest possible neighborhood that one can produce as a response to the \( \epsilon  \) challenge. 
        \end{proof}
    \item[(b)] If \( \lim_{ x \to a } f(x) = L  \) and \( a \) happens to be in the domain of \( f \), then \( f(a) = L  \).
        \begin{proof}[Solution]
        This is false. I have trouble thinking of an example. Will get back to this one soon. 
        \end{proof}
    \item[(c)] If \( \lim_{ x \to a } f(x) = L  \), then \( \lim_{ x \to a } 3[f(x) - 2 ]^2 = 3(L-2)^2 \).
        \begin{proof}[Solution]
        This is true based on the Algebraic Functional Limit Theorem.
        \end{proof}
    \item[(d)] If \( \lim_{ x \to a } f(x) = 0  \), then \( \lim_{ x\to a } f(x)g(x) = 0   \) for any function \( g \) (with domain equal to the domain of \( f \)).
        \begin{proof}[Solution]
        This is not true in general since we can easily produce \( g(x)  \) that is unbounded. For this to work, we would need \( g(x) \) to be bounded.
        \end{proof}
\end{enumerate}

\subsubsection{Exercise 4.2.7} Let \( g: A \to \R  \) and assume that \( f \) is a bounded function on \( A \) in the sense that there exists \( M > 0  \) satisfying \( | f(x) | \leq  M  \) for all \( x \in A  \). Show that if \( \lim_{ x \to c } g(x) = 0  \), then \( \lim_{ x \to c } g(x)f(x) = 0  \) as well.
\begin{proof}
Let \( g: A \to \R  \). Suppose \( \lim_{ x \to c } g(x) = 0   \) and suppose \( f(x) \) is bounded; that is, there exists \( M > 0  \) such that \( | f(x) | \leq M  \) for all \( x \in A  \). Using the Functional Limit Definition, suppose there exists \( \delta > 0  \) such that \( 0 < | x - c  | < \delta \). Hence, we have 
\begin{align*}
    | f(x)g(x) - 0  | &= | f(x) g(x) |  \\
                      &= | f(x) | | g(x) | \\
                      &\leq M | g(x) | \\
                      &< M \cdot \frac{ \epsilon  }{ M } \\
                      &= \epsilon.
\end{align*}
It follows from this that \( \lim_{ x \to c } f(x)g(x) = 0  \).
\end{proof}

\subsubsection{Exercise 4.2.9 (Infinite Limits).} The statement \( \lim_{ x \to 0 } 1/x^2 = \infty  \) certainly makes intuitive sense. To construct a rigorous definition in the challenge response style of Definition 4.1.1 for an infinite limit statement of this form, we replace the (arbitrarily small) \( \epsilon > 0  \) challenge with an (arbitrarily large) \( M > 0  \) challenge: 

\begin{tcolorbox}
\begin{defn}
We say \( \lim_{ x \to c } f(x) = \infty  \) to mean that for all \( M > 0  \), we can find a \( \delta > 0  \) such that whenever \( 0 < | x - c  | < \delta \), it follows that \( f(x) > M  \).
\end{defn}
\end{tcolorbox}
\begin{enumerate}
    \item[(a)] Show \( \lim_{ x \to 0 } 1/x^2 = \infty  \) in the sense described in the previous definition.
        \begin{proof}
        Let \( M > 0  \) and \( f(x) = 1/x^2 \). Choose \( \delta = 1 / \sqrt{ M }  \). Since \( 0 < | x  | < \delta \), we have 
        \[ x^2 < \delta^2 \iff \frac{ 1 }{ x^2 } > \frac{ 1 }{ \delta^2  }. \]
        But this means that 
        \[ f(x) = \frac{ 1 }{ x^2 } > \frac{ 1 }{ \delta^2 } = \frac{ 1 }{ 1/ M } = M. \]
        Hence, we have 
        \[  \lim_{ x \to 0 } f(x) = \infty. \]
        \end{proof} \item[(b)] Now, construct a definition for the statement \( \lim_{ x \to \infty  } f(x) = L  \). Show that \( \lim_{ x \to \infty  } 1/x = 0  \).
       \begin{tcolorbox}
       \begin{defn}
      Let \( f: A \to \R  \). We say \( \lim_{ x \to \infty  } f(x) = L  \) to mean for any \( \epsilon > 0  \), there exists \( x_0 \in A  \) where \( A \subseteq \R  \) such that for any \( x \geq x_0  \), we have 
      \[  | f(x) - L  | < \epsilon. \]
       \end{defn}
       \end{tcolorbox} 

       \begin{proof}[Solution]
        We start with some sketch work for what \( x_0  \) might be. Observe that 
        \begin{align*}
            | f(x) - L  | &= | \frac{ 1 }{ x } - 0  |  \\
                          &= \frac{ 1 }{ x } \\ 
                          &< \epsilon.
        \end{align*}
        Solving for \( x \), we get that 
        \[  x > \frac{ 1 }{ \epsilon  }.  \] Hence, \( x_0 = 1 / \epsilon   \).

        Let \( \epsilon > 0  \) and let \( f(x) = 1 / x  \). Choose \( x_0 = 1 / \epsilon  \). Suppose \( x > x_0 = 1 / \epsilon  \). Then we have that 
        \[ x > \frac{ 1 }{ \epsilon  } \iff \frac{ 1 }{ x } < \epsilon. \]
        Hence, we have that 
        \[  | f(x) - 0  | < \epsilon. \]
        This means \( \lim_{ x \to \infty  } f(x) = 0  \).
        \end{proof}
\end{enumerate}


\subsubsection{Exercise 4.2.10}

Introductory calculus courses typically refer to the \textit{right-hand limit} of a function as the limit obtained by "letting \( x \) approach \( a \) from the right-hand side"

\begin{enumerate}
    \item[(a)] Give a proper definition in the style of Definition 4.1.1 for the right-hand and left-hand limit statements: 

        \begin{center}
            \( \lim_{ x \to a^{+} } f(x) = L  \) and \( \lim_{ x \to a^{-} } f(x) = M. \) 
        \end{center}
        \begin{tcolorbox}
        \begin{defn}
        We say that
        \[ \lim_{ x \to a^{+}  } f(x) = L  \text{~and~} \lim_{ x \to a^{-} } f(x) = M  \]  
        if for all \( \epsilon > 0  \), there exists a \(  \delta > 0 \) such that whenever \( 0 < | x - c  | < \delta \)  and \( 0 < | c - x  | < \delta  \) we have 
        \begin{align*}
            | f(x) - L  | &< \epsilon, \\
            | f(x) - M  | &< \epsilon
        \end{align*}
        respectively.
        \end{defn}
        \end{tcolorbox}
    \item[(b)] Prove that \( \lim_{ x \to a } f(x) = L  \) if and only if both the right and left-hand limits equal to \( L  \).
        \begin{proof}
        Suppose \( \lim_{ x \to a } f(x) = L  \). Let \( \epsilon > 0  \). Then there exists \( \delta > 0  \) such that whenever \( 0 < | x - a  | < \delta  \), we have 
        \[ | f(x) - L  | < \epsilon. \tag{1} \]
        This denotes our right-hand limit. Note that this also works if we flip the order of our \( \delta \) assumption. Hence, we have \( 0 < | a - x  | < \delta \) such that (1) holds. 
        \end{proof}
        But this means our right-hand and left-hand limits are equal to each other.

        Now we show the converse. Suppose the right-hand and left-hand limits are equal to each other; that is, 
        \[ \lim_{ x \to a^{+}  } f(x) = \lim_{ x \to a^{-}  } f(x).\]
        We want to show that \( \lim_{ x \to a } f(x) = L  \). Since both \( | x - a  | < \delta \) and \( | a - x   | < \delta  \) hold for both limits, we immediately have that 
        \[  | f(x) - L  | < \epsilon. \]
        Hence, we have \( \lim_{ x \to a } f(x) = L  \).
\end{enumerate}




\subsubsection{Exercise 4.2.11(Squeeze Theorem).} Let \( f,g, \) and \( h \) satisfy \( f(x) \leq g(x) \leq h(x) \) for all \( x  \) in some common domain \( A  \). Suppose \( f(x) \leq g(x) \leq h(x) \ \). If \( \lim_{ x \to c  } f(x) = L  \) and \( \lim_{ x \to c } h(x) = L  \) at some limit point \( c \) of \( A  \), show 
\[  \lim_{ x \to c } g(x) = L  \] as well.
\begin{proof}
Let \( (x_n), (y_n), (z_n) \subseteq A   \). Suppose \( \lim_{ x \to c } f(x) = L  \) and \( \lim_{ x \to c } h(x) = L  \) at some limit point \( c \) of \( A  \). Using the Sequential Criterion for Functional Limits, let \( f(x_n) \to L  \) and \( h(z_n) \to L  \). There exists \( N \in \N  \) such that for any \( n \geq N  \), we have 
\[ f(x_n) \leq g(y_n) \leq h(z_n) \iff L \leq g(y_n) \leq L. \]
Hence, we have \( g(y_n) \to L  \) by the Squeeze Theorem for Sequences. By the Sequential Criterion, we must have \( \lim_{ x \to c } g(x) = L  \) as well.
\end{proof}

Another proof using the definition of Functional Limits directly.

\begin{proof}
Let \( f, g, h \) satisfy \( f(x) \leq g(x) \leq h(x) \) for all \( x  \) in some common domain \( A  \). Since \( \lim_{ x \to c } f(x) = L  \) and \( \lim_{ x \to c } h(x) = L  \), we know that 
\[ \lim_{ x \to c } [h(x) - f(x)] = 0.  \]
Suppose there exists \( \delta > 0  \) such that \(  0 < | x - c  | < \delta  \). Choose \( \delta = \min \{ \delta_1, \delta_2  \}  \). Since \( f(x) \leq g(x) \leq h(x) \), we have 
\begin{align*}
    | g(x) - L  | &\leq | h(x) - L  |  \\
                  &= | h(x) - f(x) + f(x) - L  | \\
                  &\leq | h(x) - f(x)  | + |  f(x) - L  | \\
                  &< \frac{ \epsilon  }{ 2 } + \frac{ \epsilon  }{ 2 } \\
                  &= \epsilon.
\end{align*}
But, this means that \( \lim_{ x \to a } g(x) = L.  \)
\end{proof}

Another proof 

\begin{proof}
Let \( f, g, h \) satisfy \( f(x) \leq g(x) \leq h(x) \) for all \( x  \) in some common domain \( A  \). 
Suppose there exists \( \delta > 0  \) such that \(  0 < | x - c  | < \delta  \). Choose \( \delta = \min \{ \delta_1, \delta_2, \delta_3, \delta_4  \}  \). Since \( f(x) \leq g(x) \leq h(x) \), we have 
\begin{align*}
    | g(x) - L  | &= | g(x) - h(x) + h(x) - L  |  \\
                  &\leq | g(x) - f(x) + f(x) - h(x)  | + | h(x) - L  | \\ 
                  &\leq | g(x) - f(x) | + | f(x) - h(x)  | + | h(x) - L  | \\
                  &\leq | h(x) - f(x)  | + | f(x) - L  | + | L - h(x) | + | h(x) - L  | \\
                  &\leq | h(x) - L| + | L - f(x) | + 2| h(x) - L  | \\
                  &< \frac{ \epsilon  }{ 4 } + \frac{ \epsilon  }{ 4 } + \frac{ 2 \epsilon  }{ 4 } \\
                  &= \epsilon.
\end{align*}
Hence, it follows that 
\[  \lim_{ x \to a } g(x) = L. \]
\end{proof}









\section{Continuous Functions}

\begin{tcolorbox}
    \begin{defn}[Continuity]
    A function \( f: A \to \R  \) is \textit{continuous at a point} \( c \in A  \) if, for all \( \epsilon > 0  \), there exists \( \delta > 0  \) such that whenever \( | x - c  | < \delta \) (and \( x \in A  \)) it follows that \( | f(x) - f(c) | < \epsilon  \). 
    If \( f \) is continuous at every point in the domain \( A  \), then we say that \( f \) \textit{continuous} on \( A  \).
    \end{defn}
\end{tcolorbox}

The difference between this definition and the definition for functional limits is that we require the limit point \( c  \) of \( A  \) to be in the domain of \(  f\). The value of \( f(c) \) is the value of \( \lim_{ x \to c } f(x)  \). It is indeed possible to shorten this definition to say that \( f  \) is continuous at \( c \in A  \) if 
\[  \lim_{ x \to c } f(x) = f(c) \tag{1}\]
so as long as \( c  \) is a limit point of \( A  \). The equation above gets undefined if \( c \) is an isolated point of \( A  \). But this does not stop \( f \) from being continuous at the point \( c \). In fact, functions can still be continuous at their isolated points such as \( c \).

We observed in the previous section that functional limits can be formulated using sequences from chapter 2. The same can be done for continuity. 

\begin{tcolorbox}
    \begin{thm}[Characterizations of Continuity]
    Let \( f: A \to \R  \), and let \( c \in A  \). The function \( f \) is continuous at \( c \) if and only if any one of the following three conditions is met:
    \begin{enumerate}
        \item[(i)] For all \( \epsilon > 0  \), there exists \( \delta > 0  \) such that \( | x - c  | < \delta \) (and \( x \in A  \)) implies \( | f(x) - f(c) | < \epsilon; \)
        \item[(ii)] For all \( V_{\epsilon }(f(c)) \), there exists a \( V_{\delta}(c)  \) with the property that \( x \in V_{\delta} (c) \) (and \( x \in A  \)) implies \( f(x) \in V_{\epsilon }(f(c))  \);
        \item[(iii)] For all \( (x_n) \to c \) (with \( x_n \in A
            \)), it follows that \( f(x_n) \to f(c) \).

            If \( c \) is limit point of \( A  \), then the above conditions are equivalent to 
    \item[(iv)] \( \lim_{ x \to c } f(x)  = f(c)\).
    \end{enumerate}
    \end{thm}
\end{tcolorbox}

\begin{proof}
    (i) \( \iff \) (ii) Let \( V_{\epsilon}(f(c)) \) and let \( \epsilon > 0  \). By assumption, there exists \( \delta > 0  \) such that \( | x - c  | < \delta \) (and \( x \in A  \)) implies 
    \[ | f(x) - f(c) | < \epsilon. \]
    We can rephrase in terms of \( \epsilon  \) and \( \delta \) neighborhoods. Hence, the statement above is just equivalent to the statement that for all \( V_{\epsilon }(f(c)) \), there exists \( V_{\delta}(c) \) such that \( x \in V_{\delta}(c) \) which implies that \( f(x) \in V_{\epsilon }(f(c)) \). 

     (iii) \( \iff \) (ii)  First we show (ii) holds. Let \( (x_n) \to c  \) such that \( f(x_n) \to f(c)  \). In other words, \( f(x_n) \in V_{\epsilon }(f(c)) \). For sake of contradiction, that \( f(x) \notin V_{\epsilon }(f(c)) \). This means there exist \( \epsilon_0  \) such that for all \( \delta > 0  \) with the property that \( | x - c  | < \delta \) that 
    \[  | f(x) - f(c)  | \geq \epsilon_0. \] Let \( \delta = 1 / n  \) and suppose we pick \( x_n \in V_{\delta}(c) \) such that \( f(x_n) \notin V_{\epsilon }(f(c)) \). But this contradicts our assumption that \( f(x_n) \in V_{\epsilon }(f(c)) \). Hence, it must be the case that \( f(x) \in V_{\epsilon }(f(c)) \).

    Now we want to show that \( (iii) \) holds. Let \( (x_n) \subseteq A  \).
    (with \( x_n \in A  \)). Since \( x_n \in A  \) \( x_n \neq c  \) or \( x_n = c  \). If \( x_n = c  \), then it immediately follows that \( f(x_n) \to f(c) \) given there exists \( | x_n  - c | < \delta \). Suppose \( x_n \neq c  \). Pick \( x_n \in V_{\delta}(c) \) for some \( \delta > 0  \). Then by assumption, we have \( f(x_n) \in V_{\epsilon }(f(c)) \). Hence, \( f(x_n) \to f(c) \).

    To show (i) \( \implies \) (iv), suppose \( c  \) is a limit point of \( A  \). Let \( x \in A  \). Since \( c \in A \), either \( x \neq c  \) or \( x = c  \). The conclusion follows immediately from the latter statement. Suppose \( x_n \neq c  \). By assumption, there exists \( \delta > 0  \) such that \( | x_n -c  | < \delta \). From this, it follows that 
    \[  |  f(x) - f(c)  | < \epsilon. \]
    And hence, 
    \[ \lim_{ x \to c } f(x) = f(c). \]
\end{proof}

\begin{tcolorbox}
    \begin{cor}[Criterion for Discontinuity]
    Let \( f: A \to \R  \), and let \( c \in A  \) be a limit point of \( A  \). If there exists a sequence \( (x_n) \subseteq A  \) where \( (x_n) \to c  \) but such that \( f(x_n) \not \to f(c) \), we may conclude that \( f \) is not continuous at \( c \).
    \end{cor}
\end{tcolorbox}

This sequential characterization of continuity allows us to use all the results that we know of when it comes to sequences from Chapter 2. 


\begin{tcolorbox}
    \begin{thm}[Algebraic Continuity Theorem] 
    Assume \( f: A \to \R  \) and \( g: A \to \R  \) are continuous at point \( c \in A  \). Then, 
    \begin{enumerate}
        \item[(i)] \( kf(x)  \) is continuous at \( c  \) for all \( k \in \R  \);
        \item[(ii)] \( f(x) + g(x) \) is continuous at \( c  \) for all \( k \in \R  \);
        \item[(iii)] \( f(x) g(x) \) is continuous at \( c  \); and 
        \item[(iv)] \( f(x) / g(x)  \) is continuous at \( c  \), provided the quotient is defined.
    \end{enumerate}
    \end{thm}
\end{tcolorbox}

\begin{proof}
All of these statements can be derived from the Characterizations of Continuity Theorem and the Algebraic Functional Limit Theorem.
\end{proof}


\begin{ex}
All polynomials are continuous on \( \R  \). In fact, rational functions (Quotients of polynomials) are continuous wherever they are defined. Consider the identity function \( g(x) = x  \). Since \( | g(x) - g(c) | = | x - c  |  \), we can respond to a given \( \epsilon > 0 \) by choosing \( \delta = \epsilon  \) such that \( g  \) is continuous on all of \( \R  \). Furthermore, this argument gets much simpler when we consider a constant function such as \( f(x) = k  \). Since any arbitrary polynomial 
\[  p(x) = a_0 + a_1 x + a_2 x^2 + \dots + a_n x^n \]
consists of sums and products of \( g(x) \) with different constant functions, we can conclude that \( p(x) \) is continuous. On the other hand, the Algebraic Continuity Theorem implies that quotients of polynomials are continuous as long as the denominator is not zero.
\end{ex}

\begin{ex}
In the sinuisodal example in the last section, we noticed that the oscillations of \( \sin (1/x ) \) are so rapid near the origin that \( \lim_{ x \to 0  } \sin (1/x) \) does not exist. Consider the function, 
\[  g(x) = 
\begin{cases}
    x \sin (1/x) &\text{~if~} x \neq 0 \\ 
    0 &\text{~if~}  x = 0. 
\end{cases} \]
\end{ex}
Suppose we want to observe the continuity of \( g \) at \( c = 0  \). We can do this by the following:
\[ | g(x) - g(0) | = | x \sin(1/x) - 0  | \leq | x |   \]
since \( | \sin (x)  |  \leq 1 \). Given \( \epsilon > 0 \), choose \( \delta = \epsilon  \) such that whenever \( | x | < \delta \) it follows that \( | g(x) - g(0) | < \epsilon  \). Thus, \( g  \) is continuous at the origin. 


\begin{ex}
    Consider the greatest integer function \( h(x) = [[x]] \) which for each \( x \in \R  \) returns the largest integer \( n \in \Z  \) such that \( n \leq x  \). In previous math classes, this step function is observed to have discontinuous jumps at each integer value of its domain. We can show this more rigorously using the tools we have at our disposal. Given \( m \in \Z  \), define the sequence \( (x_n)  \) by \( x_n = m -  1/n \). It follows that \( (x_n) \to m \), but not that 
    \[  h(x_n) \to (m-1), \]
    which does not equal \( m = h(m) \). By the Criterion for Discontinuity, we see that \( h \) fails to be continuous at each \( m \in \Z  \). Suppose we want to see why \( h  \) is continuous at a point \( c \notin \Z \). Given any \( \epsilon > 0  \), we must find a \( \delta-\)neighborhood \( V_{\delta}(c) \) such that \( x \in V_{\delta}(c) \) implies \( h(x) \in V_{\epsilon }(h(c)) \). We know that \( c \in \R  \) falls between consecutive integers \( n < c < n+1 \) for some \(  n \in \Z  \). Taking \( \delta = \min \{ c -n , (n+1) - c  \}  \), then it follows from definition of \( h \) that \( h(x) = h(c) \). Thus, we certainly have that \( h(x) \in V_{\epsilon }(h(c)) \) whenever \( x \in V_{\delta}(c) \). This proof actually implies that our \( \delta \) is not dependent on the value of \( \epsilon > 0  \). 
\end{ex}

\begin{ex}
Consider \( f(x) = \sqrt{ x }  \) defined on \( A = \{ x \in \R : x \geq 0 \}  \). Exercise 2.3.1 outlines a sequential proof that \( f  \) is continuous on \( A  \). Show \( f \) is continuous below.
\end{ex}

What about functions like \( h(x) = \sqrt{ 3x^2 + 5  }   \) is continuous. Hence, a Compositions of Continuous functions type theorem is needed to show that \( h(x) \) is continuous on its domain.


\begin{tcolorbox}
    \begin{thm}[Composition of Continuous Functions]
    Given \( f: A \to \R  \) and \( g: B \to \R  \), assume that the range \( f(A) = \{ f(x) : x \in A  \}  \) is contained in the domain \( B  \) so that the composition \( g \circ f(x) = g(f(x)) \) is defined on \( A  \). If \( f \) is continuous at \( c \in A  \), and \( g  \) is continuous at \( f(c) \in B  \), then \( g \circ f  \) is continuous at \( c  \).

    \end{thm}
\end{tcolorbox}

\begin{proof}
Exercise 4.3.3.
\end{proof}

\subsection{Exercises}

\subsubsection{Exercise 4.3.1} Let \( g(x) = \sqrt[3]{x }  \).
\begin{enumerate}
    \item[(a)] Prove that \( g \) is continuous at \( c = 0  \).
        \begin{proof}
            Let \( \epsilon > 0  \). Suppose \( c = 0  \). Let \( c \in A  \) and \( g(x) = \sqrt[3]{ x }  \). Since \( | x | < \delta  \), we can say that \( |\sqrt[3]{ x }| < \sqrt[3]{ \delta }   \). Then choose \( \delta = \epsilon^3 \) such that 
            \begin{align*}
                | g(x)  - g(c) | &= | \sqrt[3]{ x } - 0  |  \\
                                 &= | \sqrt[3]{ x }  |  \\
                                 &< \sqrt[3]{ \epsilon^3 } \\
                                 &= \epsilon.
            \end{align*}
            Hence, we have that 
            \[  \lim_{ x \to 0 } \sqrt[3]{ x } = 0 . \]
        \end{proof}
    \item[(b)] Prove that \( g  \) is continuous at a point \( c \neq 0  \). (The identity \( a^3 - b^3 = (a-b)(a^2 + ab + b^2)  \) will be helpful.)
        \begin{proof}
        Now let \( c \neq  0  \). Let \( \epsilon > 0  \) and suppose \( | x - c  | < \delta \). By the given identity \( a^3 - b^3 = (a-b)(a^2 +ab +b^2) \), we can write 
        \begin{align*}
            | g(x) - g(c) | &= | \sqrt[3]{x} - \sqrt[3]{c}     |  \\
                            &= \Big| x^{3/9} - c^{3/9} \Big| \\  
                            &= | x^{1/9} - c^{1/9} | | x^{2/9} + x^{2/9}c^{2/9} + c^{2/9} |.
        \end{align*}
        Since \( | x - c  | < \delta \), we have 
        \begin{align*}
            x^{1/9} - c^{1/9}  &< (\delta + c)^{1/9} - c^{1/9} \tag{1}  \\
            x^{2/9} +x^{1/9}c^{1/9} + c^{2/9}  &< (\delta + c )^{2/9} + x^{1/9}c^{1/9} + c^{1/9} \tag{2} \\
            x^{1/9} c^{1/9} &< (\delta + c )^{1/9} c^{1/9} \tag{3}.
        \end{align*}
        By using the identity again, we write
        \begin{align*}
            | g(x) - g(c) | &= | x^{1/9} - c^{1/9} | | x^{2/9} + x^{2/9}c^{2/9} + c^{2/9} |  \\
                            &< [(\delta + c)^{1/9} - c^{1/9} ] [(\delta + c )^{2/9} + (\delta + c )^{1/9} c^{1/9} + c^{1/9}] \\
                            &= (\delta + c )^{3/9} - c^{3/9}  \\  
                            &= (\delta + c)^{1/3} - c^{1/3}.  \tag{4} \\
        \end{align*}
        Now let \( \delta = \min \{ \delta_1, \delta_2  \}  \) such that (4) implies
        \[ (\delta + c)^{1/3} - c^{1/3} = (\delta^3 - c + c )^{1/3} - c^{1/3} = \epsilon + c^{1/3} - c^{1/3} = \epsilon. \]
        Hence, we have \( | g(x) - g(c) | < \epsilon  \) implying that
        \[ \lim_{ x \to c } g(x) = g(c). \] 
        \end{proof}
\end{enumerate}

\subsubsection{Exercise 4.3.3} 
\begin{enumerate}
    \item[(a)] Supply a proof for Theorem 4.2.3 using the \( \epsilon - \delta \) characterization of continuity.
        \begin{proof}
        Let \( f: A \to \R  \) and \( g: B \to \R  \). Assume that the range 
        \[  f(A) = \{ f(x) : x \in A  \}  \] is contained in the domain \( B  \) so that the composition \( g \circ f (x) = g(f(x)) \) is defined on \( A  \). Suppose \( f  \) is continuous at \( c \in A  \) and \( g  \) is continuous at \( f(c) \in B  \). Let \( \epsilon > 0  \). Since \( f \) is continuous at \( c \in  A  \), we can construct \( \delta > 0  \) such that  \( | x - c | < \delta  \) where
        \[ | f(x) - f(c)   | < \epsilon. \] Let \( \epsilon  = \delta  \). Then assume 
        \[ | f(x) - f(c)  | < \delta \] since \( g \) is continuous at \( f(c) \in B  \). Then we immediately have that 
        \[ | g \circ f (x) - g \circ f(c) | = | g(f(x)) - g(f(c)) | < \epsilon. \]
        Hence, we have 
        \[  \lim_{ x  \to c } g \circ f (x) = g \circ f(c). \]
        \end{proof}
    \item[(b)] Give another proof of this theorem using the sequential characterization of continuity.
        \begin{proof}
        Let \( f(x_n) \subseteq f(A) \) and \( (x_n) \subseteq A \) where the image 
        \[ f(A) = \{ f(x) : x \in A  \} .\]
        Let \( \epsilon > 0  \). Since \( (x_n) \to c  \), there exists \( N \in \N  \) such that for any \( n \geq N  \), we have 
        \[  | f(x_n)  - f(c)| < \epsilon. \]
        But \( g(f(x_n0) \to g(f(c)) \) since \( g \) is continuous at \( f(c)\in B  \) so we end up having 
        \begin{align*}
           | g \circ f(x_n) - g \circ f(c) |  &= |  g(f(x_n)) - g(f(c))| < \epsilon.
        \end{align*}
        Hence, we have 
        \[ \lim g \circ f(x_n) = g \circ f (c).  \]
        Note that this fact follows immediately if \( f(x_n) = f(c) \) and \( x_n = c  \).
        \end{proof}
\end{enumerate}


\subsubsection{Exercise 4.3.5} Show using definition 4.3.1 that if \( c \) is an isolated point of \( A \subseteq \R  \), then \( f: A \to \R  \) is continuous at \( c \).
\begin{proof}
    Since \( c \in A  \) and \( c  \) is an isolated point of \( A \subseteq \R  \), we must have \( x = c  \) and \( f(x) = f(c) \) where \( x \in A  \) is an arbitrary point. This follows from the fact that there exists an \( \epsilon - \)neighborhood such that \( V_{\epsilon }(x) \cap A = \emptyset \). Let \( \epsilon > 0  \). Then choose \( \delta = \epsilon  \) such that whenever \( | x - c  | < \delta \), we have that 
\[ | f(x) - f(c) | = | f(c) - f(c) | = 0 < \epsilon. \]
\end{proof}




\subsubsection{Exercise 4.3.7} Assume \( h: \R \to \R  \) is continuous on \( \R  \) and let \( K = \{ x : h(x) = 0  \}  \). Show \( K \) is a closed set.
\begin{proof}
Let \( h: \R \to \R  \) be a continuous function on \( \R  \).  Let \( c \in K  \). Since \( h \) is continuous on \( \R  \) we can use the Sequential Criterion. Let \( x_n \in K  \). There exists  \( (x_n) \subseteq K\) such that \( (x_n) \to  c \) implying that \( h(x_n) \to h(c) \). Since every \( x \in K  \) has the property that \( h(x) = 0  \) and \( c \in K  \), we have that \( h(c) = 0  \). Hence, \( h(x_n) \to h(c) \) is contained in \( K \). Thus, \( K \) is closed. 
\end{proof}






\subsubsection{Exercise 4.3.11 (Contraction Mapping Theorem).} Let \( f  \) be a function defined on all of \( \R  \), and assume there is a constant \( c  \) such that \(  0 < c < 1 \) and 
\[  | f(x) - f(y) | \leq c | x - y |  \]
for all \( x,y \in \R  \).

\begin{enumerate}
    \item[(a)] Show that \( f  \) is continuous on \( \R  \).
        \begin{proof}
        Let \( f: \R  \to \R  \) and let \( x,c \in \R   \) where \( c  \) is a limit point of \( A  \). We want to show that \( f \) is continuous on \( \R  \); that is, we want to show that 
        \[  \lim_{ x \to c } f(x) = f(c). \]
        Let \( \epsilon > 0  \). Choose \( \delta = \epsilon / c  \) such that whenever \( | x - c  | < \delta \) we have 
        \[  | f(x) - f(c) | \leq k | x - c  | < k \cdot \frac{ \epsilon  }{ k } = \epsilon. \]
        Hence, we have 
        \[ \lim_{ x \to c } f(x) = f(c). \]
        \end{proof}
    \item[(b)] Pick some point \( y_1 \in \R  \) and construct the sequence 
        \[  (y_1, f(y_1), f(f(y_1)), ...).\]
        In general, if \( y_{n+1} = f(y_n) \), show that the resulting sequence \( (y_n) \) is a Cauchy sequence. Hence, we may let \( y = \lim y_n \).
        \begin{proof}
        Observe that for any fixed \( n \in \N  \), 
        \[ | y_{m+1} - y_{m+1} | = | f(y_m) - f(y_{m+1}) | \leq c | y_m - y_{m_+1} |. \] We can show this via induction that 
        \begin{align*}
            | y_{m+1} - y_{m+2} | &\leq c | y_{m} - y_{m+1}  |  \\
                                  &\leq c^2 | y_{m-1} - y_m | \\
                                  &\vdots \\
                                  &\leq c^m | y-1 - y_2 |.
        \end{align*}
        The fact that \( 0 < c <  1 \) means that \( \sum_{ n=1 }^{ \infty  } c^n  \) converges (because it is a geometric series) enables us to conclude that \( (y_n) \) is Cauchy sequence. To see this, observe that 
        \begin{align*}
            | y_m  - y_n  | &= | y_m - y_{m+1} + y_{m+1} + \dots + y_{n-1} - y_n |  \\
                            &\leq c^{m-1} | y_1 - y_2  | + c^m | y_1 - y_2  | + \dots c^{n-2} | y_1 - y_2  | \\ 
                            &= c^{m-1} | y_1 - y_2  | (1 + c + \dots + c^{n - m - r }) \\
                            &< c^{m-1} | y_1 - y_2  | \Big( \frac{ 1 }{ 1 - c  }  \Big).
        \end{align*}
    Let \( \epsilon > 0  \), and choose \( N \in \N  \) large enough so that \( c^{N-1} < \epsilon 1 - c / | y_1 - y_2  |   \). Then the previous calculation shows that \( n > m \geq N  \) implies \( | y_m - y_n  | < \epsilon  \); that is, we have
    \begin{align*}
        | y_{m+1} - y_{n+1} | &\leq | y_m - y_n  |  \\
                            &< c^{N-1} | y_1 - y_2  | \Big( \frac{ 1 }{ 1 - c  }   \Big) \\
                            &< \frac{ \epsilon (1 - c ) }{ | y_1 - y_2  |  } \cdot \frac{ | y_1 - y_2  |  }{ 1 -c  } \\ 
                            &= \epsilon.
    \end{align*}
    Hence, \( (y_n) \) is Cauchy Sequence.
\end{proof}
\item[(c)] Prove that \( y  \) is a fixed point of \( f \); that is, \( f(y) = y  \) and that it is unique in this regard.
    \begin{proof}
    Since \( f  \) is continuous on \( \R  \) and \( (y_n) \subseteq \R  \) is a Cauchy sequence and that \( \lim y_n = y  \) for all \( n \in \N  \), we have that 
    \[ f(y_n) = y_{n+1} \to f(y) = y.  \]
    Hence, \( y \) is a fixed point of \( f \).
    \end{proof}
    \item[(d)] Finally, prove that if \( x \) is \textit{any} arbitrary point in \( \R  \) then the sequence \( (x, f(x), f(f(x)),...) \) converges to \( y \) defined in (b).
        \begin{proof}
            Let us fix \( x \in \R  \) where \( (x_n) \subseteq \R  \) is a sequence of points. Since the \( f \) is defined like  \( f(x_{n}) = x_{n+1}   \). Since \( x \in \R  \) is fixed, we know that \( f(x_n) \to f(x) \) and that \( (x_{n+1}) \to  x  \). Hence, \( f(x) = x  \) as defined in (b). 
        \end{proof}
\end{enumerate}


\subsubsection{Exercise 4.3.12} Let \( F \subseteq \R  \) be a nonempty closed set and define \( g(x) = \inf \{ | x - a  | : a \in  F  \}  \). Show that \( g \) is continuous on all of \( \R  \) and \( g(x) \neq  0 \) for all \( x \notin F  \).
\begin{proof}
Let \( F \subseteq \R  \) be a nonempty closed set. Let \( x \in F  \). Since \( F \) is a closed, let \( (x_n)  \) be a Cauchy sequence such that \( (x_n) \to x  \) that is contained in \( F \). But since \( g: F \to \R  \), we must have 
\[ | g(x_n) - g(x) | < \epsilon \]
by the Sequential Criterion of Continuity. This means \( g \) is continuous on all of \( \R  \).

Let \( x \notin F  \). Suppose for sake of contradiction that \( g(x) = 0  \). Let \( (x_n) \subseteq F  \) where \( x_n \in F  \). Since \( F  \) is closed, \( (x_n) \) is a Cauchy sequence such that \( (x_n) \to x  \). But since \( g \) is continuous \( x  \) has to be contained in \( F  \) which is a contradiction. Hence, it must be that \( g(x) \neq 0  \).
\end{proof}









\subsubsection{Exercise 4.3.13} Let \( f \) be a function defined on all of \( \R  \) that satisfies the additive condition \( f(x+y) = f(x) + f(y) \) for all \( x,y \in \R  \).

\begin{enumerate}
    \item[(a)] Show that \( f(0) = 0  \) and that \( f(-x) = - f(x) \) for all \( x \in \R  \).
        \begin{proof}
        By using the linearity property of \( f \), we have \( f(0 + 0) = f(0) + f(0)  \) which implies \( f(0) = 0 \). For the inverse property, suppose \( x \in \R  \) such that \( x + (-x) = 0  \). Then using the linearity property of \( f \), we have \( f(x) + (-x)) = f(x) + f(-x)   \). Since \( f(0) = 0  \), we have that \( f(x) + f(-x) \) implies \( f(-x) = -f(x)   \).
        \end{proof}
    \item[(b)] Let \( k = f(1) \). Show that \( f(n) = k n \) for all \( n \in \N  \), and then prove that \( f(z) = kz \) for all \( z \in \Z  \). Now, prove that \( f(r) = kr \) for any rational number \( r \).
        \begin{proof}
        Let \( k = f(1) \). We proceed to show \( f(n) = kn \) by inducting on \( n \in \N  \). Let our base case be \( n = 1  \). If \( n = 1  \), then \( f(1) =  k \). Now suppose \( f(n) \) holds for \( 1 \leq n \leq \ell - 1  \). Then 
        \begin{align*}
           f(\ell) &= k \ell \\
                   &= k (\ell + 1 - 1 ) \\
                   &= k(\ell - 1) + k \\
                   &= f(\ell - 1 ) + f(1).
        \end{align*}
        Hence, \( f(n) = kn \) for all \( n \in \N  \). To prove \( f(z) = kz \) for all \( z \in \ Z \) we can just prove it for all negative integers and zero. This is easy to see when \( z = 0  \). It's also easy to see that \( f \) holds for \( z \in \Z^{-} \). Since \( f(-x) = -f(x) \), we have that \( f(-\ell) = - f(\ell) \). Since \( f(n)  \) holds for all \( n \in \Z^{+} \). Hence, \( f(z) = nz \) for all \( z \in \Z \). 
    Before proving \( f(r) = kr  \) for any rational number \( r  \), let us consider \( 1/n  \) where \( n \in \N  \). Note that 
    \begin{align*}
         k &= f(1)  \\
           &= f \Big( \frac{ 1 }{ n } + \frac{ 1 }{ n }  + \dots + \frac{ 1 }{ n }  \Big) \\
           &= n f \Big( \frac{ 1 }{ n }  \Big).
    \end{align*}
    Applying this for any given \( r \in \Q  \), we can see that 
    \begin{align*}
        f(m/n) &= f \Big( \frac{ 1 }{ n } + \frac{ 1 }{ n } + \dots + \frac{ 1 }{ n }  \Big) \\
               &= m f \Big( \frac{ 1 }{ n }  \Big) \\ 
               &= k \Big( \frac{ m }{ n }  \Big). \\
    \end{align*}
    We can prove that this holds for any rational number \( r < 0  \) by using a similar strategy to the used to prove the negative integers case above.
        \end{proof}
    \item[(c)] Show that if \( f \) is continuous at \( x = 0  \), then \( f \) is continuous at every point in \( \R  \) and conclude that \( f(x) = k x  \) for all \( x \in \R  \). Thus, any additive function that is continuous at \( x = 0  \) must necessarily be a linear function through the origin.
        \begin{proof}
        Assume \( f \) is continuous at \( x = 0  \). Let \( \epsilon > 0  \). Choose \( \delta = \epsilon / k  \) such that whenever \( | x - c  | < \delta  \), we have that 
        \begin{align*}
            | f(x) - f(c) | &= | kx - kc  |  \\
                            &= k | x - c  | \\
                            &< k \cdot \frac{ \epsilon  }{  k } \\
                            &= \epsilon.
        \end{align*}
        Hence, \( f  \) is continuous for \( c \neq 0  \). But \( f \) is also continuous at \( x = 0  \). Hence, \( f \) is continuous at every point in \( \R  \) and thus \( f(x) = kx \) for all \( x \in \R  \).
        \end{proof}
\end{enumerate}









% !TEX root =  ../../../main.tex 

\section{Continuous Functions on Compact Sets}

Given a function \( f: A \to \R  \) and a given subset \( B \subseteq A  \), the notation \( f(B) \) can be defined as the range of \( f \) over the set \( B  \); in other words, we have that 
\[  f(B) = \{ f(x) : x \in B  \}.  \]
We can describe properties such as subsets of \( \R  \) being open, closed, bounded, compact, perfect, and connected, but a more interesting analysis arises when see which ones are preserved when mapping \( B  \) to \( f(B) \) via a continuous function. 

For example, if \( B  \) is an open set and \( f \) is continuous, is the mapping \( f(B) \) necessarily open? The answer to this is no.  

Suppose \( f(x) =x^2  \) and \( B = (-1,1) \) is an open interval, then we have that the interval \( [0,1) \) is not open. What if \( B  \) is closed? The same conjecture actually leads to the same conclusion that \( f(B) \) is not closed as well. Consider the function
\[  g(x) = \frac{ 1 }{ 1 + x^2  }   \]
and the closed set \( B = [0, \infty ) = \{ x : x \geq 0  \}  \). Because we have that \( g(B) = (0,1] \) is not closed, we must conclude that continuous functions do not generally map from closed sets to closed sets. However, if \( B  \) is compact, then \( B  \) gets mapped to closed and bounded subsets by continuous functions.

\begin{tcolorbox}
    \begin{thm}[Preservation of Compact Sets]
    Let \( f: A \to \R  \) be continuous on \( A  \). If \( K \subseteq A  \) is compact, then \( f(K)  \) is compact as well.
    \end{thm}
\end{tcolorbox}

\begin{proof}
    Let \(  f: A \to \R  \) and \( K \subseteq A  \) be a compact set. Let \( (x_n) \subseteq K \) and \( (y_n) \subseteq f(K) \). Since \( K  \) is a compact set, there exists \( (x_{n_k}) \to x  \) such that \( x  \) is contained in \( K  \). Suppose \( f  \) is a continuous function. Define \( f(x_n) = y_n \). Since \( (x_{n_k}) \) converges to \( x  \) and \( f  \) is a continuous function, we have that 
    \[  f(x_{n_k}) = y_{n_k} \to f(x) = y. \]
    This means our subsequence \( (y_{n_k}) \subseteq f(K) \) converges to a limit \( y \) that is contained in \( f(K) \). Thus, \( f(K)  \) is a compact set.
\end{proof}

An extremely important result from this theorem deals with how compact sets are bounded and how they contain their supremums and infimums.




\begin{tcolorbox}
    \begin{thm}[Extreme Value Theorem]
    If \( f: K \to \R  \) is continuous on a compact set \( K \subseteq \R  \), then \( f  \) attains a maximum and minimum value. In other words, there exists \( x_0, x_1 \in K  \) such that \( f(x_0) \leq f(x) \leq f(x_1) \) for all \( x \in K  \).
    \end{thm}
\end{tcolorbox}

\begin{proof}
    Since \( f(K) \) is a compact set, we can set \( \alpha = \sup f(K) \) and know that \( \alpha \in f(K) \) from Exercise 3.3.1. It immediately follows that for some \( x_1 \in K  \), this element gets mapped to \( \alpha = f(x_1) \) since \( f  \) is a continuous function. Likewise, we have \( \beta \in f(K) \) such that for some \( x_0 \in K  \), \( x_0  \) gets mapped to \( \beta = f(x_0 ) \) by the same reasoning above. Hence, we have that for any \( x \in K  \), 
    \[  f(x_0) \leq f(x) \leq f(x_1). \]
\end{proof}

\subsection{Uniform Continuity}

We learned in the last section that polynomials are always continuous on \( \R  \). In this section, we show that these functions are actually uniform continuous on \( \R  \); that is, they are everywhere continuous.

\begin{ex}
\begin{enumerate}
    \item[(i)] Let \( f: \R \to \R  \) such that \( f(x) = 3x + 1  \). We want to show that this is continuous for any point \( c \in \R  \). Let \( \epsilon  > 0  \). Then choose \( \delta = \epsilon  / 3  \) such that whenever \( | x - c  | < \delta  \), we have that 
\begin{align*}
    | f(x) - f(c) | &= |  (3x + 1 ) - (3c + 1 ) |  \\
                    &= 3 | x - c  | \\
                    &< 3 \cdot \frac{ \epsilon  }{ 3 } \\
                    &= \epsilon.
\end{align*}
Hence, \( \lim_{ x \to c } f(x) = f(c) \). The key observation here is that our choice of \( \delta  \) is the same regardless of the point \( c \in \R  \), we are considering.
    \item[(ii)] Suppose how the situation changes when we consider another function, say, \( g(x) = x^2  \) and see how the choice of \( \delta  \) changes with each point \( c \in \R  \). Given \( c \in \R  \), observe that 
        \[  | g(x) - g(c) | = | x^2 - c^2  | = | x -c | | x + c  |.  \]
As was discussed in section 4.2, we need to upper bound \( | x + c  |  \), which, in this case, can be obtained by letting our choice \( \delta \) not exceed \( 1 \). This implies that all values of \( x  \) under consideration will fall in the interval \( (c - 1, c + 1 ) \). By using our assumption that \( | x - c  | < \delta  \) and letting \( \delta = 1  \), we have that 
\[  | x + c  | \leq | x | + | c  | \leq (| c  | + 1 ) + | c  | = 2 | c | + 1  . \] Now let \( \epsilon  > 0  \). If we choose \( \delta = \min \{ 1, \epsilon / (2 | c  | + 1 ) \} \), then assume \( | x - c  | < \delta  \) such that 

\[ | f(x) - f(c)  | = | x - c  | | x + c  | < \Big( \frac{ \epsilon  }{ 2 | c  | + 1  }  \Big) \cdot ( 2 | c  | + 1) = \epsilon. \]

Notice how our choice of \( \delta \) depended on our choice of \( c \in \R  \) where 
\[  \delta = \frac{ \epsilon  }{ 2 | c  | + 1  }. \]
This means that as our choice of \( c \in \R  \) gets bigger and bigger, our \( \delta - \)neighborhood must get smaller and smaller.
\end{enumerate}
\end{ex}


This leads us to our rigorous definition of what it means for a function to be uniform continuous.

\begin{tcolorbox}
\begin{defn}
A function \( f: A \to \R  \) is \textit{uniformly continuous} on \( A  \) if for every \( \epsilon > 0  \) there exists a \( \delta > 0  \) such that for all \( x,y \in A  \), \( | x - y  | < \delta  \) implies \( | f(x) - f(y) | < \epsilon. \)
\end{defn}
\end{tcolorbox}

The difference between regular continuity and uniform continuity is that regular implies continuity of a function at any point \( c \in \R  \) granted that our choice of \( \delta  \) is dependent on our choice of \( c \in \R  \) while uniform continuity implies that our choice of \( \delta  \) stays the same regardless of our choice of \( c \in \R  \).

On the other hand, saying that a function is not uniform continuous is to say that given some \( \epsilon > 0  \), there is not suitable choice of \( \delta > 0  \) that will be a valid response to our \( \epsilon  \) challenge. That is, every point \( c \in \R  \) has a unique \( \delta > 0   \).   

\begin{tcolorbox}
    \begin{thm}[Sequential Criterion for Absence of Uniform Continuity]
    A function \( f: A \to \R  \) fails to be uniformly continuous on \( A  \) if and only if there exists a particular \( \epsilon_0 > 0  \) and two sequences \( (x_n) \) and \( (y_n) \) in \( A  \) satisfying 
    \[ | x_n - y_n  | \to 0 \text{~but~} | f(x_n) - f(y_n) | \geq \epsilon_0. \]
    \end{thm}
\end{tcolorbox}

\begin{proof}
    (\( \Rightarrow \)) We can negate the definition of uniform continuity to help us prove this direction. Hence, for some \( \epsilon_0 > 0  \), choose \( \delta_n = 1/n  \) such that whenever we have  sequences \( (x_n) \) and \( (y_n) \) that satisfy
    \[ | x_n - y_n  | < \frac{ 1 }{ n }, \]
we have that 
\[ | f(x_n) - f(y_n) | \geq \epsilon_0. \]
Clearly, we have \( | x_n - y_n | \to 0  \) by the Squeeze Theorem for sequences.

(\( \Leftarrow \) ) Since \( | x_n - y_n | \to 0  \) for any \( n \geq N   \) for some \( N \in \N  \), we can see that any choice of \( \delta > 0  \) will not be a suitable response to the \( \epsilon_0  \) challenge; that is, \( | x_n - y_n | \to 0  \) implies that \( f  \) fails to be uniformly continuous on \( A  \). 
\end{proof}


\begin{ex}
    Consider the function \( h(x) = \sin (1/x) \). We can see that \( h(x) \) is continuous at every point in the open interval \( (0,1) \) but is not uniformly continuous on this interval. We can prove this by defining the following sequence \( (x_n) \) and \( (y_n) \) to be 
    \[  x_n = \frac{ 1 }{ \pi / 2 + 2n \pi } \text{~and~} y_n = \frac{ 1 }{ 3\pi / 2 + 2n \pi }. \]
\end{ex}
Since both sequences tend to zero, we have that \( | x_n - y_n | \to 0  \) such that letting \( \epsilon_0 = 2  \) leads to 
\begin{align*}
    | h(x_n) - h(y_n) | &= | \sin(\pi / 2 + 2n \pi ) - \sin( 3 \pi / 2 + 2n \pi)|  \\
                        &= | 2\cos(n \pi) - (-2\cos(n \pi)  | \\ 
                        &= 2 | \cos(n \pi) | \\  
                        &= 2.
\end{align*}

\begin{tcolorbox}
    \begin{thm}[Uniform Continuity on Compact Sets]
    A function that is continuous on a compact set \( K  \) is uniformly continuous on \( K \).
    \end{thm}
\end{tcolorbox}

\begin{proof}
Assume \( f: K \to \R  \) is continuous at every point of a compact set \( K \subseteq \R  \). Suppose for sake of contradiction that \( f \) is not uniformly continuous on \( K \). Then by the Sequential Criterion for Absence of Uniform Continuity, we have that for some \( \epsilon_0 > 0  \), we have two sequences \( (x_n) \) and \( (y_n) \) that satisfy the following property 
\[  | x_n - y_n | \to 0  \] which implies that 
\[  | f(x_n) - f(y_n) | \geq \epsilon_0. \]
Since \( K  \) is a compact set, we can find a subsequence \( x_{n_k} \) such that \( (x_{n_k}) \to x  \) where \( x \in K  \). Suppose we want to show that \( y_{n_k} \to x  \). Hence, we can use the Algebraic Limit Theorem to say that 
\[ \lim (y_{n_k}) = \lim ((y_{n_k} - x_{n_k}) + x_{n_k}) = \lim (y_{n_k} - x_{n_k}) + \lim x_{n_k} = 0 + x = x.  \]
Since \( f \) is continuous on \( x \in K  \), we have that \( f(x_{n_k}) = f(x) \) and \( f(y_{n_k}) = f(x) \). But this means that 
\[ \lim (f(x_{n_k}) - f(y_{n_k})) = 0   \]
that is, \( | f(x_{n_k}) - f(y_{n_k})  | < \epsilon \)
for all \( \epsilon > 0  \)
which contradicts our original assumption that 
\[  | f(x_n) - f(y_n) | \geq \epsilon_0 \]
for all \( n \in \N  \). Hence, \( f \) must be uniformly continuous on \( K \).
\end{proof}



\subsection{Definitions}


\begin{tcolorbox}
    \begin{thm}[Preservation of Compact Sets]
    Let \( f: A \to \R  \) be continuous on \( A  \). If \( K \subseteq A  \) is compact, then \( f(K)  \) is compact as well.
    \end{thm}
\end{tcolorbox}


\begin{tcolorbox}
    \begin{thm}[Extreme Value Theorem]
    If \( f: K \to \R  \) is continuous on a compact set \( K \subseteq \R  \), then \( f  \) attains a maximum and minimum value. In other words, there exists \( x_0, x_1 \in K  \) such that \( f(x_0) \leq f(x) \leq f(x_1) \) for all \( x \in K  \).
    \end{thm}
\end{tcolorbox}

\begin{tcolorbox}
\begin{defn}
A function \( f: A \to \R  \) is \textit{uniformly continuous} on \( A  \) if for every \( \epsilon > 0  \) there exists a \( \delta > 0  \) such that for all \( x,y \in A  \), \( | x - y  | < \delta  \) implies \( | f(x) - f(y) | < \epsilon. \)
\end{defn}
\end{tcolorbox}

\begin{tcolorbox}
    \begin{thm}[Sequential Criterion for Absence of Uniform Continuity]
    A function \( f: A \to \R  \) fails to be uniformly continuous on \( A  \) if and only if there exists a particular \( \epsilon_0 > 0  \) and two sequences \( (x_n) \) and \( (y_n) \) in \( A  \) satisfying 
    \[ | x_n - y_n  | \to 0 \text{~but~} | f(x_n) - f(y_n) | \geq \epsilon_0. \]
    \end{thm}
\end{tcolorbox}

\begin{tcolorbox}
    \begin{thm}[Uniform Continuity on Compact Sets]
    A function that is continuous on a compact set \( K  \) is uniformly continuous on \( K \).
    \end{thm}
\end{tcolorbox}
\subsection{Exercises}

\subsubsection{Exercise 4.4.1}

\begin{enumerate}
    \item[(a)] Show that \( f(x) = x^3  \) is continuous on all \( c \in \R  \).
        \begin{proof}
Let \( \epsilon > 0  \) and let \( c \in \R  \). Choose \( \delta = \min \{ 1, \epsilon / | (c+1)^2 + (c+1)c + c^2 |   \}   \). Let \( f(x) = x^3  \). Then whenever \( | x - c  | < \delta  \), we have that 
\begin{align*}
    | f(x) - f(c)  | &= |  x^3 - c^3  |  \\
                     &=|x - c   | | x^2 + xc + c^2  | \\ 
                     &< \frac{ \epsilon  }{ | (c+1)^2 + (c+1)c + c^2 |  }  \cdot | (c + 1)^2 + (c + 1)c + c^2  |  \\
                     &= \epsilon.
\end{align*}
Hence, we have that \(\lim_{ x  \to c  } f(x) = f(c) \) for any \( c \in \R  \).
        \end{proof}
    \item[(b)] Argue, using Theorem 4.4.5, that \( f  \) is not uniformly continuous on \( \R  \).
        \begin{proof}
            Let \( (x_n) = n    \) and \( (y_n) = n + 1 / n   \), then 
            \[  | x_n - y_n | = \Big| \frac{ 1 }{ n }  \Big| \to 0.  \]
            Then we have that 
            \begin{align*}
                | f(x_n) - f(y_n)  | &= \Big| n^3 - \Big( n + \frac{ 1 }{ n }  \Big)^3  \Big|  \\
                                     &= \Big| - \Big( 3n + \frac{ 3 }{ n } + \frac{ 1 }{ n^3 } \Big) \Big| \\
                                     &= 3n + \frac{ 3 }{ n } + \frac{ 1 }{ n^3 } \\  
                                     &\geq 3.
            \end{align*}
            Hence, there exists \( \epsilon_0 = 3  \) such that \( | f(x_n) - f(y_n) | \geq \epsilon_0. \) which implies \( f(x) = x^3  \) is not uniform continuous.
        \end{proof}
    \item[(c)] Show that \( f \) is uniformly continuous on any bounded subset of \( \R  \). 
        \begin{proof}
            Let \( (a,b) \) be a bounded subset of \( \R  \). Then let \( \epsilon > 0 \). Let \( x,y \in [a,b] \). Choose \( \delta =  \). Hence, whenever \( | x - y  | < \delta  \) we have that 
            \begin{align*}
                | f(x) - f(y) | &= | x^3 - y^3  |  \\
                                &= | x - y  | | x^2 + xy + y^2  | \\
                                &< \delta \cdot 3b^2 \\ 
                                &= \frac{ \epsilon  }{ 3b^2 }  \cdot 3b^2 \\
                                &= \epsilon.
            \end{align*}
            Hence, we have that \( f \) is uniformly continuous on the bounded set \( (a,b) \subseteq \R  \).
        \end{proof}
\end{enumerate}

\subsubsection{Exercise 4.4.2} 
\begin{enumerate}
    \item[(a)] Is \( f(x) = 1 / x  \) uniformly continuous on \( (0,1) \)? 
        \begin{proof}[Solution]
        No it is not. Let \( (x_n), (y_n)  \) be two sequences such that \( x_n = 1 / 2n \) and \( y_n = 1 / (2n+1) \). Observe that \( | x_n - y_n  | \to 0  \) since both sequences tend to zero. Now we have that 
        \begin{align*}
            | f(x_n) - f(y_n) | &= | 2n - 2n - 1  |
                                = 1 = \epsilon_0.
        \end{align*}
        Hence, \( f(x)  \) is not uniformly continuous on \( (0,1) \).
        \end{proof}
    \item[(b)] Is \( g(x) = \sqrt{ x^2 + 1  }  \) uniformly continuous on (0,1)?
        \begin{proof}
        Let \( \epsilon > 0  \) and choose \( \delta = \epsilon. \) Assume \( f(x) \) is defined on \( (0,1)  \). Then whenever \( | x - y  | < \delta  \), we have that 
            \begin{align*}
                | g(x) - g(y) | &= | \sqrt{ x^2 + 1  } - \sqrt{ y^2 + 1  }  |  \\
                                &=  \frac{ | x^2 - y^2  |  }{ | \sqrt{ x^2 + 1  } + \sqrt{ y^2 + 1  }  |  }   \\
                                &=  \frac{ | x - y  | | x + y  |  }{ | \sqrt{ x^2 + 1  } + \sqrt{ y^2 + 1  }   |  }  \\  
                                &< \delta \cdot \frac{ | x+ y  |  }{ |  \sqrt{ x^2 + 1  } + \sqrt{ y^2 + 1  }   |  }  \\
                                &< \delta \cdot \frac{ 2 }{ 2 }  \\
                                &= \epsilon .
            \end{align*}
            Hence, we have that \( g(x)  \) is uniform continuous on \( (0,1) \subseteq \R  \).
        \end{proof}
\end{enumerate}


\subsubsection{Exercise 4.4.3} Show that \( f(x) = 1 / x^2  \) is uniformly continuous on the set \( [1, \infty  ) \) but not on the set \( (0,1] \).
\begin{proof}
    Let \( \epsilon > 0  \). Choose \( \delta = \delta / 2  \) such that whenever \( | x - y  | < \delta  \) for any \( x,y \in [1,\infty ) \) we have that 
    \begin{align*}
        | f(x) - f(y) | &= \Big| \frac{ 1 }{ x^2  } - \frac{ 1 }{ y^2  }  \Big|   \\
                        &= \frac{ | y^2 - x^2  |  }{  | x^2 y^2  |  } \\
                        &= \frac{ | x - y  | | x + y  |  }{ | x^2  | | y^2  |  }   \\
                        &\leq \frac{ | x - y  |  }{ | x^2  | | y^2  |  } (| x | + | y | ) \tag{Triangle Inequality}\\
                        &< \delta \cdot \frac{ 1 }{ | x^2 y^2  |  } (| x  | + | y | ) \tag{\( | x - y  | < \delta  \)}\\ 
                        &= \delta \cdot \Big( \frac{ 1 }{ x y^2  } + \frac{ 1 }{ x^2 y  }  \Big) \\
                        &< \frac{ \epsilon  }{ 2 }  \cdot 2 \tag{\( x \geq  1  \) \text{and} \( \delta = \epsilon / 2  \) } \\  
                        &= \epsilon.
    \end{align*}
    Hence, \( f(x) = 1 / x^2  \) is a continuous function on \( [1, \infty) \).
    Now we want to show that \( f(x) \) is not uniformly continuous on \( (0,1] \). Hence, let \( (x_n), (y_n) \) be two sequences that are contained in \( (0,1] \). Defined these two sequences as follows: 
    \begin{align*}
        x_n &= \frac{ 1 }{ \sqrt{ 2n }  },  \\
        y_n &= \frac{ 1 }{ \sqrt{ 2n + 1  }  }.
    \end{align*}
    Since \( (x_n) \) and \( (y_n) \) both tend towards zero, we have that \( | x_n - y_n  | \to 0  \). Now consider \( | f(x_n) - f(y_n) |  \). We want to construct an \( \epsilon_0  \) such that \( | f(x_n) - f(y_n) | \geq \epsilon_0  \). Hence, observe that 
    \begin{align*}
        | f(x_n) - f(y_n)  | &= \Big| \Big( \frac{ 1 }{ 1 / \sqrt{ 2n }  } \Big)^2  - \Big( \frac{ 1  }{ 1 / \sqrt{ 2n + 1  }  } \Big)^2   \Big|   \\
                             &= |  2n - 2n - 1  | \\
                             &= 1 \\
                             &= \epsilon_0.
    \end{align*}
    Hence, we have that \( f(x) \) cannot be uniformly continuous on the interval \( (0, 1] \).
\end{proof}



\subsubsection{Exercise 4.4.4} Decide whether each of the following statements is true or false, justifying each conclusion.

\begin{enumerate}
    \item[(a)] If \( f \) is continuous on \( [a,b] \) with \( f(x) > 0  \) for all \( a \leq x \leq b  \), then \( 1 /  f \) is bounded on \( [a,b] \) (meaning \( 1/ f \) has bounded range).
        \begin{proof}[Solution]
            Let \( f  \) be a continuous function \( [a,b] \) with \( f(x) > 0  \) for all \( a \leq x \leq b  \). Since \( [a,b] \) are compact sets, and \( f  \) is continuous on \( [a,b]  \), we have \( f(a) \leq f(x) \leq f(b) \). Since \( f > 0  \) for all \( a \leq x \leq  b  \), we have that 
            \[  \frac{ 1 }{ f(b) } \leq \frac{ 1 }{ f(x) } \leq \frac{ 1 }{ f(a) }   \]
            which means \( 1 / f   \) is a bounded function on \( [a,b] \).
        \end{proof}
    \item[(b)] If \( f \) is uniformly continuous on a bounded set \( A  \), then \( f(A) \) is bounded.
        \begin{proof}[Solution]
        Let \( f  \) be a uniformly continuous function on a bounded set \( A  \). Hence, let \( \epsilon  = 1  \). There exists \( \delta > 0  \) such that whenever \( | x - y  | < \delta  \), we have that 
        \[  | f(x) -f(y) | < \epsilon. \]
Since \( A  \) is bounded, there exists a subsequence \( (x_n) \to x  \) where \( x  \) is a limit point of \( A  \). Suppose for sake of contradiction that \( f(A) \) is unbounded. Let \( \delta = 1  \). Since \( f  \) is unbounded, we have that whenever \( | x_n - x_m  | <  1 \) where \(  n \neq m  \), we have 
    \[ | f(x_n) - f(x_m) | > 1.\]
    But this means that our function is not uniformly continuous on \( A  \) which is a contradiction. Thus, \( f  \) must have a bounded range.

        \end{proof}
    \item[(c)] If \( f  \) is defined on \( \R  \) and \( f(K) \) is compact whenever \( K  \) is compact, then \( f  \) is continuous on \( \R  \).
        \begin{proof}[Solution]
        This is false. Suppose we have the function \( f  \) defined as follows:
        \begin{align*}
            f(x) &= 
            \begin{cases}
                1 &\text{if } x \in \Q \\ 
                0 &\text{if } x \in \mathbb{I}.
            \end{cases} \\
        \end{align*}
        We have that for any compact set \( K  \), we have that every element of \( x  \) gets mapped to \( f(K) \) but \( f(x) \) is not a continuous function.
        \end{proof}
\end{enumerate}


\subsubsection{Exercise 4.4.5} Assume that \( g  \) is defined on an open interval \( (a,c) \) a nd it is known to be uniformly continuous on \( (a,b]  \) and \( [b, c) \). Let \( x \in (a,b] \) and \( y \in [b,c) \). Note that \( (a,c) = (a,b] \cup [b,c) \). Since \( f  \) is uniformly continuous on \( (a,b] \), we have that for any \( x, b \in (a,b] \), there exists \( \delta_1 > 0  \) such that whenever \( | x - b  | < \delta_2   \), we have that 
\[ | f(x) - f(b)  | < \frac{ \epsilon  }{ 2 } .\] Similarly, there exists \( \delta_2 > 0  \) such that whenever \( | b - y  | < \delta_2 \) such that whenever 
\[  | f(b) - f(y) | < \frac{ \epsilon  }{ 2 } . \]Choose \( \delta  =  \min \{ \delta_1, \delta_2  \}  \) such that whenever \( | x - y  | < \delta  \), we have that 
\begin{align*}
    | f(x) - f(y)  | &= | f(x) - f(b) + f(b) - f(y)  |  \\
                     &= | f(x) - f(b)  | + | f(b) - f(y) | \\
                     &< \frac{ \epsilon  }{ 2 } + \frac{ \epsilon  }{ 2 } \\ 
                     &= \epsilon.
\end{align*}
Hence, \( f(x)  \) is uniformly continuous on \( (a,c) \).





\subsubsection{Exercise 4.4.7} Prove that \( f(x) = \sqrt{ x  }  \) is uniformly continuous on \( [0,\infty ) \).
\begin{proof}
    Let \( \epsilon > 0  \) and let \( x, y \in [0, \infty ) \). If \( x,y = 0   \), then it immediately follows that \( f  \) is uniformly continuous since choosing \( \delta = \epsilon  \) implies that whenever \( | x - y  | < \delta  \), we have 
    \[ | f(x) - f(y) | = | \sqrt{ 0 } - \sqrt{ 0 }   | = 0 < \epsilon.   \]Suppose \( x,y \neq  0  \). Then choose \( \delta = \epsilon  \cdot 2\sqrt{ c  }  \) for any \( c \in [0, \infty)\) such that whenever \( | x - y  | < \delta  \) we have that
    \begin{align*}
        | f(x) - f(y)  | &= | \sqrt{ x } - \sqrt{ y }  |  \\
                         &= \Big| \frac{ x - y  }{ \sqrt{ x  } + \sqrt{ y }  }  \Big| \\
                         &= \frac{ | x - y  |  }{  \sqrt{ x  } + \sqrt{ y }    } \\
                         &\leq \frac{ | x - y  |  }{   2 \sqrt{ c }    }  \\
                         &< \frac{ \delta  }{ 2 \sqrt{ c }  }  \\
                         &= \epsilon. 
    \end{align*}
    Hence, we have that \( f(x)  = \sqrt{ x  }  \) is continuous on \( [0,\infty) \).
\end{proof}

\subsubsection{Exercise 4.4.8} Give an example of each of the following, or provide a short argument for why the request is impossible.
\begin{enumerate}
    \item[(a)] A continuous function defined on \( [0,1] \) with range \( (0,1) \).
        \begin{proof}[Solution]
        Let \( f(x) = 1 / x(x+1) \). The range of \( f(x) \) is \( (0,1) \).   
        \end{proof}
    \item[(b)] A continuous function defined on \( (0,1) \) with range \( [0,1] \).
        \begin{proof}[Solution]
        Let the following function \( f(x) \) be defined as follows:
        \[ f(x) = 
        \begin{cases}
            0 &\text{ if } x \in (0, \frac{ 1 }{ 4 } ) \\
            2x - \frac{ 1 }{ 2 } &\text{ if } x \in [ \frac{ 1 }{ 4 }, \frac{ 3 }{ 4 } ] \\
            1 &\text{ if } x \in (3/4, 1)
        \end{cases} \]
        \end{proof}
    \item[(c)] A continuous function defined on \( (0,1] \) with range \( (0,1) \).
        \begin{proof}[Solution]
        
        \end{proof}
\end{enumerate}


\subsubsection{Exercise 4.4.9}
\begin{tcolorbox}
    \begin{defn}[Lipschitz Functions]
        A function \( f: A \to \R  \) is called \textit{Lipschitz} if there exists a bound \( M > 0  \) such that 
        \[ \Big| \frac{ f(x) - f(y) }{ x - y  }   \Big| \leq M  \]
        for all \( x \neq y \in A  \).
    \end{defn}
\end{tcolorbox}
Geometrically speaking, a function \( f \) is Lipschitz if there is a uniform bound on the magnitude of the slopes of lines drawn through any two points on the graph of \( f \).

\begin{enumerate}
    \item[(a)] Show that if \( f: A \to \R  \) is Lipschitz, then it is uniformly continuous on \( A  \).
        \begin{proof}
        Suppose \( f: A \to \R  \) is Lipschitz and let \( x \neq y \in A  \). Let \( \epsilon > 0  \) and choose \( \delta = \epsilon / M  \). Assume \( | x - y  | < \delta  \). Using the fact that \( f  \) is Lipschitz, we have 
        \[ \Big| \frac{ f(x) - f(y)  }{ x - y  }   \Big| \leq M . \]
        But this is equivalent to 
        \[ | f(x) - f(y)  | \leq M | x - y  | < M \cdot \delta = M \cdot \frac{ \epsilon  }{ M  }  = \epsilon. \]
        Hence, we have that \( | f(x) - f(y) | < \epsilon  \) for all \( x \neq y \in A  \) implying that \( f: A \to \R   \) is a uniformly continuous function.
        \end{proof}
    \item[(b)] Is the converse statement true? Are all uniformly continuous functions necessarily Lipschitz?
        \begin{proof}[Solution]
        No the converse statement is not true. Take the function \( f(x) = \sqrt{ x  }  \). This function contains a slope that gets arbitrarly steep as the values of \( x  \) tend toward infinity. Another example is the function \( g(x) = 1 / x^2  \). We can see that if \( g(x)  \) is defined on \( (0,1 ) \) then the slope of the \( g(x)  \) gets arbitrary big as \( x \to 0  \).
        \end{proof}
\end{enumerate}

\subsubsection{Exercise 4.4.10} Assume that \( f \) and \( g  \) are uniformly continuous functions defined on a common domain \( A  \). Which of the following combinations are necessarily uniformly continuous on \( A  \):

\[ f(x) + g(x), ~ f(x)g(x), ~ \frac{ f(x) }{ g(x) }, ~ g \circ f (x).  \]


\begin{enumerate}
    \item[(a)] \( f(x) + g(x) \).
        \begin{proof}
        The addition of two uniformly continuous functions \( f(x), g(x) \) defined on the common domain \( A  \) is continuous. Let \( \epsilon > 0  \). Then define \( \delta = \min \{ \delta_1, \delta_2  \}  \) such that whenever 
        \begin{align*}
            | x - y  | &< \delta_1,  \\
            | x - y  | &< \delta_2 
        \end{align*}
        we have that 
        \begin{align*}
            | f(x) + g(x) - (f(y) - g(y)) | &= | (f(x) - f(y)) + (g(x) - g(y)) |  \\
                                            &\leq | f(x) - f(y)  | + | g(x) - g(y) | \\
                                            &< \frac{ \epsilon  }{ 2 } + \frac{ \epsilon  }{ 2 } \\
                                            &= \epsilon.
        \end{align*}
        Hence, the sum of two uniformly continuous functions is uniformly continuous.
        \end{proof}
    \item[(b)] \( f(x)g(x) \).
        \begin{proof}[Solution]
        Not necessarily true unless both of the functions are both bounded.

        \end{proof}
    \item[(c)] \( f(x) / g(x)  \) where \( g(x) > 0  \).
        \begin{proof}[Solution]
        Like the product of two uniformly continuous functions, the quotient is also not necessarily true unless both \( g(x) \) and \( f(x) \) are bounded below and above respectively.
        \end{proof}
    \item[(d)] \( f(g(x))  \).
        \begin{proof}
        Given \( f: A \to \R  \) and \( g:  A \to \R  \) uniformly continuous, assume that the range \( f(A) = \{ f(x) : x \in A  \}  \) is contained in \( A  \) so that composition is defined on \( A  \). Let \( \epsilon > 0  \). Let \( x \neq y \in A  \). Since \( f(x) \) is uniformly continuous on \( A  \), choose \( \epsilon  = \delta   \) such that whenever \( | f(x) - f(y)  | < \delta  \) and \( g : A \to \R  \) being uniformly continuous on \( A  \), we have that
        \begin{align*}
            | g \circ f (x) - g \circ f(y) | &= | g(f(x)) - g(f(y)) |  \\
                                             &< \epsilon.
        \end{align*}
        \end{proof}
\end{enumerate}


\subsubsection{Exercise 4.4.11 (Topological Characterization of Continuity).} Let \( g  \) be defined on all of \( \R  \). If \( B  \) is a subset of \( \R  \), define the set \( g^{-1}(B)  \) by 
\[  g^{-1}(B) = \{ x \in \R : g(x) \in B  \}. \]
Show that \( g  \) is continuous if and only if \( g^{-1}(O)  \) is open whenever \( O \subseteq \R  \) is an open set.
\begin{proof}
    \( (\Leftarrow) \) Let us construct an \( \epsilon - \)neighborhood around \( g(c) \); that is, \( V_{\epsilon }(g(c)) \). Suppose \( V_{\delta}(c) \) with \( x \in V_{\delta}(c) \). Since \( g^{-1}(O) \) is an open set, we have that \( V_{\delta}(c) \subseteq g^{-1}(O) \). But this means that \( x \in g^{-1}(O) \) implying that \( g(x) \in V_{\epsilon }(g(c))  \) since \( g(x) \in O  \).


    \( (\Rightarrow) \) Let \( V_{\epsilon }(g(c))  \) where \( c  \) is a limit point of \( O  \). Since \( g \) is a continuous function, there exists \( V_{\delta}(c)  \) such that whenever \( x \in V_{\delta}(c)  \), we have that \( g(x) \in V_{\epsilon}(g(c)) \). Since \( O  \) is an open set, we have that \( V_{\epsilon}(g(c)) \subseteq O  \). But this means that \( g(x) \in O  \) as well implying that \( x  \) is also an element of \( g^{-1}(O) \). Hence, we have that \( V_{\delta (c) }  \subseteq g^{-1}(O)\) and thus \( g^{-1}(O)  \) is an open set.
\end{proof}




% !TEX root =  ../../../main.tex 
\section{The Intermediate Value Theorem}

In intuitive terms, the Intermediate Value Theorem is an observation that explains how a continuous function \( f \) on a closed interval \( [a,b] \) attains every value that falls between two range values \( f(a) \) and \( f(b) \).


\begin{tcolorbox}
    \begin{thm}[Intermediate Value Theorem]
    Let \( f: [a,b] \to \R  \) be continuous. If \( L  \) is a real number satisfying \( f(a) < L < f(b) \) or \( f(a) > L > f(b) \), then there exists a point \( c \in (a,b) \) where \( f(c) = L  \).
    \end{thm}
\end{tcolorbox}

\subsection{Preservation of Connected Sets}

Before we prove the Intermediate Value Theorem, we should understand that it is a special case of continuous functions mapping connected sets to connected sets. In the last section, we saw how continuous functions on compact sets \( K  \) produces range sets \( f(K) \) that are also compact. This just so happened to also hold for connected sets.


\begin{tcolorbox}
    \begin{thm}[Preservation of Connected Sets]
    Let \( f: G \to \R  \) be continuous. If \( E \subseteq G  \) is connected, then \( f(E) \) is connected as well.
    \end{thm}
\end{tcolorbox}

\begin{proof}
Our goal is to use the characterization of connected sets to prove this theorem. Let \( f(E) = A \cup B  \) where \( A  \) and \( B  \) are disjoint and nonempty. We want to produce a sequence that is contained in either \( A  \) or \( B  \) that converges to a limit contained in the other. Define the following sets:
\begin{center}
    \( C = \{ x \in E: f(x) \in A  \}  \) and \( D = \{ x \in E : f(x) \in B  \}  \).
\end{center}
These sets are the \textit{preimages} of \( A  \) and \( B  \) respectively. We can see, from the properties of both \( A  \) and \( B  \), that they are nonempty and disjoint such that they satisfy \( E = C \cup D  \). Now assume \( E  \) is a connected set. This means there exists a convergent sequence \( (x_n)  \) that is contained in either \( C  \) or \( D  \) with \( \lim x_n = x  \). Since \( f \) is continuous at \( x  \), we have that \( f(x) = \lim f(x_n) \). Thus, it follows that \( f(x_n) \) is a convergent sequence contained in either \( A  \) or \( B  \) while the limit \( f(x)  \)is an element of the other. Hence, \( f(E ) \) is also connected.
\end{proof}

In \( \R  \), a set is connected if and only if it is an interval (which is possibly unbounded). This in addition to the theorem above leads to a short proof of the Intermediate Value Theorem. 

\subsection{Completeness}

A typical application of the Intermediate Value Theorem involves proving the existence of roots. Given a function \( f(x) = x^2 - 2  \), we can see that \( f(1) = -1  \) and \( f(2) = 2  \). Hence, there exists a point \( c \in (1,2) \) where \( f(c) = 0  \). In Chapter 1, we prove the existence of \( \sqrt{ 2 }  \) using the Axiom of Completeness as our main assumption about the properties of \( \R  \). This fact sheds light on the relationship between the continuity of functions and the completeness of \( \R  \).

Proof of the Intermediate Value Theorem using the Axiom of Completeness.
\begin{proof}
First, let us consider a special case where \( f  \) is a continuous function satisfying the property that \( f(a) < 0 < f(b) \). We want to show that \( f(c) = 0  \) for some \( c \in (a,b) \). Let us define the following set 
\[  K = \{ x \in [a,b] : f(x) \leq 0  \}. \]
This is the set of negative values that \( f(x) \) takes on the interval \( [a,b] \). Note that \( K  \) is bounded above by \( b \), and \( a \in K  \) so that \(  K \) is nonempty. Hence, we are allowed to assert that \( \sup K  \) exists and that \( \sup K = c  \). We have three cases to consider: 
\begin{center}
    \( f(c) > 0  \), \( f(c) < 0  \), and \( f(c) = 0  \).
\end{center}
By the fact that \( c  \) is the least upper bound of \( K  \) rules out the first two cases. Hence, we reach our desired conclusion that \( f(c) = 0  \). The details are requested in Exercise 4.5.5(a).
\end{proof}

Below is the second proof of the Intermediate Value Theorem using the Nested Interval Property.

\begin{proof}
    Consider the special case where \( L = 0  \) and \( f(a) < 0 < f(b) \). Let \( I_0 = [a,b] \), and consider the midpoint \( z = (a+b)/ 2  \). If \( f(z) \geq 0  \), then set \( a_1 = a  \) and \( b_1 = z  \). If \( f(z) < 0  \), then set \( a_1 = z  \) and \( b_1 = b  \). We have that, in either case, the interval \( I_1 =  [a_1, b_1 ] \) has the property that \( f  \) is negative at the left endpoint and nonnegative at the right. This procedure can be inductively repeated such that the Nested Interval Property can be applied to gain the conclusion of the theorem. The remainder of the argument is left to the reader in Exercise 4.5.5(b).
\end{proof}

\subsection{The Intermediate Value Property} 

An interesting question we can ask is does the Intermediate Value Theorem have a converse that is true?

\begin{tcolorbox}
\begin{defn}
    A function \( f  \) has the \textit{intermediate value property} on an interval \( [a,b] \) if for all \( x < y  \) in \( [a,b] \) and all \( L   \) between \( f(x)  \) and \( f(y) \), it is always possible to find a point \( c \in (x,y) \) where \( f(c) = L  \).
\end{defn}
\end{tcolorbox}

This is to say that every continuous function \( f \) on an interval \( [a,b] \) must have the intermediate value property. We must be careful since this is not always true that a function that contains this property must necessarily be continuous. An example of this is the function 
\[  g(x) = 
\begin{cases}
    \sin(1/x) &\text{if } x \neq 0 \\
    0 &\text{if } x = 0
\end{cases}  \]
is not continuous at zero, but it does have the intermediate value property on \( [0,1] \). 


\subsection{Definitions}


\begin{tcolorbox}
    \begin{thm}[Intermediate Value Theorem]
    Let \( f: [a,b] \to \R  \) be continuous. If \( L  \) is a real number satisfying \( f(a) < L < f(b) \) or \( f(a) > L > f(b) \), then there exists a point \( c \in (a,b) \) where \( f(c) = L  \).
    \end{thm}
\end{tcolorbox}


\begin{tcolorbox}
    \begin{thm}[Preservation of Connected Sets]
    Let \( f: G \to \R  \) be continuous. If \( E \subseteq G  \) is connected, then \( f(E) \) is connected as well.
    \end{thm}
\end{tcolorbox}


\begin{tcolorbox}
\begin{defn}
    A function \( f  \) has the \textit{intermediate value property} on an interval \( [a,b] \) if for all \( x < y  \) in \( [a,b] \) and all \( L   \) between \( f(x)  \) and \( f(y) \), it is always possible to find a point \( c \in (x,y) \) where \( f(c) = L  \).
\end{defn}
\end{tcolorbox}

\subsection{Exercises}

\subsubsection{Exercise 4.5.3} 
\begin{tcolorbox}
\begin{defn}
A function \( f  \) is \textit{increasing} on \( A  \) if \( f(x) \leq f(y) \) for all \( x < y  \) in \( A  \).
\end{defn}
\end{tcolorbox}
Show that if \( f  \) is increasing on \( [a,b]  \) and satisfies the intermediate value property, then \( f  \) is continuous on \( [a,b] \).
\begin{proof}
    Let \( f  \) be an increasing function on \( [a,b] \). Since \( f  \) satisfies the intermediate value property, we know that \( f(a) \leq f(c) \leq f(b) \) for some \( c \in (x,y)  \). Since we know that \( f(a) \leq f(c)  \), let us suppose two cases; that is, let us suppose either \( f(c) - \epsilon  / 2 < f(a)  \) or \( f(a) \leq f(c) - \epsilon / 2  \). Then the former implies that we can set \( x_1 = a  \) and the latter implies that we can set \( f(c) - \epsilon / 2 = f(x_1) \). Taking the latter case, we can set \( f(x_1) = f(c) - \epsilon  / 2   \) and let \( x \in (x_1, c] \) because \( f  \) satisfies the intermediate value property. Hence, we have that 
    \[  f(c) - \epsilon / 2 \leq f(x) \leq f(c) \tag{1}. \]
    Likewise, we know that \( f(c) \leq f(b)  \). Let us suppose two cases again; either \( f(b) < f(c) + \epsilon  / 2  \) or \( f(b) \geq f(c) + \epsilon / 2  \). The former we can set \( b = x_2  \) and the latter we can set \( f(x_2) = f(c) + \epsilon  / 2  \) because \( f  \) satisfies the intermediate value theorem. If we let \( x \in [c, x_2) \), then we have that 
    \[  f(c) \leq f(x) \leq f(c) + \epsilon / 2 = f(x_2) \tag{2}. \]
    Taking (1) and (2) together then choose \( \delta = \min \{ x_1 - c, x_2 - c  \}  \), we have that 
    \[   \epsilon / 2 \leq f(x) - f(c) \leq \epsilon / 2 \]
    which is equivalent to 
    \[  | f(x) - f(c)  | \leq \epsilon / 2. \]
\end{proof}



\subsubsection{Exercise 4.5.5} 
\begin{enumerate}
    \item[(a)] Finish the proof of the Intermediate Value Theorem using the Axiom of Completeness started previously.
        \begin{proof}
        Considering the special case where \( f  \) is a continuous function satisfying the property that \( f(a) < 0 < f(b) \). Our goal is to show that \( f(c) \) for some \( c \in (a,b) \). Define the set 
        \[ K = \{ x \in [a,b] : f(x) \leq 0  \}. \] 
Note that \( f(b)  \) is an upper bound for \(  K \) and we know \( a \in K  \) since \( f(a) < 0  \). Hence, we have that \( \sup K  \) exists by the Axiom of Completeness. Our goal is to show that \( \sup  K = f(c) = 0  \). Consider the cases where \( f(c) > 0 , f(c) < 0,  \) and \( f(c) = 0  \). We will show that the first two cases contradict our notion that \( \sup K  \) is the least upper bound. Assume \( f(c) < 0  \) for some \( c \in (a,b) \). But this means that \( f(c)  \) would not be an upper bound of \(  K \) since there exists some \( \alpha  \) such that \( \sup K < \alpha  \)where \( \alpha  \) is not an upper bound of \( K  \). Assume \( f(c) > 0  \) for some \( c \in (a,b)  \). This implies that \( f(c) > 0  \) is an upper bound of \( K  \) but is not the least upper bound of \( K  \) which is a contradiction. Hence, it must be that \( f(c) = 0  \). 
        \end{proof}
    \item[(b)] Finish the proof of the Intermediate Value Theorem using the Nested Interval Property started previously.
        \begin{proof}
            Consider the special case where \( L = 0  \) and \( f(a) < 0 < f(b)  \). Let \( I_0 = [a,b] \) and consider the midpoint \( z = (a+b) / 2  \). If \( f(z) \geq 0  \), then set \( a_1 = z  \) and \( b_1 = b  \). In either case, the interval \( I_1 = [a_1, b_1 ] \) has the property that \( f  \) is negative at the left endpoint and positive at the right. We can extend this inductively to produce a sequence of closed intervals \( I_n = [a_n, b_n] \) with the property described above to make a nested sequence 
            \[  I_0 \supseteq I_2 \supseteq I_3 \supseteq I_4 \supseteq \dots ~ . \]
            Since the intersection of these nested intervals \( I_n  \) for all \( n \) is nonempty, we can find a \( c \in \bigcup_{n = 0 }^{\infty} I_n \) such that \( f(c) = 0  \) since \( f(x)  \) is a continuous function.
        \end{proof}
\end{enumerate}


\subsubsection{Exercise 4.5.6} Let \( f: [0,1] \to \R  \) be continuous with \( f(0) = f(1)  \).
\begin{enumerate}
    \item[(a)] Show that there must exist \( x,y \in [0,1]  \) satisfying \( | x - y  | = 1 / 2  \) and \( f(x) = f(y)  \).
        \begin{proof}
            Define \( g(x) = f(x) - f(x + 1/2)   \). Note that \( g  \) is continuous over \( [0,1/2 ] \). Hence, we have 
            \begin{align*}
                g(0) &= f(0) - f(1/2) \tag{1} \\
                g(1/2) &= f(1/2) - f(1) \\
                       &= -g(0) \tag{2}
            \end{align*}
            By the Intermediate Value Theorem, we can see that there must exists \( c \in [0,1/2]  \) such that \( g(c) = 0  \). Hence, we must have \(g(c) =  f(c) - f(c + 1/2) = 0  \) which implies \(  f(c) = f(c + 1/2) \). 
        \end{proof}
    \item[(b)] Show that for each \( n \in \N  \) there exists \( x_n ,y_n \in [0,1]  \) with \( | x_n - y_n  | = 1/n  \) and \( f(x_n) = f(y_n)  \).

        \begin{proof}[Initial Attempt at Solution]
            Define \( g(x_n) = f(x_n) - f(x_n + 1/n) \). Note that \( g(x_n)  \) is a continuous function over \( [0, 1] \). Since \( g  \) is continuous, let \( (x_n) \to 0  \) and observe that \( g(0) = f(0) - f(0) = 0  \) and let \( (x_n) \to 1  \) such that \( g(1) = f(1) - f(1) = 0  \). Hence, we can see that \( f(0) = f(1) = 0  \). Hence, we can use the Intermediate Value Theorem to state that there exists \( c \in [0,1]  \) such that \( g(c) = 0  \). Hence, we have 
            \[  g(c) = 0 \iff f(x_n) = f(x_n + 1/n).\]
        \end{proof}

        \begin{proof}[Corrected Solution]
            For fixed \( n \in \N  \), define \( g(x_n) = f(x_n) - f(x_n + 1/n) \) such that \( | x_n - y_n  | = 1/n  \) where \( y_n = x_n + 1 / n  \). Note that \( g  \) is continuous over \( [0, (n - 1) / n ] \). We want to show that there exists a root in the interval \( [0, (n-1) / n ] \). Consider the following
             \begin{align*}
                \sum_{ k=0  }^{ n-1 } g \Big( \frac{ k  }{ n }  \Big)  
                                      &= \sum_{ k=0  }^{ n-1 } f \Big( \frac{ k+1  }{ n  }  \Big) - f \Big( \frac{ k  }{ n }  \Big) \\
                                      &= f(1) - f(0) \\
                                      &= 0 
            \end{align*}
            and note that if there exists \(  0 \leq k \leq n - 1  \) such that \( g(k/n) = 0 \) then we are done. Otherwise, if \( g(k/n) \neq 0  \) for \( 0 \leq k \leq n -1  \), then there must exist \( 0 \leq k_1 \leq n-1  \) and \( 0 \leq k_2 \leq n - 1  \) such that \( g(k_1 / n ) \) and \( g(k_2 / n ) \) have opposite  sign. Hence, we can use the intermediate value theorem to posit the existence of \( c \in [0, (n-1)/n] \) such that \( g(c) = 0  \). Hence, we have 
            \[  f(x_n) = f\Big(x_n + \frac{ 1 }{ n }  \Big).\]

        \end{proof}
\end{enumerate}

\subsubsection{Exercise 4.5.7} Let \( f  \) be a continuous function on the closed interval \( [0,1] \) with range also contained in \( [0,1]  \). Prove that \( f  \) must have a fixed point; that is, show \( f(x) = x  \) for at least one value of \( x \in [0,1]  \). 
\begin{proof}
    Let \( g \) be defined by \( g(x) = x - f(x)\). We can see that \( g  \) is continuous over \( [0,1] \) since \( f(x)  \) is continuous over \( [0,1]  \). Now observe that 
    \begin{align*}
        g(0) &= -f(x), \tag{1}\\
        g(1) &= 1 - f(x). \tag{2}
    \end{align*}
    Since \( g(1) = 1  + g(0)  \), we have that \( g(1) \geq g(0)  \). Since \( g  \) is continuous over \( [0,1] \), we can use the Intermediate Value Theorem to find a \( c \in [0,1] \) such that \( g(c) = 0  \). Hence, we must have 
    \[  g(c) = c - f(c) = 0  \iff f(c) = c.  \]
\end{proof}

\subsubsection{Exercise 4.5.8 (Inverse Functions).} If a function \( f: A \to \R  \) is one-to-one, then we can define the inverse function \( f^{-1}  \) on the range of \( f  \) in the natural way: \( f^{-1}(y) = x  \) where \( y = f(x)  \).

Show that if \( f  \) is continuous on an interval \( [a,b]  \) and injective, then \( f^{-1}  \) is also continuous.
\begin{proof}
    Assume \( f  \) is continuous on an interval \( [a,b] \) and injective. Let \( \epsilon > 0  \). Assume \( f  \) is monotone so that we can use the intermediate value theorem to posit the existence of an \( y_0  \) such that it is between \( f(x)  \) and \( f(y) \). Choose \( f  \) to be an increasing function such that there exists some \( x_1 \in [a,b] \) such that \( f(x_1) < f(c)  \). Hence, there exists \(  y_1  \) in the image of \( f  \) such that \( x_1 = f^{-1}(y_1) =  c - \epsilon  \) where \( x_1 = c - \epsilon  < c  \). Let \( x \in (x_1,c] \) such that 
    \begin{align}
        x_1 = c - \epsilon < x \leq c. 
    \end{align}
    Likewise, there exists some \( x_2 \in [a,b]  \) such that \( f(x_2) > f(c) \). Define \( x_2 = c + \epsilon  \) such that \( x_2 > c  \). Let \( x \in [c, x_2) \), we have that 
    \begin{align}
        c \leq x < x_2 = c + \epsilon. 
    \end{align}
    Since \( f  \) is injective, we have \( f^{-1}(x') = x \) for all \( x' \in \text{im}(f) \) and \( f^{-1}(c') = c  \) for some \( c' \in \text{im}(f) \). Hence, we have that 
    \[ | f^{-1}(x') - f^{-1}(c')  | < \epsilon  \]
    whenever \( | f(x) - f(c) | < \delta \).
\end{proof}




\section{Sets of Discontinuity} 

\begin{tcolorbox}
\begin{defn}
Given a function \( f: \R \to \R  \), we call the set \( D_f \subseteq \R  \) to be the set of points where the function \( f  \) fails to be continuous.
\end{defn}
\end{tcolorbox}

Some examples of sets of discontinuous points are
\begin{enumerate}
    \item[(a)] \( D_g = \R  \) in the case for Dirichlet's function, 
    \item[(b)] and \( D_{h} = \R \setminus \{ 0 \}  \) in the case of the modified Dirichlet's function, and 
    \item[(c)] lastly, \( D_{t} = \Q  \) for Thomae's function \( t(x)  \).
\end{enumerate}

We can always write the set of discontinuous points for a function \( D_f  \) as a countable union of closed sets. For monotone functions, these closed sets can taken as single points.


\chapter{The Derivative}

% !TEX root =  ../../../main.tex 


\section{Are Derivatives Continuous?}

The derivative of a function \( g(x)  \), namely \( g'(x)  \), can be defined as the slope of \( g  \) at each point \( x \in \text{Dom}(f)  \). As we have learned in our previous studies, the derivative is just the following limit
\[  g'(c) = \lim_{ x \to c } \frac{ g(x) - g(c)  }{ x - c  }.\]
A couple questions we can ask about the relationship between continuity and differentiability of functions is that: 
\begin{enumerate}
    \item[(i)] Are they continuous? 
    \item[(ii)] Are continuous functions differentiable? 
    \item[(iii)] How nondifferentiable can a continuous function be? 
\end{enumerate}
In the last section, we identified the discontinuous points of a monotone function and expressed them in terms of countable closed sets. Some examples of such functions are of the form 
\[  g_n(x) = 
\begin{cases}
    x^n \sin(1/x) &\text{if } x \neq 0 \\ 
    0 &\text{if } x = 0.
\end{cases} \]
When \( n = 0  \), we can see the oscillations of \( \sin(1/x)  \) prevent \( g  \) from being continuous at \( x = 0  \). But when \( n = 1  \), the oscillations of \( g  \) are sandwiched between \( | x  |   \) and \( - |  x  |  \) which implies that \( g  \) is continuous at \( x = 0  \). What can we say about \( g'_2(0) \)? Is it defined? Using our intuitive definition above, we have that 
\[ g'_1(0) = \lim_{ x \to 0 } \frac{ g_1(x)  }{ x  } = \lim_{ x \to 0 } \sin( 1 / x ) \] which, in this case, does not exist. Thus, we have that \( g_1  \) is not differentiable at zero.
However, if we let \( n = 2  \), then we have the following 
\[  g'_2(0) = \lim_{ x \to 0 } x \sin (1/x) = 0. \]
At nonzero points in the domain of \( g  \), we can use rules of differentiation (that will be justified later) to conclude the \( g_2  \) is differentiable everywhere in \( \R  \) with 
\[  g'_2(x) = 
\begin{cases}
    - \cos(1/x) + 2x \sin (1/x) &\text{if } x \neq 0 \\
    0 &\text{if } x = 0.
\end{cases} \]
But if we now consider the limit 
\[ \lim_{ x \to 0  } g'_2(x)   \]
we will find that it does not exist because for every \( x \neq 0  \), the \( \cos (1/x ) \) term is not preceded by a factor of \( x  \).

In summary, when \( n = 2  \), \( g_2(x)  \) is continuous and differentiable everywhere on \( \R  \), but the derivative function \( g_2'(x)   \) is defined everywhere but is not continuous at \( x = 0  \). The conclusion is that we don't the derivative of a function to be continuous in general.

The discontinuity we found from \( g'_2  \) is an \textit{essential} discontinuity; that is, the limit as \( x \to 0  \) does not exist as a one sided limit. What about a function with a simple jump discontinuity like 
\[  h'(x) = 
\begin{cases}
    -1 &\text{if } x \leq 0 \\
    1 &\text{if } x> 0.
\end{cases} \]

Notice that this function is actually the slopes of the absolute value function \( | x |  \) which is not differentiable at \( x = 0  \). How can we imply differentiability of \( h'  \) at \( x = 0  \)? Our main point here is that continuity is not a sufficient condition for derivatives to be possible.



\section{Derivatives and the IVP}

\subsection{Definition of the Derivative}

\begin{tcolorbox}
    \begin{defn}[Differentiability]
    Let \( g : A \to \R  \) be a function defined on an interval \( A  \). Given \( c \in A  \), the \textit{derivative} of \( g  \) at \( c  \) is defined by 
    \[ g'(c) = \lim_{ x \to c } \frac{ g(x) - g(c)  }{ x - c  }, \]
    provided this limit exists. In this case, we say that \( g  \) is \textit{differentiable} at \( c  \). If \( g'  \) exists for all points \( c \in A  \), we say that \( g  \) is \textit{differentible} on \( A  \).
    \end{defn}
\end{tcolorbox}


\begin{ex}
\begin{enumerate}
    \item[(i)] Consider the function \( f(x) = x^n  \), where \( n \in \N  \), and let \( c  \) be any arbitrary point in \( \R  \). Using the following identity, 
        \[  x^n - c^n = (x -c )(x^{n-1} + cx^{n-2} + c^2 x^{n-3} + \dots + c^{n-1}) \]
        we can take the limit 
        \begin{align*}
            f'(c) &= \lim_{ x \to c  } \frac{ x^n - c^n  }{ x - c  }  \\
                  &= \lim_{ x \to c } (x^{n-1} + cx^{n-2} + c^2 x^{n-3} + \dots + c^{n-1}) \\
                  &= c^{n-1} + c^{n-1} + c^{n-1} + \dots + c^{n-1} \\
                  &= nc^{n-1}\\
        \end{align*}
    \item[(ii)] If \( g(x) = | x  |  \), then if we want to take the derivative at \( c =0  \) produces the following limit
        \[  g'(0) = \lim_{ x \to 0 } \frac{ | x |  }{ x }  \]
        which is \( 1 \) if we approach from the right and \( -1 \) if we approach from the left. Hence, we have that \(g'(c) =  0  \) does not exist.
\end{enumerate}
\end{ex}

This last example should remind us that continuity of a function does not necessarily imply that a function is differentiable. On the other hand, we can say that if \( g \) is differentiable at a point then \( g  \) is continuous at that point. 

\begin{tcolorbox}
\begin{thm}
If \( g: A \to \R  \) is differentiable at a point \( c \in A  \), then \( g  \) is continuous at \( c  \) as well.
\end{thm}
\end{tcolorbox}

\begin{proof}
Assume \( g: A \to \R  \) is differentiable at a point \( c \in A  \). Hence, we have that the following limit exists
    \[ g'(c) = \lim_{ x \to c } \frac{ g(x) - g(c)  }{ x - c  }. \]
    Using the Algebraic Limit Theorem for functional limits, we have that 
    \[  \lim_{ x \to c  } (g(x) - g(c) ) = \lim_{ x \to c  } \Big( \frac{ g(x) - g(c)  }{ x - c  }  \Big) (x - c) = g'(c) \cdot 0 = 0. \]
    Hence, it follows that \( \lim_{ x \to c  } g(x) = g(c). \)
\end{proof}

We can prove the same fact using the epsilon-delta definition for functional limits. 
\begin{proof}
Assume \( g: A \to \R  \) is differentiable at at a point \( c \in A  \). Let \( \epsilon > 0  \). Then we can find a \( \delta > 0  \) such that whenever \( 0 < | x - c  | < \delta  \), we have that
    \[ g'(c) = \lim_{ x \to c } \frac{ g(x) - g(c)  }{ x - c  }. \]
    With a few algebraic manipulations, we can manipulate the above to state that
    \[  | g(x) - g(c) - g(c)(x-c)  | <  | x - c  | \tag{1}  \]
    with \( \epsilon = 1  \).
Using the triangle inequality and choosing \( \delta = \min \{ 1 , \epsilon / (1   + | g(c) | ) \}  \) 
\begin{align*}
    | g(x) - g(c)  | &= | g(x) - g(c)(x-c) + g(c)(x-c) -  g(c) |  \\
                     &\leq | g(x) - g(c)(x-c) | + | g(c)(x-c) - g(c) | \\
                     &< | x - c  | + | g(c)  | | x - c  | \\
                     &= | x - c  | (1 + | g(c) | ) \\
                     &<  \delta \cdot (1 + | g(c)  | ) \\
                     &= \frac{ \epsilon  }{ 1 + | g(c)  |  }  \cdot (1 + | g(c)  | ) \\
                     &= \epsilon.
\end{align*}
Hence, \( g  \) is continuous at \( c \in A  \).
\end{proof}

\subsection{Combinations of Differentiable Functions}

We can use the Algebraic Limit Theorem for functional limits to prove some basic algebraic combinations of differentiable functions.

\begin{tcolorbox}
    \begin{thm}[Algebraic Differentiability Theorem]
    Let \( f  \) and \( g  \) be functions defined on an interval \( A  \), and assume both are differentiable at some point \( c \in A \). Then, 
    \begin{enumerate}
        \item[(i)] \( (f+g)'(c) = f'(c) + g'(c), \)
        \item[(ii)] \( (kf)'(c) = kf'(c), \) for all \( k \in \R  \),
        \item[(iii)] \( (fg)'(c) = f'(c)g(c) + f(c)g'(c),  \) and 
        \item[(iv)] \( (f/g)'(c) = \frac{ g(c)f'(c) - f(c)g'(c)  }{ [g(c)]^2 }  \) provided that \( g(c) \neq  0 \).
    \end{enumerate}
    \end{thm}
\end{tcolorbox}

\begin{enumerate}
    \item[(i)] \( (f+g)'(c) = f'(c) + g'(c)  \).
        \begin{proof}
        Assume \( f  \) and \( g  \) are functions that are both differentiable at some point \( c \in A  \). Since \( (f+g)(x) = f(x) + g(x)  \) and the Algebraic Function Limit Theorem, we have that
        \begin{align*}
            (f+g)'(x) &= \lim_{ x \to c } \frac{ (f+g)(x) - (f+g)(c) }{ x - c  }  \\
                      &= \lim_{ x \to c } \frac{ f(x) + g(x) - (f(c) + g(c) ) }{ x - c  } \\
                      &= \lim_{ x \to c  } \frac{ ( f(x) - f(c)) + (g(x) - g(c)) }{ x - c  } \\
                      &= \lim_{ x \to c  } \Big(  \frac{ f(x) - f(c)  }{ x - c  } + \frac{ g(x) - g(c)   }{ x - c  }  \Big) \\
                      &= \lim_{ x \to c  }  \frac{ f(x) - f(c)  }{ x - c  } + \lim_{ x \to c  } \frac{ g(x) - g(c)  }{ x - c  } \\
                      &= f'(c) + g'(c).
        \end{align*}
        \end{proof}
    \item[(ii)] \( (kf)'(c) = k f'(c)  \) for all \( k \in \R  \).
        \begin{proof}
        Since \( f  \) is differentiable at \( c \in A  \), we have that 
        \begin{align*}
            (kf)'(c) &= \lim_{ x \to c  } \frac{ (kf)(x) - (kf)(c)  }{ x - c  }  \\
                     &= \lim_{ x \to c  }  \frac{ k f(x) - k f(c)  }{ x - c  } \\
                     &= \lim_{ x \to c  } \frac{ k (f(x) - f(c) ) }{ x - c  } \\
                     &=k \cdot  \lim_{ x \to c  } \frac{ f(x) - f(c) }{ x - c  } \\
                     &= k f'(c).
        \end{align*}
        \end{proof}
    \item[(iii)] \( (fg)'(c)  = f'(c)g(c) + f(c)g'(c). \)
        \begin{proof}
        Let \( f  \) and \( g  \) be differentiable at some point \( c \in A  \).  By using the Algebraic Function Limit Theorem, we have that 
        \begin{align*}
            (fg)'(c)  &= \lim_{ x \to c  } \frac{ (fg)(x) - (fg)(c) }{ x - c  }  \\
                      &= \lim_{ x \to c  } \frac{ f(x)g(x) - f(c)g(c)  }{ x - c  } \\
                      &= \lim_{ x \to c  } \frac{ f(x)g(x) - f(x)g(c) + f(x)g(c) - f(c)g(c) }{ x - c   } \\
                      &= \lim_{ x \to c  }  \Big(  \frac{ f(x) (g(x) - g(c))  }{ x - c   }  + \frac{  g(c) (f(x) - f(c))  }{ x - c  } \Big) \\ 
                      &= \lim_{ x \to c  } \frac{ f(x) (g(x) - g(c) ) }{ x - c  } + \lim_{ x \to c  } \frac{ g(c) (f(x) - f(c) ) }{ x - c  } \tag{ALFT} \\ 
                      &= \lim_{ x \to c  } f(x)  \Big( \lim_{ x \to c  } \frac{ g(x) - g(c)  }{ x - c  }  \Big) + g(c) \cdot \lim_{ x \to c  } \frac{ f(x) - f(c)  }{ x - c  }   \\
                      &= f(c) g'(c) + g(c)f'(c).
        \end{align*}
        Hence, we have that \( (fg)'(c) = f(c)g'(c) + f'(c)g(c) \).
        \end{proof}
    \item[(iv)] \( (f/g)'(c) = \frac{ g(c)f'(c) - f(c)g'(c)  }{ [g(c)]^2  }  \).
        \begin{proof}
        Let \( f  \) and \( g  \) be differentiable functions where \( g(x) \neq 0  \) for all \( x \in A   \). Note that since \(f \) and \( g  \) are differentiable, they are also continuous on \( A  \). Hence, \( \lim_{ x \to c  } f(x) = f(c)  \) and \( \lim_{ x \to c  } g(x) = g(c)  \). Then observe that 
        \begin{align*}
            (f/g)'(c) &= \lim_{ x \to c  } \frac{ (f/g)(x) - (f/g)(c)  }{ x - c  }  \\
                      &= \lim_{ x \to c  } \frac{ f(x) / g(x) - f(c) / g(c)  }{ x - c  } \\
                      &= \lim_{ x \to c  } \Big[ \frac{ 1 }{ g(x) g(c)  } \cdot \frac{ g(x) (f(x) - f(c)) - f(x) (g(x) - g(c) ) }{ x - c   } \Big] \\ 
                      &= \lim_{ x  \to c  } \Big( \frac{ 1 }{ g(x) g(c) }  \Big) \lim_{ x \to c  } \Big( \frac{ g(x) (f(x) - f(c) )  - f(x) (g(x) - g(c) )}{ x - c  }  \Big) \\
                      &= \lim_{ x \to c  } \Big( \frac{ 1 }{ g(x) g(c)  }  \Big) \lim_{ x \to c  } \Big( g(x) \frac{ f(x) - f(c)  }{ x - c  } - f(x) \frac{ g(x) - g(c)  }{ x - c  }  \Big) \\
                      &= \lim_{ x \to c  } \Big( \frac{ 1 }{ g(x) g(c)  }   \Big) \Big( \lim_{ x \to c  } g(x) \frac{ f(x) - f(c)  }{ x - c  } - \lim_{ x \to c  } f(x) \frac{ g(x) - g(c)  }{ x - c  }  \Big) \\
                      &= \frac{ 1 }{ [g(c)]^2  } \cdot (g(c) f'(c) - f(c)g'(c) ). \\ 
        \end{align*}
        \end{proof}
\end{enumerate}

We can also compose two differentiable functions together and still get a differentiable function. This next fact is called the chain rule. A way to prove this fact is to use the following:
\begin{align*}
   (g \circ f)'(c) &= \lim_{ x \to c  } \frac{ g(f(x)) - g(f(c) ) }{ x - c  }  \\
                   &= \lim_{ x \to c  } \frac{ g(f(x) ) - g(f(c) ) }{ x - c  } \\
                   &= g'(f(c) ) \cdot f'(c).
\end{align*}
But an issue with this proof is that the expression \( f(x) - f(c)  \) might be zero in the denominator for arbitrarily small neighborhoods of \( c  \).

\begin{tcolorbox}
    \begin{thm}[Chain Rule]
    Let \( f: A \to \R  \) and \( g: B \to \R  \) satisfy \( f(A) \subseteq B  \) so that the composition \(  g  \circ f \) is defined. If \( f  \) is differentiable at \( c \in A  \) and if \( g  \) is differentiable at \( f(c) \in B   \), then \( g \circ f  \) is differentiable at \( c  \) with \( (g \circ f)'(c) = g'(f(c)) \cdot f'(c)\).
    \end{thm}
\end{tcolorbox}

\begin{proof}
Assume \( g  \) is differentiable at \( f(c)  \). Then we have that 
\[  g'(f(c) )  = \lim_{ y  \to f(c)  }  \frac{ g(y) - g(f(c)) }{ y - f(c)  }.\]
We can rewrite the quotient in the limit above by setting \( d(y)  \) to be the following:
\[  d(y) = \frac{ g(y) -  g(f(c) ) }{ y - f(c)  }. \]
This is equivalent to \( \lim_{ y \to f(c)  } d(y) = g'(f(c) ) \). The issue at the moment is when we set \( y = f(c)   \), \( d(y)  \) becomes undefined. But this can be mitigated by rewriting \( d(y)  \) like 
\[  g(y) - g(f(c)) = d(y) (y - f(c)) \tag{1} \]
which hold for all \( y \in B  \) including \( y = f(c)  \). Hence, we are allowed to substitute \( y = f(t)  \) for any arbitrary \( t \in A  \) into \( d(y)  \). If \( t \neq c  \), we can divide (1) by \( (t - c ) \) to get 
\[  \frac{ g(f(t) - g(f(c) )) }{ t - c  } = d(f(t)) \frac{ f(t) - f(c)  }{ t - c  }. \]
Hence, we have that 
\begin{align*}
    (g \circ f)'(c) &= \lim_{ x \to c  }  \frac{ g(f(x)) - g(f(c)) }{ x - c   } \\
                    &= \lim_{ x \to c  } \Big( \frac{ g(f(x))  - g(f(c) )}{f(x) - f(c)  } \cdot \frac{ f(x) - f(c) }{ x - c  } \Big) \\
                    &= g'(f(c)) \cdot f'(c)
\end{align*}
\end{proof}

\subsection{Darboux's Theorem}

A takeaway from our introduction to Derivatives is that differentiable functions need always be continuous and that our main example for this is for \( n = 2  \) with the function \( g_2(x) = x^2 \sin( 1 / x) \) where \( g_2(0)  \). However, differentiable functions do posses the intermediate value property. This leads us to next theorem that says that functions attain their max and mins at points where the derivative is zero. 

\begin{tcolorbox}
    \begin{thm}[Interior Extremum Theorem] Let \( f  \) be differentiable on an open interval \( (a,b)  \). If \( f  \) attains a maximum value at some point \( c \in (a,b)  \); that is, \( f(c) \geq f(x)  \) for all \( x \in (a,b)  \), then \( f'(c) = 0  \). The same is true if \( f(c)  \) is a minimum value.

\end{thm}
\end{tcolorbox}

\begin{proof}
    Since \( c  \) is in an open interval \( (a,b) \), we can construct two sequences \( (x_n), (y_n) \subseteq (a,b)  \) such that both sequences converge to \( c \in (a,b)  \) and satisfy \( x_n < c < y_n  \) for all \( n \in \N  \). Since \( f  \) attains its maximum value at some point \( c \in (a,b)  \), we have that for all \( n \in \N  \), \( f(y_n) \leq f(c)  \). Hence, we can say that 
    \[  f(y_n) - f(c) \leq 0. \]
    Dividing by \( (y_n - c ) \) and using the Order Limit Theorem gives us the following
    \[  f'(c) = \lim_{ n \to \infty   } \frac{ f(y_n) - f(c)  }{ y_n - c  } \leq 0 \tag{1}. \]
    Now if \( f  \) attains a minimum, we have \( f(x_n) - f(c) \geq 0  \) for all \( n \in \N  \) which implies further that 
    \[  f'(c) = \lim_{ n \to \infty  } \frac{ f(x_n) - f(c)  }{ x_n - c  } \geq 0 \]
    by the Order Limit Theorem.
\end{proof}

This theorem leads to a very important result about differntiable functions containing the intermediate value property.

\begin{tcolorbox}
    \begin{thm}[Darboux's Theorem]
    If \( f  \) is differentiable on an interval \( [a,b]  \), and if \( \alpha  \) satisfies \( f'(a) < \alpha < f'(b)  \) (or \( f'(a) > \alpha > f'(b)  \) ), then there exists a point \( c \in (a,b)  \) where \( f'(c) = \alpha \).
    \end{thm}
\end{tcolorbox}

\begin{proof}
    We can define a new function \( g(x) = f(x) - \alpha x  \) on \( [a,b]  \). Since \( g  \) is differentiable on \( [a,b]  \) with \( g'(x)  = f'(x) - \alpha  \) and that \( g'(a) < 0 < g'(b)  \), we want to show that \( g'(c) = 0  \) for some \( c \in (a,b)  \).
\end{proof}

\subsection{Exercises}













\section{The Mean Value Theorems}

\subsection{Mean Value Theorem}

\begin{enumerate}
    \item[(a)] We can find a point along some interval \( [a,b]  \) of a differentiable function \( f  \) such that we will have a slope of \( f  \) where 
        \[  f'(c) = \frac{ f(b) - f(a)  }{  b - a  }  \]
        for at least one point \( c \in (a,b) \).
    \item[(b)] Used to prove L'hopital's rule for limits of quotients of differentiable functions.
    \item[(c)] Used in the study of infinite series of differentiable functions.
    \item[(d)] One of mechanisms needed to show Lagrange's Remainder Theorem and used to approximate the error between a Taylor polynomial.
\end{enumerate}

\begin{theorem}{Rolle's Theorem}{}
        Let \( f : [a,b] \to \R  \) be continuous on \( [a,b]  \) and differentiable on \( (a,b)  \). If \( f(a) = f(b) \), then there exists a point \( c \in (a,b)  \) where \( f'(c) = 0  \). 
    \end{theorem}


\begin{proof}
    Since \( f  \) is continuous on a compact set, we know that \( f  \) attains a maximum and a minimum. If \( f  \) attains a maximum and minimum at the endpoints and the fact that \( f(a) = f(b)  \), we know that \( f  \) must be a constant function. Hence, we can choose any \( x \in [a,b]  \) such that \( f'(x) = 0  \). If \( f  \) attains a maximum or minimum in the interior of \( f  \) then there exists \( c \in (a,b)  \) such that \( f'(c) = 0  \).
\end{proof}

\begin{theorem}{Mean Value Theorem}{}
        If \( f: [a,b] \to \R  \) is continuous on \( [a,b]  \) and differentiable on \( (a,b)  \), then there exists a point \( c \in (a,b)  \) where 
        \[  f'(c) = \frac{ f(b) - f(a)  }{ b - a  }. \]
    \end{theorem}

\begin{proof}
    Notice that the Mean Value Theorem reduces to Rolle's Theorem in the case where \( f(a) = f(b)  \).
Consider the equation of a line through \( (a, f(a) ) \) and \( (b , f(b) ) \) is 
\[  y = \Big( \frac{ f(b) - f(a)  }{ b - a  } (x -a ) \Big) + f(a). \]
Furthermore, we want to consider the difference between this line and the function \( f(x)  \). Define a new function \( d   \) where 
\[  d(x) = f(x) - \Big[ \Big( \frac{ f(b) - f(a)  }{ b - a  }  \Big) (x - a ) + f(a)     \Big], \]
Observe that \( d  \) is continuous on \( [a,b]  \) since \( f  \) is continuous on \( [a,b]  \) and differentiable on \( (a,b)  \) and satisfies \( d(a) = 0 = d(b)  \). By differentiating \( d(x)  \), we have that 
\[  d'(x) = f'(x) - \frac{ f(b) - f(a)  }{ b - a  }. \]
Now, using Rolle's Theorem, we can find a \( c \in (a,b)  \) such that \( d'(c) = 0   \). Hence, 
\[  0 = f'(c) - \frac{ f(b) - f(a)  }{ b -a  } \iff f'(c) = \frac{ f(b) - f(a)  }{ b - a  }.  \] 

\end{proof}

Now consider a constant function \( f(x) = k  \) for any \( k  \). Intuition suggests that for all \( x \in A  \), we have \( f'(x) = 0  \). Is there any way we can prove that \( f(x)  \) is constant given \( f'(x) = 0  \) for all \( x \in A  \)? Indeed, we can using the Mean Value Theorem.

\begin{cor}
If \( g: A \to \R  \) is differentiable on an interval \( A  \) and satisfies \( g'(x) = 0  \) for all \( x \in A  \), then \( g(x) = k  \) for some constant \(  k \in \R  \).
\end{cor}

\begin{proof}
    Take \( x, y \in A  \) and assume \( x < y  \). Applying the Mean Value Theorem to \( g  \) on the interval \( [a,b ] \), we can see that 
    \[  g'(c) = \frac{ g(y) - g(x)   }{ y- x  }.  \]
    Since \( g(x) = 0  \) for all \( x \in A  \), we have that 
    \[  \frac{ g(y) - g(x)  }{ y -x  } = 0  \iff g(y) = g(x). \]
    Set \(  k  \) equal to this common value. Since \( x,y \in A  \) are arbitrary, it follows that \( g(x) = k \) for all \( x \in A  \).
\end{proof}

\begin{cor}{}{}
If \( f  \) and \( g  \) are differentiable functions on an interval \( A  \) and satisfy \( f'(x) = g'(x)   \) for all \( x \in A  \), then \( f(x) = g(x) + k  \) for some interval \(  k \in \R  \).
\end{cor}

\begin{proof}
Suppose \( f \) and \( g  \) are differentiable functions on an interval \( A  \) and satisfy \( f'(x) = g'(x)  \). Let us define a new function \( h(x) = f(x) - g(x)  \). Differentiating this function gives us the following \( h'(x) = f'(x) - g'(x)  \). But since \( g'(x) = f'(x)  \), we have that \( h'(x) = 0  \). Since \( h  \) is differentiable on an interval \( A  \), we know that \( h(x) = k  \). Hence, we have that 
\[  k = f(x) - g(x) \iff f(x) = g(x) + k. \]
\end{proof}

We can build a more general form of the Mean Value Theorem which can be used to prove L'hopital's rules and the Lagrange Remainder Theorem.

\begin{theorem}{Generalized Mean Value Theorem}{}
    If \( f  \) and \( g  \) are continuous on the closed interval \( [a,b]  \) and differentiable on the open interval \( (a,b)  \), then there exists a point \( c \in (a,b)  \) where 
    \[  [f(b) -f(a)]g'(c) = [g(b) -g(a)]f'(c). \]
    If \( g'  \) is never zero on \( (a,b)  \), then the conclusion can be stated as 
    \[  \frac{ f'(c)  }{ g'(c)  } = \frac{ f(b) - f(a)  }{ g(b) - g(a)  }. \]
    \end{theorem}

\begin{proof}
This result follows by applying the Mean Value Theorem to the function 
\[  h(x) = [f(b) - f(a)] g(x) = [g(b) - g(a)] f(x).  \]
The details are requested in Exercise 5.3.5.
\end{proof}

\subsection{L'Hopital's Rules}

The Algebraic Limit Theorem asserts that when taking a limit of a quotient of functions we can write 
\[  \lim_{ x \to c  } \frac{ f(x)  }{ g(x)  } = \frac{ \lim_{ x \to c  } f(x)  }{ \lim_{ x \to c  } g(x)  }  \]
provided that the quotient is well-defined. What happens when the denominator has a limit that goes to zero while the numerator goes to a limit that is non-zero? Both zero? Both infinite?

\begin{theorem}{L'Hopital's Rule: 0/0 case}{}
    Let \( f  \) and \( g  \) be continuous on an interval containing \( a  \), and assume \( f  \) and \( g  \) are differentiable on this interval with the possible exception of the point \( a  \). If \( f(a) = g(a) = 0  \) and \( g'(x) \neq 0  \) for all \( x \neq a  \), then 
    \begin{center}
        \( \lim_{ x \to a } \frac{ f'(x)  }{ g'(x)  } = L  \) implies \( \lim_{ x \to a } \frac{ f(x)  }{ g(x)  } = L. \)
    \end{center}
    \end{theorem}

\begin{proof}
Of course they would leave this to the reader to finish. It is requested in Exercise 5.3.11.
\end{proof}

Next is the case when we replace the assumption of the last theorem with the case that \( \lim_{ x \to a } g(x) = \infty \). We can define what it means to have an infinite limit.

\begin{definition}{}{}
Given \( g: A \to \R \) and a limit point \( c  \) of \( A  \), we say that \( \lim_{ x \to c  } g(x) = \infty  \) if, for every \( M > 0  \), there exists a \( \delta > 0  \) such that whenever \( 0 < | x -c  |  < \delta  \) it follows that \( g(x) \geq M  \). We can define \( \lim_{ x \to c  } g(x)  \) in a similar way.
\end{definition}
 Next is the case of L'hopital's rule when applied to the case when both the numerator and the denominator go to infinity.
 \begin{theorem}{L'Hopital's Rule: \( \infty / \infty  \) case}
    Assume \( f  \) and \( g  \) are differentiable on \( (a,b)  \) and that \( g'(x) \neq 0  \) for all \( x \in (a,b) \). If \( \lim_{ x \to a } g(x) = \infty  \) (or \( - \infty  \)), then 
    \[  \lim_{ x \to a } \frac{ f'(x)  }{ g'(x)  } = L \text{~implies~} \lim_{ x \to a } \frac{ f(x)  }{ g(x)  } = L. \]
    \end{theorem}

\begin{proof}
Let \( \epsilon > 0  \). Since \( \lim_{ x \to a } \frac{ f'(x)  }{ g'(x)  } = L  \), there exists \( \delta_1 > 0  \) such that 
\[   \Big| \frac{ f'(x)  }{ g'(x) } - L  \Big| < \frac{ \epsilon  }{ 2 } \tag{1} \]
for all \( a < x < a + \delta_1  \). For convenience of notation, let \( t = a + \delta_1  \) and note that \( t  \) is fixed for the remainder of the argument. Let our functions \( f  \) and \( g  \) be defined on the interval \( [x,t] \) for any \( x \in (a,t)  \). We can use the Generalized Mean Value Theorem on the interval \( [x,t]  \) to get that 
\[  \frac{ f'(c)  }{ g'(c)  } = \frac{ f(x) - f(t)  }{ g(x) - g(t)  }  \] for some \( c \in (x,t ) \). Since we are considering \( t = a + \delta_1  \), we have that 
\[  L - \frac{ \epsilon  }{ 2 } < \frac{ f(x) - f(t)  }{ g(x) - g(t) } < L + \frac{ \epsilon  }{ 2 } \tag{2}  \] for all \( x \in (a,t). \) 
Our goal is to isolate the fraction \( f(x) / g(x)  \) by multiplying (2) by \( (g(x) - g(t)) / g(x)  \). We need to assume that \( g(x) \geq g(t)  \) so that the quantity we are multiplying by is positive (or else we will switch the order of the inequality which we don't want). Carrying our our plan results in the following inequality
\[  L - \frac{ \epsilon  }{ 2 }  + \frac{ -Lg(t) + \frac{ \epsilon  }{ 2 } g(t) + f(t)  }{ g(x)  } < \frac{ f(x)  }{ g(x)  } < L + \frac{ \epsilon  }{ 2 } + \frac{ -L g(t) - \frac{ \epsilon  }{ 2 } g(t)  + f(t)  }{ g(x)  }.\]
Since \( t  \) is fixed and that \( \lim_{ x \to a } g(x) = \infty  \), we can choose \( \delta_2 > 0  \) such that this our choice of multiplying by the above quantity will satisfy \( g(x) \geq g(t)  \) for all \( a < x < a + \delta_2 \). By the same fact, we can also choose \( \delta_3  \) such that \( a < c < a + \delta_3  \) implies that \( g(x)  \) is large enough to ensure that both 
\[ \frac{ -Lg(t) + \frac{ \epsilon }{ 2 } g(t) + f(t)   }{ g(x)  } \text{ ~and~ } \frac{ -L g(t) - \frac{ \epsilon  }{ 2 } g(t) + f(t)   }{ g(x)  } \]
are less than  \( \epsilon / 2   \) in absolute value. Choosing \( \delta = \min \{ \delta_1, \delta_2, \delta_3  \}  \) guarantees that 
\[  \Big| \frac{ f(x)  }{ g(x)  } - L  \Big| < \epsilon \]
for all \( a < x < a  + \delta. \)
\end{proof}



\section{Continuous Nowhere-Differentiable Functions}

\subsection{The Sawtooth Function}

In the last few sections, we have seen that continuity does not immediately imply that a function is differentiable. The prime example for this is the absolute value function \( f(x) = | x  |  \) where differentiable at every point except \(  x = 0  \). Can we construct functions that are continuous at every point but non-differentiable everywhere? The answer is yes and it is in the form of 
\[  f(x) = \sum_{ n = 0 }^{ \infty  } a^n \cos(b^n x) \] where the values of \( a  \) and \( b  \) are carefully chosen. All of us may notice that this is just the Fourier Series. Define a function \( h(x) = | x |  \) that replaces the oscillations provided by \( \cos(x)  \) on the interval \( [-1,1] \) and extend \( h  \) to be defined on all of \( \R  \) such that \( h(x+2) = h(x)   \). Instead of the smooth oscillations that we see with the \( \cos (x)  \) graph, we have a periodic "sawtooth" looking graph provided by \( h(x) = | x |  \).

\subsubsection{Exercise 5.4.1} Sketch a graph of \( (1/2) h(2x)  \) on \( [-2,3]  \). Give a qualitative description of the functions 
\[  h_n(x) = \frac{ 1 }{ 2^n  } h(2^n x ) \]
as \( n  \) gets larger.
\begin{proof}[Solution]
    The graph of \( h_1(x)  \) is just the Sawtooth function \( h(x)  \) but with a maximum height of \( 1/2  \) and the length of the period is just \( 1  \). Notice that as \(  n \) gets bigger and bigger our height for \( h_n(x)  \) is just \( 1 / 2^n  \) while the period is \( 1 / 2^{n-1}  \). The slopes of all the segments that make up \( h_n(x)  \) is just \( \pm 1  \) for all \( n \in \N  \).
\end{proof}
Let's define 
\[  g(x) = \sum_{ n=0  }^{ \infty  } h_n(x) = \sum_{ n=0  }^{ \infty  } \frac{ 1 }{ 2^n  } h(2^n x ) \]
which we claim as having the property that it is continuous for all \( x \in \R  \) but non-differentiable for all \( x \in \R  \).

\subsection{Infinite Series of Functions and Continuity}
As we can see, the way we defined \( g(x)  \) is very different from the way the functions we have been defining thus far where for each \( x \in \R  \), \( g(x)   \) is the value of an infinite series.

\subsubsection{Exercise 5.4.2} Fix \( x \in \R  \). Argue that the series 
\[  \sum_{ n=0  }^{ \infty  } \frac{ 1 }{ 2^n } h(2^n x) \]
converges and thus \( g(x)  \) is properly defined. 

\begin{proof}[Solution]
Since the maximum height of \( h_n(x)  \) is \( 1 / 2^n  \), it follows that for all \( n \in \N  \), the sequence of partial sums is bounded; that is
\[ 0 \leq \frac{ 1 }{ 2^n  } h_n(2^n x) \leq \frac{ 1 }{ 2^n  }.\]
Since \( \frac{ 1 }{ 2^n  }  \) produces a geometric series which converges, we know that \( \frac{ 1 }{ 2^n  } h_n (2^n x) \) must also converge by the Comparison Test. This means the series that defines \( g(x)  \) converges and is, therefore, well defined.
\end{proof}

We can ask a couple of questions about functions that are defined by infinite series. 
\begin{enumerate}
    \item[(a)] Certainly, finite sum of a continuous functions is continuous but is the infinite sum of continuous functions necessarily continuous? We will see that this is not always the case in the next chapter. 
\end{enumerate}

 \subsubsection{Exercise 5.4.4} As the graph in Figure 5.7 suggests, the structure of \( g(x)  \) is quite intricate. Answer the following questions, assuming that \( g(x) \) is indeed continuous. \begin{enumerate}
    \item[(a)] how do we know \( g  \) attains a maximum value \( M  \) on \( [0,2]  \)?  What is the value? 
        \begin{proof}[Solution]
        Since \(  g \) is continuous on a compact set \( [0,2]  \), we know that \( g  \) must attain its maximum and minimum on the interval. Since \( g  \) converges, we can use associativity to redefine \( g \) in the following way:
        \[  f_n (x) = h_{2n}(x) + h_{2n+1}(x)  \] for all \(  n \in \N  \). A scaling argument is used to show that \( g(x) \leq \sum_{ k=0 }^{ \infty  } \frac{ 1 }{ 4^k } = \frac{ 4 }{ 3 }. \) Hence, the max of \( g(x)  = \frac{ 4 }{ 3 }. \) 
        \end{proof}
\end{enumerate}

\subsubsection{Nondifferentiability} 

Let us try to prove that \( g  \) is not differentiable for all \( x \in \R  \). Looking at \( x = 0  \), we can see that \( g  \) appears to not be differentiable here. Consider the sequence \( x_m = 1/2^m \), where \( m = 0,1,2, \dots \).

\subsubsection{Exercise 5.4.5} Show that 
\[  \frac{ g(x_m) - g(0)  }{ x_m - 0  } = m+1,  \]
and use this to prove that \( g'(0)  \) does not exist.
\begin{proof}
For \( g'(0)  \) to exist, the sequential criterion for limits requires that 
\[ g'(0) = \lim_{ m \to \infty  }  \frac{ g(x_m) - g(0)  }{ x_m - 0  }  \] exist for any sequence \( (x_m) \to 0  \). Let us fix \( m \in \N  \) and consider \( x_m  = 1 / 2^m \). Then plugging this in to \( g  \), we get that
\[ g(x_m) = \sum_{ n=0 }^{ \infty  } \frac{ 1 }{ 2^n  } h(2^{n-m}).\] Now we have to consider a few cases. Suppose \( n > m  \), then \( h(2^{n-m})  = 0  \) because the sawtooth function is zero for even \( n  \). If \(  n \leq m  \), then we have \( h(x) = x  \) and we get that 
\[  \frac{ 1 }{ 2^n } h(^{n-m} ) = \frac{ 1 }{ 2^n } 2^{n-m} = \frac{ 1 }{ 2^m } \]
by the definition of \( h(x)  \). This means we can represent \( g(x_m)  \) as a finite sum
\[  g(x_n) = \sum_{ n=0 }^{ m } \frac{ 1 }{ 2^m }. \] Using the difference quotient, we can see that 
\[  \frac{ g(x_m) - g(0)  }{ x_m - 0  } = \frac{ \sum_{ n=0 }^{ m } 1/2^m }{ 1/2^m } = \sum_{ n=0 }^{ m }1 = m + 1. \]
Notice that the \( g(x_m) = m + 1  \) is a sequence that diverges, we know that \( g  \) is not differentiable at zero and therefore does not exist.
\end{proof}

Although intuition might lead us astray by telling us that \( g'(0) = \infty  \), we need to remember that for \( x_m = -(1/2^m) \) we can use the same argument above to produce a difference quotient that approaches \( - \infty  \). This is that "cusp" that we see at \( x = 0  \) in the graph of \( g  \). 

Using the same argument above, we can prove \( g' \) does not exists for \( x = 1  \) and \( x = 1 / 2 \). In fact, \( g'(x)  \) does not exist for any \( x \in \Q  \) such that \( x = p/2^k \) where \( p \in \Z  \) and \( k \in \N \cup \{ 0 \}  \). These values of \( x \in \Q  \) are called \textit{dyadic} points. For every \( x  \) that is \textit{dyadic}, \( h_n  \) contains a corner at \( x  \) as long as \( n \geq k  \). 

What if \( x  \) is not \textit{dyadic}? For fixed values of \( m \in \N \cup \{ 0  \}  \), \( x  \) can be between two dyadic points; that is, 
\[  \frac{ p_m  }{ 2^m } < x < \frac{ p_m + 1  }{ 2^m }. \] Setting \( x_m = p_m / 2^m \) and \( y_m = (p_m) + 1) / 2^m  \), we can repeat this for each \( m  \) that produces two sequences \( (x_m)  \) and \( (y_m)  \) that satisfy the following:
\begin{enumerate}
    \item[(a)] \( \lim x_m = \lim y_m = x   \), 
    \item[(b)] \( x_m < x < y_m \).
\end{enumerate}

We can prove the following lemma:

\begin{tcolorbox}
\begin{lem}
Let \( f  \) be defined on an open interval \( J  \) and assume \( f  \) is differentiable at \( a \in J  \). If \( (a_n)  \) and \( (b_n)  \) are sequences satisfying \( a_n < a < b_n  \) and \( \lim a_n = \lim b_n = a  \), then 
\[ f'(a) = \lim_{ n \to \infty  } \frac{ f(b_n) - f(a_n)  }{ b_n - a_n  }. \]
\end{lem}
\end{tcolorbox}
\begin{proof}

\end{proof}
We can use this lemma to show that \( g'(x)  \) does not exist.






\chapter{Sequences and Series of Functions}

\section{Discussion: Power Series}

\subsection{Geometric Series}

Series that are easily summable are the Geometric series. One example of such a series is the following:
\[  \sum_{ n=0 }^{ \infty  } x^n = 1 + x + x^2 + x^3 + \dots = \frac{ 1 }{ 1 -x  } \tag{1} \]
for all \( | x  | < 1  \). A few more examples include the series 
\[  \sum_{ n=0  }^{ \infty  } \frac{ 1 }{ 2^n }  \] and 
\[  \sum_{ n=0  }^{ \infty  } \Big( \frac{ -1 }{ 3 }  \Big)^n = \frac{ 3 }{ 4 }. \]
We can take derivatives of both sides of (1) and get 
\[  \frac{ 1 }{ (1-x)^2 } = 0 + 1 + 2x + 3x^2 + 4x^3 + \dots. \tag{2} \]
A question we can ask ourselves is whether or not this formula is valid at least on the open interval \( (-1,1) \)? It turns out that it is true for (2) to hold along \( (-1,1) \) (we will find out later why this is the case).  


There is another surprising connection of (1) when we replace \( x  \) term with an \( x^2 \) term and then take the integral of the left side. What we end up getting is a relationship between circles and infinite series; that is, 
\[ (\arctan(x))' = \frac{ 1 }{ 1 + x^2  } \text{~ and ~} \arctan(0) = 0 \]
where setting \( x = 1  \) gives us 
\[ \frac{ \pi  }{ 4  } = 1 - \frac{ 1 }{ 3 } + \frac{ 1 }{ 5 } - \frac{ 1 }{ 7 }  + \frac{ 1 }{ 9 } \- \dots .  \]
Does treating the infinite series above like a polynomial really a valid way to produce a formula like the one above? It turns out we can treat these series as if we were just adding up an infinite amount of polynomials. Hence, which is why we have a whole section dedicated to \textit{power series}. What are some applications of power series? Well for one, we can express famous functions such as \( e^x  \), \( \sqrt{ 1 + x  }  \), or \( \sin(x)   \) in terms of an infinite series of polynomial terms. 
A notable example of power series is the generalization of the binomial formula. For any \( n \in \N  \), we have that 
\[  (1+x)^n  = 1 + nx + \frac{ n(n-1) }{ 2! } x^2 + \frac{ n(n-1) (n-2) }{ 3! } x^3 + \dots + x^n.\]
Say, we set \( n = -1  \), then our series is written as 
\[  \frac{ 1 }{ 1+x  } = 1 - x + x^2 - x^3 + x^4 - \dots, \]
which is equivalent (1). Setting \( n = 1/2 \) then our infinite series becomes
\[ \sqrt{ 1 + x  } = 1 + \frac{ 1 }{ 2 } x - \frac{ 1 }{ 2^2 2! } x^2 + \frac{ 3 }{ 2^3 3! } x^3 - \frac{ 3 \cdot 5  }{ 2^4 3! } x^4 + \dots ~ .  \] There are many more examples such as this that uses some sophisticated machinery that we do not quite know yet. One very important question we can ask ourselves is what properties of power series allows them to be manipulated in such a way that is so impervious to the infinite? We will explore this more in the upcoming sections.







\section{Uniform Convergence of a Sequence of Functions}


Just like our studies demonstrated in Chapter 2, we will first study the behaviors and properties of converging \textit{sequences} of functions. The results that we have gathered about sequences and series so fat will be immediately applicable to our study of sequences of functions.

\subsection{Pointwise Convergence} 

\begin{tcolorbox}
\begin{defn}
For each \( n \in \N  \), let \( f_n \) be a function defined on a set \( A \subseteq \R  \). The sequence \( (f_n)  \) of functions \textit{converges pointwise }  on \( A  \) to a function \( f  \) if, for all \( x \in A  \), the sequence of real numbers \( f_n(x)  \) converges to \( f(x)  \). 

In this case, the following notations are all equivalent to each other
\begin{enumerate}
    \item[(i)] \( f_n \to f  \) 
    \item[(ii)] \( \lim f_n = f  \)
    \item[(iii)] \( \lim_{ n \to \infty  } f_n(x) = f(x)  \).
\end{enumerate}
\end{defn}
\end{tcolorbox}

(iii) of the definition above is especially useful if there are any confusions the may arise as to whether or not \( x  \) or \( n  \) is the limiting variable.

\begin{ex}
\begin{enumerate}
    \item[(i)] Consider the sequence of functions \( f_n  \) defined by 
        \[  f_n(x) = \frac{ x^2 + nx  }{ n } \]
        on all of \( \R  \). We can compute the limit of \( f_n  \)
        \[  \lim_{ n \to \infty  } f_n(x) = \lim_{ n \to \infty  } \frac{ x^2 + nx  }{ n  } = \lim_{ n \to \infty  } \frac{ x^2  }{ n } + x  = x. \] Thus, we have that \( (f_n) \) converges \textit{pointwise} to \( f(x) = x  \) on \( \R  \).
    \item[(ii)] Let \( g_n(x) = x^n  \) on the set \( [0,1]  \) where we consider the situation as \( n \to \infty  \). If \( 0 \leq x < 1  \), then we know that \( x^n \to 0  \). On the other hand, suppose \( x = 1  \), then we have that \( x^n \to 1  \). It follows that \( g_n \to g  \) converges pointwise on \( [0,1] \), where 
        \[  g(x) = 
        \begin{cases}
            0 &\text{for } 0 \leq x < 1 \\
            1 &\text{for } x = 1.
        \end{cases} \]
        We have a problem when considering continuity at \( x = 1  \).
    \item[(iii)] Consider \( h_n(x) = x^{1+ \frac{ 1 }{ 2n - 1  } } \) on the set \( [-1,1]  \). For a fixed \( x \in [-1,1]  \), we have 
        \[  \lim_{ n \to \infty  } h_n(x) = x \lim_{ n \to \infty  } x^{\frac{ 1 }{ 2n-1 } }  = | x  |.\]
        Note that this function is not differentiable at \( x = 0  \).
\end{enumerate}
\end{ex}

\subsection{Continuity of the Limit Function}

We will begin this section by failing to prove that the pointwise limit of continuous functions is continuous. We will then find the holes of the subsequent argument so that we may understand why we need a stronger footing on the meaning of convergence for a sequence of functions.

Let \( (f_n)  \) be a sequence of continuous functions on a set \( A \subseteq \R  \) and let us assume that \( (f_n)  \) converges to a pointwise limit \( f  \). We will try to argue that the limit \( f  \) is continuous. Let us fix \( c \in A  \), and let \( \epsilon > 0  \). Our objective is to find  \( \delta > 0 \) such that whenever \( | x - c  | < \delta  \), we have 
\[  | f(x) - f(c) | < \epsilon. \]
We may use the Triangle Inequality to write
\begin{align*}
    | f(x) - f(c)  | &= | f(x) - f_n(x) + f_n(x) - f_n(c) + f_n(c) - f(c)   |  \\
                     &\leq | f(x) - f_n(x)  | + | f_n(x) - f_n(c)  | + | f_n(c) - f(c)  |.
\end{align*}
Our impression is to make each term of the right hand side of this inequality small by using the fact that \( f_n \to f  \) and the continuity of \( f_n  \).  Since \( c \in A  \) is fixed, let us choose \( N \in \N  \) such that 
\[  | f_N(c) - f(c)  | < \frac{ \epsilon  }{ 3 }. \]
Since \( N  \) is chosen, the continuity of our particular choice \( f_N  \) implies that there exists a \(  \delta > 0  \) such that 
\[  | f_N(x) - f_N(c)  | < \frac{ \epsilon  }{ 3 } \]
for all \( x  \) whenever \( | x - c  | < \delta  \). But here lies the problem of using the continuity of \( f_n  \); that is, we also need the following to hold:
\[  | f_N(x) - f(x)  | < \frac{ \epsilon  }{ 3 }  \] for all \( x  \) satisfying \( | x - c  | < \delta  \). A few problems with this argument include 
\begin{enumerate}
    \item[(i)] Our choice of \( x  \) depends on \( \delta  \) which also depends on our choice of \( N  \). This means for every choice of \( x  \) along \( (c - \delta, c + \delta ) \), we will get a different \( N  \). We want our choice of \( \delta  \) to be uniform for any \( x  \).
    \item[(ii)] The choice of \( x  \) is not fixed the way \( c  \) is on the interval \( (c - \delta, c + \delta ) \). This means that our choice \( x  \) has to work along the interval.
\end{enumerate}
This problem is apparent in our second example at the beginning of this section where the inequality
\[  | g_n(1/2 ) - g(1/2)  | < \frac{ 1 }{ 3 } \]
for \( n \geq 2  \) whereas 
\[  |g_n(9/10) - g(9/10)   | < \frac{ 1 }{ 3 }  \]
is true only after \( n \geq 11 \).

\subsection{Uniform Convergence} 

To solve our the problems of pointwise convergence of functions, we introduce a stronger notion for convergence of functions.

\begin{tcolorbox}
    \begin{defn}[Uniform Convergence]
    Let \( (f_n)  \) be a sequence of functions defined on a set \( A \subseteq \R  \). Then, \( (f_n)  \) \textit{converges uniformly} on \( A  \) to a limit function \( f  \) defined on \( A  \) if, for every \( \epsilon  > 0 \), there exists an \( N \in \N  \) such that 
    \[  | f_n(x) - f(x)  | < \epsilon   \]
    whenever \( n \geq N  \) and \( x \in A  \).
    \end{defn}
\end{tcolorbox}

Let us restate the definition of Pointwise convergence so that we are able to distinguish the key differences between the two.

\begin{tcolorbox}
    \begin{defn}[Pointwise Convergence]
    Let \( (f_n)  \) be a sequence of functions defined on a set \( A \subseteq \R  \). Then, \( (f_n)  \) \textit{converges pointwise on} A to a limit \( f  \) defined on \( A  \) if, for every \( \epsilon > 0   \) and \( x \in A  \), there exists an \( N \in \N  \) (may depend on x ) such that 
    \[  | f_n(x) - f(x)  | < \epsilon  \] whenever \( n \geq N  \).
    \end{defn}
\end{tcolorbox}

Key Differences:
\begin{enumerate}
    \item[(i)] In uniform convergence, notice that we only need 
        \[ | f_n(x) - f(x)  | < \epsilon  \] to hold for all \( \epsilon > 0  \); that is, our choice of \( x    \) will not affect our choice of \( N  \). Another way to state this is \( N \neq N(\epsilon, x ) \)
    \item[(ii)] In pointwise convergence, not only do we need convergence to hold for all \( \epsilon > 0   \), we also need it to hold for all \( x  \).
\end{enumerate}

\begin{ex}
\begin{enumerate}
    \item[(i)] Let 
        \[  g_n(x) = \frac{ 1 }{ n(1 + x^2 ) }.  \]
        For any fixed \( x \in \R  \), it is apparent that \( \lim_{ n \to \infty  } g_n(x) = 0  \) so that \( g(x) = 0   \) is the pointwise limit of the sequence \( (g_n)  \) on \( \R  \). We want to know if \( (g_n)  \) uniformly convergent. Since \( 1 / (1 + x^2 )  \leq 1  \) for all \( x \in \R  \) implies that 
        \[  | g_n(x) - g(x)  | = \Big| \frac{ 1 }{ n(1+x^2 )  } - 0  \Big| \leq \frac{ 1 }{ n }.  \]
Hence, any given \( \epsilon > 0  \), we can choose \( N > 1 / \epsilon  \) (which does not depend on \( x  \)), we have that
\begin{center}
    \( n \geq N  \) implies \( | g_n(x) - g(x)  | < \epsilon  \)
\end{center}
for all \( x \in \R  \). Hence, \( g_n \to 0 \) uniformly on \( \R  \).
\item[(ii)] What about our first example from the very beginning of this section? Does it converge uniformly as well?  Let \( f_n(x) = (x^2 + n x ) / n  \). Since \( (f_n) \to f  \) pointwise where \( f(x) =x  \). It turns our that \( f_n  \) is not uniformly convergent. To see why this is the case, we write that
    \[  | f_n(x) - f(x)  | = \Big| \frac{ x^2 + nx  }{ n  } - x  \Big| = \frac{ x^2  }{ n }. \]
    For \( | f_n(x) - f(x)  | < \epsilon  \) to hold, we would need to create a choice of \( N  \) such that 
    \[  N  > \frac{ x^2  }{ \epsilon  }. \]
    While we certainly have convergence for every \( x \in \R  \), we still have our choice of \( N  \) not uniform. Although not uniformly convergent on all of \( \R  \), we do end up having uniform convergence when we consider \( f_n  \) over a closed interval \( [-b,b] \). Hence, we have that 
    \[  \frac{ x^2  }{ n }  \leq \frac{ b^2  }{ n  }. \]
    Given any \( \epsilon > 0  \), we can choose \( N > b^2 / \epsilon  \) that is not dependent on any \( x \in [-b,b]  \).
\end{enumerate}
\end{ex}

Graphically speaking, the uniform convergence of \( f_n  \) to a limit \( f  \) on a set \( A  \) can be visualized by constructing an \( \epsilon -  \)neighborhood around the limit \( f  \) for which all of \( f_n  \) is completely contained within the neighborhood for all \(  n \geq N  \) for some point \( N \in \N  \).

\subsection{Cauchy Criterion}

Recall that the Cauchy Criterion states an equivalence between convergent sequences and Cauchy sequences without stating the limit of the sequence. The usefulness of such a theorem creates an opportunity for an analogous  characterization of uniformly convergent sequences of functions.

\begin{tcolorbox}
    \begin{thm}[Cauchy Criterion for Uniform Convergence]
    A sequence of functions \( (f_n)  \) defined on a set \( A \subseteq \R  \) converges uniformly on \( A  \) if and only if for every \( \epsilon > 0  \), there exists an \( N \in \N  \) such that 
    \[  | f_n(x) - f_m(x)  | < \epsilon  \]
    whenever \( m,n \geq N  \) and \( x \in A  \).
    \end{thm}
\end{tcolorbox}
\begin{proof}
Exercise 6.2.5.
\end{proof}

\subsection{Continuity Revisited}

Let us now prove that the limit function of a sequence of continuous functions is continuous.

\begin{tcolorbox}
    \begin{thm}[Continuous Limit Theorem]
    Let \( (f_n)  \) be a sequence of functions defined on \( A \subseteq \R  \) that converges uniformly on \( A  \) to a function \( f  \), If each \( f_n   \) is continuous at \( c \in A  \), then \( f  \) is continuous at \( c  \).
    \end{thm}
\end{tcolorbox}

\begin{proof}
Fix \( c \in A  \) and let \( \epsilon > 0  \). Since \( (f_n) \to f   \) is uniformly convergent on \( \R  \), we can choose an \( N \in \N   \) (that does not depend on \( x  \)) such that 
\[  | f_N(x) - f(x)  | < \frac{ \epsilon  }{ 3  }  \]
for all \( x \in A  \). Since \( f_N  \) is continuous, there exists \( \delta > 0  \) for which 
\[  | f_N(x) - f_N(c)  | < \frac{ \epsilon  }{ 3 }  \]
is true whenever \( | x - c  | < \delta  \). Just like our argument at the beginning of this section, we have that
\begin{align*}
    | f(x) - f(c)  | &= | f(x) - f_N(x) + f_N(x) - f_N(c) + f_N(c) - f(c)  |  \\
                     &\leq | f(x) - f_N(x)  | + | f_N(x) - f_N(c)  | + | f_N(c) - f(c)  | \\
                     &< \frac{ \epsilon  }{ 3  } + \frac{ \epsilon  }{ 3 } + \frac{ \epsilon  }{ 3  } \\
                     &= \epsilon.
\end{align*}
Hence, \( f  \) is continuous at \( c \in  A  \).
\end{proof}




\subsection{Definitions and Theorems}


\begin{tcolorbox}
\begin{defn}
For each \( n \in \N  \), let \( f_n \) be a function defined on a set \( A \subseteq \R  \). The sequence \( (f_n)  \) of functions \textit{converges pointwise }  on \( A  \) to a function \( f  \) if, for all \( x \in A  \), the sequence of real numbers \( f_n(x)  \) converges to \( f(x)  \). 

In this case, the following notations are all equivalent to each other
\begin{enumerate}
    \item[(i)] \( f_n \to f  \) 
    \item[(ii)] \( \lim f_n = f  \)
    \item[(iii)] \( \lim_{ n \to \infty  } f_n(x) = f(x)  \).
\end{enumerate}
\end{defn}
\end{tcolorbox}


\begin{tcolorbox}
    \begin{defn}[Uniform Convergence]
    Let \( (f_n)  \) be a sequence of functions defined on a set \( A \subseteq \R  \). Then, \( (f_n)  \) \textit{converges uniformly} on \( A  \) to a limit function \( f  \) defined on \( A  \) if, for every \( \epsilon  > 0 \), there exists an \( N \in \N  \) such that 
    \[  | f_n(x) - f(x)  | < \epsilon   \]
    whenever \( n \geq N  \) and \( x \in A  \).
    \end{defn}
\end{tcolorbox}


\begin{tcolorbox}
    \begin{defn}[Pointwise Convergence]
    Let \( (f_n)  \) be a sequence of functions defined on a set \( A \subseteq \R  \). Then, \( (f_n)  \) \textit{converges pointwise on} A to a limit \( f  \) defined on \( A  \) if, for every \( \epsilon > 0   \) and \( x \in A  \), there exists an \( N \in \N  \) (may depend on x ) such that 
    \[  | f_n(x) - f(x)  | < \epsilon  \] whenever \( n \geq N  \).
    \end{defn}
\end{tcolorbox}


\begin{tcolorbox}
    \begin{thm}[Cauchy Criterion for Uniform Convergence]
    A sequence of functions \( (f_n)  \) defined on a set \( A \subseteq \R  \) converges uniformly on \( A  \) if and only if for every \( \epsilon > 0  \), there exists an \( N \in \N  \) such that 
    \[  | f_n(x) - f_m(x)  | < \epsilon  \]
    whenever \( m,n \geq N  \) and \( x \in A  \).
    \end{thm}
\end{tcolorbox}


\begin{tcolorbox}
    \begin{thm}[Continuous Limit Theorem]
    Let \( (f_n)  \) be a sequence of functions defined on \( A \subseteq \R  \) that converges uniformly on \( A  \) to a function \( f  \), If each \( f_n   \) is continuous at \( c \in A  \), then \( f  \) is continuous at \( c  \).
    \end{thm}
\end{tcolorbox}





\section{Uniform Convergence and Differentiation}

We shall start this section by asking what is the effect of having a pointwise converging sequence of functions that are differentiable? It turns out that if we require the sequence of derivatives of some function to be uniformly convergent, then the limit of the sequence of derivative is the derivative of the original function.

\begin{theorem}{Differentiable Limit Theorem}{}
        Let \( f_n \to f  \) pointwise on the closed interval \( [a,b] \), and assume that each \( f_n  \) differentiable. If \( (f'_n) \) converges uniformly on \( [a,b] \) to a function \( g  \), then the function \( f  \) is differentiable and \( f' = g  \).
    \end{theorem}%

\begin{proof}
    Fix \( c \in [a,b]  \) and let \( \epsilon > 0  \). We want to show that \( f'(c)  \) exists and equals \( g(c)  \); that is, we want to show that for all \( \epsilon > 0  \), there exists \( \delta > 0  \) such that 
    \[ \Big| \frac{ f(x) - f(c) }{ x - c  } - g(c)  \Big| < \epsilon  \]
    whenever \( 0 < | x -c  | < \delta  \). We can do this by observing that for all \( n \geq N  \) and \( x \neq c  \), we can use the triangle inequality to say that 
    \begin{align*}
        \Big| \frac{ f(x) - f(c)  }{ x - c  } - g(c)  \Big| &\leq \Big| \frac{ f(x) - f(c)  }{  x- c  } - \frac{ f_n(x) - f_(c)  }{ x - c  }  \Big| \\ &+ \Big| \frac{ f_n(x) - f_n(c)  }{ x - c  } - f_n'(c)  \Big|  + | f_n'(c) - g(c)   |. \\
    \end{align*}
    We can make the last two terms "small" by having them be both less than \( \epsilon / 3  \). Since \( (f'_n) \to g  \) uniformly, we can choose an \( N_1 \in \N  \) such that for any \( n \geq N  \) that 
    \[ | f'_n(c) - g(c)  | < \frac{ \epsilon  }{ 3 }. \tag{1} \]
We can also invoke the uniform convergence of \( f'_n  \) to state that for any \( m,n \geq N_2  \) for some \( N_2 \in \N  \) that 
\[ | f_n(x) - f_m(x)  | < \frac{ \epsilon  }{ 3 }  \]
for all \( x \in [a,b] \).
    Furthermore, for \( x \neq c  \), we can make 
    \[  \Big| \frac{ f_n(x) - f_n(c)  }{ x -c  } - f_n'(c)    \Big| < \frac{ \epsilon  }{ 3  }. \tag{2}  \]
    whenever \( 0 < | x - c  | < \delta  \).
    
    \[   \]
    The first term can be made small by using the Mean Value Theorem. Fix an \( x  \) satisfying \( 0 < | x - c  | < \delta  \) and let \( m \geq N  \), and apply the Mean Value Theorem to \( f_m - f_N  \) on the interval \( [c,x] \). Suppose \( x > c  \), then there exists an \( \alpha \in (c,x) \) such that 
    \[  f'_m(\alpha) - f'_N(\alpha)  = \frac{ (f_m(x) - f_N(x)) - (f_m(c) - f_N(c) ) }{ x - c  }. \]
    Since we have \( m \geq N  \) for some \( N \in \N  \), we can have 
    \[  | f'_m(\alpha) - f'_N(\alpha)  | < \frac{ \epsilon  }{ 3 }, \]
    which means that 
    \[  \Bigg| \frac{ f_m(x) - f_m(c)  }{ x -c  } - \frac{ f_N(x) - f_N(c)  }{  x- c  }  \Bigg| < \frac{ \epsilon  }{ 3 }. \tag{3} \] 
    Since \( f_m \to f  \) pointwise, we can use the Order Limit Theorem to imply that 
    \[  \Big| \frac{ f(x) - f(c)   }{ x- c  } - \frac{ f_n(x) - f_n(c)  }{ x - c  }  \Big| \leq \frac{ \epsilon  }{ 3 }. \]
    Using (1), (2), and (3), we can now conclude that  
    \begin{align*}
        \Big| \frac{ f(x) - f(c)  }{ x - c  } - g(c)  \Big| &\leq \Big| \frac{ f(x) - f(c)  }{  x- c  } - \frac{ f_n(x) - f_(c)  }{ x - c  }  \Big| \\ &+ \Big| \frac{ f_n(x) - f_n(c)  }{ x - c  } - f_n'(c)  \Big|  + | f_n'(c) - g(c)   | \\
                                                            &< \frac{ \epsilon  }{ 3  }  + \frac{ \epsilon  }{ 3  } + \frac{ \epsilon  }{ 3  } \\
                                                            &=\epsilon.
    \end{align*}
    Hence, \( (f'_n) \to g  \) and \( f' = g  \).
\end{proof}

As it turns out, we don't really need to assume that \( f_n(x) \to f(x)  \) for the conclusion above to be true. We only need uniform convergence of \( (f'_n)  \) for the theorem above to work. Two functions with the same derivative may differ by a constant, so we must suppose that there is at least one point \( x_0  \) where \( f_n(x_0) \to f(x_0)  \).

\begin{theorem}{}{}
    Let \( (f_n)  \) be a sequence of differentiable functions defined on the closed interval \( [a,b]  \), and assume \( (f'_n)  \) converges uniformly on \( [a,b]  \). If there exists a point \( x_0 \in [a,b]  \) where \( f_n(x_0)  \) is convergent, then \( (f_n)  \) converges uniformly on \( [a,b]  \).
\end{theorem}

\begin{proof}
    Let \( (f_n)  \) be a sequence of differentiable functions defined on the closed interval \( [a,b]  \), and assume \( (f'_n) \) converges uniformly on \( [a,b]  \). Furthermore, assume that there exists a point \( x_0 \in [a,b] \) where \( f_n(x_0)  \) is convergent. Since \( (f'_n)  \) converges uniformly on \( [a,b] \), let \( \epsilon = 1  \) such that  there exists \( N \in \N \) such that for any \( m,n \geq N \) and \( x \in [a,b]  \), we have that 
    \[  | f_n'(x) - f_m'(x)   | < \epsilon. \tag{1}\] Since \( f_n  \) is differentiable on \( [a,b]  \), we can use the Mean Value Theorem to state that there exists an \( \alpha \in (x_0, x)   \) such that 
    \[  f'_n(\alpha) = \frac{ f_n(x) - f_n(x_0)  }{ x - x_0  }  \]
    and 
    \[  f'_m(\alpha) = \frac{ f_m(x) - f_m(x_0)  }{ x - x_0 }. \]
   Using the fact that \( (f'_n)  \) converges uniformly, we know that 
   \[ | f_n'(\alpha) - f_m'(\alpha) | < 1. \] This implies that
   \[  \Big| \frac{ f_n(x) - f_n(x_0)  }{ x - x_0  } - \frac{ f_m(x) - f_m(x_0) }{ x - x_0 }  \Big| < 1. \] This implies that 
   \[  | f_n(x) - f_n(x_0) - (f_m(x) - f_m(x_0) ) | < | x - x_0  |  \]
   for which we can assume \( 0 < | x -x_0 | < \delta  \) since \( f_n  \) is differentiable. Using the reverse triangle inequality, we can take the left side of the above inequality and state that
   \[   | f_n(x) - f_m(x)  |  - | f_m(x_0) - f_n(x_0) | \leq | f_n(x) - f_m(x) - (f_m(x_0) - f_n(x_0) ) | < | x - x_0  |  \]
   which manipulating even further implies 
   \begin{align*}  | f_n(x) - f_m(x)  | &\leq | f_n(x) - f_m(x) - (f_m(x_0) - f_n(x_0) ) |  \\
       &+ | f_m(x_0) - f_n(x_0)  | \\
       &< | x - x_0 | + | f_m(x_0) - f_n(x_0) |  \tag{2}. 
   \end{align*}
   Using the triangle inequality of the left side of (2), using the fact that \( f_n(x_0) \to f(x_0)  \) and setting \( \delta = \epsilon / 3  \), we can say that for some \( N \in \N  \) where, we have that for any \( m,n \geq N  \) 
   \begin{align*}  | f_n(x) - f_m(x)  | &\leq | f_n(x) - f_m(x) - (f_m(x_0) - f_n(x_0) ) |  \\
       &+ | f_m(x_0) - f_n(x_0)  | \\
       &< | x - x_0 | + | f_m(x_0) - f_n(x_0) |  \\ 
       &= | x -x_0  | + | f_m(x_0) - f(x_0)  | + | f(x_0) - f_n(x_0) | \\
       &< \frac{ \epsilon  }{ 3 } + \frac{ \epsilon  }{ 3 } + \frac{ \epsilon  }{ 3  } = \epsilon.
   \end{align*}
   Hence, this means that \( (f_n)  \) is uniformly convergent.
\end{proof}%

Now we have a stronger version of the first theorem of this section.

\begin{theorem}{}{}
    Let \( (f_n)  \) be a sequence of differentiable functions defined on the closed interval \( [a,b]  \), and assume \( (f'_n)  \) converges uniformly to a function \( g  \) on \( [a,b]  \). If there exists a point \( x_0 \in [a,b]  \) for which \( f_n(x_0)  \) is convergent, then \( (f_n)  \) converges uniformly. Moreover, the limit function \( f = \lim f_n  \) is differentiable and satisfies \( f' = g  \).
\end{theorem}



\section{Series of Functions}

\begin{tcolorbox}
\begin{defn}
    For each \( n \in \N  \), let \( f_n  \) and \( f  \) be functions defined on a set \( A \subseteq \R  \). The infinite series 
    \[  \sum_{ n=1 }^{ \infty  } f_n(x) = f_1(x) + f_2(x) + f_3(x) + \dots \]
    \textit{converges pointwise}  on \( A  \) to \( f(x)  \) if the sequence \( s_k(x)  \) of partial sums defined by 
    \[  s_k(x) = f_1(x) + f_2(x) + \dots + f_k(x) \]
    converges pointwise to \( f(x)  \). The series \textit{converges uniformly} on \( A  \) to \( f  \) if the sequence \( s_k(x)  \) converges uniformly on \( A  \) to \( f(x)  \). In either case, we write 
    \[  f = \sum_{ n=1 }^{ \infty  } f_n  \] or 
    \[  f(x) = \sum_{ n=1 }^{ \infty  } f_n(x)  \] always being explicit about the type of convergence involved.
\end{defn}
\end{tcolorbox}

Suppose we have a series \( \sum_{ n=1 }^{ \infty  }f_n  \) where the functions \( f_n  \) are continuous. We can guarantee that the partial sums of this series will be continuous as well by using the Algebraic Continuity Theorem. If the each \( f_n  \) is differentiable, then we can use the same reasoning to prove that the partial sums are also differentiable.

\begin{tcolorbox}
    \begin{thm}[Term-by-term Continuity Theorem]
    Let \( f_n  \) be continuous functions defined on a set \( A \subseteq \R  \), and assume \( \sum_{ n=1 }^{ \infty  } f_n  \) converges uniformly on \( A  \) to a function \( f  \). Then, \( f  \) is continuous on \( A  \).
    \end{thm}
\end{tcolorbox}

\begin{proof}
Since \( \sum_{ n=1 }^{ \infty  } f_n  \) converges uniformly on \( A  \) to a function \( f  \), the sequence of partial sums 
\[  s_k = f_1 + f_2 + \dots + f_k \]
converge uniformly to some function \( s  \) on \( A  \). Furthermore, \( f_n \) being a sequence of continuous functions also implies that \( s_k  \) is continuous. Since \( s_k \to s  \) uniformly and continuous, we must also have that \( s  \) is continuous by the Continuous Limit Theorem. Hence, \( \sum_{ n=1 }^{ \infty  } f_n = f  \) must be continuous.
\end{proof}

\begin{tcolorbox}
    \begin{thm}[Term-by-term Differentiability Theorem]
        Let \( f_n  \) be differentiable functions defined on an interval \( A  \), and assume \( \sum_{ n=1 }^{ \infty  } f_n'(x)  \) converges uniformly to a limit \( g(x) \). If there exists a point \( x_0 \in [a,b]  \) where \( \sum_{ n=1 }^{ \infty  } f_n(x_0)  \) converges, then the series \( \sum_{ n=1 }^{ \infty  } f_n(x)  \) converges uniformly to a differentiable function \( f(x)  \) satisfying \( f'(x)  = g(x)  \) on \( A  \), In other words, 
        \[  f(x) = \sum_{ n=1 }^{ \infty  } f_n(x) \text{ ~and~ } f'(x) = \sum_{ n=1 }^{ \infty  } f_n'(x).  \] 
    \end{thm}
\end{tcolorbox}

\begin{proof}
Our goal is to use the stronger version of the Differentiable Limit Theorem to state that the partial sums 
\[  s_k = f_1 + f_2 + \dots + f_k  \]
of the series \( \sum_{ n=1 }^{ \infty  } f_n(x)  \) converges uniformly. Since \( (f_n)  \) is a sequence of differentiable functions defined on an interval \( A  \), we know that the partial sums 
\[  s_k' = f_1'  + f_2' + \dots + f_k' \tag{1}\]
holds by the Algebraic Differentiability Theorem. Since \( \sum_{ n=1 }^{ \infty  } f'(x) \) converges uniformly to \( g(x)  \) on \( A  \), we know that (1) must converge to some function \( \ell(x)  \) on \( A  \). Furthermore, there exists \(  x_0 \in [a,b] \) such that 
\[  s_k(x_0)  = f_1(x_0) + f_2(x_0) + \dots + f_k(x_0) \] converges to \( s(x_0)  \).
Since \( s_k' \to \ell \) uniformly and \( s_k(x_0) \to s(x_0)  \) for some \( x_0 \in [a,b]  \), we know that \( s_k  \) must converge uniformly to the function \( s  \) where \( \lim s_k = s  \) by the stronger version of the Differentiable Limit Theorem and that \( s' = \ell  \). By definition, this means that 
\[  f(x) = \sum_{ n=1 }^{ \infty  } f_n(x) \text{ ~and~ } f'(x) = \sum_{ n=1 }^{ \infty  } f_n'(x) \]
and \( f'(x) = g(x)  \) on \( A  \).
\end{proof}

We can characterize the convergence of series of functions \( f_n \) using the Cauchy Criterion.

\begin{tcolorbox}
    \begin{thm}[Cauchy Criterion for Uniform Convergence of Series]
    A series \( \sum_{ n=1 }^{ \infty  } f_n  \) converges uniformly on \( A \subseteq \R  \) if and only if for every \( \epsilon > 0  \) there exists an \( N \in \N  \) such that 
    \[  | f_{m+1}(x) + f_{m+2} + f_{m+3} + \dots + f_n(x)  | < \epsilon \]
    whenever \( n > m \geq N  \) and \( x \in A  \).
    \end{thm}
\end{tcolorbox}

Uniform convergence allows us to develop a tool to determine when a series converges uniformly.

\begin{tcolorbox}
    \begin{cor}[Weierstrass M-Test]
    For each \( n \in \N  \), let \( f_n  \) be a function defined on a set \( A \subseteq \R  \), and let \( M_n > 0  \) be a real number satisfying
    \[  | f_n(x)  | \leq M_n \] for all \( x \in A  \). If \( \sum_{ n=1 }^{ \infty  } M_n  \) converges, then \( \sum_{ n=1 }^{ \infty  } f_n  \) converges uniformly on \( A  \).
    \end{cor}
\end{tcolorbox}

\begin{proof}
Exercise 6.4.1.
\end{proof}

\subsection{Exercises}

\subsubsection{Exercise 6.4.1} Supply the details for the proof of the Weierstrass M-test (Corollary 6.4.5).

This proof will use the Comparison Test and Absolute Convergence to show that \( \sum_{ n=1 }^{ \infty  } f_n  \) converges uniformly.
\begin{proof}
Let \( n \in \N  \) be arbitrary such that \( f_n  \) is defined on a set \( A \subseteq \R  \) and let \( M_n > 0  \) be a real number such that 
\[  | f_n(x)  |  \leq M_n \]
for all \( x \in A  \). Let \( x \in A  \) be arbitrary. Suppose \( \sum_{ n=1 }^{ \infty  } M_n  \) converges. Our goal is to use to Comparison Test to show that the series 
\[  \sum_{ n=1 }^{ \infty  } f_n  \] converges absolutely. Since \( | f_n(x)  |  \leq M_n  \) and \( M_n > 0  \), we know that
\[  \sum_{ n=1 }^{ \infty  } | f_n(x)  | \leq \sum_{ n=1 }^{ \infty  } M_n. \tag{1}\]
But \( \sum_{ n=1 }^{ \infty  } M_n  \) converges. Thus, we know that the left side of (1) must also converge via the Comparison Test. This means that the series 
\[  \sum_{ n=1 }^{ \infty  } f_n(x)  \] converges. Since \( x \in A  \) is arbitrary, the series above must converge uniformly on \( A  \).
\end{proof}

This proof uses the Cauchy Criterion to prove that the series converges uniformly.

\begin{proof}
Let \( M_n > 0  \) be a real number satisfying \( | f_n(x)  | \leq M_n  \) for all \( x \in A  \).Let \( \epsilon > 0 \). Our goal is to use the Cauchy Criterion for Uniform Convergence of Series to prove that \( \sum_{ n=1 }^{ \infty  } f_n  \) converges uniformly. Suppose \( \sum_{ n=1 }^{ \infty  } M_n  \) converges, then for some \( N \in \N  \), we have for any \( n > m \geq N  \) implies 
\[ \Big| \sum_{ k=m+1 }^{ n } M_k \Big| < \epsilon.   \]
Since \( M_n > 0  \), this result can be restated as follows
\[  \sum_{ k=m+1 }^{ n } M_k < \epsilon. \]
By letting \( n > m \geq N  \) and \( x \in A  \) arbitrary as before, we can use the triangle inequality to state that 
\begin{align*}
    \Big| \sum_{ k=m+1 }^{ n } f_k(x)  \Big| &\leq \sum_{ k=m+1 }^{ n }| f_k(x)  |  \\
                                             &\leq  \sum_{ k=m+1 }^{ n } M_k \\
                                             &< \epsilon.
\end{align*}
By the Cauchy Criterion for Uniform Convergence, we know that \( \sum_{ n=1 }^{ \infty  } f_m(x)  \) must converge uniformly on \( A  \),
\end{proof}

\subsubsection{Exercise 6.4.2} Decide whether each proposition is true or false, providing a short justification or counterexample as appropriate.  

\begin{enumerate}
    \item[(a)] If \( \sum_{ n=1 }^{ \infty  } g_n  \) converges uniformly, then \( (g_n)  \) converges uniformly to zero.
        \begin{proof}[Solution]
         Since \( \sum_{ n=1 }^{ \infty  } g_n  \) converges uniformly, we know that 
        \[  \lim_{ n \to \infty  } \sum_{ k=1 }^{ n } g_k(x)  \]
        converges uniformly. If we consider the case when \( n = m - 1  \), then we have that 
        \[  \lim_{ n \to \infty  } \sum_{ k=1 }^{ n } g_k(x) = \lim_{ n \to \infty  } \Big( g_n(x) + \sum_{ k=1 }^{ n - 1  } g_n(x)  \Big)  \tag{1}. \]
        Furthermore, (1) implies that 
        \[  \lim_{ n \to \infty  }  \sum_{ k=1 }^{  n } g_k(x) = \lim_{ n \to \infty  }  g_n(x) + \lim_{ n \to \infty  } \sum_{ k=1 }^{ n-1 } g_n(x). \tag{2} \]
        Since the sequence of partial sums for \( \sum_{ n=1 }^{ \infty  } g_n(x)  \), we know that subtracting the second term on the right side of (2) results in 
        \[  \lim_{ n \to \infty  }  g_n(x) = 0. \]
        Hence, \( g_n \to 0  \).
        \end{proof}
        
    \item[(b)] If \( 0 \leq f_n(x) \leq g_n(x)  \) and \( \sum_{ n=1 }^{ \infty  } g_n  \) converges uniformly, then \( \sum_{ n=1 }^{ \infty  } f_n  \) converges uniformly.
        \begin{proof}
        Let \( x \in A  \) be arbitrary. Since \( 0 \leq f_n(x) \leq g_n(x)  \) and \( \sum_{ n=1 }^{ \infty  } g_n  \) satisfies the Cauchy Criterion for uniform convergence, we know that for any \( n > m \geq N  \) and \( x \in A  \), we have that
        \begin{align*}
            \Big| \sum_{ k=m+1 }^{ n } f_n(x)  \Big|  &\leq \Big| \sum_{ k=m+1 }^{ n } g_n(x) \Big| \\ 
                                                      &< \epsilon.
        \end{align*}
        Hence, \( \sum_{ n=1 }^{ \infty  } f_n(x)  \) converges uniformly.
        \end{proof}
    \item[(c)] If \( \sum_{ n=1 }^{ \infty  } f_n(x)  \) converges uniformly on \( A  \), then there exists constants \( M_n  \) such that \( | f_n(x)  | \leq M_n  \) for all \( x \in  A \) and \( \sum_{ n=1 }^{ \infty  } M_n  \) converges.
        \begin{proof}[Solution]
        Suppose our series is defined by 
        \[  \sum_{ n=1 }^{ \infty  } \frac{ (-1)^{n+1}  }{  n } \tag{1}\]
        where \( f_n(x) = (-1)^{n+1} / n  \). We can see by the Alternating Series test that (1) converges uniformly. But note that \( | f_n(x)  | \leq 1 / n = M_n   \) which produces the harmonic series 
        \[  \sum_{ n=1 }^{ \infty  } \frac{ 1 }{ n }  \]
        which diverges.
        
        \end{proof}
\end{enumerate}

\subsubsection{Exercise 6.4.3} 
\begin{enumerate}
    \item[(a)] Show that 
        \[  g(x) = \sum_{ n=0 }^{ \infty  } \frac{ \cos(2^n x) }{ 2^n }  \]
        is continuous on all of \( \R  \).
        \begin{proof}
            Our goal is to use the Term-by-term Continuity Theorem to show that \( g  \) is continuous on \( \R  \); that is, we want to show that \( g_n  \) is continuous and \( \sum_{ n=0 }^{ \infty  }  g_n  \) converges uniformly. Since \( \cos(x)  \) is a continuous function, we know that for each \( n \in \N  \), \( g_n  \) must be a sequence of continuous functions. All that is left is to show that \( \sum_{ n=0 }^{ \infty  } g_n(x)  \) converges uniformly. We shall do this by using the Weierstrass M-test. Since \( | \cos(2^n x ) | \leq 1  \) for all \( x \in \R  \), we know that \[  \sum_{ n=0 }^{ \infty  } \Big| \frac{ \cos(2^n x ) }{  2^n  }  \Big|  \leq \sum_{ n=0  }^{ \infty  } \Big( \frac{ 1 }{ 2 }  \Big)^n \tag{1}.  \]
            Since the right side of (1) is a geometric series that converges, we know that the series \( \sum_{ n=0 }^{ \infty  } g_n(x)  \) must converge uniformly to \( g(x)  \). Hence, \( g(x) \) is continuous on all of \( \R  \).
        \end{proof}
    \item[(b)] The function \( g  \) was cited in Section 5.4 as an example of a nowhere differentiable function. What happens if we try to use Theorem 6.4.3 to explore whether \( g  \) is differentiable?
    \item[(b)] We can define the series above via the sawtooth function from section 5.4 where 
        \[  g(x) = \sum_{ n=0 }^{  \infty  } h_n(x) = \sum_{ n=0  }^{  \infty  } 2^n h(2^n x ) \tag{1}.\]
        If we look at the terms \( h_n(x)  \), we can see that 
        \[ | h_n(x)  |  \leq \frac{ 1 }{ 2^n  } \tag{2} \] which forms a geometric series on the right side of (2). Hence, we must have (1) converge uniformly via the Weierstrass M-test.
\end{enumerate}

\subsubsection{Exercise 6.4.4} Define 
\[  g(x) = \sum_{ n=0  }^{ \infty  } \frac{ x^{2n} }{ (1+x^{2n}) }. \]
Find the values of \( x  \) where the series converges and show that we get a continuous function on this set.
\begin{proof}
    Let \( h_n(x) = \frac{ x^{2n} }{ (1 + x^{2n}) }    \) be the terms of the series 
    \[  \sum_{ n=0  }^{ \infty  } h_n(x).\] When \( | x  |  \geq 1  \), we observe that the terms of \( (h_n)  \) does not converge to \( 0  \) as \(  n \to \infty  \). If \( | x  |  < 1  \), then we have that 
    \[  | h_n(x)  |  \leq x^{2n}. \]
    Since \( | x  |  < 1  \), \( x^{2n} \) forms a geometric series that converges. Hence, we must have that the series of \( h_n(x)  \) must be uniformly convergent. Furthermore, for any \( 0 \leq a < 1  \), we will find that the infinite series of \( h_n(x)  \) converges uniformly where 
    \[  \sum_{ n=0  }^{ \infty  } \frac{ a^{2n} }{  1 + a^{2n} }. \]
\end{proof}

\subsubsection{Exercise 6.4.5} 
\begin{enumerate}
    \item[(a)] Prove that 
        \[  h(x) = \sum_{ n=1 }^{ \infty  } \frac{ x^n  }{ n^2  } = x + \frac{ x^2  }{ 4  } + \frac{ x^3  }{ 9  } + \frac{ x^4  }{ 16 } + \dots\]
        is continuous on \( [-1,1] \).
        \begin{proof}
        Using the Weierstrass M-test, we have that
        \[  \sum_{ n=1 }^{ \infty  } \Big| \frac{ x^n  }{ n^2  }  \Big|   \leq \sum_{ n=1 }^{ \infty  } \frac{ 1 }{ n^2  } \tag{1}.\]
        Since the series on the right side of (1) is a P-series, we know that it must converge which means that 
        \[  \sum_{ n=1 }^{ \infty  } \frac{ x^n  }{ n^2  }   \] must converge. Note the sequence of functions
        \[  h_n(x) = \frac{ x^n  }{ n^2  }  \] is continuous for each \( n \in \N  \) because for each \( n \in \N  \) \( h_n(x)  \) is just a polynomial which is continuous. Hence, we must have that 
        \[  h(x) = \sum_{ n=1  }^{ \infty  } \frac{ x^n  }{ n^2  }  \]
        is continuous on \( [-1,1] \).
        \end{proof}
    \item[(b)] The series 
        \[  f(x) = \sum_{ n=1 }^{ \infty  } \frac{ x^n  }{  n  }  = x + \frac{ x^2  }{ n  }  + \frac{ x^3  }{ 3  }  + \frac{ x^4  }{  4  }  + \dots \]
        converges for every \( x  \) in the half-open interval \( [-1,1) \) but does not converge when \( x = 1  \). For a fixed \( x_0 \in (-1,1)  \), explain how we can still use the Weierstrass M-test to prove that \( f  \) is continuous at \( x_0  \).
        \begin{proof}[Solution]
            Fix \( x_0  \), then consider any \( | x_0  |  < 1  \). Using the Weierstrass M-test, we will find that the terms 
            \[  \Big| \frac{ x_0^n  }{ n }  \Big|   \leq | x_0^2 |^n    \]
            forms a geometric series for which the right side will converge implying that the series of the left side will converge.
        \end{proof}
\end{enumerate}

\subsubsection{Exercise 6.4.7} Let 
\[ f(x) = \sum_{ k=1 }^{ \infty  } \frac{ \sin (kx)  }{  k^{3} }.  \]
\begin{enumerate}
    \item[(a)] Show that \( f(x)  \) is differentiable and that the derivative \( f'(x)  \) is continuous. 
        \begin{proof}
        We will use the term-by-term differentiability theorem to show that 
        \[  f(x) = \sum_{ k=1 }^{ \infty  } \frac{ \sin(kx) }{ k^3  }  \]
        is differentiable. We need to show that \( (f_k)  \) is differentiable for every \( k \in \N  \), \( \sum_{ k=1 }^{ \infty  } f'_n(x)  \) converges uniformly to some \( g(x)  \) in \( \R  \), and that for some \( x_0 \in [a,b] \subseteq \R  \) that \( \sum_{ k=1 }^{ \infty  } f_k(x_0)  \) converges to \( f(x_0)  \). 

        Note that for any \(k \in \N  \), we know that \( f_k(x)  \) is differentiable since \( \sin(kx)  \) is a differentiable function for all \( x \in \R  \). Now we compute \( f'_n(x)  \) which results in 
        \[  f'_n(x) = \frac{ \cos(kx)  }{  k^2  }. \]
        Since \( | \cos(kx)  |  \leq 1  \), we can state that 
        \[  \sum_{ k=1 }^{ \infty  } \Big| \frac{ \cos(kx)  }{ k^2  }  \Big|  \leq \sum_{ k=1 }^{ \infty  }  \frac{ 1 }{ k^2  } \tag{1}. \]
        Since the p-series on the right of (1) converges, we know that 
        \[ \sum_{ k=1 }^{ \infty  } f'_k(x)  \] 
        must converge uniformly to some \( g(x)  \) on \( \R  \). By the same reasoning, we can show that \( \sum_{ k=1 }^{ \infty  } f_n(x)  \) converges for all \( x \in \R  \); that is, since \( | \sin(kx)   |  \leq 1  \), we know that
        \[  \sum_{ k=1 }^{ \infty  } \Big| \frac{ \sin(kx)  }{ k^3  }  \Big|  \leq \sum_{ k=1 }^{ \infty  } \frac{ 1 }{ k^3  }  \]
        where the p-series on the right converges which implies that \( \sum_{ k=1 }^{ \infty  } f_n(x)  \) converges. By the Term-by-term differentiability theorem, we know that 
        \[  \sum_{ k=1 }^{ \infty  } \frac{ \sin(kx)  }{ k^3  }   \]
        must converge uniformly to a differentiable function \( f(x) \). Furthermore, we know that \( f'(x)  \) is continuous since \( f'_n  \) for all \( n \in \N  \) is continuous and 
        \[  \sum_{ k=1 }^{ \infty  } \frac{ \cos(kx) }{ k^2  }  \]
        converges uniformly on \( \R  \).
        \end{proof}
    \item[(b)] Can we determine if \( f  \) is twice-differentiable?
        We claim that \( f  \) is not twice-differentiable. If we compute \( f"_n(x)  \) and end up with the series
        \[  \sum_{ k=1 }^{ \infty  } \frac{ \sin(kx)  }{ k }  \]
        , then the series above fails the Weierstrass M-test since the constant series 
        \[ \sum_{ k=1 }^{ \infty  } \frac{ 1 }{ k}  \]
        diverges (this is a harmonic series). 

\end{enumerate}

\subsubsection{Exercise 6.4.8} Consider the function 
\[  f(x) = \sum_{ k=1 }^{  \infty  } \frac{ \sin(x / k) }{ k }.  \]
Where is \( f  \) defined? Continuous? Differentiable? Twice-differentiable?
\begin{proof}[Solution]
The function \( f  \) is defined on every \( x \in \R  \). We first claim that \( f  \) is continuous. We do this by showing that 
\[  \sum_{ k=1 }^{ \infty  } \frac{ \sin(x/k)  }{ k  }  \] converges uniformly and that each \( f_n  \) is continuous. Since \( | \sin(k/x)  |  \leq | kx  |  \), we know that 
\[ \sum_{ k=1 }^{ \infty  } \Big| \frac{ \sin(k/x)  }{  k  }   \Big|  \leq \sum_{ k=1 }^{ \infty  } \frac{ | x/k |   }{ k  } \leq \sum_{ k=1 }^{ \infty  } \frac{ 1 }{ k^2 }.    \]
Since \( \sum_{ k=1 }^{ \infty  } 1 / k^2  \) converges, we know that 
\[  \sum_{ k=1 }^{ \infty  } f_n(x)  \] must converge uniformly. Since \( \sin(x)  \) is continuous (trig functions are continuous), we know that \( f(x)   \) must be a continuous function. 

We clam that \( f  \) is also differentiable. To do this we show that 
\[  \sum_{ k=1 }^{ \infty  } f_n(x_0)  \]
converges to some function \( f(x_0) \) and then show that 
\[  \sum_{ k=1 }^{ \infty  } \frac{ \cos(x / k ) }{ k^2 }  \] converges uniformly where \( f'_n(x) = \cos(x / k ) / k^2  \). Using the same process to show that \( \sum_{ k=1  }^{  \infty  } f_n \) converges but only for some \( x_0 \in (a,b) \), we know that 
\[  \sum_{ k=1 }^{ \infty  } f_n(x_0) = f(x_0). \]
Now to show that \( \sum_{ k=1 }^{ \infty  } f_n'(x)  \) converges uniformly to some \( \ell(x)  \) in \( \R  \), we can see that \( | \cos(x/k)  |  \leq 1  \) such that 
\[  \sum_{ k=1 }^{ \infty  } \Big| \frac{ \cos(k/x) }{ k^2  }   \Big| \leq \sum_{ k=1 }^{ \infty  } \frac{ 1 }{ k^2  }. \]
Since the series on the right of the above converges (p-series), we have that 
\( \sum_{ k=1 }^{ \infty  } f_n'(x)  \) must converge uniformly using the Weierstrass M-test. Since \( (f_n)  \) differentiable and \( \sum_{ k=1 }^{ \infty  } f_n'(x)  \) converges uniformly to some \( \alpha (x)  \) on \( \R  \), we know that \( f(x)  \) must be differentiable. 

We can use the same process above to show that \( f   \) is twice-differentiable.
\end{proof}

\subsubsection{Exercise 6.4.9} Let 
\[  h(x) = \sum_{ n=1 }^{ \infty  } \frac{ 1 }{ x^2 +n^2  }. \]
\begin{enumerate}
    \item[(a)] Show that \( h  \) is a continuous function defined on all of \( \R  \).
        \begin{proof}
        Note that \( (h_n)  \) is a sequence of rational functions which are continuous. Our goal is to show that \( \sum_{ n=1 }^{ \infty  } h_n(x)  \) converges uniformly where 
        \[  h_n(x) = \frac{ 1 }{ x^2 + n^2  }\] differentiable for all \(  n \in \N  \).  Let \( x \in \R  \) be arbitrary. Since \( h_n(x)  \) reaches a maximum at \( x = 0  \), we know that \( h_n(x) \leq 1 / n^2  \). Thus,
        \[  \sum_{ n=1 }^{ \infty  } \Big| \frac{ 1 }{ x^2 + n^2 } \Big|  \leq \sum_{ n=1 }^{ \infty  } \frac{ 1 }{ n^2  },   \]
        and the fact that \( \sum_{ n=1 }^{ \infty  } 1 / n^{2} \) is a p-series which converges, we know that \( \sum_{ n=1 }^{ \infty  } h_n(x)   \) must converge uniformly to \( h(x)  \). This means that \( h  \) must be continuous on \( \R  \) by the Term-by-term Continuity Theorem.
        \end{proof}
    \item[(b)] Is \( h  \) differentiable? If so, is the derivative function \( h' \) continuous? 
        \begin{proof}[Solution]
        We claim that \( h  \) is a differentiable function. First we compute \( h_n'(x)  \). Using our differentiation rules, we arrive at
        \[  h_n'(x) = \frac{ -2x  }{ (x^2 + n^2 )^2 }.  \]
        Our goal is to show that 
        \[  \sum_{ n=1 }^{ \infty  } h_n'(x)   \]
        converges uniformly to some \( \ell(x)  \) in \( \R  \). Since \( h_n(x)  \) is also secondly differentiable, we can use the Interior Extremum Theorem to find the points on \( \R  \) such that \( h"(x) = 0   \) such that \( h'_n(x)  \) is at its maximum. Differentiating \( h_n'(x)  \) again, we arrive arrive at 
        \[  h_n"(x) = \frac{ 6x^2 - 2n^2  }{ (x^2 + n^2 )^3 }.  \]
        Setting \( h_n''(x_0) = 0  \) produces the following values where \( h'(x)  \) reaches its extrema:
        \[  x_0 = \pm \frac{ n }{ \sqrt{ 3 }  }. \] Plugging this point into \( h_n'(x)  \) where
        \[  h_n'(x_0) = \frac{ 9 }{ 8\sqrt{ 3 }  n^3  }.  \]
        Hence, we can bound \( h_n'(x)  \) by this value. So we have that 
        \[ \sum_{ n=1 }^{ \infty  } | h_n"(x) |  \leq \sum_{ n=1 }^{ \infty  } \frac{ 9 }{ 8 \sqrt{ 3 } n^3  }.  \]
        Since the series on the right side of the above inequality converges, we know that by the  Weierstrass M-test that 
        \[  \sum_{ n=1 }^{ \infty  } h_n'(x) = \ell(x).  \]
        Furthermore, since \( \sum_{ n = 1 }^{  \infty } h_n(x)  \) converges uniformly for all \( x \in \R  \), we know that \( h(x)  \) must be twice-differentiable.
        
        \end{proof}
\end{enumerate}







\section{Power Series}

We can express functions in the form of power series where it takes the form of 
\[  f(x) = \sum_{ n=0  }^{  \infty  } a_n x^n = a_0 + a_1 x + a_2 x^2 + \dots . \]
We want to be able to find all possible \( x \in \R \) such that the above series converges. 

\begin{tcolorbox}
\begin{thm}
If a power series \( \sum_{ n=0  }^{ \infty  } a_n x^n  \) converges at some point \( x_0 \in \R  \), then it converges absolutely for any \( x  \) satisfying \( | x  |  < | x_0  |  \).
\end{thm}
\end{tcolorbox}

\begin{proof}
Assume the power series \( \sum_{ n=0  }^{ \infty  } a_n x^n  \) converges at some point \( x_0 \in \R  \). Since the sequence of terms \( (a_n x_0^n) \) converges to zero, we know that they must be bounded. Hence, there exists \( M > 0  \) such that \( | a_n x_0^n |  \leq M  \) for all \( n \in \N  \). If \( x \in \R  \) satisfies the property that \( | x  |  < | x_0  |  \), then we have that 
\[  | a_n x^n  |  = | a_n x_0^n  | \Big| \frac{ x  }{  x_0  }  \Big|^n \leq M \Big| \frac{ x  }{  x_0  }  \Big|^n.\] This tells us that the series 
\[  \sum_{ n=0  }^{  \infty   } M \Big| \frac{ x  }{  x_0  }  \Big|^n  \] is a geometric series with \( | x / x_0  |  <  1  \) which converges. Hence, we can use the comparison test, to state that the original series \( \sum_{ n = 0  }^{  \infty  } a_n x^n  \) converges absolutely.
\end{proof}

This theorem tells us a few things:
\begin{enumerate}
    \item[(a)] The set of points for which a given power series converges must be either \( \{ 0  \}  \), \( \R  \), or some bounded interval that is centered at \( x = 0  \). 
    \item[(b)] The strict inequality in the condition tells us that the intervals may come in the following forms; either, \( (-R , R ) \), \( [-R, R ] \), \( (-R, R ] \), or \( [-R , R ] \).
    \item[(c)] We denote the value \( R  \) in the intervals above as the \textit{radius of convergence} of a power series which can be either \( 0  \) or \( \infty  \) to represent \( \{ 0  \}  \) or \( \R   \) respectively.
\end{enumerate} 

Below are questions to be answered about the properties of power series:
\begin{enumerate}
    \item[(a)] Continuity
    \item[(b)] Differentiability
    \item[(c)] Term-by-term differentiability
    \item[(d)] Behavior of endpoints.
\end{enumerate}
\subsection{Establishing Uniform Convergence}

\begin{tcolorbox}
\begin{thm}
    If a power series \( \sum_{ n= 0 }^{  \infty   } a_n x^n \) converges absolutely at a point \( x_0  \), then it converges uniformly on the closed interval \( [-c , c ] \), where \( c  = | x_0  |  \).
\end{thm}
\end{tcolorbox}

\begin{proof}
    Suppose a power series \( \sum_{ n=0  }^{  \infty   } a_n x^n  \) converges absolutely at a point \( x_0  \). Then the series \(  \sum_{ n=0  }^{  \infty  } | a_n x_0^n |  \) converges. Let \( x \in [-c , c ] \) where \( c = | x_0  |  \). We proceed via the Weierstrass M-test to show that \( \sum_{ n=0  }^{ \infty  } a_n x^n  \) converges uniformly. We observe that 
    \[  | a_n x^n  |  \leq a_n c^n = a_n | x_0  |^n = a_n | x_0^n | .  \]
    This tells us that 
    \[  \sum_{ n=0 }^{  \infty   } | a_n x^n | \leq \sum_{ n=0  }^{ \infty  } a_n | x_0^n  |.   \]
    Since the right side of the above inequality converges, we know that \( \sum_{ n=0  }^{ \infty  } a_n x^n  \) must converge uniformly on any \( x \in [-c ,c ]  \).
\end{proof}
A few remarks about this result:
\begin{enumerate}
    \item[(a)] Any \( x \in (-R, R ) \) is contained in the interior of a closed interval \( [-c , c ]  \subseteq (-R ,R )\).
    \item[(b)] If the interval above was open instead of closed, then the limit of the series above is necessarily continuous on this interval. 
\end{enumerate}

Some questions we can ask about this result are:
\begin{enumerate}
    \item[(a)] Can a power series converge at an endpoint of the interval of convergence? 
    \item[(b)] Does the behavior of the power series on an open interval necessarily imply that it will be convergent at \( x = R  \)? 
    \item[(c)] What happens when we \textit{conditionally} convergent power series?
\end{enumerate}

\subsection{Abel's Theorem}

\begin{tcolorbox}
    \begin{lem}[Abel's Lemma]
    Let \( b_n  \) satisfy \( b_1 \geq b_2 \geq b_3 \geq \dots \geq 0,  \) and  let \( \sum_{ n=1 }^{ \infty  } a_n  \) be a series for which the partial sums are bounded. In other words, assume there exists \( A > 0  \) such that 
    \[  | a_1 + a_2 + \dots + a_n  |  \leq A  \] for all \( n \in \N  \). Then, for all \( n \in \N  \), 
    \[  \Big| \sum_{ n=1 }^{ \infty  } a_n b_n  \Big|  \leq Ab_1. \]
    \end{lem}
\end{tcolorbox}

\begin{proof}
Let \( s_n = \sum_{ k=1 }^{ n  } a_k  \) be a bounded sequence of partial sums for the series 
\[  \sum_{ n=1  }^{  \infty  } a_n.  \] Hence, there exists some \( A > 0  \) such that \( | x_n  | \leq A  \). Using the Summation-by-parts formula, we have that 
\begin{align*}
    \Big| \sum_{ k=1 }^{ n } a_k b_k  \Big| &= \Big| s_n y_{n+1} + \sum_{ k=1 }^{ n } s_k (y_k - y_{k+1}) \Big|  \\
                                            &\leq | s_n | | y_{n+1} | + \Big| \sum_{ k=1 }^{ n } s_k (y_k - y_{k+1}) \Big| \\
                                            &\leq Ay_1 + \sum_{ k=1 }^{ n } A (y_k - y_{k+1}) \\
                                            &= Ay_1 + A(y_1 - y_{n+1}) \\
                                            &\leq Ay_1.
\end{align*}
\end{proof}

We can use this bound to prove the next theorem about proving convergence at one of the endpoints of an interval.

\begin{tcolorbox}
    \begin{thm}[Abel's Theorem]
        Let \( g(x) = \sum_{ n=0  }^{  \infty  } a_n x^n  \) be a power series that converges at the point \( x = R > 0  \). Then the series converges uniformly on the interval \( [0,R] \). A similar result holds if the series converges at \( x = -R  \).
    \end{thm}
\end{tcolorbox}

\begin{proof}
Let us rewrite \( g(x)  \) into the following form:
\[  g(x) = \sum_{ n=0  }^{  \infty  } a_n x^n = \sum_{ n=0  }^{ \infty  } (a_n R^n ) \Big( \frac{ x  }{ R  }  \Big)^n. \]
Let \( \epsilon > 0  \). We can show that the series above converges uniformly by showing that it satisfies the Cauchy Criterion. Since \( \sum_{ n=0  }^{ \infty  } a_n R^n  \) converges where \( R > 0  \), we know that there exists an \( N \in \N  \) such that for any \( n > m \geq N \), we have that
\[  \Big| \sum_{ k = m+1    }^{ n } a_n R^n  \Big| < \frac{ \epsilon  }{ 2  }. \]
Using Abel's lemma, we can fix any \( m \in \N  \) and use \( \epsilon / 2  \) as an upper bound. Furthermore, we utilize the fact that \(  (x / R )^{m+j}  \) is a monotone decreasing sequence of functions. Hence, we can write 
\begin{align*}
    \Big| \sum_{ k=m+1  }^{ n  } (a_k R^k ) \Big( \frac{ x  }{ R  }  \Big)^k \Big| &\leq \frac{ \epsilon  }{ 2  }  \Big( \frac{ x  }{ R  }  \Big)^{m+1} < \epsilon \\ 
\end{align*}
by Abel's Lemma. The same process goes for when \( x = -R  \).
\end{proof}

\subsection{The Success of Power Series}

We can summarize the two theorems above in the following theorem.

\begin{tcolorbox}
\begin{thm}
If a power series converges pointwise on the set \( A \subseteq \R  \), then it converges uniformly on any compact set \( K \subseteq A  \).
\end{thm}
\end{tcolorbox}

\begin{proof}
    A compact set \( K \subseteq A  \) contains both a maximum \( x_1  \) and minimum \( x_0  \). Since \( K \subseteq A  \), we know that \( x_0, x_1 \in A  \). Since \( g(x) = \sum_{ n=0  }^{  \infty  } a_n x^n  \) converges pointwise on \( A  \), we know that the series converges uniformly on \( [x_0, x_1 ] \) and hence also on \( K  \) by Abel's Theorem.
\end{proof}

We can utilize this fact about a power series converging on a compact set to show that the power series is continuous at every point in the compact set. To show differentiability, however, requires a slightly more complicated set of assumptions. In order to do this, we need to show that 
\[  \sum_{ n=0  }^{ \infty  } a_n x^n  \] is differentiable, and that we can differentiate each term in the infinite series given we know that 
\[  \sum_{ n=1  }^{  \infty  } na_n x^{n-1} \] converges uniformly.

\begin{tcolorbox}
\begin{thm}
    If \( \sum_{ n=0  }^{  \infty  } a_n x^n  \) converges for all \( x \in (-R , R ) \), then the differentiated series \( \sum_{ n=1  }^{  \infty  } na_n x^{n-1} \) converges at each \( x \in (-R , R ) \) as well. Consequently, the convergence is uniform on compact sets contained in \( (-R , R ) \).
\end{thm}
\end{tcolorbox}

\begin{proof}
Exercise 6.5.5.
\end{proof}

A couple things to note about this result.
\begin{enumerate}
    \item[(i)] It is possible to have a power series converge at \( x =R  \), but its differentiated series to diverge at this point. An example of a power series that satisfies this property is the series 
        \[  \sum_{ n=1  }^{  \infty  } \frac{ x^n  }{ n } \]
        at \( x = -1  \). 
    \item[(ii)] If we happen to have the differentiated series converge at the point \( x = R  \), then we can use Abel's Theorem to imply uniform convergence of the differentiated series on any compact set that contains \( R  \).
\end{enumerate}

\begin{tcolorbox}
\begin{thm}
Assume 
\[  f(x) = \sum_{ n=0 }^{ \infty  } a_n x^n  \] converges on an interval \( A \subseteq \R  \). The function \( f  \) is continuous on \( A  \) and differentiable on any open interval \( (-R , R ) \subseteq A \). The derivative is given by 
\[  f'(x) = \sum_{ n=1  }^{  \infty  } n a_n x^{n-1}. \]
Moreover, \( f  \) is infinitely differentiable on \( (-R ,R ) \), and the successive derivatives can be obtained via term-by-term differentiation of the appropriate series.
\end{thm}
\end{tcolorbox}

\begin{proof}
We can use Theorem 6.5.4 to explain why \( f  \) is continuous. By Theorem 6.5.5, we can use the Term-By-Term Differentiability Theorem to verify the formula \( f' \). Despite not having our differentiated power series diverge at the endpoints of our interval, the radius of convergence is not altered. We can use an induction argument to show that the power series can be differentiated an infinite number of times.
\end{proof}

\subsection{Exercises}

\subsubsection{Exercise 6.5.1} Consider the function \( g  \) defined by the power series 
\[  g(x) = x - \frac{ x^2  }{ 2  }  + \frac{ x^3  }{  3  }  - \frac{ x^4  }{ 4  }  + \frac{ x^5  }{  5  }  - \dotsb . \]
\begin{enumerate}
    \item[(a)] Is \( g  \) defined on \( (-1,1)?  \) Is it continuous on this set? Is \( g  \) defined on \( (-1,1]  \)? Is it continuous on this set? What happens on \( [-1,1]  \)? Can the power series for \( g(x) \) possibly converge for any other points \( | x  | > 1  \)? Explain.
        \begin{proof}[Solution]
        
        \end{proof}
    \item[(b)] For what values of \( x \) is \( g'(x)  \) defined? Find a formula for \( g' \).
        \begin{proof}[Solution]
        
        \end{proof}
\end{enumerate}

\subsubsection{Exercise 6.5.3} Use the Weierstrass M-test to prove Theorem 6.5.2.
\begin{proof}
    Suppose a power series \( \sum_{ n=0  }^{  \infty   } a_n x^n  \) converges absolutely at a point \( x_0  \). Then the series \(  \sum_{ n=0  }^{  \infty  } | a_n x_0^n |  \) converges. Let \( x \in [-c , c ] \) where \( c = | x_0  |  \). We proceed via the Weierstrass M-test to show that \( \sum_{ n=0  }^{ \infty  } a_n x^n  \) converges uniformly. We observe that 
    \[  | a_n x^n  |  \leq a_n c^n = a_n | x_0  |^n = a_n | x_0^n | .  \]
    This tells us that 
    \[  \sum_{ n=0 }^{  \infty   } | a_n x^n | \leq \sum_{ n=0  }^{ \infty  } a_n | x_0^n  |.   \]
    Since the right side of the above inequality converges, we know that \( \sum_{ n=0  }^{ \infty  } a_n x^n  \) must converge uniformly on any \( x \in [-c ,c ]  \).
\end{proof}



\subsubsection{Exercise 6.5.3 (Term-by-term Antidifferentiation).} Assume \( f(x) = \sum_{ n=0  }^{  \infty  } a_n x^n  \) converges on \( (-R,R ) \). 
\begin{enumerate}
    \item[(a)] Show 
        \[  F(x) = \sum_{ n=0  }^{ \infty  } \frac{ a_n  }{  n + 1 } x^{n+1} \]
        is defined on \( (-R,R ) \), find a power series representation for \( g  \).
        \begin{proof}
            First we show that \( F(x)  \) converges uniformly on \( (-R,R ) \). Let \( \epsilon > 0   \). Choose \( N \in \N  \) such that for any \(  n > m \geq N  \), we have that 
            \begin{align*}
                \Big| \sum_{ k = m +1  }^{  n } \frac{ a_k  }{  k +1  } x^{k+1} \Big| &= \Big| \sum_{ k= m+1  }^{ n } (a_k x^k) \Big( \frac{ x }{ k+1  }  \Big) \Big|  \\
                                                                                      &\leq \frac{ \epsilon (m+2) }{  R  } \Big( \frac{ x }{ m+2  }  \Big) \\
                                                                                      &< \epsilon \tag{\( x < R  \)}.
            \end{align*}
            Since the power series \( \sum_{ n=0  }^{ \infty  } \frac{ a_n  }{ n + 1  }  x^{n+1}  \) converges on \( (-R,R ) \), we know that \( F(x)  \) must be continuous on \( (-R ,R ) \) and differentiable by Theorem 6.5.6. Hence, we can differentiate according to the formula in Theorem 6.5.6 to write that \[  F'(x) = \sum_{ n=0  }^{ \infty  } a_n x^n = f(x). \]
        \end{proof}
    \item[(b)] Antiderivatives are not unique. If \( g  \) is an arbitrary function satisfying \( g'(x) = f(x)  \) on \( (-R,R ) \), find a power series representation for \( g  \).
        \begin{proof}[Solution]
        Suppose \( g  \) is an arbitrary function satisfying \( g'(x) = f(x)  \) on \( (-R ,R ) \); that is, 
        \[  g'(x) = \sum_{ n=0  }^{ \infty  } a_n x^n. \]
        Since \( f(x) = F'(x)  \) and the fact that Antiderivatives are not unique, we can write 
        \[  g(x) = c +  \sum_{ n=0  }^{ \infty  } \frac{ a_n  }{  n+1  } x^{n+1}\]
        where \( c  \) is some constant.
        \end{proof}
\end{enumerate} 



\section{Taylor Series}

In this section, our goal is to develop some theory for infinitely differentiable functions such as 
    \[  \sin(x) = a_0 + a_{0} x  + a_{2} x^{2} + a_{3} x^{3} + a_{4} x^{4} + \dotsb \]
so that we can find suitable coefficients \( a_{n} \) given some nonzero values of \( x  \).

\subsection{Manipulating Series}
In section 6.1, we encountered 
\[  \frac{ 1 }{ 1-x  } = 1 + x + x^{2} + x^{3} + x^{4} + \dotsb, \text{ for all } | x  | < 1. \tag{1} \]
We can apply Theorem 6.5.7 to arrive at the following series representation
\[  \frac{ 1 }{ (1-x)^2  } = 1 + 2x + 3 x^{2} + 4 x^{3} + 5 x^{4} + \dotsb, \text{ for all } | x  | < 1.   \]

We can use term-by-term antidifferentiation (proven in Exercise 6.5.4) to arrive at the original function. An example of this is 
\[  \frac{ 1 }{ 1 + x^2  }  = 1 - x^{2} + x^{4 } - x^{6 } + x^{8 } - \dotsb, \text{ for all } | x  | < 1. \]
Antidifferentiating each term of the power series above, we arrive at 
\[  \arctan(x) = x - \frac{ 1 }{ 3 }  x^{3} + \frac{ 1 }{ 5 } x^{5} - \frac{ 1 }{ 7 } x^{7} + \dotsb,  \]
for all \( x \in (-1,1) \). Note that the power series representation above is also valid when \( x = \pm 1  \). The same methods can be used to find the series representations for functions such as \( \ln ( 1 + x ) \) and \( x  / (1 + x^2 )^2 \).

\subsection{Taylor's Formula for the Coefficients} 

Given an infinitely differentiable function \( f  \) defined on some interval centered at zero, if we assume that a function \( f  \) has a power series expansion, we can be able to find the every coefficient.

\begin{tcolorbox}
    \begin{thm}[Taylor's Formula]
    Let 
    \[  f(x) = a_0 + a_{1}x + a_{2} x^{2} + a_{3} x^{3} + \dotsb \]
    be defined on some nontrivial interval centered at zero. Then, 
    \[ a_n = \frac{ f^{(n)}(0) }{ n! }. \]
    \end{thm}
\end{tcolorbox}

\begin{proof}
Exercise 6.6.3
\end{proof}

We can use our new formula to derive the \textit{Taylor Series} for \( \sin(x)  \). To get \( a_{0} \), all we have to do is let \( x = 0  \) into the formula above so that 
\[  a_{0} = \sin(0) = 0. \] Then for \( n =1  \), we get 
\[  a_{1} = \frac{ f^{(1)}(0) }{ 1! } = \cos(0) = 1.  \]
Then likewise we have \( a_{1} = \cos(0) = 1  \), \( a_{2} = - \sin(0) / 2! = 0  \), and then \( a_{3} = - \cos(0) / 3! = -1/ 3! \) and so on. Hence, we are left with the following series
\[  x - \frac{ x^{3} }{ 3! }  + \frac{ x^{5} }{ 5! }  - \frac{ x^{7} }{ 7! }  + \dotsb .  \]
Note that this is the power series representation of \( \sin(x)  \). Generally, if a function \( f(x)  \) can be expressed as a power series
\[  f(x) = \sum_{ n=0 }^{ \infty  } a_{n} x^{n} \] then we are guaranteed to have 
\[  a_n = \frac{ f^{(n)}(0) }{ n! } \]
if the power series is centered at \( x = 0  \). But is the converse true? 

A few questions: 
     If we have 
        \[  a_n = \frac{ f^{(n)}(0) }{ n! }  \] for all \( n \geq 0  \) does the series
        \[  \sum_{ n=0 }^{ \infty  } a_n x^{n} \] converge to \( f(x)  \) on some nontrivial set of points?  Does it even converge at all? The limit different from \( f(x)  \)? Our question as of now is whether or not the following sequence of partial sums
        \[  S_{N}(x) = a_{0} + a_{1} x  + a_{2} x^{2} + \dotsb + a_{N} x^{N }  \] for the Taylor series expansion of \( f(x)  \) actually converges to \( f(x)  \); that is, 
        \[ \lim_{ N \to \infty  }  S_{N}(x) = f(x) \]
        for some values of \( x  \) besides zero.

    \subsection{Lagrange's Remainder Theorem}

The idea of the Remainder theorem is to express the error between the function \( f  \) and partial sum \( S_N  \) in terms 
\[  E_N (x) = f(x) - S_N(x).\]

\begin{tcolorbox}
    \begin{thm}[Lagrange's Remainder Theorem]
    Let \( f  \) be differentiable \( N + 1  \) times on \( (-R ,R ) \), define \( a_n = f^{n}(0) / n!  \) for \( n = 0, 1 , \dots , N  \), and let 
    \[  S_N(x) = a_0 + a_{1} x + a_{2}x^{2} + \dotsb + a_{N}x^{N}. \]
    Given \( x \neq 0  \) in \( (-R ,R ) \), there exists a point \( c  \) satisfying \( | c  |  < | x  |  \) where the error function \( E_N(x) = f(x) - S_N(x)  \) satisfies 
    \[  E_N(x) = \frac{ f^{(N+1)}(c) }{ (N+1)!  } x^{N+1}. \]
    \end{thm}
\end{tcolorbox}

Components that make up the theorem:
\begin{enumerate}
    \item[(i)] Showing that \( S_N(x) \to  f(x)  \) is equivalent to showing that \( E_N(x) \to 0  \).
    \item[(ii)] The factorial \( (N+1)! \) on the denominator helps to make the error small as \( N  \) tends to infinity. 
    \item[(iii)] The \( x^{N+1}  \) term on the numerator has the potential to grow depending on how far \(  x  \) is chosen from the origin.
    \item[(iv)] The second term on the numerator \( f^{(N+1)} (c)  \) can be handled by introducing some upper bound either from a compact set or based on the behavior of \( f  \).
\end{enumerate}

\begin{ex}
Consider the Taylor series for \( \sin (x)  \) from earlier. We can ask how well does our sequence of partial sums approximate \( \sin(x)  \) when \( N=5  \); that is, how well does  
\[  S_{5}(x) = x - \frac{ 1 }{ 3! } x^{3} + \frac{ 1 }{ 5! } x^{5} \]
approximate \( \sin(x)  \) on the interval \( [-2, 2 ] \). By using Lagrange's Remainder Theorem we can assert that 
\[  E_5(x) = \sin(x) - S_5(x) = \frac{ -\sin(c)  }{ 6! } x^{6} \]
for some \( c \in (-| x |, | x | ) \). Since we don't know the value of \( c  \), we can still use the fact that \( | \sin(c)  | \leq 1 \) to assert 
\[  | E_5(x) | = \Big| \frac{ - \sin(c)  }{ 6! } x^{6} \Big| \leq \frac{ 2^{6} }{ 6! } \tag{\( x \in [-2,2] \)}.  \]
We can show that \( S_N(x)  \) converges uniformly to \( \sin(x)  \) on \( [-2,2] \) by observing that \( | f^{(N+1)}(c) | \leq 1  \). Thus, 
\[  | E_N(x)  | = \Big| \frac{ f^{(N+1)}(c) }{ (N+1)! } x^{N+1} \Big| \leq \frac{ 1 }{ (N+1)! }  2^{N+1} \] for \( x \in [-2,2] \). We know that factorials grow faster than exponentials. Hence, we know that \( E_N(x) \to 0  \) on \( [-2,2] \).
\end{ex}

\begin{proof}[Proof of Lagrange's Remainder Theorem]
    Note that the Taylor coefficients are chosen so that the function \( f  \) and the polynomial \( S_N  \) have the same derivatives at \( 0  \), at least up to through the \( N \)th derivative, after which \( S_N  \) becomes the zero function. That is, we have \( f^{(n)}(0) = S_N^{(n)}(0) \) for all \( 0 \leq n \leq N  \), which implies the error function \( E_{N}(x) = f(x) - S_{N}(x)  \) satisfies 
    \[  E_N^{(n)}(0) = 0  \] for all \( n = 0,1,2, \dots, N .  \)
    Our goal is to use the Generalized Mean Value Theorem from Chapter 5. To simplify our notation, let us assume \( x > 0  \) and apply the theorem to the error function \( E_{N}(x)  \) and the polynomial \( x^{N+1} \) on the interval \( [0,x] \). Thus, there exists \( x_1 \in (0,x ) \) such that 
    \[  \frac{ E_{N}(x) }{ x^{N+1}  } = \frac{ E'_N(x_1) }{ (N+1) x_{1}^N  }. \]
    Now apply the Generalized Mean Value Theorem again to the functions \( E'_{N }(x)  \) and \( (N+1) x^N  \) on the interval \( [0,x_{1}] \) to get that there exists a point \( x_2 \in (0,x_1) \) where
    \[  \frac{ E_{N}(x) }{ x^{N+1}  } = \frac{ E'_N(x_1) }{ (N+1) x_{1}^N  } = \frac{ E_N"(x_2) }{ (N+1)Nx_2^{N-1} } . \]
    Continuing in this manner we find 
    \[  \frac{ E_{N}(x)  }{ x^{N+1} }  =  \frac{ E_{N}^{N+1} (x_{N+1}) }{ (N+1)! x_{N+1}^{N-N} } =  \frac{ E_{N}^{N+1} (x_{N+1}) }{ (N+1)! }  \] 
    where \( x_{N+1} \in (0, x_{N}) \subseteq \dots \subseteq (0,x) \). Now set \( c = x_{N+1} \). Since \( S_{N}^{N+1}(x) = 0  \), we know that \( E_{N}^{N+1}(x) = f^{(N+1)}(x) \) and it follows that 
    \[ E_{N}(x) = \frac{ f^{(N+1)}(c) }{ (N+1)! } x^{N+1}  \] as desired.
\end{proof} 

\subsection{Taylor Series Centered at \( a \neq 0  \).}

The series expansion of a function need not be centered only at \( a = 0  \). If the function \( f \) is defined at any other neighborhood of \( a \neq 0  \) and infinitely differentiable at \( a  \), then the Taylor series expansion takes the following form where 
\[  \sum_{ n=0 }^{ \infty  } c_{n} (x-a)^{n} \text{ where } c_{n} = \frac{ f^{(n)}(a) }{ n! }. \]
Setting up our Error function \( E_{N}(x) = f(x) - S_{N}(x)  \) as before, we can reformulate Lagrange's Remainder Theorem in the following fashion where there exists some value \( c \in (a, x) \) such that 
\[  E_N(x) = \frac{ f^{(N+1)}(c) }{ (N+1)! } (x-a)^{N+1}. \]

\subsection{A Counterexample}

The Lagrange's Remainder Theorem is useful in determining how well behaved the sums of the Taylor series \( S_N(x) \) approximate \( f(x) \), but it leaves the to question whether or not the sequence of partial sums actually converges to \( f(x) \). Let 
\[  g(x) = 
\begin{cases}
    e^{-1/x^2} &\text{ for } x \neq 0  \\
    0 &\text{ for } x = 0. 
\end{cases} \]
We can compute the Taylor coefficients for this function. It can be found that \( a_{0 }= g(0) = 0  \). For take the derivative of \( g(x)  \) at \( x = 0  \)  and get 
\[  a_1 = g'(0) = \lim_{ x \to 0 }  \frac{ g(x) - g(0)  }{ x - 0  } = \lim_{ x \to 0 }  \frac{ e^{-1/x^2} }{ x  } = \lim_{ x \to 0 }  \frac{ 1 /x  }{ e^{1/x^2} }   \] where taking the limit produces an \( \infty / \infty \) situation that calls for L'Hopital's rule to be applied as \( x \to 0  \). Hence, we have that 
\[  a_{1} = \lim_{ x \to 0 }  \frac{ -1/x^2  }{  e^{1/x^2} } (-2/x^3) = \lim_{ x \to 0  } \frac{ x }{ 2 e^{1/x^2} } = 0.\]
Since \( a_1 = 0  \), we know that \( g  \) must be flat at the origin. Furthermore, we find that \( g^{(n)}(0) = 0  \) for all \( n \in \N  \).
However, there is a caveat to this conclusion. We have a function that is infinitely differentiable whose Taylor series expansion converges uniformly to \( 0  \) function but \( g(x) \neq 0  \) everywhere except for \( x = 0  \). This means our convergence does not lead to \( g(x)  \) but leads to something else all together. This unfortunately tells us that not all infinitely differentiable function can be represented in terms of its Taylor series.

\subsection{Exercises}

\subsubsection{Exercise 6.6.1} The derivation in Example 6.6.1 shows the Taylor series for \( \arctan(x)  \) is valid for all \(  x \in (-1,1) \). Notice, however, that the series also converges when \( x =1  \). Assuming that \( \arctan(x)  \) is continuous, explain why the value of the series at \( x = 1  \) must necessarily be \( \arctan(1)  \). What interesting identity do we get in this case?
\begin{proof}[Solution]
We know the series equals the value of \( \arctan(1) \) at \( x = 1  \) because we know that the term-by-term antidifferentiation of the series
\[  \frac{ 1 }{ 1 + x^2  } = 1 - x^{2} + x^{4} - x^{6} + x^{8} - \dotsb  \tag{1}\]
that converges for all \( x \in (-1,1)  \) produces the series 
\[  \arctan(x) = x - \frac{ x^{3} }{ 3  } + \frac{ x^{5} }{ 5 } - \frac{ x^{7} }{ 7 } + \dotsb .  \tag{2}\]
that also converges for \( x \in (-1 ,1 ) \). Since we have convergence of (2) for all \( x \in (-1,1) \) and the fact that each term in the series representation of (2) is continuous shows that at \( x = 1  \), we must have equality of the series representation to \( \arctan(1)  \). This leads us to the interesting identity that 
\[  \arctan(1) = \frac{ \pi  }{ 4 }  = 1 - \frac{ 1 }{ 3 } + \frac{ 1 }{ 5 }  - \frac{ 1 }{ 7 }  + \frac{ 1 }{ 9 } - \dotsb . \]
\end{proof}

\subsubsection{Exercise 6.6.2} Starting from one of the previously generated series in this section, use manipulations similar to those in Example 6.6.1 to find Taylor series representations for each of the following functions. For precisely what values of \( x \) is each series representation valid? 
\begin{enumerate}
    \item[(a)] \( x \cos(x^2)  \)
        \begin{proof}[Solution]
        For this problem, we will use the \( \frac{ d }{ dx }  \) notation to compute our derivatives. We know that 
        \[  \sin (x) = \sum_{ n=0 }^{ \infty  } \frac{ (-1)^n x^{2n+1} }{ (2n+1)! }.\]
        Since \( \sin(x)  \) is an infinitely differentiable function, we can take the derivative of its Taylor expansion. Hence, we have 
        \begin{align*}
            \frac{d  }{d x } [ \sin(x) ] &= \frac{d  }{d x } \Big[ \sum_{ n=0 }^{ \infty  } \frac{ (-1)^{n}  }{ (2n+1)! } x^{2n+1} \Big] \\
                                         &= \sum_{ n=0  }^{ \infty  } \frac{d  }{d x }  \Big[  \frac{ (-1)^{n}  }{ (2n+1)! } x^{2n+1}\Big] \\
                                         &= \sum_{ n=0  }^{ \infty  } \frac{ (-1)^n }{ (2n)! }  x^{2n}.
        \end{align*}
        This means that 
        \[  \cos(x) = \sum_{ n=0  }^{ \infty  } \frac{ (-1)^n  }{ (2n)! } x^{2n}. \] Substituting \( x = x^2  \) and multiplying by \( x  \) gives us 
        \[  x \cos(x^2) = \sum_{ n=0  }^{ \infty  } \frac{ (-1)^{n} }{ (2n)! } x^{4n+1} \]
        which holds for any \( x \in (-R ,R ) \) where \( R > 0  \).
        \end{proof}
    \item[(b)] \( x / (1+ 4x^2)^2 \)
        \begin{proof}[Solution]
        Take the function \( f(x) = \frac{ 1 }{ 1 - x  }  \) which has the following Taylor Expansion 
        \[  \frac{ 1 }{ 1 - x  } = \sum_{ n=0  }^{ \infty  } x^n  \] which is defined for all \( | x  |  < 1  \). Taking the derivative of \( f(x)  \) we get 
        \[  \frac{\text{d}  }{\text{d} x }  \Big[ \frac{ 1 }{ 1 - x  }    \Big] =  \frac{ 1   }{ (1 - x )^2  }   \]
        which has a Taylor series expansion of 
        \[  \frac{ 1 }{ (1-x)^2 } = \sum_{ n=1 }^{ \infty  } n x^{n-1} .  \]
        Letting \( x = -4x^2  \) and multiplying by \( x  \), we arrive at the following Taylor series expansion 
        \[  \frac{ x  }{ (1 + 4x^2 )^2  } = \sum_{ n=1  }^{ \infty  } (-4)^{n-1} n  x^{2n - 1 }  \]
        which holds for all \( x \in (-R ,R ) \).
        \end{proof}
    \item[(c)] \( \text{log}(1+x^2) \)
        \begin{proof}[Solution]
            Our goal is to use the Term-by-term Anti-differentiation to write a Taylor series expansion for  \(  \text{log}(1+x^2) \). Set \( F(x) = \text{log}(1-x) \) and \( f(x) = \frac{ 1 }{ 1 - x  }  \). Since the series expansion for \( f(x)  \); that is, 
            \[  f(x) =\sum_{ n = 0  }^{ \infty  } x^n \] is defined for any \( | x  | < 1  \), we have that 
            \[  F(x) = \sum_{ n=0 }^{ \infty  } \frac{ 1 }{ n+1  } x^{n+1} \]
            which satisfies  \( F'(x) = f(x)  \). Letting \( x = - x^2  \) we get that 
            \[  \text{log}(1 + x^2 ) =  \sum_{ n=0  }^{ \infty  } \frac{ (-1)^{n+1}  }{ n+1  } x^{2n+2}.\]
        \end{proof}
\end{enumerate}






\subsubsection{Exercise 6.6.3} Derive the formula for the Taylor Coefficients given in Theorem 6.6.2.
\begin{proof}
    Suppose \( f(x) = \sum_{ n=0 }^{ \infty  } a_n x^n  \) converges for all \( x \in (-R ,R ) \) Since \( f  \) is infinitely differentiable by Theorem 6.5.7, we can take derivatives of \( f  \) where \( f^{n}(0)= n! a_n    \) implies that 
    \[  a_n = \frac{ f^{(n)}(0) }{ n! }. \]
    \[   \]
\end{proof}

\subsubsection{Exercise 6.6.4} Explain how Lagrange's Remainder Theorem can be modified to prove 
\[  1 - \frac{ 1 }{ 2 } + \frac{ 1 }{ 3 } - \frac{ 1 }{ 4 }  + \frac{ 1 }{ 5 } - \frac{ 1 }{ 6 } + \dotsb = \text{log}(2). \]
\begin{proof}[Solution]

\end{proof}

\subsubsection{Exercise 6.6.5} 
\begin{enumerate}
    \item[(a)] Generate the Taylor coefficients for the exponential function \( f(x) = e^{x} \), and then prove that the corresponding Taylor series converges uniformly to \( e^{x} \) on any interval of the form \( [-R,R] \).
        \begin{proof}
        To generate the Taylor coefficients for \( f(x) = e^{x} \) we can just use the formula given to us via Theorem 6.6.2 and the fact that \( f^{(n)}(0) = f^{(n+1)}(0) \) for all \( n \geq 0  \) where \( f^{(n)}(0) = e^{0}= 1  \), to write 
        \[  a_n = \frac{ f(0)  }{ n! } = \frac{ 1 }{ n! }. \]
        Hence, we can define the following power series 
        \[  S_N(x) = \sum_{ N=0  }^{ \infty  } \frac{ x^N  }{ N! }. \]
        To show that \( S_n(x) \to f(x)  \) where \( f(x) = e^{x} \), we will use Lagrange's Remainder theorem. Given an \(  x \in (-R ,R ) \) non-zero, suppose there exists a point \( c  \) satisfying \( | c  | < | x  |  \) such that \( E_N(x) = e^{x} - S_{N}(x)   \) satisfies  
        \[  E_{N}(x) =  \frac{ f^{(N+1)}(c) x^{N+1} }{ (N+1)! }  =  \frac{ e^c x^{N+1} }{ (N+1)! } .   \]
        Since \( x \in (-R ,R ) \), we can produce the following bound 
        \[  | E_N(x)  | = \Big| \frac{ e^c x^{N+1}  }{ (N+1)!  }  \Big| \leq \frac{ e^{c} R^{N+1} }{ (N+1)! }.   \]
        Since the term on the right side converges to zero uniformly on \( (-R ,R ) \), we know that \( E_N(x) \to 0  \) which means that \(  S_N(x) \to e^{x} \) uniformly on \( (-R ,R ) \).
        \end{proof}
    \item[(b)] Verify the formula \( f'(x) = e^{x} \).
        \begin{proof}[Solution]
            If we take the Taylor Expansion of \( e^{x} \) which is defined for all \( x \in [-R ,R ] \), we differentiate 
            \[  e^{x} = \sum_{ n=0  }^{ \infty  } \frac{ x^{n} }{ n! }  \] via Theorem 6.5.7 to get the following series representation 
            \[  f'(x) = \sum_{ n=1 }^{ \infty  } \frac{ n x^{n-1} }{ n! } =   \sum_{ n=1 }^{ \infty  } \frac{ x^{n-1} }{ (n-1)! }. \]
            We can reorder our indices to get 
            \[  f'(x) = \sum_{ n=0  }^{ \infty  } \frac{ x^{n} }{ n! } = e^{x}. \]
        \end{proof}
    \item[(c)] Use a substitution to generate the series for \( e^{-x} \), and then informally calculate \( e^{x} \cdot e^{-x} \) by multiplying together the two series and collecting common powers of \( x \).
        \begin{proof}[Solution]
        To generate the series for \( e^{-x} \), let \( x = -x  \). Then 
        \[  f(-x) = e^{-x} = \sum_{ n=0  }^{ \infty  } \frac{ (-1)^{n}  }{ n! } x^{n} \]
        To attain the Taylor expansion of \( e^{x} \cdot e^{-x } \) we can multiply the two series together. Since we are just collecting powers of \( x  \), we can try and form a formula for the summation of the coefficients of the Taylor expansions of \( e^{x} \) and \( e^{-x } \). By using the formula from section 2.7 where 
        \[  \sum_{ i=0 }^{ n } d_k \]
        with 
        \[  d_n = \sum_{ i=0 }^{ n } a_{i} b_{i -n}. \]
        Then we have 
        \begin{align*}
            e^{x} \cdot e^{-x} &=  \Big( \sum_{ n=0 }^{ \infty  } \frac{ x^{n} }{ n! }  \Big) \Big( \sum_{ m=0 }^{ \infty  } \frac{ (-1)^{m} x^{m} }{ m! }  \Big)  \\
                               &= \sum_{ n=0 }^{ \infty  } \Big( \sum_{ i=0 }^{ n } \frac{ (-1)^{n-i } }{ i! (n-i)! }  \Big) x^n.
        \end{align*}
        \end{proof}
\end{enumerate}

\subsubsection{Exercise 6.6.6} Review the proof that \( g'(0) = 0 \) for the function 
\[  g(x) = 
\begin{cases}
     e^{-1/x^2} &\text{ for } x \neq 0 \\
    0 &\text{ for } x = 0
\end{cases} \]
introduced at the end of this section.
\begin{enumerate}
    \item[(a)] Compute \( g'(x)  \) for \( x \neq 0  \). Then use the definition of the derivative to find \( g"(0) \).
        \begin{proof}[Solution]
        Since \( x \neq 0  \) we can use the Chain Rule to get 
        \[  g'(x) = \frac{ 2 }{ x^3  } e^{-1/x^2 } \]
        \end{proof}
    \item[(b)] Compute \( g"(x)  \) and \( g^{(3)}(x)  \) for \( x \neq 0  \). Use these observations and invent whatever notation is needed to give a general description for the \( n \)th derivative \( g^{(n)}(x)  \) at points different from zero.
        \begin{proof}[Solution]
        
        \end{proof}
    \item[(c)] Construct a general statement argument for why \( g^{(n)}(0) = 0  \) for all \( n \in \N  \).
        \begin{proof}[Solution]
        
        \end{proof}
\end{enumerate}







\section{The Weierstrass Approximation Theorem}

\begin{tcolorbox}
    \begin{thm}[Weierstrass Approximation Theorem]
        Let \( f: [a,b] \to \R  \) be continuous. Given \( \epsilon > 0  \), there exists a polynomial \( p(x) \) satisfying 
        \[  | f(x) - p(x) | < \epsilon \] for all \( x \in [a,b] \).
    \end{thm}
\end{tcolorbox}

This means that every continuous function over a closed interval can be uniformly approximated by a polynomial.

\subsubsection{Exercise 6.7.1} Assuming WAT, show that if \( f \) is continuous on \( [a,b] \), then there exists a sequence \( (p_n) \) of polynomials such that \( p_n \to f  \) uniformly on \( [a,b] \).

\begin{proof}
    Using the Weierstrass Approximation Theorem, we can let \( \epsilon  = \frac{ 1 }{ n }  \). By choosing an \( N = 1 / \epsilon   \) such that \( n \geq N  \), we can have a sequence of polynomials \( (p_n) \) such that 
    \[ | p_{n}(x) - f(x)  | < \frac{ 1 }{ n  } \leq \frac{ 1 }{ N } < \epsilon.  \]
\end{proof}

\subsection{Interpolation}

The purpose of Weiertrass's theorem is to approximate polynomials. We can get try to understand this a little more by looking at the collection of continuous, piecewise-linear functions instead of polynomials.

\begin{tcolorbox}
\begin{defn}
    A continuous function \( \phi : [a,b] \to \R  \) is \textit{polygonal} if there is a partition 
    \[  a = x_{0} < x_{1} < \dotsb < x_{n} = b   \] of \( [a,b] \) such that \( \phi  \) is linear on each subinterval \( [x_{i-1}, x_{i}] \) where \(  i = 1, \dots n. \)
\end{defn}
\end{tcolorbox}

The goal of interpolation is to find a function whose graph passes through a given set of points. We can do this by using line segments.

\begin{tcolorbox}
\begin{thm}
    Let \( f: [a,b] \to \R  \) be continuous. Given \( \epsilon > 0   \), there exists a polygonal function \( \phi  \) satisfying 
    \[  | f(x) - \phi(x) | < \epsilon \] for all \( x \in [a,b] \).
\end{thm}
\end{tcolorbox}

\subsubsection{Exercise 6.7.2} Prove Theorem 6.7.3.
\begin{proof}
    We can partition the closed interval \( [a,b]  \) into 
    \[  a = x_{0} < x_{1} < \dotsb < x_{n} = b  \] where each subinterval is defined as \( [x_{i-1}, x_{i}] \) where \( i \in \N  \). Since \( f  \) is continuous over \( [a,b]  \) which is a compact set, we know that \( f  \) must be uniformly continuous on \( [a,b] \). Hence, \( f  \) takes on a maximum and a minimum value on \( [a,b] \). We can do this on each subinterval of \( [a,b]  \) where 
    \[  f(x_{i-1}) \leq f(x) \leq f(x_{i}) \iff f(x) - f(x_{i-1}) \leq f(x_{i}) - f(x_{i-1}) .\] We can define \( \phi(x)  \) at the endpoints of \( [a,b]  \) to be linear as a way of interpolating between the endpoints of each subinterval. Then for any \( x \in (a,b)  \), let \( q  \) be the largest segment endpoint that is less  than \( x  \), and \( r  \) be the following segment endpoint. Using the uniform continuity of \( f \) over \( [a,b]  \), we can choose \( \delta > 0  \) such that whenever 
    \( | x - q  | <  \delta \), we have 
    \begin{align*}
        | f(x) - \phi(q)  | &\leq |  \phi(q) - \phi(r) | < \epsilon. \\
    \end{align*}
\end{proof}

This is essentially the same thing as the WAT but with the substitution of polygonal functions being used to approximate a function instead of polynomials.

\subsubsection{Exercise 6.7.3} 
\begin{enumerate}
    \item[(a)] Find the second degree polynomial \( p(x) = q_{0} + q_{1} + q_{2}x^{2} \) that interpolates the three points \( (-1,1) \), \( (0,0) \), and \( (1,1) \) on the graph of \( g(x) = | x  |  \). Sketch \( g(x)  \) and \( p(x)  \) over \( [-1,1] \) on the same set of axes.
        \begin{proof}[Solution]
        Using the points given to us, we can set up a system of linear equations where 
        \begin{align*}
            1 &= q_{0} - q_{1} + q_{2} \\
            0 &= q_{0} \\
            1 &= q_{0} + q_{1} + q_{2}.
        \end{align*}
        Solving this set of equations gives us the coefficients 
        \begin{align*}
            q_{0} &= 0  \\
            q_{1} &= 0 \\
            q_{2} &= 1
        \end{align*}
        which gives us the following interpolating quadratic polynomial 
        \[  p(x) = x^2. \]
        \end{proof}
\end{enumerate}

It turns out that interpolating with polynomials is not a fruitful approach when it comes to approximating functions as it leads to rapid oscillations.

\subsection{Approximating the Absolute Value Function}

We can use Theorem 6.7.3 which asserts that every continuous function can be uniformly approximated by a polygonal function. The goal is to find a polynomial representation of the Absolute Value Function to prove the Weierstrass Approximation Theorem. This is because unlike polynomials, Absolute Value Functions do not produce rapid oscillations. 

\subsection{Cauchy's Remainder Formula for Taylor Series} 

We can show that the function \( g(x) = | x |  \) is the uniform limit of polynomials is via the Taylor series. This is surprising because we know that \( | x  |   \) is not a differentiable function but we can, however find the Taylor series of the infinitely differentiable function \( \sqrt{ 1 - x  }.  \)

\subsubsection{Exercise 6.7.4} Show that \( f(x) = \sqrt{ 1 - x  }  \) has Taylor series coefficients \( a_{n} \) where \( a_{0} = 1  \) and 
\[  a_{n} = \frac{  -1 \cdot 3 \cdot 5 \dotsb (2n-3)  }{ 2 \cdot 4 \cdot \dotsb 2n }  \] for \( n \geq 1  \).
\begin{proof}
Since \( f(x) = \sqrt{ 1 - x  }    \) is infinitely differentiable we can use Taylor's Formula 
\[  a_n = \frac{ f^{(n)}(0)  }{ n!  }  \] to produce the coefficients of the Taylor series of \( f(x)  \). Note that \( a_0 = 1  \) because \( f(0) = \sqrt{ 1 - 0  } = 1   \). Taking the first derivative of \( f \), we find that 
\[  f^{(1)}(x) = \frac{ -(1-x)^{-1/2} }{ 2 } \]
which produces the Taylor coefficient 
\[  a_1 = \frac{ -1 }{ 2 }. \] We can take the second derivative (\( n=2 \)) 
\[  f^{(2)}(x) = \frac{ -1 }{ 2 } (1-x)^{-3/2}  \] which produces the Taylor Coefficient at \( n =2  \) 
\[  a_2 = \frac{ f^{(2)}(0)  }{ 2!  } = \frac{ -1  }{ 4 }. \]
For \( n \geq 1  \), we find that 
\[  f^{(n)}(x) = \frac{ 1 }{ 2^{n} } (1-x)^{-(2n-1)/2 } \prod_{i=1}^{n} 2i-3  \]
where plugging in \( x = 0  \) yields 
\[  f^{(n)}(0) = \frac{ 1 }{ 2^{n} } \prod_{i=1}^{n} 2i -3. \]
Then using Taylor's formula, we have for \( n \geq 1  \) 
\begin{align*}
    a_n &= \frac{ f^{(n)}(0) }{ n! }  \\
        &= \frac{ 1 }{ 2^{n}n! } \prod_{i=1}^{n} (2i-3) \\
        &= \frac{ \prod_{i=1}^{n} (2i-3)}{ \Big( \prod_{i=1}^{n} 2 \Big) \Big( \prod_{i=1}^{n} i \Big)  }   \\
        &=  \prod_{i=1}^{n } \frac{ 2i-3 }{ 2i }  .
\end{align*}
\end{proof} 
Our goal now is to show that the error function of \( f(x) = \sqrt{ 1 - x  }  \) for all \( x \in [-1,1] \) where 
\[ E_{N}(x) = f(x) - \sum_{ n=0  }^{ N  } a_{n} x^{n} \]
goes to \( 0  \) uniformly as \( N \to \infty  \). Normally, we can use Lagrange's Remainder Theorem to show that this is the case. But this is an unfruitful approach since fixing \( x \in (0,1]  \) produces a situation where the max of \( f(x)  \) is largest at \( x = c  \) where \( (x / 1 - x)^{N + 1 /2 } \) grows exponentially to infinity whenever \( x > 1 / 2  \); that is 
\begin{align*}
   E_{N}(x)  &= \frac{ f^{(N+1)}(c)  }{ (N+1)! } x^{N+1}  \\
             &= \frac{ 1 }{ (N+1)! } \Big( \frac{ -1 \cdot 3 \cdot 5 \dotsb (2N-1) }{ 2^{N+1} (1-c)^{(N+1)/2} }  \Big) \\
             &= \Big( \frac{ -1 \cdot 3 \cdot 5 \dotsb (2N-1) }{ 2 \cdot 4 \cdot 6 \dotsb (2N+2) }  \Big) \Big( \frac{ x  }{ 1 -c  }  \Big)^{(N+1)/2} x^{1/2}.
\end{align*}

\subsubsection{Exercise 6.7.5} 
\begin{enumerate}
    \item[(a)] Follow the advice in Exercise 6.6.9 to prove the Cauchy form of the remainder:
        \[ E_{N}(x) = \frac{ f^{(N+1)}(c)  }{ N! } (x-c)^{N}x \]
        for some \( c  \) between \( 0  \) and \( x  \).
        \begin{proof}
        See exercise 6.6.9 for the solution.
        \end{proof}

    \item[(b)] Use this result to prove 
      \[  \sqrt{ 1-x } = \sum_{ n=0 }^{ N } a_{n} x^{n} \] 
        is valid for all \( x \in (-1,1) \).
        \begin{proof}
        Let \( x \in (-1,1) \). Using Cauchy's Remainder Theorem, there exists some \( c   \in (0,x) \) such that 
        \begin{align*}
           E_{N}(x) &= \frac{ f^{(N+1)}(c) }{ N! } (x-c)^{N} x.\\
        \end{align*}
        Our goal is to make our error function \( E_{N}(x) \to 0   \). Note that 
        \begin{align*}
            f^{(N+1)}(c) &= \frac{ (1-c)^{-(2N+1)/2} }{ 2^{N+1} } \prod_{n=1}^{N+1} 2n-3. \\
        \end{align*}
        Then using Cauchy's Remainder Theorem, we have 
        \begin{align*}
            | E_{N}(x) | &= \Big| \frac{ \prod_{n=1}^{N+1} 2n-3 }{ 2^{N+1} N! } (1-c)^{-(2N+1)/2} (x-c)^{N} x \Big|  \\
                         &= \Big| \frac{ \prod_{n=1}^{N+1} 2n-3 }{ 2 \cdot \prod_{n=1}^{N} 2n } (1-c)^{-(2N+1)/2} (x-c)^{N} x  \Big|  \\
                         &<  \Big| \frac{ \prod_{n=1}^{N+1} 2n-3 }{  2 \cdot \prod_{n=1}^{N} 2n } (1-c)^{-(2N+1)/2} (1-c)^{N} \Big| \\
                         &= \frac{ | \prod_{n=1}^{N+1} 2n-3 | }{  2 \cdot \prod_{n=1}^{N} 2n } (1-c)^{-1/2}  \\
                         &< \frac{ (1-c)^{-1/2} }{ \prod_{n=1}^{N} 2n } \to 0. 
        \end{align*}
        Hence, we conclude that \( E_{N}(x) \to 0  \) which implies that 
        \[  \sqrt{ 1-x } = \sum_{ n=0  }^{ N } a_{n} x^{n} \]
        is valid for all \( x \in (-1,1) \).
        \end{proof}
\end{enumerate}

\subsubsection{Exercise 6.7.6} 
\begin{enumerate}
    \item[(a)] Let 
        \[  c_{n} = \frac{ 1 \cdot 3 \cdot 5 \dotsb (2n-1) }{ 2 \cdot 4 \cdot 6 \dotsb 2n }  \] for \( n \geq 1  \). Show \( c_{n} < \frac{ 2 }{ \sqrt{ 2n+1 }  }. \)
        \begin{proof}
        We proceed by induction to show 
        \[  c_{n} < \frac{ 2  }{ \sqrt{ 2n+1 }  } \tag{1}  \] for all \( n \geq 1  \). Let our base case be \( n = 1  \), then 
        \[  c_{1} = \frac{ 1 }{ 2 } < \frac{ 2 }{ \sqrt{ 3 } }. \]
        Now let us assume that (1) holds for \( n \geq 1  \). Let us show that (1) holds for the \( n+1 \) case. Using the definition of \( c_{n} \), observe that
        \begin{align*}
            c_{n+1} &= \frac{ 1 \cdot 3 \cdot 5 \dotsb 2n+1 }{ 2 \cdot 4 \cdot 6 \dotsb 2n+2 }  \\
                    &= \frac{ 1 \cdot 3 \cdot 5 \dotsb (2n-1)(2n+1) }{ 2 \cdot 4 \cdot 6 \dotsb (2n)(2n+2) } \\
                    &= c_{n} \cdot \frac{ 2n+1 }{ 2n+2 } \\
                    &< \frac{ 2 }{ \sqrt{ 2n+1 }  } \cdot \frac{ 2n+1 }{ 2n+2 } \\
                    &= \frac{ 2n+1 }{ n+1 \sqrt{ 2n+1 }  } \\
                    &< \frac{ 2 }{ \sqrt{ 2n^2 + 3n + 1  }  } \\
                    &< \frac{ 2 }{ \sqrt{ 2n+3 }  }.
        \end{align*}
        Hence, we conclude that \( c_n  \) satisfies the inequality
        \[  c_{n} < \frac{ 2 }{ \sqrt{ 2n+1 }  }\]
        for all \( n \geq 1  \).
        \end{proof}
    \item[(b)] Use (a) to show that \( \sum_{ n=0  }^{ \infty  } a_n  \) converges (absolutely, in fact) where \( a_{n} \) is the sequence of Taylor coefficients generated in Exercise 6.7.4.
        \begin{proof}[Solution]
        Our goal is to show that \( \sum_{ n=0 }^{ \infty  } a_n  \) converges absolutely where 
        \[  a_{n}= \prod_{i=1}^{n} \frac{ 2i-3 }{ 2i } \]
        is the sequence of Taylor coefficients. Then observe that
        \begin{align*} a_{n} &= \frac{-1 \cdot 3 \cdot 5 \dotsb (2n-3)   }{ 2 \cdot 4 \cdot 6 \dotsb 2n} \cdot \frac{ (2n-1) }{  (2n-1) } \\ 
        &= \frac{ -1  }{ (2n-1)  }  \cdot \frac{ 1 \cdot 3 \cdot 5 \dotsb (2n-1)  }{ 2 \cdot 4 \cdot 6 \dotsb 2n  } \\     
        &= -\frac{ c_n }{ 2n-1 }.  
        \end{align*}
        Since \( c_n < \frac{ 2 }{ \sqrt{ 2n-1 }  }  \) for \( n \geq 1  \), we can write 
        \[  | a_n |  = \frac{ c_{n} }{ 2n-1 } < \frac{ 2 }{ (2n-1)\sqrt{ 2n+1 }  }    \]
        which creates a series 
        \[  \sum_{ n=1 }^{ \infty  } \frac{ 2 }{ (2n-1) \sqrt{ 2n+1 }  }     \]
        that converges via the root test. Hence, we have \( \sum_{ n=0 }^{ \infty  } a_n  \) converges absolutely.
        \end{proof}
    \item[(c)] Carefully explain how this verifies that equation (1) holds for all \( x \in [-1,1] \).
        \begin{proof}
        Since \( \sum_{ n=0 }^{ \infty  } a_n  \) converges absolutely, we can use the Weirstrass M-test to show that 
        \[  \sum_{ n=0 }^{ \infty  } a_n x^{n}  \] converges uniformly on \( [-1,1] \). Observe that for any \( x \in [-1,1]  \), we have 
        \[  | a_{n} x^{n} | \leq a_{n}.\] Since the right hand side of the above inequality produces a series that converges absolutely (from part(b)), we know that the power series \[  \sum_{ n=0 }^{ \infty  } a_{n} x^{n}  \] must converge uniformly on \( [-1,1] \). 
        \end{proof}
\end{enumerate}

Our goal is to find polynomials that approximate the absolute value function on an interval containing the non-differentiable point at the origin.

\subsubsection{Exercise 6.7.7} 
\begin{enumerate}
    \item[(a)] Use the fact that \( | a  |  = \sqrt{ a^{2} }  \) to prove that, given \( \epsilon > 0  \), there exists a polynomial \( q(x)  \) satisfying 
        \[  | | x  | - q(x)  | < \epsilon \]
        for all \( x \in [-1,1] \).
        \begin{proof}
            Let \( \epsilon > 0   \). Note that \( | x  |  = \sqrt{ x^{2}  }  = \sqrt{ 1 - (1 - x^2)  }  \) ha a series representation 
            \[  \sqrt{ 1 - (1-x^2) } = \sum_{ n=0  }^{ \infty  } a_n (1 -x^2)^n \tag{1} \] where 
            \[  a_n =  \prod_{i=1}^{n} \frac{ 2i - 3  }{ 2i } \]which holds for all \( x \in [-1,1] \). Since the right hand side of (1) is just a polynomial, we can set 
            \[  q(x) = \sum_{ n=0 }^{ \infty  } a_n (1-x^{2})^{n} \] such that 
            \[  | | x  | - q(x)   | = | \sqrt{ 1 - (1-x^2) } - q(x) | < \epsilon.  \]
        \end{proof}

    \item[(b)] Generalize this conclusion to an arbitrary interval \( [a,b] \).
        \begin{proof}
			Let \( c = \max \{ a, b  \}  \) and let \( x \in [-1,1] \). Then let \( \epsilon / c > 0    \) such that there exists a polynomial \( q(x)  \) such that
			\[ \Big| \Big| \frac{ x }{ c }  \Big| - q \Big( \frac{ x }{ c }  \Big) \Big| < \frac{ \epsilon  }{ c  } \tag{1}. \]
		Then let \( x \in [a,b]  \) be arbitrary. Then multiplying by \( c  \) on both sides of (1), we get that
		\[ \Big| \Big| \frac{ x }{ c } \Big| - q \Big( \frac{ x }{ c }  \Big) \Big| < \frac{ \epsilon  }{ c  }  \iff  \Big| | x | - c \cdot q \Big( \frac{ x }{ c } \Big) \Big| < \epsilon   \]
		where we have found the polynomial \( c \cdot q (x/c) \). Hence, we have 
		\[  | | x | - q(x)  | < \epsilon  \]
		for all \( x \in [a,b] \).
        \end{proof}
\end{enumerate}



\subsection{Proving WAT} 

Knowing that the absolute value function is integral to the proof of WAT, we can now fill in the details of the proof. Fix \( a \in [-1,1]  \) and set 
\[  h_{a}(x) = \frac{ 1 }{ 2 }  ( | x -a  | + (x-a)) \] over \( [-1,1]  \). Note that \( h_{a}  \) is polygonal and satisfies \( h_{a}(x) = 0  \) for all \( x \in [-1,a] \).

\subsubsection{Exercise 6.7.8} 
\begin{enumerate}
    \item[(a)] Explain why we know \( h_{a}(x)  \) can be uniformly approximated with a polynomial on \( [-1,1] \).
        \begin{proof}[Solution]
        We know that \( h_{a}(x)  \) can be uniformly approximated with a polynomial \( q(x)  \) since 
		\[  | x - a  | = \sqrt{ (x-a)^2  } = \sqrt{ 1 - (1 - (x-a)^2) } \tag{1}\]
		is continuous on \( [-1,1] \) and the fact that (1) contains a Taylor series representation on \( [-1,1] \).
		Using the same techniques as exercise 6.7.7, we can show that given an \( \epsilon > 0  \)that we can find a polynomial \( p(x)  \) such that 
		\[  | h_a(x) - q(x)   | < \epsilon. \]
        \end{proof}
	\item[(b)] Let \( \phi  \) be a polygonal function that is linear on each subinterval of the partition 
		\[  -1 = a_{0} < a_{1} < a_{2} < \dotsb < a_{n} = 1. \]
		Show that there exists constants \( b_{0}, b_{1}, \dots, b_{n-1} \) so that 
		\[  \phi(x) = \phi(-1) + b_{0} h_{a_0 }(x) + b_{1} h_{a_{1}}(x) + \dotsb + b_{n-1}h_{a_{n-1}}(x) \]
		for all \( x \in [-1,1] \).
		\begin{proof}
			Define \( \phi(x)  \) at the endpoints of the interval \( [-1,1] \). Then using polynomial approximation of \( h_{a_n}(x)  \), we can define the series
			\[  \sum_{ i=1 }^{ n  } b_{i-1} h_{a_{i-1}}(x) \]
			where the coefficients \( b_{0}  \) and \( b_{n-1}  \) for \( n \geq 1  \) can be found by 
			\begin{align*}
				b_{i-1} &= \frac{ \phi(a_i) - \phi(a_{i-1}) }{ a_i - a_{i-1}  }  - b_{n-1}.\\
				b_{0} &= \frac{ \phi(a_{1}) - \phi(a_{0})  }{ a_{1} - a_{0} }.
			\end{align*}
			Putting everything together, we define 
			\[  \phi(x) = \phi(-1) +  \sum_{ i=1 }^{ n  } b_{i-1} h_{a_{i-1}}(x). \]
		\end{proof}
	\item[(c)] Complete the proof of WAT for the interval \( [-1,1]  \), and then generalize to an arbitrary interval \( [a,b]  \).
		\begin{proof}
		
		\end{proof}
\end{enumerate}














\chapter{The Riemann Integral}

\section{Dicussion: How should Integration be Defined?}


Recall the Fundamental Theorem of Calculus:

\begin{align*}
	\int_{ a }^{ b } F'(x) \ dx  &= F(b) - F(a) \text{ and } \\
	\text{ if } G(x) &= \int_{ a }^{ x } f(t) \ dt, \text{ then } G'(x) = f(x)
\end{align*}
 
which tells us that there is an inverse relationship between differentiation and integration. Before Analysis, the integral of some function \( f  \) is satisfied \( F' = f  \). But we need to build a rigorous foundation for the statements above. 

Around the time of Cauchy and Riemann, the notion that a theory built around integrals having an inverse relationship to derivatives were thrown out the window in favor of the more intuitive notion of the "area under the curve", the concept we mostly associate integrals to today.

The Riemann integral as it is called today can be explained as taking some function \( f  \) on some closed interval \( [a,b] \), where this interval is partitioned into smaller subintervals, say, \( [ x_{k-1}, x_{k}] \). Picking some point \( c_{k } \in [x_{k-1} , x_{k }] \), we can use the \( y \)-value \( f(c_{k }) \) as an approximation for \( f  \) on \( [x_{k-1}, x_{k }] \). Graphically, this process creates a row of thin rectangles constructed to approximate the area between \( f  \) and the \( x  \)-axis. Since the area of each rectangle is just the base multiplied by its height, we have that \( f(c_{k })(x_{ k } - x_{k - 1 }) \). The total area of all the rectangles in the interval \( [a,b] \) is given by the \textit{Riemann sum} 
\[  \sum_{ k=1 }^{ n } f(c_{k })( x_{k } - x_{k -1}).\] It should be noted that area in this context can be assigned negative values if we are taking the areas under a curve for which it is below the \( x \)-axis.

Taking this concept further, the accuracy of the Riemann sum approximation gets better as the width of the rectangles tends to zero. If this limit exists, then we just end up with Riemann's definition of \( \int_{ a }^{ b } f \ dx \). 

Bringing forth a rigorous footing for this concept is not very difficult given our extensive study of the theory dealing with limits and infinite series. What is more interesting to us, however, is deciding under what conditions is \( f  \) allowed to be integrated? 
We will see that the notion of approximating the function \( f  \) using these Riemann sums wherer the quality of the approximation is relate to the difference
\[  | f(x) - f(c_{k }) | \] is connected to the continuity of \( f  \). But is continuity necessarily sufficient to prove that our Riemann sums converge to a well-defined limit? Can it still integrate discontinuous functions such as the Dirichlet functions on \( [0,1] \)?  





\section{The Definition of the Riemann Integral}

Before we have \textit{Riemann sums}, we need to construct \textit{upper sums}  and \textit{lower sums} using the notion of the supremum and infimum. In this section, let us assume that \( f  \) is defined on a closed interval \( [a,b]  \) where \( f  \) is bounded my some \( M > 0  \) on this interval; that is, \( | f(x)  | \leq M  \) for all \( x \in [a,b] \). 

\subsection{Partitions, Upper Sums, and Lower Sums}

\begin{tcolorbox}
\begin{defn}
	A \textit{partition} \( P  \) of \( [a,b]  \) is a finite set of points from \( [a,b]  \) that includes both \( a  \) and \( b  \). The notational convention is to always list the points of a partition \( P = \{ x_{0}, x_{1}, x_{2}, \dots, x_{n} \}  \) in increasing order; thus, \[  a = x_{0} < x_{1} < x_{2} < \dotsb < x_{n} = b. \]
	For each subinterval \( [x_{k-1}, x_{k} ] \) of \( P  \), let 
	\[  m_{k } = \inf \{ f(x) : x \in [x_{k-1} , x_{k } ] \} \ \text{ and } \  M_{k } = \sup \{ f(x) : x \in [x_{k-1}, x_{k }] \}. \]
	The \textit{lower sum} of \( f  \) respect to \( P  \) is given by 
	\[  L(f, P ) = \sum_{ k=1 }^{ n } m_{ k } ( x_{k } -  x_{ k -1 }). \]
	Likewise, we define the \textit{upper sum} of \( f  \) with respect to \( P  \) by 
	\[  U(f, P ) = \sum_{ k=1 }^{ n } M_k ( x_{k } - x_{ k -1 }). \]
\end{defn}
\end{tcolorbox}

It is clear from this definition that 
\[  U(f, P ) \leq L(f, P ).\]  This inequality holds even with respect to different partitions.

\begin{tcolorbox}
\begin{defn}
A partition \( Q  \) is a \textit{refinement} of a partition \( P  \) if \( Q  \) contains all of the points of \( P  \); that is, if \( P \subseteq Q  \).



\end{defn}
\end{tcolorbox}

\begin{tcolorbox}
\begin{lem}
If \( P \subseteq Q  \), then \( L(f,P) \leq L(f,Q)  \), and \( U(f,P ) \geq U(f,Q) \).
\end{lem}
\end{tcolorbox}

\begin{proof}
	First let us prove the inequality for the lower sums. Let \( P \subseteq Q  \). Suppose we refine \( P  \) by a adding a point \( z  \) to some subinterval \(  [x_{k-1}, x_{k }] \) of P. Then we have that 
	\begin{align*}
	    m_{k } ( x_{k } - x_{ k -1 })&= m_{  k } (x_{k } - z ) + m_{ k } (z - z _{k -1 })  \\
									 &\leq m'_{k }( x_{k } - z ) + m"_{k} (z - x_{k -1 }),
	\end{align*}
	where 
	\[  m'_{k} = \inf \{ f(x) : x \in [z, x_{k }] \} \  \ \text{ and } \  \ m"_{k} = \inf \{ f(x) : x \in [x_{k -1 }, z ] \}  \]
	are each necessarily as large or larget than \( m_k  \). We can use an induction argument to show that \(  L(f, P ) \leq L(f,Q) \). The same can be done for the upper sums.

	That is, take some point \( \ell \in [x_{k-1}, x_{k }] \) such that 
	\begin{align*}
	    M_{k } (x_{k} - x_{k -1 }) &= M_{k } (x_{ k } - \ell ) + M_{k } ( \ell - x_{k -1 }) \\
								   &\geq M_k' (x_{k } - \ell ) + M"_{k } (\ell - x_{k -1 }) \\
	\end{align*}
	\[  M'_{k} = \sup \{ f(x) : x \in [\ell, x_{k }] \} \  \ \text{ and } \  \ M"_{k}= \sup \{ f(x) : x \in [x_{k -1 }, \ell ] \}.  \]
	Using induction again to repeat the argument above, we can show that \( U(f, P) \geq U(f,Q) \).
\end{proof}

\begin{tcolorbox}
\begin{lem}
	If \( P_1  \) and \( P_2  \) are any two partitions of \( [a,b]  \), then \( L(f, P_{1}) \leq U(f, P_2) \).
\end{lem}
\end{tcolorbox}

\begin{proof}
Let \( Q = P_{1} \cup P_{2}  \) be the so-called \textit{common refinement} of \( P_{1} \) and \( P_{2} \). Because \(  P_{1} \subseteq Q  \) and \( P_{2} \subseteq Q  \), it follows that 
\[ L(f, P_{1}) \leq L(f,Q) \leq U(f,Q) \leq U(f, P_{2}).  \]
\end{proof}


\subsection{Integrability}

Another way of thinking Integrability is to think of the upper sums as an overestimate of the value of the integral and lower sums as an underestimate of the value of the integral. We can see that as the we continually refine our partitions, the upper sums become smaller and smaller and the lower sums become larger and larger until they meet at some common point in the middle. 
	Rather than thinking of this whole process as the limit of these sums, we will instead make use of the Axiom of Completeness and consider the \textit{infimum} of the upper sums and the \textit{supremum} of the lower sums.

\begin{tcolorbox}
\begin{defn}
	Let \( \mathcal{P}  \) be the collection of all possible partitions of the interval \( [a,b] \). The \textit{upper integral} of \( f \) is defined to be 
	\[  U(f) = \inf \{ U(f,P) : P \in \mathcal{P} \}.  \]
	In a similar way, define the \textit{lower integral} of \( f  \) by 
	\[  L(f) = \sup \{ L(f,P) : P \in \mathcal{P}. \}  \]
\end{defn}
\end{tcolorbox}
 The following fact is not surprising. 
\begin{tcolorbox}
\begin{lem}
	For any bounded function \( f  \) on \( [a,b]  \), it is always the case that \( U(f) \leq L(f) \).
\end{lem}
\end{tcolorbox}
Why is this not surprising you ask? It is because \( U(f) \) is the exact definition for what it means to be the least upper bound and \( L(f)  \) is the exact definition for what it means to be the greatest upper bound.

\begin{tcolorbox}
	\begin{defn}[Riemann Integrability]
	A bounded function \( f  \) defined on the interval \( [a,b] \) is \textit{Riemann-integrable} if \( U(f) = L(f) \). In this case, we define \( \int_{ a }^{ b } f   \) or \( \int_{ a }^{ b } f(x) \ dx \) to be this common value; namely, 
	\[  \int_{ a }^{ b } f  = U(f) = L(f). \]
	\end{defn}
\end{tcolorbox}

\subsection{Criteria for Integrability} 

To summarize, we know that it is always the case that for a bounded function  \( f  \) on a closed interval \( [a,b]  \), we have 
\[  \sup \{ L(f,P) : P \in \mathcal{P} \} = L(f) \leq U(f) = \inf \{ U(f,P) : P \in \mathcal{P} \}.\] For \( f  \) to be integrable, the inequaltity above must be an equality. More rigorously, finding our if a function is integrable is equivalent to the existence of partitions whose upper and lower sums are arbitrarly close together.

\begin{tcolorbox}
	\begin{thm}[Integrability Criterion]
		A bounded function \(  f \) is integrable on \( [a,b]  \) if and only if, for every \( \epsilon > 0  \), there exists a partition \( P_{\epsilon } \) of \( [a,b] \) such that 
		\[  U(f, P_{\epsilon }) - L(f, P_{\epsilon }) < \epsilon. \]
	\end{thm}
\end{tcolorbox}

\begin{proof}
	(\( \Leftarrow \))Let \( \epsilon > 0  \). If such a partition \( P_{\epsilon } \) of \( [a,b]  \) exists, then we have
	\[ U(f) - L(f) \leq U(f, P_{\epsilon }) - L(f, P_{\epsilon }) < \epsilon. \]
	Since \( \epsilon  \) is arbitrary, we know by Theorem 1.2.6 that \( U(f) = L(f) \).Hence, \( f  \) is integrable.  
	(\( \Rightarrow \)) 

	The proof for the forwards direction works more like using the triangle inequality but this time with parentheses in place of absolute values. Since \( U(f)  \) is just the greatest lower bound of the upper sums, we know that, given some \( \epsilon > 0  \), there must exists a partition \( P_{1} \) such that 
	\[  U(f, P_{1}) < U(f) + \frac{ \epsilon  }{ 2 }. \tag{1} \]
	Likewise, there exists a partition \( P_{2}  \) satisfying 
	\[  L(f, P_{2}) > L(f) - \frac{ \epsilon  }{ 2 } \iff -L(f, P_{2}  ) < \frac{ \epsilon  }{ 2 } - L(f). \tag{2} \]
	Adding (1) and (2) together, we end up with 
	\[  U(f, P_{1}) - L(f, P_{2}) < ( U(f) - L(f)) + \epsilon.  \]
	Because \( f  \) is integrable on \( [a,b]  \), we have \( L(f) = U(f)  \) which implies that 
	\begin{align*}
	    U(f, P_{\epsilon }) - L(f, P_{\epsilon}) &\leq  U(f, P_{1}) - L(f, P_{2})  \\
												 &< (U(f) - L(f)) + \epsilon \\
												 &= \epsilon.
	\end{align*}
\end{proof}

At the beginning of this section, it was clear that a function's integrability was tied to its continuity. To make this precise, let us consider the arbitrary partition \( P = \{ x_{0}, x_{1}, x_{2}, \dots, x_{n} \}  \) of \( [a,b]  \), and let us define \( \delta x_{k } = x_{k } - x_{k-1 } \). Then, 
\[  U(f,P) - L(f,P) = \sum_{ k=1 }^{ n } (M_{k } - m_{k } ) \Delta x_{ k }, \]
where \( M _{k }  \) and \( m_{ k }  \) are the supremum and infimum of \( f  \) on the subinterval \( [x_{k -1 } , x_{ k  }] \) (as defined before) respectively. The size of 
\[  U(f,P) - L(f,P)   \] is dependent upon the size of the differences of our extremums \(  M_{k } - m_{k } \), which can be interpreted as the variation of the range of \( f  \) over the interval \( [x_{k-1 }, x_{k }] \). This variation and its restriction of our function \( f  \) on arbitrarly small intervals on \( [a,b]  \) is \textit{precisely} what it means fopr \( f  \) to be uniformly continuous on \( [a,b] \).

\begin{tcolorbox}
\begin{thm}
	If \( f  \) is continuous on \( [a,b] \), then it is integrable.
\end{thm}
\end{tcolorbox}

\begin{proof}
	Since \( f  \) is continuous on the compact set \( [a,b]  \), we know that \( f  \) must be uniformly continuous on \( [a,b]  \). Let \( \epsilon > 0  \) and choose \( \delta = \epsilon / b -a  \) where \( b = x_{n}   \) and \( a = x_{1}  \) are the endpoints of the interval \( [a,b ] \). Then whenever \( | x - y  | < \delta  \), we have that 
	\[  | f(x) - f(y)  | < \frac{ \epsilon  }{  b - a  }. \]
	Now, let \( P  \) be a partition of \( [a,b]  \) where \( \Delta x_{ k } = x_{ k } - x _{ k -1 } \) is less than \( \delta  \) for every subinterval \( [x_{k-1 } , x_{k }] \) of \( P  \). Given a particular subinterval \( [x_{k-1}, x_{ k } ] \subseteq [a,b]   \), we can use the Extreme Value Theorem, to state that the supremum \(  M_{k} = f(z_{k })  \) for some \( z_{k } \in [x_{ k-1 } , x_{k}]  \) as well as the infimum \( m_{k } = f(y_{k }) \) for some \( y_{ k } \in [x_{k -1 } , x_{ k }]  \). But this means that whenever \( | z_{k } - y_{k } | < \delta  \), we have 
	\[  M_{k } - m_{k } = f(z_{k}) - f(y_{k } ) < \frac{ \epsilon  }{  b -a  }. \]
	This implies that 
	\[  U(f, P ) - L(f,P ) = \sum_{ k=1 }^{ n } (M_{k } - m_{ k } ) \Delta x_{k} < \frac{ \epsilon  }{ b -a  }  \sum_{ k=1 }^{ n } \Delta x_{k} = \epsilon \]
	and \( f  \) is integrable by the criterion provided in Theorem 7.2.8.
\end{proof}



\subsection{Definitions and Theorems}


\begin{tcolorbox}
\begin{defn}
	A \textit{partition} \( P  \) of \( [a,b]  \) is a finite set of points from \( [a,b]  \) that includes both \( a  \) and \( b  \). The notational convention is to always list the points of a partition \( P = \{ x_{0}, x_{1}, x_{2}, \dots, x_{n} \}  \) in increasing order; thus, \[  a = x_{0} < x_{1} < x_{2} < \dotsb < x_{n} = b. \]
	For each subinterval \( [x_{k-1}, x_{k} ] \) of \( P  \), let 
	\[  m_{k } = \inf \{ f(x) : x \in [x_{k-1} , x_{k } ] \} \ \text{ and } \  M_{k } = \sup \{ f(x) : x \in [x_{k-1}, x_{k }] \}. \]
	The \textit{lower sum} of \( f  \) respect to \( P  \) is given by 
	\[  L(f, P ) = \sum_{ k=1 }^{ n } m_{ k } ( x_{k } -  x_{ k -1 }). \]
	Likewise, we define the \textit{upper sum} of \( f  \) with respect to \( P  \) by 
	\[  U(f, P ) = \sum_{ k=1 }^{ n } M_k ( x_{k } - x_{ k -1 }). \]
\end{defn}
\end{tcolorbox}


\begin{tcolorbox}
\begin{defn}
A partition \( Q  \) is a \textit{refinement} of a partition \( P  \) if \( Q  \) contains all of the points of \( P  \); that is, if \( P \subseteq Q  \).



\end{defn}
\end{tcolorbox}


\begin{tcolorbox}
\begin{lem}
If \( P \subseteq Q  \), then \( L(f,P) \leq L(f,Q)  \), and \( U(f,P ) \geq U(f,Q) \).
\end{lem}
\end{tcolorbox}


\begin{tcolorbox}
\begin{lem}
	If \( P_1  \) and \( P_2  \) are any two partitions of \( [a,b]  \), then \( L(f, P_{1}) \leq U(f, P_2) \).
\end{lem}
\end{tcolorbox}


\begin{tcolorbox}
\begin{defn}
	Let \( \mathcal{P}  \) be the collection of all possible partitions of the interval \( [a,b] \). The \textit{upper integral} of \( f \) is defined to be 
	\[  U(f) = \inf \{ U(f,P) : P \in \mathcal{P} \}.  \]
	In a similar way, define the \textit{lower integral} of \( f  \) by 
	\[  L(f) = \sup \{ L(f,P) : P \in \mathcal{P}. \}  \]
\end{defn}
\end{tcolorbox}


\begin{tcolorbox}
\begin{lem}
	For any bounded function \( f  \) on \( [a,b]  \), it is always the case that \( U(f) \leq L(f) \).
\end{lem}
\end{tcolorbox}


\begin{tcolorbox}
	\begin{defn}[Riemann Integrability]
	A bounded function \( f  \) defined on the interval \( [a,b] \) is \textit{Riemann-integrable} if \( U(f) = L(f) \). In this case, we define \( \int_{ a }^{ b } f   \) or \( \int_{ a }^{ b } f(x) \ dx \) to be this common value; namely, 
	\[  \int_{ a }^{ b } f  = U(f) = L(f). \]
	\end{defn}
\end{tcolorbox}


\begin{tcolorbox}
	\begin{thm}[Integrability Criterion]
		A bounded function \(  f \) is integrable on \( [a,b]  \) if and only if, for every \( \epsilon > 0  \), there exists a partition \( P_{\epsilon } \) of \( [a,b] \) such that 
		\[  U(f, P_{\epsilon }) - L(f, P_{\epsilon }) < \epsilon. \]
	\end{thm}
\end{tcolorbox}


\begin{tcolorbox}
\begin{thm}
	If \( f  \) is continuous on \( [a,b] \), then it is integrable.
\end{thm}
\end{tcolorbox}




\subsection{Exercises}

\subsubsection{Exercise 7.2.1} Let \( f  \) be a bounded function on \( [a,b]  \), and let \( P  \) be an arbitrary parititon of \(  [a,b]  \). First, explain why \( U(f) \geq L(f,P)  \). Now, prove Lemma 7.2.6.
\begin{proof}
By definition, we know that 
\[  U(f) = \inf \{ U(f,P) : P \in \mathcal{P} \}  \] and \( U(f,P)   \), for any given partition \( P \in \mathcal{P} \), is an upper bound of the set \( \mathcal {L} = \{ L(f,P) 
: P \in \mathcal{P}\}  \). Since \( U(f)  \) is any upper bound of \( \mathcal{L} \) and \( L(f)  \) is the \textit{least upper bound} then, we know that \( U(f) \geq L(f,P) \). Then given an arbitrary \( \epsilon > 0  \), we can find some partition \( P_{\epsilon } \in \mathcal{P} \) such that 
\[  L(f) - \epsilon < L(f, P_{\epsilon }) \leq U(f)  \]
which implies that 
\[  L(f) \leq U(f) + \epsilon. \]
Since \( \epsilon > 0  \) is arbitrary, we know that \( U(f) \geq L(f) \).
\end{proof}


\subsubsection{Exercise 7.2.3 (Sequential Criterion for Integrability)} 
\begin{enumerate}
	\item[(a)] Prove that a bounded function \( f  \) is integrable on \( [a,b]  \) if and only if there exists a sequence of partitions \( (P_{n})_{n=1}^{\infty } \) satisfying 
		\[  \lim_{ n \to \infty  }  [U(f, P_{n}) - L(f, P_{n})] = 0,  \]
		and in this case \( \int_{ a }^{ b }  f = \lim_{ n \to \infty  }  U(f, P_{n} ) = \lim_{ n \to \infty  }  L(f, P_{n}). \)
		\begin{proof}
		First we prove the forwards direction. We can find a \textit{common refinement} where \( Q = P_{1} \cup P_{2} \) such that \( P_{1} \subseteq Q  \) and \( P_{2} \subseteq Q  \). Given that \( f  \) is bounded, let \( \epsilon > 0  \) such that the following statements can be made:
		\begin{align*}
			U(f, P_1) &< U(f) + \frac{ \epsilon  }{ 2  }  \iff U(f, P_{1}) - U(f) < \frac{ \epsilon  }{ 2 } \\
			L(f, P_{2}) &> L(f) - \frac{ \epsilon  }{ 2 } \iff L(f) - L(f, P_{2}) < \frac{ \epsilon  }{ 2 } .
		\end{align*}
		Now assume \( L(f) = U(f) \). Then for some \( N \in \N  \), we assume that for any \( n \geq N  \), we have
		\begin{align*}
		    | U(f, P_{n}) - L(f, P_{n}) | &\leq | U(f, P_{1}) - L(f, P_{2}) |  \\
										  &= | U(f, P_{1}) - U(f)  +  U(f) - L(f)   +  L(f) - L(f, P ) | \\
										  &\leq | U(f, P_{1}) - U(f)  |  + | U(f) - L(f)  | + | L(f) - L(f, P ) | \\
										  &= | U(f,P_{1}) - U(f)  |  + | L(f) - L(f,P) | \\
										  &< \frac{ \epsilon  }{ 2  } + \frac{ \epsilon  }{ 2  } = \epsilon.
		\end{align*}

		Now we shall prove the backwards direction. Choose \( \epsilon = 1/n   \). Assume there exists a partition \( P_{n} \) such that for some \( N \in \N  \), we have that for any \( n \geq N  \)
		\[  U(f) - L(f) \leq U(f, P_{n})  - L(f, P_{n}) < \frac{ 1 }{ n } \to 0. \]
		But by Theorem 1.2.6, we have that \( U(f) = L(f) \). Hence, \( f  \) must be integrable.
		\end{proof}
	\item[(b)] For each \( n  \), let \( P_{n} \) be the partition of \( [0,1] \) into \( n \) equal subintervals. Find formulas for \( U(f, P_{n}) \) and \( L(f,P_{n}) \) if \( f(x) = x  \). The formula \( 1 + 2 + 3 + \dotsb + n = n(n+1)/2  \) will be useful.
		\begin{proof}[Solution]
			Let \( f(x) = x  \). Then for each \( n \in \N  \), let \( x_{k } = k / n  \). Then using the formula \( 1 + 2 + 3 + \dotsb + n = n(n+1) / 2 \), we attain the formula for the \textit{Upper Sum}  
			\begin{align*}
			   U(f, P_{n}) &= \sum_{ k=1 }^{ n } M_{k } ( x_{k } -  x_{k -1 } ) \\
						   &= \sum_{ k=1 }^{ n } \frac{ k  }{ n } \Big(  \frac{ k  }{ n } - \frac{ k-1 }{ n } \ \Big) \\
						   &= \sum_{ k=1 }^{ n } \frac{ k  }{ n^2 } \\
						   &= \frac{ n(n+1) }{ 2n^2 } \\
						   &= \frac{ n+1 }{ 2n }. \\ 
			\end{align*}
		Similarly, we can let \( x_{k-1} = (k-1)/n \) to attain the formula for the \textit{Lower Sum} 
		\begin{align*}
		    L(f, P_{n})&= \sum_{ k=1 }^{ n } m_{k } ( x_{k} - x_{k-1}) \\
					   &= \sum_{ k=1 }^{ n } \frac{ k-1 }{ n } \Big( \frac{ k-1 }{ n } - \frac{ k-2 }{ n }  \Big) \\
					   &= \sum_{ k=1 }^{ n } \frac{ k-1 }{ n^2 } \\
					   &= \frac{ n^{2} + n - 2  }{ 2 n^{2} }. 
		\end{align*}
		Hence, we have the following formulas:
		\begin{align*}
		    U(f, P_{n}) &= \frac{ n^2 + n  }{ 2n^2 } \\
			L(f, P_{n}) &=  \frac{ n^{2} + n - 2  }{ 2n^2 } 
		\end{align*}
		\end{proof}
	\item[(c)] Use the sequential criterion for integrability from (a) to show directly that \( f(x) = x  \) is integrable on \( [0,1]  \) and compute \( \int_{ 0 }^{ 1 } f   \).
		\begin{proof}[Solution]
			To show that \( f(x)  \) integrable on \( [0,1] \), all we need to show is 
			\[  \lim_{ n \to \infty  }  [ U(f, P_{n}) - L(f, P_{n })] = 0.  \]
			Indeed, we have 
			\begin{align*}
			    \lim_{ n \to \infty  }  [ U(f, P_{n}) - L(f, P_{n })] &= \lim_{ n \to \infty  } \frac{ 1 }{ n^2 }  \\
																	  &= 0. 
			\end{align*}
			Hence, \( f(x) = x  \) is integrable on \( [0,1] \). For large \( n  \), we know that 
			\( U(f) \leq 1 /2  \) and \( L(f) \geq 1 / 2  \). Since \( U(f) = L(f)  \), we know that 
			\[  \int_{ a }^{ b } f  = \frac{ 1 }{ 2 }. \]
		\end{proof}
\end{enumerate}


\subsubsection{Exercise 7.2.4} Let \( g  \) be bounded on \( [a,b]  \) and assume there exists a partition \( P  \) with \( L(f, P) = U(g,P)  \). Is \( g  \) necessarily continuous? Is it integrable? If so, what is the value of \( \int_{ a }^{ b }  g \).
\begin{proof}[Solution]
	Yes, \( g  \) is continuous since it is just the constant function and constant functions are continuous. Let \( \epsilon > 0  \). It is integrable becuse there exists a partition \( P_{\epsilon } \) such that \( L(g,P_{\epsilon }) = U(g,P_{\epsilon })  \) implies that 
	\[ U(g,P) - L(g,P) < \epsilon.  \] Furthemore, we have \( \int_{ a }^{ b }  g = g(x_{k }) \Delta x_{k } \).
\end{proof}





\subsubsection{Exercise 7.2.5} Assume that, for each \( n  \), \( f_{n} \) is an integrable function on \( [a,b]  \). If \( (f_n) \to f  \) uniformly on \( [a,b]   \), prove that \( f  \) is also integrable on this set. (We will see that this conclusion does not necessarily follow if the convergence is pointwise.)
\begin{proof}
	Let \( \epsilon > 0  \). Since \( f_{n} \) is integrable on \( [a,b]  \), there exists a partition \( P_{\epsilon } \) such that 
	\[ U(f_{n}, P_{\epsilon }) - L(f_{n}, P_{\epsilon }) < \epsilon. \]
	Since \( (f_{n}) \to f  \) uniformly, we can bound the difference of the upper sums between \( f_{n} \) and \( f  \) by choosing \( N_{1} \in \N  \) such that for any \( n \geq N_{1}  \), we have
	\[  U(f, P_{\epsilon }) - U(f_{n}, P_{\epsilon }) = (M_{k} - f_{n}(\ell_k)) \sum_{ k=1 }^{ n } \Delta x_{k } < \frac{ \epsilon  }{  3(b -a)  } \sum_{ k=1 }^{ n } \Delta x_{ k }  \] for some \( \ell_{k } \in [x_{k-1}, x_{k }] \) and similarly choosing \( N_{2} \in \N  \) such that for any \( n \geq N_{2} \)
	\[  L(f_n, P_{\epsilon }) - L(f, P_{\epsilon }) = ( f_{n}(s_{k }) - m_{k}) \sum_{ k=1 }^{ n } \Delta x_{k } < \frac{ \epsilon  }{  3(b -a)  } \sum_{ k=1 }^{ n } \Delta x_{ k }.\] Furthermore, we must also choose \( N_{3} \in \N  \) such that for any \( n \geq N_{3} \) 
	\[  | U(f_{n}, P_{\epsilon }) - L(f_{n}, P_{\epsilon }) | < \frac{ \epsilon  }{ 3 (b-a) } . \]
	Then observe that
	\begin{align*}
		U(f, P_{\epsilon }) - L(f, P_{\epsilon }) &= U(f, P_{\epsilon }) - U(f_n, P_{\epsilon }) + U(f_{n}, P_{\epsilon }) - L(f_{n}, P_{\epsilon }) \\ 
												  &+ L(f_{n}, P_{\epsilon }) - L(f, P_{\epsilon }) \\
												  &< \frac{ \epsilon  }{ b - a  } \sum_{ k=1 }^{ n } \Delta x_{k} = \epsilon.
	\end{align*}
\end{proof}


\subsubsection{Exercise 7.2.6} A \textit{tagged partition} \( (P, \{ c_{k } \} ) \) is one where in addition to a partition \( P  \) we choose a sampling point \( c_{k } \) in each of the subintervals \( [ x_{k-1} , x_{k }]  \). The corresponding \textit{Riemann Sum}, 
\[  R(f, P ) = \sum_{ k=1 }^{ n } f(c_{k }) \Delta x_{k },  \] is discussed in Section 7.1, where the following definition is alluded to. 
\begin{tcolorbox}
	\begin{defn}[Riemann's Original Definition of the Integral]
	A bounded function \( f  \) is \textit{integrable} on \( [a,b]  \) with \( \int_{ a }^{ b } f = A  \) if for all \( \epsilon > 0   \) there exists \( \delta > 0  \) such that for any tagged partition \( (P, \{ c_{k } \} ) \) satisfying \( \Delta x_{k } < \delta  \) for all \( k  \), it follows that 
	\[  | R(f,P) - A  | < \epsilon. \]
	\end{defn}
\end{tcolorbox}

Show that if \( f  \) satisfies Riemann's definition above, then \( f  \) is integrable in the sense of Definition 7.2.7.

\begin{proof}
Let \( \epsilon > 0  \). Choose \( \delta = \min \{ \delta_1  , \delta_2  \}  \). Let \( (P, \{ c_{k} \} ) \) be an arbitrary tagged partition satisfying \( \Delta x_{k } < \delta  \) for all \( k  \). Let 
\[ R(f, P) = \sum_{ k=1 }^{ n } f(c_{k }) \Delta x_{k }.  \]
We want to show that \( L(f) = U(f)  \) so observe that 
\begin{align*}
    | U(f) - L(f)  | &= | U(f) - R(f,P) + R(f,P) - L(f)   |  \\
					 &\leq | U(f) - R(f,P)  | + | R(f,P) - L(f)  | \\
					 &< \frac{ \epsilon  }{ 2 }  + \frac{ \epsilon  }{ 2 } \\
					 &= \epsilon.
\end{align*}
By Theorem 1.2.6, we must have \( U(f) = L(f) \).
\end{proof}










\section{Integrating Functions with Discontinuities}

In the last section, we saw how the integrability of \( f  \) is heavily dependent on the continuity of \( f  \). 
\begin{ex}
Consider the function
\[  f(x) = 
\begin{cases}
	1 \ &\text{ for } x \neq 1 \\
	0 \ &\text{ for } x = 1
\end{cases} \] on the interval \( [0,2]  \). Let \( P  \) be any partition of \( [0,2] \). Then we see that \( U(f,P) = 2  \). Notice that the lower sum \( L(f,P ) \) will be less than \( 2  \) for any subinterval containing \( x = 1  \). TO show that \( f  \) is integrable, we can construct a partition that minimizes the effect of the discontinuity by embedding \( x = 1  \) into some very small subinterval.
Let \( \epsilon > 0  \), and consider the partition \( P_{\epsilon } = \{ 0, 1 - \epsilon / 3, 1 + \epsilon / 3, 2  \}. \) Then, 
\begin{align*}
    L(f, P_{\epsilon }) &= \sum_{ k=1 }^{ 3 } m_{k} \Delta x_{k} \\
						&= 1 \Big( 1 - \frac{ \epsilon  }{ 3 }  \Big) + 0(\epsilon ) + 1 \Big( 1 - \frac{ \epsilon  }{ 3 }  \Big) \\
						&= 2 - \frac{ 2 }{ 3 } \epsilon.
\end{align*}
Since \( U(f, P_{\epsilon }) = 2  \), we have 
\[  U(f, P_{\epsilon }) - L(f, P_{\epsilon }) = 2 - 2 +  \frac{ 2  }{ 3  } \epsilon = \frac{  2 }{ 3  } \epsilon < \epsilon. \] By theorem 7.2.8, we conclude that \( f  \) is integrable.
\end{ex}

In the last example, we find that integrating simple functions with a discontinuity is as simple as isolating the discontinuity in some particularly small subinterval of the partition. 

\begin{tcolorbox}
\begin{thm}
	If \( f: [a,b] \to \R  \) is bounded, and \( f  \) is integrable on \( [c,b] \) for all \( c \in (a,b)  \), then \( f  \) is integrable on \( [a,b]  \). An analogous result holds at the other endpoint. 
\end{thm}
\end{tcolorbox}

\begin{proof}
Let \( \epsilon > 0  \). Our goal is to produce a partition \( P  \) such that \( U(f, P ) - L(f, P ) < \epsilon. \) For any partition, we can write 
\begin{align*}
    U(f, P ) - L(f,P ) &= \sum_{ k=1 }^{ n } (M_{k } - m_{k } ) \Delta x_{ k } \\
					   &- (M_{1} - m_{1})(x_{1} - a) + \sum_{ k=2 }^{ n } (M_{k} - m_{ k }) \Delta x_{k }.
\end{align*}
Our goal now is to choose \( x_{1} \) that is close enough to \( a \) such that 
\[  (M_{1} - m_{1})(x_{1} - a ) < \frac{ \epsilon  }{ 2 }. \] Since \( f  \) is bounded, we know there exists \( M > 0  \) satisfying \( | f(x)  | \leq M  \) for all \( x \in [a,b]  \). Noting that \( M_{1} - m_{1} \leq 2M \), we can pick \( x_{1} \) such that 
\[  x_{1} - a < \frac{ \epsilon  }{ 4 M  }. \] By hypothesis, \( f  \) is integrable on \( [x_{1}, b ] \), so there exists a partition \( P_{1}  \) of \( [x_{1}, b] \) for which 
\[  U(f, P_{1}) - L(f, P_{1}) < \frac{ \epsilon  }{ 2 }. \] Finally, let \( P = \{ a \} \cup P_{1} \) be a partition of \(  [a,b]  \) from which it follows that 
\begin{align*}
    U(f,P) - L(f,P) &= (M_{1} - m_{1})(x_{1} - a ) + \sum_{ k=1 }^{ n } (M_{k } - m_{k }) \Delta x_{k } \\
					&\leq 2M ( x_{1} - a ) + (U(f, P_{1}) - L(f,P_{1})) \\
					&< \frac{ \epsilon  }{ 2 }  + \frac{ \epsilon  }{ 2  } \\
					&= \epsilon.
\end{align*}
The same argument can be applied to the other endpoint.
\end{proof}

This theorem allows us to integrate bounded functions on some closed interval with a single discontinuity at its endpoint. Later on, we will prove that a function with \textit{finite} number of discotinuities is still integrable. What about infinite? 

\begin{ex}
Consider Dirichlet's function 
\[  g(x) = 
\begin{cases}
	1 \ &\text{ for } x \in \Q \\
	0 \ &\text{ for } x \in \mathbb{I}.
\end{cases}  \]
Let \( P  \) be some partition of \( [0,1]  \). The density of the rationals in \( \R  \) implies that we can always find an \( x   \) in any subinterval such that \( g(x) = 1  \) which means that \( U(f,P) = 1  \). On the other hand, the irrationals are also dense in \( \R  \) and hence, we can always find an \( x  \) in every subinterval such that \( g(x) = 0  \). Since this holds for every partition \( P  \), we cannot possibly have \( U(f) = L(f)   \). Hence, \( g  \) cannot be an integrable function.
\end{ex}

Another function that is similar to \( g  \) but is actually integrable is Thomae's function (introduced in 4.1).  

\subsection{Exercises} 

\subsubsection{Exercise 7.3.1} Consider the function 
\[  h(x) = 
\begin{cases}
	1 \ &\text{ for } 0 \leq x < 1 \\
	2 \ &\text{ for } x = 1 
\end{cases} \]
over the interval \( [0,1]  \).
\begin{enumerate}
	\item[(a)] Show that \( L(f,P) = 1  \) for every partition \( P  \) of \( [0,1] \).
		\begin{proof}[Solution]
		By definition of \( h(x)  \), we know that for any \( 0 \leq x < 1  \), we have \( h(x) = 1  \). If \( x = 1  \), then a quick calculation will produce \( U(f, P) = 2  \). Since \( L(f,P) \leq U(f,P)  \), we get that \( L(f,P) = 1  \) for every \( 0 \leq x < 1  \).
		\end{proof}
	\item[(b)] Construct a partition \( P  \) for which \( U(f,P) < 1 + 1/ 10 \).
		\begin{proof}[Solution]
			Let \( P = \{ 0,  1 - \frac{ 1 }{ 11 }, 1 + \frac{ 1 }{ 11 },  1 \}  \)
			and let our subintervals be \( [0, 1 - \frac{ 1 }{ 11 } ], [1 - \frac{ 1 }{ 11 }, 1 + \frac{ 1 }{ 11 } ], \) and  \( [1 + \frac{ 1 }{ 10 }, 1] \). Using the definition of \( U(f,P)  \), we write 
			\begin{align*}
				U(f,P) &= \sum_{ k=1 }^{ 3 } M_{k } \Delta x_{k }   \\
					   &= \Big( 1 - \frac{ 1 }{ 11 }  \Big) + 2 \Big( \frac{ 2 }{ 11 }  \Big) - 2 \Big( \frac{ 1 }{ 11 }  \Big) \\
					   &= 1 + \frac{ 1 }{ 11 } \\
					   &< 1 + \frac{ 1 }{ 10 }. 
			\end{align*}
			Hence, we have 
			\[  U(f,P) < 1 + \frac{ 1 }{ 10 }. \]
		\end{proof}
	\item[(c)] Given \( \epsilon > 0  \), construct partition \( P_{\epsilon } \) for which \( U(f,P_{\epsilon }) < 1 + \epsilon \).
		\begin{proof}[Solution]
		Let \( \epsilon > 0  \) and let \( P = \{ 0, 1 - \frac{ \epsilon  }{ 2 } , 1 + \frac{ \epsilon  }{ 2 }, 1  \}  \). Then using the definition of \( U(f,P)  \), we have
		\begin{align*}
		    U(f,P) &= \sum_{ k=1 }^{ 3 } M_{k } \Delta x_{ k }  \\
				   &= \Big( 1 - \frac{ \epsilon  }{ 2 }  \Big) + 2 \epsilon - 2 \cdot \frac{ \epsilon  }{ 2 } \\
				   &= 1 + \frac{ \epsilon  }{ 2 } \\
				   &< 1 + \epsilon. 
		\end{align*}
		Hence, we have 
		\[ U(f,P) < 1 + \epsilon. \]

		\end{proof}
\end{enumerate}


\subsubsection{Exercise 7.3.3} Let 
\[  f(x) = 
\begin{cases}
	1 \ &\text{ if } x = 1 / n \ \text{ for some } n \in \N \\
	0 \ &\text{ otherwise }.
\end{cases}  \]
Show that \( f  \) is integrable on \( [0,1]  \) and compute \( \int_{ 0 }^{ 1 }  f . \)
\begin{proof}
	Let \( \epsilon > 0  \). To show that \( f  \) is integrable, it suffices to show \( U(f, P_{\epsilon }) - L(f, P_{\epsilon }) < \epsilon  \) for some partition \( P_{\epsilon } \) of \( [0,1] \). Note that \( L(f, P ) = 0  \) for all \( x \in [0,1] \). Let us construct \( P_{\epsilon } \) by taking 
	\[  P_{\epsilon } = \Big\{ 0 , \frac{ 1 }{ n } - \epsilon , \frac{ 1 }{ n }  + \epsilon , 1  \Big\}. \]
	Then taking the definition of \( U(f,P_{\epsilon })  \), we have that 
	\begin{align*}
	    U(f,P_{\epsilon }) &= \sum_{ k=1 }^{ 3 } M_{k } \Delta x_{k }   \\
						   &= 0 \Big( \frac{ 1 }{ n }  - \epsilon \Big) + 2 \epsilon  + 0 \Big( 1 - \frac{ 1 }{ n } - \epsilon  \Big) \\
						   &= 2 \epsilon. \\
	\end{align*}
	Since \( L(f, P_{\epsilon }) = 0  \), we have that 
	\[  U(f, P_{\epsilon }) - L(f, P_{\epsilon }) = 2 \epsilon - 0  < \epsilon.  \]
	Hence, \( f  \) is integrable on \( [0,1] \). Note that \( U(f) = 0  \) and hence, we must have 
	\[  \int_{ 0 }^{ 1 } f = 0. \]
\end{proof}

\subsubsection{Exercise 7.3.4} Let \( f  \) and \( g  \) be functions defined on (possibly different) closed intervals, and assume the range of \( f  \) is contained in the domain of \( g  \) so that the composition \( g \circ f   \) is properly defined.
\begin{enumerate}
    \item[(a)] Show, by example, that it is not the case that if \( f  \) and \( g  \) are integrable, then \( g \circ f  \) is integrable.
		\begin{proof}[Solution]
		Let 
		\[ g(x) = 
		\begin{cases}
			1 \ &\text{ for } x \neq 0 \\
			0 \ &\text{ for } x = 0 
		\end{cases}  \] and 
		\[  f(x) = 
		\begin{cases}
			1 / q \ &\text{ for } x \in \Q \\
			0  \ &\text{ for } x \notin \Q 
		\end{cases}.\] The function \( g(x)  \) is continuous everywhere except at \( 0  \) and \( f  \) is continuous at every \( x \in \mathbb{I} \). But the composition of these functions,
		\begin{align*}
			(g \circ f) (x) &= 
			\begin{cases}
				1 \ &\text{ for } x \in \Q \\
				0 \ &\text{ for } x \notin \Q 
			\end{cases}  \\
		\end{align*}
		which is nowhere continuous and hence not integrable on any interval.
		\end{proof}

		Now decide on the validity of each of the following conjectures, supplying a proof or counterexample as appropriate.
	\item[(b)] If \( f  \) is increasing and \( g  \) is integrable, then \( g \circ f  \) is integrable.
		\begin{proof}[Solution]
		Let \( f(x) = x^{2}  \) and define 
		\[  g(x) = 
		\begin{cases}
			\frac{ 1 }{ \sqrt{ x }  } \ &\text{ for } x \neq 0 \\ 
			0 \ &\text{ for } x = 0.
		\end{cases} \]
		Note that \( f  \) is an increasing function while \( g  \) is an integrable function. But note that 
		\[  (g \circ f)(x) = 
		\begin{cases}
			\frac{ 1 }{ | x |  } \ &\text{ for } x \neq 0 \\
			0 \ &\text{ for } x = 0
		\end{cases} \]
		is non-integrable.
		\end{proof}
		
	\item[(c)] If \( f  \) is integrable and \( g  \) is increasing, then \( g \circ f  \) is integrable.
		\begin{proof}
			Let \( \epsilon > 0 \) and \( P_{\epsilon } \) be an arbitrary partition of an arbitrary closed interval \( [a,b] \). Let us first define \( U(g \circ f , P_{\epsilon }) \) and \( L(g \circ f, P_{\epsilon }) \). So we have 
		\begin{align*}
		    U(g \circ f, P_{\epsilon }) &= \sum_{ k=1 }^{ n } g(M_{k}) \Delta y_{ k }   \\
							L(g \circ f, P_{\epsilon })			&= \sum_{ k=1 }^{ n } g(m_{k }) \Delta y_{k }
		\end{align*}
 Let \( \epsilon > 0  \).  Our goal is to show, using our partition \( P_{\epsilon } \), 
		\[ U(g \circ f, P_{\epsilon }) - L(g \circ f, P_{\epsilon }) < \epsilon.  \] Since \( g  \) is an increasing function on \( [a,b] \) and the range of \( f  \) is contained in the domain of \( g  \), we have that \( g  \) is a bounded function. But this means that we can create sequence \( (\sigma_{k } ) \) where \( \sigma_{k } = g(t_{k }) \). Since \( (\sigma_{k }) \) is increasing and bounded, we know it must be convergent and hence it must be Cauchy. So choose \( N \in \N  \) such that for any \( k,k-1 \geq N  \), we 
		\begin{align*}
		    | M'_{k} - m'_{k}   | &= | g(t_{k}) - g(t_{k-1}) | \\ 
								  &= | \sigma_{k} - \sigma_{k-1} | \\ 
									   &< \epsilon
		\end{align*}
		where \( t_{k} \) and \( t_{k-1}  \) are contained in the domain of \( g \).
	Observe that by our properties of the upper and lower sum of \( f  \), we have 
	\[ y_{k} - y_{k-1} \leq M_{k} - m_{k } \ \text{ for all } k.  \]
	Since \( f  \) is integrable, we can choose \( \epsilon = b -a  \) so that
	\begin{align*}
	    U(f, P_{\epsilon }) - L(f, P_{\epsilon }) &< b - a . \\
	\end{align*}
	Putting everything together, we have 
	\begin{align*}
	    U(g \circ f, P_{\epsilon }) - L(g \circ f, P_{\epsilon }) &= \sum_{ k=1 }^{ n } (M'_{k} - m'_{k}) \Delta y_{k} \\
								&= \sum_{ k=1 }^{ n } (M'_{k} - m'_{k }) \Delta y_{k} \cdot  \frac{ \Delta x_{k }  }{  \Delta x_{k } } \\ 
								&=  \frac{ 1 }{ b-a } \sum_{ k=1 }^{ n }  (M'_{k} - m'_{k }) \Delta y_{k} \Delta x_{k } \\
								&\leq \frac{ 1 }{ b-a } \sum_{ k=1 }^{ n } (M'_{k } - m'_{k } )  (M_{k } - m_{k } ) \Delta x_{k } \\
								&< \frac{ \epsilon  }{ b -a  } \sum_{ k=1 }^{ n } (M_{k } - m_{k }) \Delta x_{k } \\
								&< \frac{ \epsilon  }{ b -a  } \cdot b -a  \\
								&= \epsilon.
 	\end{align*}
	Hence, the composition \( g \circ f \) is an integrable function on \( [a,b] \) by Theorem 7.2.8.
	\end{proof}
	Here is a correction of the above proof
	\begin{proof}
		Suppose \( f  \) is an integrable function on \( [a,b] \) and \( g  \) is an increasing function on \( [a,b]  \). Let \( \epsilon > 0  \). Then we shall show that for an arbitrary partition \( P_{\epsilon } \), we have 
		\[  U( g \circ f, P_{\epsilon }) - L(g \circ f, P_{\epsilon }) < \epsilon. \] 
		Let \( P_{\epsilon } \) be an arbitrary partition. Since \( f  \) is an integrable function, we know that
		\[  U(f, P_{\epsilon }) - L(f, P_{\epsilon }) = \sum_{ k=1 }^{ n } [M_{k} - m_{k }] \Delta x_{k} < \epsilon. \]
		Note that the range of \( f  \) is contained within the domain of \( g  \). Define \( y_{k} = f(x_{k}) \). Since \( g  \) is an increasing function, we can write 
		\begin{align*}
			U(g \circ f, P_{\epsilon }) - L(g \circ f, P_{\epsilon }) &= \sum_{ k=1 }^{ n } [ M'_{k} - m'_{k}] \Delta y_{k} \\ 
																	  &= \sum_{ k=1 }^{ n } [ g(y_{k } ) - g(y_{k-1})] \Delta y_{k} \cdot \frac{ \Delta x_{k } }{ \Delta x_{k}  } \\
																	  &= \frac{ g(f(b)) - g(f(a))   }{ b - a  } \sum_{ k=1 }^{ n } \Delta y_{k} \Delta x_{k } \\ 
		\end{align*}
		By the properties of upper and lower sum of \( f  \), observe that for every \( k  \)
		\[ \Delta y_{k} \leq M_{k } - m_{k }.  \]
		Then we just have 
		\begin{align*}
			\frac{ g(f(b)) - g(f(a))  }{ b -a  }  \sum_{ k=1 }^{ n }  \Delta y_{k} \Delta x_{k} &\leq \frac{ g(f(b)) - g(f(a))  }{ b - a  } \sum_{ k=1 }^{ n } [ M_{k} - m_{k }] \Delta x_{k }  \\
		\end{align*}
		Using the assumption that \( f  \) is integrable on \( [a,b]  \), we have that the right side of the inequality above leads to 
		\[  U(g \circ f, P_{\epsilon }) - L(g \circ f, P_{\epsilon }) < \epsilon. \]
		Using Theorem 7.2.8, we must have \( g \circ f  \) integrable on \( [a,b] \).

	\end{proof}
\end{enumerate}





\section{Properties of the Integral}

For our first property of integration, integrating over a closed interval \( [a,b]  \) is equivalent to taking the sum of the closed intervals, say, \( [a,c]  \) and \( [c,b]  \) that make up \( [a,b]  \).

\begin{tcolorbox}
\begin{thm}
	Assume \( f: [a,b] \to \R   \) is bounded, and let \( c \in (a,b)  \). Then, \( f  \) is integrable on \( [a,b]  \) if and only if \( f  \) integrable on \( [a,c]  \) and \( [c,b]  \). In this case, we have 
	\[  \int_{ a }^{ b } f = \int_{ a }^{ c }  f  + \int_{ b }^{ c }  f. \]
\end{thm}
\end{tcolorbox}

\begin{proof}
	Suppose \( f \) is integrable on \( [a,b]  \). Then for every \( \epsilon > 0   \), there exists a partition \( P  \) such that 
	\[  U(f,P) - L(f,P) < \epsilon. \]
	Since the refinement of a paritition might cause the upper and lower sums to be closet together, all we need to do is simply add the point \( c  \) to \( P  \) if it does not already exists. Then, letting \( P_{1} = P \cap [a,c]  \) be a partition of \( [a,c]  \), and \( P_{2} = P \cap [c,b]  \) be a partition of \( [c,b]  \), it follows that 
	\[  U(f,P_{1}) - L(f, P_{1}) < \epsilon \ \text{ and } \ U(f, P_{2}) - L(f, P_{2}) < \epsilon. \] This implies that \( f  \) is integrable on \( [a,c]  \) and \( [c,b]  \).

	Conversely, if we are given that \( f  \) is integrable on the two smaller intervals \( [a,c]  \) and \( [c,b] \), then given an \( \epsilon > 0  \), we can create partitions \( P_{1}   \) and \( P_{2} \) of \( [a,c] \) and \( [c,b] \), respectively, such that 
	\[  U(f, P_{1}) - L(f, P_{1}) < \frac{ \epsilon  }{ 2  } \ \text{ and } U(f, P_{2}) - L(f,P_{2}) < \frac{ \epsilon  }{ 2 }.  \]
	Since \( P = P_{1} \cup P_{2}  \) produces a partition of \( [a,b]  \), we must have 
	\[ U(f, P) - L(f,P) < \epsilon.  \] Hence, \( f  \) is integrable on \( [a,b]  \).
	Now let \( P = P_{1} \cup P_{2} \) as before, we have 
	\begin{align*}
		\int_{ a }^{ b } f  \leq U(f, P) &< L(f, P)  + \epsilon  \\
										 &= L(f, P_{1}) + L(f, P_{2}) + \epsilon \\
										 &leq \int_{ a }^{ c } f  + \int_{ c }^{ b } f + \epsilon,
	\end{align*}
	This implies that \( \int_{ a }^{ b } f \leq \int_{ a }^{ c }f + \int_{ c }^{ b }  f \). To get the other inequality, observe that 
	\begin{align*}
	    \int_{ a }^{ c } f + \int_{ c }^{ b } f   &\leq U(f, P_{1}) + U(f, P_{2})  \\
												  &< L(f, P_{1}) + L(f, P_{2}) + \epsilon \\
												  &= L(f, P) + \epsilon \\
												  &\leq \int_{ a }^{ b } f + \epsilon.
	\end{align*}
	Since \( \epsilon > 0  \) is arbitrary, we must have that 
	\[  \int_{ a }^{ c } f + \int_{ c }^{ b } f \leq \int_{ a }^{ b } f, \]
	so hence, we have 
	\[  \int_{ a }^{ c } f  + \int_{ c }^{ b } f = \int_{ a }^{ b } f, \]
	as desired.
\end{proof}

Some more properties of integration is outlined in the next theorem.

\begin{tcolorbox}
\begin{thm}
	Assume \( f  \) and \( g  \)  are integrable functions on the interval \( [a,b]  \).
	\begin{enumerate}
		\item[(i)] The function \( f + g  \) is integrable on \( [a,b]  \) with \( \int_{ a }^{ b } (f + g) = \int_{ a }^{ b }  f + \int_{ a }^{ b } g  \).
		\item[(ii)] For \( k \in \R  \), the function \( kf  \) is integrable with \( \int_{ a }^{ b } kf  = k \int_{ a }^{ b }  f. \)
		\item[(iii)] If \( m \leq f(x) \leq M  \) on \( [a,b]  \), then \( m(b-a) \leq \int_{ a }^{ b } f \leq M(b-a) \).
		\item[(iv)] If \( f(x) \leq g(x)  \) on \( [a,b]  \), then \( \int_{ a }^{ b } f  \leq \int_{ a }^{ b } g  \).
		\item[(v)] The function \( | f |   \) is integrable and \( | \int_{ a }^{ b }  f | \leq \int_{ a }^{ b } | f |. \) 
	\end{enumerate}
\end{thm}
\end{tcolorbox}

\begin{proof}\label{proof:theproof}
	Let \( \epsilon > 0 \). Since \( f \) and \( g  \) are integrable on \( [a,b]  \), there exists a partition \( P_{\epsilon } \) such that 
	\[  U(f, P_{\epsilon }) - L(f, P_{\epsilon }) < \frac{ \epsilon  }{ 2 }   \]
	and 
	\[  U(g, P_{\epsilon }) - L(g, P_{\epsilon }) < \frac{ \epsilon  }{ 2 } . \]
	To show that \( (f+g)  \) is integrable on \( [a,b] \), we must show that there exists a partition \( P_{\epsilon } \)
	\[  U(f+g, P_{\epsilon }) - L(f +g, P_{\epsilon }) < \epsilon. \]
	First, observe that by the properties of the supremum and infimum, we have 
	that 
	\[  U(f+g, P_{\epsilon }) \leq U(f, P_{\epsilon }) + U(g, P_{\epsilon }) \]
	and 
	\[  L(f+g, P_{\epsilon }) \geq L(f, P_{\epsilon }) + L(g, P_{\epsilon }). \]
	Now observe that
	\begin{align*}
		U(f+g, P_{\epsilon }) - L(f+g, P_{\epsilon }) &\leq [U(f,P_{\epsilon }) + U(g, P_{\epsilon })] - [L(f,P_{\epsilon }) + L(g, P_{\epsilon })] \\
													  &= [U(f,P_{\epsilon }) - L(f,P_{\epsilon })] + [U(g, P_{\epsilon }) - L(g, P_{\epsilon })] \\
													  &< \frac{ \epsilon  }{ 2 }  + \frac{ \epsilon  }{ 2 } \\
													  &= \epsilon.
	\end{align*}
	Hence, \( (f+g)  \) integrable on \( [a,b]  \) by Theorem 7.2.8.  

	Now to show 
	\[  \int_{ a }^{ b } (f+g)  = \int_{ a }^{ b } f + \int_{ a }^{ b } g \]
	we must show 
	\[  \int_{ a }^{ b } (f+g) \leq \int_{ a }^{ b } f + \int_{ a }^{ b } g \ \text{ and } \int_{ a }^{ b } (f+g) \geq \int_{ a }^{ b } f + \int_{ a }^{ b } g  \]
	Since \( (f+g)  \) integrable on \( [a,b] \), we know that \( U(f+g) = L(f+g) = \int_{ a }^{ b } (f+g)  \). Then for any partition \( P  \), we can use the properties of the lower and upper sum derived back in section 7.2 to get 
	\begin{align*}
	    \int_{ a }^{ b } (f+g)  &\leq U(f+g, P) \\
								&< L(f+g, P) + \epsilon \\
								&\leq L(f) + L(g) + \epsilon \\
								&= \int_{ a }^{ b } f  + \int_{ a }^{ b } g + \epsilon.
	\end{align*}
	Since \( \epsilon > 0   \) is arbitrary, we have \( \int_{ a }^{ b } (f+g)  \leq \int_{ a }^{ b } f + \int_{ a }^{ b } g  \). To get the other inequality, we employ a similar process as above. Observe that, 
	\begin{align*}
	    \int_{ a }^{ b } f + \int_{ a }^{ b } g &\leq U(f,P) + U(g, P)\\
												&< L(f,P) + L(g,P) + \epsilon \\
												&\leq L(f) + L(g) + \epsilon \\
												&= L(f+g) + \epsilon \\
												&= \int_{ a }^{ b } (f+g) + \epsilon.
	\end{align*}
	Again, \( \epsilon > 0    \) implies \( \int_{ a }^{ b } f + \int_{ a }^{ b }  g \leq \int_{ a }^{ b } (f+g) \). Hence, we conclude
	\[  \int_{ a }^{ b } (f+g) = \int_{ a }^{ b } f + \int_{ a }^{ b } g. \]
	
\end{proof}

\begin{proof}[Proof of (ii)]
	Let \( k \in \R  \). By our supremum and infimum properties derived back in chapter 1, we know that 
	\[  \sup \{ k f(x) : x \in [x_{k-1}, x_{k }]  \} = k \sup \{ f(x) : x \in \{ x_{k-1}, x_{k} \}  \}   \]
	and likewise, 
	\[  \inf \{ kf(x) : x \in [x_{k-1}, x_{k}] \} = k \inf \{ f(x) : x  \in [x_{k-1}, x_{k }]  \}.  \]
	Hence, we have
	\[  U(kf, P_{n}) = k U(f, P_{n}) \ \text{ and } \ L(kf, P_{n}) = k L(f, P_{n}).  \]
	Since \( f  \) is integrable on \( [a,b]  \), there exists a partition \( P_{n} \) such that 
	\begin{align*}
	    | U(kf, P_{n}) - L(kf, P_{n}) | &= k | U(f, P_{n}) - L(f, P_{n}) | \to 0.  \\
	\end{align*}
	Hence, \( kf  \) is integrable on \( [a,b] \).

\end{proof}

\begin{proof}[Proof of (iii)]
	Suppose \( m \leq f(x) \leq M  \) on \( [a,b]  \). Since \( f  \) is integrable on \( [a,b]  \), we know that \( U(f) = L(f) = \int_{ a }^{ b } f  \). Let \( P  \) be a parition of \( [a,b] \). By using the properties of the upper sums and lower sums derived back in 7.2, we know that 
	\begin{align*}
	    \int_{ a }^{ b } f = U(f) &\leq U(f,P) \\
			 &= \sum_{ k=1 }^{ n } M_{k } \Delta x_{k } \\
			 &\leq M \sum_{ k=1 }^{ n } \Delta x_{k } \\ 
			 &= M(b-a).
	\end{align*}
	Likewise, we have 
	\begin{align*}
	    \int_{ a }^{ b } f = L(f) &\geq L(f, P) \\
								  &= \sum_{ k=1 }^{ n } m_{k } \Delta x_{k } \\ 
								  &\geq m \sum_{ k=1 }^{ n } \Delta x_{k } \\
								  &= m (b-a).
	\end{align*}
	We conclude 
	\[  m(b-a) \leq \int_{ a }^{ b } f \leq M(b-a) \]
\end{proof}


\begin{proof}[Proof of (iv)]
	Suppose \( f(x) \leq g(x) \) on \( [a,b]  \). Since \( f  \) and \( g \) are integrable, we know that \( U(f) = L(f) = \int_{ a }^{ b } f  \) and \( U(g) = L(g) = \int_{ a }^{ b } g  \). Let \( \epsilon > 0  \) and let \( P  \) be a partition of \( [a,b] \). Then observe that 
	\begin{align*}
		\int_{ a }^{ b } f \leq U(f,P) &\leq U(g, P) \\
									 &< L(g,P) + \epsilon \\ 
									 &\leq \int_{ a }^{ b } g  + \epsilon \\
	\end{align*}
	Since \( \epsilon > 0 \) is arbitrary, we can conclude 
	\[  \int_{ a }^{ b } f \leq \int_{ a }^{ b } g. \]
\end{proof}

\begin{proof}[Proof of (v)]
	Suppose \( | f |  \) is integrable on \( [a,b]  \) and let \( P \) be an arbitrary partition of \( [a,b] \). Let \( \epsilon > 0 \). Then using the properties of the upper sums, we have 
	\begin{align*}
	    \Big| \int_{ a }^{ b } f  \Big| \leq | U(f,P) |
										&= \Big| \sum_{ k=1 }^{ n } M_{k } \Delta x_{k }  \Big| \\
										&\leq \sum_{ k=1 }^{ n } | M_{k } \Delta x_{k } | \\
										&= U(| f |, P) \\
										&\leq \int_{ a }^{ b } | f |   + \epsilon \\
	\end{align*}
	Since \( \epsilon > 0  \) is arbitrary, we conclude that 
	\[  \Big| \int_{ a }^{ b } f  \Big| \leq \int_{ a }^{ b } | f |. \]
\end{proof}

\begin{tcolorbox}
\begin{defn}
	If \( f  \) is integrable on the interval \( [a,b]  \), define 
	\[  \int_{ a }^{ b } f = - \int_{ a }^{b  } f. \]
	Also, for \( c \in [a,b]  \) define 
	\[  \int_{ c }^{ c } f = 0. \]
\end{defn}
\end{tcolorbox}

\subsection{Uniform Convergence and Integration}

An interesting question we can ask is that when we have a sequence of function \( (f_{n}) \) on \( [a,b]  \) where \( f_{n} \to f  \), then does 
\[  \int_{ a }^{ b } f_{n} \to \int_{ a }^{ b } f  \]
hold? Suppose \( f_{n} \to f  \) pointwise, then consider 
\[  f_{n}(x) = 
\begin{cases}
	n \ &\text{ if } 0 < x < \frac{ 1 }{ n } \\
	0 \ &\text{ if } x = 0 \ \text{ or } x \geq \frac{ 1 }{ n }
\end{cases} \]
as a counter-example. Notice that each \( f_{n} \) contains two discontinuities on \( [0,1] \) and is integrable with \( \int_{ 0 }^{ 1 } f_{n} = 1  \). For every \( x \in [0,1] \), note that \( \lim f_{n}(x) = 0  \) pointwise.  Then observe that the limit function \( 0  \) clearly integrates to \( 0 \). But this means that 
\[  \lim_{ n \to \infty  }  \int_{ a }^{ b } f_{n} \neq 0. \]
To fix this problem caused by pointwise convergence, we require the assumption of uniform convergence. 


\begin{tcolorbox}
	\begin{thm}[Integrable Limit Theorem]
	Assume that \( f_{n} \to f  \) uniformly on \( [a,b]  \) and that each \( f_{n}  \) integrable. Then, \( f \) is integrable and 
	\[  \lim_{ n \to \infty  }  \int_{ a }^{ b } f_{n} = \int_{ a }^{ b } f. \]
	\end{thm}
\end{tcolorbox}
\begin{proof} From exercise 7.2.5, we have proven that \( f  \) is integrable on \( [a,b] \). 
	Using part (v) of Theorem 7.4.2, we can make the following statement:
	\[ \Big| \int_{ a }^{ b } f_{n} - \int_{ a }^{ b } f  \Big| = \Big| \int_{ a }^{ b }  (f_{n} - f )  \Big| \leq \int_{ a }^{ b }  | f_{n} -f  |.   \]
	Since \( f_{n} \to f  \) uniformly on \( [a,b] \), we can let \( \epsilon > 0  \) such that there exists \( N \in \N \) implies 
	\[  | f_{n}(x) - f(x)  | < \frac{ \epsilon  }{ b -a  }  \ \text{ for all } n \geq N \ \text{and} \ x \in [a,b].\] Then observe that 
	\begin{align*}
	    \Big| \int_{ a }^{ b } f_{n} - \int_{ a }^{ b } f  \Big| &\leq \int_{ a }^{ b } | f_{n} - f  |   \\
																 &< \int_{ a }^{ b }  \frac{ \epsilon  }{ b -a  } \\
																 &= \epsilon.
	\end{align*}
	We conclude that 
	\[  \lim_{ n \to \infty  }  \int_{ a }^{ b } f_{n} = \int_{ a }^{ b } f. \]
\end{proof}


\subsection{Definitions and Theorems}


\begin{tcolorbox}
\begin{thm}
	Assume \( f: [a,b] \to \R   \) is bounded, and let \( c \in (a,b)  \). Then, \( f  \) is integrable on \( [a,b]  \) if and only if \( f  \) integrable on \( [a,c]  \) and \( [c,b]  \). In this case, we have 
	\[  \int_{ a }^{ b } f = \int_{ a }^{ c }  f  + \int_{ b }^{ c }  f. \]
\end{thm}
\end{tcolorbox}


\begin{tcolorbox}
\begin{thm}
	Assume \( f  \) and \( g  \)  are integrable functions on the interval \( [a,b]  \).
	\begin{enumerate}
		\item[(i)] The function \( f + g  \) is integrable on \( [a,b]  \) with \( \int_{ a }^{ b } (f + g) = \int_{ a }^{ b }  f + \int_{ a }^{ b } g  \).
		\item[(ii)] For \( k \in \R  \), the function \( kf  \) is integrable with \( \int_{ a }^{ b } kf  = k \int_{ a }^{ b }  f. \)
		\item[(iii)] If \( m \leq f(x) \leq M  \) on \( [a,b]  \), then \( m(b-a) \leq \int_{ a }^{ b } f \leq M(b-a) \).
		\item[(iv)] If \( f(x) \leq g(x)  \) on \( [a,b]  \), then \( \int_{ a }^{ b } f  \leq \int_{ a }^{ b } g  \).
		\item[(v)] The function \( | f |   \) is integrable and \( | \int_{ a }^{ b }  f | \leq \int_{ a }^{ b } | f |. \) 
	\end{enumerate}
\end{thm}
\end{tcolorbox}



\begin{tcolorbox}
\begin{defn}
	If \( f  \) is integrable on the interval \( [a,b]  \), define 
	\[  \int_{ a }^{ b } f = - \int_{ a }^{b  } f. \]
	Also, for \( c \in [a,b]  \) define 
	\[  \int_{ c }^{ c } f = 0. \]
\end{defn}
\end{tcolorbox}


\begin{tcolorbox}
	\begin{thm}[Integrable Limit Theorem]
	Assume that \( f_{n} \to f  \) uniformly on \( [a,b]  \) and that each \( f_{n}  \) integrable. Then, \( f \) is integrable and 
	\[  \lim_{ n \to \infty  }  \int_{ a }^{ b } f_{n} = \int_{ a }^{ b } f. \]
	\end{thm}
\end{tcolorbox}



\subsection{Exercises}





\subsubsection{Exercise 7.4.1} Let \( f  \) be a bounded function on a set \( A  \), and set 
\[  M = \sup \{ f(x) : x \in A  \},  \ m = \inf \{ f(x) : x \in A \}, \]
\[  M' = \sup \{ | f(x)  | : x \in A  \} \ \text{ and } \ m' = \inf \{ | f(x)  | : x \in A \}. \]

\begin{enumerate}
    \item[(a)] Show that \( M - m \geq M' - m' \).
		\begin{proof}
		Observe that \( f(x) \leq M' \leq M \) and likewise \( m \leq m'  \). Then we have  
		\[  -m \geq -m' \iff M' - m \geq M' - m'. \tag{1}\]
		Since \( M' \leq M \), (1) implies 
		\[  M - m \geq M' - m'. \]
		\end{proof}
	\item[(b)] Show that if \( f  \) is integrable on the interval \( [a,b]  \), then \( | f |  \) is also integrable on this interval.
		\begin{proof}
			Suppose \( f \) is integrable on \( [a,b] \). Let \( \epsilon > 0  \). Then there exists a partition \(  P_{\epsilon } \) such that 
			\[  U(f,P_{\epsilon } ) - L(f, P_{\epsilon }) = \sum_{ k=1 }^{ n } [M_{k } - m_{k }] \Delta x_{k} < \epsilon. \]
			By using part (a), we know that 
			\[  \sum_{ k=1 }^{ n } [M'_{k } - m'_{k }] \Delta x_{k } \leq \sum_{ k=1 }^{ n } [ M_{k } - m_{k } ] \Delta x_{k }. \]
			Then using the same partition \( P_{\epsilon } \) that we found, we have 
			\[ U(| f | , P_{\epsilon } ) - L(| f | , P_{\epsilon } ) \leq U(f , P_{\epsilon }) - L( f  , P_{\epsilon }) < \epsilon. \] Hence, we conclude that \( | f |  \) is an integrable function on \( [a,b]  \). 
		\end{proof}
	\item[(b)] Provide the details for the argument that in this case we have \( | \int_{ a }^{ b } f  | \leq \int_{ a }^{ b } | f | \).
		\begin{proof}
			Since \( | f |  \) integrable, we know that \( U(| f | ) = L(| f | ) = \int_{ a }^{ b } | f |    \). Likewise \( f \) being integrable implies \( U( f ) = L(f) = \int_{ a }^{ b } f   \). Let \( \epsilon > 0 \) and let \( P \) be a partition of \( [a,b]  \). Using the properties of the upper and lower integral, we must have 
		\begin{align*}
		    \Big| \int_{ a }^{ b } f  \Big| = | U(f)  |   
											&\leq | U(f,P)  | \\
											&< \Big| L(f,P) + \epsilon  \Big| \\  
											&\leq | L(f,P)  | + \epsilon \\
											&\leq L(| f | , P) + \epsilon \\
											&\leq \int_{ a }^{ b } | f | + \epsilon. 
		\end{align*}
		Since \( \epsilon > 0  \) is arbitrary, we must have \( | \int_{ a }^{ b } f  |  \leq \int_{ a }^{ b } | f |   \). 
		\end{proof}
\end{enumerate}


\subsubsection{Exercies 7.4.2}  
\begin{enumerate}
    \item[(a)] Let \( g(x) = x^{3}  \), and classify each of the following as positive, negative, or zero.
		\[  \text{ (i) } \int_{ 0 }^{ -1 } g + \int_{ 0 }^{ 1 } g \ \ \   \text{ (ii) } \int_{ 1 }^{ 0 } g + \int_{ 0 }^{ 1 }  g \ \ \ \text{ (iii) } \int_{ 1 }^{ -2 } g + \int_{ 0 }^{ 1 } g.  \]
		\begin{proof}[Solution]
			(i) zero, (ii) zero, (iii) positive 
		\end{proof}
	\item[(b)] Show that if \( b \leq a \leq c  \) and \( f  \) is integrable on the interval \( [b,c]  \), then it is still the case that \( \int_{ a }^{ b } f = \int_{ a }^{ c } f + \int_{ b }^{ c } f  \).
		\begin{proof}
			Since \( f  \) is integrable on the interval \( [b,c]  \), we have 
			\[  \int_{ b }^{ c } f   = \int_{ b }^{ a } f  + \int_{ a }^{ c } f. \tag{1} \]
			Rearranging (1), we have 
			\[  - \int_{ b }^{ a } f = \int_{ a }^{ c } f  - \int_{ b }^{ c } f.\] By using Definition 7.4.1, we know that 
			\[  - \int_{ b }^{ a } f = \int_{ a }^{ b } f  \ \text{ and }  -\int_{ b }^{ c } f = \int_{ c }^{ b } f.  \]
			Hence, we conclude that 
			\[ \int_{ a }^{ b } f = \int_{ a }^{ c } f + \int_{ b }^{ c } f  \]

		\end{proof}
\end{enumerate}

\subsubsection{Exercise 7.4.3} Decide which of the following conjectures is true and supply a short proof. For those that are not true, give a counter-example.
\begin{enumerate}
	\item[(a)] If \( | f |  \) is integrable on \( [a,b]  \), then \( f  \) is also integrable on this set. 
		\begin{proof}[Solution]
		Define 
		\[  f(x) = 
		\begin{cases}
			1  \ \text{ for } x \in \Q \\
			-1 \ \text{ for } x \notin \Q.
		\end{cases} \]
		Notice that \( | f | \) is integrable, but not \( f \). 
		\end{proof}
	\item[(b)] Assume \( g  \) is integrable and \( g(x) \geq 0  \) on \( [a,b]  \). If \( g(x) > 0  \) for an infinite number of points \( x \in [a,b]  \), then \( \int_{ a }^{ b } g > 0  \).
		\begin{proof}[Solution]
		We can use Thomae's function in the last section 
		\[  g(x) = 
		\begin{cases}
			1 \ &\text{ if } x = 0 \\ 
			1 / n \ &\text{ if } x = m/n \in \Q \setminus \{ 0 \} \ \text{ is in lowest terms with } n > 0 \\
			0 \ &\text{ if } x \notin \Q.
		\end{cases} \]
		We see that \( g(x) \geq 0  \)  and \( g(x) > 0  \) for an infinite number of points \( x \in [a,b]  \), but \( \int_{ a }^{ b } g = 0  \). 
		\end{proof}
	\item[(c)] If \( g  \) is continuous on \( [a,b]  \) and \( g(x) \geq 0  \) with \( g(y_{0}) > 0  \) for at least one point \( y_{0} \in [a,b]  \), then \( \int_{ a }^{ b } g > 0. \)
		\begin{proof}
			Since \( g  \) is continuous on \( [a,b]  \) and \( g(x) \geq 0  \), we know that \( g  \) must be integrable on \( [a,b]  \). Furthermore, \( g  \) reaches its maximum and minimum on \( [a,b]  \) since \( [a,b]  \) is a compact interval. Hence, there exists at least one point \( y_{0} \) such that \( g(y_{0}) \) is the minimum of \( g  \) on \( [a,b] \). Since \( g  \) is integrable on \( [a,b]  \), we have that 
			\begin{align*}
				\int_{ a }^{ b } g  = L(g) &\geq L(g,P) \\
										   &= \sum_{ k=1 }^{ n } m_{k } \Delta x_{k } \\
											&\geq \sum_{ k=1 }^{ n } g(y_{0}) \Delta x_{k } \\ 
											&> 0. 
			\end{align*}
			Hence, we conclude that \( \int_{ a }^{ b } g > 0 \).
		\end{proof}
\end{enumerate}



\subsubsection{Exercise 7.4.4} Show that if \( f(x) > 0  \) for all \( x \in [a,b]  \) and \( f  \) is integrable, then \( \int_{ a }^{ b } f > 0 . \)
\begin{proof}
	Let \( x \in [a,b]  \). Since \( f \) is integrable on \( [a,b]  \), we have \( U(f) = L(f) = \int_{ a }^{ b } f   \). Let \( P \) be an arbitrary partition of \( [a,b]  \). Since \( f(x) > 0  \), we have
	\begin{align*}
		\int_{ a }^{ b } f = L(f) \geq L(f, P)  
								  = \sum_{ k=1 }^{ n } m_{k } \Delta x_{k }  > 0 \\
	\end{align*}
	Hence, \( \int_{ a }^{ b } f > 0  \).
\end{proof}

\subsubsection{Exercise 7.4.5} Let \( f  \) and \( g  \) be integrable functions on \( [a,b] \). 
\begin{enumerate}
	\item[(a)] Show that if \( P  \) is any partition of \( [a,b]  \), then 
		\[  U(f+g, P ) \leq U(f,P) + U(g,P) \tag{1}. \]
		Provide a specific example where the inequality is strict. What does the corresponding inequality for lower sums look like? 

		\begin{proof}
		In exercise 1.3.6, we proved for any two sets \( A,B \neq \emptyset  \), we have 
		\[  \sup(A + B) \leq \sup A + \sup B. \]
		In this context, we have 
		\[ A =  \{ f(x) : x \in [x_{k-1}, x_{k}] \} \ \text{and} \ B =  \{ g(x) : x \in [x_{k-1}, x_{k }] \} \]
		with 
		\[  A + B = \{ f(x) + g(x) : x \in [x_{k-1} , x_{k }] \} . \]
		Let \( P  \) be any partition of \( [a,b]  \). Then observe that 
		\begin{align*}
			U(f+g, P) &=\sum_{ k=1 }^{ n } \sup_{x \in [x_{k-1}, x_{k }]}(f+g)(x) \Delta x_{k } \\
					  &\leq \sum_{ k=1 }^{ n } \sup_{x \in [x_{k-1}, x_{k }]}f(x) \Delta x_{k}  + \sum_{ k=1 }^{ n } \sup_{x \in [x_{k-1}, x_{k}]} g(x) \Delta x_{k } \\  
					  &= U(f,P) + U(g,P).
		\end{align*}
		We have strict inequality whenever \( f(x) > g(x)  \) and the corresponding inequality to (1) is 
		\[  L(f+g,P) \geq L(f, P) + L(g, P). \]
		\end{proof}
	\item[(b)] Review the proof of Theorem 7.4.2 (ii), and provide an argument for part (i) of this theorem.
		\begin{proof}
			See part (i) of Theorem 7.4.2 in the section notes.
		\end{proof}
\end{enumerate}
\subsubsection{Exercise 7.4.6} Although not part of Theorem 7.4.2, it is true that the porduct of integrable functions is integrable. Provide the details for each step in the following proof of this fact:
\begin{enumerate}
	\item[(a)] If \( f  \) satisfies \( | f(x)  | \leq M  \) on \( [a,b]  \), show 
		\[  | (f(x))^{2} + (f(y))^{2} |  \leq 2M | f(x) - f(y) |. \]
		\begin{proof}
			Let \( x,y \in [a,b]  \). Since \( f  \) satisfies \( | f(x)  | \leq M  \) and \( | f(y) | \leq M  \), we have 
			\[  | f(x) + f(y)  | \leq | f(x)  | + | f(y)  | \leq 2M. \]
	Multiplying the inequality above by \( | f(x) - f(y)  |  \) yields 
	\[  | (f(x))^{2} - (f(y))^{2} | \leq 2M | f(x) - f(y) |. \]
		\end{proof}
	\item[(b)] Prove that if \( f  \) is integrable on \( [a,b]  \), then so is \( f^{2} \). 
		\begin{proof}
			Let \( \epsilon > 0 \). Since \( f  \) is integrable on \( [a,b]  \), there exists a partition \( P_{\epsilon } \) of \( [a,b] \) such that 
			\[ U(f, P_{\epsilon }) - L(f, P_{\epsilon }) < \frac{ \epsilon  }{ 2M }. \]
			Then using the same partition \( P_{\epsilon } \) and using part (a), we have that  
			\begin{align*}
				U(f^{2}, P_{\epsilon }) - L(f^{2}, P_{\epsilon }) &= \sum_{ k=1 }^{ n } [ (M'_{k})^{2} - (m'_{k })^{2}] \Delta x_{k } \\
																  &\leq 2M \sum_{ k=1 }^{ n } [M'_{k } - m'_{k}] \Delta x_{k } \\
																  &< 2M \cdot \frac{ \epsilon  }{ 2M } = \epsilon. 
			\end{align*}
			Hence, \( f^{2} \) is also integrable on \( [a,b]  \).
		\end{proof}
	\item[(c)] Now show that if \( f  \) and \( g \) are integrable, then \( fg \) is integrable. (Consider \( (f+g)^{2} \)).
		\begin{proof}
		Observe that 
		\[  (f+g)^{2} = f^{2} + 2fg + g^{2} \]
		and solving for \( fg \) yields 
		\[  fg = \frac{ 1 }{ 2 }  [ (f+g)^{2} - (f^2 + g^{2})]. \]
		Since \( f  \) and \( g  \) are integrable, we know by part (i) of Theorem 7.4.2 that \( f + g  \) is integrable.  Let \( h = f +g   \). Then by part (b), we have that \( h^{2} \) is integrable as well as \( f^{2} \) and \( g^{2} \). By using part (i) and (ii) of Theorem 7.4.2, we find that \( fg \) is integrable. 
		\end{proof}
\end{enumerate}

\subsubsection{Exercise 7.4.8} For each \( n \in \N  \), let 
\[  h_{n}(x) = 
\begin{cases}
	1 / 2^{n} \ &\text{ if } 1/2^{n} < x \leq 1 \\
	0 \ &\text{ if } 0 \leq x \leq 1 / 2^{n},
\end{cases} \]
and set \( H(x) = \sum_{ n=1 }^{ \infty  } h_{n}(x)  \). Show that \( H  \) is integrable and compute \( \int_{ 0 }^{ 1 } H  \).
\begin{proof}
	Notice that \( h_{n}  \) is a sequence of continuous functions on the set \( [0,1] \setminus \{  1 / 2^{n}\}  \) and that \( \sum_{ n=1 }^{ \infty  } h_{n} = 0  \) uniformly.  By the Integrability Limit Theorem, we conclude that \( H  \) must be integrale and that \( \int_{ 0 }^{ 1 } H(x) = 0  \). 
\end{proof}

\subsubsection{Exercise 7.4.9} Let \( g_{n}  \) and \( g  \) be uniformly bounded on \( [0,1]  \), meaning that there exists a single \( M > 0  \) satisfying \( | g(x) | \leq M  \) and \( | g_{n}(x)  | \leq M  \) for all \( n \in \N  \) and \( x \in [0,1] \). Assume \( g_{n} \to g  \) pointwise on \( [0,1]  \) and uniformly on any set of the form \( [0, \alpha]  \), where \( 0 < \alpha < 1  \). 

If all the functions are integrable, show that \( \lim_{ n \to \infty  }  \int_{ 0 }^{ 1 } g_{n} = \int_{ 0 }^{ 1 } g  \).
\begin{proof}
Let \( \epsilon > 0  \). Our goal is to show  
\[  \lim_{ n \to \infty  } \int_{ 0 }^{ 1 } g_{n} = \int_{ 0 }^{ 1 }  g \]
which can be done by finding an \( N \in \N  \) such that for any \(  n \geq N  \), we have 
\[  \Big| \int_{ 0 }^{ 1 } g_{n} - \int_{ 0 }^{ 1 }  g \Big| < \epsilon. \]
Suppose \( g_{n} \to g  \) pointwise on \( [0,1]  \) and uniformly on any set of the form \( [0, \alpha] \), where \( 0 < \alpha < 1  \). Looking at \( g  \) on \( [0, \alpha] \), we know there exists an \(  N \in \N  \) such that for any \( n \geq N  \) and \( x \in [0, \alpha ] \), we haThen we can pick \( x_{1} \) such that ve 
\[  | g_{n} - g  | < \frac{ \epsilon  }{  2 \alpha }.  \] Utilizing the integrability of both \( g_{n} \) and \( g  \) on \( [0,1] \), we can also state that \( g_{n} - g  \) is integrable and hence, \( | g_{n} -g  |  \). We can use the triangle inequality and Theorem 7.4.1, to write the following:
\begin{align*}
    \Big| \int_{ 0 }^{ 1 } g_{n} - \int_{ 0 }^{ 1 } g  \Big| &= \Big| \int_{ 0 }^{ 1 } g_{n} - g   \Big|  \\
															 &\leq \int_{ 0 }^{ 1 } | g_{n} - g  |  \\
															 &= \int_{ 0 }^{ \alpha } | g_{n} - g |  + \int_{ \alpha  }^{ 1 }  | g_{n} - g  |. \\ 
\end{align*}
We can easily make the first term small by finding \( N_{1} \in \N  \) such that for any \( n \geq N_1 \), we have 
\[  \int_{ 0 }^{ \alpha  } | g_{n} -g  |  < \frac{ \epsilon  }{ 2 \alpha  } \cdot \alpha = \frac{ \epsilon  }{ 2 } . \] To make the second term small, we can utilize the integrability of \( | g_{n} -g  |  \) on \( [0,1] \) to state that for any partition \( P \) of \( [\alpha,1]  \), we have 
\begin{align*}
    \int_{ \alpha  }^{ 1 } | g_{n} -g  |  = U(| g_{n} - g  | ) 
										  &\leq U( | g_{n} -g  |, P) \\
										  &= \sum_{ k=1 }^{ n } \sup_{x \in [x_{k-1}, x_k]} | g_{n} - g  | \cdot \Delta x_{k }.\\
\end{align*}
Since \( g_{n}  \) and \( g  \) are uniformly bounded by a single \( M > 0  \) satisying \( | g(x)  |  \leq M  \) and \( | g_{n}(x) | \leq M  \), we know that 
\[  \sup_{x \in [x_{k-1}, x_{k }]} | g_{n} - g  | \leq 2M.\] Furthermore, utilizing pointwise convergence of \( g_{n} \to g  \) on \( [\alpha, 1 ] \), we can say there exists \( N_{2} \in \N  \) such that for any \( n \geq N_{2} \)
\[ \sup_{x \in [\alpha, x_{1}]} | g_{1} - g  | < \frac{ \epsilon  }{ 4 (x_{1} - \alpha) } .  \]
Then we have 
\begin{align*}
	\int_{ \alpha }^{ 1 }  | g_{n} -g  |  &\leq \sup_{x \in [\alpha, x_{1}]} | g_{1} - g  | (x_{1} - \alpha) + \sum_{ k=1 }^{ n } \sup_{x\in [x_{k-1},x_{k }]} | g_{n} -g  | \Delta x_{k }  \\
										  &< \frac{ \epsilon  }{ 4 ( x_{1} - \alpha) } \cdot (x_{1} - \alpha) + 2M \cdot \frac{ \epsilon  }{ 8M  } \\ 
										  &= \frac{ \epsilon  }{ 2 }.
\end{align*}
Letting \( N = \max \{ N_{1}, N_{2} \}  \), assuming that \( n \geq N   \), we have that 
\begin{align*}
    \Big| \int_{ 0 }^{ 1 } g_{n} - \int_{ 0 }^{ 1 } g  \Big| &= \Big| \int_{ 0 }^{ 1 } g_{n} - g   \Big|  \\
															 &\leq \int_{ 0 }^{ 1 } | g_{n} - g  |  \\
															 &= \int_{ 0 }^{ \alpha } | g_{n} - g |  + \int_{ \alpha  }^{ 1 }  | g_{n} - g  | \\ 
															 &< \frac{ \epsilon  }{ 2 \alpha } \cdot \alpha + \frac{ \epsilon  }{ 2  } \\
															 &= \epsilon.
\end{align*}
Hence, we conclude that
\[  \lim_{ n \to \infty  }  \int_{ 0 }^{ 1 } g  = \int_{ 0 }^{ 1 } g.  \]
\end{proof}


\subsubsection{Exercise 7.4.10} Assume \( g  \) is integrable on \( [0,1]  \) and continuous at \( 0  \). Show 
\[  \lim_{ n \to \infty  } \int_{ 0 }^{ 1 }  g(x^{n}) \ dx = g(0). \]
\begin{proof}
\end{proof}





\end{document}



