\section{Linear Transformations, Null Spaces, and Ranges}

Suppose we have a function \( T  \) with domain \( V  \) and codomain \( W  \) denoted by \( T: V \to W  \).


\begin{definition}[Linear Transformation]\label{Linear Transformation}
   Let \( V  \) and \( W  \) be vector spaces (over \( F  \)). We call a function \( T: V \to W  \) a \textbf{linear transformation from \( V  \) to \( W  \)}, for all \( x,y \in V  \) and \( c \in F  \), we have
   \begin{enumerate}
       \item[(a)] \( T(x+y) = T(x) + T(y) \) and
        \item[(b)] \( T(cx) = c T(x) \).
   \end{enumerate}
\end{definition}

\begin{remark}
    If \( F = \Q   \), then (a) implies (b) in the definition above. Otherwise, (a) and (b) are logically independent statements.  
\end{remark}
The following are a list of properties for linear functions:
\begin{enumerate}
    \item If \( T  \) is linear, then \( T(0) = 0  \).
        \begin{proof}
        Suppose \( T  \) is linear, then \( T(0 \cdot 0 ) = 0 T(0) = 0  \).
        \end{proof}
    \item We have \( T  \) is linear if and only if \( T(cx+y) = cT(x) + T(y)  \) for all \( x,y \in V  \) and \( c \in F  \).
        \begin{proof}
        Suppose \( T  \) is linear. Let \( x,y \in V  \) and \( c \in F  \). Then 
        \[  T(cx+y) = T(cx) + T(y) = cT(x) + T(y). \]
        Conversely, if \( c = 1  \) then 
        \[  T(x + y) = T(x) + T(y). \]
        If  \( y = 0  \), then 
        \[ T(cx) = cT(x).  \]
        Hence, \( T  \) is a linear transformation.
        \end{proof}
    \item If \( T  \) is linear, then \( T(x-y) = T(x) - T(y)  \) for all \( x,y \in V  \).
        \begin{proof}
        Let \( x,y \in V  \). Suppose \( T  \) is linear, then 
        \[ T(x - y) = T(x) + T(-y) = T(x) - T(y).  \]
        \end{proof}
    \item \( T  \) is linear if and only if, for \( x_{1}, x_{2} , \dots, x_{n} \in V  \) and \( a_{1}, a_{2}, \dots, a_{n} \in F  \), 
        we have
        \[  T \Big(  \sum_{ i=1 }^{ n } a_{i} x_{i}  \Big) = \sum_{ i=1 }^{ n } a_{i} T(x_{i}). \]
        \begin{proof}
        Suppose \( T  \) is linear. Let \( x_{1}, x_{2}, \dots, x_{n} \in V  \) and \( a_{1}, a_{2}, \dots, a_{n} \in F  \) such that
        \[  v = \sum_{ i=1 }^{ n } a_{i} x_{i}. \]
        Then observe that 
        \[
            T \Big( \sum_{ i=1 }^{ n } a_{i} x_{i} \Big) = \sum_{ i=1 }^{ n } T( a_{i} x_{i}) 
                                                         = \sum_{ i=1 }^{ n } a_{i} T(x_{i}).
        \]
        Conversely, for \( i = 2  \) and denote \( a_{i} x_{i} = y_{i} \). Then
        \[  T\Big(\sum_{ i=1 }^{ 2 } a_{i} x_{i} \Big) = T(a_{1} x_{1}) + T(a_{2} x_{2}) = T(y_{1}) + T(y_{2})  \]  
        and so property 1 is satisfied. 
        If \( i = 1  \), then 
        \[  T \Big( \sum_{ i=1 }^{ 1 } a_{i}x_{i}  \Big) = a_{1} T(x_{1}) \]
        and so property 2 is satisfied. Hence, \( T  \) is a linear transformation.
        \end{proof}
\end{enumerate}

\begin{eg}
   Define  
   \begin{center}
       \( T: \R^{2} \to \R^{2}  \) by \( T(a_{1}, a_{2}) = (2a_{1} + a_{2}, a_{1}) \).
   \end{center}
   Show that \( T  \) is linear. Let \( c \in \R  \) and \( x,y \in \R^{2} \), where \( x = (a_{1}, a_{2}) \) and \( y = (b_{1}, b_{2}) \). Then
   \begin{align*}
       cx + y &= c(a_{1}, a_{2}) + (b_{1}, b_{2})  \\
              &=  (ca_{1}, ca_{2} ) + (b_{1}, b_{2}) \\
              &=  (ca_{1} + b_{1} , ca_{2} + b_{2}).
   \end{align*}
   So, we have
   \begin{align*}
                                       T(cx + y)  &= T(ca_{1} + b_{1} , ca_{2} + b_{2})  \\
                                         &= (2(ca_{1} + b_{1})  + ca_{2} + b_{2}, ca_{1} + b_{1} ) \\
                                          &= ((2ca_{1} + ca_{2}) +  (2b_{1} + b_{2}), ca_{1} + b_{1}  ) \\
                                          &= (2ca_{1} + ca_{2}, ca_{1}) + (2b_{1} + b_{2}, b_{1} ) \\
                                          &= c(2a_{1} + a_{2}, a_{1}) + (2b_{1} + b_{2}, b_{1}) \\
                                          &= cT(a_{1}, a_{2}) + T(b_{1}, b_{2}) \\
                                          &= cT(x) + T(y).
   \end{align*}
   Hence, we have that \( T(cx + y) = cT(x) + T(y) \) so \( T: \R^{2} \to \R^{2}   \) is a linear.
\end{eg}

