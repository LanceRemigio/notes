\section{The Matrix Representation of a Linear Transformation}

In this section, we will develop a one-to-one correspondence between matrices and linear transformations so that we may study the properties of one utilizing the properties of another. Before we do this, we shall develop the concept of an \textit{ordered basis} for a vector space \( V  \).

\subsection{Ordered Bases}

\begin{definition}[Ordered Bases]
    Let \( V  \) be a finite-dimensional vector space. An \textbf{ordered basis} for \( V  \) is a basis for \( V  \) endowed with a specific order; that is, an ordered basis for \( V  \) is a finite sequence of linearly independent vectors in \( V  \) that generates \( V  \).
\end{definition}

\begin{eg}
    In \( F^{3} \), let \( \beta = \{ e_{1}, e_{2}, e_{3} \}  \) be an ordered basis for \( F^{3} \). If we take another set, say, \( \gamma = \{ e_{2}, e_{1}, e_{3} \}  \), be a basis for \( F^{3} \), we will see that even though these two bases are equal in terms of the vectors within it, we still end up with different ORDERED bases.
\end{eg}

\begin{itemize}
   \item Note that \( e_{i}  \) for all \( 1 \leq  i \leq n  \) are the \textbf{standard basis vectors} for \( F^{n} \). The set \( \{  e_{1}, e_{2}, \dots, e_{n} \}  \) in \( F^{n} \) is the \textbf{standard ordered basis} for \( F^{n} \). Likewise, \( \{ 1,x,\dots, x^{n} \}  \) is the \textbf{standard ordered basis} for \( P_{n}(F) \).
    \item We can now identify vectors in some finite-dimensional vector space of dimension \( n  \) by using \( n- \)tuples.
    \item This is done through what is called \textit{coordinate vectors}.
\end{itemize}

\begin{definition}[Coordinate Vectors]
    Let \( \beta = \{ u_{1}, u_{2}, \dots, u_{n} \}  \) be an ordered basis for a finite-dimensional vector space \( V  \). For \( x \in V  \), let \( a_{1}, a_{2}, \dots, a_{n} \) be the unique scalars such that 
    \[  x = \sum_{ i=1 }^{ n } a_{i} u_{i}. \]
    We define the \textbf{coordinate vector of \( x \) relative to \( \beta \)}, denoted \( [x]_{\beta} \), by
    \[ [x]_{\beta} = \begin{pmatrix}
        a_{1} \\
        a_{2} \\
        \vdots \\
        a_{n}
    \end{pmatrix}. \]
\end{definition}
\begin{itemize}
    \item In our definition of standard basis vectors, we see that \( [u_{i}]_{\beta} = e_{i}  \).
    \item It is quite easy to show that \( x \to [x]_{\beta} \) provides us with a linear transformation from \( T: V \to F^{n} \).
\end{itemize}

\begin{eg}
    Let \( V = {P}_{2}(\R) \) and let \( \beta  = \{ 1 , x , x^{2} \}  \) be the standard ordered basis for \( V  \). If \( f(x) = 4 + 6x - 7 x^{2} \), then 
    \[  {[f]}_{\beta} = \begin{pmatrix}
        4 \\
        6 \\
        -7
    \end{pmatrix}. \]
\end{eg}
    
Suppose that \( V  \) and \( W  \) are finite-dimensional vector spaces with ordered bases 
    \( \beta = \{  {v}_{1}, {v}_{2}, \dots , {v}_{n} \}  \) and \( \gamma = \{ {w}_{1}, {w}_{2}, \dots, {w}_{m} \} \), respectively. Let \( T : V \to W  \) be linear. Then for each \( j  \), \( 1 \leq j \leq n  \), there exists unique scalars ({\hyperref[Unique combinations from bases]{Theorem 1.5.1}}) \( {a}_{ij} \in F  \), \( 1 \leq i \leq m  \) such that
    \[  T({v}_{j}) = \sum_{ i=1 }^{ m } {a}_{ij} {w}_{i} \ \ \text{for} \ \ 1 \leq j \leq n. \]

\subsection{Matrix Representations}

\begin{definition}[Matrix Representation of \( T  \)]
    Using the notation above, we call the \( m \times n  \) matrix \( A  \) defined by \( {A}_{ij} = {a}_{ij} \), the \textbf{matrix representation of \( T  \) in the ordered bases \( \beta  \) and \( \gamma \) } and write \( A = {[T]}_{\beta}^{\gamma}  \). If \( V = W  \) and \( \beta = \gamma  \), then we write \( A = [T]_{\beta}^{}  \).
\end{definition}

\begin{itemize}
    \item The \( j \)th column of \( A  \) is just \( {[T({v}_{j})]}_{\gamma} \). 
    \item If \( U: V \to W  \) is linear such that \( {[U]}_{\beta}^{\gamma} = [T]_{\beta}^{\gamma}  \), then \( U = T  \) by {\hyperref[Theorem 2.6]{corollary to Theorem 2.6.}}.
\end{itemize}

\subsection{Examples of Computing Matrix Representations}

\begin{eg}
   Let \( T: \R^{2} \to \R^{3}  \) be the linear transformation defined by  
   \[  T({a}_{1}, {a}_{2}) = ({a}_{1} + {3a}_{2}, 0 , {2a}_{1} - {4a}_{2}) \] 
   Let \(  \beta  \) and \( \gamma  \) be the standard ordered bases for \( \R^{2}  \) and \( \R^{3}  \), respectively. Now, 
   \[  T(1,0) = (1,0,2) = {1e}_{1} + {0e}_{2} + {2e}_{3}  \]
    and 
    \[  T(0,1) = (3,0,-4) = {3e}_{1} + {0e}_{2} - {4e}_{3}. \]
    Hence, we have
    \[  [T]_{\beta}^{\gamma}  = \begin{pmatrix}
        1 & 3 \\
        0 & 0 \\
        2 & -4
    \end{pmatrix}. \]
    Suppose we re-ordered our basis \( \gamma \) to be the basis \( \gamma' = \{ {e}_{3} , {e}_{2}, {e}_{1} \}  \). Then we will see that
    \[  [T]_{\beta}^{\gamma'} = \begin{pmatrix}
        2 & -4 \\ 
        0 & 0 \\ 
        1 & 3 
    \end{pmatrix}.  \]
\end{eg}

The example serves to demonstrate how different orderings of bases can lead to different matrix representations.

\begin{eg}\label{Example 4 of Section 2.2}
    Let \( T: {P}_{3}(\R) \to {P}_{2}(\R) \) be the linear transformation defined by \( T(f(x)) = f'(x) \). Let \( \beta \) and \( \gamma  \) be the standard ordered bases for \( {P}_{3}(\R) \) and \( {P}_{2}(\R) \), respectively. Since \( \beta = \{ 1, x , x^{2}, x^{3} \}  \) is a basis for \( {P}_{3}(\R) \) and \( \gamma = \{ 1,x,x^{2} \}  \) for \( {P}_{2}(\R) \) , observe that
    \begin{align*}
        T(1) &= 0 \cdot 1 + 0 \cdot x + 0 \cdot x^{2} \\
        T(x) &= 1 \cdot 1 + 0 \cdot x + 0 \cdot x^{2} \\
        T(x^{2}) &= 0 \cdot 1 + 2 \cdot x + 0 \cdot x^{2} \\
        T(x^{3}) &= 0 \cdot 1 + 0 \cdot x + 2 \cdot x^{2}.
    \end{align*}
    So, 
    \[  [T]_{\beta}^{\gamma}  = \begin{pmatrix}
        0 & 1 & 0 & 0 \\
        0 & 0 & 2 & 0 \\
        0 & 0 & 0 & 3
    \end{pmatrix}. \]
    Observe that writing \( T(x^{j}) \) is written as a linear combination of the vectors of \( \gamma  \), its coefficients give the entries of the \( j \)th column of \( {[T]}_{\beta}^{\gamma} \).
\end{eg}

Later on, we will prove a theorem about how associating matrices with linear transformations leads to preservation of addition and scalar multiplication. 

\subsection{Addition and Scalar Multiplication of Matrix Representations}

\begin{definition}
   Let \( T, U: V \to W  \) be arbitrary functions, where \( V  \) and \( W  \) are vector spaces over \( F  \), and let \( a \in F  \). We define \( T + U : V \to W  \) by \( (T+U)(x) =  T(x) + U(x)   \) for all \( x \in V  \), and \( aT: V \to W  \) by \( (aT)(x) = aT(x) \) for all \( x \in V  \). 
\end{definition}

These operations of addition and scalar multiplication lead to the preservation of linear transformations.

\begin{theorem}[The Set of Linear Maps Over \( V  \) is a Vector Space]
   Let \( V  \) and \( W  \) be vector spaces over a field \( F  \), and let \( T,U: V \to W  \) be linear. Then we have
   \begin{enumerate}
       \item[(a)] For all \( a \in F  \), \( aT + U  \) is linear.
       \item[(b)] Using the operations of addition and scalar multiplication in the preceding definition, the collection of all linear transformations from \( V  \) to \( W  \) is a vector space over \( F  \).
   \end{enumerate}
\end{theorem}
\begin{proof}
    \begin{enumerate}
        \item[(a)] To show linearity of \( aT + U  \),  let \( x,y \in V  \) and \( a \in F  \). Then the linearity of \( T  \) and \( U  \) implies that 
    \begin{align*}
        (aT + U)(x+y) &= (aT)(x+y) + U(x+y)  \\
                      &= (aT)(x) + (aT)(y)  + U(x) + U(y) \\
                      &= (aT)(x) + U(x) + (aT)(y) + U(y) \\
                      &= (aT+U)(x) + (aT +  U)(y). 
    \end{align*}
    Let \( c \in F  \). Then 
    \begin{align*}
        (c(aT+U))(x) &= (caT + cU)(x) \\
                     &= (caT)(x) + (cU)(x) \\
                     &= c((aT)(x) + cU(x) \\
                     &= c ((aT)(x) + U(x)) \\
                     &= c (aT + U)(x).
    \end{align*}
    Hence, we can see that \( T  \) is linear.
\item[(b)] Let \( S  \) denote the collection of all linear transformations from \( V  \) to \( W  \). Note that the zero transformation \( {T}_{0} \) is in \( S  \). The vector axioms follow from \( S  \) being closed under addition and scalar multiplication; that is, 
    \begin{center}
       \( U + T \in S  \) and \( cT \in S  \). 
    \end{center}
    Hence, \( S  \) is a vector space over a field \( F  \).
    \end{enumerate}
\end{proof}

\begin{definition}
    Let \( V  \) and \( W  \) be vector spaces over \( F  \). We denote the vector space of all linear transformations from \( V  \) into \( W   \) by \( \mathcal{L}(V,W)  \). In the case that \( V = W  \), we write \( \mathcal{L}(V)  \) instead of \( \mathcal{L}(V,W) \). 
\end{definition}

\begin{theorem}[Addition and Scalar Mulitplication of Matrix Representations]
   Let \( V  \) and \( W \) be finite-dimensional vector spaces with ordered bases \( \beta  \) and \( \gamma \), respectively, and let \( T,U : V \to W  \) be linear transformations. Then 
   \begin{enumerate}
       \item[(a)] \( [T+U]_{\beta}^{\gamma}  = [T]_{\beta}^{\gamma} + [U]_{\beta}^{\gamma}  \) and
        \item[(b)]\( [aT]_{\beta}^{\gamma}  = a [T]_{\beta}^{\gamma}  \) for all scalars \( a  \).
   \end{enumerate}
\end{theorem}
\begin{proof}
    \begin{enumerate}
        \item[(a)] 
    Let \( \beta = \{ {v}_{1}, {v}_{2}, \dots, {v}_{m} \}  \) and \( \gamma = \{ {w}_{1}, {w}_{2}, \dots, {w}_{n} \}  \). There exists unique scalars \( {a}_{ij}  \) and \( {b}_{ij}  \) with \( 1 \leq i \leq m  \) and \( 1 \leq j \leq n  \) such that
    \[  T({v}_{j}) = \sum_{ i=1 }^{ m } {a}_{ij} {w}_{i} \ \text{ and } \ U({v}_{j}) = \sum_{ i=1 }^{ m } {b}_{ij} {w}_{i} \ \text{ for } 1 \leq j \leq n.   \]
    Observe that 
    \begin{align*}
        (T + U)({v}_{j}) &= \sum_{ i=1 }^{ m } ({a}_{ij} + {b}_{ij}) {w}_{i} \\
        &= \sum_{ i=1 }^{ m } {a}_{ij} {w}_{i} + \sum_{ i=1 }^{ m } {b}_{ij} {w}_{i} \\
        &= T({v}_{j}) + U({v}_{j}).
    \end{align*}
    Hence, we can write that 
    \[ {([T+U]_{\beta}^{\gamma} )}_{ij} = {a}_{ij} + {b}_{ij} = {([T]_{\beta}^{\gamma} + [U]_{\beta}^{\gamma} )}_{ij}   \]

\item[(b)] Using the same bases as above, the matrix representation of \( [aT]_{\beta}^{\gamma}  \) is written as 
    \[  (aT)({v}_{j}) = \sum_{ i=1 }^{ n } ({ca}_{ij}) {w}_{i} = c \sum_{ i=1 }^{ n } {a}_{ij} {w}_{i} = cT({v}_{j}) \ \text{ for } \ 1 \leq j \leq n. \]
    Hence, we can write
    \[  ({[aT]_{\beta}^{\gamma} })_{ij} = a {([T]_{\beta}^{\gamma} )}_{ij}. \]
    \end{enumerate}
\end{proof}

\begin{eg}
    Let \( T: \R^{2} \to \R^{3}  \) and \( U: \R^{2} \to \R^{3} \) be the linear transformations respectively defined by 
    \begin{center}
        \( T({a}_{1},{a}_{2}) = ({a}_{1} + {3a}_{2}, {2a}_{1} - {4a}_{2}) \) and \( U({a}_{1}, {a}_{2}) = ({a}_{1} - {a}_{2}, {2a}_{1}, {3a}_{1} + {2a}_{2}) \).
    \end{center}
    Let \( \beta \) and \( \gamma \) be the standard ordered bases of \( \R^{2}  \) and \( \R^{3}  \), respectively. Computing \( [T]_{\beta}^{\gamma}  \), we get
    \[ [T]_{\beta}^{\gamma}  = \begin{pmatrix}
        1 & 3 \\
        0 & 0 \\
        2 & -4 
    \end{pmatrix},  \]
    and 
    \[ [U]_{\beta}^{\gamma}  = \begin{pmatrix}
        1 & -1 \\
        2 & 0 \\
        3 & 2
    \end{pmatrix}.  \]
    Since \( T + U  \) is linear, we can write 
    \[  (T+U)({a}_{1}, {a}_{2}) = ({2a}_{1} + {2a}_{2} , {2a}_{1}, {5a}_{1} - {2a}_{2}) \]
    and so we can write
    \[  [T+U]_{\beta}^{\gamma}  = \begin{pmatrix}
        2 & 2 \\
        2 & 0 \\
        5 & -2 
    \end{pmatrix} \]
    using Theorem 2.8.
\end{eg}

