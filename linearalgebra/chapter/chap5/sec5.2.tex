\section{Diagonalizability}
Our goals in this section is to: 
\begin{itemize}
    \item Create a simple test to determine whether an operator or a matrix can be diagonalized.
    \item Develop a method for finding a basis of eigenvectors.
\end{itemize}

The next theorem that any constructed set that consists of eigenvectors is linearly independent.

\begin{theorem}\label{Theorem 5.5}
    Let \( T  \) be a linear operator on a vector space \( V  \), and let \( {\lambda}_{1}, {\lambda}_{2}, \dots, {\lambda}_{k} \) be distinct eigenvalues of \( T  \). If \( {v}_{1}, {v}_{2}, \dots, {v}_{k} \) are eigenvectors of \( T  \) such that \( {\lambda}_{i} \) corresponds to \( {v}_{i} \) (\( 1 \leq i \leq k  \)), then \( \{ {v}_{1}, {v}_{2}, \dots, {v}_{k } \}  \) is linearly independent.
\end{theorem}
\begin{proof}
We proceed via mathematical induction on \( k  \). Suppose that \( k = 1  \). Let \( \lambda_1  \) be an eigenvalue corresponding to \( {v}_{1} \). Since \( {v}_{1} \neq 0  \), we have that \( \{ {v}_{1} \}  \) is linearly independent. Now, assume that the theorem holds for \( k - 1  \) case. Note that \( k - 1 \geq 1  \). Our goal is to show that for some scalars \( {a}_{1}, {a}_{2}, \dots, {a}_{k}  \), we have  
\[  {a}_{1} {v}_{1} + {a}_{2} {v}_{2} + \cdots + {a}_{k} {v}_{k} = 0 \tag{1}  \]
where \( {a}_{1} = {a}_{2} = \cdots = {a}_{k} = 0  \). Applying \( T - {\lambda}_{k} I  \) on both sides of (1), we have
\[ (T - {\lambda}_{k} I) ({a}_{1} {v}_{1} + {a}_{2} {v}_{2} + \cdots + {a}_{k} {v}_{k} ) = 0   \]
implies
\[  {a}_{1} ({\lambda}_{1} - {\lambda}_{k}) {v}_{1} + {a}_{2} ({\lambda}_{2} - {\lambda}_{k}) {v}_{2} + \cdots + {a}_{k-1} ({\lambda}_{k-1} - {\lambda}_{k}) {v}_{k-1} = 0.  \]
Using our induction hypothesis, we have that \( \{ {v}_{1}, {v}_{2}, \dots, {v}_{k-1} \}  \) implies that 
\[  {a}_{1} ({\lambda}_{1} - {\lambda}_{k}) = {a}_{2} ({\lambda}_{2} - {\lambda}_{k})  = \cdots + {a}_{k-1} ( {\lambda}_{k-1} - {\lambda}_{k} ) = 0.\]
Since \( {\lambda}_{i}  \) for \( 1 \leq i \leq k   \) is distinct, we have that \( {\lambda}_{i-1} - \lambda_{i} \neq  0   \) for all \( 1 \leq i \leq k - 1  \). Consequently, this results in \( {a}_{i} = 0  \) for all \( 1 \leq i \leq k - 1 \) which leaves us with \( {a}_{k} {v}_{k} = 0  \). Since \( {v}_{k} \) is an eigenvector, we have \( {v}_{k} \neq  0  \) so \( {a}_{k} = 0  \). Thus, we have \( {a}_{1} = {a}_{2} = \cdots = {a}_{k-1} = {a}_{k} = 0  \) implies that \( \{ {v}_{1}, {v}_{2}, \dots, {v}_{k} \}   \) is a linearly independent set.   
\end{proof}

\begin{corollary}
  Let \( T  \) be a linear operator on an \( n- \)dimensional  vector space \( V  \). If \( T  \) has \( n  \) distinct eigenvalues, then \( T  \) is diagonalizable.  
\end{corollary}
\begin{proof}
Suppose that \( T  \) has \( n  \) distinct eigenvalues \( {\lambda}_{1}, {\lambda}_{2}, \dots, {\lambda}_{n} \). We can choose an eigenvalue \( {\lambda}_{i} \) for each corresponding eigenvector \( {v}_{i} \) for all \( i  \). Note that each \( {\lambda}_{i} \) is distinct. Using {\hyperref[Theorem 5.5]{Theorem 5.5}}, the set \( \{ {v}_{1}, \dots, {v}_{n} \}  \) is linearly independent. Since \( \text{dim}(V) = n  \), this set is a basis for \( V  \). Thus, \( T  \) is diagonalizable via Theorem 5.1. 
\end{proof}

\begin{eg}
   Let  
   \[  A = \begin{pmatrix} 
       1 & 1 \\
       1 & 1 
             \end{pmatrix}  \in {M}_{2 \times 2}(\R). \]
    The characteristic polynomial of \( A  \) (and hence of \( {L}_{A} \)) is
    \[  \text{det}(A - t I ) = \text{det} \begin{pmatrix} 
        1 - t & 1 \\
        1 & 1 - t 
              \end{pmatrix}  = t (t-2). \]
        We can see that the eigenvalues of \( {L}_{A} \) are \( 0  \) and \( 2 \). These eigenvalues correspond to the eigenvectors of \( {L}_{A} \) which form a basis such that \( A  \) is a diagonal matrix. Thus, \( {L}_{A} \) is a linear-operator that is diagonalizable (and hence \( A  \) is also diagonalizable).
\end{eg}

Note that it is not necessarily true that a diagonalizable linear operator contains \( n  \) distinct eigenvalues. A quick counter-example would be the identity operator. Even though \( I  \) is diagonalizable, it only contains one eigenvalue, namely, \( \lambda = 1  \).

This tells us that diagonalizability requires a much stronger condition on the characteristic polynomial.

\begin{definition}[Splits Over]
   A polynomial \( f(t)  \) in \( P(F)  \) \textbf{splits over} \( F  \) if there are scalars \( c, {a}_{1}, {a}_{2}, \dots, {a}_{n} \) (not necessarily distinct) in \( F  \) such that 
   \[  f(t) = c(t- {a}_{1})(t - {a}_{2})\cdots (t - {a}_{n}). \]
\end{definition}

The splitting behavior of a polynomial is different based on which field the polynomial is defined on. For example, we see that \( t^{2} - 1  \) splits over \( \R  \), but \( (t^{2} + 1)(t-2) \) does not since \( t^{2} + 1  \) has no solutions in the real line. However, \( t^{2} + 1  \) can further be split if it was defined over \( \C  \). In this case, \( (t^{2} +1)(t-2) \) does split over \( \C  \), namely, it splits into \( (t+i)(t-i)(t-2) \).

\begin{theorem}
   The characteristic polynomial of any diagonalizable linear operator splits.
\end{theorem}
\begin{proof}
Let \( T  \) be a diagonalizable linear operator on the \( n- \)dimensional vector space \( V  \), and let \( \beta \) be an ordered basis for \( V  \) such that \( [T]_{\beta} = D  \) is a diagonal matrix. Suppose that  
\[ \begin{pmatrix} 
{\lambda}_{1} & 0 & \cdots & 0 \\
0  & {\lambda}_{2} & \cdots & 0 \\
\vdots & \vdots &   & \vdots \\  
0 & 0 & \cdots & {\lambda}_{n}
\end{pmatrix}, \]
and let \( f(t)  \) be the characteristic polynomial of \( T  \). Then
\begin{align*} 
    f(t) = \text{det}(D - tI) = \text{det} \begin{pmatrix} 
{\lambda}_{1} - t  & 0 & \cdots & 0 \\
0  & {\lambda}_{2} - t & \cdots & 0 \\
\vdots & \vdots &   & \vdots \\  
0 & 0 & \cdots & {\lambda}_{n} - t
\end{pmatrix} \\  
= ({\lambda}_{1} - t)({\lambda}_{2} -t)\cdots ({\lambda}_{n} -t) = (-1)^{n}(t - {\lambda}_{1})(t - {\lambda}_{2})\cdots(t- {\lambda}_{n}). 
    \end{align*}
\end{proof}
\begin{itemize}
    \item If \( T  \) is a diagonalizable linear operator but fails to have distinct eigenvalues, then the characteristic polynomial of \( T  \) must have repeated zeros.
    \item The converse of the theorem above is not true since not every characteristic polynomial of a linear operator of \( T  \) guarantees that \( T  \) be diagonalizable.
\end{itemize}

\begin{definition}[Algebraic Multiplicity]
    Let \( \lambda  \) be an eigenvalue of a linear operator or matrix with characteristic polynomial \( f(t) \). The \textbf{(algebraic) multiplicity} of \( \lambda  \) is the largest positive integer \( k  \) for which \( (t- \lambda)^{k }  \) is a factor of \( f(t) \).
\end{definition}

Recall that a diagonalizable linear operator \( T  \) that is defined over a finite-dimensional vector space \( V  \) contains an ordered basis \( \beta \) for \( V \) consisting of eigenvectors of \( T  \). By Theorem 5.1, \( [T]_{\beta} \) is a diagonal matrix in which the diagonal entries are the eigenvalues of \( T  \). Remember that each eigenvalue of \( T  \) corresponds to the diagonal entry of \( [T]_{\beta} \) as many times as its multiplicity permits. 

We can investigate the exact amount of independent eigenvectors that are associated with a given eigenvalue. A way we can do this is to look at the null space of \( T - \lambda I  \).

\begin{definition}[Eigenspace]
    Let \( T  \) be a linear operator on a vector space \( V  \) ,and let \( \lambda  \) be an eigenvalue of \( T  \). Define \( {E}_{\lambda} = \{ x \in V : T(x) = \lambda x \}  =  N(T - \lambda {I}_{V}) \). The set \( {E}_{\lambda} \) is called the \textbf{eigenspace} of \( T  \) corresponding to the eigenvalue \( \lambda  \). Analogously, we define the \textbf{eigenspace} of a square matrix \( A  \) to be the eigenspace of \( {L}_{A} \).
\end{definition}

It can easily be proven that \( {E}_{\lambda} \) is a subspace of \( V  \). The maximum number of linearly independent eigenvectors that correspond to a given eigenvalue can therefore be seen by taking the dimension of the given eigenspace.

\begin{theorem}
    Let \( T  \) be a linear operator on a finite-dimensional vector space \( V  \), and let \( \lambda  \) be an eigenvalue of \( T  \) having multiplicity \(  m  \). Then \( 1 \leq \text{dim}({E}_{\lambda}) \leq  m \).
\end{theorem} 
\begin{proof}

\end{proof}

