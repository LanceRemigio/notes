\section{The Orthogonalization Process and Orthogonal Complements}

Just as bases are building blocks for vector spaces, orthonormal bases are the building blocks for inner product spaces.

\begin{definition}[Orthonormal Bases for Inner Product Spaces]
   Let \( V  \) be an inner product space. A subset of \( V  \) is an \textbf{orthonormal basis} for \( V  \) if it is an ordered basis that is orthonormal. 
\end{definition}

The next theorem illustrates the importance of orthonormal sets and orthonormal bases in general.

\begin{theorem}
    Let \( V  \) be an inner product space and \( S = \{ {v}_{1}, {v}_{2}, \dots, {v}_{k } \}  \) be an orthogonal subset of \( V  \) consisting of nonzero vectors. If \( y \in \text{span}(S) \), then
    \[  y = \sum_{ i=1  }^{ k  } \frac{ \langle y , {v}_{i} \rangle }{ \|{v}_{i}\|^{2}  }  {v}_{i}. \]
\end{theorem}
\begin{proof}
Since \( y \in \text{span}(S) \), we can find scalars \( {a}_{1}, {a}_{2}, \dots, {a}_{k } \) such that
\[  y = \sum_{ i=1  }^{ k  }{a}_{i} {v}_{i}. \]
Then for \( 1 \leq i \leq k  \), we must have
\[  \langle y , {v}_{j} \rangle = \Big\langle \sum_{ i=1  }^{ k  } {a}_{i} {v}_{i} , {v}_{j} \Big\rangle  =\sum_{ i=1  }^{ k  } {a}_{i} \langle {v}_{i} , {v}_{j} \rangle = {a}_{j} \|{v}_{j}\|^{2}. \]
Thus, we have that
\[ {a}_{j} = \frac{ \langle y , {v}_{j} \rangle }{ \|{v}_{j}\|^{2} }  \]
which leads to our desired result 
\[  y = \sum_{ i=1  }^{ k  } \frac{ \langle y ,  {v}_{i} \rangle }{ \|{v}_{i}\|^{2} } {v}_{i}. \]
\end{proof}


\begin{corollary}
   If, in addition to the hypotheses of Theorem 6.3, \( S  \) is orthonormal and \( y \in \text{span}(S) \), then 
   \[  y = \sum_{ i=1  }^{ k  } \langle y , {v}_{i} \rangle {v}_{i}. \]
\end{corollary}

\begin{proof}
Since \( S  \) is orthonormal, we have \( \|{v}_{i}\| = 1  \) for all \( 1 \leq i \leq k  \). So,
\[  y = \sum_{ i=1  }^{ k  } \langle y , {v}_{i} \rangle {v}_{i}. \]
\end{proof}

\begin{corollary}
    Let \( V  \) be an inner product space, and let \( S  \) be an orthogonal subset of \( V  \) consisting of nonzero vectors. Then \( S  \) is linearly independent.
\end{corollary}

\begin{proof}
Suppose that \( {v}_{1}, {v}_{2}, \dots, {v}_{k} \in S  \) and 
\[  \sum_{ i=1  }^{ k  } {a}_{i} {v}_{i} = 0.  \]
Observing that \( y = 0  \), we can apply Theorem 6.3 to obtain
\[  {a}_{j} = \frac{ \langle 0 , {v}_{j} \rangle }{  \|{v}_{j}\|^{2} }  = 0\]
for all \( j  \). Thus, \( S  \) is linearly independent.
\end{proof} 

\begin{itemize}
    \item The main takeaway from these results is that if we have some orthonormal basis, we can solve for the coefficients by using the formula described in the first corollary.
    \item The second corollary tells us that vector space \( H  \) in the Section 6.1 contains an infinite linearly independent set and therefore is not a finite-dimensional vector space.
    \item Later in this section, we will prove that it is possible for finite-dimensional vector spaces to posses an orthonormal basis from a linearly independent set of vectors.
\end{itemize}

\begin{theorem}
    Let \( V  \) be an inner product space and \( S = \{ {w}_{1}, {w}_{2}, \dots, {w}_{n} \}  \) be a linearly independent subset of \( V  \). Define \( S' = \{ {v}_{1}, {v}_{2}, \dots, {v}_{n} \},  \) where \( {v}_{1} = {w}_{1} \) and
    \[  {v}_{k} = {w}_{k} - \sum_{ j=1 }^{ k - 1  } \frac{ \langle {w}_{k} , {v}_{j} \rangle }{ \|{v}_{j}\|^{2} } {v}_{j} \ \ \text{for } 2 \leq k \leq n. \]
    Then \( S'  \) is an orthogonal set of nonzero vectors such that \( \text{span}(S') = \text{span}(S) \).
\end{theorem}
\begin{proof}

\end{proof}



