\section{Linear Combinations and Systems of Linear Equations}
\begin{definition}[Linear Combinations]
   Let \( V  \) be a vector space and \( S  \) a nonempty subset of \( V  \). A vector \( v \in V  \) is called \textbf{linear combination} of vectors of \( S  \) if there exist a finite number of vectors \( u_{1} , u_{2}, \dots, u_{n} \in S  \) and scalars \( a_{1}, a_{2} , \dots, a_{n} \in F  \) such that 
   \[  v = a_{1} u_{1} + a_{2} u_{2} + \cdots + a_{n} u_{n}. \] In this case, we say that \( v \) is a linear combination of \( u_{1} , u_{2} , \dots, u_{n} \in V   \) and call \( a_{1} , a_{2} , \dots, a_{n} \in F   \) the \textbf{coefficients} of the linear combination. 
\end{definition}

\begin{itemize}
    \item An easy example of a vector expressed as a linear combination is the zero vector \( O  \) where \( 0 v = O  \) for each \( v \in V  \).
    \item Determining whether a vector is a linear combination of other vectors often involves solving a system of linear equations.
\end{itemize}

\begin{eg}
    Suppose we wanted to express the vector \( (2,6,8) \in \R^{3}  \) as a linear combination of 
    \begin{center}
    \( u_{1} = (1,2,1)  \), \( u_{2} = (-2, -4, -2) \), \( u_{3} = (0,2,3) \), \( u_{4} = (2,0,-3) \), and \( u_{5} = (-3, 8, 16) \).
    \end{center}
    Our goal is to find scalars \( a_{1}, a_{2}, a_{3}, a_{4} \) and \( a_{5}  \) such that 
    \begin{align*}
        (2,6,8) &= a_{1} u_{1} + a_{2} u_{2} + a_{3} u_{3} + a_{4} u_{4} + a_{5}  u_{5}.   \\
    \end{align*}
    Doing a bit of algebra, we would need to solve the following system of linear equations 
    \begin{align*}
        a_{1} - 2a_{2}  \ \ \  + 2a_{4} - 3a_{5} &= 2  \\
        2a_{1} - 4a_{2} + 2a_{3}  \ \ \  + 8a_{5} &= 6  \\
        a_{1} - 2a_{2} + 3a_{3} + 16a_{5} &= 8
    \end{align*}
\end{eg}

Solving the system of equations above involves three types of operations:
\begin{enumerate}
    \item Interchanging the order of any two equations in the system;
    \item multiplying any equation in the system by some non-zero constant;
    \item adding a constant multiple of any equation to another equation in the system.
\end{enumerate}
We will learn in a later section that the operations listed above do not change the set of solutions to the original system.

\begin{eg}
    We claim that 
    \[  2x^{3} - 2x^{2} + 12x - 6 \] is a linear combination of
    \begin{center}
        \( x^{3} - 2x^{2} - 5x - 3  \) and \( 3x^{3} - 5x^{2} - 4x - 9  \) 
    \end{center}
    in \( P_{3}(\R ) \), but that 
    \[  3x^{3} - 2x^{2} + 7x + 8  \] is not. In the first case we wish to find scalars \( a \) and \( b  \) such that 
    \begin{align*}
        2x^{3} - 2x^{2} + 12x - 6 &= a(x^{3} - 2x^{2} - 5x - 3 ) \\ 
                                  &+ b(3x^{3} - 5x^{2} - 4x - 9)  \\
                                  &= (a + 3b)x^{3} + (-2a-5b)x^{2} \\  
                                  &+ (-5a-4b)x + (-3a - 9b). 
    \end{align*}
    Thus, we have the following system of linear equations:
    \begin{align*}
       a + 3b  &= 2 \\
       -2a - 5b &= -2 \\ 
       -5a - 4b = 12 \\
       -3a - 9b = -6.
    \end{align*}
    Adding the appropriate multiples of the first equation to the others in order to eliminate \( a  \), we find that 
    \begin{align*}
        a + 3b &= 2  \\
        b &= 2  \\
        11b &= 22 \\
        0b &= 0.
    \end{align*}
    Then we get that \( a = -4  , b = 2 , 0 = 0 , 0= 0  \). Hence, we have
 \[ 2x^{3} - 2x^{2} + 12x - 6 = -4(x^{3} - 2x^{2} - 5x - 3 )  + 2(3x^{3} - 5x^{2} - 4x - 9). \]  
    Looking at the second case now, we observe that using the preceding technique leads us to the following system of linear equations
    \begin{align*}
        a + 3b &= 3  \\
        -2a - 5b &= -2 \\
        -5a - 4b &= 7  \\
        -3a - 9b &= 8. 
    \end{align*}
    Eliminating \( a  \) as before yields the following:
    \begin{align*}
        a + 3b &= 3  \\
        b &= 4  \\
        11b &= 22 \\ 
        0 &= 17.
    \end{align*}
    The presence of the non-sensical result in the last equality tells us that there are no solutions to the system and that the second polynomial cannot be written as a linear combination of the two polynomials.
\end{eg}

We can denote the set of all linear combinations of some set of vectors in the following:

\begin{definition}[Span]
    Let \( S  \) be a nonempty subset of a vector space \( V  \). The \textbf{span} of \( S  \), denoted \( \text{span}(S) \), is the set consisting of all linear combinations of the vectors in \( S  \). For convenience, we define \( \text{span}(\emptyset) = \{ 0  \}  \). 
\end{definition}

Some immediate examples of spans are:
\begin{eg}
      In \( \R^{3}   \), the span of the set \( S =  \{ (1,0,0) , (0,1,0)  \}  \) consist of all vectors in \( \R^{3}  \) such that for some scalars \( a,b \in \R  \), we have 
        \[  a(1,0,0) + b (0,1,0) = (a,b , 0). \]
        This tells us that \( S  \) contains all the points in the \( xy \)-plane. One can show that \( S  \) is a subspace of \( \R^{3} \).
\end{eg}

\begin{theorem}\label{Theorem 1.5}
    The span of any subset \( S  \) of a vector space \( v  \) is a subspace of \( V  \) that contains \( S  \). Moreover, any subspace of \( V  \) that contains \( S  \) must also contain the span of \( S  \).
\end{theorem}
\begin{proof}
Suppose \( S = \emptyset \). Then the span of \( S  \) is just \( \text{span}(\emptyset) = \{ 0  \}  \) which is a subspace in which \( S  \) is contained in. Moreover, \( \text{span}(\emptyset) = \{ 0  \}   \) is always contained in any subspace \( X  \) of \( V  \) because every subspace contains the zero vector \( 0  \). 
Suppose \( S \neq \emptyset \), then \( S  \) contains a vector \( z  \). We need to show that \( \text{span}(S) \) is a subspace of \( V  \). 
\begin{enumerate}
    \item[(a)] Since \( S \neq \emptyset \), we know that \( S  \)  contains a vector \( v  \) such that \( 0 z = 0  \). Hence, \( 0 \in \text{span}(S) \).
    \item[(b)] Let \( x,y \in \text{span}(S) \). We need to show that \( x + y \in \text{span}(S) \). If \( x \in \text{span}(S) \), then we can find \( a_{1}, a_{2}, \dots, a_{n} \in F  \) and \( x_{1}  , x_{2}, \dots, x_{n} \in S  \) such that 
        \[  a_{1} x_{1} + a_{2} x_{2} + \cdots + a_{n} x_{n}. \]
        Likewise, \( y \in \text{span}(S) \) implies that we can find scalars \( b_{1} , b_{2} , \dots , b_{m}   \) and vectors  \( y_{1}, y_{2} , \dots, y_{m} \) such that 
        \[  b_{1} y_{1} + b_{2} y_{2} + \dots + b_{m } y_{m}. \]
        Then  
        \begin{align*}
            x + y &= a_{1} x_{1} + a_{2} x_{2} + \cdots + a_{n} x_{n} + b_{1} y_{1} + b_{2} y_{2} + \dots + b_{m } y_{m} \\
        \end{align*}
        is a linear combination and thus \( x + y \in \text{span}(S) \).
    \item[(c)] Note that \( c \in F  \) implies that
        \begin{align*}
            cx &= c(a_{1} x_{1} + a_{2} x_{2} + \cdots + a_{n} x_{n}) \\
               &= (c a_{1}) x_{1} + (ca_{2})  x_{2} + \cdots + (ca_{n}) x_{n} \\
        \end{align*}
        is a linear combination and thus \( cx \in \text{span}(S) \).
\end{enumerate}
Hence, \( \text{span}(S)  \) is a subspace of \( V  \). Now we need to show that \( S  \) is contained within \( \text{span}(S) \); that is, \( S \subseteq \text{span}(S) \). Let \( v \in S  \). Then using (VS 5), we can see that \( 1 \cdot v = v \) is a linear combination; so we have \( v \in \text{span}(S) \) and hence, the span of \( S  \) contains \( S  \). 

Now let \( W  \) be any subspace of \( V  \) that contains \( S  \).  We need to show that \( \text{span}(S) \subseteq W  \). Let \( v \in \text{span}(S) \). Then we can find scalars \( a_{1}, a_{2}, \dots, a_{n} \in F    \) and \( x_{1} , x_{2} , \dots, x_{n} \in S  \) such that 
\[  v = a_{1} x_{1} + a_{2} x_{2} + \dots + a_{n} x_{n}.   \]
Since \( S \subseteq W  \), we know that \( x_{1}, x_{2} , \dots , x_{n} \in W  \). Using exercise 20 from section 1.3 and using the same set of scalars \( a_{1}, a_{2}, \dots, a_{n} \in F  \), we have \( v = a_{1} x_{1} + a_{2} x_{2} + \cdots + a_{n} x_{n} \in W  \). Hence, \( \text{span}(S) \subseteq W  \).
\end{proof}

\begin{definition}[ ]
    A subset \( S  \) of a vector space \( V  \) \textbf{generates} (or \textbf{spans}) \( V  \) if \( \text{span}(S) = V  \). In this case, we also say that the vectors of \( S  \) generates (or span) \( V  \).
\end{definition}

\begin{eg}[Vectors in \( \R^3 \)]
    The vectors \( (1,1,0) , (1,0,1) , (0,1,1)   \) generate \( \R^{3} \) since any given vector \( v \in \R^{3}  \) is a linear combination of the three given vectors. Furthermore, there exists scalars \( r,s,t \in \R \) such that
    \[  r(1,1,0) + s(1,0,1) + t(0,1,1) = (a_{1}, a_{2} , a_{3}) = v \]
    where
    \begin{center}
       \( r = \frac{ 1 }{ 2 }  (a_{1} + a_{2} - a_{3}), s = \frac{ 1 }{ 2 }   (a_{1} - a_{2} + a_{3}) \) and \( t = \frac{ 1 }{ 2 }  (-a_{1} + a_{2} + a_{3}) \). 
    \end{center}
\end{eg}

\begin{eg}[Polynomials]\label{Generating Polynomials section 1.4}
   The polynomials \( x^{2} + 3x - 2, 2x^{2} + 5x - 3,   \) and \( -x^{2} - 4x + 4  \) generate \( P_{2}(\R ) \) because each of the three given polynomials belongs to \( P_{2}(\R) \) and each polynomial \( ax^{2} + bx + c \in P_{2}(\R) \) is a linear combination of these three. It can be shown that we can find \( a,b,c   \in \R  \) such that 
   \begin{align*}
   a (x^{2} + 3x - 2)   + b (2x^{2} + 5x - 3) +  c (-x^{2} - 4x + 4) &= ax^{2} + bx + c.   
\end{align*}
\end{eg}

\begin{eg}[Matrices]\label{Generating Matrices section 1.4}
    Note that the matrices
    \begin{center}
        \( \begin{pmatrix}
            1 & 0 \\
            1 & 0 
        \end{pmatrix} \),  \( \begin{pmatrix}
            1 & 1 \\ 
            0 & 1 
        \end{pmatrix} \),  \( \begin{pmatrix}
            1 & 0 \\ 
            1 & 1 
        \end{pmatrix} \), and \( \begin{pmatrix}
            0 & 1 \\
            1 & 1 
        \end{pmatrix} \) 
        
    \end{center}
    generate \( M_{2 \times 2 }(\R ) \) because an every \( A \in M_{2 \times 2}(\R)  \) can be expressed as a linear combination of the four given matrices found below where there exists scalars \( a, b , c , d \in \R  \) such that
    \begin{align*}
        \begin{pmatrix}
            a_{11} & a_{12} \\
            a_{21} & a_{22} 
        \end{pmatrix} &= a  \begin{pmatrix}
        1 & 1 \\
        1 & 0 
        \end{pmatrix} 
        + b \begin{pmatrix}
            1 &  1 \\
            0 & 1 
        \end{pmatrix} + c \begin{pmatrix}
            1 & 0 \\
            1 & 1 
        \end{pmatrix} 
        + d \begin{pmatrix}
            0 & 1 \\
            1 & 1 
        \end{pmatrix}
    \end{align*}
    where it can be show that 
    \begin{align*}
        a &= \frac{ 1 }{ 3 }  a_{11} + \frac{ 1 }{ 3 }  a_{12} + \frac{ 1 }{ 3 }  a_{21} - \frac{ 2 }{ 3 }  a_{22}   \\
        b &= \frac{ 1 }{ 3 }  a_{11} + \frac{ 1 }{ 3 }  a_{12} - \frac{ 2 }{ 3 }  a_{21} + \frac{ 1 }{ 3 }  a_{22} \\
        c &= \frac{ 1 }{ 3 }  a_{11} - \frac{ 2 }{ 3 }  a_{12} + \frac{ 1 }{ 3 }  a_{21} + \frac{ 1 }{ 3 }  a_{22} \\
        d &= - \frac{ 2 }{ 3 }  a_{11} + \frac{ 1 }{ 3 }  a_{12} + \frac{ 1 }{ 3 }  a_{21} + \frac{ 1 }{ 3 }  a_{22}.
    \end{align*}

A non-example set of matrices are
\begin{center}
    \( \begin{pmatrix}
        1 & 0 \\ 
        0 & 1 
    \end{pmatrix} \), \( \begin{pmatrix}
        1 & 1 \\
        0 & 1 
        \end{pmatrix} \), and \( \begin{pmatrix}
        1 & 0 \\
        1 & 1 
    \end{pmatrix} \)
\end{center}
do not generate \( M_{2 \times 2 }(\R ) \) since all of the matrices have equal diagonal entries. Thus, every the set of matrices above cannot generate every  \( A \in M_{2 \times 2 }(\R )  \).
\end{eg}


