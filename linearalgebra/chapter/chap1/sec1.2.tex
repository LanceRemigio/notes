\section{Vector Spaces}

\subsection{Basics}


\begin{definition}[Vector Spaces]
    A \textbf{vector space} (or \textbf{linear space}) over a field \( F \) consists of a set on which two operations (called \textbf{addition} and \textbf{scalar multiplication}, respectively) are defined so that for each pair of elements \( x,y, \) in \( V \) there is a unique element \( ax  \) in \( V  \), such that the following conditions hold:
    \begin{enumerate}
        \item[(VS 1)] For all \( x,y   \in V \), \( x + y = y + x  \) (commutativity of addition).
        \item[(VS 2)] For all \( x, y, z  \in V  \) , \( (x+y) + z = x + (y + z)  \) (associativity of addition).
        \item[(VS 3)] There exists an element in \( V  \) denoted by \( O  \) such that \( x + O = x  \) for each \( x  \in V \) 
        \item[(VS 4)] For each element \( x \in V  \), there exists an element \( y \in V  \) such that \( x + y = O \).
        \item[(VS 5)] For each element \( x \in V  \), we have \( 1x = x  \).
        \item[(VS 6)] For each \( a,b \in F  \) and each element \( x \in V  \), then \( (ab)x = a(bx) \). 
        \item[(VS 7)] For each element \( a \in F  \) and each pair \( x,y \in V  \), we have \( a(x+y) = ax + ay \).
        \item[(VS 8)]  For each pair \( a,b \in F  \) and each \( x \in V  \), we have \( (a+b)x = ax + bx \).
    \end{enumerate} 
The elements \( x + y  \) and \( a x  \) are called the \textbf{sum} of \( x  \) and \( y  \) and the \textbf{product} of \( a  \) and \( x  \), respectively.
\end{definition}

\begin{itemize}
    \item The elements of a field \( F  \) are called \textbf{scalars} and the elements of a vector space \( V  \) are called \textbf{vectors} (these should not be confused!).
    \item Every vector space will always be defined over a given field, mostly defined over the real numbers \( \R  \) or the complex numbers \( \C  \) unless otherwise noted.
    \item Every vector space should specify the operations of addition and scalar multiplication.
\end{itemize}

\begin{definition}[n-tuples]
    An object of the form \( (a_{1}, a_{2}, \dots ,a_{n}) \), where the entries \( a_{1}, a_{2} , \dots, a_{n} \) are elements of a field \( F  \), is called an \textbf{n-tuple} with entries from \( F  \). The elements \( a_{1}, a_{2} , \dots, a_{n} \) are called \textbf{entries} or \textbf{components} of the \( n \)-tuple.
\end{definition}


\begin{definition}
    We say that two \( n \)-tuples, \( (a_{1}, a_{2}, \dots, a_{n}) \) and \( (b_{1}, b_{2}, \dots , b_{n}) \),  are \textbf{equal} if \( a_{i} = b_{i}  \) for \( i = 1,2, \dots, n \).  
\end{definition}

\begin{eg}
    The set of all \( n \)-tuples with entries from a field \( F  \) denoted by \( F_{n} \) is a vector space. To see why, suppose \( u, v \in F_{n}  \) where \( u = (a_{1}, a_{2}, \dots a_{n}) \) and \( v = (b_{1}, b_{2}, \dots, b_{n}) \). If we take term-by-term addition of the entries in both \( u  \) and \(  v  \), then we end up with 
    \[  u + v = (a_{1} + b_{1}, a_{2} + b_{2}, \dots , a_{n} + b_{n}) \]
and likewise,
    \[ cu = (ca_{1}, ca_{2}, \dots, ca_{n}).   \]
    These same set of operations define \( \R ^{3} \) as a vector space over \( \R  \) and likewise, \( \C ^{2} \) is a vector space over \( \C  \).

\end{eg}

\begin{itemize}
    \item 
    Note that vectors in \( F^{n} \) can be written as \textbf{column vectors} 
            \[  \begin{pmatrix}
                a_{1} \\
                a_{2} \\
                \vdots \\ 
                a_{n}
            \end{pmatrix} \]
    rather than \textbf{row vectors} \( (a_{1} , a_{2}, \dots , a_{n}) \). 
\item 1-tuples are are just scalars or an just an element from \( F \). 
\end{itemize}


\begin{definition}
An \( m \times n  \) \textbf{matrix} with entries from a field \( F  \) is a rectangular array of the form
    \[  \begin{pmatrix}
        a_{11} & a_{12} & \cdots & a_{1n} \\
        a_{21} & a_{22} & \cdots & a_{2n}  \\
        \vdots & \vdots &           & \vdots \\
        a_{m1} & a_{m2} & \cdots     & a_{mn}
    \end{pmatrix} \]
    where each entry \( a_{ij} \) with \( (1 \leq i \leq m , 1 \leq j \leq n  ) \) is an element of \( F  \). We call the entries \( a_{ij}  \) with \( i = j  \) the \textbf{diagonal entries} of the matrix. The entries \( a_{i1}, a_{i2}, \dots a_{in} \) compose the \textbf{\( i \)th row}of the matrix, and the entries \( a_{1j}, a_{2j}, \dots a_{mj } \) compose the \( j \)th column of the matrix.
\end{definition}

\begin{itemize}
    \item The rows make a vector space which we denote \( F^{n} \).
    \item Likewise, the columns make a vector space we denote \( F^{m} \)
\end{itemize}

\begin{definition}[Zero Matrix]   
    The \( m \times n  \) matrix in which each entry equals zero is called the \textbf{zero matrix} and is denoted by \( O \).
\end{definition}

\begin{definition}[Square Matrix]
    A matrix is \textbf{square} if the number of rows and columns of a matrix are equal.
\end{definition}

\begin{itemize}
    \item Just like our tuple example, the set of all \( m \times n  \) matrices with entries from a field \( F  \) form a vector space. Denote this vector space as \( M_{m \times n}(F) \) endowed with two operations; that is, \textbf{matrix addition} and \textbf{scalar multiplication}. Suppose for \( A,B \in M_{m \times n}(F) \) and \( c \in F  \), we have 
        \[   (A + B)_{ij} = A_{ij} + B_{ij} \]
        and 
        \[  (cA)_{ij} = c A_{ij} \] for \(  1 \leq i \leq m  \) and \( 1 \leq j \leq n  \). In other words, the two operations can be performed entry-wise.
    \item The operations from our tuple case extends very naturally to \( M_{m \times n}(F) \). In other words, if we add two matrices \( A_{ij}  \) and \( B_{ij }  \), then we would expect to that \( A_{ij} + B_{ij} \in  M_{m \times n }(F) \) as well and likewise for the scalar multiplication case.
\end{itemize}

\begin{definition}[Set of All Functions]
    Let \( S \) be any nonempty set and \( F  \) be any field, and let \( \mathcal{F}(S,F) \) denote the set of all functions from \( S  \) to \( F  \).
\end{definition}

\begin{definition}
    Two functions \( f,g \in \mathcal{F}(S,F)  \) are called \textbf{equal} if \( f(s) = g(s) \) for each \( s \in S  \). 
\end{definition}

The set of functions above forms a vector space with the operations of addition and scalar multiplication defined for every \( f,g \in \mathcal{F}(S,F) \) and \( c \in F  \) with 
\begin{center}
    \( (f+g)(s) = f(s) + g(s)  \) and \( (cf)(s) = cf(s)  \)
\end{center}

\begin{definition}[Set of All Polynomials]
    A \textbf{polynomial} with coefficients from a field \( F  \) is an expression of the form 
    \[  f(x) = a_{n} x^{n} + a_{n-1} x^{n-1} + \cdots + a_{1} x + a_{0}, \]
    where \( n  \) is a nonnegative integer and each \( a_{k } \in F   \) is called the \textbf{coefficient } of \( x_{k }  \).
\end{definition}

\begin{definition}[Zero Polynomial]
    We call \( f(x) = 0  \) the \textbf{zero polynomial} if \( a_{n} = a_{n-1} = \cdots = a_{0} = 0  \).
\end{definition}

\begin{definition}[Degree]
    The \textbf{degree} of a given polynomial \( f  \) is defined to be the largest exponent of \( x  \) that appears in the representation
    \[  f(x) = a_n x^{n} + a_{n-1} x^{n-1} + \cdots + a_{1}x + a_{0}.  \]
\end{definition}

\begin{definition}[Equality of Polynomials]
    We call two polynomials \( f,g  \), where 
    \[ f(x) = a_n x^{n} +  a_{n-1} x^{n-1} + \cdots + a_{1}x + a_{0}   \] and 
    \[  g(x) = b_m x^{m} + b_{m-1} x^{m-1} + \cdots + b_{1}x + b_{0}, \],
    \textbf{equal} if \( m =n  \) and \( a_{i} = b_{i}  \) for all \( i = 0,1, \dots , n  \).
\end{definition}


Suppose we have \( c \in F  \) and say we evaluated the polynomial \( f \in F   \) at \( c  \). Then we would have the following 
\[  f(c) = a_n c^{n} + a_{n-1} c^{n-1} + \cdots + a_{1}c + a_{0} \]
where \( f(c) \in F  \).

\begin{definition}[Basic Operations of Polynomials]
    Define polynomial \textbf{addition} \( f + g  \) as the following:
        \[  f(x) + g(x) = (a_{n} + b_{n} ) x^{n} + (a_{n-1} + b_{n-1}) x^{n-1} + \cdots + (a_{1} + b_{1})x + (a_{0} + b_{0}). \] Let \( c \in F  \). Let scalar \textbf{multiplication} be defined by 
        \[  c f(x) = ca_n x^{n} + ca_{n-1} x^{n-1} + \cdots + ca_{1}x + ca_{0}. \]
\end{definition}

The operations above form a vector space for \( P(F) \) (the set of all polynomials).

\begin{definition}[Sequences]
    A \textbf{sequence} in \( F  \) is a function \( \sigma : \Z^{+} \to F   \). A given sequence \( \sigma  \) such that \( \sigma (n) = a_{n}  \) for \( n = 1,2,\dots  \) is denoted \( (a_{n}) \).   
\end{definition}

Let \( V  \) be the set of all sequences \( \sigma(n) \in F  \). For every \( (a_{n}), (b_{n} ) \in V  \) with \( t \in F  \), we have the following operations
\begin{center}
        \( (a_{n}) + (b_{n}) = (a_{n} + b_{n}) \) and \( t(a_{n}) = (t a_{n}) \).
\end{center}

\subsection{Non-examples}

\begin{eg}
    Let \( S = \{  (a_{1}, a_{2}) : a_{1} , a_{2} \in R  \}   \) where \( R  \) is a field. For every \( (a_{1} , a_{2}), (b_{1} , b_{2} ) \in S  \) and \( c \in R  \), define 
    \begin{center}
        \( (a_{1}, a_{2} ) + (b_{1}, b_{2}) = (a_{1} + b_{1} , a_{2} - b_{2}) \) and \( c(a_{1}, a_{2}) = (ca_{1}  , ca_{2}) \). 
    \end{center}
    Note that \( S \) is \textbf{NOT} a vector space since  (VS 1), (VS 2), and (VS 8) fail. 
\end{eg}

% add the second example here

\begin{eg}
    Let \( S  \) be the same set as in the last example. For \( (a_{1}, a_{2} ) , (b_{1}, b_{2}) \in S  \) and \( c \in R  \), define 
    \[  (a_{1}, a_{2} ) + (b_{1} , b_{2})  = ( a_{1} + b_{1} , 0 ) \]
    and 
    \[  c(a_{1}, a_{2}) = (ca_{1}, 0 ). \]
    Note that \( S  \) is \textbf{NOT} a vector space with these operations since (VS 3), (VS 4), and (VS 5) fail.
\end{eg}

\subsection{Basic Extensions from Definition}

\begin{theorem}[Cancellation Law for Vector Addtion]
    If \( x,y,z \in V  \) such that \( x + z = y + z  \), then \( x = y  \).
\end{theorem}
\begin{proof}
There exists a vector \( v \in V  \) such that \( z + v = O  \) (VS 4). Thus, we have
    \begin{align*}
        x &= x + O  \\
            &= x + (z + v ) \\
            &= (x+ z) + v  \\
            &= (y+z) + v \\
            &= y + (z + v) \\
            &= y + O \\ 
            &= y.
    \end{align*}
    Hence, we have \( x = y  \).
\end{proof}

\begin{corollary}
    The vector described \( O  \) described in (VS 3) is unique.
\end{corollary}
\begin{proof}
\textbf{TO DO.}
\end{proof}

\begin{corollary}
   The vector \( y  \) described in (VS 4) is unique. 
\end{corollary}
\begin{proof}
\textbf{TO DO.}
\end{proof}



