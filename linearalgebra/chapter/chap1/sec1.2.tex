
\section{Vector Spaces}

\begin{definition}{}{}
    A \textbf{vector space} (or \textbf{linear space}) over a field \( F \) consists of a set on which two operations (called \textbf{addition} and \textbf{scalar multiplication}, respectively) are defined so that for each pair of elements \( x,y, \) in \( V \) there is a unique element \( ax  \) in \( V  \), such that the following conditions hold:
    \begin{enumerate}
        \item[(VS 1)] For all \( x,y   \in V \), \( x + y = y + x  \) (commutativity of addition).
        \item[(VS 2)] For all \( x, y, z  \in V  \) , \( (x+y) + z = x + (y + z)  \) (associativity of addition).
        \item[(VS 3)] There exists an element in \( V  \) denoted by \( O  \) such that \( x + O = x  \) for each \( x  \in V \) 
        \item[(VS 4)] For each element \( x \in V  \), there exists an element \( y \in V  \) such that \( x + y = O \).
        \item[(VS 5)] For each element \( x \in V  \), we have \( 1x = x  \).
        \item[(VS 6)] For each \( a,b \in F  \) and each element \( x \in V  \), then \( (ab)x = a(bx) \). 
        \item[(VS 7)] For each element \( a \in F  \) and each pair \( x,y \in V  \), we have \( a(x+y) = ax + ay \).
        \item[(VS 8)]  For each pair \( a,b \in F  \) and each \( x \in V  \), we have \( (a+b)x = ax + bx \).
    \end{enumerate} 
The elements \( x + y  \) and \( a x  \) are called the \textbf{sum} of \( x  \) and \( y  \) and the \textbf{product} of \( a  \) and \( x  \), respectively.
\end{definition}

\begin{itemize}
    \item The elements of a field \( F  \) are called \textbf{scalars} and the elements of a vector space \( V  \) are called \textbf{vectors} (these should not be confused!).
    \item Every vector space will always be defined over a given field, mostly defined over the real numbers \( \R  \) or the complex numbers \( \C  \) unless otherwise noted.
    \item Every vector space should specify the operations of addition and scalar multiplication.
\end{itemize}

\begin{definition}{}{}
    An object of the form \( (a_{1}, a_{2}, \dots ,a_{n}) \), where the entries \( a_{1}, a_{2} , \dots, a_{n} \) are elements of a field \( F  \), is called an \textbf{n-tuple} with entries from \( F  \). The elements \( a_{1}, a_{2} , \dots, a_{n} \) are called \textbf{entries} or \textbf{components} of the \( n \)-tuple.
\end{definition}


\begin{definition}{}{}
    We say that two \( n \)-tuples, \( (a_{1}, a_{2}, \dots, a_{n}) \) and \( (b_{1}, b_{2}, \dots , b_{n}) \),  are \textbf{equal} if \( a_{i} = b_{i}  \) for \( i = 1,2, \dots, n \).  
\end{definition}

\begin{eg}{}{}
    The set of all \( n \)-tuples with entries from a field \( F  \) denoted by \( F_{n} \) is a vector space. To see why, suppose \( u, v \in F_{n}  \) where \( u = (a_{1}, a_{2}, \dots a_{n}) \) and \( v = (b_{1}, b_{2}, \dots, b_{n}) \). If we take term-by-term addition of the entries in both \( u  \) and \(  v  \), then we end up with 
    \[  u + v = (a_{1} + b_{1}, a_{2} + b_{2}, \dots , a_{n} + b_{n}) \]
and likewise,
    \[ cu = (ca_{1}, ca_{2}, \dots, ca_{n}).   \]
    These same set of operations define \( \R ^{3} \) as a vector space over \( \R  \) and likewise, \( \C ^{2} \) is a vector space over \( \C  \).


\end{eg}




