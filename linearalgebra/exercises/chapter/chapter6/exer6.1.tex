\section{Inner Products and Norms}

\subsection*{Exercise 6.1.6} Complete the proof of Theorem 6.1.
\begin{proof}
\begin{enumerate}
    \item[(b)] Let \( x,y \in V  \). Using the linearity of the first component, we must have 
        \begin{align*}
            \langle x, cy \rangle &= \overline{\langle cy, x  \rangle} \\
                                  &= \overline{c \langle y,x \rangle} \\ 
                                  &= \overline{c} \overline{\langle y,x \rangle} \\
                                  &= \overline{c} \langle  x,y  \rangle.
        \end{align*}
    \item[(c)] Let \( x \in V  \). Note that for any \( v \in V   \), we have \( 0 \cdot v = 0  \). So, we have
        \begin{align*}
            \langle x,0 \rangle &= \langle x, 0 \cdot v  \rangle = \overline{0} \langle x , v  \rangle  = 0 \langle x, v  \rangle = 0  
        \end{align*}
        Likewise, we have
        \[  \langle 0,x  \rangle = \langle  0 \cdot v , x  \rangle =  0 \langle v , x  \rangle = 0. \]
        Thus, we have \( \langle x,0  \rangle = \langle 0, x  \rangle = 0  \).
    \item[(d)] Suppose \( \langle x,x \rangle = 0  \). By part (c), we can see that 
        \[  \langle x , x \rangle = \langle 0  , x  \rangle = \langle x , 0  \rangle = 0    \]
        which is true if and only if \( x = 0  \). The converse is trivial.
    \item[(e)] Suppose \( \langle x , y \rangle = \langle x , z \rangle  \) for all \( x \in V  \). Then observe that 
        \begin{align*}
            \langle x , y \rangle  = \langle x , z \rangle &\implies \langle x , y \rangle - \langle x , z \rangle = 0   \\
                                                           &\implies \langle 0  ,  y -z  \rangle = 0.
        \end{align*}
        By part (c), we know that the above is true if and only if \( y - z = 0  \). So, we have \( y = z  \).
\end{enumerate}
\end{proof}

\subsection*{Exercise 6.1.7} Complete the proof of Theorem 6.2.
\begin{proof}
\begin{enumerate}
    \item[(a)] Let \( x \in V  \) and \( c \in F  \). Observe that
        \begin{align*}
            \| cx \| = \sqrt{ \langle  cx , cx \rangle }  
                   &= \sqrt{c \langle x , cx \rangle }  \\
                   &= \sqrt{ c \overline{c} \langle x , x \rangle } \\
                   &= \sqrt{ | c |^{2} \cdot \langle x , x \rangle } \\
                   &= | c |  \cdot \sqrt{ \langle x , x \rangle } \\
                   &= | c |  \cdot \| x \|.
        \end{align*}
        Thus, we have that \( \| cx \| = | c  |  \cdot \| x \| \).
    \item[(b)] Let \( x \in V  \). Suppose \( \| x \| = 0   \). Then by definition of norm, we have
        \[ \| x \| = \sqrt{ \langle x , x \rangle } = 0. \]
        Squaring both sides, we can see that 
        \[  \| x \|^{2} = \langle  x , x \rangle = 0. \]
        This is true if and only if \( x = 0  \) by part (d) of Theorem 6.1. Conversely, suppose \( x = 0   \). So by definition of norm and part (d) of Theorem 6.1, we have  
        \[  \sqrt{ \langle x , x \rangle }  = 0 \iff \| x \| = 0. \]
\end{enumerate}
\end{proof}

\subsection*{Exercise 6.1.9}\label{Exercise 6.1.9} Let \( \beta \) be a basis for a finite-dimensional inner product space.
\begin{enumerate}
    \item[(a)] Prove that if \( \langle x , z \rangle = 0  \) for all \( z \in \beta  \), then \( x = 0  \).
        \begin{proof}
            Let \( z \in \beta  \) where \( z \neq 0  \). By part (c) of the definition of inner product, \( \langle x , z \rangle = 0  \) if and only if \( x = 0  \).
        \end{proof}
    \item[(b)] Prove that if \( \langle x , z \rangle = \langle y , z \rangle \) for all \( z \in \beta  \), then \( x = y  \).
        \begin{proof}
        Let \( z \in \beta  \). Observe that
        \begin{align*}
            \langle x , z \rangle = \langle y , z \rangle &\iff \langle  x - y  , z  \rangle = 0.
        \end{align*}
        By part (a), we find that \( x - y = 0  \) if and only if \( x = y  \).
        \end{proof}
\end{enumerate}

\subsection*{Exercise 6.1.10} Let \( V  \) be an inner product space, and suppose that \( x  \) and \( y  \) are orthogonal vectors in \( V  \). Prove that \( \| x + y \|^{2} = \|x\|^{2} + \|y\|^{2}. \) Deduce the Pythagorean Theorem in \( \R^{2} \).
\begin{proof}
Let \( x,y \in V  \) be orthogonal. Thus, \( \langle x , y \rangle = 0 \) implies that
\begin{align*}
    \|x + y\|^{2}  &= \|x\|^{2} +  \langle y , x \rangle + \langle x , y \rangle + \|y\|^{2} \\
                  &= \|x\|^{2} + \overline{\langle x , y \rangle} + \langle x , y \rangle + \|y\|^{2} \\
                  &= \|x\|^{2} + \|y\|^{2}
\end{align*}
which is our desired result. Let \( x' = (x,0)  \) and \( y' = (0,y)  \) where \( x',y' \in \R^{2}  \). Denote \( \| x' + y' \| = z  \). Using the Euclidean definition of length along with our result, we find that 
\begin{align*}
    z^{2} = \| x' + y'\|^{2} &= \|x'\|^{2} + \|y'\|^{2}  \\
          &= | x |^{2} + | y |^{2} \\ 
          &= x^{2} + y^{2}.
\end{align*}
Thus, we have obtained the Pythagorean Theorem in \( \R^{2} \).
\end{proof}

\subsection*{Exercise 6.1.11} Prove the \textit{parallelogram law} on an inner product space \( V  \); that is, show that
\[  \|x + y \|^{2} + \|x - y \|^{2} = 2 \|x\|^{2} + 2 \|y\|^{2} \ \ \text{for all} \ \ x,y \in V. \]
\begin{proof}
Let \( x,y \in V  \). Using the definition of norm and properties of inner product, we get that
\begin{align*}
    \|x + y\|^{2} + \|x - y\|^{2} &= \langle x + y  ,  x + y  \rangle + \langle x -y  , x - y  \rangle \\
                                  &= \langle x + y  , x  \rangle + \langle x + y  , y \rangle + \langle x - y  , x  \rangle + \langle x - y  , -y  \rangle \\
                                  &= \langle x  , x \rangle + \langle y , x \rangle + \langle x  , y \rangle + \langle y  , y \rangle + \langle x  , x \rangle - \langle y , x \rangle \\
                                  &+ - \langle x , y \rangle + \langle y , y \rangle \\
                                  &= 2 \langle x , x \rangle + 2 \langle y , y \rangle \\
                                  &= 2 \|x\|^{2} + 2 \|y\|^{2}.
\end{align*}
Thus, we obtain
\[  \|x + y \|^{2} + \|x - y\|^{2} = 2 \|x\|^{2} + 2 \|y\|^{2}. \]
\end{proof}

\subsection*{Exercise 6.1.12}\label{Exercise 6.1.12} Let \( \{ {v}_{1}, {v}_{2}, \dots, {v}_{k} \}  \) be an orthogonal set in \( V  \), and let \( {a}_{1}, {a}_{2}, \dots, {a}_{k }  \) be scalars. Prove that
\[  \Big\| \sum_{ i=1 }^{ k  } {a}_{i} {v}_{i} \Big\|^{2} = \sum_{ i=1 }^{ k  } | {a}_{i} |^{2} \|{v}_{i}\|^{2}. \]
\begin{proof}
Let \( \{ {v}_{1}, {v}_{2}, \dots, {v}_{k } \}  \) be an orthogonal set in \( V  \) and let \( {a}_{1}, {a}_{2}, \dots, {a}_{k} \) be scalars. By the linearity in the first component and conjugate linearity in the second component, we obtain
\begin{align*}
    \Big\| \sum_{ i=1  }^{ k   } {a}_{i} {v}_{i} \Big\|^{2} &= \Big\langle \sum_{ i=1  }^{ k  } {a}_{i} {v}_{i}  ,  \sum_{ j=1  }^{ k  } {a}_{j} {v}_{j} \Big\rangle \\
                                                        &= \sum_{ i=1  }^{ k  } {a}_{i} \Big\langle  {v}_{i}  , \sum_{ j=1  }^{ k  } {a}_{j} {v}_{j}  \Big \rangle \\
                                                        &= \sum_{ i=1  }^{ k   } {a}_{i} \Big(  \sum_{ j=1  }^{ k  } \overline{{a}_{j}} \langle {v}_{i} ,  {v}_{j} \rangle   \Big).
\end{align*}
Since \( \{ {v}_{1}, {v}_{2}, \dots, {v}_{k } \}   \) is orthogonal in \( V  \), we have that \( \langle {v}_{i} , {v}_{j} \rangle = {\delta}_{ij} \) where \( {\delta}_{ij} = 1  \) or \( {\delta}_{ij} = 0   \) if \( i = j  \) or \( i \neq j  \), respectively. Thus, we have
\[  \Big\| \sum_{ i=1  }^{ k   } {a}_{i} {v}_{i} \Big\|^{2} = \sum_{ i=1  }^{ k  } {a}_{i} \overline{{a}_{i}} \langle {v}_{i}  , {v}_{i} \rangle = \sum_{ i=1  }^{ k  } | {a}_{i} | ^{2} \|{v}_{i}\|^{2}  \] which is our desired result.
\end{proof}

\subsection*{Exercise 6.1.13} Suppose that \( \langle \cdot  ,  \cdot  \rangle_{1} \) and \( \langle \cdot  ,  \cdot  \rangle_{2} \) is another inner product on \( V  \). Show that \( \langle \cdot ,  \cdot  \rangle \) is an inner product on \( V  \).
\begin{proof}
  Suppose that \( \langle \cdot  ,  \cdot  \rangle_{1} \) and \( \langle \cdot  ,  \cdot  \rangle_{2} \) is another inner product on \( V  \). Let \( x,y,z \in V  \) and \( c \in F  \).
  \begin{enumerate}
      \item[(a)] By linearity of the first component, we obtain
          \begin{align*}
              \langle x + z  , y  \rangle &= \langle x + z  , y  \rangle_{1} + {\langle x + z  , y  \rangle}_{2}  \\
                                          &= \langle x  , y \rangle_{1} + \langle z , y \rangle_{1}  + \langle x  , y \rangle_{2} + \langle z  , y \rangle_{2} \\
                                          &= ({\langle x , y \rangle}_{1} + {\langle x , y \rangle}_{2}) + (\langle z , y \rangle_{1} + \langle z  , y \rangle_{2}  ) \\
                                          &= \langle x  , y \rangle + \langle z , y \rangle.
          \end{align*}
        \item[(b)] Observe that
            \begin{align*}
                \langle cx , y \rangle &= \langle cx , y \rangle_{1} + \langle cx , y \rangle_{2} \\
                                       &=  c \langle x , y \rangle_{1} + c \langle x , y \rangle_{2} \\
                                       &= c (\langle x , y \rangle_{1} + \langle x , y \rangle_{2}) \\
                                       &=  c \langle x , y \rangle.
            \end{align*}
        \item[(c)] 
            \begin{align*}
                \overline{\langle x , y \rangle} &= \overline{\langle x , y \rangle_{1} + \langle x , y \rangle_{2}} \\
                                                 &=  \overline{\langle x , y \rangle_{1}} + \overline{\langle x , y \rangle_{2}} \\
                                                 &=  \langle y , x \rangle_{1} + \langle y , x \rangle_{2} \\
                                                 &= \langle y , x \rangle.
            \end{align*}
        \item[(d)] For any \( x \neq 0  \), we have
            \begin{align*}
                \langle x , x \rangle &= {\langle x , x \rangle}_{1} + {\langle x , x \rangle}_{2}. 
            \end{align*}
            Note that \( {\langle x , x \rangle}_{1} > 0  \) and \( \langle x , x \rangle_{2} > 0  \). Thus, we must also have \( \langle x  , x  \rangle > 0  \).
  \end{enumerate}
  We conclude that \( \langle \cdot  ,  \cdot  \rangle \) is a inner product on \( V  \).
\end{proof}

\subsection*{Exercise 6.1.14} Let \( A  \) and \( B  \) be \( n \times n  \) matrices, and let \( c  \) be a scalar. Prove that \( (A + cB)^{*} = A^{*} + \overline{c} B^{*} \).
\begin{proof}
Let \( A  \) and \( B  \) be \( n \times n  \) matrices and let \( c \in F  \). By using the definition of adjoint, we get that for all \( i,j \) we get
\begin{align*}
    (A + cB)_{ij}^{*} &= \overline{{(A + cB)}_{ji}}   \\
                        &=  \overline{{A}_{ji} + c {B}_{ji}} \\
                        &= \overline{{A}_{ji}} + \overline{c} \overline{{B}_{ji}} \\
                        &=  {(A^{*})}_{ij} + \overline{c} {(B^{*})}_{ij}.
\end{align*} 
Thus, we conclude that \( (A + cB)^{*} = A^{*} + \overline{c} B^{*} \).
\end{proof}

\subsection*{Exercise 6.1.15} 
\begin{enumerate}
    \item[(a)] Prove that if \( V  \) is an inner product space, then \( | \langle x , y \rangle |  = \| x \| \cdot \| y \| \) if and only if one of the vectors \( x  \) or \( y  \) is a multiple of the other. \textit{Hint:} If the identity holds and \(  y \neq 0  \), let  
        \[  a = \frac{ \langle x , y \rangle }{ \|y\|^{2} },  \]
        and let \( z = x - ay  \). Prove that \( y  \) and \( z  \) are orthogonal and 
        \[  | a  |  = \frac{ \|x\| }{ \|y\| }. \]
        Then apply Exercise 10 to \( \|x\|^{2} = \|ay + z\|^{2} \) to obtain \( \|z\| = 0  \). 
    \begin{proof}
    Suppose \( V  \) is an inner product space. For the forwards direction, assume \( |  \langle  x  , y \rangle |  = \| x \| \cdot \| y\| \). We will show that either \( x  \) or \( y  \) is a multiple of the other. Observe that \( y \neq 0  \) implies that
    \begin{align*}
        | \langle x , y \rangle | = \|x\| \cdot \|y\| &\iff \frac{ | \langle x , y \rangle |  }{ \|y\|^{2} } \cdot \|y\| = \|x\|.  
    \end{align*}
    We can see a similar process proves the other case
    \begin{align*}
        | \langle x , y \rangle | = \|x\| \cdot \|y\| &\iff \frac{ | \langle x , y \rangle |  }{ \|x\|^{2} } \cdot \|x\| = \|y\|.  
    \end{align*}
    For the backwards direction, suppose either \( x  \) or \( y  \) is a multiple of the other. Thus, for some \( c \neq 0  \), we have
    \begin{align*}
        | \langle x , y \rangle |  &= \sqrt{ \overline{\langle x , y \rangle} \cdot \langle x , y \rangle }  \\
                                   &= \sqrt{ \overline{\langle cy , y \rangle} \cdot \langle x , \frac{ 1 }{ c } x  \rangle } \\
                                   &= \sqrt{ \overline{c} \cdot \frac{ 1 }{ \overline{c} } \cdot \overline{\langle y , y \rangle}  \langle x , x \rangle } \\
                                   &= \sqrt{ \langle y , y \rangle \cdot \langle x , x \rangle }  \\
                                   &=  \sqrt{ \langle y , y \rangle } \cdot \sqrt{ \langle x , x \rangle } \\
                                   &= \|y\| \cdot \|x\| = \|x\| \cdot \|y\|.
    \end{align*}
    Note that the other case is similar. Thus, we have
    \[  | \langle x , y \rangle | = \|x\| \cdot \|y\|. \]

    Now, let \( y \neq 0  \) and \( z = x - ay \). Then set 
    \[  a  = \frac{ \langle x , y \rangle }{ \|y\|^{2} }. \]

    We will show that \( y  \) and \( z  \) are orthogonal and that 
    \[  | a  |  = \frac{ \|x\| }{ \|y\| }. \]
    It suffices to show that \( \langle y , z \rangle = 0  \). Observe 
    \begin{align*}
        \langle y , z \rangle &= \langle y , x - ay \rangle \\
                              &= \langle y , x \rangle - \overline{a} \langle y , y \rangle \\
                              &=  \overline{\langle x , y \rangle} - \overline{a} \|y\|^{2}.
    \end{align*}
    Note that we have
    \[  a = \frac{ \langle x , y \rangle }{  \|y\|^{2} }  \iff \langle x , y \rangle = a \|y\|^{2} \]
    and
    \[  \overline{ \|y\|^{2} } = \overline{\langle y , y \rangle} = \langle y , y \rangle. \]
    So, we must have that 
    \[  \langle y , z \rangle = \overline{\langle x , y \rangle} - \overline{a} \|y\|^{2} =  \overline{a} \|y\|^{2} - \overline{a} \|y\|^{2} = 0. \]
    Thus, \( y  \) and \( z  \) are orthogonal. Using  
    \[  a = \frac{ \langle x , y \rangle }{  \|y\|^{2} }, \]
    we see that 
    \begin{align*}
        | \langle x , y \rangle | = \|x\| \cdot \|y\| &\iff  \langle x , y \rangle \cdot \overline{\langle x , y \rangle} = \|x\|^{2} \cdot \|y\|^{2} \\
                                                      &\iff | a |^{2} \|y\|^{4} = \|x\|^{2} \cdot \|y\|^{2} \\ 
                                                      &\iff | a |^{2} = \frac{ \|x\|^{2} }{ \|y\|^{2} } \\
                                                      &\iff | a |  = \frac{ \|x\| }{ \|y\| }.
    \end{align*}

    Finally, we show that \( \|z\| = 0  \) using Exercise 10. Since \( \|x\|^{2} = \|ay + z \|^{2} \), we see that
    \[ \|x\|^{2} =  \|ay  + z \|^{2} = a^{2} \|y\|^{2} + \|z\|^{2}. \]
    Since \( | a  |  = \|x\| / \|y\| \), we can re-write the statement above into
    \[ a^{2} \|y\|^{2} = a^{2} \|y\|^{2} + \|z\|^{2} \iff \|z\|^{2} = 0.  \]
    Thus, we get \( \|z\| = 0  \) which is our desired result.
    \[   \]
    \end{proof}
    \item[(b)] Derive a similar result for the equality \( \|x + y\| = \|x\| + \|y\| \), and generalize it to the case of \( n  \) vectors.
        \begin{proof}
        
        \end{proof}
\end{enumerate}

\subsection*{Exercise 6.1.16} 
\begin{enumerate}
    \item[(a)] Show that the vector space \( H  \) with \( \langle  \cdot  ,  \cdot  \rangle \) defined on page 332 is an inner product space.
        \begin{proof}
            Let \( f,g,h \in H  \). Then
        \begin{enumerate}
            \item[(a)] 
                \begin{align*}
                    \langle  f + h  , g  \rangle &= \frac{ 1  }{  2 \pi  } \int_{ 0  }^{  2 \pi  } (f + h)(t) \overline{g(t)} \ dt  \\
                                                 &= \frac{ 1 }{  2 \pi }  \int_{ 0 }^{ 2 \pi  } (f(t) + h(t)) \overline{g(t)} \ dt \\
                                                 &= \frac{ 1 }{ 2 \pi  }  \int_{ 0 }^{  2\pi  } (f(t)\overline{g(t)} + h(t) \overline{g(t)}) \ dt \\
                                                 &= \frac{ 1 }{ 2 \pi  } \int_{ 0 }^{ 2 \pi  } \overline{g(t)} \ dt + \frac{ 1 }{ 2 \pi  }  \int_{ 0 }^{  2 \pi } h(t) \overline{g(t)}  \ dt \\
                                                 &= \langle f , g \rangle + \langle h   , g \rangle.
                \end{align*}
            \item[(b)] For \( c \in F  \), we have 
                \begin{align*}
                    \langle cf , g \rangle &= \frac{ 1 }{ 2\pi } \int_{ 0 }^{ 2 \pi } (cf)(t) \overline{g(t)} \ dt \\
                                           &= \frac{ 1  }{ 2 \pi  }  \int_{ 0 }^{ 2 \pi }  c f(t) \overline{g(t)} \ dt \\
                                           &= c \cdot  \frac{ 1 }{ 2 \pi }  \int_{ 0 }^{  2\pi } f(t) \overline{g(t)}   \ dt \\
                                           &=  c \cdot \langle f  , g \rangle.
                \end{align*}
            \item[(c)] 
                \begin{align*}
                    \overline{\langle f , g \rangle} &= \frac{ 1 }{ 2 \pi  } \overline{\int_{ 0 }^{ 2 \pi } f(t) g(t)  \ dt } \\
                                                     &= \frac{ 1  }{ 2 \pi }  \int_{ 0 }^{ 2 \pi }  \overline{f(t) \overline{g(t)}} \ dt \\
                                                     &= \frac{ 1 }{ 2 \pi }  \int_{ 0 }^{ 2 \pi }  \overline{f(t)} g(t) \ dt \tag{\(  \overline{\overline{g(t)}} = g(t)\)} \\
                                                     &= \frac{ 1 }{ 2 \pi }  \int_{ 0 }^{ 2 \pi } g(t) \overline{f(t)}  \ dt  \\
                                                     &= \frac{ 1 }{ 2 \pi }  \int_{ 0 }^{ 2 \pi }  \overline{f(t)} g(t) \ dt \\
                                                     &= \langle g , f \rangle.
                \end{align*}
            \item[(d)] For \( f > 0  \), we know that \( \int_{ 0 }^{ 2 \pi  } f  \ dt > 0  \) by our integration properties. Thus, we have
                \begin{align*}
                    \langle f , f \rangle &= \frac{ 1  }{ 2 \pi } \int_{ 0  }^{ 2 \pi }  f(t) \overline{f(t)} \ dt > 0. 
                \end{align*}
        \end{enumerate} 
    We conclude that \( \langle \cdot  ,  \cdot  \rangle \) satisfies all properties of an inner product.
        \end{proof}
    \item[(b)] Let \( V = C([0,1]) \), and define 
        \[  \langle f , g \rangle = \int_{ 0 }^{ 1 / 2  }  f(t) g(t)  \ dt. \]
        Is this an inner product on \( V  \)?
        \begin{proof}
        This is not an inner product because it fails property (c) of the definition of inner product.
        \end{proof}
\end{enumerate}

\subsection*{Exercise 6.1.17} Let \( T  \) be a linear operator on an inner product space \( V  \), and suppose that \( \|T(x)\| = \|x \|  \) for all \( x  \). Prove that \( T  \) is injective.
\begin{proof}
Let \( x,y \in V  \) and suppose \( T(x) = T(y) \). Taking advantage of the linearity of \( T  \), we can write 
\[  T(x) = T(y) \iff T(x - y) = 0.  \]
Using our assumption, we see that
\[ 0 = \|T(x-y)\| = \|x-y\|.  \]
By using the definition of norm, we can see further that
\[  \| x - y \|^{2} = 0 \iff \langle x - y  ,  x - y  \rangle = 0   \]
Since \( V  \) is an inner product space, we can see that the above is true if and only if \( x - y = 0  \) and thus \( x = y  \). Therefore, we conclude that \( T  \) is injective.
\end{proof}

\subsection*{Exercise 6.1.18} Let \( V  \) be a vector space over \( F  \), where \( F = \R  \) or \( F = \C  \), and let \( W  \) be an inner product space over \( F  \) with inner product \( \langle \cdot  ,  \cdot  \rangle \). If \( T: V \to W  \) is linear, prove that \( \langle x , y \rangle' = \langle T(x)  , T(y) \rangle \) defines an inner product on \( V  \) if and only if \( T  \) is one-to-one.
\begin{proof}
    Let \( x,y \in V  \). For the forwards direction, suppose that \( \langle x , y \rangle' = \langle  T(x)  , T(y) \rangle \) defines an inner product on \( V  \). Suppose that \( T(x) = T(y) \). Consider the norm
    \[  \|T(x)\| = \|T(y)\|.  \]
Since \( \langle x , y \rangle'  \) defines an inner product on \( V  \), we can write
\begin{align*}
    \|T(x) \|^{2} &= \|T(y)\|^{2} \\
    \langle T(x)  , T(x)  \rangle &= \langle T(y) , T(y) \rangle\\
    \langle T(x)  , T(x)  \rangle &= \langle T(x) , T(y) \rangle\\
    \langle x , x \rangle' &= \langle x , y \rangle'.
\end{align*}
Using part (e) Theorem 6.1, we have that \( x = y  \). Thus, \( T  \) is one-to-one. 

For the backwards direction, suppose \( T  \) is one-to-one. We need to show that \( \langle \cdot , \cdot \rangle' \) defines an inner product on \( V  \). Let \( x,y,z \in V  \) and let \( c \in F  \). Suppose \( T  \) is linear and \( \langle \cdot ,  \cdot  \rangle  \) defines an inner product over \( W  \). Then
\begin{enumerate}
    \item[(a)] 
        \begin{align*}
            \langle x + y  ,  z  \rangle' &= \langle  T(x+y)  , T(z) \rangle \\
                                          &=  \langle T(x) + T(y)  , T(z) \rangle \\
                                          &= \langle T(x)  , T(y) \rangle + \langle T(z)  , T(y)   \rangle \\
                                          &=  \langle x , y \rangle' + \langle z , y \rangle'.
        \end{align*}
    \item[(b)] 
        \begin{align*}
            \langle cx , y \rangle' &= \langle  T(cx)  , T(y) \rangle  \\
                                    &=  \langle c T(x)  , T(y) \rangle \\
                                    &=  c \langle T(x)  , T(y) \rangle \\
                                    &=  c \langle x  , y \rangle'.
        \end{align*}
    \item[(c)] 
        \begin{align*}
            \overline{\langle x , y \rangle'} &= \overline{\langle T(x)  , T(y) \rangle}  \\
                                              &=  \langle T(y)  , T(x) \rangle \\
                                              &=  \langle y , x \rangle'.
        \end{align*}
    \item[(d)] Suppose \( x \neq 0  \). Since \( \langle \cdot , \cdot \rangle \) is an inner product on \( W  \), we have 
        \[  \langle x , x \rangle' = \langle T(x)  , T(x)  \rangle > 0. \]
\end{enumerate}
Thus, we have that \( \langle \cdot , \cdot \rangle'  \) is an inner product on \( V  \).
\end{proof}

\subsection*{Exercise 6.1.19} Let \( V  \) be an inner product space. Prove that
\begin{enumerate}
    \item[(a)] \( \|x \pm y\|^{2} = \|x\|^{2} \pm 2 \Re \langle x , y \rangle + \|y\|^{2}  \) for all \( x,y \in V  \), where \( \Re \langle x , y \rangle \) denotes the real part of the complex number \( \langle x , y \rangle \).
        \begin{proof}
        Let \( x,y \in V  \) where \( V  \) is an inner product space. Note that 
        \[  \|x + y\|^{2} = \|x\|^{2} + 2 \Re \langle x , y \rangle + \|y\|^{2} \]
        by the proof for part (d) of Theorem 6.2. We will show the other case. By the properties found in Theorem 6.1, we have
        \begin{align*}
            \|x - y\|^{2} &= \langle x - y  ,  x - y  \rangle \\
                          &= \langle x , x \rangle - \langle x , y \rangle - ( \langle y , x \rangle)  \\
                          &= \langle x , x \rangle - (\langle x , y \rangle + \overline{\langle x , y \rangle} ) + \langle y , y \rangle \\
                          &= \langle x , x \rangle - 2 \Re \langle x , y \rangle + \|y\|^{2}.
        \end{align*}
        \end{proof}
    \item[(b)] \( | \|x\| - \|y\| |  \leq \|x - y\| \) for all \( x,y \in V  \).
        \begin{proof}
            Let \( x,y \in V  \). Using part (a) and part (c) of Theorem 6.2, we can see that
            \begin{align*}
               \|x - y\|^{2}  &= \|x\|^{2} - 2 \Re \langle x , y \rangle + \|y\|^{2} \\
                              &\geq \|x\|^{2} - 2 |  \langle x , y \rangle | + \|y\|^{2} \\
                              &\geq \|x\|^{2} - 2 \|x\| \cdot \|y\| + \|y\|^{2} \tag{\( -2 | \langle x , y \rangle |  \geq -2 \|x\| \cdot \|y\| \)} \\ 
                              &= (\|x\| - \|y\|)^{2}.
            \end{align*}
            Taking the square root of both sides of our inequality, we obtain
            \[  \|x - y \| \geq |  \| x \| - \|y\| |. \]
            Note that 
            \[ \sqrt{ (\|x\| - \|y\|)^{2} } = | \|x\| - \|y\| |.    \]
        \end{proof}
\end{enumerate}


\subsection*{Exercise 6.1.20} Let \( V \) be an inner product space over \( F  \). Prove the \textit{polar identities}: For all \( x,y \in V  \),  
\begin{enumerate}
    \item[(a)] \( \langle x , y \rangle = \frac{ 1 }{ 4 }  \|x + y \|^{2} - \frac{ 1 }{ 4 }  \|x - y\|^{2} \) if \( F = \R  \).
        \begin{proof}
        Let \( x,y \in V  \) and let \( F = \R  \). Using part (a) of Exercise 19, we can write
        \begin{align*}
            \frac{ 1 }{ 4 }  \|x + y\|^{2} - \frac{ 1 }{ 4 }  \|x + y\|^{2} &= \frac{ 1 }{ 4 }  \Big(  \|x\|^{2} + 2 \langle x , y \rangle + \|y\|^{2}  \Big)  \\
                                                                            &- \frac{ 1 }{ 4 }  \Big(  \|x\|^{2} - 2 \langle x , y \rangle + \|y\|^{2} \Big) \\
                                                                            &= \frac{ 1 }{ 4 }  \cdot 4 \langle x , y \rangle = \langle x , y \rangle.
        \end{align*}
        Thus, we have that
        \[  \langle x , y \rangle = \frac{ 1 }{ 4 }  \|x + y\|^{2} - \frac{ 1 }{ 4 }  \|x + y\|^{2}. \]
        \end{proof}
    \item[(b)] \( \langle x , y \rangle = \frac{ 1 }{ 4 }  \sum_{ k=1  }^{ 4  } i^{k} \|x + i^{k} y \|^{2}  \) if \( F = \C  \), where \( i^{2} = - 1  \).
        \begin{proof}
        Let \( x,y \in V  \) and \( F = \C  \). Using part (a), we can write
        \begin{align*}
            \frac{ 1 }{ 4 } \sum_{ k=1  }^{ 4  } i^{k } \|x + i^{k } y\|^{2} &= \frac{ 1 }{ 4 }  \Big[ i \|x + iy\|^{2} + i^{2} \|x + i^{2} y\|^{2} \\ 
                                                                             &+ i^{3} \|x + i^{3} y\|^{2} + i^{4} \|x + i^{4} y\|^{2} \Big]  \\
                                                                             &= \frac{ 1 }{ 4 } \Big[   i \|x + iy \|^{2} - \|x - y \|^{2} - i \|x - iy\|^{2} + \|x + y\|^{2} \Big] \\
                                                                             &= \frac{ 1 }{ 4 }  \Big[ \|x + y\|^{2} - \|x - y \|^{2} \Big] + \frac{ i }{ 4 }  \Big[ \|x + iy\|^{2} - \|x - iy\|^{2} \Big]
        \end{align*}
        where
        \[  \Re \langle x , y \rangle = \frac{ 1 }{ 4 }  \Big[ \|x + y \|^{2} - \|x - y\|^{2} \Big] \tag{part(a)} \]
        and 
        \[ \Im \langle x , y \rangle = \frac{ 1 }{ 4 }  \Big[ \|x + iy\|^{2} - \|x - iy\|^{2} \Big].   \]
        Thus, we have that
        \[  \langle x , y \rangle = \frac{ 1 }{ 4 }  \sum_{ k=1  }^{ 4  } i^{k } \|x + i^{k } y \|^{2}. \]
        \end{proof}
\end{enumerate}


\subsection*{Exercise 6.1.21} Let \( A  \) be an \( n \times n  \) matrix. Define
\[  {A}_{1} = \frac{ 1 }{ 2 }  (A + A^{*}) \ \ \text{and} \ \ {A}_{2} = \frac{ 1 }{ 2i }  (A - A^{*}). \]
\begin{enumerate}
    \item[(a)] Prove that \( {A}_{1}^{*} = {A}_{1} \), \( {A}_{2}^{*} = {A}_{2} \), and \( A = {A}_{1} + i {A}_{2} \). Would it be reasonable to define \( {A}_{1}  \) and \( {A}_{2}  \) to be the real and imaginary parts, respectively, of the matrix \( A  \)?
        \begin{proof}
        We will first show that \( {A}_{1}^{*} = {A}_{1} \) and \( {A}_{2}^{*} = {A}_{2} \). For all \( i , j \), we see that
        \begin{align*}
            ({A}_{1}^{*})_{ij} = \overline{{({A}_{1})}_{ji}} &= \frac{ 1 }{ 2 }  \overline{{(A + A^{*})}_{ji}}  \\
                                                             &= \frac{ 1 }{ 2 }  \overline{ ( {A}_{ji} + A_{ji}^{*} ) } \\
                                                             &= \frac{ 1 }{ 2 }  (\overline{{A}_{ji}} + \overline{{A}_{ji}^{*}} ) \\
                                                             &= \frac{ 1 }{ 2 }  ({(A^{*})}_{ij} + {A}_{ij}) \tag{\( \overline{\overline{{A}_{ij} }} = {A}_{ij} \)} \\
                                                             &= \frac{ 1 }{ 2 }  {(A^{*} + A )}_{ij} \\
                                                             &= ({A}_{1})_{ij}.
        \end{align*}
        Thus, \( {A}_{1}^{*} = {A}_{1} \). Similarly, we have
        \begin{align*}
            {({A}_{2}^{*})}_{ij} = \overline{{({A}_{2})}_{ji} } &= \overline{\frac{ 1 }{ 2i } (A - A^{*})_{ji}}  \\
                                                                &= - \frac{ 1 }{ 2i   } \overline{({A}_{ji} - {A}_{ji}^{*})} \\
                                                                &= - \frac{ 1 }{ 2i }  (\overline{{A}_{ji}} - \overline{{A}_{ji}^{*}}) \\ 
                                                                &= - \frac{1  }{ 2i } ( {A}_{ij}^{*} - {A}_{ij} ) \tag{\( \overline{\overline{{A}_{ij} }} = {A}_{ij} \)} \\
                                                                &= \frac{ 1 }{ 2i }  ({A}_{ij} - {A}_{ij}^{*}) \\
                                                                &= \frac{ 1 }{ 2i } {(A - A^{*})}_{ij} = {({A}_{2})}_{ij}.
        \end{align*}
        Thus, we have that \( {A}_{2}^{*} = {A}_{2} \).

        Now, we will show that \( A = {A}_{1} + i {A}_{2} \). Observe that
        \begin{align*}
           {A}_{1} + i {A}_{2} &= \frac{ 1 }{ 2 }  (A + A^{*}) + i \cdot \frac{ 1 }{ 2i  } (A - A^{*}) \\
                               &= \frac{ 1 }{ 2 }  A + \frac{ 1 }{ 2 }  A^{*} + \frac{ 1 }{ 2  } A -  \frac{ 1 }{ 2 }  A^{*} \\
                               &= A.
        \end{align*}
        Thus, we have \( A = {A}_{1} + i {A}_{2} \). This is a reasonable definition since we can just define the corresponds parts of the complex entries of \( A  \) as entries in \( \R  \) corresponding to two different matrices with each dedicated to the real and imaginary parts, respectively.
        \end{proof}
    \item[(b)] Let \( A  \) be an \( n \times n  \) matrix. Prove that the representation in (a) is unique. That is, prove that if \( A = {B}_{1} + i {B}_{2} \), where \( {B}_{1}^{*} = {B}_{1} \) and \( {B}_{2}^{*} = {B}_{2} \), then \( {B}_{1} = {A}_{1} \) and \( {B}_{2} = {A}_{2} \).
        \begin{proof}
        Let \( A  \) be an \( n \times n  \) matrix. Suppose there exists another representation of \( A  \) such that \( A = {B}_{1} + i {B}_{2}  \) where \( {B}_{1} = {B}_{1}^{*} \) and \( {B}_{2} = {B}_{2}^{*} \). Equating this representation to that of the one found in part (a), we obtain that
        \begin{align*}
            {A}_{1} + i {A}_{2} = {B}_{1} + i {B}_{2} \  \iff \ {A}_{1} = {A}_{2} \ , \ {B}_{1} = {B}_{2}.
        \end{align*}
        Thus, the representation in (a) is unique.
        \end{proof}
\end{enumerate}

\subsection*{Exercise 6.1.22} Let \( V  \) be a real or complex vector space (possibly infinite-dimensional), and let \( \beta \) be a basis for \( V  \). For \( x,y \in V  \) there exists \( {v}_{1}, {v}_{2}, \dots, {v}_{n} \in \beta  \) such that
\[  x = \sum_{ i=1  }^{ n  } {a}_{i} {v}_{i} \ \text{and} \ y = \sum_{ i=1  }^{ n } {b}_{i} {v}_{i}. \]
Define 
\[  \langle x , y \rangle = \sum_{ i=1  }^{ n } {a}_{i}\overline{b}_{i}. \]
\begin{enumerate}
    \item[(a)] Prove that \( \langle \cdot  ,  \cdot  \rangle \) is an inner product on \( V  \) and that \( \beta  \) is an orthonormal basis for \( V  \). Thus every real or complex vector space may be regarded as an inner product space.
        \begin{proof}
        Let \( \beta \) be an orthonormal basis for \( V  \). We will show that \( \langle \cdot  , \cdot  \rangle  \) is an inner product on \( V  \). Let \( x,y,z \in V  \) and \( c \in F  \). 
        \begin{enumerate}
            \item[(a)] We can find scalars \( {a}_{i}, {b}_{i}, {c}_{i} \in F  \) and \( {v}_{i} \in \beta  \) for \( 1 \leq i \leq n  \) such that 
                \[  x = \sum_{ i=1  }^{ n  } {a}_{i} {v}_{i},  \ \  y = \sum_{ i=1  }^{ n } {b}_{i} {v}_{i},  \ \ z = \sum_{ i=1  }^{ n } {c}_{i} {v}_{i}. \]
                Using our definition of \( \langle \cdot  ,  \cdot \rangle \), we obtain that
                \begin{align*}
                    \langle x + y  , z  \rangle = \sum_{ i=1  }^{ n  } ({a}_{i} + {b}_{i}) \overline{{c}_{i}}  
                                                &= \sum_{ i=1  }^{ n  } ({a}_{i} \overline{{c}_{i}} + {b}_{i} \overline{{c}_{i}}) \\
                                                &= \sum_{ i=1  }^{ n  } {a}_{i} \overline{{c}_{i}} + \sum_{ i=1  }^{ n } {b}_{i} \overline{{c}_{i}} \\
                                                &= \langle x , z \rangle + \langle y , z \rangle.
                \end{align*}
                Thus, part (a) is satisfied.
            \item[(b)] Using the representation of \( x,y \in V  \), we obtain
                \begin{align*}
                    \langle cx ,  y  \rangle = \sum_{ i=1  }^{ n } c {a}_{i} \overline{{b}_{i}} 
                                             = c \sum_{ i=1  }^{ n } {a}_{i} \overline{{b}_{i}} 
                                             =  c \langle x , y \rangle.
                \end{align*}
                Thus, part (b) is satisfied.
            \item[(c)] 
                \begin{align*}
                   \overline{\langle x , y \rangle} = \overline{\sum_{ i=1 }^{ n } {a}_{i} \overline{{b}_{i}}} 
                                                    = \sum_{ i=1  }^{ n } \overline{{a}_{i} \overline{{b}_{i}}} 
                                                    &= \sum_{ i=1  }^{ n } \overline{{a}_{i}} {b}_{i} \tag{\( \overline{\overline{{b}_{i}}} = {b}_{i} \)} \\
                                                    &= \sum_{ i=1  }^{ n } {b}_{i} \overline{{a}_{i}}   \\
                                                    &= \langle y , x \rangle.
                \end{align*}
                Thus, part (d) is satisfied.
            \item[(d)] Suppose \( x \neq 0  \). Then by definition of \( \langle \cdot  ,  \cdot \rangle \), we must have
                \[  \langle x , x \rangle = \sum_{ i=1  }^{ n } {a}_{i} \overline{{a}_{i}} = \sum_{ i=1  }^{ n } | {a}_{i} |^{2} > 0.  \]
                Thus, part (d) is satisfied.
\end{enumerate}
        \end{proof}
    \item[(b)] Prove that if \( V = \R^{n}  \) or \( V = C^{n} \) and \( \beta  \) is the standard ordered basis, then the inner product defined above is the standard inner product.
        \begin{proof}
        Let \( F = \R^{n}  \) or \( F = \C^{n} \). Let \( x,y,z \in F^{n} \) and \( c \in F  \). Repeat the same process above but with fixing \( x = ({a}_{1}, {a}_{2}, \cdots, {a}_{n}) \), \( y = ({b}_{1}, {b}_{2}, \dots, {b}_{n})  \), and \( z = ({c}_{1}, {c}_{2}, \dots, {c}_{n}) \). Thus, the inner product above is the standard inner product.
        \end{proof}
\end{enumerate}

\subsection*{Exercise 6.1.23} Let \( V = F^{n} \), and let \( A \in {M}_{n \times n }(F) \). 
\begin{enumerate}
    \item[(a)] Prove that \( \langle x  , Ay \rangle = \langle A^{*} x  , y \rangle \) for all \( x,y \in V  \).
        \begin{proof}
        Let \( x,y \in V  \). By definition of the adjoint for matrices, we obtain, for all \( i,j \), through the linearity of the first component and the conjugate linearity of the second component that 
        \begin{align*}
            \langle x  ,  {A}_{ij} y  \rangle = \overline{{A}_{ij}} \langle x , y \rangle &= {(A^{*})}_{ji} \langle x , y \rangle  \\
                                                                                          &= \langle  (A^{*})_{ji} \  x  , y \rangle.
        \end{align*}
        Thus, we have that \( \langle x , Ay  \rangle = \langle A^{*} x  , y \rangle \) for all \( x,y \in V  \).
        \end{proof}
    \item[(b)] Suppose that for some \( B \in {M}_{n \times n}(F) \), we have \( \langle x  , Ay \rangle = \langle B x  , y \rangle \) for all \( x, y \in V  \). Prove that \( B = A^{*} \).
        \begin{proof}
            Let \( x,y \in V  \). By part (a), we have 
            \[  \langle x  , Ay  \rangle = \langle  A^{*} x  , y \rangle. \]
            So, by linearity of the first component, we find that
            \[  \langle B x  , y \rangle = \langle A^{*} x  , y \rangle \ \iff \ B \langle x , y \rangle  = A^{*} \langle x , y \rangle.  \]
            by assumption. If \( \langle x , y \rangle \neq 0  \), then \( B = A^{*} \). Otherwise, \( B  \) is just the zero matrix. Thus, we must have \( B = A^{*} \).
            
        \end{proof}
    \item[(c)] Let \( \alpha  \) be the standard ordered basis for \( V  \). For any orthonormal basis \( \beta  \) for \( V  \), let \( Q  \) be the \( n \times n  \) matrix whose columns are the vectors in \( \beta  \). Prove that \( Q^{*} = Q^{-1}  \).
    \begin{proof}
       Let \( V = F^{n} \) and \( \alpha  \) as the standard ordered basis for \( V  \). It suffices to show that \( Q Q^{*} = Q Q^{*} = I  \). Let \( \beta = \{ {v}_{1}, {v}_{2}, \dots, {v}_{n} \}  \) be an orthonormal basis for \( V \). Since the columns of \( Q  \) are just the vectors of \( \beta \), we can write 
        \begin{align*}
            {(Q Q^{*})}_{ij} &= \sum_{ k=1  }^{ n } {Q}_{ik } Q^{*}_{kj} \\
                             &= \sum_{ k=1  }^{ n } {Q}_{ik } \overline{{Q}_{jk }} \\
                             &= \langle {v}_{i} ,  {v}_{j} \rangle \\
                             &= {\delta}_{ij}
        \end{align*}
        Note that the above is equal to 1 whenever \( i = j  \) and \( 0  \) otherwise. So, we must have \( Q Q^{*} = I   \). Showing \( Q^{*} Q  \) is done similarly as above. Thus, we conclude that \( Q^{*} = Q^{-1} \).
     \end{proof}

     \item[(d)] Define linear operators \( T  \) and \( U  \) on \( V  \) by \( T(x) = Ax  \) and \( U(x) = A^{*}x  \). Show that \( [U]_{\beta} = [T]^{*}_{\beta} \) for any orthonormal basis \( \beta  \) for \( V  \).
        \begin{proof}
        Observe that \( T = {L}_{A} \) and \( U = {L}_{A^{*}} \) since \( T(x) = Ax  \) and \( U(x) = A^{*}x \). By Theorem 2.15, we can see that 
        \[  [T]_{\beta} = A \iff [T]_{\beta}^{*} = A^{*}.   \]
        But note that \(  [U]_{\beta} = A^{*} \). Thus, we conclude that
        \[  [T]_{\beta}^{*} = [U]_{\beta}.\]
        \end{proof}
\end{enumerate}

\begin{definition}[Norm Spaces]
    Let \( V  \) be a vector space over \( F  \), where \( F  \) is either \( \R  \) or \( \C \). Regardless of whether \( V  \) is or not an inner product space, we may still define a \textbf{norm} \( \|\cdot\| \) as a real-valued function on \( V  \) satisfying the following three conditions for all \( x,y \in V  \) and \( a \in F  \).
    \begin{enumerate}
        \item[(1)] \( \|x \| \geq 0  \), and \( \|x\| = 0  \) if and only if \( x = 0  \).
        \item[(2)] \( \|ax\| = | a | \|x\| \).
        \item[(3)] \( \|x + y\| \leq \|x\| + \|y\| \).
    \end{enumerate}
\end{definition}

\subsection*{Exercise 6.1.24} Prove that the following are norms on the given vector spaces \( V  \).
\begin{enumerate}
    \item[(a)] \( V = {M}_{m \times n}(F); \ \ \ \|A \| = \text{max}_{i,j} | {A}_{ij} |    \) \ \ \  for all \( A \in V  \).
        \begin{proof}
            Let \( A \in {M}_{m \times n }(F) \). For (1), let \( A \neq O  \) where \( O  \) is the zero matrix. Thus, \( | {A}_{ij} |  > 0  \) and so \( \|A \| = \max\limits_{i,j} | {A}_{ij} | > 0   \). On the other hand, \( A = O \) implies that \( {A}_{ij} = 0  \) for all \( i,j  \). This is true if and only if \( \|A \| = 0   \). Hence, (1) is satisfied.


            Let \( a \in F  \). By properties of \( \max\limits_{i,j}  \)
            \begin{align*}
                \|a A \| = \max\limits_{i,j} | (aA)_{ij} | &= \max\limits_{i,j} | a {A}_{ij} | \\ 
                                                           &= \max\limits_{i,j} | a | | {A}_{ij} | = | a | \max\limits_{i,j} | {A}_{ij} |  \\   
                                                           &= | a | \|A\|.
        \end{align*}
        Thus, part (2) is satisfied.

        Now, let \( A, B \in {M}_{m \times n}(F) \). Then by triangle inequality of the absolute value, we have for all \( i,j \)
        \begin{align*}
            \|A + B\| &= \max\limits_{i,j} | {(A + B)}_{ij}  |  \\
                      &= \max\limits_{i,j} | {A}_{ij} + {B}_{ij} | \\
                      &\leq \max\limits_{i,j} | {A}_{ij} |  + \max\limits_{i,j} | {B}_{ij} | = \|A \| + \|B\|.
        \end{align*}
        Thus, (3) is satisfied. We conclude that \( \|A\| = \max\limits_{i,j} | {A}_{ij} |  \) is a norm for all \( A \in V  \).
        \end{proof}
    \item[(b)] \( V = C([0,1]);  \ \ \  \|f\| = \max\limits_{t \in [0,1]} | f(t) |  \) for all \( f \in V  \). 
        \begin{proof}
            Let \( V = C([0,1]) \) and let \( f \in V  \). Suppose \( f \neq 0  \), then \( f(t) > 0  \) for all \( t \in [0,1] \). Thus, \( \|f\| = \max\limits_{t \in [0,1]} | f(t) | > 0   \). Clearly, \( f = 0  \) implies that for all \( t \in [0,1]  \), \( f(t) = 0  \). This is true if and only if \( \|f\| = 0  \). 

            Let \( a \in F  \). Then observe that
            \begin{align*}
                \|af\| = \max\limits_{t \in [0,1]} | (af)(t) | &= \max\limits_{t \in [0,1]} | a f(t) |   \\
                                                               &= \max\limits_{t \in [0,1]} | a |  | f(t) | \\
                                                               &= | a |  \max\limits_{ t \in [0,1]} | f(t) | = | a |  \|f\|.
            \end{align*}
            Thus, (2) is satisfied.

            Now, let \( f,g \in C([0,1]) \). By using the properties of the absolute value, we have
            \begin{align*}
                \|f + g\|  = \max\limits_{t \in [0,1]} | (f+g)(t) |  &= \max\limits_{t \in [0,1]} | f(t) + g(t) |  \\
                                                                     &\leq \max\limits_{t \in [0,1]} | f(t) |  + \max\limits_{t \in [0,1]} | g(t) | \\ 
                                                                     &= \|f\| + \|g\|. 
            \end{align*}
            Thus, we have that \( \|f +g\| \leq  \|f\| + \|g\| \) for all \( t \in [0,1] \). Thus, (3) is satisfied. Hence, \( \|f \| = \max\limits_{t \in [0,1]} | f(t) |  \) is a norm over \( V  \).
        \end{proof}
    \item[(c)] \( V = C([0,1]); \ \ \ \|f\| = \int_{ 0 }^{ 1 } | f(t) |  \ dt \) for all \( f \in V  \).
        \begin{proof}
            If \( f \geq 0  \), then \( | f |  \geq 0  \). Thus,
        \[  \|f \| = \int_{ 0 }^{ 1 }  | f(t) |  \ dt > 0  \]
        by our integration properties. Otherwise \( f = 0  \) implies that
        \[  \|f\| = \int_{ 0 }^{ 1 } | f(t) |  \ dt = 0. \]
        So, property (1) is satisfied. Conversely, \( f = 0  \). Since this is course on linear algebra, I will not bother with the technicalities of the argument. Thus, property (1) is satisfied.  

        Now, for property (2), let \( f \in C([0,1]) \) and \( a \in F  \). Then by our properties of integration, we must have
        \begin{align*}
            \|af\| = \int_{ 0 }^{ 1 } | (af)(t) |  \ dt &= \int_{ 0 }^{ 1 } | a f(t) |  \ dt \\  
                                                        &= \int_{ 0 }^{ 1 }  | a |  | f(t) |  \ dt \\
                                                        &= | a | \int_{ 0 }^{ 1 }  | f(t) |  \ dt \\
                                                        &= | a |  \|f\|.
        \end{align*}
        Thus, property (2) is satisfied.

        Now, let \( f,g \in C([0,1]) \). Then using the triangle inequality, we obtain
        \begin{align*}
            \| f + g \| = \int_{ 0 }^{ 1 } | (f+g)(t) |  \ dt &= \int_{ 0 }^{ 1 }  | f(t) + g(t) |  \ dt \\
                                                              &\leq \int_{ 0 }^{ 1 } | f(t) |  + | g(t) |  \ dt \\
                                                              &= \int_{ 0 }^{ 1 } f(t)  \ dt + \int_{ 0 }^{ 1 }  g(t) \ dt \\
                                                              &= \|f\| + \|g\|.
        \end{align*}
        Thus, \( \|f+g\| \leq \|f\| + \|g\| \) for all \( f,g \in C([0,1]) \) and hence we conclude that
        \[ \|f\| = \int_{ 0 }^{ 1 } | f(t) |  \ dt  \]
        defines a norm on \( V = C([0,1])  \).
        \end{proof}
\end{enumerate}

\subsection*{Exercise 6.1.25} Use Exercise 20 to show that there is no inner product \( \langle \cdot , \cdot \rangle \) on \( \R^{2} \) such that \( \|x\|^{2} = \langle x , x \rangle \) for all \( x \in \R^{2} \) if the norm is defined as in Exercise 24(d).
\begin{proof}

\end{proof}

\subsection*{Exercise 6.1.27} Let \( \| \cdot \|  \) be a norm on a real vector space \( V  \) satisfying the parallelogram law given in Exercise 11. Define 
\[  \langle  x , y \rangle = \frac{ 1 }{ 4 }  [ \|x + y\|^{2} - \|x - y\|^{2}]. \]
Prove that \( \langle \cdot , \cdot \rangle  \) defines an inner product on \( V  \) such that \( \|x \|^{2} = \langle x , x \rangle \) for all \( x \in V  \).

\begin{enumerate}
    \item[(a)] Prove \( \langle x , 2y \rangle = 2 \langle x , y \rangle \) for all \( x,y \in V  \).
        \begin{proof}
        
        \end{proof}
\end{enumerate}

\subsection*{Exercise 6.1.28} Let \( V  \) be a complex inner product space with an inner product \( \langle \cdot ,  \cdot \rangle \). Let \( [\cdot, \cdot] \) be the real-valued function such that \( [x,y] \) is the real part of the complex number \( \langle x , y \rangle \) for all \( x,y \in V  \). Prove that \( [\cdot, \cdot] \) is an inner product for \( V  \), where \( V  \) is regarded as a vector space over \( \R  \). Prove, furthermore, that \( [x,ix] = 0  \) for all \( x \in V  \).
\begin{proof}
Let \( x,y,z \in V  \) and \( c \in \R   \). 
\begin{enumerate}
    \item[(a)] Since \( \langle  \cdot  ,  \cdot \rangle \) is linear in the first component, we must have that
        \begin{align*}
            [x + y, z] = \Re \langle x + y , z  \rangle &= \Re (\langle x  , z \rangle + \langle y  , z \rangle) \\
                                                        &= \Re \langle x , z \rangle + \Re \langle y , z \rangle \\
                                                        &= [x,z] + [y,z].
        \end{align*}
    \item[(b)]
        \begin{align*}
            [cx, y] = \Re \langle cx , y \rangle &= \Re c \langle x , y \rangle  \\
                                                 &=  c \Re \langle x , y \rangle \\
                                                 &= c [x,y].
        \end{align*}
    \item[(c)] 
        \begin{align*}
            \overline{[x,y]} = \Re \overline{\langle x , y \rangle} = \Re \langle y , x \rangle  = [y,x].
        \end{align*}
    \item[(d)] Suppose that \( x \neq 0  \). Since \( \langle x , x \rangle > 0  \), we also have that
        \[  [x,x] = \Re \langle x , x \rangle > 0.  \]
        So, property (d) is satisfied. Now, let us show that \( [x,ix] = 0  \) for any \( x \in V  \). Now we can see that \( [ \cdot , \cdot ] \) is a inner product. Since \( \Re(i) = 0  \), we obtain that
        \begin{align*}
            [x,ix] = \Re \langle x , ix \rangle &= \Re (\overline{i} \langle x , x \rangle)\\
                   &= \Re(-i) \Re \langle x , x \rangle \\
                   &= 0 \cdot \langle x , x \rangle = 0.
        \end{align*}
        So, \( [x,ix] = 0  \) for all \( x \in V  \).
\end{enumerate}
\end{proof}

\subsection*{Exercise 6.1.29} Let \( V  \) be a vector space over \( \C  \), and suppose that \( [\cdot,\cdot] \) is a real inner product on \( V  \), where \( V  \) is regarded as a vector space over \( \R  \), such that \( [x,ix] = 0  \) for all \( x \in V  \). Let \( \langle \cdot , \cdot \rangle \) be the complex-valued function defined by
\[  \langle x , y \rangle = [x,y] + i [x,y] \ \ \text{for} \ x,y \in V. \]
Prove that \( \langle \cdot , \cdot \rangle  \) is a complex inner product on \( V  \).
\begin{proof}
    Let \( x,y,z \in V  \) and \( c \in F  \).
    \begin{enumerate}
        \item[(a)] Using the linearity of the first component of the inner product \( [\cdot, \cdot] \), we must have
            \begin{align*}
                \langle x + y  , z  \rangle &= [ x + y, z ] + i [x + y, iz] \\
                                            &=  ([x ,z] + [y,z]) + i ([x,iz] + [y,iz]) \\
                                            &= ([x,z] + i[x,iz]) + ([y,z] + i[x,iz]) \\
                                            &= \langle x , z \rangle + \langle y , z \rangle.
            \end{align*}
            So we have \( \langle x +y  , z  \rangle = \langle x  , z \rangle + \langle y , z \rangle \).
        \item[(b)] By the same reasoning, we also have
            \begin{align*}
                \langle cx  , y \rangle &= [cx,y] + i[cx,iy] \\
                                        &= c [x,y] + ci[ x,iy] \\
                                        &= c ([x,y] + i[x,iy]) \\
                                        &= c \langle x , y \rangle.
            \end{align*}
            So, part (b) is satisfied.
        \item[(c)] 
            \begin{align*}
                \overline{\langle x , y \rangle} &= \overline{[x,y] + i [x,iy]} \\
                                                 &= \overline{[x,y]} + \overline{i [x,iy]} \\
                                                 &= [y,x] + \overline{i} [iy,x] \\
                                                 &= [y,x] + i[y,ix] \\
                                                 &= \langle y , x \rangle.
            \end{align*}
        \item[(d)] Suppose \( x \neq 0  \). Since \( [x,ix] = 0  \) for any \( x \in V  \) and that \( [x,x] > 0  \) (Since \( [\cdot,\cdot]   \) is a real inner product for \( V  \)) , we can write
            \[   \langle x , x \rangle = [x,x] + i[x,ix] = [x,x] > 0.  \]
           Thus, property (d) is satisfied. Hence, we conclude that \( \langle \cdot ,  \cdot \rangle  \) is a complex inner product on \( V  \). 
    \end{enumerate}
\end{proof} 
