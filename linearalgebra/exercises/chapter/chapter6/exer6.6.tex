\section{Orthogonal Projections and the Spectral Theorem}

\subsection*{Exercise 6.6.4} Let \( W  \) be a finite-dimensional subspace of an inner product space \( V  \). Show that if \( T  \) is the orthogonal projection of \( V  \) on \( W  \), then \( I - T  \) is the orthogonal projection of \( V  \) on \( W^{\perp} \).
\begin{proof}
Suppose \( T  \) is the orthogonal projection of \( V  \) on \( W  \). Then, by definition, we have
\[  N(T) = R(T)^{\perp} \ \text{and} \ R(T) = N(T)^{\perp}. \] 
In order to show that \( I - T  \) is an orthogonal projection of \( V  \) on \( W  \), it suffices to show that \( I -T  \) is a projection and that 
\begin{center}
    \( R(I - T) = N(T) \) and \( N(I - T) = R(T) \).
\end{center}
Since \( T  \) is a projection of \( V  \) on \( W  \), we have \( T^{2} = T  \). So, observe that 
\[  (I-T)^{2} = I^{2} - 2IT + T^{2} = I - 2T + T = I -T.  \]
Thus, we have that \( I - T  \) is a projection. Let \( x \in R(I - T) \). Thus,  
\[  x = (I - T)(x) = I(x) - T(x) = x - T(x)  \]
which implies \( T(x) = 0  \). Thus, \( x \in N(T) \). Now, let \( x \in N(T) \). Then
\[  T(x) = 0 = I(x) - x \implies (I - T)(x) = x.  \]
Thus, \( x \in R(I -T)  \) since \( I - T  \) is a projection. Therefore, we conclude that \( R(I -T) = N(T) \). Now, let \( x \in N(I -T) \). Then we have \( (I - T)(x) = 0  \). So, \( T(x) = I(x) = x  \). Since \( T  \) is a projection, we must also have \(  x \in R(T)  \). On the other hand, let \( x \in R(T) \). Then \( T(x) = x   \) and so reversing the steps from the prior argument, we must have \( (I -T)(x) = 0  \). So, \( x \in N(T - I) \). Thus, we have \( N(I -T) = R(T) \). Thus, \( I -T   \) is an orthogonal projection. Note that \( R(T) = W  \) and so with our results, we must have
\[  R(I - T) = N(T) = R(T)^{\perp} = W^{\perp}. \]
Thus, \( I - T  \) is an orthogonal projection on \( W^{\perp} \).
\end{proof}

\subsection*{Exercise 6.6.5} Let \( T  \) be a linear operator on a finite-dimensional inner product space \( V  \).
\begin{enumerate}
    \item[(a)] If \( T  \) is an orthogonal projection, prove that \( \|T(x)\| \leq \|x\| \) for all \( x \in V  \). Give an example of a projection for which this inequality does not hold. What can be be concluded about a projection for which the inequality is actually an equality for \( x \in V  \).
        \begin{proof}
        
        \end{proof}
    \item[(b)] Suppose that \( T  \) is a projection such that \( \|T(x)\| \leq \|x\| \) for all \( x \in V  \). Prove that \( T  \) is an orthogonal projection.
        \begin{proof}
        
        \end{proof}
\end{enumerate}

\subsection*{Exercise 6.6.6} Let \( T \) be a normal operator on a finite-dimensional inner product space. Prove that if \( T  \) is a projection, then \( T  \) is also an orthogonal projection.
\begin{proof}

\end{proof}

\subsection*{Exercise 6.6.7} Let \( T  \) be a normal operator on a finite-dimensional complex inner product space \( V  \). Use the spectrael decomposition \( {\lambda}_{1} {T}_{1} + {\lambda}_{2} {T}_{2} + \cdots + {\lambda}_{k} {T}_{k} \) of \( T  \) to prove the following results.
\begin{enumerate}
    \item[(a)] If \( g \) is a polynomial, then 
        \[  g(T) = \sum_{ i=1  }^{ k  } g({\lambda}_{i}) {T}_{i}. \]
        \begin{proof}
        
        \end{proof}
    \item[(b)] If \( T^{n} = {T}_{0} \) for some \( n  \), then \( T = {T}_{0} \). 
        \begin{proof}
        
        \end{proof}
    \item[(c)] Let \( U  \) be a linear operator on \( V  \). Then \( U  \) commutes with \( T  \) if and only if \( U  \) commutes with each \( {T}_{i} \). 
        \begin{proof}
        
        \end{proof}
    \item[(d)] There exists a normal operator \( U  \) on \( V  \) such that \( U^{2} = T  \).
        \begin{proof}
        
        \end{proof}
    \item[(e)] \( T  \) is invertible if and only if \( {\lambda}_{i} \neq 0  \) for \( 1 \leq i \leq k  \).
        \begin{proof}
        
        \end{proof}
    \item[(f)] \( T  \) is a projection if and only if every eigenvalue of \( T  \) is \( 1  \) or \( 0  \).
        \begin{proof}
        
        \end{proof}
    \item[(g)] \( T = - T^{*} \) if and only if every \( {\lambda}_{i} \) is an imaginary number.
        \begin{proof}
        
        \end{proof}
\end{enumerate}
