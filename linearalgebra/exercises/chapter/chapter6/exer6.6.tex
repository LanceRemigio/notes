\section{Orthogonal Projections and the Spectral Theorem}

\subsection*{Exercise 6.6.4} Let \( W  \) be a finite-dimensional subspace of an inner product space \( V  \). Show that if \( T  \) is the orthogonal projection of \( V  \) on \( W  \), then \( I - T  \) is the orthogonal projection of \( V  \) on \( W^{\perp} \).
\begin{proof}
Suppose \( T  \) is the orthogonal projection of \( V  \) on \( W  \). Then, by definition, we have
\[  N(T) = R(T)^{\perp} \ \text{and} \ R(T) = N(T)^{\perp}. \] 
In order to show that \( I - T  \) is an orthogonal projection of \( V  \) on \( W  \), it suffices to show that \( I -T  \) is a projection and that 
\begin{center}
    \( R(I - T) = N(T) \) and \( N(I - T) = R(T) \).
\end{center}
Since \( T  \) is a projection of \( V  \) on \( W  \), we have \( T^{2} = T  \). So, observe that 
\[  (I-T)^{2} = I^{2} - 2IT + T^{2} = I - 2T + T = I -T.  \]
Thus, we have that \( I - T  \) is a projection. Let \( x \in R(I - T) \). Thus,  
\[  x = (I - T)(x) = I(x) - T(x) = x - T(x)  \]
which implies \( T(x) = 0  \). Thus, \( x \in N(T) \). Now, let \( x \in N(T) \). Then
\[  T(x) = 0 = I(x) - x \implies (I - T)(x) = x.  \]
Thus, \( x \in R(I -T)  \) since \( I - T  \) is a projection. Therefore, we conclude that \( R(I -T) = N(T) \). Now, let \( x \in N(I -T) \). Then we have \( (I - T)(x) = 0  \). So, \( T(x) = I(x) = x  \). Since \( T  \) is a projection, we must also have \(  x \in R(T)  \). On the other hand, let \( x \in R(T) \). Then \( T(x) = x   \) and so reversing the steps from the prior argument, we must have \( (I -T)(x) = 0  \). So, \( x \in N(T - I) \). Thus, we have \( N(I -T) = R(T) \). Thus, \( I -T   \) is an orthogonal projection. Note that \( R(T) = W  \) and so with our results, we must have
\[  R(I - T) = N(T) = R(T)^{\perp} = W^{\perp}. \]
Thus, \( I - T  \) is an orthogonal projection on \( W^{\perp} \).
\end{proof}

\subsection*{Exercise 6.6.5} Let \( T  \) be a linear operator on a finite-dimensional inner product space \( V  \).
\begin{enumerate}
    \item[(a)] If \( T  \) is an orthogonal projection, prove that \( \|T(x)\| \leq \|x\| \) for all \( x \in V  \). Give an example of a projection for which this inequality does not hold. What can be be concluded about a projection for which the inequality is actually an equality for \( x \in V  \).
        \begin{proof}
        Suppose \( T  \) is an orthogonal projection, we have
        \begin{center}
            \( R(T) = N(T)^{\perp} \) and \( N(T) = R(T)^{\perp} \)
        \end{center}
        with \( V = R(T) \oplus R(T)^{\perp} \). Let \( x \in V  \). Then \( x = u + z  \) where \( u \in R(T) \) and \( z \in R(T)^{\perp} \). Since \( u  \) and \( z  \) are orthogonal to each other, we can use Exercise 10 from Section 6.1 to write
        \[  \|x\|^{2} = \| u   + z \|^{2} = \|u\|^{2} + \|z\|^{2} \geq \|u\|^{2} =  \|T(x)\|^{2}. \]
        So, we conclude that \( \|T(x)\| \leq \|x\|  \). If \( \|T(x)\| = \|x\|  \), then \( T  \) must be unitary.
        \end{proof}
    \item[(b)] Suppose that \( T  \) is a projection such that \( \|T(x)\| \leq \|x\| \) for all \( x \in V  \). Prove that \( T  \) is an orthogonal projection.
        \begin{proof}
        Since \( T \) is a projection with the property that \( \|T(x)\| \leq \|x \| \) for all \( x \in V  \), we have that \( N(T) = R(T)^{\perp} \) by Exercise 10 of Section 6.2. We only have to show that \( R(T) = N(T)^{\perp} \). Since \( V  \) is finite-dimensional, we must have
        \[  R(T) = (R(T)^{\perp})^{\perp} = N(T)^{\perp}.   \]
        Thus, we see that \( T  \) is an orthogonal projection.
        \end{proof}
\end{enumerate}

\subsection*{Exercise 6.6.6} Let \( T \) be a normal operator on a finite-dimensional inner product space. Prove that if \( T  \) is a projection, then \( T  \) is also an orthogonal projection.
\begin{proof}
Suppose \( T  \) is a projection. Then \( T^{2} = T  \). In order to show that \( T  \) is an orthogonal projection, we have 
\[  R(T) = N(T)^{\perp} \ \ \text{and} \ \ N(T) = R(T)^{\perp}.  \]
Let \( x \in R(T) \). Since \( T  \) is a projection, \( T(x) = x  \). We must show that \( x \in N(T)^{\perp} \). Let \( y \in N(T) \). Since \( T  \) is a projection, then   
\begin{align*}
    \langle x , y \rangle = \langle T(x) , y \rangle = \langle x  , T^{*}(y) \rangle.  
\end{align*}
By Exercise 6.4.9, we must have that \( T^{*}(y) = 0  \) since \( N(T) = N(T^{*}) \). So, \( \langle x  ,  T^{*}(y) \rangle = \langle x  , 0  \rangle = 0  \) implies that \( \langle x  , y \rangle = 0 \). So, \( x \in R(T)^{\perp} \). Now, let \( x \in N(T)^{\perp} \). Then we must show that \( x = T(x) \) so that \( x \in R(T) \). Consider the norm \( \|x  - T(x)\|^{2} \). Then observe that 
\begin{align*}
    \|x - T(x)\|^{2} &= \langle  x - T(x ) ,  x - T(x) \rangle \\
                     &= \langle x  ,  x - T(x) \rangle - \langle  T(x) ,   x - T(x) \rangle.
\end{align*}
Since \( x - T(x) \in N(T) \), the first term cancels out. With the second term, we can write
\[  \langle T(x)  ,  x - T(x) \rangle = \langle  x  ,  T^{*}(x - T(x)) \rangle. \]
Since \( N(T) = N(T^{*}) \), the equation above is equal to zero. Thus, we have \( \|x - T(x)\|^{2} = 0  \) which implies that \( x = T(x) \in R(T) \). Since \( V  \) is a finite-dimensional inner product space, we must have  
\[  N(T) = (N(T)^{\perp} )^{\perp} = R(T)^{\perp}  \]
and we are done.

\end{proof}

\subsection*{Exercise 6.6.7} Let \( T  \) be a normal operator on a finite-dimensional complex inner product space \( V  \). Use the spectrael decomposition \( {\lambda}_{1} {T}_{1} + {\lambda}_{2} {T}_{2} + \cdots + {\lambda}_{k} {T}_{k} \) of \( T  \) to prove the following results.
\begin{enumerate}
    \item[(a)] If \( g \) is a polynomial, then 
        \[  g(T) = \sum_{ i=1  }^{ k  } g({\lambda}_{i}) {T}_{i}. \]
        \begin{proof}
        Let \( T  \) be a normal operator and let 
        \[  T= {\lambda}_{1} {T}_{1} + {\lambda}_{2} {T}_{2} + \cdots + {\lambda}_{k} {T}_{k} \]
        be the spectral decomposition of \( T  \). Since \( T  \) is normal, we must have that
        \[  g(T) = T^{*} = \sum_{ i=1  }^{ k } g({\lambda}_{i}) {T}_{i}  \]
        by Corollary 1 to Theorem 6.18 where \( g({\lambda}_{i}) = \overline{{\lambda}_{i}} \).
        \end{proof}
    \item[(b)] If \( T^{n} = {T}_{0} \) for some \( n  \), then \( T = {T}_{0} \). 
        \begin{proof}
        Suppose \( T^{n} = {T}_{0} \) for some \( n \). Since \( V  \) is the direct sum of eigenspaces \( {W}_{i} \) for \( 1 \leq i \leq k \) of \( T \), let \( x = {x}_{1} + {x}_{2} + \cdots + {x}_{k} \) with \( {x}_{i} \in {W}_{i} \) for each \( 1 \leq i \leq k \). Let each \( {x}_{i} \) have a corresponding eigenvalue \( {\lambda}_{i} \). By Exercise 5.1.15, we can see that \( T^{n}({x}_{i}) = {\lambda}_{i}^{n} {x}_{i}  \). Since \( T^{n} = {T}_{0} \), notice that \( {\lambda}_{i}^{n} {x}_{i} = {T}_{0}({x}_{i}) = 0  \). Since \( {x}_{i} \neq 0  \), we have \( {\lambda}_{i}^{n} = 0    \) if and only if \( {\lambda}_{i} = 0  \) for each \( i  \). Using the spectral decomposition of \( T  \), we can see that
        \begin{align*}
            T &= {\lambda}_{1} {T}_{1} + {\lambda}_{2} {T}_{2}  + \cdots + {\lambda}_{k} {T}_{k}  \\
              &= 0 {T}_{1} + 0{T}_{1} + \cdots  + 0 {T}_{k} \\
              &= 0.
        \end{align*}
        Thus, we conclude that \( T = {T}_{0} \).
\end{proof}
    \item[(c)] Let \( U  \) be a linear operator on \( V  \). Then \( U  \) commutes with \( T  \) if and only if \( U  \) commutes with each \( {T}_{i} \). 
        \begin{proof}
        Suppose \( U  \) commutes with \( T  \). Then we have \( UT = TU  \). Since 
        \[  T = {\lambda}_{1} {T}_{1} + {\lambda}_{2} {T}_{2} + \cdots + {\lambda}_{k} {T}_{k},  \]
        we have 
        \[  UT = U \Big( \sum_{ i = 1   }^{ k  } {\lambda}_{i} {T}_{i} \Big) = \sum_{ i=1  }^{ k  } {\lambda}_{i} U {T}_{i} \]
        and 
        \[  TU = \Big(  \sum_{ i=1  }^{ k  } {\lambda}_{i} {T}_{i} \Big) U = \sum_{ i=1  }^{ k  } {\lambda}_{i} {T}_{i} U.  \]
        Since \( TU = UT  \), \( U  \) must commute with every \( {T}_{i} \).

        On the other hand, suppose \( U  \) commutes with every \( {T}_{i} \).

        \end{proof}
    \item[(d)] There exists a normal operator \( U  \) on \( V  \) such that \( U^{2} = T  \).
        \begin{proof}
        Define the linear operator \( U: V \to V  \) by 
        \[  U = \sum_{ i=1  }^{ k  } \sqrt{ {\lambda}_{i} } {T}_{i} \]
        with each \( {T}_{i} \) being an orthogonal operator. Since each \( {T}_{i} \) is self-adjoint and therefore normal as well as \( {T}_{i} {T}_{j} = {\delta}_{ij} {T}_{i}  \), we must have \( U U^{*} = U^{*} U   \). So, \( U  \) is normal. Furthermore, we have  
        \begin{align*}
            U^{2} &= \Big(  \sum_{ i=1  }^{ k  } \sqrt{ {\lambda}_{i} }  {T}_{i} \Big) \Big(  \sum_{ i=1  }^{ k  } \sqrt{ {\lambda}_{i} }  {T}_{i} \Big) \\
                  &= \sum_{ i=1  }^{ k  } (\sqrt{ {\lambda}_{i} } )^{2} {T}_{i} \\  
                  &=\sum_{ i=1  }^{ k  } {\lambda}_{i} {T}_{i} \\
                  &= T.
        \end{align*}
        So, we conclude that \( U^{2} = T  \).
        \end{proof}
    \item[(e)] \( T  \) is invertible if and only if \( {\lambda}_{i} \neq 0  \) for \( 1 \leq i \leq k  \).
        \begin{proof}
        Suppose \( T  \) is invertible. Since \( T  \) is normal, there exists an orthonormal basis consisting of eigenvectors corresponding to eigenvalues \( {\lambda}_{i}    \) for \( 1 \leq i \leq n  \) not all necessarily distinct. Let \( A = [T]_{\beta} \). Since \( T \) is invertible, \( [T]_{\beta}  \) is invertible. By Corollary to Theorem 4.7, we must have \( \text{det}(A) \neq  0  \). Thus, the eigenvalues lying on the diagonal of \( A  \) must be nonzero. So, \( {\lambda}_{i} \neq 0  \) for all \( 1 \leq i \leq k  \). We can prove the backwards result by reversing this argument.
        \end{proof}
    \item[(f)] \( T  \) is a projection if and only if every eigenvalue of \( T  \) is \( 1  \) or \( 0  \).
        \begin{proof}
            (\( \Leftarrow \)) Suppose that every eigenvalue of \( T  \) is 1 or 0. If \( {\lambda}_{i} = 0  \), then \( T  \) is just the zero transformation which is a projection. 
        On the other hand, suppose that \( {\lambda}_{i} = 1  \) for all \( i \). Using the spectral decomposition of \( T  \), we must have 
        \[  T = \sum_{ i=1  }^{ k  } {\lambda}_{i} {T}_{i} = \sum_{ i=1  }^{ k  } {T}_{i} = I. \]
        Since \( I  \) is a projection, we see that \( T  \) must be a projection as well.
        
        (\( \Rightarrow \)) Suppose that \( T  \) is a projection. Then \( T(x) = x  \) and \( x = {x}_{1} + {x}_{2} + \cdots + {x}_{k} \) where each \( {x}_{i} \in {W}_{i}  \) with \( {W}_{i} \) being an eigenspace of \( T  \). Since each \( {T}_{i} \) is an orthogonal projection, we must have
        \begin{align*}
            x  = T(x) &=  \Big( \sum_{ i=1  }^{ k  } {\lambda}_{i} {T}_{i} \Big)(x)  \\
                      &= \sum_{ i=1  }^{ k  } {\lambda}_{i} {T}_{i}(x) \\ 
                      &= \sum_{ i=1  }^{  k  } {\lambda}_{i} {x}_{i}. 
        \end{align*}
        Since \( x = {x}_{1} + {x}_{2} + \cdots + {x}_{k} \), each \( {\lambda}_{i} = 1  \) for all \( i \) by matching up corresponding coefficients. If \( x = 0  \), then we can see that each \( {\lambda}_{i} = 0  \) for all \( i \) and we are done.
        
        \end{proof}
    \item[(g)] \( T = - T^{*} \) if and only if every \( {\lambda}_{i} \) is an imaginary number.
        \begin{proof}
        Suppose \( T = - T^{*}  \) and let  
        \[ T = \sum_{ i=1  }^{ k  } {\lambda}_{i} {T}_{i}   \]
        be the spectral decomposition of \( T  \) with each \( {T}_{i} \) being an orthogonal projection. By Corollary 1 of the Spectral Theorem, we see that \( T^{*} = g(T)  \), for some polynomial \( g \), with \( g({\lambda}_{i}) = \overline{{\lambda}_{i}} \) for all \( i \). By part (a), we know that   
        \[ T^{*} =  g(T) = \sum_{ i=1  }^{ k  } g({\lambda}_{i}) {T}_{i} = \sum_{ i=1  }^{ k  } \overline{{\lambda}_{i}} {T}_{i}. \]
        Since \( T = - T^{*} \), we know that
        \[  \sum_{ i=1  }^{ k  } {\lambda}_{i} {T}_{i} = \sum_{ i=1  }^{ k } - \overline{{\lambda}_{i}} {T}_{i}.  \]
        So, we have \( - {\lambda}_{i} =  \overline{{\lambda}_{i}}  \) for all \( i \) by setting each corresponding coefficient equal to each other. But this means that each \( {\lambda}_{i} \) is an imaginary number.

        Conversely, suppose each \( {\lambda}_{i}  \) is an imaginary number. This implies that 
        \[  \overline{{\lambda}_{i}} = - {\lambda}_{i} \implies {\lambda}_{i} = - \overline{{\lambda}_{i}} \]
        for all \( i  \). Using the first corollary to the Spectral Theorem, we have \( g(T) = T^{*} \) with \( g({\lambda}_{i}) = \overline{{\lambda}_{i}} \) for some polynomial \( g \). So, we have 
        \begin{align*}
           T = \sum_{ i=1  }^{ k  } {\lambda}_{i} &= \sum_{ i=1  }^{ k  } - \overline{{\lambda}_{i}} {T}_{i}   \\
                                                  &= \sum_{ i=1  }^{ k  } - g({\lambda}_{i}) {T}_{i} \\
                                                  &= - \sum_{ i=1  }^{ k   } g({\lambda}_{i}) {T}_{i} \\
                                                  &= - T^{*}
        \end{align*}
        and we are done.
    \end{proof}
\end{enumerate}

\subsection*{Exercise 6.6.8} Use Corollary 1 of the Spectral Theorem to show that if \( T  \) is a normal operator on a complex finite-dimensional inner product space and \( U  \) is a linear operator that commutes with \( T  \), then \( U  \) commutes with \( T^{*} \). 
\begin{proof}
Suppose \( T  \) is a normal operator on a complex finite-dimensional inner product space and \( U  \) is a linear operator that commutes with \( T  \). We will show that \( U T^{*} = T^{*}U  \). By part (c), \( U  \) must commute with each \( {T}_{i} \). Using the first corollary of the Spectral Theorem, we have \( T^{*} = g(T) \) for some polynomial \( g \). So, observe that
\begin{align*}
    U T^{*}  = U (g(T)) &= U \Big(  \sum_{ i=1  }^{ k  } g({\lambda}_{i}) {T}_{i} \Big) \\
                        &= \sum_{ i=1  }^{ k  } g({\lambda}_{i}) (U {T}_{i}) \\ 
                        &= \sum_{ i=1  }^{ k  } g({\lambda}_{i}) ({T}_{i} U) \\ 
                        &= \Big(  \sum_{ i=1  }^{ k  } g({\lambda}_{i}) {T}_{i} \Big) U \\
                        &= (g(T))U \\
                        &= T^{*} U.
\end{align*}
Thus, \( U  \) commutes with \( T^{*} \).
\end{proof}

\subsection*{Exercise 6.6.9} Referring to Exercise 20 of Section 6.5, prove the following facts about a partial isometry \( U \). 
\begin{enumerate}
    \item[(a)] \( U^{*} U  \) is an orthogonal projection on \( W  \).
        \begin{proof}
        Our goal is to show that 
        \begin{center}
           \( R(U^{*}U) = N(U^{*}U)^{\perp} \) and \( N(U^{*}U) = R(U^{*}U)^{\perp} \). 
        \end{center}
        Suppose \( y \in R(U^{*}U) \). Then \( y = U^{*}U(x) \) for \( x \in V  \). We need to show that \( \langle y , z  \rangle = 0  \) for all \( z \in N(U^{*}U) \). Since \( U  \) is a partial isometry, we must have
        \begin{align*}
            \langle y , z  \rangle = \langle U^{*}U(x) , z  \rangle  &= \langle U(x) , U(z) \rangle \\
                                                                     &= \langle x  ,  U^{*}U(z) \rangle \\
                                                                     &= \langle x  , 0  \rangle \\
                                                                     &= 0.
       \end{align*}
       Thus, \( y \in N(U^{*}U)^{\perp} \). On the other hand, suppose \( y \in N(U^{*}U)^{\perp}  \). We need to show that \( y  - U^{*}U(y) = 0 \). Consider the norm \( \| y - U^{*}U(y)\|^{2} \). Then observe that 
       \begin{align*}
           \|y - U^{*}U(y)\|^{2} &= \langle y - U^{*}U(y) ,  y - U^{*}U(y) \rangle   \\
                                 &= \langle y  ,  y - U^{*}U(y) \rangle - \langle U^{*}U(y) , y - U^{*}U(y) \rangle \\
                                 &= \langle y , y  - U^{*}U(y) \rangle - \langle U(y)  ,  U(y - U^{*}U(y)) \rangle \\ 
                                 &= \langle y , y - U^{*}U(y) \rangle - \langle y  , y - U^{*}U(y) \rangle \\
                                 &= 0.
       \end{align*}
       So, we conclude that \( y - U^{*}U(y) = 0  \) and thus \(  U^{*}U(y) = y  \). Therefore, we conclude that \( y \in R(U^{*}U) \) which proves our first equation. Using this equation and the fact that \( V  \) is a finite-dimensional inner product space, we can write
       \[   N(U^{*}U) = (N(U^{*}U)^{\perp})^{\perp}  = R(U^{*}U)^{\perp}  \]
       which shows the second result.
        \end{proof}
    \item[(b)] \( U U^{*} U = U  \).
        \begin{proof}
        Consider the norm \( \| U(x) - U U^{*} U (x) \|^{2} \). Then we have
        \begin{align*}
            \|U(x) - U U^{*} U (x) \|^{2} &= \langle U(x) - U U^{*} U (x)  ,  U (x) - U U^{*} U(x)  \rangle \\
                                          &=  \langle U(x) , U(x) - U U^{*} U(x)   \rangle - \langle U U^{*} U(x)  , U(x) - U U^{*} U(x)  \rangle.
        \end{align*}
        Since \( U  \) is an partial isometry, we can write the first term as
        \begin{align*}
            \langle U(x)  , U ( x - U^{*}U(x)) \rangle = \langle x  ,  x - U^{*}U(x)  \rangle.
        \end{align*}
        Likewise, the second term can be written as
        \begin{align*}
            \langle U(U^{*}U(x)) , U(x -  U^{*} U(x))  \rangle &= \langle U^{*}U(x)  , x -  U^{*} U(x)  \rangle \\
                                                                &= \langle U(x)  , U(x -  U^{*} U(x)) \rangle \\
                                                                &= \langle x  ,  x -  U^{*} U (x)  \rangle.
        \end{align*}
        Thus, \( \|U(x) - U U^{*} U(x) \|^{2} = 0  \). So, we get that \( U U^{*} U(x) = U(x)  \) for all \( x \in V  \). Therefore, we conclude that \( U U^{*} U = U  \).
        \end{proof}
\end{enumerate}

\subsection*{Exercise 6.6.10} Let \( U \) and \( T  \) be normal operators on a finite-dimensional complex inner product space \( V  \) such that \( TU = UT  \). Prove that there exists an orthonormal basis for \( V  \) consisting of vectors that are eigenvectors of both \( T  \) and \( U  \).
\begin{proof}

\end{proof}

\subsection*{Exercise 6.6.11} Prove (c) of the spectral theorem.
\begin{proof}
    Let each \( {W}_{i}  \) be the eigenspace of \( T \). Suppose \( i = j  \). Since each \( {T}_{i}  \) is an orthogonal projection, we have \( T_{i}^{2} = {T}_{i}  \). Let \( x \in V  \). Since \( x = {x}_{1} + {x}_{2} + \cdots + {x}_{k } \) and \( {T}_{i}(x) = {x}_{i} \) for each \( i \) where \( {x}_{i} \in {W}_{i} \), we have
    \[ {T}_{i}{T}_{j}(x) = {T}_{i}^{2}(x) = {T}_{i}({T}_{i}(x)) = {T}_{i}({x}_{i}) = 1 \cdot {x}_{i} = {\delta}_{ii}  \cdot {T}_{i}(x).  \]
    On the other hand, suppose \( i \neq j  \). By part (b) of the spectral theorem, we know that \( {W}_{i}^{\perp} = {W}_{i}'  \). Thus, we have \( {T}_{i}(x) = 0   \) for \( i \neq j  \). So, observe that 
    \[  {T}_{i}{T}_{j}(x) = {T}_{i}({T}_{j}(x)) = {T}_{i}({x}_{j}) = 0 = 0 \cdot {T}_{i}(x) = {\delta}_{ij} {T}_{i}(x). \]
    Therefore, we conclude that \( {T}_{i}{T}_{j} = {\delta}_{i} {T}_{i}  \).
\end{proof}

