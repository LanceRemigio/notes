\section{Gram-Schmidt Orthogonalization Process}

\subsection*{Exercise 6.2.6} Let \( V  \) be an inner product space, and let \( W  \) be a finite-dimensional subspace of \( V  \). If \( x \notin W  \), prove that there exists \( y \in V  \) such that \( y \in W^{\perp} \), but \( \langle x , y \rangle \neq 0  \).
\begin{proof}
Let \( x \in V  \) but not in \( W  \). By Theorem 6.6, there exists unique scalars \( u \in W  \) and \( y \in W^{\perp} \) such that \( x = u + y \) if and only if \( y = x - u  \). Then observe that
\[ \langle x , y \rangle = \langle x  , x - u  \rangle = \langle x , x \rangle - \langle x , u \rangle.\]
If \( x = 0  \), then we would find that \( x \in W  \) since \( W  \) is a finite-dimensional subspace of \( V  \) which is a contradiction. Does \( x  \) must be non-zero as well as \( x - u \neq 0  \) since \( U  \) is a unique scalar. Otherwise, we gain the same contradiction. Thus, we have that \( \langle x , y \rangle \neq 0  \). 
\end{proof}

\subsection*{Exercise 6.2.7} Let \( \beta \) be a basis for a subspace \( W  \) an inner product space \( V  \), and let \( z \in V  \). Prove that \( z \in W^{\perp} \) if and only if \( \langle z  ,  v  \rangle = 0  \) for every \( v \in \beta  \).
\begin{proof}
    Let \( z \in W^{\perp} \) and \( v \in \beta \). Then by definition of \( W^{\perp} \) and the fact that \( \beta \subseteq W  \), we must have \( \langle z , v  \rangle = 0  \).

    Conversely, suppose \( \langle z , v \rangle = 0  \) for all \( v \in \beta \). Since \( \beta \subseteq W \), we immediately have \( z \in W^{\perp} \).
\end{proof}

\subsection*{Exercise 6.2.8} Prove that if \( \{ {w}_{1}, {w}_{2}, \dots, {w}_{n} \}  \) is an orthogonal set of nonzero vectors, then the vectors \( {v}_{1}, {v}_{2}, \dots, {v}_{n} \) derived from the Gram-Schmidt process satisfy \( {v}_{i} = {w}_{i} \) for \( i = 1,2,\dots, n \).
\begin{proof}
    We proceed by mathematical induction on \( n  \). Let \(  k = 1,2,\dots, n \) and let \( S_k = \{ {w}_{1}, {w}_{2}, \dots, {w}_{k} \}  \) be an orthogonal set of non-zero vectors. Note that \( S_k  \) is also linearly independent by Corollary 1 to Theorem 6.3. Define \( {S}_{k}' = \{ {v}_{1}, {v}_{2}, \dots, {v}_{k} \}  \) with the same properties as seen in the setup of Theorem 6.4. Let \(  n = 1  \) be our base case. Then we the construction of \( {S}_{k}' \), we immediately have \( {v}_{1} = {w}_{1} \). Now, suppose that this result holds the \( k - 1  \) case. Define \( {S}_{k-1} = \{ {w}_{2}, {w}_{3}, \dots, {w}_{k} \}  \) and \( {S}_{k-1}' = \{ {w}_{2}, {w}_{3}, \dots, {w}_{k} \}  \). Notice that \( {S}_{k}' \) is also orthogonal and that \( \text{span}(S_k) = \text{span}({S}_{k}') \) by Theorem 6.4. So, if \( y \in \text{span}({S}_{k}) \), then
    \[  y = \sum_{ i=1  }^{ k  } \frac{ \langle y , {v}_{i} \rangle  }{ \|{v}_{i}\|^{2} } {v}_{i} = \sum_{ i=1  }^{ k  } \frac{ \langle y , {w}_{i} \rangle }{ \|{w}_{i}\|^{2} } {w}_{i}. \]
    We can see that 
    \[  \frac{ \langle y , {v}_{1} \rangle }{  \|{v}_{1}\|^{2} } {v}_{1}  + \sum_{ i=2  }^{ k } \frac{ \langle y , {v}_{i} \rangle }{ \|{v}_{i}\|^{2} } {v}_{i} =  \frac{ \langle y , {w}_{1}  \rangle }{ \|{w}_{1}\|^{2} } {w}_{1} + \sum_{ i=2  }^{ k  } \frac{ \langle y , {w}_{i} \rangle }{ \|{w}_{i}\|^{2} } {w}_{i}.   \]
    By our induction hypothesis that \( {v}_{i} = {w}_{i}  \) for all \( k = 2,3, \dots, n \), we see that the two summations above must cancel out. Thus, we are left with
    \[  \frac{ \langle y , {v}_{1} \rangle }{ \|{v}_{1}\|^{2} }  {v}_{1} = \frac{ \langle y , {w}_{1} \rangle }{ \|{w}_{1}\|^{2} } {w}_{1}. \]
    But since \( {v}_{1} = {w}_{1} \) by our base case, we must have \( {v}_{i} = {w}_{i}  \) for all \( 1 \leq i \leq k  \) which ends our induction proof. 
\end{proof}

\subsection*{Exercise 6.2.10} Let \( W  \) be a finite-dimensional subspace of an inner product space \( V  \). Prove that \( V = W \oplus W^{\perp} \). Prove that there exists a projection \( T  \) on \( W  \) along \( W^{\perp} \) that satisfies \( N(T) = W^{\perp} \). In addition, prove that \( \|T(x)\| \leq \|x\| \) for all \( x \in V  \).
\begin{proof}
  Let \( S = \{ {v}_{1}, {v}_{2}, \dots, {v}_{k} \}   \) be an orthonormal basis for \( W  \). Let \( x \in V  \). By Theorem 6.6, we can find \( u \in W  \) and \( z \in W^{\perp} \) such that \( x = u + z  \). Thus, \( V = W + W^{\perp} \). Note that \( W \cap W^{\perp} = \{ 0  \}  \). So, \( V = W \oplus W^{\perp}  \). Now, define the map
  \[  T: V \to V  \ \ \text{by} \ \ T(x) = u    \]
  with \( x \in V   \). We will show that this map is a projection on \( W  \) along \( W^{\perp} \). Note that it can be easily shown that this map is linear. Since \( V = W \oplus W^{\perp}  \), we have \( x = u + z  \) where \( u \in W  \) and \( z \in W^{\perp} \). Thus, \( T  \) must be a projection on \( W  \) along \( W^{\perp} \). 

  Next, we will show that \( T  \) satisfies \( N(T) = W^{\perp}  \). Let \( x \in W^{\perp} \). Since \( T  \) is a projection on \( W  \) along \( W^{\perp} \), we have \( T(x) = 0  \). So, \( x \in N(T) \). For the other containment, let \( x \in N(T) \). Then by definition of \( N(T) \), we have \( T(x) = 0  \). With \( T  \) being a projection on \( W  \) along \( W^{\perp} \), we have \( x = 0 + z  \) where \( z \in W^{\perp} \). But this tells us that \( x \in W^{\perp} \). Thus, we conclude that \( N(T) = W^{\perp}  \).  

  Let \( x \in V  \). Then \( x = u + z  \) since \( V = W \oplus W^{\perp} \). By Exercise 10 from Section 6.1, we have 
  \begin{align*}
      \|x\|^{2} = \|u + z\|^{2} = \|u\|^{2} + \|z\|^{2} 
                                \geq \|u\|^{2} 
                                = \|T(x)\|^{2}.
  \end{align*}
  Thus, we have \( \|T(x)\| \leq \|x\| \) for all \( x \in V  \).
\end{proof}

\subsection*{Exercise 6.2.13} Let \( V  \) be an inner product space, \( S  \) and \( {S}_{0} \) be subsets of \( V  \), and \( W  \) be a finite-dimensional subspace of \( V  \). Prove the following results.
\begin{enumerate}
    \item[(a)] \( {S}_{0} \subseteq S  \) implies that \( S^{\perp} \subseteq {S}_{0}^{\perp} \).
        \begin{proof}
            Suppose \( {S}_{0} \subseteq S \). Let \( {s}_{0} \in {S}_{0} \). Then \( {s}_{0} \in S  \) as well. If we let \( x \in S^{\perp} \), then  we see that \( \langle x ,  {s}_0  \rangle = 0  \). But \( {s}_{0} \) in an element of \( {S}_{0} \), so \( x \in {S}_{0}^{\perp} \) and we are done.  
        \end{proof}
    \item[(b)] \( S \subseteq (S^{\perp})^{\perp} \); so \( \text{span}(S) \subseteq (S^{\perp})^{\perp} \).
        \begin{proof}
        Let \( {v}_{1}, {v}_{2}, \dots, {v}_{k} \in S  \). If \( y \in \text{span}(S) \), then 
        \[  y = \sum_{ i=1  }^{  k  } {a}_{i} {v}_{i}  \]
        for some scalars \( {a}_{1}, {a}_{2}, \dots, {a}_{k }  \). Since \( S \subseteq (S^\perp)^{\perp} \), we see that \( {v}_{1}, {v}_{2}, \dots, {v}_{k} \in (S^{\perp})^{\perp} \). So, \( y  \) must also be an element of \( (S^{\perp})^{\perp} \) and we are done.
        \end{proof}
    \item[(c)] \( W = (W^{\perp})^{\perp} \).
        \begin{proof}
        By part (b), we have \( W \subseteq (W^{\perp})^{\perp} \). We need only show the other containment now, namely, \( (W^{\perp})^{\perp} \subseteq W  \). Suppose for sake of contradiction that \( x \notin W  \). Now, let \( x \in (W^{\perp})^{\perp} \). By definition, we have   
        \begin{center}
            \( \langle x , w \rangle = 0   \) for all \( w \in W^{\perp} \).
        \end{center}
        However, we have \( x \notin W  \) implies that \( w \in W^{\perp} \) such that \( \langle x , w \rangle \neq 0  \) for all \( w \in W^{\perp} \) which contradicts the assumption that \( x \in (W^{\perp})^{\perp}  \). Thus, we must have \( (W^{\perp})^{\perp} \subseteq W  \). Hence, we conclude that 
        \[   W = (W^{\perp})^{\perp}.  \]
        \end{proof}
    \item[(d)] \( V = W \oplus W^{\perp} \).
        \begin{proof}
        Since \( W  \) is a subspace and finite-dimensional, let \( \beta = \{ {v}_{1}, {v}_{2}, \dots, {v}_{k} \}  \) be an orthonormal basis for \( W \). By Theorem 6.7, we have \( \text{dim}(V) = \text{dim}(W) + \text{dim}(W^{\perp}) \). Note that \(W \cap W^{\perp} = \{ 0 \}  \). By Exercise 29 (a), we have  
        \begin{align*}
            \text{dim}(V) &= \text{dim}(W) + \text{dim}(W^{\perp}) - \text{dim}(W \cap W^{\perp}) \\
                          &= \text{dim}(W + W^{\perp}).
        \end{align*}
        Hence, \( V = W + W^{\perp} \). Thus, we conclude that
        \[  V = W \oplus W^{\perp}. \]
        \end{proof}
\end{enumerate}

\subsection*{Exercise 6.2.14} Let \( {W}_{1} \) and \( {W}_{2} \) be subspaces of a finite-dimensional inner product space. Prove that \( ({W}_{1} + {W}_{2})^{\perp} = {W}_{1}^{\perp} \cap {W}_{2}^{\perp} \) and \( ({W}_{1} \cap {W}_{2})^{\perp} = {W}_{1}^{\perp} + {W}_{2}^{\perp} \). 
\begin{proof}
    Let \( V \) be a finite-dimensional vector space. We will first show that \( ({W}_{1} + {W}_{2})^{\perp} \subseteq {W}_{1}^{\perp} \cap {W}_{2}^{\perp} \). Since \( {W}_{1} \cap {W}_{2} \subseteq {W}_{1} + {W}_{2} \), we can see by part (a) that \( ({W}_{1} + {W}_{2})^{\perp} \subseteq {W}_{1}^{\perp} \cap {W}_{2}^{\perp}  \). Let \( x \in {W}_{1}^{\perp} \cap {W}_{2}^{\perp} \). Then \( x \in {W}_{1}^{\perp} \) and  \( x \in {W}_{2}^{\perp} \). So, \( \langle x , {w}_{1}  \rangle = 0  \) and \( \langle x  ,  {w}_{2} \rangle = 0  \) for all \( {w}_{1} \in {W}_{1} \) and \( {w}_{2} \in {W}_{2} \), respectively. Since 
    \[  \langle x , {w}_{1} \rangle + \langle x  , {w}_{2} \rangle = \langle x  ,  {w}_{1} + {w}_{2} \rangle = 0, \]
    we can see that \( x \in ({W}_{1} + {W}_{2})^{\perp} \). So, \( {W}_{1}^{\perp} \cap {W}_{2}^{\perp} \subseteq ({W}_{1} + {W}_{2})^{\perp} \). Thus, 
    \[  ({W}_{1} + {W}_{2})^{\perp} = {W}_{1}^{\perp} \cap {W}_{2}^{\perp}. \tag{1} \]

    Now, we can use the first equation to show the second equation. Then by using part(c) of Exercise 13 and (1), we can write that
    \begin{align*}
        ({W}_{1} \cap {W}_{2})^{\perp} &= \Big( ({W}_{1}^{\perp})^{\perp} \cap ({W}_{2}^{\perp})^{\perp} \Big)  \\
                                       &= \Big( (W_{1}^{\perp} + {W}_{2}^{\perp})^{\perp}  \Big)^{\perp} \\
                                       &= {W}_{1}^{\perp} + {W}_{2}^{\perp}
    \end{align*}
    which is our desired result.


\end{proof}

\subsection*{Exercise 6.2.15} Let \( V  \) be a finite-dimensional inner product space over \( F  \).
\begin{enumerate}
    \item[(a)] \textit{Parseval's Identity}. Let \( \{ {v}_{1}, {v}_{2}, \dots, {v}_{n} \}  \) be an orthonormal basis for \( V  \). For any \( x,y \in V  \) prove that 
        \[  \langle x , y \rangle = \sum_{ i=1  }^{ n } \langle x , {v}_{i} \rangle \overline{\langle y , {v}_{i} \rangle}. \] 
        \begin{proof}
        Let \( x,y \in V  \). Since \( \{ {v}_{1}, {v}_{2}, \dots, {v}_{n} \}   \) is an orthonormal basis for \( V  \), write  
        \[   x = \sum_{ i=1  }^{ n } \langle x , {v}_{i} \rangle {v}_{i} \ \ \text{and} \ \ y = \sum_{ j=1  }^{ n  } \langle y , {v}_{j} \rangle {v}_{j}. \]
        Since \( \langle {v}_{i}  ,  {v}_{j} \rangle = {\delta}_{ij}  \) where \( {\delta}_{ij} = 1 \) whenever \( i = j  \) and \( {\delta}_{ij} = 0  \) otherwise, we have
        \begin{align*}
            \langle x , y \rangle &= \Big\langle \sum_{ i=1  }^{ n } \langle x , {v}_{i} \rangle, \sum_{ j=1 }^{ n  } \langle y , {v}_{j} \rangle {v}_{j}  \Big\rangle \\
                                  &= \sum_{ i=1  }^{ n } \langle x , {v}_{i} \rangle \Big\langle {v}_{i} ,  \sum_{ j=1 }^{ n } \langle y , {v}_{j} \rangle {v}_{j} \Big\rangle \\ 
                                  &= \sum_{ i=1  }^{ n } \langle x , {v}_{i} \rangle \sum_{ j=1  }^{ n }  \overline{\langle y , {v}_{j} \rangle } \langle {v}_{i} ,  {v}_{j} \rangle  \\
                                  &= \sum_{ i=1  }^{ n } \langle x  ,  {v}_{i} \rangle \overline{\langle y , {v}_{j} \rangle}.
        \end{align*}
        Thus, we have
            \[  \langle x , y \rangle = \sum_{ i=1  }^{ n } \langle x  , {v}_{i} \rangle \overline{\langle y , {v}_{i} \rangle}. \]
        \end{proof}
    \item[(b)] Use (a) to prove that if \( \beta  \) is an orthonormal basis for \( V  \) with inner product \( \langle \cdot , \cdot \rangle \), then for any \( x,y \in V  \)
        \[  \langle {\phi}_{\beta}(x) , \phi_{\beta}(y) \rangle' = \langle [x]_{\beta} , [y]_{\beta} \rangle' = \langle x , y \rangle. \]
        \begin{proof}
            Let \( x,y \in V  \). Since \( \beta  \) is an orthonormal basis for \( V  \), write
        \[   x = \sum_{ i=1  }^{ n } \langle x , {v}_{i} \rangle {v}_{i} \ \ \text{and} \ \ y = \sum_{ j=1  }^{ n  } \langle y , {v}_{j} \rangle {v}_{j}. \]
        Since \( \langle \cdot , \cdot \rangle'  \) is the standard inner product on \( F^{n} \), we have
        \[  \langle \phi_{\beta}(x) , {\phi}_{\beta}(y)  \rangle' = \langle [x]_{\beta} , [y]_{\beta} \rangle' = \sum_{ i=1  }^{ n } \langle x , {v}_{i} \rangle \overline{\langle y , {v}_{i} \rangle} = \langle x , y \rangle \]
        which is our desired result.
        \end{proof}
\end{enumerate}

\subsection*{Exercise 6.2.16}
\begin{enumerate}
    \item[(a)] \textit{Bessel's Inequality}. Let \( V  \) be an inner product space, and let \( S = \{ {v}_{1}, {v}_{2}, \dots, {v}_{n} \}  \) be an orthonormal subset of \( V  \). Prove that for any \( x \in V  \) we have 
        \[  \|x\|^{2} \geq \sum_{ i=1  }^{ n } | \langle x , {v}_{i} \rangle |^{2}. \]
        \begin{proof} Suppose that the orthonormal subset \( S = \{ {v}_{1}, {v}_{2}, \dots, {v}_{n} \}  \) generates some subspace \( W  \) of \( V  \). Note that this implies that \( W  \) is a finite-dimensional subspace of \( V  \). Let \( x \in V  \). Using Theorem 6.6, there exists unique vectors \( u \in W  \) and \( z \in W^{\perp} \) such that \( x = u + z  \). Since \( S  \) is an orthonormal basis for \( W  \), we must have that
            \[  u = \sum_{ i=1  }^{ k  } \langle  x ,  {v}_{i} \rangle {v}_{i}. \]
            By using {\hyperref[Exercise 6.1.10]{Exercise 10 from Section 6.1}} and the fact that \( S  \) is an orthonormal set, we can see that 
            \begin{align*}
                \|x\|^{2} = \|u + z\|^{2} \geq \|u\|^{2}  &= \Big\| \sum_{ i=1  }^{ k  } \langle x , {v}_{i} \rangle {v}_{i} \Big\|^{2} \\
                                          &= \sum_{ i=1  }^{ n  } | \langle x , {v}_{i} \rangle |^{2} \|{v}_{i}\|^{2} \\
                                          &= \sum_{ i=1 }^{ n  } | \langle x , {v}_{i} \rangle |^{2}
            \end{align*}
            which establishes our result.
        \end{proof}
    \item[(b)] In the context of (a), prove that Bessel's inequality is an equality if and only if \( x \in \text{span}(S) \).
        \begin{proof}
        Suppose that Bessel's inequality is an equality. Since \( S  \) is an orthonormal subset of \( V  \) and {\hyperref[Exercise 6.1.10]{Exercise 10 from Section 6.1}}, we see that
        \begin{align*}
            \|x\|^{2} = \sum_{ i=1  }^{ k  } | \langle x , {v}_{i} \rangle |^{2}   
                      = \sum_{ i=1  }^{ n  } | \langle x , {v}_{i} \rangle |^{2} \|{v}_{i}\|^{2} 
                      = \Big\| \sum_{ i=1  }^{ n  } \langle x , {v}_{i} \rangle {v}_{i} \Big\|^{2}.
        \end{align*}
        This implies that 
        \[  x = \sum_{ i=1  }^{ n  } \langle x , {v}_{i} \rangle {v}_{i} \]
        and thus \( x \in \text{span}(S) \). 

        Conversely, suppose \( x \in \text{span}(S) \). Then
        \[  x = \sum_{ i=1  }^{ n  } \langle x , {v}_{i} \rangle {v}_{i}.  \]
        Using Exercise 10 again, we have
        \begin{align*}
            \|x\|^{2} = \Big\| \sum_{ i=1  }^{ n  } \langle x , {v}_{i} \rangle {v}_{i} \Big\|^{2} = \sum_{ i=1  }^{ n } | \langle x , {v}_{i} \rangle |^{2} \|{v}_{i}\|^{2} = \sum_{ i=1  }^{ n  } | \langle x , {v}_{i} \rangle |^{2} 
        \end{align*}
        which achieves the equality of (a).
        \end{proof}
\end{enumerate}

\subsection*{Exercise 6.2.17} Let \( T  \) be a linear operator on an inner product space \( V  \). If \( \langle T(x) , y \rangle = 0  \) for all \( x,y \in V  \), prove that \( T = {T}_{0} \). In fact, prove this result if the equality holds for all \( x  \) and \( y  \) in some basis for \( V  \).
\begin{proof}
Let \( x,y \in V  \). Our goal is to show that \( {T}_{0} = T \). Consider \( {T}_{0}  \). Then \( V  \) being an inner product space implies that we have \( \langle {T}_{0}(x) , y \rangle = \langle 0 , y \rangle = 0  \) for all \( x \in V  \) where \( {T}_{0}(x) = 0  \). Since \( \langle T(x) , y \rangle = 0  \), this must imply that \( T(x) = {T}_{0}(x)   \). Thus, \( T = {T}_{0} \). Now, suppose \( \beta  \) is a basis for \( V  \) where \( x,y \in \beta \). Using Exercise 9 from the last section, we see that \( \langle {T}_{0}(x) , y \rangle = \langle T(x) , y \rangle  \) implies that \( {T}_{0}(x) = T(x) \). So, \( T = {T}_{0} \).
\end{proof}

\subsection*{Exercise 6.2.18} Let \( V = C([-1,1]) \). Suppose that \( {W}_{e} \) and \( {W}_{o} \) denote the subspaces of \( V  \) consisting of the even and odd functions, respectively. Prove that \( W^{\perp}_{e} = W_{o} \), where the inner product on \( V  \) is defined by 
\[  \langle f , g \rangle = \int_{ -1 }^{ 1 }  f(t) g(t) \ dt. \]
\begin{proof}
 Our goal is to show that \( {W}_{o} \subseteq {W}_{e}^{\perp} \) and \( {W}_{e}^{\perp} \subseteq {W}_{o} \). Let \( f \in {W}_{o} \). Consider \( \langle f , g \rangle \). Then by definition 
 \[  \langle f , g \rangle = \int_{ -1 }^{ 1 }  f(t)g(t) \ dt \] where \( g \in {W}_{e} \). We need to show that \( \langle f , g \rangle = 0  \) for all \( g \in {W}_{e} \). Since \( g  \) is an even function and \( f \) is an odd function, then the product \( fg  \) is an odd function. By our integration properties, we have that 
 \[  \langle f , g \rangle = \int_{ -1 }^{ 1 }  f(t) g(t) \ dt = \int_{ -1 }^{ 1 }  fg(t) \ dt = 0.   \]
 Hence, \( f \in {W}_{e}^{\perp} \). 

 Let \( f \in {W}_{e}^{\perp} \). Suppose for sake of contradiction that \( f \notin {W}_{o} \); that is, \( f  \) is an even function. By definition of \( {W}_{e}^{\perp} \), we have
 \[   0 = \langle f , g \rangle = \int_{ -1 }^{ 1 }  f(t) g(t) \ dt = \int_{ -1 }^{ 1  } fg(t)  \ dt = 2 \int_{ 0 }^{ 1 } fg(t) \ dt. \]
 But we have \( fg  \) is a continuous even function where
 \[  \int_{ -1 }^{ 1 }  fg(t) \ dt - 2 \int_{ 0 }^{ 1 } fg(t) \ dt \neq 0 \]
 which is a contradiction since we have assumed that \( f \in {W}_{e}^{\perp} \). Hence, we must have \( f \in {W}_{o} \). Thus, we conclude that \( {W}_{o} = {W}_{e}^{\perp} \).
\end{proof}


\subsection*{Exercise 6.2.23} Let \( V  \) be the vector space defined in Example 5 of Section 1.2, the space of all sequences \( \sigma \) in \( F  \) (where \( F = \R  \) or \( F = \C  \)) such that \( \sigma(n) \neq 0  \) for only finitely many positive integers \( n  \). For \( \sigma, \mu \in V  \), we define \( \langle \sigma  , \mu \rangle = \sum_{ n=1  }^{ \infty   } \sigma(n) \overline{\mu(n)}.  \) Since all but a finite number of terms of the series are zero, the series converges.
\begin{enumerate}
    \item[(a)] Prove that \( \langle \cdot , \cdot \rangle \) is an inner product on \( V  \), and hence \( V  \) is an inner product space.
        \begin{proof}
        
        \end{proof}
    \item[(b)] For each positive integer \( n \), let \( {e}_{n} \) be the sequence defined by \( {e}_{n}(k) = {\delta}_{nk} \), where \( {\delta}_{nk}  \) is the Kronecker delta. Prove that \( \{ {e}_{1}, {e}_{2}, \dots, \dots \}  \) is an orthonormal basis for \( V  \).
        \begin{proof}
        
        \end{proof}
    \item[(c)] Let \( \sigma_n = {e}_{1} + {e}_{n} \) and \( W = \text{span}(\{ {\sigma}_{n}: n \geq 2 \} ) \).
        \begin{enumerate}
            \item[(i)] Prove that \( {e}_{1} \notin W  \), so \( W \neq V  \).
                \begin{proof}
                
                \end{proof}
            \item[(ii)] Prove that \( W^{\perp} = \{ 0  \}  \), and conclude that \( W \neq (W^{\perp})^{\perp}  \).
                Thus the assumption in Exercise 13(c) that \( W  \) is finite-dimensional is essential.
                \begin{proof}
                
                \end{proof}
        \end{enumerate}
\end{enumerate}
