\section{Normal and Self-Adjoint Operators}

\subsection*{Exercise 6.4.4} Let \( T  \) and \( U  \) be self-adjoint operators on an inner product space \( V  \). Prove that \( TU  \) is self-adjoint if and only if \( TU = UT  \).
\begin{proof}
Let \( T  \) and \( U  \) be self-adjoint operators on an inner product space \( V  \). For the forwards direction, suppose \( TU  \) is self-adjoint. Then we have
\[  TU = (TU)^{*} = U^{*} T^{*} = UT. \]
Conversely, suppose \( TU = UT \). Then
\[  (TU)^{*} = U^{*}T^{*} = UT = TU.  \]
Thus, \( TU  \) is self-adjoint.
\end{proof}

\subsection*{Exercise 6.4.5} Prove (b) of Theorem 6.15.

\begin{proof}
Suppose \( c \in F  \). Then we have
        \begin{align*}
            (T - cI)(T - cI)^{*} &= (T - cI)(T^{*} - \overline{c}I) \\
                                 &= TT^{*} - c T^{*} - \overline{c} (IT) + c \overline{c} I\\
                                 &= T^{*}T - c T^{*} - \overline{c} (IT) + \overline{c} c   I \\
                                 &= T^{*} (T - cI) - \overline{c} I  (T - cI) \\
                                 &= (T - cI) (T^{*} - \overline{c}I) \\
                                 &= (T - cI) (T  - cI)^{*}.
        \end{align*}
        Thus, the operator \( T - cI \) is normal.
    \end{proof}

\subsection*{Exercise 6.4.6} Let \( V  \) be a complex inner product space, and let \( T  \) be a linear operator on \( V  \). Define 
\[  {T}_{1} = \frac{ 1 }{ 2 }  (T + T^{*}) \ \ \text{and} \ \ {T}_{2} = \frac{ 1 }{ 2i }  (T - T^{*}). \]
\begin{enumerate}
    \item[(a)] Prove that \( {T}_{1}  \) and \( {T}_{2}  \) are self-adjoint and that \( T = {T}_{1} + i {T}_{2} \).
        \begin{proof}
            By definition of \( {T}_{1} \), we have
            \[  {T}_{1}^{*} = \frac{ 1 }{ 2 }  (T + T^{*})^{*} = \frac{ 1 }{ 2 }  ( T^{*} + T^{* * }) = \frac{ 1 }{ 2 }  (T  + T^{*}) = {T}_{1}. \]
            Thus, \( {T}_{1} \) is self-adjoint. Likewise, we have
            \begin{align*}
            {T}_{2}^{*} = \Big(  \frac{ 1 }{ 2i }  (T - T^{*}) \Big)^{*} &=  - \frac{ 1 }{ 2i } (T - T^{*})^{*} \\  
                                                                         &=  \frac{ 1 }{ 2i }  ( T^{* * } -  T^{*}) \\  
                                                                         &=  \frac{ 1 }{ 2i }  ( T -  T^{*}) \\  
                                                                         &= {T}_{2}.
\end{align*}
Thus, \( {T}_{2} \) is self-adjoint. Observe that \( T = {T}_{1} + i {T}_{2} \). Hence, we have
\begin{align*}
    {T}_{1} + i {T}_{2} &= \frac{ 1 }{ 2 }  ( T  + T^{*}) + i \Big(  \frac{ 1 }{ 2i }  (T - T^{*}) \Big)   \\
                        &= \frac{ 1 }{ 2 }  (T + T^{*}) + \frac{ 1 }{ 2 } (T - T^{*}) \\
                        &= \frac{ 1 }{ 2 } \cdot  2 T  \\
                        &= T.
\end{align*}

        \end{proof}
    \item[(b)] Suppose that \( T = {U}_{1} + i {U}_{2} \), where \( {U}_{1} \) and \( {U}_{2} \) are self-adjoint. Prove that \( {U}_{1} = {T}_{1} \) and \( {U}_{2} = {T}_{2} \).
        \begin{proof}
            Observe that \( T^{*} = {U}_{1}^{*} - i {U}_{2}^{*} \). First, we will show \( {U}_{1} = {T}_{1} \). Since \( {U}_{1}  \) and \( {U}_{2} \) are self-adjoint, we have \( T^{*} = {U}_{1} - i {U}_{2} \). So, we have
            \begin{align*}
                {T}_{1} = \frac{ 1 }{ 2 }  (T + T^{*})  &= \frac{ 1 }{ 2 }  ({U}_{1} + i {U}_{2}) + \frac{ 1 }{ 2 }  ({U}_{1}^{*} - i {U}_{2}^{*}) \\
                                                        &=\frac{ 1 }{ 2 }  ({U}_{1} + i {U}_{2}) + \frac{ 1 }{ 2 } ( {U}_{1} - i {U}_{2}) \\
                                                        &= \frac{ 1 }{ 2 }  \cdot 2 {U}_{1} \\
                                                        &=  {U}_{1}.
\end{align*}
Thus, we have \( {T}_{1} = {U}_{1} \). For the second equality, observe that
\begin{align*}
    {T}_{2} = \frac{ 1 }{ 2i }  (T - T^{*}) &= \frac{ 1 }{ 2i } ( {U}_{1} + i {U}_{2}) - \frac{ 1 }{ 2i } ( {U}_{1}^{*} - i {U}_{2}^{*} )  \\
                                            &= \frac{ 1 }{ 2i }  ({U}_{1} + i {U}_{2}) - \frac{ 1  }{ 2i }  ({U}_{1} - i {U}_{2}) \\
                                            &= \frac{ 1 }{ 2i } \cdot 2i {U}_{2} \\
                                            &= {U}_{2}.
\end{align*}
        \end{proof}
    \item[(c)] Prove that \( T  \) is normal if and only \( {T}_{1} {T}_{2} = {T}_{2} {T}_{1} \).
        \begin{proof}
        Suppose that \( T  \) is normal. Thus, \( T T^{*} = T^{*} T  \) by definition. Our goal is to show \( {T}_{1} {T}_{2}  = {T}_{2} {T}_{1} \). Thus, we have
        \begin{align*}
            {T}_{1} {T}_{2} &= \frac{ 1 }{ 2 }  (T  + T^{*}) \frac{ 1 }{ 2i }  (T - T^{*})  \\
                                                                                          &=  \frac{ 1 }{ 2 }  \cdot \frac{ 1 }{ 2i }  (T^{2} + T^{*}T - T T^{*} - (T^{*})^{2}) \\
                                                                                          &= \frac{ 1 }{ 2 }  \cdot \frac{ 1 }{ 2i } (T^{2} - T^{*} T + T^{*}T - (T^{*})^{2}) \\
                                                                                          &= \frac{ 1 }{ 2 }  \cdot \frac{ 1 }{ 2i }  \Big(  T (T  - T^{*}) + T^{*} (T - T^{*})  \Big) \\
                                                                                          &=  \frac{ 1 }{ 2i }  (T - T^{*}) \frac{ 1 }{ 2 }  (T + T^{*}) \\
                                                                                          &= {T}_{2} {T}_{1}.
        \end{align*}
        Thus, \( {T}_{1} {T}_{2} = {T}_{2} {T}_{1} \). Conversely, suppose \( {T}_{1} {T}_{2} = {T}_{2} {T}_{1} \). Observe that
        \begin{align*}
            T T^{*} &= ({T}_{1} + i {T}_{2})({T}_{1} - {iT}_{2}) \\
                    &= ({T}_{1}^{2} + i {T}_{2} {T}_{1} - i {T}_{1} {T}_{2} - i^{2} {T}_{2}^{2}) \\
                    &= ({T}_{1}^{2} - i {T}_{1} {T}_{2} + i {T}_{1} {T}_{2} - {i}^{2} T^{2}_2) \\
                    &= {T}_{1} ({T}_{1} - i {T}_{2}) +  i {T}_{2} ({T}_{1} - i {T}_{2}) \\
                    &= ({T}_{1} - i {T}_{2}) (T + i {T}_{2}) \\
                    &= T^{*} T.
         \end{align*}
         Thus, we conclude that \( T^{*} T = T T^{*} \).
        \end{proof}
\end{enumerate}

\subsection*{Exercise 6.4.7} Let \( T  \) be a linear operator on an inner product space \( V  \), and let \( W  \) be a \( T- \)invariant subspace of \( V  \). Prove the following results. 
\begin{enumerate}
    \item[(a)] If \( T  \) is self-adjoint, then \( {T}_{W}  \) is self-adjoint. 
        \begin{proof}
        Suppose that \( T  \) is self-adjoint. Our goal is to show that \( {T}_{W} \) is also self-adjoint. First, we show that \( W  \) is \( T^{*}  \) invariant. Let \( y \in T^{*}(W) \). Then \( y = T^{*}(x) \) for \( x \in W  \). But \( T  \) is self-adjoint, so \( y = T^{*}(x) = T(x)  \) where \( T(x) \in W  \). Thus, \( y \in W  \) and so \( W  \) is \( T^{*} \) invariant. Therefore, we may place a restriction \( {T}_{W}  \) such that \( {T}_{W}(x) = T(x) \) and \({T}_{W}^{*}(x) = T^{*}(x)  \). Thus,
     for any \( x \in W  \), we have
        \begin{align*}
            {T}_{W}(x) = T(x) = T^{*}(x) = {T}_{W}^{*}(x)
        \end{align*}
        and we are done.
        \end{proof}
    \item[(b)] \( W^{\perp}  \) is \( T^{*}- \)invariant.
        \begin{proof}
        Our goal is to show that \( T^{*}(W^{\perp}) \subseteq W^{\perp} \). Let \( y \in T^{*}(W^{\perp}) \). Thus, \( y = T^{*}(x)  \) for \( x \in W^{\perp}  \). Note that \( y \in W   \) since \( W   \) is \( T^{*}- \)invariant. We need to show that \( \langle y , w  \rangle = 0  \) for all \( w \in W  \). Then observe that
        \[  \langle y , w \rangle = \langle T^{*}(x) , w  \rangle = \langle x  , T(w) \rangle.  \]
        Since \( x \in W^{\perp} \) and that \( W  \) is \( T- \)invariant (that is, \( T(w) \in W  \)), \( \langle x  , T(w) \rangle = 0  \). So, we conclude that \( \langle  y  , w  \rangle = 0  \) and that \( y \in W^{\perp} \). So, \( W^{\perp} \) is \( T^{*} -  \)invariant.
        \end{proof}
    \item[(c)] If \( W  \) is both \( T- \) and \( T^{*}- \)invariant, then \( ({T}_{W})^{*} = {(T^{*})}_{W} \).
        \begin{proof}
        Suppose that \( W  \) is both \( T-  \) and \( T^{*}- \) invariant. We can place restrictions on \( T  \) by having functions \( {T}_{W} \) and \( (T^{*})_W  \). Let \( x \in W  \). Then
        \[  ({T}_{W})^{*} = T^{*}(x) = (T^{*})_W (x). \]
        Thus, we conclude that \( ({T}_{W})^{*} = (T^{*})_W   \).
        \end{proof}
    \item[(d)] If \( W  \) is both \( T- \) and \( T^{*}-  \)invariant and \( T  \) is normal, then \( {T}_{W} \) is normal.
        \begin{proof}
        Suppose that \( W  \) both \( T- \) and \( T^{*}- \) invariant. Using part (c) and the fact that \( T  \) is normal, we can write
        \begin{align*}
            {T}_{W} ({T}_{W})^{*} = {T}_{W} (T^{*})_W = T T^{*} = T^{*}T = (T^{*})_W {T}_{W} = ({T}_{W})^{*} {T}_{W}.
        \end{align*}
        Thus, we conclude that \( {T}_{W} \) is normal.
\end{proof}
\end{enumerate}

\subsection*{Exercise 6.4.8} Let \( T  \) be a normal operator on a finite-dimensional complex inner product space \( V  \), and let \( W  \) be a subspace of \( V  \). Prove that if \( W  \) is \( T- \)invariant, then \( W  \) is also \( T^{*}- \)invariant.

\begin{proof}

\end{proof}

\subsection*{Exercise 6.4.9} Let \( T  \) be a normal operator on a finite-dimensional inner product space \( V  \). Prove that \( N(T) = N(T^{*}) \) and \( R(T) = R(T^{*}) \).

\begin{proof}
Let \( T  \) be a normal operator on a finite-dimensional inner product space \( V  \). Let \( x \in N(T)  \). Then \( T(x) = 0  \) for all \( x \in V  \). By part (a) of Theorem 6.15, we have
\[  0 = \|T(x)\| = \|T^{*}(x)\|. \]
By Theorem 6.1, we have \( T^{*}(x) =0  \). So, \( x \in N (T^{*}) \) and thus \( N(T) \subseteq N(T^{*}) \). The other containment is just the reverse of this argument. Thus, \( N(T) = N(T^{*}) \). Using part (b) of Exercise 12 from Section 6.3, we have 
\[  R(T^{*}) = N(T)^{\perp} = N(T^{*})^{\perp} = R(T^{**}) = R(T). \]
Thus, \( R(T^{*}) = R(T) \).
\end{proof}


\subsection*{Exercise 6.4.10} Let \( T  \) be a self-adjoint operator on a finite-dimensional inner product space \( V  \). Prove that for all \( x \in V  \).
\[ \|T(x) \pm ix\|^{2} = \|T(x)\|^{2} + \|x\|^{2}.  \]
Deduce that \( T - i I  \) is invertible and that the adjoint of \( (T - iI )^{-1} \) is \( (T + i I )^{-1} \).
\begin{proof}
Let \( T  \) be a self-adjoint operator and \( x \in V  \). Note that \( \langle T(x)  , x  \rangle = \langle x , T(x) \rangle \). By Exercise 19 of Section 6.1, we have
\begin{align*}
    \|T(x) + ix \|^{2} &= \|T(x)\|^{2} + 2 \Re \langle T(x) , ix  \rangle + \|ix \|^{2} \\
                       &= \|T(x)\|^{2} + \langle T(x)  , ix  \rangle + \langle ix  , T(x)  \rangle + \|x\|^{2}  \\
                       &= \|T(x)\|^{2} +  i \langle T(x)   , x \rangle - i \langle T(x)  , x  \rangle + \|x\|^{2} \\  
                       &= \|T(x)\|^{2} + \|x\|^{2}.
\end{align*}
Similarly, we have
\begin{align*}
    \|T(x) - ix \|^{2} &= \|T(x)\|^{2} - 2 \Re \langle T(x) , ix  \rangle + \|ix\|^{2} \\
                       &= \|T(x)\|^{2} - \langle T(x)  , ix  \rangle - \langle ix , T(x) \rangle + \|x\|^{2} \\
                       &= \|T(x)\|^{2} + i \langle T(x)  , x  \rangle - i \langle T(x)  , x  \rangle + \|x\|^{2} \\
                       &= \|T(x)\|^{2} + \|x\|^{2}.
\end{align*}
Thus, we conclude that \( \|T(x) \pm ix\|^{2} = \|T(x)\|^{2} + \|x\|^{2} \). Note that \( T - iI  \) is invertible since \( \text{det}(T - iI) \neq 0  \). By Exercise 8 of Section 6.3, we see that \( (T - iI)^{*} = T + iI \) is invertible and that 
\[ (T + iI)^{*} = \Big( (T - iI)^{*} \Big)^{-1}.  \]
\end{proof}

\subsection*{Exercise 6.4.11} Assume that \( T  \) is a linear operator on a complex (not necessarily finite-dimensional) inner product space \( V  \) with an adjoint \( T^{*} \). Prove the following results.
\begin{enumerate}
    \item[(a)] If \( T  \) is self-adjoint, then \( \langle T(x) , x \rangle \) is real for all \( x \in V  \).
        \begin{proof}
        Suppose \( T  \) is self-adjoint. Then observe that, for any \( x \in V  \), we have
        \[  \langle T(x) , x \rangle = \langle x  , T^{*}(x) \rangle = \langle x  , T(x) \rangle = \overline{\langle T(x) , x \rangle}. \]
        Since \( \langle T(x)  ,  x  \rangle  \) is equal to its conjugate, we conclude that \( \langle T(x)  , x  \rangle  \) is real for all \( x \in V  \).
        \end{proof}
    \item[(b)] If \( T  \) satisfies \( \langle T(x) , x \rangle = 0  \) for all \( x \in V  \), then \( T = {T}_{0} \).
        \begin{proof}
        Suppose \( T  \) satisfies \( \langle T(x)  , x  \rangle = 0  \) for all \( x \in V  \). Replacing \( x  \) with \( x + y \) yields the inner product \( \langle T(x+y) , x+y  \rangle = 0  \) for \( x,y \in V  \). Expanding this inner product yields
        \begin{align*}
            0 = \langle T(x+y) , x + y  \rangle &= \langle T(x) + T(y) ,  x + y  \rangle \\  
                                                &= \langle T(x)  , x  \rangle + \langle T(x)  , y  \rangle + \langle T(y ) , x  \rangle + \langle T(y) , y \rangle  \\
                                                &= \langle T(x) , y \rangle + \langle T(y) , x \rangle
        \end{align*}
         Thus, \( \langle T(x) , y \rangle = - \langle T(y) , x  \rangle \).with \( \langle T(x)  , x  \rangle = 0  \) and \( \langle T(y) , y  \rangle = 0  \) by assumption. So, we have
         \[  \langle T(x) , y \rangle + \langle T(y) , x \rangle = 0 \implies \langle T(x) , y \rangle = - \langle T(y) , x \rangle \tag{1} \]
Now, replace \( x + y  \) with \( x + iy  \). Similarly, we have
        \begin{align*}
            0 = \langle T(x+iy) , x + iy  \rangle &= - i \langle T(x) , y \rangle + i \langle T(y) , x \rangle.   
        \end{align*}
        So, 
        \[  0  =  -\langle T(x) , y \rangle + \langle T(y) , x \rangle \implies \langle T(y) , x \rangle = \langle T(x) , y \rangle \tag{2} \]
        Using (1) and (2), we can now write
        \[  \langle T(x)  , y  \rangle = - \langle T(x) , y \rangle \implies \langle T(x) , y \rangle = 0 \  \text{for any} \  x,y \in V.  \]
        By {\hyperref[Exercise 6.2.17]{Exercise 17 of Section 6.2}}, we conclude that \( T = {T}_{0} \).
        \end{proof}
    \item[(c)] If \( \langle T(x) , x \rangle  \) is real for all \( x \in V  \), then \( T  \) is self-adjoint.
        \begin{proof}
        Suppose that \( \langle T(x)  , x  \rangle  \) is real for all \( x \in V  \). Then
        \begin{align*}
            \langle x , T^{*}(x) \rangle = \langle T(x) , x \rangle = \overline{\langle T(x) , x  \rangle}  = \langle x  ,  T(x) \rangle.
        \end{align*}
        Using Theorem 6.1, we conclude that \( T^{*}(x) = T(x) \) for all \( x \in V  \). So, \( T^{*}  = T  \).
        \end{proof}
\end{enumerate}

\subsection*{Exercise 6.4.12} Let \( T  \) be a normal operator on a finite-dimensional real inner product space \( V  \) whose characteristic polynomial splits. Prove that \( V  \) has an orthonormal basis of eigenvectors of \( T  \). Hence prove that \( T  \) is self-adjoint.

\begin{proof}
Let \( T  \) be a normal operator on a finite-dimensional real inner product space \( V  \) whose characteristic polynomial splits. By Theorem 6.16, we know there exists an orthonormal basis \( \beta = \{ {v}_{1}, {v}_{2}, \dots, {v}_{n} \}  \) consisting of eigenvectors of \( T  \). Thus, we have the corresponding to eigenvalues \( {\lambda}_{1}, {\lambda}_{2}, \dots, {\lambda}_{n} \in \R  \) of \( {v}_{1}, {v}_{2}, \dots, {v}_{n} \), respectively. Furthermore, \( [T]_{\beta} \) is a diagonal matrix and so is \( [T]_{\beta}^{*} = [T^{*}]_{\beta} \). Note that \( {\lambda}_{j} = \overline{{\lambda}_{j}}  \) for all \( j  \) since \( V \) is a real inner product space. Hence, by part (c) of Theorem 6.15, we can see that  \[  T({v}_{j}) = {\lambda}_{j} {v}_{j} = \overline{\lambda_j} {v}_{j} = T^{*}({v}_{j}). \]
So, \( T = T^{*}  \) which implies that \( T  \) is self-adjoint.
\end{proof}

\subsection*{Exercise 6.4.14} Let \( V  \) be a finite-dimensional real inner product space, and let \( U \) and \( T  \) be self-adjoint linear operators on \( V  \) such that \( UT = TU  \). Prove that there exists an orthonormal basis for \( V  \) consisting of vectors that are eigenvectors of both \( U  \) and \( T  \).
\begin{proof}

\end{proof}




\begin{definition}[Positive Definite]
    A linear operator \( T  \) on a finite-dimensional inner product space is called \textbf{positive definite [positive semidefinite]} if \( T  \) is self-adjoint and \( \langle T(x) , x \rangle > 0  \) [\( \langle T(x) , x \rangle \geq 0  \)] for all \( x \neq 0  \). 
    An \( n \times n  \) matrix \( A  \) with entries from \( \R  \) or \( \C  \) is called \textbf{positive definite [positive semidefinite]} if \( {L}_{A} \) is positive definite [positive definite].
\end{definition}

\subsection*{Exercise 6.4.17} Let \( T  \) and \( U  \) be self-adjoint linear operators on an \( n- \)dimensional inner product space \( V  \), and let \( A = [T]_{\beta} \), where \( \beta  \) is an orthonormal basis for \( V  \). Prove the following results.

\begin{enumerate}
    \item[(a)] \( T  \) is positive definite [semidefinite] if and only if all of its eigenvalues are positive [nonnegative].
        \begin{proof}
        Suppose \( T  \) is positive definite and let \( \beta = \{ {v}_{1}, {v}_{2}, \dots, {v}_{n} \}  \) be an orthonormal basis for \( V  \) consisting of eigenvectors. Let \( \lambda_j \) for \( 1 \leq j \leq n  \) be the eigenvalues of each corresponding eigenvector \( {v}_{1}, {v}_{2}, \dots, {v}_{n}  \). Since each \( {v}_{j} \neq 0  \), we have
        \[  {\lambda}_{j} = {\lambda}_{j} \langle {v}_{j} , {v}_{j}  \rangle = \langle {\lambda}_{j} {v}_{j} , {v}_{j} \rangle = \langle T({v}_{j}) , {v}_{j} \rangle > 0. \]
        Thus, we see that each \( \lambda_j > 0  \) for all \( j  \). Conversely, if each \( {\lambda}_{j} > 0  \) for all \( j  \), then we must have
        \[  \langle T({v}_{j}) , {v}_{j} \rangle = \langle {\lambda}_{j} {v}_{j} , {v}_{j} \rangle = {\lambda}_{j} \langle {v}_{j} , {v}_{j} \rangle = \lambda_j > 0. \]
        Thus, \( T  \) is positive definite.

        \end{proof}
    \item[(b)] \( T  \) is positive definite if and only if 
        \[ \sum_{ i,j }^{  } {A}_{ij} {a}_{j} \overline{{a}_{i} } \ \ \text{for all nonzero} \ n  \text{-tuples} \  ({a}_{1}, {a}_{2}, \dots, {a}_{n}).  \]
        \begin{proof}
        Suppose \( T  \) is positive definite. Let \( A = [T]_{\beta} \) where \( \beta  \) is an orthonormal basis of \( V  \) consisting of eigenvectors of \( T  \). Note that \( A  \) must be a diagonal matrix, so each \( {A}_{ij} = {\lambda}_{j} \). Furthermore, \( \lambda_j > 0  \) for all \( j  \) since \( T  \) is positive definite by part (a). Let \( ({a}_{1}, {a}_{2}, \dots, {a}_{n}) \) be a nonzero tuple. Using the corollary to Theorem 6.5, we have 
        \[  {A}_{ij} = \langle T({v}_{j}) , {v}_{i} \rangle. \]
        Since \( \beta  \) is an orthonormal basis, we find that
        \begin{align*}
           \sum_{ i,j }^{  } {A}_{ij} {a}_{j} \overline{{a}_{i}} = \sum_{ i,j }^{  } \langle T({v}_{j}) , {v}_{i}  \rangle {a}_{j} \overline{{a}_{i}} 
                                                                 &= \sum_{ i,j  }^{  } \langle {\lambda}_{j} {v}_{j} , {v}_{i}  \rangle {a}_{j} \overline{{a}_{i}} \\
                                                                 &= \sum_{ i,j  }^{  } {\lambda}_{j} \langle {v}_{j} , {v}_{i} \rangle {a}_{j} \overline{{a}_{i}} \\
                                                                 &= \sum_{ i=1  }^{ n } {\lambda}_{i} {a}_{i} \overline{{a}_{i}} \\ 
                                                                 &= \sum_{ i=1  }^{ n } {\lambda}_{i} | {a}_{i} |^{2} > 0
        \end{align*}
        which is our desired result for the forwards direction. Conversely, suppose that 
        \[ \sum_{ i,j }^{  } {A}_{ij} {a}_{j} \overline{{a}_{i}} > 0   \]
        for all nonzero tuples \( ({a}_{1}, {a}_{2}, \dots, {a}_{n})  \). Let \( x \neq 0  \) in \( V  \). Since \( \beta  \) is an orthonormal basis for \( V  \), we have 
        \[  x = \sum_{ i=1  }^{ n } \langle x , {v}_{i} \rangle {v}_{i}. \]
        Denote \( {a}_{i} = \langle x  , {v}_{i}  \rangle \). Our goal is to show that \( \langle T(x) , x \rangle > 0  \). Using the corollary to Theorem 6.5 again, we can see that
        \begin{align*}
            \langle T(x) , x  \rangle &= \Big\langle T \Big(  \sum_{ j=1  }^{ n } \langle x , {v}_{j}  \rangle {v}_{j} \Big)  , \sum_{ i=1  }^{ n } \langle x , {v}_{i} \rangle {v}_{i}  \Big\rangle \\
&= \Big\langle \sum_{ j=1  }^{ n } \langle x , {v}_{j} \rangle T({v}_{j}), \sum_{ i=1  }^{ n } \langle x , {v}_{i} \rangle {v}_{i} \Big\rangle \\
&= \sum_{ j=1  }^{ n } \langle x , {v}_{j} \rangle \sum_{ i=1 }^{ n  } \overline{\langle x , {v}_{i} \rangle} \langle T({v}_{j}) , {v}_{i} \rangle \\
&= \sum_{ i,j }^{  } \langle T({v}_{j}) , {v}_{i} \rangle \langle x , {v}_{j} \rangle \overline{\langle x , {v}_{i} \rangle} \\
&= \sum_{ i,j }^{  } {A}_{ij} \langle x , {v}_{j} \rangle \overline{\langle x , {v}_{i} \rangle} > 0.
\end{align*}
Note that \( T  \) is self-adjoint by assumption. Thus, we conclude that \( T  \) is positive definite.
\end{proof}
    \item[(c)] \( T  \) is positive semidefinite if and only if \( A = B^{*} B  \) for some square matrix \( B  \).
        \begin{proof}
        
        \end{proof}
    \item[(d)] If \( T  \) and \( U  \) are positive semidefinite operators such that \( T^{2} = U^{2} \), then \( U = T  \).     
        \begin{proof}
        Let \( \beta = \{ {v}_{1}, {v}_{2}, \dots, {v}_{n} \}   \) be an orthonormal basis for \( V  \) consisting of eigenvectors. Let \( {\lambda}_{i} \) and \( {\lambda}_{i}' \) be eigenvalues of these eigenvectors \( T  \) and \( U  \), respectively. Since \( T  \) and \( U  \) are positive semidefinite operators such that \( T^{2} = U^{2} \), we must have 
       \begin{align*}
           \langle T^{2}({v}_{i}) , {v}_{i} \rangle = \langle U^{2}({v}_{i}) , {v}_{i} \rangle &\implies \langle \lambda_i^{2} {v}_{i}  ,  {v}_{i}  \rangle = \langle  {\lambda}_{i}'^{2} {v}_{i}  ,  {v}_{i}  \rangle \\
                                                                                               &\implies {\lambda}_{i}^{2} \langle {v}_{i}  , {v}_{i}  \rangle = {\lambda}_{i}'^{2} \langle {v}_{i}  , {v}_{i}  \rangle \\
                                                                                               &\implies {\lambda}_{i}^{2} = {\lambda}_{i}'^{2}. 
       \end{align*} 
       Thus, \( {\lambda}_{i} = {\lambda}_{i}' \). This implies that \( T = U  \).

        \end{proof}
    \item[(e)] If \( T  \) and \( U  \) are positive definite operators such that \( TU = UT  \), then \( TU \) is positive definite.
        \begin{proof}
        Suppose \( T  \) and \( U  \) are positive definite operators such that \( TU = UT  \). Let \( x \) be an eigenvector of \( T  \) and \( U  \) where \( T(x) = {\lambda}_{1}x  \) and \( U(x) = {\lambda}_{2}x  \). Since \( T  \) and \( U  \) are positive definite operators, we have \( {\lambda}_{1} > 0  \) and \( {\lambda}_{2} > 0  \). So, we have
        \begin{align*}
            \langle TU(x) , x  \rangle = \langle UT(x) , x  \rangle &= \langle U(T(x)) , x  \rangle \\
                                                                    &= \langle U({\lambda}_{1} x ) , x  \rangle \\
                                                                    &= \langle {\lambda}_{1} U(x) , x  \rangle \\
                                                                    &= \langle {\lambda}_{1} {\lambda}_{2} x  , x  \rangle \\
                                                                    &= {\lambda}_{1} {\lambda}_{2} \langle x  , x  \rangle >0. 
        \end{align*}
        Furthermore, \( UT = TU  \) is self-adjoint since 
        \[  TU = T^{*}U^{*} = (UT)^{*} = (TU)^{*} \] where \( T  \) and \( U  \) are self-adjoint. Thus, \( TU  \) is positive definite.
        \end{proof}
    \item[(f)] \( T  \) is positive definite [semidefinite] if and only if \( A  \) is positive definite [semidefinite].
        \begin{proof}
            Suppose \( T  \) is positive definite and let \( A = [T]_{\beta} \) where \( \beta  \) is an orthonormal basis for \( V  \) consisting of eigenvectors of \( T  \). Let \( {\lambda}_{1}, {\lambda}_{2}, \dots, {\lambda}_{n} \) be the eigenvalues of eigenvectors \( {v}_{1}, {v}_{2}, \dots, {v}_{n} \), respectively. By Exercise 6 of Section 5.1, the \( {\lambda}_{1}, {\lambda}_{2}, \dots, {\lambda}_{n} \) are also eigenvalues of \( A  \). Since \( T  \) is positive definite, \( {\lambda}_{i} > 0  \) for all \( 1 \leq i \leq n  \). Thus, the eigenvalues \( {\lambda}_{i} \) of \( {L}_{A}  \) are all greater than zero, and therefore, \( A  \) must be positive definite. To show the converse, the argument is reversible. 
        \end{proof}
\end{enumerate}

\subsection*{Exercise 6.4.18} Let \( T: V \to W  \) be a linear transformation, where \( V  \) and \( W  \) are finite-dimensional inner product spaces. Prove the following results.
\begin{enumerate}
    \item[(a)] \( T^{*} T \) and \(  T T^{*} \) are positive semidefinite.
        \begin{proof}
            Since \( V  \) and \( W  \) are finite-dimensional inner product spaces, we can construct orthonormal bases \( \beta  \) and \( \gamma \) for \( V  \) and \( W  \), respectively via the Gram-Schmidt Process. That is, let \( \beta = \{ {v}_{1}, {v}_{2}, \dots, {v}_{m} \}  \) and \( \gamma = \{ {w}_{1}, {w}_{2}, \dots, {w}_{n} \}  \) where \( T({v}_{i}) = {w}_{i} \) for all \( i \). Let \( \|\cdot\|_1  \) and \( \|\cdot\|_2 \) be the norms for \( V  \) and \( W  \), respectively. If \( x \in V  \), then 
            \[  x = \sum_{ i=1  }^{ n } \langle x , {v}_{i} \rangle {v}_{i}. \]
            Our goal is to show that \( \langle T^{*}T(x) , x \rangle \geq 0 \). First, observe that
            \[  \langle T^{*}T(x) , x \rangle_1 = \overline{\langle x , T^{*}T(x) \rangle_1} =  \overline{\langle T(x) , T(x) \rangle_2} - \langle T(x) , T(x) \rangle_2 = \|T(x)\|_2^{2}. \]
            Since \( \gamma \) is an orthonormal basis, we can use {\hyperref[Exercise 6.1.12]{Exercise 6.1.12}}, to write
            \begin{align*}
                \|T(x)\|^{2}_2 = \Big\| T \Big(  \sum_{ i=1  }^{ n } \langle x , {v}_{i} \rangle_1 {v}_{i} \Big) \Big\|^{2}_2 &= \Big\| \sum_{ i=1  }^{ n } \langle x , {v}_{i} \rangle_1 {w}_{i} \Big\|^{2}_2 \\
                &= \sum_{ i=1  }^{ n } |\langle x , {v}_{i} \rangle_1|^{2} \|{w}_{i}\|^{2}_2 \\
                &= \sum_{ i=1  }^{ n } | \langle x , {v}_{i} \rangle_1 |^{2} \geq 0.
            \end{align*}
            Thus, we have that \( \langle T^{*}T(x) , x \rangle \geq 0 \). Note that \( T T^{*} \) is self adjoint because  
        \[  (T T^{*})^{*} = T^{**} T^{*} = T T^{*}. \]
        Thus, \( T^{*} T  \) is positive semidefinite.  

        Now, let \( y \in W  \) and define \( T^{*}({w}_{i}) = {v}_{i} \).
        Observe that \[  \langle T T^{*}(y) , y  \rangle_2 = \langle T^{*}(y) , T^{*}(y) \rangle_1 = \|T^{*}(y)\|^{2}_1.  \]
    By {\hyperref[Exercise 6.1.12]{Exercise 6.1.12}}, we can write
    \begin{align*}
        \|T^{*}(y)\|^{2}_1 &= \Big\| T^{*} \Big(  \sum_{ i=1  }^{ n } \langle y , {w}_{i} \rangle_2 {w}_{i} \Big) \Big\|^{2}_1  \\
                           &= \Big\| \sum_{ i=1  }^{ n } \langle y , {w}_{i} \rangle_2 T^{*}({w}_{i})  \Big\|^{2}_1 \\
                           &= \sum_{ i=1  }^{ n } | \langle y , {w}_{i} \rangle_2 |^{2} \|T^{*}({w}_{i})\|^{2}_1 \\
                           &= \sum_{ i=1  }^{ n } \|\langle y , {w}_{i} \rangle_2\|^{2} \|{v}_{i}\|^{2}_1 \\
                           &= \sum_{ i=1  }^{ n } | \langle y , {w}_{i} \rangle_2 |^{2} \geq 0.
    \end{align*}
    Furthermore, \( T T^{*} \) is self-adjoint since
    \[  (T T^{*})^{*} = T^{**} T^{*} = T T^{*}. \]
    Thus, \( T T^{*} \) is positive semidefinite.
\end{proof}
    \item[(b)] \( \text{rank}(T^{*}T) = \text{rank}(T T^{*}) = \text{rank}(T) \).        Since \(   \) 
        \begin{proof}
        Apply parts (a) and (b) of Exercise 6.3.13 to obtain the result.
        \end{proof}
\end{enumerate}



\subsection*{Exercise 6.4.19} Let \( T  \) and \( U  \) be positive definite operators on an inner product space \( V  \). Prove the following results.
\begin{enumerate}
    \item[(a)] \( T + U  \) is positive definite.
        \begin{proof}
         Let \( x \in V  \) be nonzero. Observe that  
        \[  \langle (T+U)(x) , x \rangle = \langle T(x) + U(x) , x  \rangle = \langle T(x) , x  \rangle + \langle U(x) , x \rangle.  \]
        Since \( T  \) and \( U  \) are positive definite operators, we must have 
        \begin{center}
            \( \langle T(x) , x \rangle > 0  \) and \( \langle U(x) , x \rangle > 0  \).
        \end{center}
        This tells us that \( \langle (T+U)(x) , x \rangle > 0   \) and thus, \( T + U  \) is positive definite.
\end{proof}
    \item[(b)] If \( c > 0  \), then \( cT  \) is positive definite.
        \begin{proof}
        Let \( c > 0  \). Then for any \( x \in V  \) nonzero, 
        \[  \langle (cT)(x) , x  \rangle = \langle c T(x) , x \rangle = c \langle T(x) , x \rangle > 0  \]
        since \( \langle T(x) , x \rangle > 0  \) by assumption. So, \( cT \) is positive definite.
        \end{proof}
    \item[(c)] \( T^{-1}  \) is positive definite.
        \begin{proof}
            Suppose that \( T  \) is invertible and \( T  \) is positive-definite. Let \( x \in V   \) be an eigenvector of \( T  \) corresponding to the eigenvalue \( \lambda \). Recall by {\hyperref[Exercise 5.1.8]{Exercise 8 of Section 5.1}} that for any eigenvalue \( \lambda \) of \( T  \), \( \lambda^{-1} \) is an eigenvalue of \( T^{-1} \). So, observe that
            \[  \langle T^{-1}(x) , x \rangle = \langle \lambda^{-1} x  , x  \rangle = \lambda ^{-1} \langle x  ,  x  \rangle.  \]
            Since \( x \neq 0  \) and \( \lambda > 0  \) (by part (a) of Exercise 6.4.18), we have \( \langle x  , x  \rangle > 0  \) and \( \lambda^{-1} > 0 \). This implies that \( \langle T^{-1}(x) , x \rangle > 0  \). So, \( T^{-1}  \) is positive-definite.  
        \end{proof}
\end{enumerate}

\subsection*{Exercise 6.4.20} Let \( V  \) be an inner product space with inner product \( \langle \cdot , \cdot \rangle \), and a positive definite linear operator on \( V  \). Prove that \( \langle x , y \rangle' = \langle T(x) , y \rangle \) defines another inner product on \( V  \).
\begin{proof}
   \begin{enumerate}
       \item[(a)] Let \( x,y,z \in V  \) be nonzero. Then we have
           \begin{align*}
               \langle x+y , z  \rangle' = \langle T(x+y) , z \rangle = \langle T(x) + T(y) , z \rangle &= \langle T(x) , z \rangle + \langle T(y) , z \rangle  \\
                                                                            &= \langle x , z \rangle' + \langle y , z \rangle'.
    \end{align*}
    \item[(b)] Let \( c \in F  \). Then
        \[  \langle cx , y \rangle' = \langle T(cx) , y \rangle = \langle cT(x) , y \rangle = c \langle T(x) , y \rangle = c \langle x , y \rangle'.  \]
        Thus, part (b) is satisfied. 
    \item[(c)] Note that \( T  \) is self-adjoint, so 
        \[  \overline{\langle x , y \rangle'} = \overline{\langle T(x) , y \rangle} = \langle y , T(x) \rangle = \langle T^{*}(x) , y \rangle = \langle T(x) , y \rangle = \langle x , y \rangle'.  \]
        So, part (c) is satisfied.
    \item[(d)] Since \( T  \) is a positive definite operator, we have \[ \langle x , x \rangle' = \langle T(x) , x \rangle > 0.  \]
    Thus, property (d) of the inner product is satisfied and so we conclude that \( \langle x , y \rangle' = \langle T(x) , y \rangle  \) defines an inner product on \( V  \).
\end{enumerate}  
Thus, we conclude that \( \langle x , y \rangle' = \langle T(x) , y \rangle \) defines an inner product on \( V  \). 

\end{proof}

\subsection*{Exercise 6.4.21} Let \( V  \) be a finite-dimensional inner product space, and let \( T  \) and \( U  \) be self-adjoint operators on \( V  \) such that \( T  \) is positive definite. Prove that both \( TU  \) and \( UT  \) are diagonalizable linear operators that have only real eigenvalues.
\begin{proof}
Let \( V  \) be a finite-dimensional inner product space, and let \( T  \) and \( U  \) be self-adjoint operators on \( V  \) such that \( T  \) is positive-definite. Since \( U  \) is self-adjoint, there exists an orthonormal basis \( \beta = \{ {v}_{1}, {v}_{2}, \dots, {v}_{n} \}  \) consisting of eigenvectors of \( U \). Using the hint available to us, suppose this orthonormal basis is given with respect to the inner product \( \langle x , y \rangle' = \langle T(x) , y \rangle \). We will show that \( UT  \) is self-adjoint by showing that \( (UT)^{*} = UT \). Let \( {A}_{ij} = \langle UT({v}_{j}) , {v}_{i} \rangle \). Since \( T  \) and \( U  \) are self-adjoint, we must have the following hold
\begin{align*}
     \langle UT({v}_{j}) , {v}_{i} \rangle = \langle T({v}_{j}) , U({v}_{i}) \rangle 
                                                     &= \langle {v}_{j} ,  U({v}_{i}) \rangle' \\
                                                     &= \langle {v}_{j} , {\lambda}_{j} {v}_{j} \rangle' \\
                                                     &= {\lambda}_{i} \langle {v}_{j} , {v}_{i} \rangle' \\
                                                     &= {\lambda}_{i} {\delta}_{ij}
\end{align*}
which implies that \( UT({v}_{j}) = {\lambda}_{j} {v}_{j} \) whenever \( i = j  \). Note that the eigenvalues of \( U  \) are only real eigenvalues by the lemma. On the other hand, we also see that
\begin{align*}
    \langle (UT)^{*}({v}_{j}) , {v}_{i}  \rangle = \langle T^{*}U^{*}({v}_{j}) , {v}_{i} \rangle &= \langle TU({v}_{j}) , {v}_{i} \rangle \\
                                                                                                  &= \langle U({v}_{j}) , {v}_{i} \rangle' \\
                                                                                                  &= \langle {\lambda}_{j}{v}_{j} , {v}_{i} \rangle' \\
                                                                                                  &= {\lambda}_{j} \langle {v}_{j} , {v}_{i} \rangle' \\
                                                                                                  &= {\lambda}_{j} {\delta}_{ij}
\end{align*}
which implies that \( (UT)^{*}({v}_{j}) = {\lambda}_{j} {v}_{j} \) for all \( i = j  \). Thus, \( UT  \) is self-adjoint which implies that \( [UT]_{\beta} \) is diagonalizable if and only if \( UT \) is diagonalizable. Furthermore, the eigenvalues of \( UT \) are only real eigenvalues given by lemma. Now, we will show that \( TU  \) is diagonalizable. We will refer to the hint again, replacing \( T^{-1} \) with \( T  \). Since \( T  \) is self-adjoint, we know that \( T^{-1} \) is also self-adjoint where \( (T^{*})^{-1} = (T^{-1})^{*}  \). Furthermore, \( T^{-1} \) is positive definite by part (c) of Exercise 6.4.19. Thus, we have
\begin{align*}
    \langle TU({v}_{j}) , {v}_{i}  \rangle' = \langle U({v}_{j}) , T({v}_{i}) \rangle' &= \langle T^{-1} U({v}_{j}) , T({v}_{i})  \rangle \\
                                                                                       &= \langle U({v}_{j}) , {v}_{i} \rangle \\
                                                                                       &= \langle {\lambda}_{j} {v}_{j} ,  {v}_{i} \rangle \\
                                                                                       &= {\lambda}_{j} \langle {v}_{j} ,  {v}_{i} \rangle \\ 
                                                                                       &= {\lambda}_{j} {\delta}_{ij}.
\end{align*}
So, \( (TU)({v}_{j}) = {\lambda}_{j} {v}_{j} \). Similarly, we have
\begin{align*}
    \langle (TU)^{*}({v}_{j}) , {v}_{i} \rangle'  = \langle U^{*}T^{*}({v}_{j}) , {v}_{i} \rangle' &= \langle UT({v}_{j}) , {v}_{i} \rangle' \\
                                                                                                   &= \langle T({v}_{j}) , U({v}_{i}) \rangle' \\
                                                                                                   &= \langle {v}_{j} ,  {\lambda}_{i} {v}_{i} \rangle \\ 
                                                                                                   &= \overline{{\lambda}_{i}} \langle {v}_{j}  ,  {v}_{i} \rangle \\
         &=  {\lambda}_{i} \langle {v}_{j} , {v}_{i} \rangle \\
         &= {\lambda}_{i} {\delta}_{ij}.
\end{align*}
Thus, \( (TU)^{*}({v}_{j}) = {\lambda}_{j} {v}_{j}  \) for all \( i = j  \). Therefore, \( (TU)^{*}  = TU  \) is self-adjoint which implies that \( TU \) is diagonalizable with only real eigenvalues. 

\end{proof}

\subsection*{Exercise 6.4.22} This exercise provides to Exercise 20. Let \( V \) be a finite-dimensional inner product space with inner product \( \langle \cdot , \cdot \rangle, \) and let \( \langle \cdot , \cdot \rangle' \) be any other inner product on \( V  \).
\begin{enumerate}
    \item[(a)] Prove that there exists a unique linear operator \( T  \) on \( V  \) such that \( \langle x , y \rangle' = \langle T(x) , y \rangle  \) for all \( x  \) and \( y  \) in \( V  \). 
        \begin{proof}
        
        \end{proof}
    \item[(b)] Prove that the operator \( T  \) of (a) is positive definite with respect to both inner products.
        \begin{proof}
       
        \end{proof}
\end{enumerate}
