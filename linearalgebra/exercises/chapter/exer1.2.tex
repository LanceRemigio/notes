\section{Vector Spaces}


\subsubsection{Exercise 1.2.1}

Label the following statements as true or false.

\begin{enumerate}
    \item[(a)] Every vector space contains a zero vector.
    \item[(b)] A vector space may have more than one zero vector.
    \item[(c)] In any vector space, \( ax = bx  \) implies that \( a = b  \).
    \item[(d)] In any vector space, \( ax = ay  \) implies that \( x = y \). 
\end{enumerate}

\subsubsection{Exercise 1.2.7} Let \( S  = \{ 0,1 \}   \) and \( F = \R   \). In \( \mathcal{F}(S,\R \), show that \( f = g  \) and where \( f(t) = 2t + 1  \), \( g(t) = 1 +4t - 2 t^{2}  \), and \( h(t) = 5^{t} + 1  \).
\begin{proof}
To show that \( f = g  \), we have to show that for each \( s \in S  \) that \( f(s) = g(s) \). Since \( S = \{ 0,1 \}  \), we can just evaluate both \( f  \) and \( g  \) for elements in \( S  \). Note that 
\[  f(0) = 2(0) + 1 = 1   \]
and likewise, 
\[  g(0) = 1 + 4(0) - 2(0)^{2}.\]
Hence, \( f(0) = g(0) \).  Now let us evaluate both functions \( f  \) and \( g  \) at \( s = 1  \). Hence, we have 
\[  f(1) = 2(1) + 1 = 3  \]
and
\[  g(1) = 1 + 4(1) - 2(1)^{2} = 3.  \]
Thus, we must have \( f(s) = g(s)  \) for all \( s \in S  \).

Now, we need to show that \( f + g = h  \). Like we did above, we have to show that this is the case for all \( s \in S  \). Note that 
\[  (f+g)(s) = f(s) + g(s). \]
Hence, we have 
\[  f(s) + g(s) = 2 + 6s  - 2s^{2}. \]
Evaluating at \( s = 0  \), we have 
\[  f(0) + g(0) = 2.\] and likewise,  
\[  h(0) = 5^{0} + 1 = 2. \] Hence, \(  (f+g)(0) = h(0) \).
Now let us evaluate \( f + g  \) at \( s = 1  \)  
\[ f(1) + g(1) = 2 + 6(1) - 2(1)^{2} = 6    \]
and likewise, we have 
\[  h(1) = 5^{1} + 1 = 6.\]
Hence, we have \( (f+g)(1) = h(1) \). Thus, we have that 
\( f+g = h \) for all \( s \in S \).

\end{proof}

\subsubsection{Exercise 1.2.8}
In any vector space \( V  \), we have 
\[  (a+b)(x+y) = ax + ay + bx + by  \] for any \( x ,y \in V  \) and any \( a,b \in F  \).
\begin{proof}
Observe the following set equalities:
\begin{align*}
    (a+b)(x+y) &= a(x+y) + b(x+y) \tag{VS 8} \\
               &= ax + ay + bx + by \tag{VS 7}.
\end{align*}
Hence, we have that 
\[  (a+b)(x+y) = ax + ay + bx + by  \] for any \( x ,y \in V  \) and any \( a,b \in F  \).
\end{proof}



\subsubsection{Exercise 1.2.10} Let \( V  \) denote the set of all differentiable real-valued functions defined on the real line. Prove that \( V  \) is a vector space with the operations of addition and scalar multiplication defined in Example 3.  
\begin{proof}
    Let \( f,g \in \mathcal{F }(S,\R ) \) and let \( x \in S  \). Using the addition operation defined in Example 3, we have  
    \[  (f+g)(x) = f(x) + g(x). \]
    Since \( f(x) , g(x) \in \R  \) where \( \R  \) is a field, we know that commutativity is preserved. Hence, we have that 
    \[  f(x) + g(x) = g(x) + f(x). \]
    By the same reasoning, we have that associativity is also preserved.

 Since \( \R  \) is a field, we also know that there exists an element \( O  \) in \( \R  \) such that for every \( y \in \R  \), we have \( y + O = y  \). Since \( f(y) \in \R  \), there exists an  \(f_{0}   \) such that   \( f(y) + f_{0} = f(y)  \). Hence, the third property is also satisfied. Likewise, the fourth property is also satisfied since \( \R  \) is a field.
Since \( \R  \) is a field in which multiplicative identities exists and that for every \( x \in S  \) with \( f(x) \in \R  \), we know that \( 1 \cdot f(x) = f(x)  \). Hence, property 5 is satisfied.

 Let \( x \in S  \). Since \( f(x) \in \R  \) and \( \R  \) is a field, we also know that properties 6 and 7 are satisfied. Now, let \( f \in \mathcal{F}(S, \R ) \) and \( x \in S  \). Let \( a,b \in \R  \). Using the operations of addition and scalar multiplication and the fact that \( \R  \) is a field, we have that 
 \[ (a+b)f(x) = a f(x) + b f(x).\]
 Hence, we conclude that \( V = \mathcal{F}(S, \R ) \) is a vector space.

\end{proof}



\subsubsection{Exercise 1.2.11} Let \( V = \{ 0  \}  \) consist of a single vector \( O  \) and define \( O + O = O  \) and \( c O = O  \) for each scalar \( c \in F  \). Prove that \( V  \) is a vector space over \( F  \).
\begin{proof}
    Let \( x,y \in O  \). Since \( V = \{ 0  \}  \), we know that \( x  \) and \( y  \) are both the \( 0  \) vector. By using the addition defined on \( V  \), we can see that 
\[  x + y = O + O = y + x. \] 
Note that \( x + y \in V \) implies that \( x+ y = O + O  \) and  likewise \( y + z = O + O  \) for every  \( x, y, z \in V  \). Hence, we have that 
\[  x + ( y + z ) = O + (O + O) = (O+ O) + O =  (x+y) + z. \]

Since \( V  \) consists of only the zero vector \( O  \), we know that (VS 3) and (VS 4) are satisfied. By the same reasoning, (VS 5) is satisfied because 
\[  1 \cdot O = O. \]
Let \( a,b \in F  \). Then (VS 6) and (VS 7) are satisfied because 
\begin{center}
    \( (ab)O = 0 = a \cdot 0 =  a (bO)  \) and \( a ( O + O ) = 0 = O + O  =  a O + a O\) respectively.
\end{center}
Let \( a,b \in F  \) again. Then we have 
\begin{align*}
    (a+b) O &= 0  \\
            &= O + O \\
            &= a O + b O. 
\end{align*}
Hence, (VS 8) is satisfied. 
\end{proof}

\subsubsection{Exercise 1.2.12}

A real-valued function \( f  \) defined on the real line is called an \textbf{even function} if \( f(-t) = f(t)  \) for all \( t \in \R  \). Prove that the set of even functions defined on \( \R  \) with the operations of addition and scalar multiplication defined in Example 3 is a vector space.

\begin{proof}
    First, let us show that for every \( t \in \R  \) that \( (f+g)(t)  \) is also an even function for every pair of even functions \( f,g  \) and likewise \( (cf)(t)  \) is an even function for every \( c \in \R  \). Observe that 
    \begin{align*}
        (f+g)(-t) &= f(-t) + g(-t) \\
                  &= f(t) + g(t) \\
                  &= (f+g)(t).
    \end{align*}
    Hence, the function \( f + g  \) is also even. Now observe that 
    \begin{align*}
        (cf)(-t) &= cf(-t)  \\
                 &= c f(t) \\
                 &= (cf)(t).
    \end{align*}
    Hence, \( cf  \) is also even function.

    Now we can show that the set of even functions is a vector space. 
    \begin{enumerate}
        \item[(VS 1)] Let  \( f,g  \) be a pair of real-valued even functions.  Let \( t \in \R  \) be arbitrary. We need to show that \( f + g = g + f  \). Since \( f(t)  \) and \( g(t)  \) are real numbers, observe that 
            \[  (f+g)(t) = f(t) + g(t) = g(t) + f(t) = (g + f)(t).   \] Hence, we have that \(  f + g = g + f  \).
        \item[(VS 2)] Let \( f,g,h  \) be even functions and let \( t \in \R  \) be arbitrary. We need to show that \( f + (g+h) = (f+g) + h \). By the same reasoning we used to prove (VS 1), observe that
            \begin{align*}
               f(t) + ((g+h)(t)) &= f(t) + (g(t) + h(t)) \\
                                 &= (f(t) + g(t)) + h(t) \\
                                 &= ((f+g)(t)) + h(t) \\
            \end{align*}
            Hence, we have that  \( f + ( g+ h ) + (f+g) + h \).
        \item[(VS 3)] Let \( f  \) be an even function. Note that the zero function is an even function. Denote the zero function as \( f_{0}  \). For every \( t \in \R  \), we have \( f_{0}(x) = O \). We need to show that \(  f + f_{0} = f  \). Since \( f_{0}(x)  \) is just a real number, we have 
            \[  (f+ f_{0})(t) = f(t) + f_{0}(t) = f(t) + O = f(t). \]
        \item[(VS 4)] Let \( f  \) be an even function and let \( f_{0}  \) be the zero function defined above. We need to show that there exists a even function \( g  \) such that \( f + g = f_{0} \). Let \( t \in \R  \) be arbitrary. Note that 
            \[ (f+g)(t) = f(t) + g(t)  \]
        Since \( f(t)  \) and \( g(t)  \) are real numbers and there exists an additive identity in the real numbers, we have
        that \(  f(t)  + g(t) = f_{0}(t)  \). Hence, (VS 5) is satisfied.
        \item[(VS 5)] Let \( x \in \R  \) and \( f  \) an even function. Then we immediately have \( (1 \cdot f)(t) = 1 f(t) = f(t).  \) 
        \item[(VS 6)] Let \( a, b \in \R  \). We need to show that \( (ab) f = a (bf)  \). Hence, let \( t \in \R  \). Then we have 
            \begin{align*}
                (ab  f)(t)  &= a (bf)(t).  \\
            \end{align*}
        \item[(VS 7)] Let \( a \in \R  \) and let \( f,g   \) be even functions. Let \( t \in \R  \). We need to show that \( a (f+g) = af + ag \). Then we have 
            \begin{align*}
                a(f+g)(t) &= a [ f(t) + g(t) ] \\
                          &= a f(t) + a g(t).
            \end{align*}
            Hence, (VS 7) is satisfied.
        \item[(VS 8)] Now let \( a,b \in \R  \) and let \( f  \) be an even function. We need to show that \(  (a+b) f =  a f + b f  \). Let \( t \in \R  \) be arbitrary.
            Observe that 
            \begin{align*}
                (a+b)f(t) &= a f(t) + b f(t)   \\
                          &=  (af)(t) + (bf)(t) \\
            \end{align*}
            Hence, (VS 8) is satisfied.
    \end{enumerate}
\end{proof}

\subsubsection{Exercise 1.2.13} Let \( V  \) denote the set of ordered pairs of real numbers. If \( (a_{1}, a_{2}) \) and \( (b_{1}, b_{2}) \) are elements of \( V  \) and \( c \in \R  \), define 
\begin{center}
    \( (a_{1}, a_{2} ) + (b_{1} , b_{2} ) = (a_{1} + a_{2} , a_{2} b_{2} ) \) and \( c (a_{1} , a_{2} ) = (ca_{1} , a_{2})  \).
\end{center}
Is \( V  \) is a vector space over \( \R  \) with these operations? Justify your answer.
\begin{proof}[Solution]
We have that \( V  \) is not a vector space of \( \R  \). To see why,  let \( (2,1), (4,2) \in V  \) where \( x = (2,1)  \) and \( y = (4,2) \). We will show that (VS 1) does not hold; that is, \( x + y \neq y + x  \). Hence, observe that 
\begin{align*}
    (2,1) + (4,2) &= (2 + 1, 2 ) = (3,2) \\
\end{align*}
and 
\begin{align*}
    (4,2) + (2,1) &= (4+2 , 2) =  (6, 2) \\
\end{align*}
Hence, we have \( x + y \neq y  + x   \) and so \( V  \) is \textbf{NOT} a vector space.
\end{proof}

\subsubsection{Exercise 1.2.14} Let \( V = \{ (a_{1}, a_{2}, \dots, a_{n} )  : a_{i} \in C  \ \text{for} \  i = 1,2, \dots, n) \}  \); so \( V  \) is a vector space over \( C   \) by Example 1.  Is \( V  \) is a vector space over the field of real numbers with the operations of coordinate-wise addition and multiplication?

\begin{proof}
Yes, \( V  \) where 
\[  V = \{ (a_{1}, a_{2}, \dots , a_{n} ): a_{i} \in \R \ \text{for} \  i = 1,2, \dots, n  \}  \] is a vector space of \( \R  \).
\begin{enumerate}
    \item[(VS 1)] Let \( x,y \in V  \) such that \( x = (a_{1}, a_{2} , \dotsm,  a_{n}) \) and \( y = (b_1, b_2, \dots, b_n)  \). Since addition is entry-wise in \( V  \) and each entry in both \( x  \) and \( y  \) are elements of \( \R  \) (where \( \R  \)  is a field), we have \( a_{i} + b_{i} = b_{i} + a_{i}   \) for all \( i = 1,2, \dots, n  \). Hence, \( x + y = y + x  \).
    \item[(VS 2)] Let \( x,y,z \in V  \) with \( x \) and \( y  \) as defined as before where \( z  \) contains entries \( c_{i}  \) for all \( i = 1,2, \dots n  \). We can see that the entries of \( x,y,z \) are elements of \( \R  \) so associativity is preserved; that is, \( a_{i} + ( b_{i} + c_{i}  )  = ( a_{i}  + b_{i} ) + c_{i}   \)  for all \( i = 1,2, \dots , n \). Hence, we have \( x + (y + z) = (x+ y) + z   \).
    \item[(VS 3)] Since \( \R  \) contains the zero element \( 0  \) and \( V  \) is the set of \( n  \)-tuples, there exists an element denoted by \( O  \) such that this element consisting of entries that only have the zero element \( 0  \); that is, \( O = (0, 0 , \dots, 0 ) \). Take \( x \in V  \). Hence, we have \( a_{i} + 0 = a_{i}  \) for every \( i = 1,2, \dots, n  \). Thus, we must have \( x + O = x  \). 
    \item[(VS 4)] Let \( x \in V  \) be arbitrary as defined before. Since every entry in \( x  \) is an element of \( \R  \); that is, every \( a_{i} \in \R   \) for all \( i = 1,2, \dots, n  \), we know that every entry contains an element \( c_{i}  \) such that \( a_{i} + c_{i} = 0   \) for every \( i = 1,2,\dots, n  \). Denote \( x' = (c_{1}, c_{2}, \dots, c_{n})  \). Hence, we have \( x + x' = O  \). 
    \item[(VS 5)] Let \( x \in V  \). Every entry \( a_{i} \in \R   \) for all \( i = 1,2, \dots, n  \), we have \( 1 \cdot a_{i} = a_{i} \) which holds for all \( i  \). Denote this identity element as \( I  \) with entries  consisting only of \( 1 \). Hence, we have \( I \cdot x = x  \).
    \item[(VS 6)] Let \( e,r  \in \R  \) and let \( x \in V  \). For every entry \( a_{i} \in \R  \) for all \( i = 1,2, \dots, n  \), we have \( (er)a_{i} = e(r a_{i}) \) for all \( i = 1,2, \dots, n  \). By using the operations of scalar multiplication for \( n \)-tuples, this tells us that \( (er)x = e(rx)  \). Hence, (VS 6 ) is satisfied.
    \item[(VS 7)]  Let \( e \in \R  \) and let \( x,y \in V  \). We need to show that \( e(x+y) = ex + ey \). Note that \( a_{i} , b_{i} \in \R  \) implies that \( e(a_{i} + b_{i} ) = ea_{i} + eb_{i} \). Hence, we have \( e(x+y) = ex + ey \).
    \item[(VS 8)] Let \( e,r \in \R  \) and \( x \in V  \) as defined before. We need to show that \( (e+r)x = ex + rx  \). Since \( a_{i} \in \R   \) with \( e,r \in \R  \), we are guaranteed to have \( (e+r)a_{i} = e a_{i} + r a_{i}  \) for all \( i = 1,2, \dots, n  \). Hence, we have \( (e+r)x = e x + r x  \).
\end{enumerate}
\end{proof}

\subsubsection{Exercise 1.2.16} Let \( V  \) denote the set of all \( m \times n  \) matrices with real entries; so \( V  \) is a vector space over \( \R  \) by Example 2. Let \( F  \) be the field of rational numbers. Is \( V  \) a vector space over \( F  \) with the usual definitions of matrix addition and scalar multiplication?

\begin{proof}
\begin{enumerate}
    \item[(VS 1)] Let \( A, B \in V  \). Since \( A,B  \) consist of elements \( A_{ij}, B_{ij} \in \R  \), we know that \( A_{ij} + B_{ij} = B_{ij} + A_{ij}  \). Hence, \( A + B = B +  A \). 
    \item[(VS 2)] Let \( A,B, C \in V  \) with \( A,B  \) defined as before and \( C   \) containing real entries \( C_{ij}  \). With the same reasoning used to prove (VS 1), we know that \( A_{ij } + ( B_{ij } + C_{ij}  ) = (A_{ij} + B_{ij}) + C_{ij}   \). Hence, we have \( A + (B +C ) = (A+B) + C  \). 
    \item[(VS 3)] Let \( A \in V  \) once again. Since the entries of \( A  \) imply that there exists an element \( O  \) such that \( A_{ij} + 0 = A_{ij}    \), we know that \( A + O = A  \) where \( O  \) is the \textbf{zero matrix} of \( V  \).
    \item[(VS 4)] Since the real  entries of \( A  \) also consists of an element \( A'_{ij}  \) such that \( A_{ij} + A'_{ij} = 0  \), this implies that \( A + A' = O \) where \( A'  \) is the additive inverse matrix of \(  V  \).
    \item[(VS 5)] Let \( x \in V  \) as defined as before. Every entry of \( A  \), \( A_{ij} \in \R  \), has the following property: \( 1 \cdot A_{ij} = A_{ij}   \) for all  for all \(  1 \leq i \leq m   \) and  for all \(  1 \leq j \leq n   \) . The matrix whose entries consists of only one we can define as the \textbf{identity matrix} denoted by \(  I  \) where \( I_{ij} = 1  \) for all \(  1 \leq i \leq m   \) and \( 1 \leq j \leq n  \). Hence, we have \( A \cdot I = A  \).
    \item[(VS 6)] Let \( r,t \in \Q  \) and let \( A \in V  \) as defined before. We need to show that \( (rt) A = r(tA) \). Since \( A_{ij} \in \R  \), entry-wise scalar multiplication implies that \( (rt) A_{ij} = r(t A_{ij})  \) for all \( 1 \leq i \leq m  \) and \( 1 \leq j \leq n  \). Hence, we must have \( (rt) A = r (tA ) \).
    \item[(VS 7)] Let \( r \in \Q  \) and \( A, B \in V  \) as defined before. Since \( A_{ij}, B_{ij}  \in \R  \), we know that \( r(A_{ij} + B_{ij} ) = r A_{ij} + r B_{ij}  \). Hence, we have \( r(A + B ) = r A + r B  \).
    \item[(VS 8)] Let \( r,t \in \Q  \) and let \( A \in V  \) as defined before. Since \( A_{ij} \in \R   \) for all \( i,j \), we must have \( (r+t)A_{ij} = r A_{ij} + t A_{ij}    \). Hence, we have \( (r+t) A  = rA + tA  \).
\end{enumerate}
Hence, \( V  \) is a vector space over \( \Q  \).
\end{proof}

\subsubsection{Exercise 1.2.17}

Let \( V = \{ (a_{1}, a_{2}) : a_{1}, a_{2} \in F   \}  \), where \( F  \) is a field. Define addition of elements of \( V  \) coordinate-wise, and for \( c \in F  \) and \( (a_{1} , a_{2}) \in V   \), define 
\[  c(a_{1}, a_{2} ) = (ca_{1}, 0 ).\]
Is \( V  \) a vector space over \( F  \) with these operations? Justify your answer.

\begin{proof}
    We claim that \( V  \) is not a vector space over \( F  \) because \( V  \) fails to satisfy (VS 5). To see why, let \( (1,2) \in V  \). Using (VS 5), we have 
    \[  1 \cdot (1,2) = (1,0) \neq (1,2). \]
    Hence, \( V  \) cannot be a vector space.
\end{proof}

\subsubsection{Exercise 1.2.18} Let \( V = \{ (a_{1}, a_{2}  ) : a_{1}, a_{2} \in \R  \}   \). For \( (a_{1}, a_{2}), (b_{1} , b_{2} ) \in V  \) and \( c \in \R  \), define 
\begin{center}
    \( (a_{1}, a_{2} ) + (b_{1} , b_{2} ) = (a_{1} + 2 b_{1} , a_{2} + 3 b_{2} ) \) and \( c(a_{1}, a_{2} ) = (ca_{1}, ca_{2})  \). 
\end{center}
Is \( V  \) a vector space over \( F = \R   \) with these operations? Justify your answer?

\begin{proof}
We claim that \( V  \) is not a vector space over \( \R   \) and we will use (VS 1) to show this. Let \( x,y \in V  \) be defined by \( x = (1,2)  \) and \( y = (3,4)  \). Observe that 
\[  x + y = (1,2) + (3,4) = (7,14)  \]  and
\[  y + x = (3,4)  + (1,2) = (5,10). \]
Clearly, we have \(  x + y  = (7,14) \neq (5,10) = y + x  \) and so (VS 1) does not hold.
\end{proof}

\subsubsection{Exercise 1.2.19}


Let \( V = \{ (a_{1}, a_{2}) : a_{1}, a_{2} \in \R  \}   \). Define addition of elements of \( V  \) coordinate-wise, and for \( (a_{1}, a_{2}) \in V  \) and \( c \in \R  \), define 
\[  c(a_{1}, a_{2} ) = 
\begin{cases}
    (0,0)  &\text{if} \ c = 0 \\
    \Big( ca_{1}, \frac{ a_{2}  }{ c  }  \Big) &\text{if } c \neq 0 .
\end{cases} \]
Is \( V  \) a vector space over \( \R  \) with these operations? Justify your answer.
\begin{proof}
We claim that \( V   \) is not a vector space over \( \R  \). To see why, consider (VS 8). If we let \( (0,1) \in V  \) with \( c = 2 + 1 = 3    \). Observe that 
\[  (2+1) (0,1) = \Big(0, \frac{ 1 }{ 2 + 1  }  \Big) = \Big( 0, \frac{ 1 }{ 3 }  \Big).   \]
Likewise, we have 
\[  2(0,1) + 1(0,1) = \Big( 0, \frac{ 1 }{ 2 }  \Big) + \Big( 0, 1 \Big) = \Big( 0, \frac{ 3 }{ 2 }  \Big).  \]
Notice that \( (2+1) (0,1) \neq 2(0,1) + 1(0,1)  \). Hence, \( V  \) cannot be a vector space over \( \R  \).  
\end{proof} 

\subsubsection{Exercise 1.2.20} Let \( V  \) denote the set of all real-valued functions \( f  \) defined on the real line such that \( f(1) = 0  \). Prove that \( V  \) is a vector space with the operations of addition and scalar multiplication defined in Example 3.
\begin{proof} 
    Define \( V  \) as a vector space with the operations of addition and scalar multiplication defined in Example 3. We must show that \( V  \) is a vector space.
\begin{enumerate}
\item[(VS 1)] Let \( f,g \in V  \). This means that \( f(1) = 0  \) and \( g(1) = 0 \). We need to show that \( f+ g = g + f  \). Since \( f(1), g(1) \in \R  \) and commutativity holds in \( \R  \), we can write
    \begin{align*}
        (f+g)(1)   &= f(1) + g(1) \\
                   &= g(1) + f(1) \\
                   &= (g+f)(1) \\
    \end{align*}
    Hence, we have \( f+g = g+f \).
\end{enumerate}
\item[(VS 2)] Let \( f,g,h \in V  \) then \( f(1) = g(1) = h(1) = 0  \). We need to show that \( f + (g+h) = (f+g) + h  \). Observe that  
    \begin{align*}
       \Big( f + (g+h)  \Big)(1)  &=  f(1) + (g+h)(1) \\
                       &= f(1) + g(1) + h(1) \\
                       &= (f+g)(1) + h(1) \\
                       &= \Big( (f+g) + h \Big)(1).
    \end{align*}
    Hence, we have \( f + (g+h) = (f+g) + h \).
\item[(VS 3)]  Let \( f \in V  \). We need to show that \( f + f_{0} = f  \) for some \( f_{0} \in V  \). Since \( V  \) contains elements of \( f \in V  \) such that \( f(1) = 0  \), we can choose \( f_{0} \) such that \( f_{0}(1) = 0  \). We can show that this is indeed the additive inverse of \( V  \) by writing the following: 
    \[  (f + f_{0})(1) = f(1) + f_{0}(1) = f(1) + 0 = f(1).  \]
    Hence, we have \( f + f_{0} = f \).
\item[(VS 4)] Let \( f \in V  \). We need to find an element \( g \in V  \) such that \( f + g = f_{0} \) By definition of \( V  \), \( f(1) = 0  \).  We need to show that \( f+ g = f_{0} \) with \( f_{0}  \) defined as before. Choose \( g = -f  \) as our additive inverse and observe that  
    \begin{align*}
        (f+g)(1)  &= (f - f)(1)  \\
                  &= f(1) - f(1) \\
                  &= 0 - 0  \\
                  &= 0 \\
                  &= f_{0}(1).
    \end{align*}
    Hence, \( g = -f  \) an element such that \( f + g = f_{0} \).
\item[(VS 5)] Let \( f \in V  \). By definition of \( V  \), we have  \( f(1) = 0   \). Since \( f(1) \in \R  \), we know that \( 1 \cdot f(1) = f(1) \). We need to show that \( 1f = f  \). Observe that \[  (1f)(1) = 1 \cdot f(1) = f(1).  \]
    Hence, (VS 5) holds.
\item[(VS 6)] Let \( a,b \in \R  \) and let \( x \in V  \). We need to show that \( (ab)f = a(bf)  \). By using scalar multiplication, we can see that 
    \[  (ab \cdot f) (1) = (ab)f(1) = a (bf(1)) = a(b \cdot f)(1). \]
    Hence,  (VS 6) must hold.
\item[(VS 7)] Let \( a \in \R  \) and let \( f,g \in V  \). We need to show that \( a(f+g) = af + ag \). Observe that 
    \begin{align*}
        a(f+g)(1) &= a (f(1) + g(1))  \\
                  &= a f(1) + a g(1) \\
                  &=  (af)(1) + (ag)(1).
    \end{align*}
    Hence, \( a(f+g) = af + ag \) and so (VS 7) is satisfied.
\item[(VS 8)] Let \( a,b \in \R  \) and let \( f \in V  \). Observe that 
    \begin{align*}
        (a+b)f(1) &= a f(1) + b f(1)  \\
                  &=  (af)(1) + (bf)(1) \\
    \end{align*}
    Hence, \( (a+b)f =  af + bf  \). 
Thus, \( V  \) must be a vector space over \( \R  \).
\end{proof}

\subsubsection{Exercise 1.2.21} 
Let \( V  \) and \( W  \) be vector spaces over a field \( F  \). Let 
\[  Z = \{ (v,w): v \in v \ \text{and} \  w \in W  \}. \]
Prove that \( Z  \) is a vector space over \( F  \) with the operations   
\begin{center}
   \( (v_{1}, w_{1}) + (v_{2}, w_{2}) = (v_{1} + v_{2} , w_{1} + w_{2} ) \) and \( c(v_{1}, w_{1} ) = (cv_{1}, cw_{1} ) \). 
\end{center}
\begin{proof}

\end{proof}

hello


