\section{Subspaces}

\subsubsection{Exercise 1.3.3}

Prove that \( (aA + bB)^{t} = a A^{t} + b B^{t}  \) for any \( A,B \in M_{m \times n } (F) \) and any \( a,b \in F  \).

\begin{proof}
Let \( A , B \in M_{m \times n } (F ) \) and let \( a, b \in F  \) be arbitrary. Using scalar multiplication defined on \( M_{m \times n }(F ) \), we have 
\begin{align*}
    (aA + bB )^{t} &= (aA)^{t} + (bB)^{t}  \\
                   &= a A^{t} + b B^{t}.
\end{align*}
Hence, we are done.
\end{proof}

\subsubsection{Exercise 1.3.5} 
Prove that \( A + A^{t}  \) is symmetric for any square matrix \( A  \).

\begin{proof}
Let \( A  \) be an arbitrary square matrix. Since square matrices are symmetric, we have that \( A^{t} = A  \). We need to show that \( (A + A^{t})^{t}  \). Observe that 
\begin{align*}
    (A + A^{t})^{t} &= A^{t} + (A^{t})^{t} \\
                    &= A + A^{t}.
\end{align*}
Hence, we have \( A + A^{t}  \) is symmetric.
\end{proof}

\subsubsection{Exercise 1.3.}

Prove that \( \text{tr}(aA + bB) = a \text{tr}(A) + b \text{tr}(B) \) for any \( A,B \in M_{n \times n }(F) \).

\begin{proof}
Let \( A , B \in M_{n \times n }(F) \) and let \( a,b \in F  \) be arbitrary.  Now, let \( i = j  \) and observe that 
\begin{align*}
    \text{tr}(aA + bB) &= \sum_{ i,j \in \N  }^{ n  } (aA + aB )_{ij} \\
                       &= \sum_{ i,j \in \N  }^{ n  } (aA)_{ij} + (bB)_{ij} \\
                       &= \sum_{ i,j \in \N  }^{ n  } (aA)_{ij} + \sum_{ i,j \in \N  }^{ n  } (bB)_{ij} \\
                       &= \sum_{ i,j \in \N  }^{ n  } a A_{ij} + \sum_{ i,j \in \N  }^{ n  } b B_{ij} \\
                       &= a \sum_{ i,j \in \N  }^{ n  } A_{ij} + b \sum_{ i,j \in \N  }^{ n } B_{ij} \\
                       &= a \text{tr}(A) + b \text{tr}(B).
\end{align*}
Hence, we conclude 
\[  \text{tr}(aA + bB ) = a \text{tr}(A) + b \text{tr}(B) \]
for any \( A, B \in M_{n \times n }(F ) \).
\end{proof}

\subsubsection{Exercise 1.3.7} 
Prove that diagonal matrices are symmetric matrices.

\begin{proof}
Let \( A \in M_{n \times n }(F) \) be diagonal. Let \( i \neq  j  \) where \( A_{ij} = 0  \). We need to show that \( A^{t} = A  \). If we apply a transpose on \( A  \), we get that \( A_{ji} = 0  \) since \( A^{t}   \) is also diagonal and square. Since \( A_{ij} = A_{ji} = 0  \) for all \( 1 \leq i \leq n  \) and \( 1 \leq j \leq n  \). Hence,  \( A^{t} = A \)
\end{proof}


\subsubsection{Exercise 1.3.8} Determine whether the following sets are subspaces of \( \R^{3}  \) under the operations of addition and scalar multiplication defined on \( \R^{3}  \). Justify your answers. 
\begin{enumerate}
    \item[(a)] \( W_{1} = \{ (a_{1}, a_{2}, a_{3}) \in \R^{3} : a_{1} = 3a_{2} \ \text{and} \ a_{3} =  -a_{2} \}  \)
        \begin{proof}
        We claim that \( W_{1}  \) is a subspace of \( \R^{3} \). 
        \begin{enumerate}
            \item[(a)] Note that \( O_{\R^{3}} \in W_{1}  \) where \( O_{\R^{3} } = (0,0,0) \) because \( 0 = 3 \cdot 0  \) and \( 0 = -1 \cdot 0  \).
            \item[(b)] Let \( x, y \in W_{1}  \) where \( x = (a_{1}, a_{2}, a_{3}) \) and \( y = (b_{1}, b_{2}, b_{3}) \). We need to show that \( x + y \in W_{1} \).  Since \( a_{1} = 3 a_{2}  \) and \( a_{3} = - a_{2} \) as well as \( b_{1} = 3 b_{2}  \) and \( b_{3} = - b_{2} \), we can write \( a_{1} + b_{1} = 3 (a_{2} + b_{2} )  \) and \( a_{3} + b_{3} = - (a_{2} + b_{2}) \). Hence, \( x + y \in W_{1} \).
            \item[(c)] Let \( c \in \R  \) and \( x \in W_{1}  \) with \( x  \) defined as before. Then observe that \( ca_{1} = c (3 a_{2}) = 3 (ca_{2}) \) and \( ca_{3} = c (-a_{2} ) = -(ca_{2}) \). Hence, \( cx \in W_{1} \).
        \end{enumerate}
        Since all the properties of a Theorem 3 have been satisfied, we can conclude that 
        \( W_{1} \) is a subspace of \( \R^{3} \).
         \end{proof}
    \item[(b)] \( W_{2} = \{ (a_{1}, a_{2}, a_{3}) \in \R^{3} : a_{1} = a_{3} + 2  \}  \)
    \item[(c)] \( W_{3} = \{ (a_{1}, a_{2}, a_{3}) \in \R^{3} : 2a_{1} - 7a_{2} + a_{3} = 0  \}  \)
    \item[(d)] \( W_{4} = \{ (a_{1}, a_{2}, a_{3}) \in \R^{3} : a_{1} - 4a_{2} - 3a_{3} = 1  \}  \)
    \item[(e)] \( W_{5} = \{ (a_{1}, a_{2} ,a_{3}) \in \R^{3} : a_{1} + 2a_{2} -  3a_{3} = 1   \}  \)
    \item[(f)] \( W_{6} = \{ (a_{1} , a_{2} ,a_{3}) \in \R^{3} : 5a_{1}^{2} - 3a_{2}^{2} + 6a_{3}^{2} = 0  \}  \)
\end{enumerate}

\subsubsection{Exercise 1.3.9} Let \( W_{1} , W_{3} , W_{4}  \) be as in Exercise 8. Describe \( W_{1} \cap W_{3}  \), \( W_{1} \cap W_{4}  \), and \( W_{3} \cap W_{4}  \) and observe that each is a subspace of \( \R^{3} \).

\begin{proof}

\end{proof}


\subsubsection{Exercise 1.3.11} Prove that the set \( W_{1} = \{ (a_{1}, a_{2} , \dots , a_{n}) \in F^{n} : a_{1} + a_{2} + \dots + a_{n} = 0  \}   \) is a subspace of \( F^{n}  \), but \( W_{2} = \{ (a_{1}, a_{2}, \dots , a_{n} )  \} \in F^{n} : a_{1} + a_{2} + \dots + a_{n} = 1   \) is not.
\begin{proof}
We need to show that \( W_{1}  \) is a subspace of \( F^{n}  \). We proceed by satisfying the properties of Theorem 3 to do this. 
\begin{enumerate}
    \item[(a)] Note that \( O_{F^{n}} \in  W_{1}    \) since \( 0 + 0 + \dots + 0 = 0  \) \( n  \) times. 
    \item[(b)] Let \( x,y \in W_{1}  \) with \( x = (a_{1}, a_{2} , \dots, a_{3} )  \) and  \( y = (b_{1}, b_{2}, \dots, b_{n})   \). By definition of \( W_{1}  \), we can see that 
        \begin{align*}
            \sum_{ i=1 }^{ n } (a_{i} + b_{i}  ) &= \sum_{ i=1 }^{ n } a_{i} + \sum_{ i=1 }^{ n } b_{i}  \\
                                                 &= 0 + 0 \\
                                                 &= 0.
        \end{align*}
        Hence, \( x + y \in W_{1} \) which tells us that \( W_{1}  \) is closed under addition.
    \item[(c)] Let \( x \in W_{1}  \) and \( c \in F  \). Then observe that 
        \[ \sum_{ i=1  }^{ n } (ca_{i}) = c \sum_{ i=1 }^{ n } a_{i} = c \cdot 0 = 0.  \]
        Hence, we have \( cx \in F^{n} \).

We claim that \( W_{2}  \) is not a subspace because \( W_{2}  \) is not closed under addition. Let \( (0,1) , (1,0) \in F^{2}  \). Observe that \( 0 + 1 = 1  \) and \( 1 + 0 = 1  \), but \(  (0+1) + (1+0) =  1 + 1 = 2   \). Hence, \( (0,1) + (1,0) \notin W_{2}  \). 
\end{enumerate}

\end{proof}


\subsubsection{Exercise 1.3.11} Is the set \( W = \{ f(x) \in P(F) : f(x) = 0 \ \text{or} \  f(x) \ \text{has degree} \ n \}  \) a subspace of \( P(F)  \) if \( n \geq 1  \)? Justify your answer.


\subsubsection{Exercise 1.3.12} Prove that the set of \( m \times n  \) upper triangular matrices is a subspace of \( M_{m \times n}(F) \).
\begin{proof}
    Let \( V  \) denote the set of \( m \times n  \) upper triangular matrices. We will show that \( V  \) is a subspace of \( M_{m \times n}(F)  \) using Theorem 3. 
    \begin{enumerate}
        \item[(a)] The zero matrix \( O \) from \( M_{m \times n}(F) \) contains entries \( O_{ij} = 0  \) whenever \( i > j  \). Hence, \( O \in V \).
        \item[(b)] Let \( A, B \in V  \). By definition of \( V  \), \( A  \) and \( B \) are upper triangular where \(  A_{ij} = 0  \) and \( B_{ij} = 0  \) whenever \( i > j   \). Observe that 
            \[ (A + B)_{ij} = A_{ij} + B_{ij} = 0 + 0 = 0  \]
            whenever \( i > j  \).
        Hence, \( A + B \in V  \).
    \item[(c)] Let \( c \in F  \) and \( A \in V  \) as defined before. Let \( i > j  \) and observe that \( (cA)_{ij} = c A_{ij} = c \cdot 0 = 0   \). Hence, \( cA \in V  \). 
    \end{enumerate}
    Since all the properties of theorem 3 have been satisfied, we conclude that \( V  \) is indeed a subspace of \( M_{m \times n}(F) \).
\end{proof}


\subsubsection{Exercise 1.3.13} Let \( S  \) be nonempty set and \( F  \) is a field. Prove that for any \( s_{0} \in S  \), the set \( \{ f \in \mathcal{F}(S,F ) : f(s_{0}) = 0  \}  \), is a subspace of \( \mathcal{F}(S,F) \).
\begin{proof}
    We will proceed to prove that \( S  \) is a subspace of \( \mathcal{F}(S,F)  \) over the field \( F   \) by satisfying the properties of Theorem 3. Let \( V = \{  f \in \mathcal{F}(S,F) : f(s_{0}) = 0  \}   \). 
    \begin{enumerate}
        \item[(a)] Note that the zero function \( f_{0} \in \mathcal{F}(S,F)  \) where \( f_{0}(s_{0}) = 0  \) for any \( s_{0} \in S  \) implies that \( f_{0} \in  V  \).
        \item[(b)] Let \( f,g \in V  \). By definition of \( V  \), \( f(s_{0}) = 0  \) and \( g(s_{0} ) = 0  \) for any \( s_{0} \in S \). We have \(  f + g \in V  \) since
            \[  (f+g)(s_{0}) = f(s_{0}) + g(s_{0}) = 0 + 0 = 0. \]
        Hence, \( V  \) is closed under addition.
        \item[(c)] Let \( f \in V  \) and \( c \in F  \). We have \( cf \in V  \) since 
            \[  (cf)(s_{0} = c f(s_{0})) = c \times 0 = 0  \] for any \( s_{0} \in S  \). Hence, \( V  \) is closed under scalar multiplication.
    \end{enumerate} 
\end{proof}

