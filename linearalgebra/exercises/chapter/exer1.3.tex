\section{Subspaces}

\subsubsection{Exercise 1.3.3}

Prove that \( (aA + bB)^{t} = a A^{t} + b B^{t}  \) for any \( A,B \in M_{m \times n } (F) \) and any \( a,b \in F  \).

\begin{proof}
Let \( A , B \in M_{m \times n } (F ) \) and let \( a, b \in F  \) be arbitrary. Using scalar multiplication defined on \( M_{m \times n }(F ) \), we have 
\begin{align*}
    (aA + bB )^{t} &= (aA)^{t} + (bB)^{t}  \\
                   &= a A^{t} + b B^{t}.
\end{align*}
Hence, we are done.
\end{proof}

\subsubsection{Exercise 1.3.5} 
Prove that \( A + A^{t}  \) is symmetric for any square matrix \( A  \).

\begin{proof}
Let \( A  \) be an arbitrary square matrix. Since square matrices are symmetric, we have that \( A^{t} = A  \). We need to show that \( (A + A^{t})^{t}  \). Observe that 
\begin{align*}
    (A + A^{t})^{t} &= A^{t} + (A^{t})^{t} \\
                    &= A + A^{t}.
\end{align*}
Hence, we have \( A + A^{t}  \) is symmetric.
\end{proof}

\subsubsection{Exercise 1.3.}

Prove that \( \text{tr}(aA + bB) = a \text{tr}(A) + b \text{tr}(B) \) for any \( A,B \in M_{n \times n }(F) \).

\begin{proof}
Let \( A , B \in M_{n \times n }(F) \) and let \( a,b \in F  \) be arbitrary.  Now, let \( i = j  \) and observe that 
\begin{align*}
    \text{tr}(aA + bB) &= \sum_{ i,j \in \N  }^{ n  } (aA + aB )_{ij} \\
                       &= \sum_{ i,j \in \N  }^{ n  } (aA)_{ij} + (bB)_{ij} \\
                       &= \sum_{ i,j \in \N  }^{ n  } (aA)_{ij} + \sum_{ i,j \in \N  }^{ n  } (bB)_{ij} \\
                       &= \sum_{ i,j \in \N  }^{ n  } a A_{ij} + \sum_{ i,j \in \N  }^{ n  } b B_{ij} \\
                       &= a \sum_{ i,j \in \N  }^{ n  } A_{ij} + b \sum_{ i,j \in \N  }^{ n } B_{ij} \\
                       &= a \text{tr}(A) + b \text{tr}(B).
\end{align*}
Hence, we conclude 
\[  \text{tr}(aA + bB ) = a \text{tr}(A) + b \text{tr}(B) \]
for any \( A, B \in M_{n \times n }(F ) \).
\end{proof}

\subsubsection{Exercise 1.3.7} 
Prove that diagonal matrices are symmetric matrices.

\begin{proof}
Let \( A \in M_{n \times n }(F) \) be diagonal. Let \( i \neq  j  \) where \( A_{ij} = 0  \). We need to show that \( A^{t} = A  \). If we apply a transpose on \( A  \), we get that \( A_{ji} = 0  \) since \( A^{t}   \) is also diagonal and square. Since \( A_{ij} = A_{ji} = 0  \) for all \( 1 \leq i \leq n  \) and \( 1 \leq j \leq n  \). Hence,  \( A^{t} = A \)
\end{proof}


\subsubsection{Exercise 1.3.8} Determine whether the following sets are subspaces of \( \R^{3}  \) under the operations of addition and scalar multiplication defined on \( \R^{3}  \). Justify your answers. 
\begin{enumerate}
    \item[(a)] \( W_{1} = \{ (a_{1}, a_{2}, a_{3}) \in \R^{3} : a_{1} = 3a_{2} \ \text{and} \ a_{3} =  -a_{2} \}  \)
    \item[(b)] \( W_{2} = \{ (a_{1}, a_{2}, a_{3}) \in \R^{3} : a_{1} = a_{3} + 2  \}  \)
    \item[(c)] \( W_{3} = \{ (a_{1}, a_{2}, a_{3}) \in \R^{3} : 2a_{1} - 7a_{2} + a_{3} = 0  \}  \)
    \item[(d)] \( W_{4} = \{ (a_{1}, a_{2}, a_{3}) \in \R^{3} : a_{1} - 4a_{2} - 3a_{3} = 1  \}  \)
    \item[(e)] \( W_{5} = \{ (a_{1}, a_{2} ,a_{3}) \in \R^{3} : a_{1} + 2a_{2} -  3a_{3} = 1   \}  \)
    \item[(f)] \( W_{6} = \{ (a_{1} , a_{2} ,a_{3}) \in \R^{3} : 5a_{1}^{2} - 3a_{2}^{2} + 6a_{3}^{2} = 0  \}  \)
\end{enumerate}

\subsubsection{Exercise 1.3.9} Let \( W_{1} , W_{3} , W_{4}  \) be as in Exercise 8. Describe \( W_{1} \cap W_{3}  \), \( W_{1} \cap W_{4}  \), and \( W_{3} \cap W_{4}  \) and observe that each is a subspace of \( \R^{3} \).










