\section{Diagonalization}

\subsection*{Exercise 5.2.5} State and prove the matrix version of Theorem 5.6.
\begin{proof}

\end{proof}

\subsection*{Exercise 5.2.8} Suppose that \( A \in {M}_{n \times n }(F) \) has two distinct eigenvalues \( {\lambda}_{1} \) and \( {\lambda}_{2} \) and that \( \text{dim}({E}_{{\lambda}_{1}}) = n - 1  \). Prove that \( A  \) is diagonalizable.
\begin{proof}

\end{proof}

\subsection*{Exercise 5.2.9} Let \( T \) be a linear operator on a finite-dimensional vector space \( V  \), and suppose there exists an ordered basis for \( V  \) such that \( [T]_{\beta} \) is an upper triangular matrix.  
\begin{enumerate}
    \item[(a)] Prove that the characteristic polynomial for \( T  \) splits.
        \begin{proof}
        
        \end{proof}
    \item[(b)] State and prove an analogous result for matrices.
        \begin{proof}
        
        \end{proof}
\end{enumerate}

\subsection*{Exercise 5.2.10} Let \( T  \) be a linear operator on a finite-dimensional vector space \( V  \) with the distinct eigenvalues \( {\lambda}_{1}, {\lambda}_{2}, \dots, {\lambda}_{k} \) and corresponding multiplicities \( {m}_{1}, {m}_{2}, \dots, {m}_{k}  \). Suppose that \( \beta \) is a basis for \( V  \) such that \( [T]_{\beta} \) is an upper triangular matrix. Prove that the diagonal entries of \( [T]_{\beta} \) are \( {\lambda}_{1}, {\lambda}_{2}, \dots, {\lambda}_{k} \) and that each \( {\lambda}_{i} \) occurs \( {m}_{i} \) times (\( 1 \leq i \leq k  \)).
\begin{proof}

\end{proof}

\subsection*{Exercise 5.2.11} Let \( A  \) be an \( n \times n  \) matrix that is similar to an upper triangular matrix and has the distinct eigenvalues \( {\lambda}_{1}, {\lambda}_{2}, \dots, {\lambda}_{k} \) with corresponding multiplicities \( {m}_{1}, {m}_{2}, \dots, {m}_{k } \). Prove the following statements. 
\begin{enumerate}
    \item[(a)] \( \text{tr}(A) = \sum_{ i=1  }^{ k  } {m}_{i} {\lambda}_{i} \).
        \begin{proof}
        
        \end{proof}
    \item[(b)] \( \text{det}(A) = ({\lambda}_{1})^{{m}_{1}} ({\lambda}_{2})^{{m}_{2}} \cdots ({\lambda}_{k})^{{m}_{k}} \).
        \begin{proof}
        
        \end{proof}
\end{enumerate}

\subsection*{Exercise 5.2.12} Let \( T  \) be an invertible linear operator on a finite-dimensional vector space \( V  \). 
\begin{enumerate}
    \item[(a)] Recall that for any eigenvalue of \( T  \), \( \lambda^{-1} \) is an eigenvalue of \( T^{-1} \) (Exercise 8 of Section 5.1). Prove that the eigenspace of \( T  \) corresponding to \( \lambda  \) is the same as the eigenspace of \( T^{-1} \) corresponding to \( \lambda^{-1} \).
        \begin{proof}
        
        \end{proof}
    \item[(b)] Prove that if \( T  \) is diagonalizable, then \( T^{-1} \) is diagonalizable.
        \begin{proof}
        
        \end{proof}
\end{enumerate}

\subsection*{Exercise 5.2.13} Let \( A \in {M}_{n \times n}(F)  \). Recall from Exercise 14 of Section 5.1 that \( A  \) and \( A^{t} \) have the same characteristic polynomial and hence share the same eigenvalues with the same multiplicities. For any eigenvalue \( \lambda  \) of \( A  \) and \( A^{t} \), let \( {E}_{\lambda} \) and \( {E}_{\lambda}' \) denote the corresponding eigenspaces for \( A  \) and \( A^{t} \), respectively.
\begin{enumerate}
    \item[(a)] Show by way of example that for a given common eigenvalue, these two eigenspaces need not be the same.
        \begin{solution}
        
        \end{solution}
    \item[(b)] Prove that for any eigenvalue \( \lambda  \), \( \text{dim}({E}_{\lambda}) = \text{dim}({E}_{\lambda}') \).
        \begin{proof}
        
        \end{proof}
    \item[(c)] Prove that if \( A  \) is diagonalizable, then \( A^{t}  \) is diagonalizable.
        \begin{proof}
        
        \end{proof}
\end{enumerate}
