\section{Diagonalization}

\subsection*{Exercise 5.2.5} State and prove the matrix version of Theorem 5.6.
\begin{proof}
Let \( A \in {M}_{n \times n}(F) \) and let \( f(t) = \text{det}(A - tI) \) is the characteristic polynomial of \( A \). Suppose \( {L}_{A} \) is diagonalizable. The proof is the same except \( T  \) is replaced with \( {L}_{A} \) instead. Thus, the characteristic polynomial \( f(t)  \) of \( A  \) splits over \( F  \).
\end{proof}

\subsection*{Exercise 5.2.8} Suppose that \( A \in {M}_{n \times n }(F) \) has two distinct eigenvalues \( {\lambda}_{1} \) and \( {\lambda}_{2} \) and that \( \text{dim}({E}_{{\lambda}_{1}}) = n - 1  \). Prove that \( A  \) is diagonalizable.
\begin{proof}
    Let \( A \in {M}_{n \times n}(F) \) and \( {\lambda}_{1} \) and \( {\lambda}_{2} \) be two distinct eigenvalues where \( {E}_{{\lambda}_{1}} \) and \( {E}_{{\lambda}_{2}} \) are the eigenspaces associated with these eigenvalues (related to the linear operator \( {L}_{A} \)). Let \( \beta_{1} \) be a basis for \( {E}_{\lambda_{1}} \). Since \( \text{dim}({E}_{\lambda_{1}}) = n - 1  \), we have
    \[  {\beta}_{1} = \{ {v}_{1}, {v}_{2}, \dots, {v}_{n-1} \}.   \]
    Note that the dimensions of \( {E}_{{\lambda}_{1}} \) and \( {E}_{{\lambda}_{2}}  \) must sum to \( n \). Thus, we must have \( \text{dim}({E}_{{\lambda}_{2}}) = 1  \) and let \( {\beta}_{2} = \{ v'  \}  \) be the basis for \( {E}_{{\lambda}_{2}} \) where \( v' \neq 0  \). So, extend \( {\beta}_{1} \) to a basis \( \beta  \) for \( V  \) taking the union of \(  {\beta}_{2}   \) with this basis. Since \( \beta = {\beta}_{1} \cup {\beta}_{2} \) is a set that contains exactly \( n  \) linearly independent vectors that are eigenvectors of \( {L}_{A} \) corresponding to eigenvalues \( {\lambda}_{1} \) and \( {\lambda}_{2} \), we have that \( {L}_{A} \) must be diagonalizable by part (b) of Theorem 5.9. Thus, \( A  \) must be diagonalizable.
\end{proof}

\subsection*{Exercise 5.2.9} Let \( T \) be a linear operator on a finite-dimensional vector space \( V  \), and suppose there exists an ordered basis for \( V  \) such that \( [T]_{\beta} \) is an upper triangular matrix.  
\begin{enumerate}
    \item[(a)] Prove that the characteristic polynomial for \( T  \) splits.
        \begin{proof}
        Let 
        \[  f(t) = \text{det}([T]_{\beta} - tI) \]
        be the characteristic polynomial of \( [T]_{\beta}  \). By Exercise 23 of Section 4.2, we can see that the determinant of this upper triangular matrix is just the product of its diagonal entries. Thus, we have 
        \[  f(t) = (-1)^{n} ({[T]_{\beta}}_{11} - t)({[T]_{\beta}}_{22} - t) \cdots ({[T]_{\beta}}_{nn}  - t). \]
        Hence, \( f(t)  \) splits.
        \end{proof}
    \item[(b)] State and prove an analogous result for matrices.
        \begin{proof}
        Substitute \( {L}_{A} \) in place of \( T  \) and apply part (a).
        \end{proof}
\end{enumerate}

\subsection*{Exercise 5.2.10} Let \( T  \) be a linear operator on a finite-dimensional vector space \( V  \) with the distinct eigenvalues \( {\lambda}_{1}, {\lambda}_{2}, \dots, {\lambda}_{k} \) and corresponding multiplicities \( {m}_{1}, {m}_{2}, \dots, {m}_{k}  \). Suppose that \( \beta \) is a basis for \( V  \) such that \( [T]_{\beta} \) is an upper triangular matrix. Prove that the diagonal entries of \( [T]_{\beta} \) are \( {\lambda}_{1}, {\lambda}_{2}, \dots, {\lambda}_{k} \) and that each \( {\lambda}_{i} \) occurs \( {m}_{i} \) times (\( 1 \leq i \leq k  \)).
\begin{proof}
Let \( [T]_{\beta} = A  \) and \(  \text{dim}(V) = n \). Since \( A  \) is an upper triangular matrix, we know that the characteristic polynomial of \( A  \) splits by Exercise 5.2.10. Thus, we have
\begin{align*}
    f(t) &= \text{det}(A - tI) \\
         &= (-1)^{n} ({A}_{11} - t)({A}_{22} - t) \cdots ({A}_{nn} -t).
\end{align*}
By Theorem 5.3, \( f(t) \) contains at most \( n   \) distinct eigenvalues. So, we must have \( f(t) = 0  \) if and only if \( \lambda = {A}_{11}, {A}_{22}, \dots, {A}_{nn} \). But this means that each \( {A}_{ii} \) for \( 1 \leq i \leq n  \) is an eigenvalue of \( A  \) by Theorem 5.2. Thus, the eigenvalues of \(T\) must occur on the diagonals of \( A  \). But we only have distinct eigenvalues \( {\lambda}_{1}, {\lambda}_{2}, \dots, {\lambda}_{k }  \) of \( T  \) with corresponding multiplicities \( {m}_{1}, {m}_{2}, \dots, {m}_{k } \) where \( k \leq n  \). So, each \( {m}_{i} \) corresponding to each \( {\lambda}_{i} \) for \( 1 \leq i \leq k  \) must sum up to \( n  \) since \( f(t) \) is an \( n \)th degree polynomial and there must be \( n  \) solutions to \( f(t) = 0  \). Thus, each \( {\lambda}_{i} \) must occur \( {m}_{i} \) times along the diagonal of \( A  \). 
\end{proof}

\subsection*{Exercise 5.2.11} Let \( A  \) be an \( n \times n  \) matrix that is similar to an upper triangular matrix and has the distinct eigenvalues \( {\lambda}_{1}, {\lambda}_{2}, \dots, {\lambda}_{k} \) with corresponding multiplicities \( {m}_{1}, {m}_{2}, \dots, {m}_{k } \). Prove the following statements. 
\begin{enumerate}
    \item[(a)] \( \text{tr}(A) = \sum_{ i=1  }^{ k  } {m}_{i} {\lambda}_{i} \).
        \begin{proof}
            Let \( A \in {M}_{n \times n}(F) \). Suppose \( A  \) is similar to some upper triangular matrix \( B  \). Let \( {\lambda}_{1}, {\lambda}_{2}, \dots, {\lambda}_{k } \) be the eigenvalues with corresponding multiplicities \( {m}_{1}, {m}_{2}, \dots, {m}_{k } \) of \( B  \). Since these eigenvalues must occur on the diagonal of \( A  \) (by Exercise 5.2.10) where each \( {\lambda}_{i} \) occurs \( {m}_{i}  \) times along with \( \text{tr}(A) = \text{tr}(B) \) (since \( A  \) is similar to \( B \)), we can take the sum of our diagonal entries to get
            \[  \text{tr}(A) = \text{tr}(B) =  \sum_{ i=1 }^{ k   } {m}_{i} {\lambda}_{i}. \]
        \end{proof}
    \item[(b)] \( \text{det}(A) = ({\lambda}_{1})^{{m}_{1}} ({\lambda}_{2})^{{m}_{2}} \cdots ({\lambda}_{k})^{{m}_{k}} \).
        \begin{proof}
        Since \( A  \) is similar to \( B  \), we know that \( \text{det}(A) = \text{det}(B) \) by {\hyperref[Exercise 2.5.10]{Exercise 2.5.10}}. Since \( B  \) is upper triangular, we know that by {\hyperref[Exercise 4.2.23]{Exercise 4.2.23}} that the determinant of \( B  \) is just the product of its diagonal entries. Since the diagonal entries of \( B  \) consists of eigenvalues \( {\lambda}_{1}, {\lambda}_{2}, \dots, {\lambda}_{k} \) corresponding to multiplicities \( {m}_{1}, {m}_{2}, \dots, {m}_{k} \), we have  
        \[  \text{det}(A) = \text{det}(B) = \prod_{i=1}^{k} ({\lambda}_{i})^{{m}_{i}}  \]
        which is our desired result.
        \end{proof}
\end{enumerate}

\subsection*{Exercise 5.2.12} Let \( T  \) be an invertible linear operator on a finite-dimensional vector space \( V  \). 
\begin{enumerate}
    \item[(a)] Recall that for any eigenvalue of \( T  \), \( \lambda^{-1} \) is an eigenvalue of \( T^{-1} \) (Exercise 8 of Section 5.1). Prove that the eigenspace of \( T  \) corresponding to \( \lambda  \) is the same as the eigenspace of \( T^{-1} \) corresponding to \( \lambda^{-1} \). % Show that \( dim(E_{\lambda}) = dim(E_{\lambda^{-1}})  \)
        \begin{proof}
            Let \( \lambda  \) be an eigenvalue of \( T  \). Denote the eigenspaces \( {E}_{\lambda} \) and \( {E}_{\lambda^{-1}} \) of \( T  \) and \( T^{-1} \) respectively. We need to show that \( {E}_{\lambda} = {E}_{\lambda^{-1}} \). Note that \( {E}_{\lambda} = N(T - \lambda I ) \) by definition.  Let \( x \in {E}_{\lambda}  \). Then this is true if and only if \( x \neq 0    \) is an eigenvector of \( T  \) corresponding to \( \lambda \) by Theorem 5.4. By {\hyperref[Exercise 5.1.8]{Exercise 5.1.8}}, this is true if and only if \( x  \) is an eigenvector of \( T^{-1} \) corresponding to \( \lambda^{-1} \). By Theorem 5.4 again, this is true if and only if \( x \in N(T^{-1} - \lambda^{-1}I) \). So, \( x \in {E}_{\lambda^{-1}} \) and thus \( {E}_{\lambda} \subseteq {E}_{\lambda^{-1}} \). We can reverse this argument to get the other containment. Thus, \( {E}_{\lambda^{-1}} \subseteq {E}_{\lambda} \) and so we conclude that \( {E}_{\lambda} = {E}_{\lambda^{-1}} \).
        \end{proof}
    \item[(b)] Prove that if \( T  \) is diagonalizable, then \( T^{-1} \) is diagonalizable.
        \begin{proof}
        Let \( {\lambda}_{1}, {\lambda}_{2}, \dots, {\lambda}_{k} \) be the eigenvalues of \( T  \) corresponding to multiplicities \( {\lambda}_{1}, {\lambda}_{2}, \dots, {\lambda}_{k} \). Suppose that \( T  \) is diagonalizable. By Theorem 5.9, the multiplicity of each \( {\lambda}_{i} \) is equal to \( \text{dim}({E}_{{\lambda}_{i}}) \) for all \(1 \leq  i  \leq k \). Using part (a), we obtain
        \[  {m}_{i} = \text{dim}({E}_{\lambda_i}) = \text{dim}({E}_{\lambda^{-1}_{i}}) \ \text{for all} \ 1 \leq i \leq k.   \]
        Since each \( {\lambda}_{i}^{-1} \) is an eigenvalue of \( T^{-1} \) and \( \text{dim}({E}_{{\lambda}_{i}^{-1}}) = {m}_{i} \), \( T^{-1} \) must be diagonalizable by part (a) of Theorem 5.9.
        \end{proof}
\end{enumerate}

\subsection*{Exercise 5.2.13} Let \( A \in {M}_{n \times n}(F)  \). Recall from Exercise 14 of Section 5.1 that \( A  \) and \( A^{t} \) have the same characteristic polynomial and hence share the same eigenvalues with the same multiplicities. For any eigenvalue \( \lambda  \) of \( A  \) and \( A^{t} \), let \( {E}_{\lambda} \) and \( {E}_{\lambda}' \) denote the corresponding eigenspaces for \( A  \) and \( A^{t} \), respectively.
\begin{enumerate}
    \item[(a)] Show by way of example that for a given common eigenvalue, these two eigenspaces need not be the same.
        \begin{solution}
        
        \end{solution}
    \item[(b)] Prove that for any eigenvalue \( \lambda  \), \( \text{dim}({E}_{\lambda}) = \text{dim}({E}_{\lambda}') \).
        \begin{proof}
        Let \( A \in {M}_{n \times n}(F) \) and denote \( {E}_{\lambda} = N({L}_{A} - \lambda I) \). Let \( x \in {E}_{\lambda}  \). By Theorem 5.9, \( x \neq 0  \) is an eigenvector of \( {L}_{A} \) corresponding to the eigenvalue \( \lambda  \). So, \( \lambda  \) must be an eigenvalue of \( A  \) by definition. By Theorem 5.2, we must have \( \text{det}(A - \lambda I ) = 0  \). By {\hyperref[Exercise 5.1.14]{Exercise 5.1.14}}, we have   
        \[ \text{det}(A - \lambda I ) = \text{det}(A^{t} - \lambda I ) = 0.  \]
        Thus, \( \lambda  \) is an eigenvalue of \( A^{t} \) by Theorem 5.2 and so \( \lambda  \) is corresponding to the eigenvector \( x  \) of \( L_{A^{t}} \). By Theorem 5.4 again, we can see that \( x \in {E}'_{\lambda} \). Thus, \( {E}_{\lambda} \subseteq {E}'_{\lambda} \). We can reverse this argument to show the other containment. Thus, \( {E}'_{\lambda} \subseteq {E}_{\lambda} \) and so we conclude that \( {E}_{\lambda} = E'_{\lambda} \).
        \end{proof}
    \item[(c)] Prove that if \( A  \) is diagonalizable, then \( A^{t}  \) is diagonalizable.
        \begin{proof}
        Apply the same argument in part (b) of Exercise 5.2.12 and part (a) of this exercise to show that \( A^{t} \) is diagonalizable.
        \end{proof}
\end{enumerate}

\begin{definition}[Simultaneously Diagonalizable]
    Two linear operators \( T  \) and \( U  \) on a finite-dimensional vector space \( V  \) are called \textbf{simultaneously diagonalizable} if there exists an ordered basis \( \beta \) for \( V  \) such that both \( [T]_{\beta} \) and \( [U]_{\beta} \) are diagonal matrices. Similarly, \( A,B \in {M}_{n \times n}(F) \) are called \textbf{simultaneously diagonalizable} if there exists an invertible matrix \( Q \in {M}_{n \times n}(F) \) such that both \( Q^{-1} A Q  \) and \( Q^{-1} B Q   \) are diagonal matrices.
\end{definition} 

\subsection*{Exercise 5.2.17} 
\begin{enumerate}
    \item[(a)] Prove that if \( T  \) and \( U  \) are simultaneously diagonalizable linear operators on a finite-dimensional vector space \( V  \), then the matrices \( [T]_{\beta} \) and \( [U]_{\beta} \) are simultaneously diagonalizable for any ordered basis \( \beta \).
        \begin{proof}
        Let \( T, U  \) be simultaneously diagonalizable linear operators. Since \( V  \) is finite-dimensional let \( \beta'  \) be another ordered basis for \( V  \) (that is distinct from \( \beta \)). Using Theorem 2.23, we have
        \[  [T]_{\beta'} = Q^{-1} [T]_{\beta} Q \]
        where \( Q^{-1}, Q  \) are invertible and diagonal where 
        \begin{center}
            \( Q = [{I}_{V}]_{\beta'}^{\beta}  \) and \( Q^{-1} = [{I}_{V}]_{\beta}^{\beta'}  \).
        \end{center}
        Since \( [T]_{\beta} \) is diagonal, \( [T]_{\beta'}  \) is diagonal as well. We can apply this same argument to \( U  \) to show that \( P^{-1} [U]_{\beta} P   \) for invertible \( P \).
        \end{proof}
    \item[(b)] Prove that if \( A  \) and \( B  \) are simultaneously diagonalizable matrices, then \( {L}_{A} \) and \( {L}_{B} \) are simultaneously diagonalizable linear operators.
        \begin{proof}
        Let \( T = {L}_{A} \) and \( U = {L}_{B} \). Apply part (a) to show that \( {L}_{A} \) and \( {L}_{B} \) are also simultaneously diagonalizable linear operators. 
        \end{proof}
\end{enumerate}

\subsection*{Exercise 5.2.18} 
\begin{enumerate}
    \item[(a)] Prove that if \( T  \) and \( U  \) are simultaneously diagonalizable operators, then \( T  \) and \( U  \) commute (i.e., \( TU = UT \)). 
        \begin{proof}
        
        \end{proof}
    \item[(b)] Show that if \( A  \) and \( B  \) are simultaneously diagonalizable matrices, then \( A  \) and \( B  \) commute.
        \begin{proof}
        
        \end{proof}
\end{enumerate}

\subsection*{Exercise 5.2.19} Let \( T  \) be a diagonalizable linear operator on a finite-dimensional vector space, and let \( m  \) be any positive integer. Prove that \( T  \) and \( T^{m} \) are simultaneously diagonalizable.
\begin{proof}

\end{proof}

\subsection*{Exercise 5.2.20} Let \( {W}_{1}, {W}_{2}, \dots, {W}_{k} \) be subspaces of a finite-dimensional vector space \( V  \) such that 
\[  \sum_{ i=1 }^{ k  } {W}_{i} = V. \]
Prove that \( V  \) is the direct sum of \( {W}_{1}, {W}_{2}, \dots, {W}_{k} \) if and only if 
\[  \text{dim}(V) = \sum_{ i=1 }^{ k } \text{dim}({W}_{i}). \]
\begin{proof}

\end{proof}


