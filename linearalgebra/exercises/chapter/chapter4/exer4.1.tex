\section{Determinants of Order 2}

\subsection*{Exercise 4.1.5} Prove that if \( B   \) is the matrix obtained by interchanging the rows of a \( 2 \times 2  \) matrix \( A  \), then \( \text{det}(B) = - \text{det}(A) \).
\begin{proof}
Suppose \( B \in {M}_{2 \times 2}(F)  \) can be obtained by interchanging the rows of \( A \in {M}_{2 \times 2}(F)  \) where  
\[  A = \begin{pmatrix}
    {A}_{11} & {A}_{12} \\
    {A}_{21} & {A}_{22}
\end{pmatrix} \ \text{ and } \ B = \begin{pmatrix}
    {A}_{21} & {A}_{22} \\
    {A}_{11} & {A}_{12}
\end{pmatrix}. \]
Applying the definition of determinant for \( 2 \times 2  \) matrices, we get that
\begin{align*}
    \text{det}(B) &= {A}_{12} {A}_{21} - {A}_{11} {A}_{22} \\
                  &= - ({A}_{11}{A}_{22} - {A}_{12} {A}_{21}) \\
                  &= - \text{det}(A).
\end{align*}
Hence, we have that \( \text{det}(B) = - \text{det}(A) \).
\end{proof}

\subsection*{Exercise 4.1.6} Prove that if the two columns of \( A \in {M}_{2 \times 2}(F)  \) are identical, then \( \text{det}(A) = 0  \).
\begin{proof}
Let \( A \in {M}_{2 \times 2}(F)  \) be defined as
\[  A = \begin{pmatrix}
    {A}_{11} & {A}_{12} \\
    {A}_{21} & {A}_{22}
\end{pmatrix} \] 
with \( {A}_{11} = {A}_{12}  \) and \( {A}_{21} = {A}_{22} \) by assumption. Applying the definition of the determinant gives us \( \text{det}(A) = 0  \).
\end{proof}

\subsection*{Exercise 4.1.7} Prove that \( \text{det}(A^{t}) = \text{det}(A) \) for any \( A \in {M}_{2 \times 2}(F)  \).
\begin{proof}
Let \( A \in {M}_{2 \times 2} (F) \) where 
\[  A = \begin{pmatrix}
    {A}_{11} & {A}_{12} \\
    {A}_{21} & {A}_{22}
\end{pmatrix} \ \text{ and } \ A^{t} = \begin{pmatrix}
    {A}_{11} & {A}_{21} \\
    {A}_{12} & {A}_{22}
\end{pmatrix}. \]
Applying the determinant to \( A  \) and \( A^{t} \), we can see that
\[  \text{det}(A) = {A}_{11} {A}_{22} - {A}_{12} {A}_{21} = \text{det}(A^{t}). \]
\end{proof}

\subsection*{Exercise 4.1.8} Prove that if \( A \in {M}_{2 \times 2}(F) \) is upper triangular, then \( \text{det}(A) \) equals the product of diagonal entries of \( A  \).
\begin{proof}
Let \( A \in {M}_{2 \times 2}(F)  \) be an upper triangular matrix. Then we have
\[  A = \begin{pmatrix}
    {A}_{11} & {A}_{12} \\
    0 & {A}_{22}
\end{pmatrix}. \]
Using the definition of determinant, we have 
\[  \text{det}(A) = {A}_{11}{A}_{22} - 0 {A}_{12} = {A}_{11} {A}_{22}. \]
Hence, \( \text{det}(A)  \) equals the product of diagonal entries of \( A  \).
\end{proof}

\subsection*{Exercise 4.1.9} Prove that \( \text{det}(AB) = \text{det}(A) \text{det}(B) \) for any \( A,B \in {M}_{2 \times 2}(F) \).
\begin{proof}
Let \( A,B \in {M}_{2 \times 2}(F) \) with 
\[  A = \begin{pmatrix}
    {A}_{11} & {A}_{12} \\
    {A}_{21} & {A}_{22}
\end{pmatrix} \ \text{ and } \ B = \begin{pmatrix}
    {B}_{11} & {B}_{12} \\ 
    {B}_{21} & {B}_{22}
\end{pmatrix}. \]
Using the definition of the matrix product, we have that
\[  AB = \begin{pmatrix}
    {A}_{11}{B}_{11} + {A}_{12}{B}_{21} & {A}_{11} {B}_{12} + {A}_{12} {B}_{22} \\
    {A}_{21} {B}_{11} + {A}_{22} {B}_{21} & {A}_{21} {B}_{12} + {A}_{22} {B}_{22} 
\end{pmatrix}. \]
By definition of the determinant, we have
\begin{align*}
    \text{det}(AB) &= ({A}_{11}{B}_{11} + {A}_{12}{B}_{21})({A}_{21}{B}_{12} + {A}_{22} {B}_{22}) \\ 
                   &- ({A}_{21} {B}_{11} + {A}_{22} {B}_{21}) ({A}_{11} {B}_{12} + {A}_{12} {B}_{22}) \\
                   &=  {A}_{11}{A}_{21} {B}_{11}{B}_{12} + {A}_{12}{A}_{21}{B}_{12}{B}_{21} \\ &+ {A}_{12}{A}_{22}{B}_{21}{B}_{22} + {A}_{12}{A}_{21} {B}_{11} {B}_{22} \\
                   &= ({A}_{11}{A}_{22} - {A}_{12}{A}_{21}) {B}_{11}{B}_{22} - ({A}_{11}{A}_{22} - {A}_{12} {A}_{21}){B}_{12}{B}_{21} \\
                   &= ({A}_{11}{A}_{22} - {A}_{12} {A}_{21})({B}_{11}{B}_{22} - {B}_{12}{B}_{21}) \\
                   &= \text{det}(A) \text{det}(B).
\end{align*}
Hence, we have that \( \text{det}(AB) = \text{det}(A) \text{det}(B) \).
\end{proof}

\subsection*{Exercise 4.1.10} The \textbf{classical adjoint} of a \( 2 \times 2  \) matrix \( A \in {M}_{2 \times 2}(F)  \) is the matrix 
\[  C = \begin{pmatrix}
    {A}_{22} & - {A}_{12} \\
    - {A}_{21} & {A}_{11}
\end{pmatrix}. \]
Prove that
\begin{enumerate}
    \item[(a)] \( CA = AC = [\text{det}(A)] I. \)
    \item[(b)] \( \text{det}(C) = \text{det}(A) \).
    \item[(c)] The classical adjoint of \( A^{t} \) is \( C^{t} \).
    \item[(d)] If \( A  \) is invertible, then \( A^{-1} = [\text{det}(A)]^{-1} C \).
\end{enumerate}

\begin{proof}
    Let \( A, C \in {M}_{2 \times 2}(F)  \)  where \( C  \) is the classical adjoint of \( A  \).
\begin{enumerate}
    \item[(a)] We show that \( CA = [\text{det}(A)]I  \). Applying the definition of the matrix product, we get that
        \begin{align*}
            CA &= \begin{pmatrix}
                {A}_{22} & - {A}_{12} \\
                -{A}_{21} & {A}_{11}
            \end{pmatrix} \begin{pmatrix} 
                {A}_{11} & {A}_{12} \\ 
                {A}_{21} & {A}_{22}
                      \end{pmatrix}  \\
               &= \begin{pmatrix}
                   {A}_{11}{A}_{22} - {A}_{12} & {A}_{22}{A}_{12} - {A}_{12}{A}_{22} \\
                   {A}_{11}{A}_{21} -{A}_{11}{A}_{21} & {A}_{11}{A}_{22} - {A}_{12}{A}_{21} 
               \end{pmatrix} \\
               &= \begin{pmatrix} 
                   \text{det}(A) & 0 \\
                   0 & \text{det}(A)
                         \end{pmatrix} \\
               &= [\text{det}(A)] I.
        \end{align*}
        Note that proving \( AC = [\text{det}(A)] I  \) is a similar process as above. Hence, we have \( AC = CA = [\text{det}(A)] I  \)
    \item[(b)] Applying the definition of determinant gives us
        \[  \text{det}(C) = {A}_{11}{A}_{22} - {A}_{12}{A}_{21} = \text{det}(A). \]
    \item[(c)] Note that the transpose of \( A \in {M}_{2 \times 2 }(F) \) is
        \[  A^{t} = \begin{pmatrix}
            {A}_{11} & {A}_{21} \\ 
            {A}_{12} & {A}_{22}
        \end{pmatrix} \]
        where the classical adjoint of \( A^{t}  \) is
        \[  W = \begin{pmatrix}
            {A}_{22} &  - {A}_{21} \\
            - {A}_{12} & {A}_{11}
        \end{pmatrix}. \]
        But note that this is just the transpose of \( C  \) which was defined earlier. So, we have \( W = C^{t} \) is the classical adjoint of \( A^{t} \). 
    \item[(d)] Since \( A \) is an invertible \( 2 \times 2 \) matrix, we can write
        \begin{align*}
            A^{-1} &= \frac{ 1 }{ \text{det}(A) } \begin{pmatrix}
                {A}_{22} & - {A}_{12} \\
                -{A}_{21} & {A}_{11}
            \end{pmatrix} \\
                   &= [\text{det}(A)]^{-1} C 
        \end{align*}
        by Theorem 4.2.
\end{enumerate}
\end{proof}

\subsection*{Exercise 4.1.11} Let \( \delta: {M}_{2 \times 2}(F) \to F  \) be a function with the following three properties.
\begin{enumerate}
    \item[(i)] \( \delta \) is a linear function of each row of the matrix when the other row is held fixed.
    \item[(ii)] If the two rows of \( A \in {M}_{2 \times 2}(F)  \) are identical \( \delta(A) = 0  \).
    \item[(iii)] If \( I  \) is the \( 2 \times 2  \) identity matrix, then \( \delta(I) = 1  \).
\end{enumerate}
Prove that \( \delta(A) = \text{det}(A)  \) for all \( A \in {M}_{2 \times 2}(F)  \). 
\begin{proof}
Let \( A \in {M}_{2 \times 2}(F)  \). Let the vectors \( u,v \in F^{2} \) for an ordered basis \( \beta \) for \( F^{2}  \). Define the row vectors of \( A  \) by \( u = ({A}_{11}, {A}_{12}) \) and \( v = ({A}_{21}, {A}_{22}) \). Note that
\[  A = \begin{pmatrix} 
           u \\
           v
          \end{pmatrix} \]
          and that
          \[   \delta \begin{pmatrix} 
                    u \\ 
                    v
                    \end{pmatrix} = O \begin{pmatrix}
                        u \\
                        v
                    \end{pmatrix} \cdot A \begin{pmatrix}
                        u \\
                        v
                    \end{pmatrix}  \]
    where \( O  \) is the \textbf{orientation} of \( \beta \) and \( A  \) is the are of the parallelogram formed by \( u \) and \( v \).
    Since
    \[  A = \Big| \text{det} \begin{pmatrix} 
               u \\
               v
              \end{pmatrix}  \Big|,    \]
              and definition of \( O  \), we find that
   \begin{align*}
      \delta(A)  = \delta \begin{pmatrix} 
                 u \\
                 v
                \end{pmatrix} &= O \begin{pmatrix} 
                           u \\
                           v
                          \end{pmatrix} \cdot  A \begin{pmatrix} 
                                     u \\
                                     v
                                    \end{pmatrix}  \\
                                    &= \frac{ \text{det} \begin{pmatrix} 
                                               u \\
                                               v
                                              \end{pmatrix}  }{ \Big| \text{det} \begin{pmatrix} 
                                                         u \\
                                                         v
                                                        \end{pmatrix}  \Big|  } \cdot  \Big| \text{det}\begin{pmatrix} 
                                                                   u \\
                                                                   v
                                                                  \end{pmatrix}  \Big| \\ 
                                    &= \text{det} \begin{pmatrix} 
                                               u \\
                                               v
                                              \end{pmatrix} \\
                                    &= \text{det}(A).
   \end{align*}
   Hence, \( \delta(A) = \text{det}(A) \).
\end{proof}
