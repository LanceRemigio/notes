\section{Determinants of Order n}

\subsection*{Exercise 4.2.23} Prove that the determinant of an upper triangular matrix is the product of its diagonal entries.
\begin{proof}
We proceed by inducting on \( n  \). The result easily follows from cases \( n = 1  \) and \( n = 2  \), with the former leading to just the single entry itself and the latter being easily shown by applying the determinant formula found in Section 4.1 for \( 2 \times 2  \) upper triangular matrices.

Let \( A \in {M}_{n \times n}(F) \) be upper triangular. Now, assume the result holds for \( (n-1) \times (n-1)  \) upper triangular matrices and that \( n \geq 3  \) and \(  1 \leq j \leq n  \). We can obtain a new matrix \( D  \) by performing a finite number of type 3 row operations on \( A  \) to convert the entries found in the first row and columns \( 1 < j \leq n    \) into zeroes. Using cofactor expansion along the first row, we find that  
\begin{align*}
    \text{det}(D) &= \sum_{ j=1 }^{ n } (-1)^{1+j} {D}_{1j} \cdot \text{det}({\tilde{D}}_{1j}) \\
                  &=  (-1)^{1+1} {D}_{11} \cdot \text{det}({\tilde{D}}_{11}) + (-1)^{1+2}{D}_{12} \cdot \text{det}({\tilde{D}}_{12}) + \cdots \\ 
                  &+ (-1)^{1+n} {D}_{1n} \cdot \text{det}({\tilde{D}}_{1n}) \\ 
                  &= (-1)^{1+1} {D}_{11} \cdot \text{det}({\tilde{D}}_{11}) + (-1)^{1+2} (0) \cdot \text{det}({\tilde{D}}_{12}) + \cdots \\ 
                  &+ (-1)^{1+n }(0) \cdot \text{det}({\tilde{D}}_{1n}) \\ 
                  &= (-1)^{1+1} {D}_{11} \cdot \text{det}({\tilde{D}}_{11}).
\end{align*}
Since \( {\tilde{D}}_{11} \) is an \( (n-1) \times (n-1) \) matrix, we can apply the induction to hypothesis to write that
\[  \text{det}({\tilde{D}}_{11}) = \prod_{i=2}^{n} {D}_{ii}.  \]
Thus, we must have that
\[  \text{det}(D) = (-1)^{1+1} {D}_{11} \cdot \text{det}({\tilde{D}}_{11}) = \prod_{i=1}^{n} {D}_{ii}.    \]
Note that \( \text{det}(A) = \text{det}(D) \) by Theorem 4.6 and that \( {D}_{ii} = {A}_{ii} \) since we only changed the entries in the first row (that were not \( {A}_{11} \)). Thus, the result holds for \( n \times n  \) upper triangular matrices. 
\end{proof}

\subsection*{Exercise 4.2.24} Prove the corollary to Theorem 4.3.
\begin{proof}
We proceed by inducting on \( n  \). Let \( {A} \in M_{n \times n}(F) \). If \( n = 1  \), then
we simply have \( {A}_{11} = 0  \) and the result follows immediately. If \( n = 2  \), then the result follows from applying the formula for \( 2 \times 2  \) determinants and choosing either the \( i=1  \) or \( i = 2  \) as the rows with all zeros. Now, assume that the corollary holds for \( (n-1) \times (n-1) \) matrices. Let \( n \geq 3  \) and \( 1 \leq j \leq n  \). Let \( r  \) represent the selected row of all zero entries. Suppose that for some \( r  \ (1 \leq r \leq n) \), we have \( {a}_{r} = u + kv  \) for some \( u,v \in F^{n} \) and \( k \in F  \). Let \( u = ({b}_{1}, {b}_{2}, \dots, {b}_{n}) \) and \( v = ({c}_{1}, {c}_{2}, \dots, {c}_{n}) \), and let \( B  \) and \( C  \) be the matrices obtained from \( A  \) by replacing row \( r  \) of \( A  \) by \( u \) and \( v  \), respectively. Furthermore, let the row \( r - 1  \) be defined as in the proof for Theorem 4.3. If \( r = 1  \), then by cofactor expansion along the first row, we have 
\begin{align*}  
    \text{det}(A) &= \sum_{ j=1 }^{ n } (-1)^{1+j} {A}_{1j} \cdot \text{det}({\tilde{A}}_{1j}) \\  
                  &= (-1)^{1+1} (0) \cdot \text{det}({\tilde{A}}_{11}) + (-1)^{1+2} (0) \cdot \text{det}({\tilde{A}}_{12}) + \cdots \\  
                  &+ (-1)^{1+n} (0) \cdot \text{det}({\tilde{A}}_{1n}) \\
                  &= 0. 
\end{align*}
Now, suppose \( r > 1  \). Then by Theorem 4.3, we get
\begin{align*}
    \text{det}(A) &= \text{det}(B) + k \text{det}(C)  \\
                  &= \sum_{ j=1 }^{ n }(-1)^{1+j} {B}_{1j} \cdot \text{det}({\tilde{B}}_{1j}) + k \sum_{ j=1 }^{ n } (-1)^{1+j} {C}_{1j} \cdot \text{det}({\tilde{C}}_{1j}).
\end{align*}
Note that \( {A}_{1j} = {B}_{1j} = {C}_{ij}  \) and that \( {\tilde{A}}_{1j} = {\tilde{B}}_{1j} = {\tilde{C}}_{1j}\) except for row \( r - 1  \). Since \( {\tilde{B}}_{1j}  \) and \( {\tilde{C}}_{1j} \) are \( (n-1) \times (n-1) \) matrices, we find that \( \text{det}({\tilde{B}}_{1j}) = 0  \) and \( \text{det}({\tilde{C}}_{1j}) = 0  \) by our induction hypothesis.  So, we get that
\[  \text{det}(A) = \text{det}(B) + k \text{det}(C) = 0 + 0 = 0 \]
and thus \( \text{det}(A) = 0  \). Hence, we can see that the corollary holds for all \( n \times n  \) matrices.
\end{proof}

\subsection*{Exercise 4.2.25} Prove that \( \text{det}(kA) = k^{n} \text{det}(A) \) for any \( A \in {M}_{n \times n}(F) \).
\begin{proof}
Let \( A \in {M}_{n \times n}(F) \) and \( k \in F  \) be nonzero. We proceed by inducting on \( n  \). If \( n = 1  \), then the result follows immediately. If \( n = 2  \), then apply the formula found in section 4.1 for \( 2 \times 2  \) matrices to \( A  \) to get \( \text{det}(kA) = k^{2} \text{det}(A) \). 

Suppose \( n > 2  \) and suppose the result holds for \( (n-1) \times (n-1) \) matrices. Since \( {\tilde{A}}_{1j} \) is an \( (n-1) \times (n-1)  \) matrix, we must have \( \text{det}({k\tilde{A}}_{1j}) = k^{n-1} \text{det}({\tilde{A}}_{1j})  \). Using cofactor expansion along the first row, we get that 
\begin{align*}
    \text{det}(A) &= \sum_{ j=1 }^{ n } (-1)^{1 + j} {kA}_{1j} \cdot \text{det}(k {\tilde{A}}_{1j})  \\
                  &= k \sum_{ j=1 }^{ n } (-1)^{1+j } {A}_{1j} \cdot k^{n-1} \text{det}({\tilde{A}}_{1j}) \\
                  &=  k \cdot k^{n-1} \sum_{ j=1 }^{ n }(-1)^{1+j} {A}_{1j} \cdot \text{det}({\tilde{A}}_{1j}) \\
                  &= k^{n} \text{det}(A).
\end{align*}
\end{proof}

\subsection*{Exercise 4.2.26} Let \( A \in {M}_{n \times n}(F) \). Under what conditions is \( \text{det}(-A) = \text{det}(A) \).
\begin{proof}
In order for \( \text{det}(-A) = \text{det}(A) \), \( A  \) must be an even matrix.
\end{proof}

\subsection*{Exercise 4.2.27} Prove that if \( A \in {M}_{n \times n}(F) \) has two identical columns, then \( \text{det}(A) = 0  \).
\begin{proof}
    Let \( A \in {M}_{n \times n}(F) \). Suppose that \( A  \) contains two identical columns at row \( r  \) and row \( s  \) where \( r \neq s  \). Let \( {u}_{1}, {u}_{2}, \dots, {u}_{n}  \) be the columns of \( A \). Note that these are the rows of \( A^{t} \). Observe that \( {u}_{r} = {u}_{s} \) for \( r \neq s  \), we can apply the Corollary to Theorem 4.4 to write  
    \[  \text{det}(A^{t}) = 0.  \]
    Since \( \text{rank}(A) = \text{rank}(A^{t}) \), we get that \( \text{det}(A) = \text{det}(A^{t}) = 0  \). Hence, \( \text{det}(A) = 0 \) for all \( A \in {M}_{n \times n}(F)  \).
\end{proof}
\begin{proof}
Let \( A \in {M}_{n \times n}(F) \). We proceed by inducting on \( n \geq 2  \). For \( n = 2  \), let \( {u}_{1} = ({A}_{11}, {A}_{21}) \) and \( {u}_{2} = ({A}_{12}, {A}_{22} ) \) denote the two columns that make up \( A  \). By Exercise 4.1.6, we get that \( \text{det}(A) = 0  \).     

Now, suppose \( n > 2  \) and that the result holds for \( (n-1) \times (n-1)  \) matrices. Let \( 1 \leq j \leq n \). Using cofactor expansion along any row \( i \), we get that 
\[  \text{det}(A) = \sum_{ j=1 }^{ n } (-1)^{i + j} {A}_{ij} \cdot \text{det}({\tilde{A}}_{1j}) \tag{1} \] by Theorem 4.4.
Since \( {\tilde{A}}_{ij} \) is an \( (n-1) \times (n-1)  \) matrix, we know that \( {\tilde{A}}_{1j} \) contains identical columns \( {u}_{r}   \) and \( {u}_{s} \) where \( r \neq s  \). Thus, we have that \( \text{det}({\tilde{A}}_{1j}) = 0  \) by our induction hypothesis. Thus, (1) implies that \( \text{det}(A) = 0  \) which ends our induction argument. 
\end{proof}

\subsection*{Exercise 4.2.28} Compute \( \text{det}({E}_{i})  \) if \( {E}_{i} \) is an elementary matrix of type \( i  \).
\begin{proof}
Let \( {E}_{i} \) be an elementary matrix of type \( i \).
\end{proof}

\subsection*{Exercise 4.2.29} Prove that if \( E  \) is an elementary matrix, then \( \text{det}(E^{t}) = \text{det}(E) \).
\begin{proof}
Suppose that \( E  \) is an \( n \times n \) elementary matrix. By Theorem 3.2, \( E  \) is an invertible matrix. Thus, \( \text{rank}(E) = n  \) and is equivalent to the rank of \( E^{t}   \) by Corollary to Theorem 3.6. Thus, we must have that \( \text{det}(E) = \text{det}(E^{t}) \).
\end{proof}


