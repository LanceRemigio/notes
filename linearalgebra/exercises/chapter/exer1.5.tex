\section{Linear Dependence and Linear Independence}

\subsubsection{Exercise 1.5.1} Label the following statements as true or false.

\begin{enumerate}
    \item[(a)] If \( S  \) is a linearly dependent set, then each vector in \( S  \) is a linear combination of other vector in \( S  \).
        \begin{proof}
        \textbf{True}
        \end{proof}
    \item[(b)] Any set containing the zero vector is linearly dependent.
        \begin{proof}
        \textbf{True}
        \end{proof}
    \item[(c)] The empty set is linearly dependent.
        \begin{proof}
        \textbf{False}. It is linearly independent.
        \end{proof}
    \item[(d)] Subsets of linearly dependent sets are linearly dependent. 
        \textbf{True} by Theorem 6.
    \item[(e)] Subsets of linearly independent sets are linearly independent.
        \begin{proof}
        \textbf{True} by corollary to Theorem 6.
        \end{proof}
    \item[(f)] If \( a_{1} x_{1} + a_{2} x_{2} + \cdots + a_{n} x_{n} = 0  \) and \( x_{1}, x_{2}, \dots, x_{n}  \) are linearly independent, then all the scalars \( a_{i}  \) are zero.
        \begin{proof}
       \textbf{True} this is by definition. 
        \end{proof}
\end{enumerate}

\subsubsection{Exercise 1.5.4} In \( F^{n}  \), let \( e_{j}  \) denote the vector whose \( j \)th coordinate is \( 1  \) and whose other coordinates are \( 0  \). Prove that \( \{ e_{1}, e_{2}, \dots, e_{n}  \}  \) is linearly independent.
\begin{proof}
    Choose a finite amount of scalars \( a_{1}, a_{2}, \dots, a_{n} \in F  \) to create the following linear combination:
    \[  a_{1} e_{1} + a_{2} e_{2} + \cdots + a_{n} e_{n} = (0,0, \dots, 0). \tag{1} \]
    To show that the set \( \{ e_{1} ,e_{2}, \dots, e_{n}  \}  \) is linearly independent, we need to show that the scalars \( a_{1}, a_{2}, \dots, a_{n} \in F  \) have the trivial representation; that is, \( a_{1} = a_{2} = \cdots = a_{n} = 0  \).
    Since the \( j \)th coordinate of \( e_{j} \) is \( 1  \) but \( 0 \) in all the other entries, we have that 
    \begin{align*}
        &a_{1} (1,0,\dots, 0) + a_{2} (0,1,\dots,0) + \cdots + a_{n} (0,0, \dots, 1 )    \\
        &= (a_{1}, 0, \dots, 0  ) + (0 , a_{2}, \dots, 0 ) + \cdots + (0,0, \dots, a_{n}) \\
        &= (a_{1}, a_{2}, \dots , a_{n}  ).
    \end{align*}
    Hence, we have 
    \[  (a_{1}, a_{2} , \dots, a_{n} ) = (0,0, \dots, 0). \]
    Equating each entry of the left side of the equation above to \( 0 \), we find that \( a_{i} = 0  \) for all \( 1 \leq j \leq n  \). Hence, the set \( \{ e_{1}, e_{2}, \dots, e_{n}  \}  \) is linearly independent.
\end{proof}


\subsubsection{Exercise 1.5.4} Show that the set \( \{ 1,x, x^{2}, \dots, x^{n}  \}  \) is linearly independent in \( P_{n}(F) \).
\begin{proof}
    Just like the prior exercise, we need to show that we can find scalars \( a_{0}, a_{1}, \dots, a_{n} \in F  \)  such that 
    \[  a_{0} + a_{1} x + a_{2} x^{2} + \cdots + a_{n} x^{n} = 0   \]
    where \( a_{i} = 0  \) for all \( 0 \leq i \leq n \). Note that the \( 0  \) polynomial is just 
    \[  0 + 0 x + 0 x^{2} + \cdots + 0 x^{n} = 0.  \]
    Hence, equating coefficients we immediately get that \( a_{i} = 0  \) for all \( 0 \leq i \leq n \). Thus, the set \( \{ 1 , x , x^{2}, \dots, x^{n} \}  \)  is linearly independent. 
\end{proof}

\subsubsection{Exercise 1.5.6} In \( M_{m \times n}(F) \), let \( E^{ij}  \) denote the matrix whose only nonzero entry is \( 1  \) in the \( i \)th row and \( j \)th column. Prove that 
\( \{ E^{ij} : 1 \leq i \leq m , 1 \leq j \leq n  \}  \) is linearly independent.
\begin{proof}
    First, we create a  linear combination of a finite amount vectors in \( E  = \{ E^{ij} : 1 \leq i  \leq m , 1 \leq j \leq n   \}  \) with scalars \( \delta_{k}  \) for \( 1 \leq k \leq N  \) with \( N = mn \) as the number of total entries in each matrix in \( \{ E^{ij} : 1 \leq i \leq m , 1 \leq j \leq n  \}  \). Note that after doing our scalar multiplication and summing up each term, we find that each  \( \delta_{k} E_{ij} = \delta_{k}   \) in our linear combination can equated with a corresponding \( i  \) and  \( j  \) entry in the zero matrix such that \( \delta_{k} =  0  \) for all \( 1 \leq k \leq N  \). Hence, \( E  \) is a linearly independent set.
\end{proof}

\subsubsection{Exercise 1.5.7} Recall from Example 3 in Section 1.3 that the set of diagonal matrices in \( M_{2 \times 2 }(F) \) is a subspace. Find a linearly independent set that generates this subspace.
\begin{proof}
Define \( W  \) as the linearly independent spanning set of the set of diagonal matrices in  \( M_{2 \times 2 } \) where 
\begin{align*}
    W &= \Bigg\{ \begin{pmatrix}
            1 & 0 \\
            0 & 0 
    \end{pmatrix} , \begin{pmatrix}
            0 & 0 \\
            0 & 1 
    \end{pmatrix} \Bigg\}. \\
\end{align*}
To see why \( W  \) is a linearly independent set, choose scalars \( \delta_{1} , \delta_{2} \in F  \) such that 
\[  \delta_{1} \begin{pmatrix}
    1 & 0 \\
    0 & 0 
\end{pmatrix}  + \delta_{2} \begin{pmatrix}
    0 & 0 \\
    0 & 1 
\end{pmatrix} = \begin{pmatrix}
    0 & 0 \\
    0 & 0 
\end{pmatrix}.\]
Performing scalar multiplication and vector addition gives us the following equation 
\[  \begin{pmatrix}
    \delta_1 & 0 \\ 
    0 & \delta_2  
\end{pmatrix}  = \begin{pmatrix}
    0 & 0 \\
    0 & 0 
\end{pmatrix}. \]
Since the zero matrix is a diagonal matrix, we know that equation entries where  \( i = j    \)yields \( \delta_{1} = \delta_{2} = 0 \). Hence, \( W  \) is a linearly independent set that generates the set of diagonal matrices of \( M_{2 \times 2 }(F ) \).
\end{proof}

\subsubsection{Exercise 1.5.8} Let \( S = \{ (1,1,0) , (1,0,1), (0,1,1) \}  \) be a subset of the vector space \( F^{3} \).
\begin{enumerate}
    \item[(a)] Prove that if \( F = \R  \), then \( S  \) is linearly independent.
        \begin{proof}
        
        \end{proof}
    \item[(b)] Prove that if \( F  \) has characteristic two, then \( S  \) is linearly dependent.
        \begin{proof}
        
        \end{proof}
\end{enumerate} 


\subsubsection{Exercise 1.5.9} Let \( u \) and \( v  \) be distinct vectors in a vector space \( V  \). Show that \( \{ u,v  \}  \) is linearly dependent if and only if \( u  \) or \( v  \) is a multiple of the other.
I have written two proofs for this:
\begin{proof}
Let \( u  \) and \( v  \) be distinct vectors in a vector space \( V  \). 

( \(  \Rightarrow \) ) Since \( \{ u,v  \}   \) is a linearly dependent set, we can find scalars \( a_{1} , a_{2} \in F   \) such that 
\[ a_{1} u + a_{2} v = 0 \tag{1} \] 
Suppose \( v  \) is not a multiple of \( u  \) and choose \( a_{1} \neq 0  \) since \( \{ u,v  \}   \) is linearly dependent. We need to show that \( u  \) is a multiple of \( v  \). Solving for \( u  \), we get that
\[  u =  - \frac{ a_{2} }{ a_{1} }   v. \]
Hence, \( u  \) is a multiple of \( v  \). 

( \( \Leftarrow \) ) Suppose \( u  \) or \( v  \) is a scalar multiple of the other. Assume \( u  \) is the scalar multiple of \( v  \). Then for some \( c \neq 0 \in F  \), we have \( u = cv  \).  Hence, we have \( u - cv = 1u - cv = 0  \). This tells us that \( \{ u,v  \}  \) is linearly dependent. 
\end{proof}


\subsubsection{Exercise 1.5.12} Prove Theorem 1.6 and its corollary.
\begin{proof}
See proof in notes.
\end{proof}

\subsubsection{Exercise 1.5.13} Let \( V  \) be a vector space over a field of characteristic not equal to two.
\begin{enumerate}
    \item[(a)] Let \( u  \) and \( v  \) be distinct vectors in \( V  \). Prove that \( \{ u,v  \}   \) is linearly independent if and only if \( \{ u + v , u - v  \}   \) is linearly independent.
        \begin{proof}
        Let \( u \) and \( v  \) be distinct vectors in \( V  \). 

        For the forwards direction, assume \( \{ u,v  \}   \) is a linearly independent set. We need to show that \( \{ u + v , u - v  \}  \) is linearly independent. Hence, we need to find \( a,b \in F  \) such that 
        \[  a(u+v) + b(u-v) = 0. \tag{1} \]
        Note that (1) leads to 
        \begin{align*}
            a(u+v) + b(u-v) &= au + av + bu - bv \\
                            &= au - bv + av + bu. 
        \end{align*}
        Since \( \{ u,v  \}   \) is a linearly independent set, we have that 
        \[  au - bv = 0  \]
        and 
        \[  av + bu = 0  \] for \( a=b = 0 \). Hence, 
        \[  a(u+v) + b(u-v) = 0  \] for \( a = b = 0  \) and so \( \{ u - v , u + v  \}   \) is a linearly independent set.

        For the backwards direction, suppose \( \{ u + v , u - v  \}  \) is linearly independent. We need to show that \( \{ u , v  \}   \) is linearly independent. Note that \( a,b \in F  \) such that 
        \[  a(u+v) + b(u-v) = 0  \]
        for \( a = b = 0  \) since \( \{ u - v, u + v  \}  \) is linearly independent. Note that
        \begin{align*}
            a(u+v) + b(u-v) &= au + av + bu - bv \\
                            &= au - bv + av + bu \\
                            &= 0 + av + bu \\
                            &= 0. 
        \end{align*}
        Thus, \( av + bu = 0  \) where \( a,b  \) both zero. Thus, the set \( \{ u,v  \}   \) is linearly independent.
        \end{proof}
\end{enumerate}

