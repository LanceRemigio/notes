\section{Linear Transformations, Null spaces, and Ranges}


\subsubsection{Exercise 2.1.7} Prove properties \( 1,2,3, \) and \( 4 \) on page 65.
\begin{proof}
See proof in notes.
\end{proof}

\subsubsection{Exercise 2.1.8} Prove that the transformations in Example 2 and 3 are linear.
\begin{proof}
    First we prove that \( T_{\theta}: \R^{2} \to \R^{2} \) defined by 
    \[  T_{\theta}(a_{1}, a_{2}) = (a_{1}\cos \theta - a_{2} \sin \theta, a_{1} \cos \theta + a_{2} \sin \theta) \]
    is linear. Let \( x,y \in \R^{2}  \) defined by \( x = (a_{1}, a_{2}) \) and \( y = (b_{1}, b_{2}) \). Let \( c \in F  \) such that \( cx = (ca_{1}, ca_{2}) \). To make the computation less difficult, we have
    \[  cx + y = (ca_{1} + b_{1}, ca_{2} + b_{2}). \]
    Using the definition \( T_{\theta} \) now, we have that 
    \begin{align*}
        T_{\theta}(cx + y) &= ((ca_{1} + b_{1})\cos \theta - (ca_{2} + b_{2}) \sin \theta,  \\
                           &(ca_{1} + b_{1}) \cos \theta + (ca_{2} + b_{2}) \sin \theta ) \\
                           &= (c (a_{1}\cos \theta  - a_{2} \sin \theta) + (b_{1} \cos \theta - b_{2} \sin \theta ), \\
                           &c ( a_{1} \cos \theta + a_{2} \sin \theta ) + (b_{1} \cos \theta + b_{2} \sin \theta)           ) \\
                           &= c (a_{1} \cos \theta - a_{2} \sin \theta, a_{1} \cos \theta + a_{2} \sin \theta)  \\
                           &+ (b_{1} \cos \theta - b_{2} \sin \theta , b_{1} \cos \theta + b_{2} \sin \theta) \\
                           &= c T_{\theta}(a_{1}, a_{2}) + T_{\theta}(b_{1}, b_{2}) \\
                           &= c T_{\theta}(x) + T_{\theta}(y).
    \end{align*}
    Hence, we get that \( T_{\theta} \) is linear.

    Using the same process, we show that \( T: \R^{2} \to \R^{2}  \) defined by \( T(a_{1}, a_{2}) = (a_{1}, - a_{2})\) is linear. That is, we have
    \begin{align*}
        T(cx + y) & (ca_{1} + b_{1}, - (ca_{2} + b_{2})) \\
                  &= (ca_{1} + b_{1}, -ca_{2} - b_{2}) \\ 
                  &= c(a_{1}, -a_{2}) + (b_{1}, -b_{2}) \\
                  &= c T(a_{1}, a_{2}) + T(b_{1}, b_{2}) \\
                  &= cT(x) + T(y).
    \end{align*}
    Hence, \( T  \) is also linear.
\end{proof}



\subsubsection{Exercise 2.1.13} Let \( V  \) and \( W  \) be vector spaces, let \( T: V \to W  \) be linear, and let \( \{ w_{1}, w_{2}, \dots, w_{k } \}  \) be a linearly independent subset of \( R(T) \). If \( S = \{ v_{1}, v_{2}, \dots, v_{k} \}  \) is chosen so that \( T(v_{i}) \) for \( i = 1,2,\dots, k  \) then \( S  \) is linearly independent.
\begin{proof}
    Consider
    \[  \sum_{ i=1 }^{ k  } a_{i} v_{i} = 0 \tag{1}  \]
    for some scalars \( a_{1}, a_{2}, \dots, a_{k } \in F  \). In order to show that \( S  \) is linearly independent, we need to show that \( a_{i} = 0   \) for all \( 1 \leq i \leq k  \). Since \( T  \) is linear, we get that \( T(0) = 0  \) implies
    \[  T \Big( \sum_{ i=1 }^{ k  } a_{i} v_{i} \Big) = 0.  \]
    Since \( T  \) is linear and \( S  \) is chosen so that \( T(v_{i}) = w_{i}  \) for \( 1 \leq i \leq k  \), we get that
    \[  \sum_{ i=1 }^{ k  } a_{i} T(v_{i}) =  0 \iff \sum_{ i=1 }^{ k  } a_{i} w_{i} = 0.   \]
    Since \( \{ w_{1}, w_{2}, \dots, w_{k} \}  \) is linearly independent, we must have \( a_{i} = 0  \) for all \( 1 \leq i \leq k  \). But this tell us that (1) must have the trivial representation. Hence, \( S  \) must also be linearly independent. 
\end{proof}

\subsubsection{Exercise 2.1.14} Let \( V \) and \( W \) be vector spaces and \( T: V \to W  \) be linear.
\begin{enumerate}
    \item[(a)] Prove that \( T \) is injective if and only if \( T  \) carries linearly independent subsets of \( V  \) onto linearly independent subsets of \( W  \).
        \begin{proof}
            (\( \Rightarrow \)) Define \( T: S \to \mathcal{W} \) where \( S  \) and \( \mathcal{W} \) are linearly independent subsets of \( V \) and \( W  \) respectively. Since \( T  \) is injective, we have that \( T  \) is an onto map by Theorem 2.5. 

            (\( \Leftarrow \)) Suppose \( T  \) carries linearly independent subsets of \( V  \) onto linearly independent subsets of \( W  \). Our goal is to show that \( T \) is injective. Suppose  
            \[  T \Big( \sum_{ i=1 }^{ n } a_{i} v_{i}  \Big) = T \Big( \sum_{ i=1 }^{ n } b_{i} v_{i} \Big) \tag{1} \]
            for some scalars \( a_{i}, b_{i}  \) and vectors \( v_{i} \in S  \) for all \( 1 \leq i \leq  k  \). Since \( T  \) is linear and \( T: S \to \mathcal{W}  \) is onto, we can re-write (1) into the following form:
            \[ \sum_{ i=1 }^{ k  } a_{i} T(v_{i}) = \sum_{ i=1 }^{ k  } b_{i} T(v_{i}) \tag{2}  \]
            which manipulating again, we get that
            \[  \sum_{ i=1 }^{ k  } (a_{i} - b_{i}) T(v_{i}) = 0. \tag{3}  \]
            Since \( T(v_{i}) \in W   \) and \( W  \) is linearly independent, we must have \( a_{i} - b_{i} = 0  \) if and only if \( a_{i} = b_{i} \). Hence, we have
            \[  \sum_{ i=1 }^{ n } a_{i} v_{i} = \sum_{ i=1 }^{ n  } b_{i} v_{i}   \]
            and so \( T  \) must be injective.
        \end{proof}
    \item[(b)] Suppose that \( T  \) is injective and that \( S  \) is a subset of \( V  \). Prove that \( S  \) is linearly independent if and only if \( T(S) \) is linearly independent.
        \begin{proof}
            (\( \Rightarrow \)) In order to show that \( T(S)  \) is linearly independent, we must show that 
            \[  \sum_{ i=1 }^{ n } a_{i} T(v_{i}) = 0   \tag{1} \] contains the trivial-representation. Since \( T  \) is linear, we can write (1) into the following form
            \[  T \Big( \sum_{ i=1 }^{ n } a_{i} v_{i} \Big) = 0. \] 
            This implies that
            \[  \sum_{ i=1 }^{ n } a_{i} v_{i} \in N(T). \]
            Since \( T  \) is injective, we know that \( N(T) = \{ 0  \}   \) (by Theorem 2.4), and so we must have
            \[  \sum_{ i=1 }^{ n } a_{i} v_{i} = 0. \tag{2} \]
            But \( v_{i} \in S  \) for \( 1 \leq i \leq k   \) is linearly independent, and so \( a_{i} = 0  \) for all \( 1 \leq i \leq k  \). This tells us that (1) contains the trivial-representation. Hence, \( T(S)  \) is linearly independent.

            (\( \Rightarrow \)) Suppose \( T(S) \) is linearly independent. Then observe that
            \[  \sum_{ i=1 }^{ n } a_{i} T(v_{i}) = 0  \]
        for some scalars \( a_{i}  \) and \( T(v_{i}) \in T(S)  \) for \( 1 \leq i \leq k  \) such that \( a_{i} =0  \). Since \( T  \) is linear and \( T  \) is injective, we can write
        \[  T \Big( \sum_{ i=1 }^{ n } a_{i} v_{i}  \Big) = 0 \iff \sum_{ i=1 }^{ n } a_{i} v_{i} = 0. \]
        Since \( a_{i} = 0  \) and \( v_{i} \in S  \), we also find that \( S  \) is a linearly independent set.
        \end{proof}
    \item[(c)] Suppose \( \beta = \{ v_{1}, v_{2}, \dots, v_{n} \}  \) is a basis for \( V  \) and \( T  \) is injective and surjective. Prove that \( T(\beta) = \{ T(v_{1}), T(v_{2}), \dots, T(v_{n}) \}  \) is a basis for \( W  \). 
        \begin{proof}
            Since \( \beta  \) is a basis for \( V  \), we get that \( \text{span}(T(\beta)) = R(T) \) by Theorem 2.2. Since \( T  \) is surjective, we have \( R(T) = W  \), so \( \text{span}(T(\beta))  = W \). Hence, \( T(\beta) \) generates \( W  \). 
            Since \( \beta \) is a linearly independent subset of \( V  \) and \( T  \) is an injective linear map, we must also have \( T(\beta) \) as a linearly independent subset of \( W \) by part (b).  Hence, \( T(\beta)  \) is a basis for \( W  \).
        \end{proof}
\end{enumerate}

\subsubsection{Exercise 2.1.15} Recall the definition of \( P(\R) \) on page 10. Define 
    \[ T: P(\R) \to P(\R)  \ \text{by} \   T(f(x)) = \int_{ 0 }^{ x }  f(t) \ dt. \]
Prove that \( T  \) is linear and injective, but not surjective.
\begin{proof}
First, we show that \( T: P(\R) \to P(\R)   \) defined by 
\[  T(f(x)) = \int_{ 0 }^{ x }  f(t) \ dt. \]
Let \( cf(x) + g(x)  \in P(\R) \) where \( c \in F  \). Then observe that
\begin{align*}
    T(cf(x) + g(x) ) &= \int_{ 0 }^{ x }  [cf(t) + g(t)] \ dt \\
                     &= \int_{ 0 }^{ x } cf(t) \ dt + \int_{ 0 }^{ x } g(t) \ dt \\
                     &= c \int_{ 0 }^{ x }  f(t) \ dt + \int_{ 0 }^{ x }  g(t) \ dt \\
                     &= c T(f(x)) + T(g(x)).
\end{align*}
Hence, \( T: P(\R) \to P(\R)  \) is a linear map. Let \( f(x) , g(x) \in P(\R) \), then
\begin{align*}
    T(f(x))&= T(g(x)) \\
    \int_{ 0 }^{ x }  f(t) \ dt &= \int_{ 0 }^{ x }  g(t) \ dt \\
    \frac{d  }{d x }  \Big[ \int_{ 0 }^{ x }  f(t) \ dt  \Big] &= \frac{d  }{d x }  \Big[ \int_{ 0 }^{ x }  g(t) \ dt  \Big] \\
    f(x) &= g(x).
\end{align*}
Hence, \( T  \) is an injective map. To see why \( T  \) fails to be surjective, observe that \( 1 \in P(\R) \) but we cannot find a polynomial \( f(x)  \) such that \( T(f(x)) = 1  \); that is, the integration of any polynomial can never yield 1. 
\end{proof}

\subsubsection{Exercise 2.1.16} Let \( T: P(\R) \to P(\R) \) be defined by \( T(f(x)) = f'(x)  \). Recall that \( T  \) is linear. Prove that \( T  \) is surjective, but not injective.
\begin{proof}
    Let \( T: P(\R) \to P(\R) \) be defined by \( T(f(x)) = f'(x) = \frac{d  }{d x } [f(x)]  \). For \( cf(x) + g(x)  \in P(\R)  \) where \( c \in F  \), observe that
    \begin{align*}
        T(cf(x) + g(x)) &= \frac{d  }{d x }  [cf(x) + g(x)] \\
                        &= \frac{d  }{d x } [cf(x)] + \frac{d  }{d x } [g(x)] \\
                        &= c \frac{d  }{d x } [f(x)] + \frac{d  }{d x } [g(x)] \\
                        &= cT(f(x)) + T(g(x)).
    \end{align*}
    Hence, \( T  \) is a linear map. Now, we want to show that \( T  \) is surjective. Define 
    \[  f(x) = \int_{ 0 }^{ x }  g(t)  \ dt. \]
    Then 
    \[  T(f(x)) = \frac{d  }{d x } [f(x)] = \frac{d  }{d x }  \Big[ \int_{ 0 }^{ x }  g(t) \ dt \Big] = g(x) \]
    by the Second Fundamental Theorem of Calculus. Hence, \( T  \) is surjective. 

    To show that \( T  \) is not injective, let \( x^{n} + a, x^{n} + b \in P(\R) \) where \( a, b \in \R  \) such that \( a \neq b   \) and \( n > 0   \). Clearly, we have \( x^{n} + a \neq x^{n} + b  \). But, we have
    \[  T(x^{n} + a) = \frac{d  }{d x }  [x^{n} + a]  = n x^{n-1}\]
    and 
    \[  T(x^{n} + b) = \frac{d  }{d x} [x^{n} + b] = nx^{n-1}.  \]
    Note that \( x^{n} + a \neq x^{n} + b \) yet \( T(x^{n} +a) = T(x^{n} +b)  \). Hence, \( T  \) is not injective.
\end{proof}

\subsubsection{Exercise 2.1.17} Let \( V  \) and \( W  \) be finite-dimensional vector spaces and \( T: V \to W  \) be linear.
\begin{enumerate}
    \item[(a)] Prove that if \( \text{dim}(V) < \text{dim}(W) \), then \( T  \) cannot be surjective.
        \begin{proof}
        Using the Dimension Theorem, we find that 
        \[  \text{rank}(T) = \text{dim}(V) - \text{nullity}(T). \]
        Since \( \text{dim}(V) < \text{dim}(W) \), we find that 
        \[  \text{rank}(T) < \text{dim}(W) - \text{nullity}(T) < \text{dim}(W). \]
        Hence, \( T  \) cannot be surjective in this case.
        \end{proof}
    \item[(b)] Prove that if \( \text{dim}(V) > \text{dim}(W) \), then \( T \) cannot be injective.
        \begin{proof}
        Using the Dimension Theorem again and \( \text{dim}(V) > \text{dim}(W) \), we find that
        \[  \text{nullity}(T) = \text{dim}(V) - \text{rank}(T) > \text{dim}(W) - \text{rank}(T) > 0. \]
        This means that \( \text{nullity}(T)  \) can never be zero, otherwise \( T  \) is injective. Hence, \( T  \) can never be injective if \( \text{dim}(V) > \text{dim}(W) \).
        \end{proof}
\end{enumerate}

\subsubsection{Exercise 2.1.20} Let \( V  \) and \( W \) be vector spaces with subspaces \( V_{1}  \) and \( W_{1}  \), respectively. If \( T:V \to W  \) is linear, prove that \( T(V_{1})  \) is a subspace of \( W  \) and that \( \{ x \in V : T(x) \in W_{1} \}   \) is a subspace of \( V  \).
\begin{proof}
First, we prove that \( T(V_{1}) \) is a subspace of \( W  \). Let \( x,y \in T(V_{1}) \) and \( c \in F  \). Since \( V_{1}  \) is a subspace of \( V  \), we know that \( 0_{V} \in V_{1} \), \( x + y \in V_{1} \), and \( cx \in V_{1} \). Observe that \( T(0_{V}) = 0_{W} \) since \( T \) is linear. Hence, \( 0_{W} \in T(V_{1})\). Let \( x,y \in T(V_{1}) \). Since \( T  \) is linear and \( x + y \in V_{1} \), we have \( T(x+y) = T(x) + T(y) \). Hence, we must have \( x + y \in T(V_{1})  \). Now, \( cx \in V_{1} \) implies \( T(cx) = c T(x)    \). Hence, \( cx \in T(V_{1}) \). This tells us that \( T(V_{1}) \) is a subspace of \( W  \).

Now, we show \( S =  \{ x \in V : T(x) \in W_{1} \}  \) is a subspace of \( V  \). Since \( 0_{W} \in W_{1} \) (because \( W_{1}  \) is a subspace of \( W \)) and \( T \) is linear, we have that \( T(0_{V}) = 0_{W}. \) Hence, \( 0_{V} \in S  \). Now, let \( x,y \in S  \). Hence, \( T(x), T(y) \in W_{1} \) implies \( T(x) + T(y) \in W_{1} \) since \( W_{1} \) is a subspace of \( W \). Since \( T  \) is linear, we have \( T(x) + T(y) = T(x+y) \), and so \( x + y \in S \). Now, let \( c \in F  \) and \( x \in S  \). Again, \( W_{1} \) is a subspace so \( c T(x) \in W_{1} \). Thus, \( T \) being linear implies that \( cT(x) = T(cx)\). Hence, \( cx \in S  \). Thus, \( S  \) is a subspace of \( V  \).
\end{proof}

\subsubsection{Exercise 2.1.21} Let \( V  \) be the vector space of sequences described in Example 5 of Section 1.2. Define the functions \( T,U : V \to V  \) by 
\[  T(a_{1}, a_{2}, \dots ) = (a_{2}, a_{3}, \dots ) \ \  \text{and} \ \ U(a_{1}, a_{2}, \dots ) = (0, a_{1}, a_{2}, \dots).  \]
\( T \) and \( U  \) are called the \textbf{left shift} and \textbf{right shift} operators on \( V  \), respectively. 
\begin{enumerate}
    \item[(a)] Prove that \( T  \) and \( U  \) are linear.
        \begin{proof}
        Let \( (x_{n}), (y_{n}) \in V  \) with \( (x_{n}) = (a_{1}, a_{2}, \dots ) \) and \( (y_{n}) = (b_{1}, b_{2}, \dots ) \). Let \( c \in F  \). Then we have 
        \begin{align*}
            T(cx_{n} + y_{n}) &= (ca_{2} + b_{2}, ca_{3} + b_{3}, \dots ) \\
                              &= (ca_{2}, ca_{3}, \dots ) + (b_{2},  b_{3}, \dots ) \\
                              &= c(a_{2}, a_{3}, \dots ) + (b_{2}, b_{3}, \dots) \\
                              &= c T(x_{n}) + T(y_{n}).
        \end{align*}
        Hence, \( T: V \to V   \) is a linear map.

        Now with \( U: V \to V  \) observe that
        \begin{align*}
            U(cx_{n} + y_{n}) &= (0, ca_{1} + b_{1}, ca_{2} + b_{1}, \dots ) \\
                              &= (0, ca_{1}, ca_{2}, \dots) + (0, b_{1}, b_{2}, \dots) \\
                              &= c(0, a_{1}, a_{2}, \dots ) + (0, b_{1}, b_{2}, \dots ) \\ 
                              &= c U(x_{n}) + U(y_{n}).
        \end{align*}
        Hence, \( U: V \to V   \) is a linear map.
        \end{proof}
    \item[(b)] Prove that \( T \) is surjective, but not injective.
        \begin{proof}
        
        \end{proof}
    \item[(c)] Prove that \( U  \) is injective, but not surjective.
        \begin{proof}
        
        \end{proof}
\end{enumerate}
