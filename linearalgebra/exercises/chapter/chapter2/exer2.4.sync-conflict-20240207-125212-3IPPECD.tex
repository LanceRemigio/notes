\section{Invertibility and Isomorphisms}

\subsection*{Exercise 2.4.4} Let \( A  \) and \( B  \) be \( n \times n  \) invertible matrices. Prove that \( AB  \) is invertible and \( (AB)^{-1} = B^{-1} A^{-1} \).
\begin{proof}
Let \( A  \) and \( B  \) be \( n \times n  \) invertible matrices. We need to show that \( (AB)B^{-1}A^{-1} = {I}_{n} \) and \( B^{-1} A^{-1} (AB) = {I}_{n} \). Observe that
\begin{align*}
    B B^{-1} = {I}_{n}   &\iff A(B B^{-1}) = A   \\
                         &\iff  (AB) B^{-1} = A  \\
                         &\iff (AB) B^{-1}A^{-1} = A A^{-1} \\ 
                         &\iff (AB) B^{-1} A^{-1} = {I}_{n}.
\end{align*}
Similarly, we have 
\begin{align*}
     A^{-1} A  = {I}_{n} &\iff B^{-1}(A^{-1} A ) = B^{-1}  \\
                         &\iff (B^{-1} A^{-1}) A = B^{-1} \\ 
                         &\iff (B^{-1} A^{-1}) AB = B^{-1}B \\
                         &\iff (B^{-1}A^{-1}) AB = {I}_{n}.
\end{align*}
This tells us that \( AB  \) is invertible and that  
\[  B^{-1}A^{-1} = (AB)^{-1}. \]
\end{proof}

\subsection*{Exercise 2.4.5} Let \( A  \) be invertible. Prove that \( A^{t}  \) is invertible and \( (A^{t})^{-1} = (A^{-1})^{t} \).
\begin{proof}
Let \( A   \) be an invertible. Observe that \( (A^{t})^{t} = A  \). Thus, we have
\[  {I}_{n} = A A^{-1} = (A^{t})^{t} A^{-1} = \Big( (A^{-1})^{t} A^{t} \Big)^{t}. \]
Taking the transpose on both sides gives us the following equation
\[  (A^{-1})^{t} A^{t} = {I}_{n} \] where \( ({I}_{n})^{t} = {I}_{n} \).
Similarly, we have
\[  {I}_{n} =  A^{-1} A = A^{-1} (A^{t})^{t} = \Big( A^{t} (A^{-1})^{t} \Big)^{t}  \]
which implies
\[ \Big( A^{t} (A^{-1})^{t} \Big)^{t} = {I}_{n}.  \]
Taking the transpose again then gives us
\[  A^{t} (A^{-1})^{t} = {I}_{n}. \]
Therefore, \( A^{t} \) is invertible and that 
% \[  (A^{-1})^{t} = (A^{t})^{-1}. \]
\end{proof}

\subsection*{Exercise 2.4.6} Prove that if \( A  \) is invertible and \( AB = O  \), then \( B = O  \).
\begin{proof}
Suppose that \( A  \) is an invertible matrix. Let \( AB  \) be a defined matrix product where \( AB = O  \). Since \( A  \) is invertible, we have \( A^{-1}A = A A^{-1} = I \). Thus, we have
\begin{align*}
    AB = O &\implies A^{-1}(AB) = A^{-1}O   \\
           &\implies (A^{-1}A)B = O \\
           &\implies IB = O \\
           &\implies B = O. 
\end{align*}
Hence, we have \( B = O  \).
\end{proof}

\subsection*{Exercise 2.4.7} Let \( A  \) be an \( n \times n  \) matrix.
\begin{enumerate}
    \item[(a)] Suppose that \( A^{2} = O  \). Prove that \( A  \) is not invertible.
        \begin{proof}
        Let \( A^{2} = O  \) where \( A  \) is an \( n \times n  \) matrix. Suppose for sake of contradiction that \( A  \) is invertible. Thus, we have \( A A^{-1} = A^{-1} A = {I}_{n}  \). Now, observe that multiplying \( A^{-1}  \) on the left side of \( A^{2} = O  \) produces the following  
        \begin{align*}
            A^{-1} (A^{2}) = A^{-1}O &\implies A^{-1}(AA ) = O       \\
                                     &\implies (A^{-1}A)A = O  \\ 
                                     &\implies {I}_{n}A = O \\ 
                                     &\implies A = O.
        \end{align*}
        \end{proof}
    \item[(b)] Suppose that \( AB = O  \) for some nonzero \( n \times n  \) matrix \( B  \). Could \( A  \) be invertible? Explain.
        \begin{solution}
        The matrix \( A  \) cannot be invertible in this case since the nonzero matrix \( B   \) implies that \( A = O   \) for \( AB = O  \) to hold and that we know that the zero matrix \( O  \) cannot be invertible.
        \end{solution}
\end{enumerate}

\subsection*{Exercise 2.4.8} Prove Corollaries 1 and 2 of Theorem 2.18.
\begin{proof}
    Let \( T: V \to V  \) be linear and let \( \beta = \{ {v}_{1}, {v}_{2}, \dots, {v}_{n} \}   \) be an ordered basis for \( V  \). Suppose \( T  \) is invertible. Then there exists a unique linear transformation denoted by \( T^{-1}: V \to V  \) such that 
    \[
         TT^{-1} = T^{-1}T = {I}_{V}. 
    \]
    By using the Corollary to Theorem 2.11 and part (d) of Theorem 2.12, we must have that
    \[ [T]_{\beta} [T^{-1}]_{\beta} = [T T^{-1}]_{\beta} = [{I}_{V}]_{\beta} = {I}_{n}  \]
    and similarly,
    \[  [T^{-1}]_{\beta} [T]_{\beta} = [T^{-1} T]_{ \beta} = [{I}_{V}]_{\beta} = {I}_{n}. \]
    This tells us that \( [T]_{\beta}  \) is invertible and that 
    \[  [T^{-1}]_{\beta} = ([T]_{\beta})^{-1}. \]

   For the backwards direction, the ordered basis \( \beta \) defined earlier implies that there exists a unique linear transformation \( U: V \to V  \) defined by  
   \[  U({v}_{j}) = \sum_{ i=1  }^{ n } {A}_{ij} {v}_{i} \ \text{for} \ 1 \leq j \leq n. \]
   We need to show that \( U = T^{-1} \). Using the fact that \( [T]_{\beta} \) is invertible, we can write
   \[  [T U]_{\beta} = [T]_{\beta} [U]_{\beta} = {I}_{n} = [{I}_{V}]_{\beta}  \]
   and similarly
   \[  [UT]_{\beta} = [U]_{\beta} [T]_{\beta} = {I}_{n} = [{I}_{V}]_{\beta}.  \]
   But this tells us that \( UT = TU  = {I}_{V} \). So, \( U = T^{-1}  \) and that \( T \) is invertible.
\end{proof}

\begin{proof}
Let \( {L}_{A}: F^{n} \to F^{n} \). Let \( \beta = \{ {v}_{1}, {v}_{2}, \dots, {v}_{n} \}  \) be a basis for \( F^{n} \). By part (a) of {\hyperref[Prop of LMT]{Theorem 2.15}}, we have that \( [{L}_{A}]_{\beta} = A   \). Since \( A \) is invertible, we know that \( {L}_{A} \) must also be invertible by Corollary to Theorem 2.18. Furthermore, we have that \( {L}_{A} {L}_{A^{-1}} = I   \) implies that 
\[  {L}_{A^{-1}} = ({L}_{A})^{-1}.  \]

Conversely, \( {L}_{A} \) invertible implies that \( [{L}_{A}]_{\beta}  \) is invertible by Corollary to Theorem 2.18. By Theorem 2.15, we must have that \( [L_{A}]_{\beta}  = A  \). But this means that \( A  \) is invertible.
\end{proof}

\subsection*{Exercise 2.4.9} Let \( A  \) and \( B  \) be \( n \times n \) matrices such that \( AB  \) is invertible. Prove that \( A  \) and \( B  \) are invertible. Given an example to show that arbitrary matrices \( A  \) and \( B  \) need not be invertible if \( AB  \) is invertible.
\begin{proof}

\end{proof}

\subsection*{Exercise 2.4.10} Let \( A  \) and \( B  \) be \( n \times n  \) matrices such that \( AB = {I}_{n} \). 
\begin{enumerate}
    \item[(a)] Use Exercise 9 to conclude that \(  A  \) and \( B  \) are invertible.
        \begin{solution}
        Apply Exercise 9.
        \end{solution}
    \item[(b)] Prove \( A = B^{-1} \) (and hence \( B = A^{-1} \)).
        \begin{proof}
        Since \( A  \) is an \( n \times n  \) invertible matrix, we can write that
        \begin{align*}
            AB = {I}_{n} &\implies A^{-1}(AB) = A^{-1} \\
                         &\implies (A^{-1}A)B = A^{-1} \\
                         &\implies {I}_{n}B = A^{-1} \\
                         &\implies B = A^{-1}.
        \end{align*}
        Likewise, \( B  \) being an \(  n \times n  \) invertible matrix implies that
        \begin{align*}
            AB = {I}_{n} &\implies (AB)B^{-1} = B^{-1}k \\
                         &\implies A(B B^{-1}) = B^{-1} \\
                         &\implies A {I}_{n} = {B}^{-1} \\
                         &\implies A = B^{-1}.
        \end{align*}
        \end{proof}
    \item[(c)] State and prove analogous results for linear transformations defined on finite-dimensional vector spaces.
        \begin{proof}
        Define \( T: V \to V  \) and \( U: V \to V  \) with \( V  \) being an arbitrary finite-dimensional vector space. Let \( \beta = \{ {v}_{1}, {v}_{2}, \dots ,{v}_{n} \}   \) be an ordered basis for \( V  \) such that \( [T]_{\beta}  \) and \( [U]_{\beta} \) are properly defined and that \( [T]_{\beta} [U]_{\beta} = {I}_{n} \). Then we have
       \begin{center}
           \( [T]_{\beta} = ([U]_{\beta})^{-1} \) and \( [U]_{\beta} = ([T]_{\beta})^{-1} \).
       \end{center} 
       To prove this, we can let \( A = [T]_{\beta} \) and \( B = [U]_{\beta} \) and use parts (a) and (b) to get our desired result.
        \end{proof}
\end{enumerate}

\subsection*{Exercise 2.4.11} Verify that the transformation in Example 5 is injective.
\begin{proof}
Since \( \text{dim}({P}_{3}(\R)) = \text{dim}({M}_{2 \times 2}(\R)) \), we must have that \( {P}_{3}(\R)  \) is isomorphic to \( {M}_{2 \times 2}(\R) \) by Theorem 2.19. By definition, this means that \( T  \) is invertible which further implies that \( T  \) is injective.
\end{proof}

\subsection*{Exercise 2.4.12} Prove Theorem 2.21.
\begin{proof}
    Let \( \text{dim}(V) = n  \). Let \( \beta = \{ {v}_{1}, { v }_{2}, \dots, {v}_{n} \}   \) be an ordered basis for \( V  \). We can show that \( {\phi}_{\beta}(y)  \) is an isomorphism by showing that \( {\phi}_{\beta} \) is a surjective and injective linear map. Suppose \( {\phi}_{\beta}(x) = {\phi}_{\beta} \). Then by definition of \( {\phi}_{\beta} \), we must have that \( [x]_{\beta} = [y]_{\beta} \) which further implies that \( x = y  \). Now, let \( y \in V  \). Since \( \beta \) is an ordered basis for \( V  \), we can find scalars \( {\delta}_{1}, {\delta}_{2}, \dots, {\delta}_{n} \) such that 
    \[  y = \sum_{ i=1  }^{ n } {\delta}_{i} {v}_{i}. \]
    This implies that we have constructed a coordinate vector such that \( [y]_{\beta} = {\phi}_{\beta}(y)\). Hence, \( {\phi}_{\beta} \) is surjective. Thus, we find that \( {\phi}_{\beta} \) is an isomorphism.
\end{proof}

\subsection*{Exercise 2.4.13} Let \( \sim \) mean "is isomorphic to." Prove that \(  \sim \) is an equivalence relation on the class of vector spaces over \( F  \). 
\begin{proof}

\end{proof}

\subsection*{Exercise 2.4.15} Let \( V  \) and \( W  \) be finite-dimensional vector spaces, and let \( T: V \to W  \) be a linear transformation. Suppose that \( \beta \) is a basis for \( V  \). Prove that \( T  \) is an isomorphism if and only if \( T(\beta)  \) is a basis for \( W  \).
\begin{proof}

\end{proof}

\subsection*{Exercise 2.4.16} Let \( B  \) be an \( n \times n  \) invertible matrix. Define \( \Phi: {M}_{n \times n}(F) \to {M}_{n \times n }(F) \) by \( \Phi(A) = B^{-1} A B  \). Prove that \( \Phi  \) is an isomorphism.
\begin{proof}

\end{proof}
