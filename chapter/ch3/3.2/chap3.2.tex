

\section{Open and Closed Sets}

Recall that given any \( \epsilon >0 \), the \( \epsilon-\)neighborhood of \( a \in \R  \) is the set 
\[ V_{\epsilon } = \{ x \in \R : | x - a | < \epsilon  \}.\]
In other words, we have an open interval \( (a -\epsilon , a + \epsilon ) \) or \( a - \epsilon < x < a + \epsilon  \) centered at \( a \) with radius \( \epsilon  \). 

\begin{tcolorbox}
\begin{defn}
    A set \( A \subseteq \R  \) is \textit{open} if for all points \( a \in A \) there exists an \( \epsilon - \)neighborhood \(  V_{\epsilon }(a) \subseteq A \). 
\end{defn}
\end{tcolorbox}

\begin{ex}
\begin{enumerate}
    \item[(i)] The set \( \R  \) is an \textit{open} set because for any \( a \in \R  \), we can pick a \( \epsilon- \)neighborhood \( V_{\epsilon }(a) \) such that \( V_{\epsilon } (a) \subseteq \R \). 
    \item[(ii)] The empty set \( \emptyset \) is an open subset of the real line. This statement is vacuously true due to the definition of an open \textit{set} i.e this set has no interior points to consider so it is true by default. 
    \item[(iii)] Take any \( c,d \in \R  \) and create an open interval as such where 
        \[  (c,d) = \{ x \in \R : c < x < d \}.\] To see why \( (c,d) \) is an \textit{open} set, let \( x \in (c,d) \) be an arbitrary point. Let \(  \epsilon = \min\{ x - c, d -x  \}  \), then we can construct the following \( \epsilon-\)neighborhood where
        \[  V_{\epsilon} = \{ x' \in \R : | x' - x  | < \epsilon \}.  \]
\end{enumerate}
\end{ex}

\begin{tcolorbox}
\begin{thm}
\begin{enumerate}
    \item[(i)] The union of an arbitrary collection open sets is open.
    \item[(ii)] The intersection of a finite collection of open sets is open. 
\end{enumerate}
\end{thm}
\end{tcolorbox}

\begin{proof}
    To prove (i), define \( \{ O_{\lambda} : \lambda \in A \}  \) be a collection of open sets and let \( O = \bigcup_{ \lambda \in A  } O_{\lambda} \). Let \( a  \) be an arbitrary element of \( O \). In order to show that \( O \) is \textit{open}, we need to show that \( V_{\epsilon }(a) \subseteq O \) where \( V_{\epsilon }(a) \) is the \( \epsilon - \)neighborhood. Let \(  a \in O_{\lambda} \) be an arbitrary element. Since we have a collection of open sets 
    \[ \{ O_{\lambda}  : \lambda \in A \}  \]
    we can create a \( \epsilon - \)neighborhood around \( a \in O_{\lambda}  \) for some \( \lambda \in A  \) such that \( V_{\epsilon }(a) \subseteq O_{\lambda}\). But note that \( O_{\lambda} \subseteq O \). Hence, we have that \( V_{\epsilon } (a) \subseteq O = \bigcup_{ \lambda \in A  } O_{\lambda}\). Hence, \( O \) is an \textit{open} set.
    
    To prove (ii), suppose \( O = \bigcap_{ i =1  }^{ N } O_{i}  \). Suppose \( a \in O_{i} \) for all \( 1 \leq  i \leq N \) where \( O_{i} \) is a collection of open sets. Hence, there exists an \( \epsilon- \)neighborhood for every \( O_i  \). We need only one value of \( \epsilon  \) to make this work so define \( \epsilon = \min \{ \epsilon_1, \epsilon_2, \epsilon_3, ... \epsilon_N \}  \).  This means that \[ V_{\epsilon_i} (a) \subseteq V_{\epsilon}(a) \subseteq O_{i} \subseteq O  \] 
Hence, we have 
\[ V_{\epsilon}(a) \subseteq \bigcap_{ i=1 }^{ N } O_i. \] 
\end{proof}

\subsection{Closed Sets} 

\begin{tcolorbox}
\begin{defn}
    A point \( x \) is a \textit{limit point} of a set \( A  \) if every \( \epsilon- \)neighborhood \( V_{\epsilon }(x) \) of \( x \) intersects the set \( A  \) at some point other than \( x \). 
\end{defn}
\end{tcolorbox}
In other words, we have the following intersection 
\[ x \notin V_{\epsilon }(x) \cap A.  \]
This is another way of saying that a sequence approaches of values approaches the limit point \( x \) where \( V_{\epsilon }(x) \) can be thought of as neighborhoods "clustering" around the point \( x \). 

\begin{tcolorbox}
\begin{thm}
A point \( x \) is a limit point of a set \( A \) if and only of \( x = \lim a_n \) for some sequence \( (a_n) \) contained in A satisfying \( a_n \neq x  \) for all \( n \in \N  \). 
\end{thm}
\end{tcolorbox}

\begin{proof}
    \( (\Rightarrow) \) Let \( V_{\epsilon }(x) \) be an \( \epsilon-\)neighborhood around \( x \). We want to show that \( \lim a_n = x  \) for some sequence \( (a_n) \) contained in \( A  \) satisfying \( a_n \neq x  \) for all \( n \in \N  \). By definition, 
    \[ V_{\epsilon}(x) = \{ x \in \R : | a_n - x  | < \epsilon  \}.  \]
    Let \( \epsilon = \frac{ 1 }{ n }  \). Since \( x  \) is a \textit{limit point}, for each \( n \in \N  \), we can pick any point 
    \[ a_n \in V_{1/n}(x) \cap A. \]Then we have 
    \[  | a_n - x | < \frac{ 1 }{ n }  \]
    which is equivalent to 
    \[ x - \frac{ 1 }{ n } < a_n < x + \frac{ 1 }{ n }. \] 
    By the Algebraic limit theorem and Squeeze Theorem, we have that \( (a_n) \to x  \) where \( a_n \neq x  \) for all \( n \in \N  \). 

    \( (\Leftarrow) \) Suppose \( x = \lim a_n \) for some sequence \( (a_n) \) contained in \( A \) satisfying \( a_n \neq x  \) for all \( n \in \N  \). We want to show the converse. Let \( \epsilon > 0  \). Then By definition of \( \lim a_n = x  \), there exists \( N \in \N  \) such that for any \( n \geq N \), we have  
    \[  | a_n - x  | < \epsilon.  \]
    But this is also the definition of an \( \epsilon-\)neighborhood. Hence, \( a_n \neq x  \) for all \( n \in \N  \) and \( x \in A  \) is a limit point implies 
    \[ V_{\epsilon }(x) \cap A \]
    for all \( \epsilon- \)neighborhoods. 
\end{proof}

Keep in mind that \( a \in A  \) means that there is a sequence in \( A  \) such that \( a_n = {a,a,a, ...} \) which is uninteresting for the most part. We can distinguish \textit{limit points} from \textit{isolated points}.  

\begin{tcolorbox}
\begin{defn}
A point \( a \in A  \) is an \textit{isolated point} of \( A \) if it is not a \textit{limit point} of \( A \).  
\end{defn}
\end{tcolorbox}

Remember that an isolated point is always in the set \( A  \), but a limit point can be sometimes be outside of the set \( A \). An example of this is the endpoint of an open interval. A sequence can approach the endpoint where \( a_n \neq x  \) for all \( n \in \N  \) but \( x  \) is not in the set.  

\begin{tcolorbox}
\begin{defn}
A set \( F \subseteq \R  \) is \textit{closed} if it contains its limit points. 
\end{defn}
\end{tcolorbox}

In other words, can say that a set \( A  \) is closed if sequences contained in \(  A  \) converge to their limits that are within the set \( A  \). 

\begin{tcolorbox}
\begin{thm}
A set \( F \subseteq \R \) is closed if and only if every Cauchy sequence contained in \( F \) has a limit that is also an element of \( F \). 
\end{thm}
\end{tcolorbox}

\begin{proof}
    Suppose \( F \subseteq \R  \) is closed. Let \( x \in F \) be a limit point. Let \( (x_n) \) be a Cauchy sequence contained in \( F  \). By the Cauchy Criterion, \( (x_n) \) converges to \( x \in F  \). 
\end{proof}

\begin{ex}
\begin{enumerate}
    \item[(i)] Consider the set 
        \[ A = \Big\{ \frac{ 1 }{ n }  : n \in \N \Big\}.   \]
        Let's show that each point of \( A  \) is isolated. We can show that each point of \( A  \) is isolated. Given \( \frac{ 1 }{ n }  \in A \). Choose \( \epsilon = \frac{ 1 }{ n }  - \frac{ 1 }{ (n+1) }. \) Then, 
        \[ V_{\epsilon } (1/n) \cap A = \Big\{ \frac{ 1 }{ n } \Big\}. \]
        It follows from Definition 3.2.4 that \( \frac{ 1 }{ n }  \) is not a limit point and so is isolated. Although all of the points of \( A  \) are isolated, the set \( A  \) does have only one limit point 0. The reason for this is can be explained by the very definition of \( A  \) where \( 0 \notin A  \). Since the limit of \( A \) is not contained in \( A \), we can say that \( A  \) is not closed. The set \( F = A \cup \{ 0 \}  \) is an example of a closed set and is called the closure of \( A \).   
    \item[(ii)] Let's prove that a closed interval 
        \[ [c,d] = \{ x \in \R : c \leq c \leq d \}  \]
        is a closed set using Definition 3.2.7. If \( x \) is a limit point of \( [c,d] \), then by Theorem 3.2.5 there exists \( (x_n) \subseteq [c,d] \) with \( (x_n) \to x  \). Since \( (x_n) \to x  \), we can use the Order Limit Theorem to say that 
        \[ c \leq x_n \leq d \iff c \leq x \leq d. \]
        This means \( x \in [c,d] \) which proves that \( [c,d] \) is a closed set. 
    \item[(iii)] Consider the set \( \Q \subseteq \R  \) of rational numbers. An interesting property of \( \Q \) is that all of its limit points is actually all of \( \R  \). To see why this is so, let us have \( y \in \R  \) be arbitrary and construct \( V_{\epsilon }(y) \) such that we have the open set \( (y- \epsilon , y + \epsilon ) \). Since \( \Q  \) is dense in \( \R \), there exists \( x \in \Q \) where \( x \neq y \) such that \(x \in (y-\epsilon , y+\epsilon ) \). Hence, \( y  \) is a limit point of \( \Q \).  
\end{enumerate}
\end{ex}

We can actually restate the Density Property from the first chapter by saying the following:

\begin{tcolorbox}
\begin{thm}
For every \( y \in \R  \), there exists a sequence of rational numbers that converges to \( y \). 
\end{thm}
\end{tcolorbox}

\begin{proof}
    Let \( y \in \R  \) and let \( \epsilon = \frac{ 1 }{ n }  \). Create the following \( \epsilon- \)neighborhood \( (y-\frac{ 1 }{ n }, y + \frac{ 1 }{ n } ) \). Since the end points of this \( \epsilon- \)neighborhood are real numbers, we can find a sequence of rational numbers \( (x_n) \subseteq (y - \epsilon , y + \epsilon  )\) by the Density of \( \Q \) in \( \R \) such that 
    \[ y - \frac{ 1 }{ n } < x_n < y + \frac{ 1 }{ n }. \]
    By the Squeeze Theorem, we can write that \( (x_n) \to y \) where \( x_n \neq y \) for all \( n \in \N  \). 
\end{proof}

\subsection{Closure}

\begin{tcolorbox}
\begin{defn}
    Given a set \( A \subseteq \R  \), let \( L \) be the set of all limit points of \( A \). The closure of \( A \) is defined to be the \( \bar{A} = A \cup L \). 
\end{defn}
\end{tcolorbox}

\begin{ex}
\begin{enumerate}
    \item[(i)] Consider \( A = \{ 1/n : n \in \N  \}  \), then the \textit{closure of} \( A \) is 
    just 
    \[ \bar{A} = A \cup \{ 0 \}. \]
\item[(ii)] In the last example, \( y \notin (y- \epsilon , y + \epsilon ) \) where \( y \in \R  \) gurantees that the closure of \( \Q  \) in \( \R  \); that is, \( \bar{\Q} = \R  \). 
\item[(iii)] If \( A \) is an open interval \( (a,b) \), then the closure is just \( \bar{A} = [a,b] \); that is, \( \bar{A} = A \cup \{ a,b \}  \) where \( a,b \in \R  \) are the endpoints of the set \( (a,b) \). 
\item[(iv)]If \( A \) is a closed interval then the closure is just \( \bar{A} = A  \). The obvious conclusion from this is that closed intervals are always closed sets. 
\end{enumerate}
\end{ex}

\begin{tcolorbox}
\begin{thm}
    For any \( A \subseteq \R  \), the closure of \( \bar{A} \) is a closed set and is the smallest closed set containing \( A \). 
\end{thm}
\end{tcolorbox}

\begin{proof}
    Since \( L \) is the set of limit points of \( A \), it follows immediately that \( \bar{A} \)
    contains its limit points of \( A \). The problem here is that taking the union of \( A \) and \( L \) could produce some new limit points. 
    \begin{center}
        The details are in exercise 3.2.7
    \end{center}
    Hence, any closed set containing \( A \) must contain \( L \) as well. Hence, we have 
    \( \bar{A} = A \cup L  \) is the smallest closed set containing \( A \). 
\end{proof}

\subsection{Complements}

The notions of open and closed imply that they are not antonyms of each other. Just because a set is not open, does not immediately imply that it is closed. We can see this in action by considering the half-open interval 
\[ (c,d] = \{ x \in \R : c \leq x \leq d \}  \]
as being neither open nor closed. Furthermore, \( \R \) and \( \emptyset \) are both simultaneously open and closed at the same time. Luckily, these are the only two sets that exhibit this confusing property. We do have a relationship between open and closed sets however.   

Recall that the complement of a set \( A \subseteq \R  \) is defined to be the set
\[ A^c = \{ x \in \R : x \notin A  \} \]
which describes all of the elements that are not in \( A \). 

\begin{tcolorbox}
\begin{thm}
A set \( A \) is open if and only if \( A^c \) is closed. Likewise, a set \( B \) is closed if and only if \( B^c \) is open. 
\end{thm}
\end{tcolorbox}

\begin{proof}
    Suppose \( A \subseteq \R  \) is an open set. We want to show that \( A^c \) is a closed set. Let \( x  \) be a limit point of \( A^c \). Hence, there exists a sequence \( (x_n) \) such that \( \lim x_n = x \) where \( x_n \neq x  \) for all \( n \in \N  \). By definition of \( \lim x_n = x  \), there is an \( \epsilon- \)neighborhood \( V_{\epsilon } (x) \), but this means that \( x \notin A \) and must be in \( A^c \) since every \( \epsilon - \)neighborhood of \( x \) intersects \( A \) at some point other than \( x \). Hence, we have \( x \in O^c \). 
    
    For the converse statement, we assume \( A^c \) is a closed set. We want to show that \( A \) is open. Hence, let \( x \in A \). Since \( x \in A \), \( x \) is not a limit point of \( A^c \) and \( A^c \) is a closed set, there must exist an \( \epsilon - \)neighborhood such that \( x \notin V_{\epsilon }(x) \cap A^c \). This means \( x \in A \) and so \( V_{\epsilon } \subseteq A \). Hence, \( A \) is an open set.  
    The second statement follows quickly when taking the complement of each going in each direction.
\end{proof}

\begin{tcolorbox}
\begin{thm}
\begin{enumerate}
    \item[(i)] The union of a finite collection of closed sets is closed. 
    \item[(ii)] The intersection of an arbitrary collection of closed sets is closed. 
\end{enumerate}
\end{thm}
\end{tcolorbox}

\begin{proof}
    De Morgan's Laws state that for any collection of sets \( \{ E_{\lambda} : \lambda \in \Lambda \}  \) it is true that 
    \[ \Big( \bigcup_{\lambda \in \Lambda} E_{\lambda}\Big)^c = \bigcap_{\lambda \in \Lambda} E_{\lambda}^c ~~ \text{ and } ~~ \Big( \bigcap_{\lambda \in \Lambda} E_{\lambda}\Big)^c = \bigcup_{\lambda \in \Lambda} E_{\lambda}^c. \]
\end{proof}


\subsection{Definitions}

\begin{tcolorbox}
\begin{defn}
    A set \( A \subseteq \R  \) is \textit{open} if for all points \( a \in A \) there exists an \( \epsilon - \)neighborhood \(  V_{\epsilon }(a) \subseteq A \). 
\end{defn}
\end{tcolorbox}

\begin{tcolorbox}
\begin{thm}
\begin{enumerate}
    \item[(i)] The union of an arbitrary collection open sets is open.
    \item[(ii)] The intersection of a finite collection of open sets is open. 
\end{enumerate}
\end{thm}

\end{tcolorbox}
\begin{tcolorbox}
\begin{defn}
    A point \( x \) is a \textit{limit point} of a set \( A  \) if every \( \epsilon- \)neighborhood \( V_{\epsilon }(x) \) of \( x \) intersects the set \( A  \) at some point other than \( x \). 
\end{defn}
\end{tcolorbox}


\begin{tcolorbox}
\begin{thm}
A point \( x \) is a limit point of a set \( A \) if and only of \( x = \lim a_n \) for some sequence \( (a_n) \) contained in A satisfying \( a_n \neq x  \) for all \( n \in \N  \). 
\end{thm}
\end{tcolorbox}

\begin{tcolorbox}
\begin{defn}
A point \( a \in A  \) is an \textit{isolated point} of \( A \) if it is not a \textit{limit point} of \( A \).  
\end{defn}
\end{tcolorbox}

\begin{tcolorbox}
\begin{defn}
A set \( F \subseteq \R  \) is \textit{closed} if it contains its limit points. 
\end{defn}
\end{tcolorbox}

\begin{tcolorbox}
\begin{thm}
A set \( F \subseteq \R \) is closed if and only if every Cauchy sequence contained in \( F \) has a limit that is also an element of \( F \). 
\end{thm}
\end{tcolorbox}

\begin{tcolorbox}
\begin{thm}
For every \( y \in \R  \), there exists a sequence of rational numbers that converges to \( y \). 
\end{thm}
\end{tcolorbox}

\begin{tcolorbox}
\begin{defn}
    Given a set \( A \subseteq \R  \), let \( L \) be the set of all limit points of \( A \). The closure of \( A \) is defined to be the \( \bar{A} = A \cup L \). 
\end{defn}
\end{tcolorbox}


\begin{tcolorbox}
\begin{thm}
    For any \( A \subseteq \R  \), the closure of \( \bar{A} \) is a closed set and is the smallest closed set containing \( A \). 
\end{thm}
\end{tcolorbox}

\begin{tcolorbox}
\begin{thm}
A set \( A \) is open if and only if \( A^c \) is closed. Likewise, a set \( B \) is closed if and only if \( B^c \) is open. 
\end{thm}
\end{tcolorbox}


\subsection{Exercises}


\subsubsection{Exercise 3.2.2} Let 
\[ A = \Big\{ (-1)^n + \frac{ 2 }{ n } : n = 1,2,3,... \Big\}   \]
and
\[ B = \{ x \in \Q : 0 < x < 1 \}.  \]
Answer the following questions for each set:
\begin{enumerate}
    \item[(a)] What are the limit points? 
        \begin{proof}[Solution]
        The limit points of \( A \) is \( L = \{ -1, 1  \}  \) and the limit points of \( B \) is \( L = \{ 0,1 \}  \).
        \end{proof}
    \item[(b)] Is the set open? Closed? 
        \begin{proof}[Solution]
            The sets \( A  \) and \( B \) are not closed since their limit points are not contained and open since we can create \( V_{\epsilon }(x) \subseteq A \) while \( B \) is not open since \( V_{\epsilon }(x) \not \subseteq B \) for every \( x \in \Q \) however small \( \epsilon  \) is.  
        \end{proof}
    \item[(c)] Does the set contain any isolated points? 
        \begin{proof}[Solution]
        From part (b), since we cannot find any points near each \( x \in \Q  \) in \( B \), we have that all the points of \( B  \) are isolated points. 
        \end{proof}
    \item[(d)] Find the closure of the set.
        \begin{proof}[Solution]
            The closure of sets \( A \) and \( B \) are \( \overline{A} = A \cup \{-1,1\}   \) and \( \overline{B} = B \cup \{ 0,1 \}  \). 
        \end{proof}
\end{enumerate}




\subsubsection{Exercise 1.2.13}
Show De Morgan's Laws where \( \{ A_i : 1 \leq i \leq n \}  \) is a collection of sets such that
\begin{align*}
    \Big( \bigcup_{i = 1}^{n} A_i \Big)^c &= \bigcap_{ i=1 }^{ n } A_i^{c} \tag{1} \\
    \Big( \bigcap_{i = 1}^{n} A_i \Big)^c &= \bigcup_{ i=1 }^{ n } A_i^{c} \tag{2} \\
\end{align*}
for any finite \( n \in \N \). 
\begin{proof}
Our goal is to show that both inclusions hold for (1) and (2). Our first step is to induct on \( n\in \N \) to show that 
\[ \Big( \bigcup_{i=1}^{n} A_i \Big)^c \subseteq \bigcap_{i=1}^{n} A_i^c. \tag{1}\]
Let \( n = 1 \) be the base case. It follows immediately that \( A_1^c \subseteq A_1^c \). Let \( n = 2 \), then it follows that \( (A_1 \cup A_2)^c \subseteq A_1^c \cap A_2^c \) by exercise 1.2.5. For the other inclusion, we also have \( A_1^c \cap A_2^c \subseteq (A_1 \cup A_2)^c \). Now suppose (1) holds for \( 1 \leq n \leq k-1 \). We want to show that (1) holds for \( k \). Let 
\[ A' = \bigcup_{ n=1 }^{ k-1 } A_n  \]
then consider the following 
\[  \Big( \bigcup_{ n=1 }^{ k } A_n \Big)^c = \Big( A_k \cup \Big[ \bigcup_{ n=1 }^{ k-1 } A_n \Big] \Big)^c = ( A_k \cup A')^c \]
Let \( x \in (A_{k} \cup A')^c \), then we know that \( x \notin (A_k \cup A') \). This means that \( x \notin A_k  \) and \( x \notin A'\). Hence, we have \( x \in A_k^c \) and \( x \in (A')^c \); that is, 
\begin{align*}
    (A_k \cup A')^c &\subseteq A_k^c \cap (A')^c  \\
                    &= A_k^c \cap \Big( \bigcup_{n=1}^{k-1} A_n  \Big)^c. \\
                    &\subseteq A_k^c \cap \Big( \bigcap_{n=1}^{k-1} A_n^c \Big) \\
                    &= A_k \cap (A_{k-1} \cap ... \cap A_1) \\
                    &= \bigcap_{n=1}^{k} A_n^c.
\end{align*}
Hence, we have 
\[ \Big( \bigcup_{i=1}^{n} A_i \Big)^c \subseteq \bigcap_{i=1}^{n} A_i^c.\]
For the other inclusion, suppose the containment 
\[ \bigcap_{n=1}^{k-1}A_k^c \subseteq \Big( \bigcup_{n=1}^{k-1} A_k \Big)^c \tag{2} \]
holds for \( 1 \leq n \leq k -1  \). We want to show that (2) holds for \( k  \). Consider the finite intersection
\[ \bigcap_{ n=1 }^{ k } A_n^c = A_k^c \cap \Big( \bigcap_{ n=1 }^{ k-1 } A_{n}^c \Big).  \]
If we know that \( x \notin \bigcap_{ n=1 }^{ k-1 } A_{n} \) and \( x \notin A_k \) then \( x \notin \Big( A_k \cup \Big( \bigcap_{ n=1 }^{ k-1 } A_n \Big) \Big) \). Hence, using our inductive hypothesis, we have
\begin{align*}
    \bigcap_{ n=1 }^{ k } A_n^c &= A_k^c \cap \Big( \bigcap_{ n=1 }^{ k-1 } A_{n}^c \Big)  \\
                                &\subseteq A_k^c \cup \Big( \bigcap_{ n=1 }^{ k-1 } A_n \Big)^c \\ 
                                &\subseteq A_k^c \cup \Big( \bigcup_{ n=1 }^{ k-1 } A_n^c \Big) \\
                                &= \Big( \bigcup_{ n=1 }^{ k } A_n \Big)^c
\end{align*}
Since both containments hold, we must have 
\[ \Big( \bigcup_{ n=1 }^{ k }A_n  \Big)^c = \bigcap_{ n=1 }^{ k } A_n^c.  \]
The proof to the other equation is similar. 
\end{proof}







\subsubsection{Exercise 3.2.4} Let \( A  \) be nonempty and bounded above so that \( s = \sup A \) exists. 
\begin{enumerate}
    \item[(a)] Show that \( s \in \overline{A} \).
        \begin{proof}
            Let \( A \neq \emptyset \) and bounded above. Since \( s = \sup A  \) exists we can let \( \epsilon > 0  \) such that for some \( \alpha \in A  \), we have \( s - \epsilon < \alpha \). Our goal is to show that \( s \in \overline{A} \). Let \( (a_n) \) be a sequence in \( A \) such that \( a_n \neq s \) for all \( n \in \N \). Let \( \epsilon = 1/n \) such that 
            \[ s - \frac{ 1 }{ n } < \alpha \leq a_n \leq s. \]
            By the Squeeze Theorem, we have \( \lim a_n = s = \sup A \). This means \( s = \sup A \) is a limit point where \( L = \{ s \}  \) such that \( \overline{A} = A \cup L  \). Hence, \( s \in \overline{A} \).

        \end{proof}
    \item[(b)] Can an open set contain its supremum? 
        \begin{proof}[Solution]
        An open set \( A \) cannot contain its supremum, which is a limit point in part (a), since otherwise \( A \) would be a closed set.  
        \end{proof}
\end{enumerate}




\subsubsection{Exercise 3.2.5} Prove Theorem 3.2.8:
Show that a set \( F \subseteq \R  \) is closed if and only if if every Cauchy sequence contained in \( F \) has a limit that is also an element of \( F \). 
\begin{proof}
\( (\Rightarrow) \) Let \( F \subseteq \R  \) be a closed set. Let \( x  \) be a limit point and let \( (x_n) \) be a Cauchy sequence be arbitrary. Since \( F \) is a closed set, the limit point \( x \in F \); that is, \( \lim x_n = x \in F  \) where \( x_n \neq x  \) for all \( n \in \N \). 

\( (\Leftarrow) \) Let \( F \subseteq \R  \). We want to show that \( F  \) is closed. Let \( (x_n) \) be a Cauchy sequence contained in \( F \) such that \( \lim x_n = x \in F \). Note that \( x_n \neq x  \) for all \( n \in \N \). Since all the limit points of \( F \) are contained in \( F \), then \( F \) must be a closed set. 
\end{proof}

\subsubsection{Exercise 3.2.7} Given \( A \subseteq \R  \), let \( L  \) be the set of all limit points of \( A \). 
\begin{enumerate}
    \item[(a)] Show that the set \( L  \) is closed. 
        \begin{proof}
            Let \( L \) be the set of limit points of \( A \), and suppose that \( x  \) is a limit point of \( L \). Our goal is to show that \( x  \) is a limit point of \( A \). Let \( V_{\epsilon }(x) \) be arbitrary. Let \( \epsilon > 0  \), then we know that \( V_{\epsilon }(x) \) intersects \( L \) at a point \( \ell \in L \) where \( \ell \neq x  \). Choose \( \epsilon' > 0  \) small enough so that \( V_{\epsilon'}(\ell) \subseteq V_{\epsilon }(x)\) and \( x \notin V_{\epsilon '}(\ell) \). Since \( \ell \in L \), we know that \( \ell  \) is a limit point of \( A \)m and therefore \( x \) is a limit point of \( A \) and thus an element of \( L \). 
        \end{proof}
    \item[(b)] Argue that if \( x  \) is a limit point \( A \cup L  \), then \( x  \) is a limit point of \( A \). Use this observation to furnish a proof for Theorem 3.2.12. 
        \begin{proof}
            Suppose \( x  \) is a limit point of \( \overline{A} = A \cup L  \). By definition, we can construct \( V_{\epsilon}(x) \) such that \( V_{\epsilon }(x) \) intersects \( a \in \overline{A} \) where \( a \neq x  \). This means \( x \in A \) or \( x \in L \). If \( x \in A \), then \( V_{\epsilon }(x) \) intersects every point \( a \in A \) where \( x \neq a \). Hence, \( x  \) is a limit point of \( A \). If \( x \in L \), then we can use the same argument from above to construct an \( \epsilon' > 0  \) small enough so that \( V_{\epsilon'}(\ell) \subseteq V_{\epsilon }(x) \) where \( x \notin V_{\epsilon'}(\ell) \). Since \( \ell \in L  \) is a limit point, this means that that \( V_{\epsilon'}(\ell) \) intersects \( A \). But since \( x \notin V_{\epsilon'}(\ell) \) this means that \( V_{\epsilon }(x) \) intersects \( A \) at every point of \( A \) that is not \( x \). Hence, \( x \) is a limit point of \( A \). 
        \end{proof}
\end{enumerate}


\subsubsection{Exercise 3.2.9}

A proof for De Morgan's Laws in the case of two sets is outlined in Exercise 1.2.5. The general argument is similar. 

Now, provide the details for the proof of Theorem 3.2.14. 
    \begin{proof}
    To prove part (i), suppose we have a finite collection of open sets where 
    \[ \{ E_{i} : 1 \leq  i \leq N \}.   \]
    Since \( E_i  \) closed, their complements \( E_i^c \) is open. Since the finite intersection of open sets is open, we have that 
    \[\Big(  \bigcup_{ i=1 }^{ N } E_{i} \Big)^c = \bigcap_{ i=1 }^{ N  } E_{i}^c  \]
    is open. But this means that 
    \[ \bigcup_{i=1}^{N} E_{i} \]
    is closed. 

    To prove part (ii), suppose we have an arbitrary collection of closed sets 
    \[ \{ E_\lambda : \lambda \in \Lambda \}.  \]
    Since \(E_{\lambda}\) is closed, we have that their complement \( E_{\lambda}^c \) is open. But this means that the union 
    \[ \bigcup_{\lambda \in \Lambda} E_{\lambda}^c = \Big( \bigcap_{ \lambda \in \Lambda } E_{\lambda}  \Big)^c \tag{1} \]
    is also open. But since the complement of the intersection of (1) is open, we have 
    \[ \bigcap_{ \lambda \in \Lambda } E_{\lambda}  \]
    is closed. 
    \end{proof}

\subsubsection{Exercise 3.2.11} 
\begin{enumerate}
    \item[(a)] Prove that \( \overline{A \cup  B} = \overline{A} \cup \overline{B} \)
        \begin{proof}
        We want to show the following containments 
        \begin{align*}
            \overline{A \cup B} &\subseteq \overline{A} \cup \overline{B}, \\
            \overline{A \cup B } &\supseteq \overline{A} \cup \overline{B}
        \end{align*}
        Suppose \( x \in \overline{A \cup B} \). Then \( x \) is a limit point of \( A \cup B \). Hence, either \( x  \in A \) or \( x \in B \). But \( x  \) is a limit point so there exists \( V_{\epsilon}(x) \) that either intersects \(a \neq x \in A \) or \( b \neq x \in B \). But this means that \( x  \) is a limit point of \( A \) or \( B \). Hence, \( x \in \overline{A} \cup \overline{B} \). 
        Suppose \( x \in \overline{A} \cup \overline{B} \). Then either \( x \in \overline{A} \) or \( x \in \overline{B} \). But this means that \( x  \) is a limit point of \( A  \) or \( B \) which imply that \( V_{\epsilon}(x) \) intersects elements of both \( A  \) or \( B \) that is not \( x \). Hence, \( x  \) must be a limit point of either \( A  \) or \( B \). Hence, \( x \in \overline{A \cup B} \).
        Since both containments are true, we have that \( \overline{A \cup B} = \overline{A} \cup \overline{B} \). 
        \end{proof}
    \item[(b)] Does this result about closures extend to infinite unions of sets? 
        \begin{proof}[Solution]
            No this result does not extend to infinite unions. Consider the counter-example where we have a closed set \( H_n = [1/n, 1] \subseteq \R  \) where 
            \begin{align*}
                \bigcup_{ i=1 }^{ \infty  } \overline{H_n} &= (0, 1] \tag{1} \\
                \overline{\bigcup_{ i=1 }^{ \infty  } H_n}  &= [0,1] \tag{2}
            \end{align*}
            It is clear that (1) and (2) are not the same sets. 
        \end{proof}
        
\end{enumerate}


\subsubsection{Exercise 3.2.14} A dual notion to the closure of a set is the interior of a set. The \textit{interior} of \( E \) is denoted \( E^{\circ} \) and is defined as 
\[ E^{\circ} = \{ x \in E: \exists V_{\epsilon}(x) \subseteq E  \}.  \]
Results about closures and interiors posses a useful symmetry. 
\begin{enumerate}
    \item[(a)] Show that \( E \) is closed if and only if \( \overline{E} = E  \). Show that \( E \) is open if and only if \( E^{\circ} = E  \).
        First we show that first statement.
        \begin{proof}
            \( (\Rightarrow) \) Suppose \( E \) is closed. We want to show that \( \overline{E} = E \); that is, we want to show that \( \overline{E} \subseteq E  \) and \( \overline{E} \supseteq E  \). Note that the first containment follows immediately since \( \overline{E} \) is the smallest set containing \( E \). 
            Now we want to show that \( \overline{E} \supseteq E \). Let \( x \in E \) be  a limit point. Since \( x \) is a limit point and \( E \) is a closed set, we know that \( x \) must be contained in \( E \). This means that set of limit points \( L \) of \( E \) must where \( x \in L \) implies that \( x \in \overline{E} \). Hence, \( \overline{E} = E \).

            \( (\Leftarrow) \) It follows that \( E \) is closed since \( \overline{E} \) contains its limit points and that \( \overline{E} = E \). 
        \end{proof}
        Now we show the second statement
        \begin{proof}
           \( (\Rightarrow) \) Suppose \( E \) is an open set. We must show the following two containments: 
            \( E^{\circ} \subseteq E  \) and \( E^{\circ} \supseteq E \). We show the first containment. Let \( x \in E^{\circ} \) be arbitrary. Then there exists \( V_{\epsilon }(x) \) such that \( V_{\epsilon } (x) \subseteq E \). Hence, \( x \in E  \) so we have \( E^{\circ} \subseteq E  \). Now we show that second containment. Since \( E \) is an open set, let \( x \in E \) be arbitrary such that there exists \( V_{\epsilon }(x) \subseteq E \). But this is by definition the interior of \( E \) so we must have \( x \in E^{\circ} \). 
           
            \( (\Leftarrow) \) Suppose \( E = E^{\circ} \). We want to show that \( E \) is an open set. Let \( x \in E  \) be arbitrary. Since \( E = E ^{\circ} \), there exists \( V_{\epsilon }(x) \) such that \( V_{\epsilon }(x) \subseteq E \). But this means \( E \) is an open set by definition.  
        \end{proof}
    \item[(b)] Show that \( \overline{E}^c = (E^c)^{\circ} \), and similarly that \( (E^{\circ})^c = \overline{E^c} \). 

        Show that \( \overline{E}^c = (E^c)^{\circ} \)
        \begin{proof}
            We want to show that first equation; that is, we want to show the following two containments \( \overline{E}^c \subseteq (E^c)^{\circ} \) and \( \overline{E}^c \supseteq (E^c)^{\circ} \). 
            First we show the former containment. Let \( x \in \overline{E}^c \) be arbitrary. If \( x \notin \overline{E} \), then \( x  \) is not a limit point of \( E \) and \( x \notin E \).  But this means that \( x \in (E^c)^{\circ} \) and hence, \( \overline{E}^c \subseteq (E^c)^{\circ} \). 
            Now we show the second containment. Let \( x \in (E^c)^{\circ} \) be arbitrary. There exists \( V_{\epsilon }(x) \subseteq E^c \). We can be sure that \( x\) is not a limit point of \( \overline{E} \) since \( \overline{E} \) contains all its limit points. Hence, we must have \( x \in \overline{E}^c \). Hence, we have \( \overline{E}^c = (E^c)^{\circ} \).   
        \end{proof}
        Now we show \( (E^{\circ})^c = \overline{E^c}\)
        \begin{proof}
            We want to show the following two containments; namely, \( (E^{\circ})^c \subseteq \overline{E^c} \) and \( \overline{E^c} \subseteq (E^{\circ})^c \).

            We start with the first containment. Let \( x \in (E^{\circ})^c \) be arbitrary. This means \( x \notin E^{\circ} \) and hence for all \( \epsilon - \)neighborhoods of \( x \), we have \( V_{\epsilon }(x) \not\subseteq E \). Our goal is to show that \( x \in \overline{E^c} \). If \( x  \) is not a limit point of \( E^c \), then we just have \( x \in E^c \) and hence \( x \in \overline{E^c} \). Otherwise, we can prove \( x  \) is a limit point of \( E^c \). Suppose \( L \) is the set of limit points of \( E^c \). Let \( \epsilon' > 0   \) be as small as possible and \( \ell \in L  \) such that \( V_{\epsilon '}(\ell) \subseteq V_{\epsilon }(x) \) where \( x \notin V_{\epsilon'}(\ell) \). Since \( \ell \) is a limit point of \( E^c \), \( V_{\epsilon '}(\ell) \) intersects \( E^c \). But this also means \( V_{\epsilon }(x) \) intersects points of \( E^c \) that is not \( x  \). Hence, \( x  \) is a limit point of \( E^c \) and thus \( x \in \overline{E^c} \).


            Now let \( x \in \overline{E^c} \) be arbitrary. Then either \( x \in E^c \) or \( x \in L  \) where \( L \) denotes the set of limit points of \( E^c \). If \( x \in E^c \), then surely \( x \notin E^{\circ} \). Hence, \( x \in (E^{\circ})^c \). If \( x \in L \) and \( \overline{E} \) is a closed set, then \( x  \) cannot be in \( E^{\circ} \). Hence, \( x \) must be in \( (E^{\circ})^c \). Hence \( \overline{E^c} \subseteq (E^{\circ})^c \)

        \end{proof}
\end{enumerate}













