\section{Sets of Discontinuity} 

\begin{tcolorbox}
\begin{defn}
Given a function \( f: \R \to \R  \), we call the set \( D_f \subseteq \R  \) to be the set of points where the function \( f  \) fails to be continuous.
\end{defn}
\end{tcolorbox}

Some examples of sets of discontinuous points are
\begin{enumerate}
    \item[(a)] \( D_g = \R  \) in the case for Dirichlet's function, 
    \item[(b)] and \( D_{h} = \R \setminus \{ 0 \}  \) in the case of the modified Dirichlet's function, and 
    \item[(c)] lastly, \( D_{t} = \Q  \) for Thomae's function \( t(x)  \).
\end{enumerate}

We can always write the set of discontinuous points for a function \( D_f  \) as a countable union of closed sets. For monotone functions, these closed sets can taken as single points.

\subsection{Monotone Functions}

\begin{tcolorbox}
\begin{defn}
A function \( f: A \to \R  \) is \textit{increasing} on \( A \) if \( f(x) \leq f(y)  \) whenever \( x < y  \) and \textit{decreasing} if \( f(x) \geq f(y)  \) whenever \( x < y  \) in \( A  \). A \textit{monotone} function is one that is either increasing or decreasing.
\end{defn}
\end{tcolorbox}

The continuity of a function at a point \( c  \) means that \( \lim_{ x \to c } f(x) = f(c) \). Discontinuities occur when right-hand limits do not equal the left-hand limits approaching \( c  \). 

\begin{tcolorbox}
\begin{defn}
Given a limit point \( c  \) of a set \( A  \) and a function \( f: A \to \R  \), we write
\[ \lim_{ x \to c^{+}   }  f(x) = L \]
if for all \( \epsilon > 0  \), there exists \( \delta > 0  \) such that \( | f(x) -  L  | < \epsilon  \) whenever \( 0 < x - c < \delta  \). Equivalently, in terms of sequences, \( \lim_{ x  \to c^{+} } f(x) = L  \) if \( \lim f(x_n) = L  \) for all sequences \( (x_n)  \) satisfying \( x_n >  c  \) and \( \lim (x_n) = c  \).
\end{defn}
\end{tcolorbox}

\subsubsection{Exercise 4.6.3} State a similar definition for the left-hand limit
\[  \lim_{ x \to c^{-} } f(x) = L. \]
\begin{proof}[Solution]
    We say \( \lim_{ x \to c^{-}  } f(x) = L  \) if for all \( \epsilon > 0  \), there exists \( \delta > 0  \) such that \( | f(x) - L  | < \epsilon  \) whenever \( 0 < | c - x  | < \delta  \). Equivalently, \( \lim_{ x \to c^{-} } f(x) = L  \) if \( \lim f(y_n) = M  \) for all sequences \( (y_n)  \) satisfying \( y_n < c  \) and \( \lim (y_n) = c  \).
\end{proof}

\begin{tcolorbox}
\begin{thm}
Given \( f: A \to \R  \) and a limit point \( c  \) of \( A  \), \( \lim_{ x \to c  } f(x) = L  \) if and only if 
\begin{center}
    \( \lim_{ x \to c^{-} } f(x) = L  \) and \( \lim_{ x \to c^{+} } f(x) = L  \).
\end{center}
\end{thm}
\end{tcolorbox}

\subsubsection{Exercise 4.6.4} Supply a proof for this proposition.
\begin{proof}
    Let \( f: A \to \R  \) and a limit point \( c  \) of \( A  \). Assume \( \lim_{ x \to c  } f(x) = L  \). Let \( \epsilon > 0  \). Then there exists \( \delta > 0  \) such that \( | f(x) - L  | < \epsilon  \) whenever \( 0 < | x - c  | < \delta   \). To show that the right-hand limit equals \( L  \). Let \( x > c  \). Then immediately, we have that \( 0 < | x - c  | < \delta  \) implies \( \lim_{ x \to c^{+}  } f(x) = L  \) since \( | f(x) - L  | < \epsilon. \) For the left-hand limit, suppose \( x < c  \). Then 
    \[  0 < | x - c  | < \delta \iff 0 < | c - x  | < \delta  \] implies \( | f(x) -  L | < \epsilon  \) holds. Hence, \( \lim_{ x \to c^{-} } f(x) = L  \).

    Conversely, the fact that \( \lim_{ x \to c^{-} } f(x) = \lim_{ x \to c^{+} } = L  \) implies for some \( \delta > 0  \) that for any \( x >  c  \) or \( x < c  \) that \( 0 < | x - c  | < \delta  \) holds. Hence, we have that \( | f(x) - L  | < \epsilon  \); that is,
    \( \lim_{ x \to c  } f(x) = L  \).
\end{proof}
Generally speaking, discontinuities can be divided into three categories: 
\begin{enumerate}
    \item[(i)] If \( \lim_{ x \to c  } f(x)  \) exists but has a value different from \( f(c) \), the discontinuity at \( c  \) is called \textit{removable}. 
    \item[(ii)] If \( \lim_{ x \to c^{+} } f(x) \neq \lim_{ x \to c^{-}  } f(x)  \), then \( f  \) has a \textit{jump} discontinuity.
    \item[(iii)] If \( \lim_{ x  \to c  }  f(x)  \) does not exist for some other reason, then the discontinuity at \( c  \) is called an \textit{essential} discontinuity.
\end{enumerate}

We now have the proper characteristics to describe our set of discontinuities \( D_f  \) for an arbitrary monotone function \( f \).

\subsubsection{Exercise 4.6.5} Prove that the only type of discontinuity a monotone function can have is a jump discontinuity.
\begin{proof}
Without loss of generality, let \( f  \) be an increasing function. Then for all \( x < y  \), \( f(y) \geq f(x)  \). If \( \lim_{ x \to c  } f(x) = L  \), but \( L \neq f(c) \), then either \( f(c) < L  \) or \( f(c) > L \). In either case, there exists an \( x \in A  \) such that for all \( y \in A  \), we have \( f(x) > f(y)  \) or \( f(x) < f(y)  \). But this contradicts our assumption that \( f  \) is increasing. Hence, the discontinuity cannot be a \textit{removable} discontinuity. If there exist an \textit{essential discontinuity} then \( f  \) cannot be monotone since either the left-hand limit or the right-hand limit does not exists. Hence, the only type of discontinuity an increasing function can have is a jump discontinuity.
\end{proof}

\subsection{\( D_f \) for an Arbitrary Function}

Review of infinite and finite closed or open sets:

\begin{enumerate}
    \item[(a)] Recall that the intersection of an infinite collection of closed sets is closed and the union of a finite collection of closed sets is closed.
    \item[(b)] On the other hand, the intersection of a finite collection of open sets must be open and the union of a infinite collection of open sets must be open.
\end{enumerate}

\begin{tcolorbox}
\begin{defn}
    A set that can be written as the countable union of closed sets is in the class \( F_{\sigma} \).
\end{defn}
\end{tcolorbox}


