\documentclass[11pt,a4paper]{article} 
\usepackage{standalone}
\usepackage{import}
\usepackage[utf8]{inputenc}
\usepackage[T1]{fontenc}
% \usepackage{fourier}
\usepackage{textcomp}
\usepackage{hyperref}
\usepackage[english]{babel}
\usepackage{url}
% \usepackage{hyperref}
% \hypersetup{
%     colorlinks,
%     linkcolor={black},
%     citecolor={black},
%     urlcolor={blue!80!black}
% }
\usepackage{graphicx} \usepackage{float}
\usepackage{booktabs}
\usepackage{enumitem}
% \usepackage{parskip}
% \usepackage{parskip}
\usepackage{emptypage}
\usepackage{subcaption}
\usepackage{multicol}
\usepackage[usenames,dvipsnames]{xcolor}
\usepackage{ocgx}
% \usepackage{cmbright}


\usepackage[margin=1in]{geometry}
\usepackage{amsmath, amsfonts, mathtools, amsthm, amssymb}
\usepackage{thmtools}
\usepackage{mathrsfs}
\usepackage{cancel}
\usepackage{bm}
\newcommand\N{\ensuremath{\mathbb{N}}}
\newcommand\R{\ensuremath{\mathbb{R}}}
\newcommand\Z{\ensuremath{\mathbb{Z}}}
\renewcommand\O{\ensuremath{\emptyset}}
\newcommand\Q{\ensuremath{\mathbb{Q}}}
\newcommand\C{\ensuremath{\mathbb{C}}}
\newcommand\F{\ensuremath{\mathbb{F}}}
\DeclareMathOperator{\sgn}{sgn}
\DeclareMathOperator{\diam}{diam}
\DeclareMathOperator{\LO}{LO}
\DeclareMathOperator{\UP}{UP}
\DeclareMathOperator{\card}{card}
\DeclareMathOperator{\Arg}{Arg}
\DeclareMathOperator{\Dom}{Dom}
\DeclareMathOperator{\Log}{Log}
\DeclareMathOperator{\dist}{dist}
% \DeclareMathOperator{\span}{span}
\usepackage{systeme}
\let\svlim\lim\def\lim{\svlim\limits}
\renewcommand\implies\Longrightarrow
\let\impliedby\Longleftarrow
\let\iff\Longleftrightarrow
\let\epsilon\varepsilon
\usepackage{stmaryrd} % for \lightning
\newcommand\contra{\scalebox{1.1}{$\lightning$}}
% \let\phi\varphi
\renewcommand\qedsymbol{$\blacksquare$}

% correct
\definecolor{correct}{HTML}{009900}
\newcommand\correct[2]{\ensuremath{\:}{\color{red}{#1}}\ensuremath{\to }{\color{correct}{#2}}\ensuremath{\:}}
\newcommand\green[1]{{\color{correct}{#1}}}

% horizontal rule
\newcommand\hr{
    \noindent\rule[0.5ex]{\linewidth}{0.5pt}
}

% hide parts
\newcommand\hide[1]{}

% si unitx
\usepackage{siunitx}
\sisetup{locale = FR}
% \renewcommand\vec[1]{\mathbf{#1}}
\newcommand\mat[1]{\mathbf{#1}}

% tikz
\usepackage{tikz}
\usepackage{tikz-cd}
\usetikzlibrary{intersections, angles, quotes, calc, positioning}
\usetikzlibrary{arrows.meta}
\usepackage{pgfplots}
\pgfplotsset{compat=1.13}

\tikzset{
    force/.style={thick, {Circle[length=2pt]}-stealth, shorten <=-1pt}
}

% theorems
\makeatother
\usepackage{thmtools}
\usepackage[framemethod=TikZ]{mdframed}
\mdfsetup{skipabove=1em,skipbelow=1em}

\theoremstyle{definition}

\declaretheoremstyle[
    headfont=\bfseries\sffamily\color{ForestGreen!70!black}, bodyfont=\normalfont,
    mdframed={
        linewidth=1pt,
        rightline=false, topline=false, bottomline=false,
        linecolor=ForestGreen, backgroundcolor=ForestGreen!5,
    }
]{thmgreenbox}

\declaretheoremstyle[
    headfont=\bfseries\sffamily\color{NavyBlue!70!black}, bodyfont=\normalfont,
    mdframed={
        linewidth=1pt,
        rightline=false, topline=false, bottomline=false,
        linecolor=NavyBlue, backgroundcolor=NavyBlue!5,
    }
]{thmbluebox}

\declaretheoremstyle[
    headfont=\bfseries\sffamily\color{NavyBlue!70!black}, bodyfont=\normalfont,
    mdframed={
        linewidth=1pt,
        rightline=false, topline=false, bottomline=false,
        linecolor=NavyBlue
    }
]{thmblueline}

\declaretheoremstyle[
    headfont=\bfseries\sffamily, bodyfont=\normalfont,
    numbered = no,
    mdframed={
        rightline=true, topline=true, bottomline=true,
    }
]{thmbox}

\declaretheoremstyle[
    headfont=\bfseries\sffamily, bodyfont=\normalfont,
    numbered=no,
    % mdframed={
    %     rightline=true, topline=false, bottomline=true,
    % },
    qed=\qedsymbol
]{thmproofbox}

\declaretheoremstyle[
    headfont=\bfseries\sffamily\color{NavyBlue!70!black}, bodyfont=\normalfont,
    numbered=no,
    mdframed={
        rightline=false, topline=false, bottomline=false,
        linecolor=NavyBlue, backgroundcolor=NavyBlue!1,
    },
]{thmexplanationbox}

\declaretheorem[
    style=thmbox, 
    % numberwithin = section,
    numbered = no,
    name=Definition
    ]{definition}

\declaretheorem[
    style=thmbox, 
    name=Example,
    ]{eg}

\declaretheorem[
    style=thmbox, 
    % numberwithin = section,
    name=Proposition]{prop}

\declaretheorem[
    style = thmbox,
    numbered=yes,
    name =Problem
    ]{problem}

\declaretheorem[style=thmbox, name=Theorem]{theorem}
\declaretheorem[style=thmbox, name=Lemma]{lemma}
\declaretheorem[style=thmbox, name=Corollary]{corollary}

\declaretheorem[style=thmproofbox, name=Proof]{replacementproof}

\declaretheorem[style=thmproofbox, 
                name = Solution
                ]{replacementsolution}

\renewenvironment{proof}[1][\proofname]{\vspace{-1pt}\begin{replacementproof}}{\end{replacementproof}}

\newenvironment{solution}
    {
        \vspace{-1pt}\begin{replacementsolution}
    }
    { 
            \end{replacementsolution}
    }

\declaretheorem[style=thmexplanationbox, name=Proof]{tmpexplanation}
\newenvironment{explanation}[1][]{\vspace{-10pt}\begin{tmpexplanation}}{\end{tmpexplanation}}

\declaretheorem[style=thmbox, numbered=no, name=Remark]{remark}
\declaretheorem[style=thmbox, numbered=no, name=Note]{note}

\newtheorem*{uovt}{UOVT}
\newtheorem*{notation}{Notation}
\newtheorem*{previouslyseen}{As previously seen}
% \newtheorem*{problem}{Problem}
\newtheorem*{observe}{Observe}
\newtheorem*{property}{Property}
\newtheorem*{intuition}{Intuition}

\usepackage{etoolbox}
\AtEndEnvironment{vb}{\null\hfill$\diamond$}%
\AtEndEnvironment{intermezzo}{\null\hfill$\diamond$}%
% \AtEndEnvironment{opmerking}{\null\hfill$\diamond$}%

% http://tex.stackexchange.com/questions/22119/how-can-i-change-the-spacing-before-theorems-with-amsthm
\makeatletter
% \def\thm@space@setup{%
%   \thm@preskip=\parskip \thm@postskip=0pt
% }
\newcommand{\oefening}[1]{%
    \def\@oefening{#1}%
    \subsection*{Oefening #1}
}

\newcommand{\suboefening}[1]{%
    \subsubsection*{Oefening \@oefening.#1}
}

\newcommand{\exercise}[1]{%
    \def\@exercise{#1}%
    \subsection*{Exercise #1}
}

\newcommand{\subexercise}[1]{%
    \subsubsection*{Exercise \@exercise.#1}
}


\usepackage{xifthen}

\def\testdateparts#1{\dateparts#1\relax}
\def\dateparts#1 #2 #3 #4 #5\relax{
    \marginpar{\small\textsf{\mbox{#1 #2 #3 #5}}}
}

\def\@lesson{}%
\newcommand{\lesson}[3]{
    \ifthenelse{\isempty{#3}}{%
        \def\@lesson{Lecture #1}%
    }{%
        \def\@lesson{Lecture #1: #3}%
    }%
    \subsection*{\@lesson}
    \testdateparts{#2}
}

% \renewcommand\date[1]{\marginpar{#1}}


% fancy headers
\usepackage{fancyhdr}
\pagestyle{fancy}

\makeatother

% notes
\usepackage{todonotes}
\usepackage{tcolorbox}

\tcbuselibrary{breakable}
\newenvironment{verbetering}{\begin{tcolorbox}[
    arc=0mm,
    colback=white,
    colframe=green!60!black,
    title=Opmerking,
    fonttitle=\sffamily,
    breakable
]}{\end{tcolorbox}}

\newenvironment{noot}[1]{\begin{tcolorbox}[
    arc=0mm,
    colback=white,
    colframe=white!60!black,
    title=#1,
    fonttitle=\sffamily,
    breakable
]}{\end{tcolorbox}}

% figure support
\usepackage{import}
\usepackage{xifthen}
\pdfminorversion=7
\usepackage{pdfpages}
\usepackage{transparent}
\newcommand{\incfig}[1]{%
    \def\svgwidth{\columnwidth}
    \import{./figures/}{#1.pdf_tex}
}

% %http://tex.stackexchange.com/questions/76273/multiple-pdfs-with-page-group-included-in-a-single-page-warning
\pdfsuppresswarningpagegroup=1



\title{Math 230A: Homework 1}
\author{Lance Remigio}

\begin{document}
\maketitle
\date
\begin{enumerate}
    \item Mark each statement True or False.  
        \begin{enumerate}
            \item[1-1)] If \( x  \) and \( y  \) are elements of an ordered field, then either \( x \leq y  \) or \( y < x  \). \textbf{False.}
            \item[1-2)] Every ordered field has the least upper bound property. \textbf{True.}
            \item[1-3)] If \( E \subseteq \R  \) is bounded above and \( \alpha = \sup E  \), then \( \alpha \in E  \). \textbf{False.}
            \item[1-4)] If \( E \subseteq \R  \) and \( \alpha \geq x  \), for all \( x \in E  \), then \( \alpha = \sup E  \). \textbf{False.}
            \item[1-5)] If \( E \subseteq \R  \) and \( \alpha \geq x  \), for all \( x \in E \), and \( \alpha \in E  \), then \( \alpha = \sup_{}E   \). \textbf{True.}
\end{enumerate}

\item Prove the following: Suppose \( \alpha  \) is an upper bound for \( E \subset \R   \). Then \( \alpha = \sup E   \) if and only if for all \( \epsilon > 0  \), there exists \( {x}_{0} \in E  \) such that \( {x}_{0} > \alpha - \epsilon  \). \label{Problem 2}  
    \begin{proof}
    Suppose \( \alpha  \) is an upper bound for \( E \subseteq \R  \). For the forwards direction, suppose \( \alpha = \sup E  \) and let \( \epsilon > 0 \). Note that \( \alpha - \epsilon < \alpha  \) implies that \( \alpha - \epsilon  \) is NOT an upper bound of \( E  \). By definition, there must exist an element \( {x}_{0} \in E  \) such that \( {x}_{0} > \alpha - \epsilon  \). Thus, \( \alpha < {x}_{0} + \epsilon  \).  

    For the backwards direction, let \( \epsilon > 0  \) and let \( \alpha  \) be an upper bound of \( E  \). Suppose there exists \( {x}_{0} \in E  \) such that \( {x}_{0} > \alpha - \epsilon  \). Let \( \gamma \) be any upper bound of \( E  \). Thus, \( {x}_{0} \leq \gamma  \) by definition. By assumption, we see that 
    \[  \alpha - \epsilon  < {x}_{0} \leq \gamma \implies \alpha \leq \gamma + \epsilon.  \]
    Since \( \epsilon > 0  \) is arbitrary, we can use Exercise 5 to conclude that \( \alpha \leq \gamma  \), showing that \( \alpha  \) is the least upper bound. Thus, \( \sup E = \alpha  \).
    \end{proof}

\item Let \( E  = \{ n / n + 1 : n \in \N  \}  \). Prove that \( \sup E  = 1  \). 
    \begin{proof}
    
    \end{proof}

\item State the analogue of the Very Useful Theorem for infimums. 
        \begin{solution}
        Suppose \( \beta  \) is a lower bound for \( E \subseteq \R    \). Then \( \beta = \inf E  \) if and only if for all \( \epsilon > 0  \), there exists \( {y}_{0} \in E  \) such that \( {y}_{0} < \beta + \epsilon  \).
        \end{solution}
    \item Suppose \( a,b \in \R  \) and \( a \leq b + \epsilon  \) for every \( \epsilon > 0  \). Prove \( a \leq b  \). \label{Problem 5}
        \begin{proof}
        Suppose \( a, b \in \R  \) and \( a \leq b + \epsilon  \) for all \( \epsilon > 0  \). Suppose for sake of contradiction that \(  a > b  \). Observe that 
        \[  b < a \leq b + \epsilon \implies b \leq b + \epsilon. \]
        Then subtracting \( b  \) on both sides gives us \( \epsilon \geq 0  \). But this is a contradiction because we had assumed that \( \epsilon > 0 \). Thus, it must be the case that \( a \leq b  \).
        \end{proof}
    \item Complete the following proof.   
\begin{theorem}[Greatest-lower-bound property of \( \R  \)]
   Every nonempty subset of \( A  \) of \( \R  \) that is bounded below has a greatest lower bound. In other words, \( \inf A  \) exists and is a real number.   
\end{theorem}
\begin{proof}
Continuation of the proof presented in homework. Set \( \beta = - \alpha  \) where \(  \alpha  = \sup (-A )   \). Thus, there exists \( a \in A  \) such that \( - \gamma < - a  \). Multiplying by a negative on both sides of this inequality, we get \(  \gamma > a  \) for some \( a \in A  \). If \( \gamma  \) was a lower bound of \( A  \), then \( \gamma \leq a   \) for all \( a \in A  \). However, we have the negation of this, so we must have that \( \gamma  \) is NOT a lower bound of \( A  \), which satisfies part (ii). Since \( \beta \leq \alpha  \) for all \( a \in A  \) and \( \beta < \gamma  \) implies \( \gamma  \) is not a lower bound for \( A  \), we conclude that \( \beta  \) must be the infimum of \( A  \).  
\end{proof}
\item \begin{enumerate}
    \item[(i)] Let \( A \subseteq \R   \) be a nonempty set, which is bounded from above. Show that if \( \sup A \notin A  \), then for all \( \epsilon > 0  \) the open interval \( (\sup A - \epsilon, \sup A ) \) contains infinitely many elements of \( A  \).
        \begin{proof}
        Since \( A  \) is nonempty and bounded above, we know that \( A  \) has the least-upper-bound property. Set \( \sup A = \alpha  \). Suppose \( \sup A \notin A  \). Suppose for sake of contradiction that there exists an \( \epsilon > 0  \) such that the open interval \( (\alpha - \epsilon, \alpha ) \) contains a finitely many elements of \( A  \). Thus, for every \( {x}_{i} \in A  \) with \( 1 \leq i \leq n  \), we see that \( {x}_{i} \in (\alpha - \epsilon, \alpha )   \) for some \( \epsilon > 0 \). Then we see that
        \[  \alpha - \epsilon < {x}_{i} < \alpha \implies \alpha < {x}_{i} < \alpha + \epsilon.  \]
        But note that for all \( {x}_{i} \in A  \), we have \( {x}_{i} > \alpha   \). This tells us that \( \alpha  \) is NOT an upper bound of \( A  \), but instead that \( \sup A \in A  \) which is a contradiction. Thus, it must be the case that \( (\alpha - \epsilon, \alpha ) \) must contain infinitely many points.
        \end{proof}
    \item[(ii)] Let \( B \subseteq \R  \) be a nonempty set, which is bounded from below. Show that if \( \inf B \notin B  \), then for all \( \epsilon > 0  \) the open interval \( (\inf B , \inf B + \epsilon ) \) contains infinitely many elements of \( B  \).
        \begin{proof}
        Since \( B  \) is nonempty and bounded below, we know that \( B   \) must contain the greatest upper bound property. Set this greatest upper bound to be \( \beta = \inf B  \). Suppose for sake of contradiction that there exists an \( \epsilon > 0 \) such that the open interval \( (\beta, \beta + \epsilon ) \) contains finitely many elements of \( B  \). Thus, for every \( {x}_{i} \in B  \) with \( 1 \leq i \leq n  \), we see that \( {x}_{i} \in (\beta, \beta + \epsilon ) \). Then we see that 
        \[  \beta < {x}_{i} < \beta + \epsilon \implies \beta - \epsilon < {x}_{i} < \beta.    \]
        But note from this inequality that \( {x}_{i} < \beta   \), implying that \( \beta  \) is not a lower bound of \( B  \). Which means that \( \beta \in B  \) which is a contradiction of our assumption that \( \beta \notin B  \). Hence, \( B  \) must contain infinitely many points.
        \end{proof}
\end{enumerate}

\item Let \( A, B \subseteq \R  \) be nonempty, bounded sets and let \( c \in \R  \). Define the following sets:
    \begin{align*}
        A + B &= \{ a + b : a \in A, b \in B \}  \\
        A - B &= \{ a - b : a \in A , b \in B  \}  \\
        A \cdot B &= \{ ab : a \in A, b \in B \} \\
        cA &= \{ ca: a \in A \}.
    \end{align*}
    Prove that 
    \begin{enumerate}
        \item[9-1)] \( \inf(A + B) = \inf(A) + \inf(B) \).
            \begin{proof}
                Since \( A  \) and \( B  \) are both nonempty and bounded below, we know that both \( \inf(A) \) and \( \inf(B) \) exists. Thus, \( a \geq \inf(A) \) and \( b \geq \inf(B) \) for all \( a \in A  \) and \( b \in B  \), respectively. Thus, 
                \[  a + b \geq \inf(A) + \inf(B) \ \ \text{for all} \ a + b \in A + B. \]
                implies that \( A + B  \) is bounded below. Since \( A + B \neq \emptyset  \) (since \( A  \) and \( B  \) are both nonempty), we see that \( \inf(A+B)  \) exists.

                Now, we will show that 
                \[  \inf(A+B) = \inf(A) + \inf(B);  \]
                that is, we need to show that 
                \[  \inf(A+B) \geq \inf(A) + \inf(B) \tag{1}  \]
                and 
                \[  \inf(A+B) \leq \inf(A) + \inf(B). \tag{2} \]

                To show (1), let \( \epsilon > 0 \). Using the lemma found in {\hyperref[Problem 2]{Problem 2}}, there exists an \( \alpha \in A  \) and \( \beta \in \beta  \) such that 
                \[  \alpha < \inf(A) + \frac{ \epsilon }{ 2 } \]
                and
                \[  \beta < \inf(B) + \frac{ \epsilon }{ 2 },  \]
                respectively. Adding these two inequalities, we get
                \[  \alpha + \beta < \inf(A) + \inf(B) + \epsilon. \]
                Since \( A + B  \) is bounded below, we have \( \inf(A+B) \leq \alpha + \beta  \) such that 
                \[  \inf(A+B) \leq \alpha + \beta < \inf(A) + \inf(B) + \epsilon. \]
                Using {\hyperref[Problem 5]{Problem 5}}, we conclude that   
                \[  \inf(A+B) \leq \inf(A) + \inf(B). \]

                To show (2), let \( \epsilon > 0  \) again. Using the same lemma, there exiss \( \varphi \in A +B  \) with \( \varphi = \gamma + \lambda  \) where \( \gamma \in A  \) and \( \lambda \in B  \) such that 
                \[  \gamma + \lambda = \varphi < \inf(A+B) + \epsilon. \]
                Since \( A  \) and \( B  \) are both bounded below, we can see that
                \[ \inf(A) + \inf(B)  \leq   \gamma + \lambda. \]
                Thus, we have
                \[  \inf(A) + \inf(B) \leq \inf(A+B)  \]
                by {\hyperref[Problem 5]{Problem 5}}.
            \end{proof}
        \item[9-2)] \( \sup (A + B) = \sup (A) + \sup (B) \) 

            \begin{proof}
            Since \( A  \) and \( B \) are both nonempty, we can see that \( A + B   \) is also nonempty. Furthermore, \( A  \) and \( B  \) are both bounded above, so \( a \leq \sup (A) \) for all \( a \in A  \) and \( b \leq \sup(B) \) for all \( b \in B  \) implies that 
            \[  a + b \leq \sup(A) + \sup(B) \ \   \text{for all} \  a \in A \  \text{and} \  b \in B.   \]
            Thus, \(  A + B  \) must be bounded above, and so by definition 1.10 in the textbook, we see that the \( \sup(A+B) \) exists. Our goal is to show that  
            \[  \sup (A + B) = \sup(A) + \sup(B); \]
            that is, it suffices to show that both
            \[ \sup(A+B) \leq \sup(A) + \sup(B) \tag{1}  \]
            and
            \[  \sup(A+B) \geq \sup(A) + \sup(B) \tag{2}. \]
            To prove (1), let \( \epsilon > 0  \). By the lemma found in {\hyperref[Problem 2]{Problem 2}}, there exists \( \varphi \in A +  B \) such that \( \varphi > \sup(A+B) - \epsilon \) with \( \varphi = \alpha + \beta  \) for some \( \alpha \in A  \) and \( \beta \in B  \). Thus, we have \( \alpha \leq \sup(A) \) and \( \beta \leq \sup(B) \), and so        
            \[  \alpha + \beta \leq \sup(A) + \sup(B) \]
            implies
            \[  \sup(A) + \sup(B) \geq \alpha + \beta > \sup(A+B) - \epsilon. \]
            Using {\hyperref[Problem 5]{Problem 5}}, we can see that  
            \[  \sup(A) + \sup(B) \geq \sup(A+B) \]
            which proves (1).

            To prove (2), let \( \epsilon > 0 \) again. Since \( \sup(A) \) and \( \sup(B) \) exists, we see that there exists \( \alpha \in A  \) and \( \beta \in B  \) such that 
        \[  \alpha > \sup(A) - \frac{ \epsilon }{ 2 }   \]
        and 
        \[  \beta > \sup(B) - \frac{ \epsilon  }{ 2 }  \]
        by the lemma found in {\hyperref[Problem 2]{Problem 2}}. Adding these two inequalities results in 
        \[  \alpha + \beta > \sup(A) + \sup(B) - \epsilon. \]
        Note that \( \alpha + \beta \in A + B  \) and \( A + B  \) is bounded above, we have that
        \[  \sup(A+B) \geq \alpha + \beta > \sup(A) + \sup(B) - \epsilon. \]
        Since \( \epsilon > 0  \) is arbitrary, we can see that
        \[  \sup(A+B) \geq \sup(A) + \sup(B) \]
        by {\hyperref[Problem 5]{Problem 5}}. Since (1) and (2) are satisfied, we can conclude that
        \[  \sup(A+B) = \sup(A) + \sup(B). \]
            \end{proof}
        \item[9-3)] \( \sup(-A) = - \inf(A) \) 
            \begin{proof}
            Since \( A \neq \emptyset  \) and \(A  \) is bounded above, we can say that \( -A  \) is also nonempty and bounded above as well from {\hyperref[Problem 6]{Problem 6}}. Thus, we know that the supremum of \( -A  \) exists. Our goal is to show that both
            \[  \sup(-A) \leq - \inf(A) \tag{1} \]
            and
            \[  \sup(-A) \geq - \inf(A) \tag{2}. \]
            Let \( \epsilon > 0  \). To prove (1), we can use the lemma from {\hyperref[Problem 2]{Problem 2}} to state that there exists an \( -a \in -A  \) such that 
            \[  \sup(-A) <  - a - \epsilon. \]
            Since \( A  \) is bounded below, we know that \( a \geq \inf(A)  \). Multiplying through with a negative gives us 
            \[  - a \leq - \inf(A) \]
            which implies that 
            \[  \sup(-A) <  - a - \epsilon \leq - \inf(A) - \epsilon. \]
            Since \( \epsilon > 0 \) is arbitrary, we have that
            \[  - \inf(A) \geq \sup(-A) \]
            which proves (1). 

            Now, to prove (2), we can use the lemma from problem 6, there exists             a \( \alpha \in A  \) such that  
            \[  \alpha < \inf(A) + \epsilon. \]
            Multiplying through by a negative on this inequality, we have 
            \[  - \alpha > - \inf(A) - \epsilon. \]
            But \( A  \) is bounded below, so we must have \( - \alpha \leq \sup(-A) \). Thus, we have
            \[  \sup(-A) \geq - \alpha > - \inf(A) - \epsilon. \]
            Since \( \epsilon > 0  \) is arbitrary, we conclude that 
            \[  \sup(-A) \geq - \inf(A) \]
            by {\hyperref[Problem 5]{Problem 5}}, which proves (2).
 
            \end{proof}
        \item[9-4)] \( \inf(-A) = - \sup(A) \).
            \begin{proof}
            Our goal is to show that  \( \inf(-A) = - \sup(A) \); that is, we need to show that 
            \[ \inf(-A) \leq - \sup(A) \tag{1} \]
            and
            \[  \inf(-A) \geq - \sup(A). \tag{2} \]
            First, we show (1). Let \( \epsilon > 0  \).  Using the lemma found in {\hyperref[Problem 2]{Problem 2}}, there exists \( - \alpha -A  \) such that 
            \[  - \alpha < \inf(-A) + \epsilon  \].
            Since \( A  \) is bounded above, \( \alpha \leq \sup(A) \). Multiplying by a negative on this inequality produces \( - \alpha \geq - \sup(A) \). Thus, we have      
            \[ - \sup(A) \leq - \alpha < \inf(-A) + \epsilon.   \]
            Since \( \epsilon > 0  \) is arbitrary, we have
            \[  - \sup(A) \leq \inf(-A) \]
            by {\hyperref[Problem 5]{Problem 5}}, showing (2). 
            \end{proof}
        \item[9-5)] \( \sup (A -B) = \sup(A) - \inf(B) \).
            \begin{proof}
            Observe that 
            \begin{align*}
                \sup(A - B) = \sup(A + (-B)) &= \sup(A) + \sup(-B) \tag{9-2)} \\
                                             &= \sup(A) - \inf(B) \tag{9-3)}
            \end{align*}
            \end{proof}
        \item[9-6)] \( \inf(A - B) = \inf(A) - \sup(B) \).
            \begin{proof}
            Observe that
            \begin{align*}
                \inf(A -B) = \inf(A + (-B)) &= \inf(A) + \inf(-B) \tag{9-1)} \\
                                            &= \inf(A) - \sup(A). \tag{9-4)}
            \end{align*}
            \end{proof}
        \item[9-7)] \( \sup(cA) = c \sup (A)  \) if \( c > 0  \).
        \item[9-8)] \( \inf(cA) = c \inf (A)  \) if \( c > 0  \).
        \item[9-9)] \( \sup(cA) = c \inf(A) \) if \( c < 0  \).
        \item[9-10)] \( \inf(cA) = c \sup (A) \) if \( c < 0  \).
        \item[9-11)] Is it true that \( \sup(A \cdot B) = \sup(A) \cdot \sup(B) \).
    \end{enumerate}
\item Recall that \( \Q  \) is a field; in particular, \( \Q  \) is closed under addition and multiplication. 
    \begin{enumerate}
        \item[10-1)] Prove that if \( p \in \Q  \) and \( t \) is an irrational number, then \(  p + t  \) is an irrational number.
            \begin{proof}
            
            \end{proof}
        \item[10-2)] Complete the following proof.
            \begin{theorem}
                Given any two real numbers \( x < y  \), there exists an irrational number \( t  \) satisfying \( x < t < y  \).
            \end{theorem}
            \begin{proof}
            
            \end{proof}
    \end{enumerate}
\end{enumerate}

\end{document}

