\section{Continuous Functions}

\begin{definition}{Continuity}{}
    A function \( f: A \to \R  \) is \textit{continuous at a point} \( c \in A  \) if, for all \( \epsilon > 0  \), there exists \( \delta > 0  \) such that whenever \( | x - c  | < \delta \) (and \( x \in A  \)) it follows that \( | f(x) - f(c) | < \epsilon  \). 
    If \( f \) is continuous at every point in the domain \( A  \), then we say that \( f \) \textit{continuous} on \( A  \).
    \end{definition}

The difference between this definition and the definition for functional limits is that we require the limit point \( c  \) of \( A  \) to be in the domain of \(  f\). The value of \( f(c) \) is the value of \( \lim_{ x \to c } f(x)  \). It is indeed possible to shorten this definition to say that \( f  \) is continuous at \( c \in A  \) if 
\[  \lim_{ x \to c } f(x) = f(c) \tag{1}\]
so as long as \( c  \) is a limit point of \( A  \). The equation above gets undefined if \( c \) is an isolated point of \( A  \). But this does not stop \( f \) from being continuous at the point \( c \). In fact, functions can still be continuous at their isolated points such as \( c \).

We observed in the previous section that functional limits can be formulated using sequences from chapter 2. The same can be done for continuity. 

\begin{theorem}{Characterizations of Continuity}{}
    Let \( f: A \to \R  \), and let \( c \in A  \). The function \( f \) is continuous at \( c \) if and only if any one of the following three conditions is met:
    \begin{enumerate}
        \item[(i)] For all \( \epsilon > 0  \), there exists \( \delta > 0  \) such that \( | x - c  | < \delta \) (and \( x \in A  \)) implies \( | f(x) - f(c) | < \epsilon; \)
        \item[(ii)] For all \( V_{\epsilon }(f(c)) \), there exists a \( V_{\delta}(c)  \) with the property that \( x \in V_{\delta} (c) \) (and \( x \in A  \)) implies \( f(x) \in V_{\epsilon }(f(c))  \);
        \item[(iii)] For all \( (x_n) \to c \) (with \( x_n \in A
            \)), it follows that \( f(x_n) \to f(c) \).

            If \( c \) is limit point of \( A  \), then the above conditions are equivalent to 
    \item[(iv)] \( \lim_{ x \to c } f(x)  = f(c)\).
    \end{enumerate}
    \end{theorem}

\begin{proof}
    (i) \( \iff \) (ii) Let \( V_{\epsilon}(f(c)) \) and let \( \epsilon > 0  \). By assumption, there exists \( \delta > 0  \) such that \( | x - c  | < \delta \) (and \( x \in A  \)) implies 
    \[ | f(x) - f(c) | < \epsilon. \]
    We can rephrase in terms of \( \epsilon  \) and \( \delta \) neighborhoods. Hence, the statement above is just equivalent to the statement that for all \( V_{\epsilon }(f(c)) \), there exists \( V_{\delta}(c) \) such that \( x \in V_{\delta}(c) \) which implies that \( f(x) \in V_{\epsilon }(f(c)) \). 

     (iii) \( \iff \) (ii)  First we show (ii) holds. Let \( (x_n) \to c  \) such that \( f(x_n) \to f(c)  \). In other words, \( f(x_n) \in V_{\epsilon }(f(c)) \). For sake of contradiction, that \( f(x) \notin V_{\epsilon }(f(c)) \). This means there exist \( \epsilon_0  \) such that for all \( \delta > 0  \) with the property that \( | x - c  | < \delta \) that 
    \[  | f(x) - f(c)  | \geq \epsilon_0. \] Let \( \delta = 1 / n  \) and suppose we pick \( x_n \in V_{\delta}(c) \) such that \( f(x_n) \notin V_{\epsilon }(f(c)) \). But this contradicts our assumption that \( f(x_n) \in V_{\epsilon }(f(c)) \). Hence, it must be the case that \( f(x) \in V_{\epsilon }(f(c)) \).

    Now we want to show that \( (iii) \) holds. Let \( (x_n) \subseteq A  \).
    (with \( x_n \in A  \)). Since \( x_n \in A  \) \( x_n \neq c  \) or \( x_n = c  \). If \( x_n = c  \), then it immediately follows that \( f(x_n) \to f(c) \) given there exists \( | x_n  - c | < \delta \). Suppose \( x_n \neq c  \). Pick \( x_n \in V_{\delta}(c) \) for some \( \delta > 0  \). Then by assumption, we have \( f(x_n) \in V_{\epsilon }(f(c)) \). Hence, \( f(x_n) \to f(c) \).

    To show (i) \( \implies \) (iv), suppose \( c  \) is a limit point of \( A  \). Let \( x \in A  \). Since \( c \in A \), either \( x \neq c  \) or \( x = c  \). The conclusion follows immediately from the latter statement. Suppose \( x_n \neq c  \). By assumption, there exists \( \delta > 0  \) such that \( | x_n -c  | < \delta \). From this, it follows that 
    \[  |  f(x) - f(c)  | < \epsilon. \]
    And hence, 
    \[ \lim_{ x \to c } f(x) = f(c). \]
\end{proof}

\begin{corollary}[Criterion for Discontinuity]
    Let \( f: A \to \R  \), and let \( c \in A  \) be a limit point of \( A  \). If there exists a sequence \( (x_n) \subseteq A  \) where \( (x_n) \to c  \) but such that \( f(x_n) \not \to f(c) \), we may conclude that \( f \) is not continuous at \( c \).
    \end{corollary}

This sequential characterization of continuity allows us to use all the results that we know of when it comes to sequences from Chapter 2. 


\begin{theorem}[Algebraic Continuity Theorem]
    Assume \( f: A \to \R  \) and \( g: A \to \R  \) are continuous at point \( c \in A  \). Then, 
    \begin{enumerate}
        \item[(i)] \( kf(x)  \) is continuous at \( c  \) for all \( k \in \R  \);
        \item[(ii)] \( f(x) + g(x) \) is continuous at \( c  \) for all \( k \in \R  \);
        \item[(iii)] \( f(x) g(x) \) is continuous at \( c  \); and 
        \item[(iv)] \( f(x) / g(x)  \) is continuous at \( c  \), provided the quotient is defined.
    \end{enumerate}
    \end{theorem}

\begin{proof}
All of these statements can be derived from the Characterizations of Continuity Theorem and the Algebraic Functional Limit Theorem.
\end{proof}


All polynomials are continuous on \( \R  \). In fact, rational functions (Quotients of polynomials) are continuous wherever they are defined. Consider the identity function \( g(x) = x  \). Since \( | g(x) - g(c) | = | x - c  |  \), we can respond to a given \( \epsilon > 0 \) by choosing \( \delta = \epsilon  \) such that \( g  \) is continuous on all of \( \R  \). Furthermore, this argument gets much simpler when we consider a constant function such as \( f(x) = k  \). Since any arbitrary polynomial 
\[  p(x) = a_0 + a_1 x + a_2 x^2 + \dots + a_n x^n \]
consists of sums and products of \( g(x) \) with different constant functions, we can conclude that \( p(x) \) is continuous. On the other hand, the Algebraic Continuity Theorem implies that quotients of polynomials are continuous as long as the denominator is not zero.

In the sinuisodal example in the last section, we noticed that the oscillations of \( \sin (1/x ) \) are so rapid near the origin that \( \lim_{ x \to 0  } \sin (1/x) \) does not exist. Consider the function, 
\[  g(x) = 
\begin{cases}
    x \sin (1/x) &\text{~if~} x \neq 0 \\ 
    0 &\text{~if~}  x = 0. 
\end{cases} \]
Suppose we want to observe the continuity of \( g \) at \( c = 0  \). We can do this by the following:
\[ | g(x) - g(0) | = | x \sin(1/x) - 0  | \leq | x |   \]
since \( | \sin (x)  |  \leq 1 \). Given \( \epsilon > 0 \), choose \( \delta = \epsilon  \) such that whenever \( | x | < \delta \) it follows that \( | g(x) - g(0) | < \epsilon  \). Thus, \( g  \) is continuous at the origin. 


    Consider the greatest integer function \( h(x) = [[x]] \) which for each \( x \in \R  \) returns the largest integer \( n \in \Z  \) such that \( n \leq x  \). In previous math classes, this step function is observed to have discontinuous jumps at each integer value of its domain. We can show this more rigorously using the tools we have at our disposal. Given \( m \in \Z  \), define the sequence \( (x_n)  \) by \( x_n = m -  1/n \). It follows that \( (x_n) \to m \), but not that 
    \[  h(x_n) \to (m-1), \]
    which does not equal \( m = h(m) \). By the Criterion for Discontinuity, we see that \( h \) fails to be continuous at each \( m \in \Z  \). Suppose we want to see why \( h  \) is continuous at a point \( c \notin \Z \). Given any \( \epsilon > 0  \), we must find a \( \delta-\)neighborhood \( V_{\delta}(c) \) such that \( x \in V_{\delta}(c) \) implies \( h(x) \in V_{\epsilon }(h(c)) \). We know that \( c \in \R  \) falls between consecutive integers \( n < c < n+1 \) for some \(  n \in \Z  \). Taking \( \delta = \min \{ c -n , (n+1) - c  \}  \), then it follows from definition of \( h \) that \( h(x) = h(c) \). Thus, we certainly have that \( h(x) \in V_{\epsilon }(h(c)) \) whenever \( x \in V_{\delta}(c) \). This proof actually implies that our \( \delta \) is not dependent on the value of \( \epsilon > 0  \). 

Consider \( f(x) = \sqrt{ x }  \) defined on \( A = \{ x \in \R : x \geq 0 \}  \). Exercise 2.3.1 outlines a sequential proof that \( f  \) is continuous on \( A  \). Show \( f \) is continuous below.

What about functions like \( h(x) = \sqrt{ 3x^2 + 5  }   \) is continuous. Hence, a Compositions of Continuous functions type theorem is needed to show that \( h(x) \) is continuous on its domain.


\begin{theorem}{Composition of Continuous Functions}{}
    Given \( f: A \to \R  \) and \( g: B \to \R  \), assume that the range \( f(A) = \{ f(x) : x \in A  \}  \) is contained in the domain \( B  \) so that the composition \( g \circ f(x) = g(f(x)) \) is defined on \( A  \). If \( f \) is continuous at \( c \in A  \), and \( g  \) is continuous at \( f(c) \in B  \), then \( g \circ f  \) is continuous at \( c  \).

    \end{theorem}

\begin{proof}
Exercise 4.3.3.
\end{proof}



