\section{Functional Limits}


\subsubsection{Exercise 4.2.1}

\begin{enumerate}
    \item[(a)] Supply the details for how Corollary 4.1.1.1 follows from the Sequential Criterion for Functional Limits in Theorem 4.1.1 and the Algebraic Limit Theorem for sequences proved in Chapter 2.

    \begin{proof}
    Let \( f \) and \( g \) be functions defined on a domain \( A \subseteq \R  \). Assume \( \lim_{ x \to c } f(x) = L  \) and \( \lim_{ x \to c } g(x) = M  \) for some limit point \( c  \) of \( A  \). By the Sequential Criterion for functional limits, let \( (x_n), (y_n) \subseteq A  \) such that \( (x_n) \to c  \) and \( (y_n) \to c  \) where \( x_n, y_n \neq c  \) implying that \(f(x_n) \to L   \) and \( g(y_n) \to M  \). By the Algebraic Limit Theorem, we can state that 
    \[  \lim [f(x_n) + g(y_n)]\lim f(x_n) + \lim g(y_n) = L  + M.\]
    Hence, we have that 
    \[  \lim_{ x \to c } [f(x) + g(x)] = L + M  \]
    by the Sequential Criterion for Functional Limits.
    \end{proof}
    \item[(b)] Now, write another proof of Corollary 4.1.1.1 part (ii) directly from Definition for functional limits without using the sequential criterion in Theorem 4.1.1.
    \begin{proof}
    We can also prove the addition rule for Functional Limits directly from the definition. Suppose \( \lim_{ x \to c } f(x) = L  \) and \( \lim_{ x \to c } g(x) = M  \). Then for some \( \delta > 0  \), suppose \( c  \) is a limit point of \( A  \) such that \( 0 < | x -c  | < \delta \). We want to show that for any arbitrary \( \epsilon > 0  \) that 
    \[  | (f(x) + g(x)) - (L + M) | < \epsilon.  \]
    Hence, choose \( \delta = \min \{ \delta_1, \delta_2  \}  \)
    \begin{align*}
        |(f(x) + g(x)) - (L + M)   | &= | (f(x) - L ) + (g(x) - M ) |  \\
       &\leq  | f(x) - L  |  + | g(x) - M  | \\ 
       &< \frac{ \epsilon  }{ 2  }  + \frac{ \epsilon  }{ 2 } \\
       &= \epsilon.
    \end{align*}
    Hence, we have 
    \[  \lim_{ x \to c } [f(x) + g(x)] = L + M.  \]
    \end{proof}
    \item[(c)] Repeat (a) and (b) for Corollary 4.1.1.1 part (iii).
    \begin{proof}
    Let \( (x_n), (y_n) \subseteq A  \) such that \( (x_n) \to c  \) and \( (y_n) \to c  \) for some limit point \( c  \) of \( A  \) where  we have 
    \( \lim f(x_n) = L   \) and \( \lim  g(y_n) = M  \). By the Algebraic Limit Theorem, we have that 
    \[ \lim [f(x_n)g(y_n)] = \lim f(x_n) \cdot \lim g(y_n) = L \cdot M. \]
   By the Sequential Criterion, this also means that 
   \[  \lim_{ x \to c } [f(x)g(x)] = L \cdot M. \]

    Now we can prove the same fact but this time we use only the Definition of functional limit.
    Let \( f: A \to \R  \). Suppose there exists \( \delta > 0  \) such that \( 0 < | x - c  | < \delta  \) where \( c  \) is a limit point of \( A  \). Let \( \epsilon > 0  \). Our goal is to show that 
    \[ | f(x)g(x) - LM  | < \epsilon. \]
    Since \( \lim_{ x \to c } f(x) = L  \) and \( \lim_{ x  \to c } g(x) = M  \), choose \( \delta = \min \{  \delta_1, \delta_2, \delta_3  \}  \) 
    \begin{align*}
        | f(x)g(x) - LM  | &= | f(x)g(x) - g(x)L + g(x)L - LM |  \\
                           &= | g(x)(f(x) - L ) + L (g(x) - M)  | \\ 
                           &\leq | g(x) | | f(x) - L  | + L | g(x) - M  |   \\
                           &< M + 1 \cdot \frac{ \epsilon  }{ 2(M + 1)  } + L \cdot \frac{ \epsilon  }{ 2L  } \\
                           &= \epsilon.
    \end{align*}
    Hence, we have that 
    \[  \lim_{ x \to c } f(x)g(x) = L M. \]
    \end{proof}
\end{enumerate}


\subsubsection{Exercise 4.2.2} For each stated limit, find the largest possible \( \delta- \)neighborhood that is a proper response to the given \( \epsilon  \) challenge.

\begin{enumerate}
    \item[(a)] \( \lim_{ x \to 3 } (5x - 6 ) = 9  \).
        \begin{proof}[Solution]
        Since \( \epsilon = 1  \), we know that 
        \[ | (5x-6) - 9  | < 1.\]
        To solve for \( \delta \), we do the following
        \begin{align*}
            | (5x - 6) - 9  | &= | 5 (x - 3 ) | < 1    \\
                              &= 5 | x - 3  | < 1 
        \end{align*}
        which implies that 
        \[  | x - 3  | < \frac{ 1 }{ 5 } = \delta. \]
        Hence, the largest possible \( \delta- \)neighborhood that is a proper response to the given \( \epsilon   \) challenge. 
        \end{proof}
    \item[(b)] \( \lim_{ x \to 4 } \sqrt{ x } = 2  \), where \( \epsilon = 1  \).
        \begin{proof}[Solution]
        Since \( \epsilon = 1  \), we know that 
        \[  | \sqrt{ x }  - 2  | < 1. \]
        To get our \( \delta \), we just do the following manipulations
        \begin{align*}
            \sqrt{ x } - 2   &< 1  \\
            \sqrt{ x } &< 3 \\
            x &< 9 \\
            | x - 4  | &< 5 = \delta. \\
        \end{align*}
        Hence, our \( \epsilon  \) response is \( \delta =  5  \).
        \end{proof}
    \item[(c)] \( \lim_{ x  \to \pi } [[x]] = 3 \), where \( \epsilon = 1  \). (The function \( [[x]] \) returns the greatest integer less than or equal to \( x  \).)
        \begin{proof}[Solution]
            Let \( \epsilon = 1  \). We want to generate a \( \delta > 0  \) that satisfies the \( \epsilon  \) challenge. Since \( [[\pi]] = 3  \), our max \( \delta- \)neighborhood can only have \( \delta = \pi - 3  \)
        \end{proof}
    \item[(d)] We have (c) but this time with \( \epsilon = .01 \).  
        \begin{proof}[Solution]
        It would still remain as \( \delta = \pi - 3  \).
        \end{proof}
\end{enumerate}

\subsubsection{Exercise 4.2.5} Use Definition 4.1.1 to supply a proper proof for the following limit statements. 
\begin{enumerate}
    \item[(a)] \( \lim_{ x \to 2 } (3x + 4 ) = 10 \).
        \begin{proof}
        Suppose there exists a \( \delta > 0  \) such that \( 0 < | x - 2  | < \delta \). Let \( \epsilon > 0  \). Then observe that by using definition 4.1.1 that 
        \begin{align*}
            | f(x) - 10 | &= | (3x+4) - 10 |  \\
                          &= 3 | x - 2  | \\
                          &< 3 \delta \\
                          &= 3 \cdot \frac{ \epsilon  }{ 3 }\\
                          &= \epsilon.
        \end{align*}
        \end{proof}
    \item[(b)] \( \lim_{ x \to 0 } x^3 = 0  \).
        \begin{proof}
            Suppose there exists \( \delta > 0  \) such that \( 0 < | x  | < \delta \). By applying the definition of functional limits, choose \( \delta = (\epsilon )^{1/3} \)
        \begin{align*}
            | f(x) - 0  | &= | x^3  |  \\
                          &< \delta^3 \\
                          &= \epsilon.
        \end{align*}
        Hence, we have \( | f(x) - 0  | < \epsilon  \) which implies that 
        \[  \lim_{ x \to 0 } x^3 = 0. \]
        \end{proof}
    \item[(c)] \( \lim_{ x \to 2 } (x^2 + x - 1 ) = 5.  \)
        \begin{proof}
        Let \( \epsilon > 0  \). Choose \( \delta = \min \{ \delta_1, \delta_2  \}  \) such that 
        \begin{align*}
            | f(x)  - 5 | &= | (x^2 + x - 1 ) - 5  |  \\
                          &= | x^2 + x - 6  | \\
                          &= | x + 3  | | x - 2  | \\
                          &< | x+3  | \delta \\
                          &< 3 \cdot \frac{ \epsilon  }{ 3 } .
                          &= \epsilon.
        \end{align*}
        \end{proof}
    \item[(d)] \( \lim_{ x \to 3 } 1/x = 1/3.\)
        \begin{proof}
        Let \( \epsilon > 0  \). Choose \( \delta = \min \{ \delta_1, \delta_2 \}  \) such that 
        \begin{align*}
            | f(x) - \frac{ 1 }{ 3 }  | &= | \frac{ 1 }{ x }  - \frac{ 1 }{ 3 }  |  \\
                                        &= \Big| \frac{ x - 3  }{ 3x }  \Big| \\
    &= \frac{ | x - 3  |  }{ 3| x |  } \\ 
    &< 12 \epsilon \cdot \frac{ 1 }{ 12 } \\
    &= \epsilon.
        \end{align*}
        \end{proof}
\end{enumerate}




\subsubsection{Exercise 4.2.6} Decide if the following claims are true or false, and give short justifications for each conclusion.

\begin{enumerate}
    \item[(a)] If a particular \( \delta \) has been constructed as a suitable response to a particular \( \epsilon  \) challenge, then any smaller positive \( \delta \) will also suffice.
        \begin{proof}[Solution]
        This is true since \( \delta \) that has been constructed is the largest possible neighborhood that one can produce as a response to the \( \epsilon  \) challenge. 
        \end{proof}
    \item[(b)] If \( \lim_{ x \to a } f(x) = L  \) and \( a \) happens to be in the domain of \( f \), then \( f(a) = L  \).
        \begin{proof}[Solution]
        This is false. I have trouble thinking of an example. Will get back to this one soon. 
        \end{proof}
    \item[(c)] If \( \lim_{ x \to a } f(x) = L  \), then \( \lim_{ x \to a } 3[f(x) - 2 ]^2 = 3(L-2)^2 \).
        \begin{proof}[Solution]
        This is true based on the Algebraic Functional Limit Theorem.
        \end{proof}
    \item[(d)] If \( \lim_{ x \to a } f(x) = 0  \), then \( \lim_{ x\to a } f(x)g(x) = 0   \) for any function \( g \) (with domain equal to the domain of \( f \)).
        \begin{proof}[Solution]
        This is not true in general since we can easily produce \( g(x)  \) that is unbounded. For this to work, we would need \( g(x) \) to be bounded.
        \end{proof}
\end{enumerate}

\subsubsection{Exercise 4.2.7} Let \( g: A \to \R  \) and assume that \( f \) is a bounded function on \( A \) in the sense that there exists \( M > 0  \) satisfying \( | f(x) | \leq  M  \) for all \( x \in A  \). Show that if \( \lim_{ x \to c } g(x) = 0  \), then \( \lim_{ x \to c } g(x)f(x) = 0  \) as well.
\begin{proof}
Let \( g: A \to \R  \). Suppose \( \lim_{ x \to c } g(x) = 0   \) and suppose \( f(x) \) is bounded; that is, there exists \( M > 0  \) such that \( | f(x) | \leq M  \) for all \( x \in A  \). Using the Functional Limit Definition, suppose there exists \( \delta > 0  \) such that \( 0 < | x - c  | < \delta \). Hence, we have 
\begin{align*}
    | f(x)g(x) - 0  | &= | f(x) g(x) |  \\
                      &= | f(x) | | g(x) | \\
                      &\leq M | g(x) | \\
                      &< M \cdot \frac{ \epsilon  }{ M } \\
                      &= \epsilon.
\end{align*}
It follows from this that \( \lim_{ x \to c } f(x)g(x) = 0  \).
\end{proof}

\subsubsection{Exercise 4.2.9 (Infinite Limits).} The statement \( \lim_{ x \to 0 } 1/x^2 = \infty  \) certainly makes intuitive sense. To construct a rigorous definition in the challenge response style of Definition 4.1.1 for an infinite limit statement of this form, we replace the (arbitrarily small) \( \epsilon > 0  \) challenge with an (arbitrarily large) \( M > 0  \) challenge: 

\begin{definition}{}{}
We say \( \lim_{ x \to c } f(x) = \infty  \) to mean that for all \( M > 0  \), we can find a \( \delta > 0  \) such that whenever \( 0 < | x - c  | < \delta \), it follows that \( f(x) > M  \).
\end{definition}
\begin{enumerate}
    \item[(a)] Show \( \lim_{ x \to 0 } 1/x^2 = \infty  \) in the sense described in the previous definition.
        \begin{proof}
        Let \( M > 0  \) and \( f(x) = 1/x^2 \). Choose \( \delta = 1 / \sqrt{ M }  \). Since \( 0 < | x  | < \delta \), we have 
        \[ x^2 < \delta^2 \iff \frac{ 1 }{ x^2 } > \frac{ 1 }{ \delta^2  }. \]
        But this means that 
        \[ f(x) = \frac{ 1 }{ x^2 } > \frac{ 1 }{ \delta^2 } = \frac{ 1 }{ 1/ M } = M. \]
        Hence, we have 
        \[  \lim_{ x \to 0 } f(x) = \infty. \]
        \end{proof} \item[(b)] Now, construct a definition for the statement \( \lim_{ x \to \infty  } f(x) = L  \). Show that \( \lim_{ x \to \infty  } 1/x = 0  \).
       \begin{tcolorbox}
       \begin{defn}
      Let \( f: A \to \R  \). We say \( \lim_{ x \to \infty  } f(x) = L  \) to mean for any \( \epsilon > 0  \), there exists \( x_0 \in A  \) where \( A \subseteq \R  \) such that for any \( x \geq x_0  \), we have 
      \[  | f(x) - L  | < \epsilon. \]
       \end{defn}
       \end{tcolorbox} 

       \begin{proof}[Solution]
        We start with some sketch work for what \( x_0  \) might be. Observe that 
        \begin{align*}
            | f(x) - L  | &= | \frac{ 1 }{ x } - 0  |  \\
                          &= \frac{ 1 }{ x } \\ 
                          &< \epsilon.
        \end{align*}
        Solving for \( x \), we get that 
        \[  x > \frac{ 1 }{ \epsilon  }.  \] Hence, \( x_0 = 1 / \epsilon   \).

        Let \( \epsilon > 0  \) and let \( f(x) = 1 / x  \). Choose \( x_0 = 1 / \epsilon  \). Suppose \( x > x_0 = 1 / \epsilon  \). Then we have that 
        \[ x > \frac{ 1 }{ \epsilon  } \iff \frac{ 1 }{ x } < \epsilon. \]
        Hence, we have that 
        \[  | f(x) - 0  | < \epsilon. \]
        This means \( \lim_{ x \to \infty  } f(x) = 0  \).
        \end{proof}
\end{enumerate}


\subsubsection{Exercise 4.2.10}

Introductory calculus courses typically refer to the \textit{right-hand limit} of a function as the limit obtained by "letting \( x \) approach \( a \) from the right-hand side"

\begin{enumerate}
    \item[(a)] Give a proper definition in the style of Definition 4.1.1 for the right-hand and left-hand limit statements: 

        \begin{center}
            \( \lim_{ x \to a^{+} } f(x) = L  \) and \( \lim_{ x \to a^{-} } f(x) = M. \) 
        \end{center}
        \begin{tcolorbox}
        \begin{defn}
        We say that
        \[ \lim_{ x \to a^{+}  } f(x) = L  \text{~and~} \lim_{ x \to a^{-} } f(x) = M  \]  
        if for all \( \epsilon > 0  \), there exists a \(  \delta > 0 \) such that whenever \( 0 < | x - c  | < \delta \)  and \( 0 < | c - x  | < \delta  \) we have 
        \begin{align*}
            | f(x) - L  | &< \epsilon, \\
            | f(x) - M  | &< \epsilon
        \end{align*}
        respectively.
        \end{defn}
        \end{tcolorbox}
    \item[(b)] Prove that \( \lim_{ x \to a } f(x) = L  \) if and only if both the right and left-hand limits equal to \( L  \).
        \begin{proof}
        Suppose \( \lim_{ x \to a } f(x) = L  \). Let \( \epsilon > 0  \). Then there exists \( \delta > 0  \) such that whenever \( 0 < | x - a  | < \delta  \), we have 
        \[ | f(x) - L  | < \epsilon. \tag{1} \]
        This denotes our right-hand limit. Note that this also works if we flip the order of our \( \delta \) assumption. Hence, we have \( 0 < | a - x  | < \delta \) such that (1) holds. 
        \end{proof}
        But this means our right-hand and left-hand limits are equal to each other.

        Now we show the converse. Suppose the right-hand and left-hand limits are equal to each other; that is, 
        \[ \lim_{ x \to a^{+}  } f(x) = \lim_{ x \to a^{-}  } f(x).\]
        We want to show that \( \lim_{ x \to a } f(x) = L  \). Since both \( | x - a  | < \delta \) and \( | a - x   | < \delta  \) hold for both limits, we immediately have that 
        \[  | f(x) - L  | < \epsilon. \]
        Hence, we have \( \lim_{ x \to a } f(x) = L  \).
\end{enumerate}




\subsubsection{Exercise 4.2.11(Squeeze Theorem).} Let \( f,g, \) and \( h \) satisfy \( f(x) \leq g(x) \leq h(x) \) for all \( x  \) in some common domain \( A  \). Suppose \( f(x) \leq g(x) \leq h(x) \ \). If \( \lim_{ x \to c  } f(x) = L  \) and \( \lim_{ x \to c } h(x) = L  \) at some limit point \( c \) of \( A  \), show 
\[  \lim_{ x \to c } g(x) = L  \] as well.
\begin{proof}
Let \( (x_n), (y_n), (z_n) \subseteq A   \). Suppose \( \lim_{ x \to c } f(x) = L  \) and \( \lim_{ x \to c } h(x) = L  \) at some limit point \( c \) of \( A  \). Using the Sequential Criterion for Functional Limits, let \( f(x_n) \to L  \) and \( h(z_n) \to L  \). There exists \( N \in \N  \) such that for any \( n \geq N  \), we have 
\[ f(x_n) \leq g(y_n) \leq h(z_n) \iff L \leq g(y_n) \leq L. \]
Hence, we have \( g(y_n) \to L  \) by the Squeeze Theorem for Sequences. By the Sequential Criterion, we must have \( \lim_{ x \to c } g(x) = L  \) as well.
\end{proof}

Another proof using the definition of Functional Limits directly.

\begin{proof}
Let \( f, g, h \) satisfy \( f(x) \leq g(x) \leq h(x) \) for all \( x  \) in some common domain \( A  \). Since \( \lim_{ x \to c } f(x) = L  \) and \( \lim_{ x \to c } h(x) = L  \), we know that 
\[ \lim_{ x \to c } [h(x) - f(x)] = 0.  \]
Suppose there exists \( \delta > 0  \) such that \(  0 < | x - c  | < \delta  \). Choose \( \delta = \min \{ \delta_1, \delta_2  \}  \). Since \( f(x) \leq g(x) \leq h(x) \), we have 
\begin{align*}
    | g(x) - L  | &\leq | h(x) - L  |  \\
                  &= | h(x) - f(x) + f(x) - L  | \\
                  &\leq | h(x) - f(x)  | + |  f(x) - L  | \\
                  &< \frac{ \epsilon  }{ 2 } + \frac{ \epsilon  }{ 2 } \\
                  &= \epsilon.
\end{align*}
But, this means that \( \lim_{ x \to a } g(x) = L.  \)
\end{proof}

Another proof 

\begin{proof}
Let \( f, g, h \) satisfy \( f(x) \leq g(x) \leq h(x) \) for all \( x  \) in some common domain \( A  \). 
Suppose there exists \( \delta > 0  \) such that \(  0 < | x - c  | < \delta  \). Choose \( \delta = \min \{ \delta_1, \delta_2, \delta_3, \delta_4  \}  \). Since \( f(x) \leq g(x) \leq h(x) \), we have 
\begin{align*}
    | g(x) - L  | &= | g(x) - h(x) + h(x) - L  |  \\
                  &\leq | g(x) - f(x) + f(x) - h(x)  | + | h(x) - L  | \\ 
                  &\leq | g(x) - f(x) | + | f(x) - h(x)  | + | h(x) - L  | \\
                  &\leq | h(x) - f(x)  | + | f(x) - L  | + | L - h(x) | + | h(x) - L  | \\
                  &\leq | h(x) - L| + | L - f(x) | + 2| h(x) - L  | \\
                  &< \frac{ \epsilon  }{ 4 } + \frac{ \epsilon  }{ 4 } + \frac{ 2 \epsilon  }{ 4 } \\
                  &= \epsilon.
\end{align*}
Hence, it follows that 
\[  \lim_{ x \to a } g(x) = L. \]
\end{proof}






