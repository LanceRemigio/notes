\section{Construction of R from Q}

\subsubsection{Exercise 8.6.1} 
\begin{enumerate}
    \item[(a)] Fix \( r \in \Q  \). Show that the set \( C_{r} = \{ t \in \Q : t < r  \}  \) is a cut.
        \begin{proof}
        For the first property (C1), we show that \( C_{r} \neq \emptyset  \) and \( C_{r} \neq \Q  \). Suppose \( r \in \Q  \) is fixed. Now define \( t = r - 1  \). Then clearly, we have \( t  <r  \). But this means \( t \in C_{r } \). Hence, \(  C_{r} \neq \emptyset \). Now, we show \( C_{r} \neq \Q  \). Define \( s = r + 1 \in \Q   \). Then clearly, \( s > r  \). Hence, \( s \notin C_{r} \) and thus \( C_{r} \neq \Q  \).

        For the second property (C2), we need to show that if \( r \in C_{r} \), then \( C_{r}  \) also contains every rational \( q < r  \). Suppose we let \( t,q  \in \Q  \) be arbitrary with \( t > q  \) with \( t \in C_{r} \). Then by definition of \( C_{r} \), we must have \( t < r  \). But we have \( q < t < r  \). Hence, we have \( q < r  \) for any \( q \in \Q  \). 

        Lastly, we need to show that \( C_{r}  \) does not have a maximum; that is, for any \( x \in C_{r}  \), there exists an \( s \in C_{r}  \) with \( x < s  \). Let \( x \in C_{r}  \). Then by definition of \( C_{r}  \), we must have \( x < r  \). Suppose we take the midpoint between \( x \) and \( r  \). Then we must have \( x < \frac{ x  +r  }{ 2  } < r  \). Setting \( s = \frac{ x +r  }{ 2  }  \) gives us our desired result.
        \end{proof}
        Avoid thinking of cuts as only having this form. Which of the following subsets of \( \Q  \) are cuts?
    \item[(b)] \( S = \{ t \in \Q : t \leq 2  \}  \)
        \begin{proof}[Solution]
        This is not a cut because \( S  \) contains a maximum. In this case, the maximum is \( 2  \).
        \end{proof}
    \item[(c)] \( T = \{ t \in \Q : t^{2} < 2 \ \text{or} \ t < 0  \}  \)
        \begin{proof}[Solution]
        The set \( T  \) is a cut. First, we show (C1). Observe that \(  0 \in T  \). Hence, \( T \neq \emptyset \). Then observe that \(  2 \notin T  \). Hence, \(  T \neq \Q  \). 

        Next, we show (C2). Let \( r \in T  \). Then by definition of \( T  \), we must have either \( r^{2} < 2  \) or \( r < 0  \). If \( r < 0  \), then we can easily pick \(  q < r  \) for any \( q < 0  \). Otherwise, we have \( r^{2} < 2  \) such that if we let \( q \in \Q  \) be arbitrary with \( q^{2} < r^{2}  \). Then we easily have \( q < r  \).

        Finally, we show (C3). Let \( r \in T  \). Then either we have \( r^{2} < 2  \) or \( r < 0  \). Consider 
        \begin{align*}
           \Big( r + \frac{ 1 }{ n }  \Big)^{2} &= r^{2} + \frac{ 2r }{ n }  + \frac{ 1 }{ n }  \\
                                                &= r^{2} +  \frac{ 2r + 1  }{ n  }.  \\
        \end{align*}
        Let us set \( s = r + \frac{ 1 }{ n_{0} }   \). If \( r > 0  \), then we can pick an \( n_{0}  \) large enough such that 
        \[ \frac{ 1 }{ n_{0}  }  < \frac{ 4 - r^{2}  }{ 2r+1  }.   \]
        Then we have 
        \[  \Big( r + \frac{ 1 }{ n_{0} }  \Big)^{2} < r^{2} + \frac{ 2r+1 }{ n_{0} } < r^{2} + \frac{ 2r+1 }{ n_{0}  }  \cdot \frac{ 4 - r^{2} }{ 2r+1 } < 4. \]
        Hence, we have \( r < s < 2   \). Otherwise, we choose \( n_{0 }  \) large enough so that 
        \[  \frac{ 1 }{ n_{0}  }  < \frac{ -r^{2} }{  2r+1 }. \] Then we have 
        \[  \Big( r + \frac{ 1 }{ n_{0} }  \Big)^{2} < r^{2} + \frac{ 2r+1 }{ n_{0} }  < 0. \]
         This implies that \( r < s < 0  \). 
     \end{proof}


    \item[(d)] \( U = \{ t \in \Q : t^{2} \leq 2 \ \text{or} \ t < 0  \}.  \)
            \begin{proof}
            Is a cut. Same reasoning can be used to prove that \( U  \) is a cut, but we can assume \( a \geq 0  \) this time.
            \end{proof}
\end{enumerate}





\subsubsection{Exercise 8.6.2} 
Let \( A  \) be a cut. Show that if \( r \in  A \) and \( s \notin A  \), then \( r < s  \).
\begin{proof}
Suppose for sake of contradiction that \( r \geq s  \). Since \( r \in A  \) and \( s \in \Q  \), we know that \( s < r   \). But this means that \( s \in A  \) which leads to a contradiction.
\end{proof}

\subsubsection{Exercise 8.6.3}  Using the usual definitions of addition and multiplication, determine which of these properties are possessed by \( \N  , \Z,  \) and \( \Q  \), respectively.
\begin{proof}[Solution]
\begin{enumerate}
    \item[(\( \N  \))] This is not a field since an additive inverse does not exist; that is, \( - 1 \notin \N \). 
    \item[(\( \Z  \))] The set of integers is not a field because every element in \( x \in \Z  \) does not contain an inverse except for \(  1  \).
    \item[(\( \Q \))] Field.
\end{enumerate}
\end{proof}



\subsubsection{Exercise 8.6.4} Show that this defines an ordering on \( \R  \) by verifying properties (o1), (o2), and (o3) from Definition 8.6.5.
\begin{proof} 
    Observe that the first property (o1) follows immediately. For the second property (o2), suppose \( A \leq B  \) and \( B \leq A  \). Then we have \( A \subseteq B  \) and \( B \subseteq A  \). This implies that \( A = B  \). Finally, for the third property (o3), let \( A, B, C \in \R   \). If \( A \leq B  \) and \( B \leq C \), then we have \( A \subseteq B  \) and \( B \subseteq C  \). Then we have \( A \subseteq B \subseteq C  \). Hence, we have \( A \subseteq C \). Hence, we have an ordering on \( \R  \).
\end{proof}


\subsubsection{Exercise 8.6.5} 
\begin{enumerate}
    \item[(a)] Show that (c1) and (c3) also hold for \( A + B  \). Conclude that \( A +  B \) is a cut.
        \begin{proof}
        For property (c1), let \( a \in A  \). Then \( a = (a-b) + b \) where \( a -b \in A  \) and \( b \in B  \) implies \( a \in A + B  \). Hence, we have that \( A + B \neq \emptyset \). Let \(  c \in C  \) where \( C \subseteq \Q   \), then we find that \( A + C \not\subseteq A + B  \). Hence, the sum \( a+c \notin A + B \). Hence, \( A + B \neq \Q  \). 
        To show property (c3), Let \( a + b \in A +  B  \) be arbitrary. Since \( A  \) is a cut, we can find an \( \alpha \in A \) such that \( a < \alpha  \). Likewise, \( B  \) being a cut implies that there exists \( \beta \in B  \) such that \( b < \beta \). Adding the two inequalities together, we get that \(  a+b < \alpha + \beta \). Denote the set \( S = \{ \alpha + \beta : \alpha \in A \ \text{and} \ \beta \in B  \}  \). Then we have \( A + B \subseteq S  \) which proves property (c3). Since properties (c1), (c2), and (c3) have been satisfied, we can now conclude that \( A + B  \) is a cut.
        \end{proof}
    \item[(b)] Check that addition in \( \R  \) is commutative (f1) and associative (f2).
        \begin{proof}
        Since \( A + B  \) is a cut and its elements \( a + b   \), where \( a \in A  \) and \( b \in B  \) are both cuts, subsets of \( \Q  \) where \( \Q  \) is an ordered field, we know that addition must be commutative and associative. This means that 
        \[  A + B = B + A.\] If we have an extra set \( C  \) that is also a cut, then we would have \( (a+b) + c = a + (b+c)  \) implying that 
        \[  (A+B) + C = A + (B+C). \] 
        Hence, we have (f1) and (f2) are satisfied.
        \end{proof}
    \item[(c)] Show that property (o4) holds.
        \begin{proof}
        Suppose we have \( A \subseteq C  \) where \( A  \) and \( C  \) are both cuts.         Let \( B  \) be a cut. Then \( A \subseteq C \) implies \( a \leq c  \). If we have \( b \in B  \) be arbitrary, then we can use the ordering of \( \Q  \) to write \( b + a \leq b + c  \). But this means that \(  B + A \subseteq B + C  \) and hence property (o4) is satisfied.
        \end{proof}
    \item[(d)] Show that the cut 
        \[  O = \{ p \in \Q : p < 0  \}  \] successfully plays the role of the additive identity (f3). (Showing \( A + O = A  \) amounts to proving that these two sets are the same. The standard way to prove such a thing is to show two inclusions : \( A  + O \subseteq A  \) and \( A \subseteq A + O  \).) 
        \begin{proof}
        First, we show \( A + O \subseteq A  \). Let \( a + p \in  A+ O  \) be arbitrary. Since \(  p < 0  \), we must have \(  a + p \leq a  \). But this means that \( A + O \subseteq A  \). For the second inclusion \( A \subseteq A + O  \), let \( a \in A  \). Then observe that \( a = (a-p) + p \). Since \( p < 0  \), we can write \(  a = (a-p) + p  \). But since \( a - p \in A  \) and \( p \in O  \), we know that \( a \in A + O  \). Hence, we have \( A \subseteq A + O  \).        
        \end{proof}
\end{enumerate}


\subsubsection{Exercise 8.6.6} 
\begin{enumerate}
    \item[(a)] Prove that \( -A  \) defines a cut.
        \begin{proof}
        To show the (c1), let \( r \in \Q  \). Since \( \Q  \) is a field, we can rewrite \( r  \) in the following way
        \[  r = r + (t - t)  = (r+t) - t = (t+r) - t < -t  \] with \( t \not\in A  \). Multiplying through the inequality by a negative leads us to \(  -r > t  \). This tells us that \( r \in -A  \) and we must have \( -A \neq \emptyset  \). However, this also implies that \( -r \notin -A  \) and hence, \( -A \neq \Q  \).
        To show (c2), let \( r \in -A  \). Then by definition of \( -A  \), there exists a \( t \notin A  \) such that \( -r  > t  \). Suppose we let \( -q \in \Q  \) be arbitrary with \( -q > -r  \). Using the fact that \( \Q  \) is a field, we can multiply the inequality by a negative to get \( r > q  \) our desired result. Hence, (c2) is satisfied.

        Finally, we show (c3). Suppose \( r \in -A  \). Then there exists \( t \notin A  \) such that \( -r > t  \). Multiplying through the inequality by a negative gives us \( r < -t  \) which is our desired result. Hence, \( -A  \) defines a cut.
        \end{proof}
    \item[(b)] What goes wrong if we set \( -A = \{ r \in \Q : -r \in A  \}  \)?
        \begin{proof}
        If we have \( -A  \) defined as the set above, then it would fail (c3). We can see this when we define \(  A = \{ r \in \Q : r < 0  \}  \) and let \( -A = \{ r \in \Q : r \leq  0 \}   \) which tells us that \( -A  \) contains a maximum.
        \end{proof}
    \item[(c)] If \( a \in A  \) and \( r \in -A  \), show \( a + r \in O  \). This shows \( A + (-A ) \subseteq O  \). Now, finish the proof of property (f4) for addition in the definition of a field.
        \begin{proof}
            Let \( a \in A  \) and \( r \in -A  \). Let \( a + r \in O  \). Since \( r \in -A  \), we know that there exists \( t \notin A  \) with \( t < -r  \). Now, we can write \( a + r < a - t    \). But \( -t < 0  \) so this tells us that \( a + r < 0  \). Hence, \( a + r \in O  \) and we have \( A + (-A) \subseteq O  \). Now we need to show \( O \subseteq A + (-A)  \). Let us fix \( o \in O  \) and finding \( a \in A  \) such that \( a + b = o  \). Let \( \epsilon  = | o | / 2 = -o / 2  \). Taking advantage of properties (c1) and (c2), we can choose a rational \( t \notin A  \) such that \( t - \epsilon \in A  \). If no such \( t  \) existed then we would either have \( A = \Q  \) or \( A = \emptyset \). Now, \( t \notin A   \) implies \( -(t+\epsilon ) \in -A  \). Then 
            \[ o = -2 \epsilon = -(t+\epsilon ) + (t-\epsilon ) \in A + (-A ), \] and hence we conclude \( O \subseteq A + (-A) \). Hence, (f4) is proven.
        \end{proof}
\end{enumerate}


\begin{remark}
   Another possible way you could show the reverse inclusion in part (c) is to rewrite \( o \in O  \) such that \( o = (o+r) - r  \) and show that \( 0+r  \in -A \) for which, in addition to \( -r \in A  \), leads to \( O \subseteq A + (-A ) \). Since there exists \( t \notin A  \) such that \( -r > t  \) and \( o < 0  \), we have 
   \[  o+r < r \implies o + r < r < -t.   \]
   Multiplying the inequality by a negative gives us \( -(o+r) < t  \), which tells us that \( o +r \in -A  \). Since we also have \( - r \in A   \), we conclude that \( o \in A + (-A )  \) and hence \( O \subseteq A + (-A) \).
\end{remark}


\subsubsection{Exercise 8.6.7} 
\begin{enumerate}
    \item[(a)] Show that \( AB  \) is a cut and that property (o5) holds.
        \begin{proof}
        To show (c1), let \( a , b \in \Q  \) with the property that \( a,b \geq  0  \). Since \( \Q \) is a field, we can take the product of \( a  \) and \(  b  \) and get \( a \cdot b \geq 0  \) implying that \( AB \neq \emptyset \). Likewise, if we take any \( p \in \Q  \) with \( p < 0  \) and multiply it by any strictly non-negative \( a \in \Q  \), then \( p \cdot a < 0  \) implying that \( AB \neq \emptyset  \) once again. We know that \( AB \neq \emptyset  \) since \( q \leq  0  \) is not in \( A B  \). Hence, (c1) is satisfied.

    To show (c2), we need to show that for all \( q \in \Q  \) that \(  r \in A B  \) implies \( q < r  \). Suppose \( r = a \cdot b  \) with \( a \in A \) and \( b \in B  \). Since \( A   \) and \( B  \) are cuts, we can use (c2) to state that for all  \( \alpha \in A  \) and for all \( \beta \in B  \), we have such \(  \alpha < a   \) and \( \beta < b  \) respectively. Then taking the product of these inequalities, we get our desired result that \( q =  \alpha \cdot \beta < a \cdot b = r \). Hence, (c2) is satisfied.

    Finally, we show (c3). Let \( r \in AB  \) where \( r = a \cdot  b \) where \( a \in A  \) and \( b \in B \). Since \( A  \) and \( B  \) are cuts, we can find \(  w \in A  \) and \( z \in B  \) such that \( r = a \cdot b < z \cdot w = \omega \). Hence, (c3) is satisfied.

    Now we need to show that  property \( (o5)   \) holds for \( AB \). Assume \( A \geq O  \) and \( B \geq O  \). We know that \( O \subseteq A   \) and \(  O \subseteq B  \). This means that if we let \(  p, w \in O  \), then \( p, w \in A  \) and \( p, w  \in B  \). Note that \( p \cdot w < 0   \). Since \( ab \geq 0  \), we know that \( pw \leq ab  \) which implies that \( pw \in AB \) and hence, \( O \subseteq AB \).
        \end{proof}
    \item[(b)] Propose a good candidate for the multiplicative identity on \( \R  \) and show that this works for all cuts \( A \geq O  \).
        \begin{proof}
        A good candidate for the multiplicative identity is \( I  \). Let us define it as follows: 
        \[  I = \{ p \in \Q : p < 1 \}.  \] 
        We want to show that \( AI =A  \); that is, we need to show the following two inclusions:     

        \[  A I \subseteq A  \ \text{and} \ A \subseteq A  I. \]
        Let \( x \in A I  \). Then we have \( xp < x \cdot 1 = x   \) which tells us \( xp \in  A  \) and we have \( A I \subseteq A   \). Now if \( x \in A  \), then we have \( x \cdot 1 > xp  \) and hence \( x \in A  I \). Thus, we have \( A \subseteq AI  \). Hence, \(  AI = A  \). 
        \end{proof}

    \item[(c)] Show the distributive property (f5) holds for non-negative cuts.
        Before moving on to the proof, define \( A(B+C)  \) as 
        \[  A(B+C) = \{ a(b+c) : a \in A, b+c \in B+C \ \text{with} \ a, b+c \geq 0   \} \cup  \{ p \in \Q : p < 0  \}. \] Then let 
        \[  P = \{ a(b+c) : a \in A , b+c \in B + C \ \text{with} \ a, b+c \geq 0   \}.  \]
        \begin{proof}
        The main goal is to show that \( A (B+C) = AB + AC \); that is, we need to show the following two inclusions:
        \[  A (B+C) \subseteq AB+AC  \ \text{and} \  AB+AC \subseteq  A (B+C). \]
        Since \( x \in A(B+C)  \), then either \( x \in  P  \) or \( x \in O  \). Suppose we have \( x \in P  \). Then we have \( x = a(b+c)  \). Since \( \Q  \) is an ordered field, we can use the distributive property to write \( a(b+c) = ab+ ac \in AB + BC   \). Hence, \( a(b+c) \in AB + BC  \) and \( A(B+C) \subseteq AB + BC  \). Now suppose \( x \in O  \), then \( x < 0  \). Since the products \( AB  \) and \( BC  \) are non-negative, we know that we must have \( ab \geq 0  \) and \( bc \geq 0  \). This tells us that their sum \( ab + ac \geq 0  \) and hence, we have \( x <  0 \leq ab + ac   \). Since \( x \leq ab + ac  \), we can write \( O \subseteq AB + AC  \).  

        Next, we show \( AB + BC \subseteq A(B+C) \). Let \( x \in AB + BC \). Then \( x = r_{1} + r_{2}  \) where \( r_{1} \in AB \) and \( r_{2} \in BC \). If \( r_{1} \in AB  \), then either \( r_{1} = ab \geq 0  \) or \( r_{1} < 0  \). Likewise, \( r_{2} \in BC \) implies that either \( r_{2} = ac \geq 0  \) or \( r_{2} < 0  \). Suppose \( r_{1} = ab \geq 0  \) and \( r_{2} = ac \geq  0  \). Then we can simply use the distributive property to write \( ab + ac = a (b+c)  \in A(B+C)  \). Hence, \( ab + ac \in A(B+C) \). Now suppose \(  r_{1} \geq 0  \) and \( r_{2} < 0  \). Then \( x = r_{1} + r_{2} < r_{1} + 0 \). But note that \( AB + O \subseteq O  \) which tells us that \( x \in A(B+C) \). The case for when \( r_{2} \geq  0   \) and \( r_{1} < 0  \) is similar. If both \( r_{1} < 0  \) and \( r_{2} < 0  \), then \( x = r_{1} + r_{2} < 0 \leq a(b+c)  \) and hence, \( x \in A(B+C)  \). Thus, we can conclude that \( AB + BC \subseteq A(B+C) \).
        \end{proof}
\end{enumerate}


\subsubsection{Exercise 8.6.8} 
Let \( \mathcal{A} \subseteq \R  \) be nonempty and bounded above, and let \( S  \) be the \textit{union} of all \( A \in \mathcal{A}  \).  
\begin{enumerate}
    \item[(a)] First, prove that \( S \in \R  \) by showing that it is a cut.
        \begin{proof}
            First, we show (c1). Since \( S  \) is defined as the union of nonempty sets \( A  \), we also have that \( S \neq \emptyset \). Since all the sets within \( S   \) are just a union of cuts, we know that these cuts also have the property that \( A \neq \Q  \) and hence \( S \neq \Q  \).

            Next, we show property (c2). Let \( x \in S  \). Since \( S  \) is defined as the union of all cuts \( A \in \mathcal{A}  \), we know that \( x \in S  \) implies that there exists an \( A'  \) such that \(  x \in A'  \). Let \( q \in \Q  \) But since \( A'  \) is a cut, we know that we can use \( (c2)  \) to say that \( q < x  \). Hence, (c2) is satisfied.
            
            Lastly, we show property (c3). Let \( x \in S  \). Using the same reasoning to show (c2), we know that \( x \in A'  \) implies that there exists \( \alpha \in Q  \) such that \( x < \alpha   \).
        
            Since all the properties of a cut are satisfied, we can now conclude that \( S  \in \R   \) is also a cut.
        \end{proof}
    \item[(b)] Now, show that \( S  \) is the least upper bound for \( \mathcal{A} \).
        \begin{proof}
            In order to show that  \( S  \) is the least upper bound of \( \mathcal{A}  \), we need to show that \( S  \) is an upper bound and that \( S  \) is the least upper bound.
        
            First, we show \( S  \) is an upper bound; that is, \( S \geq A  \) for all \( A \in \mathcal{A} \). Let \( A \in \mathcal{A} \) be arbitrary. Observe that since \( S = \bigcup A  \), we know that \( A \subseteq \bigcup A = S  \). But this tells us that \( A \leq S  \). Since \( A  \) is arbitrary, we know that \( S  \) must be an upper bound.      

            Lastly, we show that \( S  \) is the \textit{least upper bound}. Let \( B \in \mathcal{A}  \) be any upper bound. Since each set \( A  \) in the union \( S = \bigcup A  \) is bounded by \( B  \) (that is, \( A \subseteq B\)), it follows that \( S \subseteq B  \). But this tells us that \(  S \leq B  \) and hence, \(  S  \) must be the least upper bound of \( \mathcal{A} \).
        \end{proof}
\end{enumerate}

\subsubsection{Exercise 8.6.9} Consider the collection of so-called "rational" cuts of the form 
\[  C_{r} = \{ t \in \Q  : t < r \}  \]
where \( r \in \Q  \). (See Exercise 8.6.1)
\begin{enumerate}
    \item[(a)] Show that \( C_{r} + C_{s} = C_{r+s} \) for all \( r,s \in \Q  \). Verify \( C_{r} C_{s} = C_{rs} \) for the case when \( r, s \geq 0  \).
        \begin{proof}
       Let \( r,s \in \Q  \). First, we show \( C_{r} + C_{s} = C_{r+s} \); that is, we need to show that following two inclusions:
       \[  C_{r } + C_{s} \subseteq C_{r+s} \ \text{and} \ C_{r+s} \subseteq C_{r} + C_{s}.   \]
    Let \( z \in C_{r} + C_{s} \). Then we have \( z = x + y   \) where \( x \in C_{r}  \) and \( y \in C_{s} \). By definition of \( C_{r} \) and \( C_{s } \), we know that \( x < r  \) and \( y < s  \) respectively. Hence, adding both of these inequalities produces \( z = x + y < r + s  \) which tells us that \( z \in C_{r+s} \). Thus, we have \( C_{r} + C_{s} \subseteq C_{r+s} \). 

    Now, we show the reverse inclusion. Let \( z \in C_{r+s} \). Then \( z < r + s  \) by definition of \( C_{r+s} \). Note that subtract \( s  \) from both sides of the inequality to get \( z - s < r  \). This tells us that \( z - s \in C_{r}  \). Likewise, subtract \( r  \) from both sides of the inequality to get \( z - r < s  \) which implies that \( z -r  \in C_{s} \). We can add these two distinct elements to get 
    \[ (z-r) + (z-s) = 2z - (r+s) < 2(r+s). \]
    Dividing by 2 then gives us 
    \[  z - \frac{ r+s }{ 2 }  < r + s. \]
    Hence, \( z \in C_{r} + C_{s}. \) and we conclude that \( C_{r+s} \subseteq C_{r} + C_{s} \).

    Let \( r, s \geq 0  \) in \( \Q  \). We want to show \( C_{r} C_{s} = C_{rs} \). To do this, we need to show the following two inclusions; that is, 
    \[  C_{r} C_{s} \subseteq C_{rs} \ \text{and} \ C_{rs} \subseteq C_{r} C_{s}. \]
    First, let \( x \in C_{r}  C_{s}   \). Then either \( x = \alpha  \beta  \) where \( \alpha \cdot \beta \geq   0  \) where \( \alpha \in C_{r} \) and \( \beta \in C_{s} \) or \( x < 0 \). If we assume the former, we can take \( \alpha \in C_{r}  \) imply \( \alpha < r  \) and \( \beta \in C_{s}  \) imply \( \beta < s  \). Multiplying these two inequalities together, we must have \( \alpha \beta < rs  \) which implies \( x  \in C_{rs}  \) and hence \( C_{r} C_{s} \subseteq C_{rs} \). Suppose \( x < 0  \). Since \( r, s \geq 0  \) implies \( rs \geq  0  \), we know that \( x < 0 < rs \). Hence, we must have \( x \in C_{rs} \). 

   Now, we show the reverse inclusion. Suppose \( x \in C_{rs} \), then \( x < rs \) for \( r,s \geq 0  \) in \( \Q  \). Suppose \( r,s > 0  \) (if \( r,s = 0  \) then the result follows immediately), then dividing by \( r  \) on both sides gives us \( x /r  < s   \) which tells us that \( x / r \in C_{s}  \). Similarly, we can divide by \(  s  \) to get \( x / s < r  \) which implies \( x / s \in C_{r}  \). Hence, taking the product of these two elements leads to 
   \[  \frac{ x }{ s }  \cdot \frac{ x }{ r }  < (rs)^{2} \iff \frac{ x  }{ s^{2}  } \cdot \frac{ x  }{ r^{2} } < rs . \] Hence, we must have \( x \in C_{r}C_{s} \) and thus we have \(  C_{rs } \subseteq C_{r}C_{s} \). 
        \end{proof}
    \item[(b)] Show that \( C_{r} \leq C_{s}  \) if and only if \( r \leq s  \) in \( \Q  \).
        \begin{proof}
        Suppose \( r \leq s  \) in \( \Q  \). Let \( z \in C_{r}  \). Using the fact that \( r \leq s  \) and using definition of \( C_{r}  \), we have \( z < r \leq s   \). This tells us that \( z \leq s  \) which subsequently tells us that \( z \in C_{s}   \). Hence, \( C_{r } \subseteq C_{s}  \) which is equivalent to \( C_{r} \leq C_{s} \). Now, let \( C_{r} \leq C_{s}  \). Suppose for sake of contradiction that \( r > s  \). Suppose we take the midpoint of \( r  \) and \( s \). Then we have \( a = \frac{ r+s }{ 2 }  \) which implies \( a < r  \). This tells us that \( a \in C_{r} \), but \( a \notin C_{s} \) since \( a > s  \). Hence, we have a contradiction. Hence, we must have \( r \leq s  \) in \( \Q  \).      
        \end{proof}
\end{enumerate}
