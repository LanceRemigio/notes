\section{Fourier Series}
Fourier's development of a special type of series involving trigonometric functions caused a rework into what it really means to have a "limit" or to "converge" towards a specific value.

\subsection{Trigonometric Series}

The basic use of series representations is to express a given function \( f(x)  \) in terms of the sum of simpler and "nicer" functions such as polynomials. In the case of power series representations, we have the following set of component functions \( \{ 1, x, x^{2}, x^{3}, \dots \}  \) such that the following series takes the form 
\[  f(x) = \sum_{ n=0  }^{ \infty  } a_{n} x^{n} = a_{0} + a_{1} x + a_{2} x^{2} + a_{3} x^{3} + \dotsb  . \]
A \textit{trigonometric series} on the other hand is an infinite series composed of component functions that are trigonometric; that is, we have 
\[  \{ 1 , \cos(x), \sin(x), \sin(2x), \cos(3x), \sin(3x), \dots \}.  \]
Thus, we can write a trigonometric series in the following way 
\begin{align*}
    f(x) &= a_{0} + a_{1}\cos(x) + b_{1} \sin(x) +  a_{2} \cos(2x) + b_{2} \sin(2x) + a_{3} \cos(3x) + \dotsb  \\
         &= a_{0} + \sum_{ n=1 }^{ \infty  } a_{n} \cos(nx) + b_{n} \sin(nx). 
\end{align*}

Now we will see how this formulation can be used to solve partial differential equations. Consider the following problem proposed by d'Alembert
\[  \frac{\partial ^{2} u  }{\partial x^{2}   } = \frac{\partial ^{2} u  }{\partial t^{2} } \tag{1} \]
which describes the motion of a vibrating string. Suppose the solution to the partial differential equation above is \( u(x,t)  \) which models that displacement of the string at time \( t \geq 0  \). Suppose the string is attached at each end of this interval. Then the boundary conditions apply
\[  u(0,t) = 0 \ \text{and} \ u(\pi, t) = 0  \tag{2} \]
for all values of \( t \geq 0  \). If we let \( t = 0  \) (at the instant the string is released), then we can assume that 
\[  \frac{\partial u }{\partial t } (x,0) = 0; \]
that is, the initial velocity of the string is defined to be \( 0 \). 

\subsubsection{Exercise 8.5.1} 
\begin{enumerate}
    \item[(a)] Verify that 
        \[  u(x,t) = b_{n} \sin(nx) \cos(nt) \] satisfies equations (1), (2), and (3) for any choice of \( n \in \N  \) and \( b_{n} \in \R  \). What goes wrong if \( n \notin \N  \).
        \begin{proof}[Solution]
        We first compute the partial derivatives of \( u(x,t)  \). Observe the first partial derivative with respect to \( x  \) of \( u(x,t) \) is 
        \begin{align*}
            \frac{\partial u }{\partial x } &= n b_{n} \cos(nx) \cos(nt). \\
        \end{align*}
        Then differentiating with respect to the same variable leads to 
        \[  \frac{\partial ^{2} u  }{\partial x^{2} } = - n^{2} b_{n} \sin(nx) \cos(nt).  \] 
        Likewise, differentiating with respect to \( t  \) leads to 

        \begin{align*}
            \frac{\partial u }{\partial t } &= -n b_{n} \sin(nx) \sin(nt). \\
            \frac{\partial ^{2} u  }{\partial  t^{2} } &= - n^{2} b_{n} \sin(nx) \cos(nt).
        \end{align*}
        We can see that (1) is satisfied using our given solution \( u(x,t) \).
        Now, using the boundary conditions found in (2) and the fact that the sine function is 0 with \( n \pi  \) for all \( n \in \N  \), we can see that 
        \[  u(0,t) =  b_{n} \sin(0) \cos(nt) = 0  \]
        and 
        \[  u(\pi, t ) = b_{n} \sin(n \pi ) \cos(nt) = 0. \] Hence, (2) is satisfied. By the same reasoning, we can see that (3) is satisfied.

        If \( n \notin \N \), then \( u(x,t)  \) would result in nonzero sine values rendering equations (1), (2), (3) invalid.
        \end{proof}
    \item[(b)] Explain why any finite sum of functions of the form given in part (a) would also satisfy (1), (2), and (3). (Incidentally, it is possible to hear the different solutions in (a) for values of \( n  \) up to 4 or 5 by isolating the harmonics on a well-made stringed instrument.)
        \begin{proof}[Solution]
        Since \( \sin(x) \) and \( \cos(x) \) are both continuous differentiable up to \( n  \) times, we can use term-by-term differentiation of the finite sum of these trigonometric functions to show that, indeed, equations (1), (2), and (3) are all satisfied.
        \end{proof}
\end{enumerate}

Observe that 
\[  u(x,t) = \sum_{ n=1 }^{ N } b_{n} \sin(nx) \cos(nt) \tag{4} \] solves the PDE in (1) which is called d'Alembert's \textit{wave equation}. The solutions to (1), however, depend on how the string is originally "plucked". Suppose at \( t = 0  \), the string is given some initial displacement 
\[  f(x) = u(x,0) .\] Letting \( t = 0  \) in our general solution in (4), we find that 
\[  f(x) = \sum_{ n=1 }^{ N } b_{n} \sin(nx).  \tag{5}\] 
Given there exists coefficients \( b_{n}   \) for all \( 1 \leq n \leq N \) such that our solution \( f(x) \) can be written in terms of the finite sum of sine functions found in  (5), we can be sure that the PDE in (1) can be solved by \( u(x,t)  \) given in (4). We can then ask, more generally, how can we construct solutions that are linear combinations of functions found in the set 
\[  \{  \sin(x), \sin(2x), \sin(3x), \dots \} . \] It turns out that we can take an \textit{infinite} sum of (5) to model the position of \( f(x)  \) for any \( x \in [0, \pi] \). The function \( f(x) \) can be thought of as the initial temperature applied to some boundary of a heat-conducting material.


\subsection{Periodic Functions}
A more general formulation of our problem is to find suitable coefficients \( (a_{n}) \) and \( (b_{n}) \) to express a function \( f(x)  \) as 
\[  f(x) = a_{0} + \sum_{ n=1 }^{ \infty  } a_{n} \cos(nx) + b_{n} \sin(nx). \tag{6} \] It is important to note that every component that makes up (6) is \( 2 \pi  \) periodic. That is, any function that is represented by trigonometric functions is necessarily periodic. Turning our attention on the interval \( (-\pi, \pi] \), we can take a function such as \( f(x) = x^{2} \), restrict its domain to \( (-\pi , \pi]  \), and then extend \( f  \) periodically to all of \( \R  \) using the rule \( f(x) = f(x+ 2 k \pi ) \) for all \( k \in \Z \)

\subsection{Types of Convergence}
The way we express functions in terms of a series of trigonometric functions depends on the type of convergence. As we have been discussing all throughout this book, we have to look at the partial sums. Hence, we have 
\[  S_{N}(x) = a_{0} + \sum_{ n=1 }^{ N } a_{n} \cos(nx) + b_{n } \sin(nx). \tag{7} \]
The idea of expressing \( f(x)  \) in terms of a trigonometric series involves finding the coefficients 
\( (a_{n})_{n=0}^{\infty } \) and \( (b_{n})_{n=1}^{\infty } \) such that  
\[  f(x) = \lim_{ N \to \infty  } S_{N}(x). \tag{8} \]
Showing that 
\[  \int_{ - \pi  }^{  \pi  } | S_{N}(x) - f(x)  |^{2} \ dx \to 0 \]
is a natural way to understand (8). This convergence is called the \( L^{2} \) \textit{convergence} (won't be discussed here). Another type of convergence that we will look at in this section is the \textit{Cesaro mean convergence} which involves taking the \textit{averages} of the partial sums and showing that they converge uniformly to \( f(x)  \).

\subsection{Fourier Coefficients}

Before discussing the Cesaro mean convergence, we will look at a few basic calculus facts.

\subsubsection{Exercise 8.5.2} Using trigonometric identities when necessary, verify the following integrals.
\begin{enumerate}
    \item[(a)] For all \( n \in \N  \), 
        \[  \int_{ - \pi  }^{ \pi  }  \cos(nx) \ dx = 0 \ \text{ and } \ \int_{ - \pi  }^{ \pi   } \sin(nx) \   dx = 0  \]
\begin{proof}[Solution]
    We will begin by showing the first integral. Hence, using the fact that \( \sin(x)  \) is odd and the fact that \( \sin(n \pi) = 0  \) for all \( n \in \N  \), we can say
    \begin{align*}
        \int_{ -\pi  }^{ \pi  } \cos(nx) \ dx &=  \Big[ \frac{ 1 }{ n }  \sin(nx)\Big]_{- \pi }^{\pi }  \\
                                              &= \frac{ 1 }{ n } [  \sin(n \pi ) - \sin(-n \pi)  ] \\
                                              &= \frac{ 2 }{ n }  \sin(n \pi) \\
                                              &= 0.
    \end{align*}
    Now we show the second integral. Using the fact that \( \cos(x) \) is an even function, we can write
    \begin{align*}
        \int_{ -\pi  }^{ \pi  } \sin(nx)  \ dx &= \Big[ -\frac{ 1 }{ n } \cos(nx) \Big]_{-\pi}^{\pi } \\
                                               &= \frac{ 1 }{ n } [  \cos(-n \pi) -\cos(n \pi)  ]  \\
                                               &= \frac{ 1 }{ n } [ \cos(n \pi) - \cos(n \pi )] \\
                                               &= 0.
    \end{align*}
\end{proof}
    \item[(b)] For all \( n \in \N  \),
        \[  \int_{ - \pi  }^{ \pi  }  \cos^{2}(nx) \ dx = \pi \ \text{ and } \ \int_{ - \pi  }^{ \pi   } \sin^{2}(nx) \   dx = \pi.  \]
        \begin{proof}[Solution]
            Using the trigonometric identity \( \cos^{2}(x) = \frac{ 1 }{ 2 }  [ 1 + \cos(2x)] \), we can write 
            \begin{align*}
                \int_{ - \pi  }^{ \pi  } \cos^{2}(2xn) \ dx &= \frac{ 1 }{ 2 } \int_{ - \pi  }^{ \pi } [ 1+ \cos(2xn) ]   \ dx \\
                                                          &= \frac{ 1 }{ 2 } \int_{ - \pi  }^{ \pi  }  \ dx + \frac{ 1 }{ 2 }  \int_{ - \pi  }^{ \pi  } \cos(2xn) \ dx \\
            \end{align*}
            The first integral evaluates to 
            \[  \frac{ 1 }{ 2 } \int_{ - \pi  }^{ \pi  }  \ dx = \frac{ 1 }{ 2 }  [ x]_{- \pi}^{\pi } = \pi  \]
            Then, using the fact that sine is an odd function and the fact that \( \sin(2 \pi n ) = 0  \) for all \( n \in \N  \), the second integral evaluates to
            \begin{align*}
                \int_{ - \pi  }^{ \pi  } \cos(2xn) \ dx &= \frac{ 1 }{ 2n } [ \sin(2x n)]_{- \pi}^{\pi } \\
                                                       &= \frac{ 1 }{ 2n } [ \sin( 2\pi n) - \sin(- 2 \pi n)] \\
                                                       &= \frac{ 1 }{ 2n } [ \sin(2 \pi)] \tag{ \( \sin(-2 \pi n) = -\sin(2 \pi n) \)} \\
                                                       &= 0.
            \end{align*}
            Hence, we have 
            \[  \int_{ - \pi  }^{ \pi  } \cos^{2}(x) \ dx = \pi. \]

            Similarly, we have
            \begin{align*}
                \int_{ - \pi  }^{ \pi  } \sin^{2}(nx)    \ dx &= \frac{ 1 }{ 2 }  \int_{ - \pi  }^{ \pi  } [ 1 - \cos(2xn) ]  \ dx \\
                                                              &= \frac{ 1 }{ 2 } \int_{ - \pi  }^{  \pi }  \ dx - \frac{ 1 }{ 2 }  \int_{ - \pi  }^{ \pi  } \cos(2xn) \ dx \\
                                                              &= \frac{ 1 }{ 2 } \cdot 2 \pi - 0  \\
                                                              &= \pi.
            \end{align*}
        \end{proof}
    \item[(c)] For all \( m,n \in \N  \), 
        \[  \int_{ - \pi  }^{ \pi  } \cos(mx) \sin(nx)  \ dx =  0. \]
        For \( m \neq n  \), 
        \[  \int_{ - \pi  }^{ \pi  } \cos(mx) \cos(nx) \ dx = 0 \ \text{and} \ \int_{ -\pi  }^{ \pi  } \sin(mx) \sin(nx)  \ dx = 0. \]
        \begin{proof}[Solution]
        Let \( m, n \in \N  \). Suppose \( m = n  \). Then using a change of variable (letting \( u = \sin(mx) \)), we get
        \begin{align*}
            \int_{ - \pi  }^{ \pi  } \cos(mx) \sin(mx)  \ dx &= \frac{ 1 }{ m } \int_{ - \pi  }^{ \pi } u    \ du \\
                                                             &= \frac{ 1 }{ 2m } [ \sin^{2}(\pi x)]_{- \pi}^{\pi}  \\
                                                             &= \frac{ 1 }{ 2m } \sin ^{2}( \pi m) + \frac{ 1 }{ 2m } \sin^{2}(\pi m) \\
                                                             &= \frac{ 1 }{ m } \sin ^{2}(\pi m) \\
                                                             &= 0.
        \end{align*}
        If \( m \neq n  \), then using the trigonometric identity 
        \[  \cos(mx)\sin(nx) =  \frac{ 1 }{ 2 } [ \sin((m+n)x) - \sin((m-n)x) ]  \]
        we obtain
        \begin{align*}
            \int_{ - \pi  }^{ \pi  } \cos(mx) \sin(nx)  \ dx &= \frac{ 1 }{ 2 } \int_{ - \pi  }^{ \pi  } [ \sin((m+n)x) - \sin((m-n)x)] \ dx \\
                                                             &= \frac{ 1 }{ 2 } \int_{ - \pi  }^{ \pi  } \sin((m+n)x)  \ dx + \frac{ 1 }{ 2 } \int_{ - \pi  }^{ \pi  } \sin((m-n)x) \ dx \\
                                                             &=\frac{ \cos((m-n)x) }{ 2(m-n) }  \Big|_{- \pi}^{\pi } + \frac{ \cos((m+n)x) }{ 2(m+n) }  \Big|_{- \pi}^{\pi } \\ 
        \end{align*}
        Using the fact that cosine is an even function we can evaluate the first term which leads to 
        \begin{align*}
            \frac{ \cos((m-n)x) }{ 2(m-n) }  \Big|_{- \pi}^{\pi } &= \frac{ 1 }{ 2(m-n)  } [ \cos((m-n) \pi) - \cos(-(m-n) \pi)] \\
                                                                  &= \frac{ 1 }{ (m-n) } [  \cos((m-n) \pi) - \cos((m-n) \pi) ] \\
                                                                  &= 0.
        \end{align*}
        Using the same reasoning, the second term can be written as 
        \begin{align*}
            \frac{ \cos((m+n)x) }{ 2(m+n) }  \Big|_{- \pi}^{\pi } &= \frac{ 1 }{ 2(m+n)  } [ \cos((m+n) \pi) - \cos(-(m+n) \pi) ] \\
                                                                  &= \frac{ 1 }{ 2(m+n) } [ \cos((m+n)\pi) - \cos((m+n)\pi)] \\
                                                                  &= 0.
        \end{align*}
        Hence, we have
        \[  \int_{ - \pi  }^{ \pi  } \cos(mx) \sin(nx)  \ dx = 0.  \]
        
        Our goal now is to show 
        \[  \int_{ - \pi  }^{ \pi  } \cos(mx) \cos(nx) \ dx = 0 \ \text{and} \ \int_{ -\pi  }^{ \pi  } \sin(mx) \sin(nx)  \ dx = 0 \]
        for all \( m \neq n  \). Let \( m \neq  n \). Using the trigonometric identity 
        \[ \cos(a)\cos(b) = \frac{ 1 }{ 2 }  [ \cos(a+b ) + \cos(a-b)],   \]
        the first integral evaluates to
        \begin{align*}
            \int_{ -\pi  }^{ \pi } \cos(mx) \cos(nx) \ dx  &= \frac{ 1 }{ 2 } \int_{ - \pi  }^{ \pi  } [ \cos((m+n)x) + \cos((m-n)x) ]  \ dx \\
                                                           &= \frac{ 1 }{ 2 } \int_{ - \pi  }^{ \pi  } \cos((m+n)x)  \ dx + \frac{ 1 }{ 2 }  \int_{ - \pi  }^{ \pi  } \cos((m-n)x) \ dx  \\
                                                           &= \frac{ \sin((m+n)x) }{ 2(m+n) }  \Big|_{-\pi}^{\pi} + \frac{ \sin((m-n)x) }{ 2(m-n) } \Big|_{- \pi}^{\pi}. \\
        \end{align*}
        Using the fact that the sine function is odd and the fact that \( \sin((m+n) \pi) = 0  \) for all \( m, n \in \N  \), we can write     
        \begin{align*}
            \frac{ \sin((m+n)x) }{ 2(m+n) }  \Big|_{-\pi}^{\pi} &= \frac{ 1 }{ 2(m+n)  } \cdot  2\sin((m+n)\pi)  \\
                                                                &= \frac{ \sin((m+n)\pi) }{ (m+n) } \\
                                                                &= 0
        \end{align*}
       Likewise, the second term evaluates to  
       \begin{align*} 
           \frac{ \sin((m-n)x) }{ 2(m-n) } \Big|_{- \pi}^{\pi} &= \frac{ 1 }{ 2(m-n) } \cdot 2\sin((m-n)\pi) \\
                                                               &= \frac{ \sin((m-n)\pi) }{ (m-n)  }   \\
                                                               &= 0.
      \end{align*}
      Hence, we have 
      \[  \int_{ -\pi  }^{ \pi  } \cos(mx) \cos(nx)  \ dx = 0. \]
      
      Now we show the second integral. Similarly, we use the trigonometric identity
      \[ \sin(a)\sin(b) = \frac{ 1 }{ 2 }  [ \sin(a+b) + \sin(a-b) ]. \]
      Then observe that
      \begin{align*}
          \int_{ -\pi  }^{ \pi  } \sin(mx) \sin(nx)  \ dx &= \frac{ 1 }{ 2 } \int_{ - \pi  }^{ \pi  } [ \sin((m+n)x) + \sin((m-n)x)]  \ dx \\
                                                          &= \frac{ 1 }{ 2 } \int_{ - \pi  }^{ \pi  } \sin((m+n)x) \ dx + \frac{ 1 }{ 2 } \int_{ - \pi  }^{ \pi  } \sin((m-n)x) \ dx \\
                                                          &= \frac{ -\cos((m+n)x) }{ 2(m+n) } \Big|_{-\pi}^{\pi} + \frac{ \cos((m-n)x) }{ 2(m-n) } \Big|_{- \pi}^{\pi}.   
      \end{align*}
      Starting with the first term, we can utilize the fact that \( \cos(x) \) is an even function to get
      \begin{align*}
          \frac{ - \cos((m+n)x)  }{ 2(m+n)  } \Big|_{-\pi }^{\pi } &= \frac{ 1 }{ 2(m+n) } [ \cos((m+n) \pi) - \cos((m+n) \pi)] \\
                                                                   &= 0.
      \end{align*}
      Likewise, the second integral evaluates to 
      \begin{align*}
          \frac{ \cos((m-n)x) }{ 2(m-n) } \Big|_{- \pi}^{\pi} &= \frac{ 1 }{ 2(m-n) } [ \cos((m-n)\pi) - \cos((m-n)\pi)] \\
                                                              &= 0.
      \end{align*}
      Hence, we have 
      \[ \int_{ - \pi  }^{ \pi  } \sin(mx) \sin(mx)  \ dx = 0. \]

        \end{proof}
\end{enumerate}

We can interpret the integrals in exercise 8.5.2 as dot products allowing us to see that all the functions from the set 
\[  \{ 1, \cos(x) , \sin(x), \cos(2x), \sin(2x), \cos(3x), \dots \}  \]
are all \textit{orthogonal} to each other. This, in turn, forms a \textit{basis} for a large class of functions.

Our ultimate goal is to find coefficients \( (a_{n}) \) and \( (b_{n}) \) in equation (6). We can assume that we are in possession of some expression described in (6) that will allow us to find formulas for both coefficients. This is similar to how we found the coefficients of a taylor series back in chapter 6, but instead of differentiating repeatedly, we aim to integrate.

In order to compute \( a_{0} \), we need to integrate each side of (6) on the closed interval \( [ - \pi, \pi ] \) to get 
\begin{align*}
    \int_{ -\pi  }^{ \pi  } f(x) \ dx &= \int_{ -\pi  }^{ \pi  } \Big[ a_{0} + \sum_{ n=1 }^{ \infty  } a_{n} \cos(nx) + b_{n} \sin(nx) \Big] \ dx \\
                                      &= \int_{ -\pi  }^{ \pi  } a_{0} \ dx + \sum_{ n=1 }^{ \infty  } \int_{ -\pi  }^{ \pi  } [ a_{n} \cos(nx) + b_{n} \sin(mx)] \ dx \\
                                      &= a_{0}2\pi + \sum_{ n=1 }^{ \infty  } a_{n} \int_{ -\pi  }^{ \pi  } \cos(nx) \ dx + \sum_{ n=1 }^{ \infty  } \int_{ -\pi  }^{ \pi  } b_{n} \sin(mx) \ dx \\
                                      &= a_{0} 2 \pi + 0 + 0 \\
                                      &= a_{0 } 2 \pi.
\end{align*}
Solving for \( a_{0} \) gives us 
\[  a_{0} = \frac{ 1 }{ 2\pi  } \int_{ - \pi  }^{ \pi  } f(x) \ dx. \tag{9} \]
One might look at the interchanging of the integral and summation used in the derivation with suspicion. The point, as of now, is not to justify each step of the derivation, but rather work backwards so that we may find a representation to prove. The rigor will follow later.

Now suppose we fix \( m \geq 1  \). To compute \( a_{m} \), we first multiply (6) by \( \cos(mx)  \) on both sides and integrate over the closed interval \( [ -\pi, \pi]  \).

\subsubsection{Exercise 8.5.3} Derive the formulas
\[  a_{m} = \frac{ 1 }{ \pi  } \int_{ -\pi  }^{ \pi  } f(x) \cos(mx) \ dx \ \text{and} \ b_{m} = \frac{ 1 }{ \pi  } \int_{ -\pi  }^{ \pi  } f(x) \sin(mx) \ dx \] for all \( m \geq 1 \).
\begin{proof}
Let \( m \geq 1  \). Multiplying by \( \cos(mx) \) on both sides of (6) gives us 
\[  f(x) \cos(mx) = a_{0} \cos(mx) + \sum_{ n=1 }^{ \infty } a_{n}\cos(mx)\cos(nx) + b_{n} \cos(mx) \sin(nx).  \]
Now, taking the integral on the closed interval \( [- \pi , \pi ] \) on (6) gives us 
\begin{align*}
    \int_{ - \pi  }^{ \pi  } f(x) \cos(mx)  \ dx &= a_{0} \int_{ -\pi  }^{ \pi  } \cos(mx)  \ dx \\ 
                                                 &+ \sum_{ n=1   }^{ \infty  } \int_{ - \pi  }^{ \pi  } [ a_{n} \cos(mx) \cos(nx) + b_{n} \cos(mx) \sin(nx)   ] \ dx. \\
\end{align*}
Using exercise 8.5.2, we know that all the integrals in the above equation evaluate to 0 for all \( n \in \N  \), except when \( n =m  \) for which the right side of the equation evaluates to
\begin{align*}
   \int_{ - \pi  }^{ \pi  } f(x) \cos(mx)  \ dx &= \int_{ -\pi  }^{ \pi } a_{m} \cos ^{2}(mx) \ dx \\
                                                &= a_{m} \cdot \pi
\end{align*}
Solving for \( a_{m}  \) gives us our result
\[  a_{m} = \frac{ 1 }{ m } \int_{ -\pi  }^{ \pi  } \cos ^{2}(mx) \ dx. \]

Using the same reasoning to show the first integral, we have that 
\[  \int_{ -\pi  }^{ \pi  } f(x) \sin(mx)  \ dx = 0  \]
for all \( n \in \N  \) except when \( n = m  \) for which the integral above evaluates to
\begin{align*}
    \int_{ -\pi  }^{ \pi  } f(x) \sin(mx)  \ dx &=  a_{0} \int_{ - \pi  }^{ \pi  } \sin(mx) \ dx \\ 
                                                &+ \sum_{ n=1 }^{ \infty  } \int_{ -\pi  }^{ \pi }   [ a_{n} \cos(nx) \sin(mx) + b_{n} \sin(nx) \sin(mx)   ] \ dx \\
                                                &= b_{m} \int_{ -\pi  }^{ \pi  } \sin^{2}(mx) \ dx \\
                                                &= b_{m} \cdot \pi.
\end{align*}
Solving for \( b_{m} \) gives us our desired result
\[  b_{m} = \frac{ 1 }{ \pi  } \int_{ - \pi  }^{ \pi  } f(x)\sin(mx) \ dx. \]

\end{proof}

Let's see how these coefficients are computed with simple functions.

\begin{example}{}{}
    Let 
    \[  f(x) = 
    \begin{cases}
        1 \ \text{if } 0 < x < \pi \\
        0 \ \text{if } x =0 \ \text{or } x = \pi \\
        -1 \ \text{if } -\pi < x < 0.
    \end{cases} \]
    Notice that \( f \) is an odd function which means that we can evaluate our integrals easily using the symmetry argument; that is, we have
    \[  a_{0} = \frac{ 1 }{ 2 \pi  } \int_{ -\pi  }^{ \pi  } f(x)  \ dx = 0 \ \text{and} \  \frac{ 1 }{ \pi  } \int_{ -\pi  }^{ \pi  } f(x) \cos(nx)  \ dx = 0  \]
    for all \( n \geq 1  \). Using the same argument, we can simplify \( b_{n} \) to get
    \begin{align*}
        b_{n} &= \frac{ 1 }{ \pi  } \int_{ -\pi  }^{ \pi  } f(x) \sin(nx)  \ dx \\
              &= \frac{ 2 }{ \pi  } \int_{ 0  }^{ \pi  } \sin(nx) \ dx \\
              &= \frac{ 2 }{ \pi  } \Big( \frac{ -1 }{ n }  \cos(nx) \Big|_{0}^{\pi} \Big)  \\
              &= \frac{ 2 }{ \pi  } \Big( \frac{ -1 }{ n }  \cos(n \pi) + \frac{ 1 }{ n } \Big) \\ 
              &= 
              \begin{cases}
                  4 / n\pi   \ \text{if } n \text{ is odd} \\
                  0   \ \text{if } n \text{ is even} \\
              \end{cases}
    \end{align*}
    Plugging in our results into (6) then gives us the following representation of \( f(x)  \)
    \[  f(x) = \frac{ 4  }{ \pi }  \sum_{ n=0  }^{ \infty  } \frac{ 1 }{ 2n+1 } \sin((2n+1)x). \]
\end{example}

\subsubsection{Exercise 8.5.4} 
\begin{enumerate}
    \item[(a)] Referring to the previous example, explain why we can be sure that the convergence of the partial sums to \( f(x)  \) is \textit{not} uniform on any interval containing \( 0  \).
        \begin{proof}[Solution]
        At the endpoints, \( f(x)  \) takes on a different value making our convergence pointwise instead of being uniform.
        \end{proof}
    \item[(b)] Repeat the computations of Example 8.5.1 for the function \( g(x) = | x |  \) and examine graphs for some partial sums. This time, make use of the fact that \( g  \) is even (\( g(x) = g(-x) \) ) to simplify the calculations. By just looking at the coefficients, how do we know this series converges uniformly to something?
        \begin{proof}[Solution]
            Our goal is to produce coefficients \( a_{0}, a_{m},   \) and \( b_{m} \) so that we have the following representation of \( f(x)  \); that is, 
            \[  f(x) = a_{0} + \sum_{ n=1 }^{ \infty  } a_{n} \cos(nx) + b_{n} \sin(nx). \] Using the formula for \( a_{0} \) derived earlier and using the fact that \( | x  |  \) is an even function, we produce 
            \begin{align*}
                a_{0} &= \frac{ 1 }{ 2 \pi  } \int_{ -\pi  }^{ \pi  } x \ dx \\
                      &= \frac{ 1 }{ \pi  } \int_{ 0  }^{ \pi  } x  \ dx \\
                      &= \frac{ 1 }{ \pi  } \cdot \frac{ 1 }{ 2  } x^{2} \Big|_{0}^{\pi } \\
                      &= \frac{ \pi  }{ 2 }.
            \end{align*}
            Using the symmetry from the absolute value function once again, we get that computing \( a_{m} \) leads to
            \begin{align*}
                a_{m} &= \frac{ 2 }{ \pi } \int_{ 0 }^{ \pi } x \cos(mx)  \ dx \\
                      &= \frac{ 2 }{ \pi  } \Big[ \frac{ 1 }{ m } x\sin(mx)    \Big]_{0}^{\pi} - \frac{ 2 }{m \pi  } \int_{ 0 }^{ \pi  } \sin(mx) \ dx \\   
                      &=  \frac{ 2 }{ m^{2}\pi  }  \cos(mx) \Big|_{0}^{\pi} \\ 
                      &= \frac{ 2 }{ m^{2} \pi  } [ \cos(m \pi) - 1 ] \\
                      &= 
                      \begin{cases}
                          0 \ &\text{if } m \text{ even} \\
                          -4/ m^{2}\pi \ &\text{if } m  \text{ odd}.
                      \end{cases}
            \end{align*}
            Since \( \sin(mx)  \) is an odd function and \( | x  |  \) is an even function, we find that their product \( | x |  \sin(mx)  \) is also odd. Hence, the integral 
            \[ \frac{ 1 }{ \pi  }  \int_{ -\pi }^{ \pi   } f(x) \sin(mx)   \ dx  \]
            evaluates to zero. Hence, \( b_{m} = 0  \). Therefore, \( f(x) = | x  |  \) can be represented by the following Fourier series
            \[  | x |  = \frac{ \pi  }{ 2 }  - \frac{ 4  }{ \pi  }  \sum_{ n=0  }^{ \infty  } \frac{ \cos((2n+1)x)  }{ (2n+1)^{2}  }.  \] By looking at the coefficients, we can see that the convergence of \( (a_{n}) \) goes to zero and does not depend on our choice of \( x \in [-\pi , \pi ] \).
        \end{proof}
\end{enumerate}

\subsection{The Riemann-Lebesgue Lemma}

Observe that by looking at the Fourier coefficients of \( (a_{n}) \) and \( (b_{n}) \) tend to zero as we take the limit as \( n  \) approaches \( \infty  \). We will see why this works in our upcoming convergence proof.

If we take a look at any function \( h(x)  \) and \( \sin(nx) \), we will notice that when we take their product \( h(x) \sin(nx) \), the oscillatory nature of \( \sin(nx) \) do not affect the value of \( h  \) too much. This means that the value from the integral 
\[  \int_{ - \pi  }^{ \pi  } h(x) \sin(nx)  \ dx \] 
leads to a small value (because of the positive and negative oscillations of \( h(x) \sin(nx) \) leads values cancelling out).

\begin{theorem}{Riemann-Lebesgue Lemma}{}
    Assume \( h(x)  \) is continuous on \( (-\pi , \pi]  \). Then, 
    \[  \int_{ -\pi  }^{ \pi  } h(x) \sin(nx) \ dx \to 0 \ \text{ and } \ \int_{ -\pi  }^{ \pi  } h(x) \cos(nx)  \ dx \to 0 \]
    as \( n \to \infty  \).
\end{theorem}

\begin{proof}
    Using our assumption that \( h  \) is continuous on the interval \( (- \pi , \pi]  \), we can periodically extend \( h  \) to be continuous on all of \( \R  \). This implies that we have to assume that \( \lim_{ n \to - \pi ^{+} } h(x) = h(\pi) \) holds.
\end{proof}

\subsubsection{Exercise 8.5.5} Explain why \( h  \) is uniformly continuous on \( \R  \).
\begin{proof}[Solution]
    We know that \( h  \) is uniformly continuous on \( \R  \) since we can extend the continuity of \( h  \) to any interval \( (-n\pi, n\pi]   \) for all \( n\in \Z^{+} \) as well as having the implication that \( \lim_{ x \to - n \pi^{+} } h(x) = h(n\pi) \).
\end{proof}

Given \( \epsilon > 0 \), choose \( \delta > 0  \) such that \( | x - y  | < \delta  \) implies \( | h(x) - h(y) | < \frac{ \epsilon }{ 2 }  \). Notice that the period of \( \sin(nx) \) is \( 2 \pi / n  \). Hence, we can choose an \( N  \) large enough such that \( \pi / n < \delta  \) whenever \( n \geq N  \). Now
consider a particular interval \( [a,b]  \) of length \( 2 \pi / n  \) over which \( \sin(nx) \) moves through one complete oscillation.

\subsubsection{Exercise 8.5.6} Show that \( \Big| \int_{ a }^{ b } h(x) \sin(nx) \ dx  \Big| < \epsilon / n  \), and use this fact to complete the proof.

\begin{proof}
    Let \( \epsilon > 0  \). Choose \( \delta = \min \{ \frac{ 1 }{ 2n } , \frac{ \epsilon  }{ 4mN  }   \} > 0   \) such that \(  | x - y  |  < \delta  \) implies \( | h(x) - h(y)  | < \epsilon / 2  \). Using the period of \( \sin(nx)  \) which is \( 2 \pi / n  \), we can choose an \( N  \) large enough such that \( \pi / n  < \delta  \) whenever \( n \geq N  \). Then using the length of \([a,b]   \) given to us above and the fact that \( | \sin(nx) | \leq 1  \), we can write  
    \[  \int_{ a }^{ b } | h(x)  |  | \sin(nx) |  \ dx \leq \int_{ a }^{ b } | h(x) |  \ dx. \]
    Since \( h  \) is integrable, we know that \( h  \) must also be bounded. Hence, there exists an \( M > 0  \) such that \( | h(y) | \leq M  \) for all \( y \in [a,b] \). Now observe that  
    \begin{align*}
        \Big| \int_{ a }^{ b } h(x) \sin(nx) \ dx  \Big| &\leq \int_{ a }^{ b } | h(x)  |  | \sin(nx) |  \ dx \\
                                                         &= \int_{ a }^{ b } | h(x) |  \ dx \\
                                                         &< \Big( \frac{ \epsilon  }{ 2 } + | h(y) |  \Big) \int_{ a }^{ b }  \ dx \\
                                                         &= \Big( \frac{ \epsilon  }{ 2 } + | h(y) |  \Big)(b-a) \\
                                                         &= \frac{ \epsilon  }{ 2 }  (b-a) + | h(y) | (b-a) \\
                                                         &\leq \frac{ \epsilon  }{ 2 } (b-a) + M (b-a) \\
                                                         &< \frac{ \epsilon  }{ 2  } \cdot \frac{ 2 \pi  }{ n }  + M \cdot \frac{ 2 \pi  }{ n } \\
                                                         &< \frac{ \epsilon  }{ 2n }  + \frac{ \epsilon  }{ 2  n  } \\
                                                         &= \frac{ \epsilon  }{ n }.
    \end{align*}
    Hence, we have that 
    \[  \Big| \int_{ a }^{ b } h(x) \sin(nx) \ dx \Big| < \frac{ \epsilon  }{ n }. \] We can use the same reasoning to show that if \( | \cos(nx) | \leq 1  \), then 
    \[  \Big| \int_{ a }^{ b } h(x) \cos(nx) \ dx \Big|  < \frac{ \epsilon  }{ n }. \]
    Letting \( \epsilon > 0  \), we know that \( \epsilon / n  \to 0  \) as \( n \to \infty  \). This tells us that we can find an \( N \in \N  \) such that for any \( n \geq N  \), we have 
    \[  \Big| \int_{ a }^{ b } h(x) \sin(nx) \ dx \Big| < \frac{ \epsilon  }{ n }  \to 0  \]
    and 
    \[  \Big| \int_{ a }^{ b } h(x) \cos(nx)  \ dx \Big| < \frac{ \epsilon  }{ n } \to 0. \]
\end{proof}

In attempt to make the proof above shorter, we can just say that since \( h  \) is uniformly continuous on \( \R  \), we know that \( h  \) must be bounded on any compact interval \( [a,b] \subseteq \R  \). Hence, there must exists an \( M > 0 \) such that \( | h(x)  | \leq M  \) for any \(  x \in [a,b]  \). Using the same set of assumptions above and letting \( \delta = \epsilon  / 4 M n  \), we can write 
\begin{align*}
    \Big| \int_{ a }^{ b } h(x) \sin(nx)  \ dx \Big| &\leq \int_{ a }^{ b } | h(x)  | | \sin(nx) |  \ dx \\
                                                     &\leq M \int_{ a }^{ b } | \sin(nx)  |  \ dx \\
                                                     &\leq M \int_{ a }^{ b }  \ dx \\
                                                     &= M (b-a) \\
                                                     &< M \cdot \frac{ 2 \pi    }{ n } \\
                                                     &< 2M \cdot \frac{ \epsilon  }{ 2 M n } = \frac{ \epsilon  }{ n }. 
\end{align*}


Note that the lemma above still holds even if we take out the continuity assumption. This makes the use of Fourier series much more versatile in terms of its application to a wider variety of functions and is capable of more interesting behavior than our power series counterpart.


\subsection{A Pointwise Convergence Proof}

The formulas for the Fourier coefficients that we have produced so far require the implicit assumption that our function needs to be Riemann Integrable. This tells us that it is advantageous to have our class of integrable functions to be as large as possible in order to properly represent \( f(x)  \) in terms of (6). The goal now is to determine how many more assumptions do we need in order to have Fourier series converge back to \( f(x)  \). This involves determining which type of convergence we hope to establish.

Consider 
\[  f(x) = a_{0} + \sum_{ n=1 }^{ \infty  } a_{n} \cos(nx) + b_{n} \sin(nx) \] once again.
On the left-hand side, we would like to have \( f(x)  \) be 
\begin{enumerate}
    \item[(i)] bounded,
    \item[(ii)] integrable,
    \item[(iii)] continuous,
    \item[(iv)] differentiable 
    \item[(vi)] \( f'  \) continuous.
\end{enumerate}
Suppose on the right-hand side, we list the types of convergences needed to properly represent \( f(x)  \) using (6). These types of convergences include 
\begin{enumerate}
    \item[(i)] pointwise convergence
    \item[(ii)] uniform convergence
    \item[(iii)] \( L^{2}  \) convergence, and
    \item[(iv)] Cesaro mean convergence.
\end{enumerate}

\begin{theorem}{}{}
    Let \( f(x)  \) be continuous on \( (- \pi , \pi]  \), and let \( S_{N}(x)  \) be the \( N \)th partial sum of the Fourier series described in equation (7), where the coefficients \( (a_{n}) \) and \( (b_{n}) \) are given by equations (9) and (10). It follows that 
    \[  \lim_{ N \to \infty  }  S_{N}(x) = f(x)  \] 
    pointwise at any \( x \in (-\pi, \pi]   \) where \( f'(x)  \) exists.
\end{theorem}

Before we embark on the proof, let us get a few known facts out of the way.
\begin{enumerate}
    \item[(i)]  \( \cos(\alpha - \theta) = \cos(\alpha) \cos(\theta) + \sin(\alpha ) \sin(\theta)  \).
    \item[(ii)] \( \sin(\alpha + \theta) = \sin(\alpha) \cos(\theta) + \cos(\alpha) \sin(\theta). \)
    \item[(iii)] For any \( \theta \neq 2n \pi \), we have  
        \[ \frac{ 1 }{ 2 } + \cos(\theta) + \cos(2 \theta) + \cos(3 \theta) + \dotsb  + \cos(N \theta) = \frac{ \sin((N+1/2) \theta)  }{ 2 \sin(\theta / 2 )  }.   \] This function is called the \textit{Dirichlet Kernel}. The proof of this fact will be omitted because it involves complex analysis; that is, it results from taking the real part of a geometric sum of complex exponentials. 
    \item[(iv)] Letting 
        \[  D_{N}(\theta) = 
        \begin{cases}
            \frac{ \sin((N+1/2) \theta) }{ 2 \sin(\theta / 2 )  } , &\text{if } \theta \neq 2 n \pi \\         
            1 / 2 + N, &\text{if } \theta = 2n\pi 
        \end{cases} 
    \]
    from (iii), we can see that
    \[  \int_{ - \pi  }^{ \pi  } D_{N}(\theta)  \ d\theta = \pi. \]
\item[(v)] The Riemann-Lebesgue Lemma.
\end{enumerate}

Let us fix a point \( x \in  (- \pi , \pi] \). Our first goal is to simplify \( S_{N}(x)  \). Since \( x  \) is a fixed constant, we can write equations (9) and (10) using \( t  \) as the "dummy" variable for integration. Using facts (i) and (iii), we can see that 
\begin{align*}
    S_{N}(x) &= a_{0 } + \sum_{ n=1  }^{ N  }a_{n} \cos(nx) + b_{n} \sin(nx) \\
             &=  \Big[ \frac{ 1 }{  2 \pi  } \int_{ -\pi  }^{ \pi  } f(t)  \ dt \Big] + \sum_{ n=1  }^{ N  } \Big[\frac{ 1 }{ \pi  } \int_{ -\pi  }^{ \pi  } f(t) \cos(nt)  \ dt \Big] \cos(nx)  \\
             &+ \sum_{ n=1  }^{ N  } \Big[ \frac{ 1 }{ \pi  } \int_{ - \pi  }^{ \pi  } f(t) \sin(nt) \ dx  \Big] \sin(nx) \\ \\
             &= \frac{ 1 }{ \pi  } \int_{ - \pi  }^{ \pi  } f(t) \Big[ \frac{ 1 }{ 2 }  + \sum_{ n=1  }^{ N  } \cos(nt) \cos(nx) + \sin(nt) \sin(nx) \Big] \ dt  \\
             &= \frac{ 1 }{ \pi  } \int_{ -\pi  }^{ \pi  } f(t) \Big[\frac{ 1 }{ 2 } + \sum_{ n =1  }^{ N  } \cos(n( t - x)) \Big] \ dt \\
             &= \frac{ 1 }{ \pi  } \int_{ -\pi  }^{ \pi  } f(t) D_{N}(t-x) \ dx.
\end{align*}
Letting \( u = t - x  \) and using the fact that \( f  \) is extended to be \( 2 \pi-\)periodic, we can express \( S_{N}(x)  \) in terms of \( D_{N}(u)  \). By fact (iv), it follows that
\[  S_{N}(x) = \frac{ 1 }{ \pi  } \int_{ -\pi - x  }^{ \pi - x  } f(u+x) D_{N}(u) \ du = \frac{ 1 }{ \pi  } \int_{ -\pi  }^{ \pi  } f(u+x) D_{N}(u)  \ du. \]
Since \( D_{N} \) is periodic, we know that our computation of the integral is independent of the choice of interval granted that we cover one full period.
    Now showing that \( S_{N}(x) \to f(x)  \) is a matter of showing \( | S_{N}(x) - f(x)  |  \) gets arbitrarily small when we let \( N \to \infty  \). Using (iv), we can express \( f(x)  \) as we have done with \( S_{N}(x)  \) in terms of \( D_{N}(u)  \). Hence, we have
    \[  f(x) = \frac{ f(x)  }{ \pi  } \int_{ -\pi  }^{ \pi  } D_{N}(u)  \ du = \frac{ 1 }{ \pi  } \int_{ - \pi  }^{ \pi  } f(x) D_{N}(u)  \ du \] and thus we have 
    \[  S_{N}(x) - f(x) = \frac{ 1 }{ \pi  } \int_{ -\pi  }^{ \pi  } (f(u+x) - f(x)) D_{N}(u) \ du. \tag{11} \]

We can use (ii), to rewrite the Dirichlet kernel \( D_{N}(u)  \) as 
\begin{align*}
    D_{N}(u) &= \frac{ \sin((N+1/2)u) }{ 2 \sin(u/2) }   \\
             &= \frac{ \sin(Nu) \cos(u/2) + \cos(Nu) \sin(u/2) }{ 2 \sin(u/2) } \\
             &= \frac{ 1 }{ 2 } \Big[ \frac{ \sin(Nu) \cos(u/2) }{ \sin(u/2) } + \cos(Nu)\Big]. \\
\end{align*}

Now observe that
\begin{align*}
    S_{N}(x) - f(x) &= \frac{ 1 }{ 2 \pi  } \int_{ - \pi  }^{ \pi  } (f(u+x) - f(x) ) D_{N}(u)     \ du \\ 
                    &= \frac{ 1 }{  2 \pi  } \int_{ -\pi  }^{ \pi  } (f(u+x) - f(x) ) \Big[ \frac{ \sin(Nu) \cos(u/2) }{ \sin(u/2) } + \cos(Nu) \Big] \ du \\
                    &= \frac{ 1 }{ 2\pi  } \int_{ -\pi  }^{ \pi  } (f(u+x) - f(x) ) \Big( \frac{ \sin(Nu) \cos(u/2) }{ \sin(u/2) }  \Big)  \\ 
                    &+ (f(u+x) - f(x)) \cos(Nu)  \ du \\
                    &= \frac{ 1 }{ 2 \pi  } \int_{ - \pi  }^{ \pi  } p_{x}(u) \sin(Nu)  \ du + \frac{ 1 }{ 2 \pi  } \int_{ -\pi  }^{ \pi  } q_{x}(u) \cos(Nu)  \ du,
\end{align*}
where in the last equality, we set 
\[ p_{x}(u) = \frac{ (f(u+x) - f(x)) \cos(u/2) }{ \sin(u/2) } \ \text{ and } \ q_{x}(u) = f(u+x) - f(x).    \]
 
\subsubsection{Exercise 8.5.7}  
\begin{enumerate}
    \item[(a)] First, argue why the integral involving \( q_{x}(u)  \) tends to zero as \( N \to \infty  \).
        \begin{proof}
        
        \end{proof} 
    \item[(b)] The first integral is a little more subtle because the function \( p_{x}(u)  \) has the \( \sin(u/2)  \) term in the denominator. Use the fact that \( f \) is differentiable at \( x  \) (and a familiar limit from calculus) to prove that the first integral goes to zero as well.
        \begin{proof}
        
        \end{proof}
\end{enumerate} 






