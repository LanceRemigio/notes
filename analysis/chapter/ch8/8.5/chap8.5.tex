\section{Fourier Series}

Fourier's development of a special type of series involving trigonometric functions caused a rework into what it really means to have a "limit" or to "converge" towards a specific value.

\subsection{Trigonometric Series}

The basic use of series representations is to express a given function \( f(x)  \) in terms of the sum of simpler and "nicer" functions such as polynomials. In the case of power series representations, we have the following set of component functions \( \{ 1, x, x^{2}, x^{3}, \dots \}  \) such that the following series takes the form 
\[  f(x) = \sum_{ n=0  }^{ \infty  } a_{n} x^{n} = a_{0} + a_{1} x + a_{2} x^{2} + a_{3} x^{3} + \dotsb / . \]
A \textit{trigonometric series} on the other hand is an infinite series composed of component functions that are trigonometric; that is, we have 
\[  \{ 1 , \cos(x), \sin(x), \sin(2x), \cos(3x), \sin(3x), \dots \}.  \]

