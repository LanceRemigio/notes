\section{Metric Spaces}

In this section, we aim to give a more generalized view of what it means to have a "distance" over sets other than \( \R  \). Do the theorems and properties we have proved about sequences, series, and functions carry over to sets like \( \R^{2} \) or even in higher dimensions like \( \R^n \)? We will be mainly testing our notions that we have developed throughout the book on sets such as \( \R^{2} \) and \( C[0,1] \), the space of continuous functions on \( [0,1] \).


\begin{definition}{Metric Spaces}{}
Given a set \( X  \), a function \( d: X \times X \to \R  \) is a \textit{metric} on \( X  \) if for all \( x,y \in X  \):
\begin{enumerate}
    \item[(i)]\( d(x,y) \geq 0  \) with \( d(x,y) =0  \) if and only if \( x = y  \),
    \item[(ii)] \( d(x,y) = d(y,x)  \), and 
    \item[(iii)] for all \( z \in X \), \( d(x,y) \leq d(x,z) + d(z,y) \).
\end{enumerate}
A \textit{metric space} is a set \( X  \) together with a metric \( d  \).
\end{definition}

\begin{itemize}
    \item Property (iii) in the definition above is just the triangle inequality.
    \item The set \( X  \) can have different metrics on it.
    \item Whenever a metric space is mentioned, we usually specify what metric are using.
\end{itemize}

\subsubsection{Exercise 8.2.1} Decide which of the following are metrics on \( X = \R^2  \). For each, we let \( x = (x_{1}, x_{2}) \) and \( y = (y_{1}, y_{2}) \) be points in the plane.
\begin{enumerate}
    \item[(a)] \( d(x,y) = \sqrt{ (x_{1}- y_{1})^2 + (x_{2} - y_{2})^{2} }. \)
    \item[(b)] \( d(x,y) = \max \{ | x_{1} - y_{1} |, | x_{2} - y_{2} |  \}. \)
    \item[(c)] \( d(x,y) = | x_{1} x_{2} + y_{1} y_{2} |  \).
\end{enumerate}

\begin{proof}[Solution]
    \begin{enumerate}
        \item[(a)]  We claim that \( d(x,y)  \) is a metric on \( X = \R^{2} \). Let \( x', y' \in \R^{2} \) where \( x = (x_{1}, x_{2}) \) and \( y = (y_{1}, y_{2}) \). For part (i), suppose \( x' \neq y' \). Then observe that by property of the square root, we know that \( d(x,y) >  0  \). Otherwise, \( d(x,y) = 0  \). 

            For part (ii), observe that 
            \begin{align*}
                d(x,y) &= \sqrt{ (x_{1} - y_{1})^2 + (x_{2} - y_{2})^{2} }  \\
                       &= \sqrt{ (y_{1} - x_{1})^2 + (y_{2} - x_{2})^{2} } \\
                       &= d(y,x).
            \end{align*} 
        For part(iii), let \( x,y, z \in \R^2  \). Then observe that 
        \begin{align*}
            d(x,y) &= \sqrt{ (x_{1} - y_{1})^2 + (x_{2} - y_{2})^2  }  \\
                   &= \sqrt{ (x_{1} - z_{1})^2 + (x_{2} - z_{2})^{2} + (z_{1} - y_{1})^{2} + (z_{2} - y_{2})^{2}  } \\ 
                   &\leq \sqrt{ (x_{1} - z_{1})^2 + (x_{2}  - z_{2})^2   }  + \sqrt{ (z_{1} - y_{1})^2 + (z_{2} - y_{2})^{2} } \\
                   &= d(x,z) + d(z,y).
        \end{align*}
        Hence, we conclude that \( d(x,y)  \) is a metric on \( \R^{2}  \).
    \item[(b)] We have \( d(x,y) = \max \{ | x_{1} - y_{1}  |, | x_{2} - y_{2} |  \}  \) is a metric on \( \R^2  \). 
        For property (i), observe that \( d(x,y) > 0  \) if either \( x \geq y  \) or \( x < y  \). This holds because \( |  \cdot | > 0  \). If \( x = y  \), then it follows immediately that \( d(x,y) = 0   \). To show the triangle inequality, we will use the formula 
        \[ \max \{ a,b \}  = a + b + | | a | - | b |  |.  \]
        Then observe that for any \( x,y,z \in \R^2 \), we have 
        \begin{align*}
            d(x,y) &= \max \{ | x_{1} - y_{1} | , | x_{2} - y_{2} |  \}  \\
                   &= \frac{ 1 }{ 2 } \Big[ | x_{1} - y_{1} |  + | x_{2} - y_{2} |  + | | x_{1} -  x_{1}  | + | x_{2} - y_{2} |   |  \Big] \\
                   &\leq \frac{ 1 }{ 2 } \Big[ | x_{1} - z_{1} | + | z_{1} - y_{1}   |   + | x_{2} - z_{2} | + | z_{2} - y_{2} |  \\ 
                   &+ \Big| | x_{1} -  z_{1}  | + | z_{1} - y_{1} |  - | x_{2} - z_{2} | + | z_{2} - y_{2} |    \Big|  \Big] \\
                   &= \frac{ 1 }{ 2 }  \Big[  | x_{1} - z_{1}  |  + | x_{2} - z_{2} | + \Big| | x_{1} - z_{1} |  - | x_{2} - z_{2} |  \Big|  \Big] \\ 
                   &+ \frac{ 1 }{ 2 }  \Big[ | z_{1} - y_{1} | + | z_{2} - y_{2} | + \Big| | z_{1} - y_{1}  |  - | z_{2} - y_{2} |  \Big|   \Big] \\
                   &= \max \{ | x_{1} -  z_{1} |, | x_{2} - z_{2} |  \} + \max \{ | z_{1} - y_{1} |, | z_{2} - y_{2} |  \} \\ 
                   &= d(x,z) +  d(z,y). 
        \end{align*}
    \item[(c)] \( d(x,y) = | x_{1}x_{2} + y_{1}y_{2} |  \) cannot be a metric since \( d(x,y) \neq 0  \) for all \( x,y \in \R^{2} \).
    \end{enumerate}
\end{proof}

\begin{itemize}
    \item The metric in part (a) is the Euclidean distance between two points in a plane.
    \item \( d(x,y) = |   x - y  |   \) is a metric over \( \R  \) (the main metric we have been working with throughout the book).
\end{itemize}

\subsubsection{Exercise 8.2.2} Let \( C[0,1]  \) be the collection of continuous functions on the closed interval \( [0,1]  \). Decide which of the following are metrics on \( C[0,1] \).
\begin{enumerate}
    \item[(a)] \( d(f,g) = \sup \{ | f(x) - g(x)  | : x \in [0,1] \}  \).
    \item[(b)] \( d(f,g) = | f(1) - g(1) |  \).
    \item[(c)] \( d(f,g) = \int_{ 0 }^{ 1 }  | f - g |   \).
\end{enumerate}

\begin{proof}[Solution]
\begin{enumerate}
    \item[(a)] Observe that for any two functions \( f, g \in C[0,1] \) that are distinct, we know that 
        \[  d(f,g) = \sup \{ | f(x) - g(x)  |: x \in [0,1]  \} \geq | f(x) - f(x)  | > 0. \]
        If \( f = g  \), then it immediately follows that \( d(f,g) = 0  \). Hence, property (i) is satisfied.  

        Observe that part (ii) is satisfied by taking 
        \begin{align*}  d(f,g) &= \sup \{ | f(x) - g(x)  | : x \in [0,1] \} \\ &= \sup \{ | g(x) - f(x)  | : x \in [0,1] \} \\  &= d(g,f). \end{align*}

        For part (iii), let \( f,g,h \in C[0,1] \), then we must have 
        \begin{align*}
            d(f,g) &= \sup | f(x) - g(x) |  \\
                   &= \sup | f(x) - h(x) + h(x) - g(x) | \\
                   &\leq \sup | f(x) - h(x)  | + \sup | h(x) - g(x) | \\
                   &= d(f,h) + d(h,g).
        \end{align*}
    \item[(b)] The first property fails (take \( f(1) = 1  \) and \( g(x) = x  \)).
    \item[(c)] We claim that \( d(f,g) = \int_{ 0 }^{ 1 } | f - g  |     \) is a metric on \( \R^{2} \).    Note that for any two distinct functions \( f,g \in C[0,1]  \), we must have \( | f -g  | > 0  \). By exercise 7.4.4, we must have \( \int_{ 0 }^{ 1 }  | f - g  |   > 0  \). Otherwise, \( f = g  \) implies \( \int_{ 0 }^{ 1 }  | f - g  |  = 0  \). If \( \int_{ 0 }^{ 1 }  | f -g  |  =0  \), then we must have \( | f -g  | = 0  \) and hence, \( f = g  \). To show the triangle inequality, let \( f,g,h \in C[0,1] \) be integrable (since they are part of a set of continuous functions that are bounded). Hence, observe that 
        \begin{align*}
            d(f,g) &= \int_{ 0 }^{ 1 } | f -g  |    \\
                   &\leq \int_{ 0 }^{ 1 }  | f- h  |  + | h - g  |  \\
                   &= \int_{ 0 }^{ 1 } | f -h  | + \int_{ 0 }^{ 1 } | h - g  |  \\
                   &= d(f,h)  + d(h,g).
        \end{align*}
\end{enumerate}
\end{proof}

Define the \textit{discrete metric} on any set \( X  \) where for any \( x, y \in X  \), let 
\[  \rho (x,y) = 
\begin{cases}
    1 \  &\text{ if } x \neq y \\
    0 \ &\text{ if } x = y.
\end{cases}  \]

\subsubsection{Exercise 8.2.3} Verify that the discrete metric is actually a metric.
\begin{proof}[Solution]
Observe that if \( x \neq y  \), then by definition we must have \( \rho(x,y) > 0  \). Otherwise, \( \rho(x,y) = 0  \) by definition. It is clear that \( \rho(x,y) = \rho(y,x) \). To show the triangle inequality, let \( x,y,z \in X  \), then we must have 
\begin{align*}
    \rho(x,y) &= 1 + 0   \\
              &\leq 1 + 1 \\
              &= \rho(x,z) + \rho(z,y). 
\end{align*}
Hence, \( \rho(x,y)  \) is a metric on any arbitrary set \(  X \).
\end{proof}

\subsection{Basic Definitions}

\begin{definition}{Convergence In A General Metric Space}{}
Let \( (X,d)  \) be a metric space. A sequence \( (x_{n}) \subseteq  X  \) \textit{converges} to an element \( x \in X  \) if for all \( \epsilon > 0  \) there exists an \( N \in \N  \) such that \( d(x_{n} ,x )  < \epsilon \) whenever \( n \geq N  \).
\end{definition}


\begin{definition}{Cauchy Sequences}{}
A sequence \( (x_{n}) \) in a metric space is a \textit{Cauchy sequence} if for all \( \epsilon > 0  \), there exists an \( N \in \N  \) such that \( d(x_{m}, x_{n}) < \epsilon \) whenever \( m, n \geq N  \).
\end{definition} 

\subsubsection{Exercise 8.2.4} Show that a convergent sequence is Cauchy.
\begin{proof}
    Since \( (x_{n}) \subseteq \) is a Cauchy sequence, we can pick an \( N \in \N  \) such that for any \( n,m \geq N  \), we must have 
    \[  d(x_{n}, x ) < \frac{ \epsilon  }{ 2  }  \ \text{ and } \ d(x, x_{m}) < \frac{ \epsilon  }{ 2 } .\] Using the same choice of \( N \in \N  \) so that \( n,m \geq N  \), we must have that
    \begin{align*}
        d(x_{n}, x_{m}) &\leq d(x_{n}, x )  + d(x, x_{m})\\ 
                        &< \frac{ \epsilon  }{ 2 }  + \frac{ \epsilon  }{ 2 }  \\
                        &= \epsilon.
    \end{align*}
    Hence, \( (x_{n}) \) is Cauchy.
\end{proof}

\begin{itemize}
    \item Notice that this is only the forwards direction of the Cauchy Criterion we studied under \( \R  \). 
    \item For metric spaces other than \( \R  \), the converse of the Cauchy Criterion does not necessarily hold.
    \item We need to develop an ordering of our space similar to how the Axiom of Completeness is used in \( \R  \) ( This is called \textit{completeness}  ). 
    \item The convergence of Cauchy sequences is taken to be the definition of completeness.
\end{itemize}


\begin{definition}{Complete Metric Spaces}{}
A metric space \( (X, d ) \) is \textit{complete} if every Cauchy sequence in \( X  \) converges to an element of \( X  \).
\end{definition}


\subsubsection{Exercise 8.2.5} 
\begin{enumerate}
    \item[(a)] Consider \( \R^{2} \) with the discrete metric \( \rho(x,y)  \) examined in Exercise 8.2.3. What do Cauchy sequences look like in this pace? Is \( \R^{2} \) complete with respect to this metric? 
        \begin{proof}[Solution]
        Cauchy sequences under the discrete metric under \( \R^{2} \) would have \( x_{n} = (x_{n_{1}}, x_{n_{2}}) \) and \( x_{m} = (x_{m_{1}}, x_{m_{2}}) \) such that 
        \[ \rho(x_{n}, x_{m}) = 
        \begin{cases}
            0 \ &\text{if } x_{n } = x_{m} \\
            1 \ &\text{if } x_{n_{i} } \neq  x_{m_{i}} \ \text{where } 1 \leq i \leq 2.
        \end{cases} \]
        Yes, \( \rho(x,y)  \) is complete under \( \R^{2} \).
        \end{proof}
    \item[(b)] Show that \( C[0,1]  \) is complete with respect to the metric in Exercise 8.2.2(a).
        \begin{proof}
        The metric from Exercise 8.2.2 (a) is 
        \[  d(f,g) = \sup_{x \in [0,1]} | f(x) - g(x) |.    \]
        Let \( f_{n}, f_{m} \in C[0,1] \). The Cauchy sequence under sup norm metric will be 
        \[  d(f_{n}, f_{m}) = \sup_{x \in [0,1]} | f_{n}(x) - f_{m}(x) | \]
        We want to show that the Cauchy sequence of functions \( (f_{n}) \) converges under \( C[0,1] \). Since \( (f_{n}) \) is a Cauchy sequence under \( \R  \), we know that is satisfies the Cauchy Criterion. Hence, \( (f_{n}) \) must converge uniformly. By choosing \( N \in \N  \), we can let \( m, n \geq N  \) and \( x \in [0,1]  \) such that 
        \begin{align*}
            d(f_{n}, f) &= \sup_{x \in [0,1]} | f_{n}(x) - f(x)  |   \\
                        &\leq \sup_{x \in [0,1] } | f_{n}(x) - f_{m}(x)  | + \sup_{x \in [0,1]} | f_{m}(x) - f(x)  | \\
                        &< \frac{ \epsilon  }{ 2  }  + \frac{ \epsilon  }{ 2 } = \epsilon.
        \end{align*}
        Hence, the sup norm metric is complete under \( C[0,1] \).
        \end{proof}
    \item[(c)] Define \( C^1[0,1] \) to be the collection of differentiable functions on \( [0,1]  \) whose derivatives are also continuous. Is \( C^{1}[0,1] \) complete with respect to the metric defined in Exercise 8.2.2(a)?
        \begin{proof}
            No, \( C^{1}[0,1] \) is not complete under metric defined in Exercise 8.2.2 (a). Define 
            \[h'_{n} = \frac{ x  }{ \sqrt{ x^{2} + 1/n  }  }. \]
            Note that the convergence of \( h'n \to h  \) where \( h(x) = x / | x |  \) is not uniform. Hence, we cannot have completeness on \( C^{1}[0,1]  \) when we have pointwise convergence instead of uniform convergence.
        \end{proof}
\end{enumerate}

The sup metric is usually written as 
\[  \lVert f -g  \rVert_{\infty }  = d(f,g) = \sup \{ | f(x) - g(x)  | : x \in [0,1] \}  \] and setting \( g = 0  \) gives us the "sup norm" 
\[  \lVert f \rVert_{\infty } = d(f,0 ) = \sup \{ | f(x)  | : x \in [0,1] \} . \]
From now on, we will assume that the space \( C[0,1]  \) is paired with the metric above unless otherwise specified.

\begin{definition}{Continuity}{}
Let \( (X, d_{1}) \) and \( (Y, d_{2}) \) be metric spaces. A function \( f: X \to Y  \) is \textit{continuous} at \( x \in  X  \) if for all \(\epsilon > 0  \), there exists a \( \delta > 0  \) such that \( d_{2}(f(x) , f(y)) < \epsilon  \) whenever \( d_{1} (x,y) < \delta \).  
\end{definition}

\subsubsection{Exercise 8.2.6} Which of these functions from \( C[0,1] \) to \( \R  \) (with the usual metric) are continuous? 
\begin{enumerate}
    \item[(a)] \( g(f) = \int_{ 0 }^{ 1 } f k  \) where \( k   \) is some fixed function in \( C[0,1]  \).
        \begin{proof}[Solution]
            We claim that \( g(f) = \int_{ 0 }^{ 1 }  fk  \) where \(  k  \) is some fixed function in \( C[0,1] \). Let \( \epsilon > 0  \). Under the usual metric under \( \R  \), suppose there exists a \( \delta > 0  \) such that \( | x -c  | < \delta  \). Since \( f \in C[0,1]  \), \( f  \) is also continuous. Hence, we can use the same \( \delta \) such that 
            \[  | f(x) - f(c)  | < \epsilon. \] Since \( k  \) is a fixed function in \( C[0,1]  \), we must have 
            \begin{align*}
                | g(f(x)) - g(f(c))  | &= \Big| k \int_{ 0 }^{ 1 } ( f(x) - f(c))  \ dx   \Big| \\
                                       &\leq M \int_{ 0 }^{ 1 }  | f(x) - f(c)  | \ dx \tag{\( k  \) is bounded} \\
                                       &< M \int_{ 0 }^{ 1 }  \frac{ \epsilon  }{ M   }  = \epsilon. \\ 
            \end{align*}
            Hence, \( g(f)  \) is continuous in \( C[0,1] \).
        \end{proof}
    \item[(b)] \( g(f) = f(1/2) \).
        \begin{proof}[Solution]
            Let \( \epsilon > 0   \). Since \( f \in C[0,1] \), we know that \( f  \) must be continuous. Hence, we can choose a \( \delta > 0  \) such that for any \( | x -c  | < \delta  \), we have
            \[  | g(f(x)) - g(f(c))  | = | f(1/2) - f(1/2)  | = 0 < \epsilon. \]
            Hence, \( g  \) is continuous on \( C[0,1] \).
        \end{proof}
    \item[(c)] \( g(f) = f(1/2)  \), but this time with respect to the metric on \( C[0,1]  \) from Exercise 8.2.2 (c).
        \begin{proof}[Solution]
        Not continuous. Let \( f = 0  \) and let \( \delta >0  \). Now define 
        \[  h_{\delta}(x) = 
        \begin{cases}
            1 /2 \ & x \in V_{\delta}(1/2) \\ 
            0 \ &\text{ otherwise }.
        \end{cases} \]
        Observe that for any \( \delta > 0  \), we have that \( d(h_{\delta}, f ) = \delta \). Using the metric from part (c), we will end up with \( d(h_{\delta}, h)= (h-f) (1/2) = 1/2   \). Thus, we can't satisfy \( \epsilon < 1/2. \)
        \end{proof}
\end{enumerate}


\subsection{Topology on Metric Spaces}

\begin{definition}{\( \epsilon - \)neighborhoods}{}
    Given \( \epsilon > 0  \) and an element \( x  \) in the metric space \( (X,d) \), the \( \epsilon- \)\textit{neighborhood of} \( x  \)  is the set 
    \[  V_{\epsilon }(x) = \{ y \in X : d(x,y) < \epsilon \}. \]
\end{definition}

\subsubsection{Exercise 8.2.7} Describe the \( \epsilon - \)neighborhoods in \( \R^{2} \) for each of the different metrics described in Exercise 8.2.1. How about the discrete metric?
\begin{proof}[Solution]

\end{proof}

Now we are able to define \textit{open sets, limit points,} and \textit{closed sets} like we did before on \( \R  \) but this time with more general spaces and different metrics. 
Reframing our definitions of these concepts in \( \R  \) in terms of a general space \( X  \), we 
\begin{itemize}
    \item call a set \( O \subseteq X  \) \textit{open} if for every \( x \in O  \) we can find a neighborhood \( V_{\epsilon }(x) \subseteq O  \). 
    \item A point \( x  \) is a \textit{limit point} of a set \( A  \) if every \( V_{\epsilon }(x)  \) intersects \( A  \) in some point other than \( x  \). A set \( C  \) is \textit{closed} if it contains its limit points.
\end{itemize}

\subsubsection{Exercise 8.2.8} Let \( (X,d) \) be a metric space.
\begin{enumerate}
    \item[(a)] Verify that a typical \( \epsilon- \)neighborhood \( V_{\epsilon }(x)  \) is an open set. Is the set 
        \[  C_{\epsilon }(x) = \{ y \in X : d(x,y) \leq \epsilon \}  \] a closed set? 
        \begin{proof}[Solution]
        Observe that \( V_{\epsilon }(x) \subseteq V_{\epsilon }(x)  \). Hence, \( V_{\epsilon }(x)  \) is an open set. 
        Yes, we have 
        \[  C_{\epsilon }(x) = \{ y \in X : d(x,y) \leq \epsilon \}  \] is a closed set. To see why, let \( x  \) be a limit point of \( C_{\epsilon }(x)  \). By Theorem 3.2.2, there exists a sequence \( x_{n} \subseteq C_{\epsilon }(x)  \) such that \( \lim x_{n} = x  \) with \( x_{n} \neq x  \) for all \( n \in \N  \). This means that we can choose an \( N \in \N  \) such that for all \( n \geq N  \), we must have \( d(x_{n}, x ) \leq \epsilon  \). Hence, \( x  \in C_{\epsilon }(x)  \) and so \( C_{\epsilon }(x)  \) must be a closed set.
        \end{proof}
    \item[(b)] Show that a set \( E \subseteq X  \) is open if and only if its complement is closed.
        \begin{proof}
        \( (\Rightarrow) \) Let \( x  \) be a limit point of \(E^{c} \). Then for all \( \epsilon- \)neighborhoods, the intersection 
        \[  V_{\epsilon }(x) \cap E^{c} \] is nonempty. Since \( E  \) is open, we know that 
        \( V_{\epsilon }(x) \subseteq E  \). But we have \( E \cap E^{c} \) is empty so we must have \( x \in E^{c}  \) because otherwise, there exists a \( V_{\epsilon }(x)  \) such that \( V_{\epsilon }(x) \subseteq E  \). Hence, \( E^{c} \) must be closed.  
       
        \( (\Leftarrow)\) Let \( E^{c} \) be a closed set. Let \( x \in E  \). Since \( x   \) is not a limit point of \( E^{c} \), \( x  \) must not be a limit point of \( E^{c} \). This means there exists an intersection \( V_{\epsilon }(x) \cap E^{c}  \)  that is empty. Hence, \( V_{\epsilon }(x) \subseteq E   \) implying that \( E  \) is open.
        \end{proof}
\end{enumerate}

\subsubsection{Exercise 8.2.9} 
\begin{enumerate}
    \item[(a)] Show that the set \( Y = \{ f \in C[0,1] :  \lVert f \rVert_{\infty } \leq 1  \}  \) is closed in \( C[0,1] \).
        \begin{proof}
        Suppose \( f  \) is a limit point of \( Y  \). Let \( (f_{n}) \) be a Cauchy sequence that converges uniformly to \( f  \). Choose \( \epsilon  =1  \). Then observe that there exists an \( N \in \N  \) such that for any \( n \geq N  \), we must have 
        \[ \lVert f_{n} - f  \rVert \leq 1. \] Then observe that 
        \begin{align*}
            | f(x) | &\leq | f_{n}(x) - f(x) + f(x)  |  \\
                  &\leq | f_{n}(x) - f(x)  | + | f(x)  | \\
                  &\leq \lVert f_{n} - f  \rVert_{\infty } \\
                  &\leq 1.
        \end{align*}
        Since \( | f(x)  |  \leq \lVert  f(x)  \rVert  \), we must also have \( \lVert f(x)  \rVert \leq 1  \).
        Hence, \( f  \) is contained in \( Y  \) and thus we must have \( Y  \) closed.
        \end{proof}
    \item[(b)] Is the set \( T = \{ f \in C[0,1] : f(0) = 0  \}  \) open, closed, or neither in \( C[0,1] \)? 
        \begin{proof}[Solution]
            \( T  \) is closed. Let \( \epsilon > 0  \). Let \( f \in C[0,1] \). Then there exists an  \( N \in \N   \) such that for any \( n \geq N  \), we have 
        \[ \lVert f  \rVert < \epsilon.   \] Hence, \( f(0) = 0  \) which implies \( f \in T \).
        \end{proof}
\end{enumerate}



\begin{definition}
A subset \( K  \) of a metric space \( (X,d)  \) is \textit{compact} if every sequence in \( K  \) has a convergent subsequence that converges to a limit in \( K  \).
\end{definition}

In \( \R  \), we came across a proposition that a set is compact if and only if it is closed and bounded. For more general metric spaces, however, this proposition only holds true in the forwards direction.

\begin{definition}
A subset \( K \) of a metric space \( (X,d)  \) is \textit{bounded} if there exists an \( R > 0  \) such that for all \( x,y \in X  \), we have \( d(x,y) < R  \).
\end{definition}


\subsubsection{Exercise 8.2.10} 
\begin{enumerate}
    \item[(a)] Show that if \( K  \) is compact subset of the metric space \( (X,d)  \), then \( K  \) is closed and bounded.       
        \begin{proof}
        Since \( K  \) is a compact subset of the metric space \( (X,d) \), every sequence \( (x_{n}) \subseteq K  \) contains a subsequence \( (x_{n_{k }})  \) that converges to a limit \( x  \) that is contained in \( K  \). Let \( (x_{n}) \) be a Cauchy sequence. Choose \( N \in \N  \) such that for any \( n > n_{k } \geq N  \), we must have 
        \begin{align*}
            d(x_{n}, x ) &\leq d(x_{n}, x_{n_{k }}) + d(x_{n_{k }}, x)  \\
                         &< \frac{ \epsilon  }{ 2  } + \frac{ \epsilon  }{ 2 } \\ 
                         &= \epsilon.\\
        \end{align*}
        Since \( x \in K  \) and \( (x_{n}) \to x  \), \( K  \) must be closed. Since every sequence \( (x_{n})   \) converges, we know that every \( (x_{n})  \) is bounded by some \( M > 0  \). Hence, \( K  \) must also be bounded.
        \end{proof}
    \item[(b)] Show that \( Y \subseteq C[0,1]  \) from Exercise 8.2.9 (a) is closed and bounded but not compact.
        \begin{proof}
            The results from part (a) of Exercise 8.2.9 imply that \( Y  \) is bounded and closed. To see why \( Y  \) is not compact, suppose we have a sequence of continuous functions \( (f_{n}) \) defined by \( f_{n} = x^{n} \). Since \( Y \subseteq C[0,1] \), we know that the \( (f_{n})  \to f \) uniformly. But the pointwise limit of \( \lim  f_{n}(x) \) is \textit{not continuous} and every subsequence of \( (f_{n}) \) will necessarily converge pointwise to \( f  \notin C[0,1] \). Hence, \( Y  \) cannot be compact in \( C[0,1] \). 
        \end{proof}
\end{enumerate}

\begin{itemize}
    \item The concept of \textit{equicontinuity} of functions is key to the solution of part (c) above.
    \item Look back to the Arzeli-Ascoli Theorem in chapter 6 before solving the exercise above.
    \item The result found in part (b) can only be made possible if, in addition to our assumptions, \( Y  \) contained a collection of functions that are equicontinuous.
\end{itemize}

\begin{definition}{Closure}{}
    Given a subset \( E  \) of a metric space \( (X,d)  \), the \textit{closure} \( \overline{E} \) is the union of \( E  \) together with its limit points. The \textit{interior} of \( E \) is denoted by \( E^{\circ} \) and is defined as 
    \[  E^{\circ} = \{ x \in E : \text{ there exists } V_{\epsilon }(x) \subseteq E  \}. \]
\end{definition}


\subsubsection{Exercise 8.2.11} 
\begin{enumerate}
    \item[(a)] Show that \( E  \) is closed if and only if \( \overline{E} = E  \). Show that \( E  \) is open if and only if \( E^{\circ} = E  \). 
        \begin{proof}
        See the solution in Exercise 3.2.14. 
        \end{proof}
    \item[(b)] Show that \( \overline{E}^{c} = (E^{c})^{\circ}  \), and similarly that \( (E^{\circ}c)^{} = \overline{E^{c}} \).
        \begin{proof}
        See the solution in Exercise 3.2.14.
        \end{proof}
\end{enumerate}


\subsubsection{Exercise 8.2.12}
\begin{enumerate}
    \item[(a)] show 
        \[  \overline{V_{\epsilon }(x)}  \subseteq \{ y \in X : d(x,y) \leq \epsilon  \},\]
        in an arbitrary metric space \( (X,d) \).
        \begin{proof}
            Let \( x  \) be a limit point of \( \overline{V_{\epsilon }(x)} \). By definition of \( \overline{V_{\epsilon }(x)}  \), we know that \( x \in \overline{V_{\epsilon }(x)} \). Then, there exist exists a sequence \( (x_{n})  \) such that \( x_{n} \to x  \) with \( x_{n} \neq x  \) for all \( n \in \N  \). Hence, for some \( N \in \N  \) we know that for any \( n \geq N  \), we have \( d(x_{n}, x) \leq \epsilon  \). But this is the definition of \( C_{\epsilon }(x) \). Hence, \( x \in C_{\epsilon }(x) \).
        \end{proof}
    \item[(b)] To keep things from sounding too familiar, find an example of a specific metric space where 
        \[  \overline{V_{\epsilon }(x)} \neq \{ y \in X : d(x,y) \leq \epsilon  \}.\]
        \begin{proof}[Solution]
        Take \( (\R, | \cdot | ) \). Then observe that \( \overline{V_{\epsilon }(x)} \neq \{ n \in \N : | 1 / n | \leq \epsilon \}  \) where \( V_{\epsilon }(x) =  \{ 0 \}  \).
        \end{proof}
\end{enumerate}

\begin{definition}{Dense Sets}{}
A set \( A \subseteq X  \) is \textit{dense} in the metric space \( (X,d) \) if \( \overline{A} = X  \). A subset \( E  \) of a metric space \( (X,d) \) is \textit{nowhere-dense} in \( X  \) if \( \overline{E}^{\circ}  \) is empty.
\end{definition}

\subsubsection{Exercise 8.2.13} If \( E  \) is a subset of a metric space \( (X,d) \), show that \( E  \) is nowhere-dense in \( X  \) if and only if \( \overline{E}^{c}  \) is dense in \( X  \). 
\begin{proof}
    \( (\Rightarrow)  \) Suppose \( E  \) is nowhere-dense in \( X  \). Then \( \overline{E}^{\circ}  \) is empty. Let \( \epsilon > 0  \). Then for any \( x \in \overline{E}^{\circ}  \) is contained in \( (\overline{E}^{\circ})^{c} \). By exercise 8.2.11, we know that \( (\overline{E}^{\circ})^{c} = \overline{(\overline{E}^{\circ})^{c}}\). But notice that we must have \( \overline{ (\overline{E}^{\circ})^{c}}  = X   \). Hence, \( \overline{E}^{c} \)  must be dense in \( X  \).

    \( (\Leftarrow) \) Suppose \( \overline{E}^c \) is dense in \( X  \). Let \( x \in \overline{E}^{\circ} \). Let \( \epsilon > 0  \). Since \( x  \) is neither an element of nor a limit point of \( \overline{E}^c \), we know that for every \( V_{\epsilon }(x)   \), we have \( V_{\epsilon }(x) \cap \overline{E} \) is empty. This tells us that \( \overline{E}^{\circ}  \) is empty. Hence, \( E  \) is nowhere dense in \( X  \).
\end{proof}








