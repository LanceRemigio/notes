\section{Inventing the Factorial Function}
The goal of this section is construct a function \( f(x)  \), defined on all of \( \R  \) with the property that \( f(n) = n!  \) for all \( n \in \N  \). This can be done easily by defining a piecewise function such that  
\[ f(x) = 
\begin{cases}
    n! \ &\text{if } n \leq x < n+1, n \in \N \\ 
    1 \ &\text{if } x < 1.
\end{cases}  \] Some questions we can explore is its continuity, differentiability (if differentiable then how many times?). Our goal now is to define a function that extends the definition of the factorial \( n!  \) in a meaningful way to non-natural \(  n \). 
\subsubsection{Exercise 8.4.1} For each \( n \in \N  \), let 
\[  n \# = n + (n -1 ) + (n-2) + \dots + 2 + 1. \]
\begin{enumerate}
    \item[(a)] Without looking ahead, decide if there is a natural way to define \( 0 \#  \). How about \( (-2) \#  \)? Conjecture a reasonable value for \(  \frac{ 7 }{ 2 }  \# \).
        \begin{proof}[Solution]
        
        \end{proof}
    \item[(b)] Now prove \( n \# = \frac{ 1 }{ 2 }  n (n+1) \) for all \( n \in \N  \), and revisit part (a).
        \begin{proof}
        The statement above is clearly true for \( n =1  \). Now assume \(  n \#   = \frac{ 1 }{ 2 }  n (n+1) \) holds for \(  1 \leq n \leq k -1  \). We want to show that \( n \#  \) holds for the \( k  \)th case. By using the definition of \( n \#  \), we can write 
        \begin{align*}
            k \# &= k + (k-1)\#  \\
                 &= k + \frac{ 1 }{ 2 }  k (k-1) \\
                 &= \frac{ 1 }{ 2 }  (k^{2} + k) \\
                 &= \frac{ 1 }{ 2 } k (k+1).
        \end{align*}
        Since \( n \# = \frac{ 1 }{ 2 }  n (n+1)  \) holds for the \( k  \)th case, we know that it holds for any \( n \in \N  \).
        \end{proof}
\end{enumerate}
We can replace the discrete variable \( n \in \N   \) for values of \( x \in \R  \) and the resulting formula 
\[  x \# = \frac{ 1 }{ 2 }  x (x+1)  \] will still make sense.

\subsection{The Exponential Function} 

How is the exponential function like \( 2^{x}  \) defined on \( \R  \)? Typically, \( 2^{x} \) is defined through a series of domain expansions. Starting with the function defined on \( \N  \), we can expand its domain by using reciprocals, then to \(  \Q  \) using roots, and then \( \R  \) using continuity. Our goal in this section is to expand the domain of \( 2^{x }   \) using a different method. 

Our first step is to properly define the natural exponential function \( e^{x}  \). Recall in chapter 6, we constructed a series expansion for \( e^{x}  \). This time, we do the opposite direction; that is, create a proper definition of \( e^{x}  \). We can do this by using the results we have found in our studies of power series expansions. 

Define 
\[  E(x) = \sum_{ n=0  }^{ \infty  } \frac{ x^{n}  }{  n!  } = 1 + x + \frac{ x^{2}  }{ 2!  } + \frac{ x^{3} }{ 3!  } + \dotsb . \]

\subsubsection{Exercise 8.4.2} Verify that the series converges absolutely for all \( x \in \R  \), that \( E(x)  \) is differentiable on \( \R  \), and \( E'(x) = E(x)  \).
\begin{proof}[Solution]
First we prove that the series above converges absolutely for all \( x \in \R  \). Let \( x \in \R  \). Observe that 
\[  \Big| \sum_{ n=0  }^{ \infty  } \frac{ x^{n}  }{  n!  }  \Big| \leq \sum_{ n=0  }^{ \infty  } \Big| \frac{ x^{n}  }{ n!  }  \Big|.    \] By using the ratio test for power series found in section 6.5, we have
\begin{align*}
    \Big| \frac{ a_{n+1} }{ a_{n} }  \Big| &= \Big| \frac{ x^{n+1} }{ (n+1)!  } \cdot \frac{ n!  }{ x^{n} }  \Big|  \\
                                           &= \frac{ | x  |  }{ n+1  } \xrightarrow{n\rightarrow\infty} 0. \\
\end{align*}
Since the limit above is \(  0  \), we know that the series 
\[  \sum_{ n=0  }^{ \infty  } \frac{ x^{n}  }{  n!  } \tag{1} \] converges absolutely for all \( x \in \R  \). Given any compact set in \( \R  \), we know that the convergence of (1) to \( E(x)  \) is uniform. Hence, it must be continuous on any \( A \subseteq \R  \) and differentiable \( n  \) times. Differentiating 
\[  E(x) = \sum_{ n= 0 }^{ \infty  } \frac{ x^{n}  }{ n!  }  \] and reordering indices we find that \( E'(x) = E(x)  \).
\end{proof}

\subsubsection{Exercise 8.4.3} 
\begin{enumerate}
    \item[(a)] Use the results of Exercise 2.8.7 and the binomial formula to show that \( E(x+y) = E(x)E(y)  \) for all \( x , y \in \R  \).
    \begin{proof}
    Let \( x, y \in \R  \). By definition of \( E(x+y)  \), using the binomial formula, we can write 
    \begin{align*}
        E(x+y)    &= \sum_{ n=0  }^{ \infty  } \frac{ (x+y)^{n}  }{  n!  }  \\
                  &= \sum_{ n=0 }^{ \infty  } \sum_{ k=0 }^{ \infty  } \frac{ y^{k } \cdot x^{n-k} }{  k! (n-k)!  }  \\
                  &= \sum_{ n=0  }^{ \infty  } \sum_{ k=0  }^{ \infty  } \frac{ y^{k }  }{  k!  }  \cdot \frac{ x^{n-k} }{ (n-k)!  } \\
                  &=  \Big[ \sum_{ m=0 }^{ \infty  } \frac{ x^{m} }{ m! }  \Big] \Big[ \sum_{ k = 0  }^{ \infty  } \frac{ y^{ k }  }{ k!  }  \Big] \tag{\( n-k = m \)}. \\
    \end{align*}
    Since 
    \begin{align*}
        E(x) &= \sum_{ m=0  }^{ \infty  } \frac{ x^{m}  }{ m!  },  \\
        E(y) &= \sum_{ k=0  }^{ \infty  } \frac{ y^{ k }  }{  k!  } \\
    \end{align*}
    both converge absolutely (by Exercise 2.8.7), we can write 
    \[  E(x+y) = E(x)E(y). \]
    \end{proof}
    \item[(b)] Show that \( E(0) = 1, E(-x) = 1 / E(x)  \), and \( E(x) > 0  \) for all \( x \in \R  \).
        \begin{proof}
        Let \( x \in \R  \). The first fact immediately follows when \( x = 0  \). Now let us show the second fact. Using the first fact and part(a), we can write 
        \[   1 = E(0) = E(x - x ) = E(x) E(-x) \iff E(x)E(-x) = 1.   \] 
        Dividing through by \( E(x)  \) on both sides leads us to our result 
        \[  E(-x) = \frac{ 1 }{ E(x)  }. \]  For the last fact, observe that \( E(x) > 0  \) follows immediately when we consider any \( x \geq 0  \). Suppose we let \( x  \) be negative, then using the fact that 
        \[  E(-x) = \frac{ 1 }{ E(x)  }   \] where \( E(x) > 0  \) for any \( x > 0  \) implies that \( E(-x) = \sum_{ n=0 }^{ \infty  } \frac{ (-1)^{n} x^{n}  }{  n!  } > 0    \).
        \end{proof}
\end{enumerate}

The takeaway here is that the power series \( E(x)  \) contains all the "normal" properties that is associated with the exponential function \( e^{x} \). 

\subsubsection{Exercise 8.4.4} Define \( e = E(1)  \). Show \( E(n) = e^{n}  \) and \( E(m/n) = e^{m/n} \) for all \( m,n \in \Z  \).
\begin{proof}

\end{proof}

To complete our list of properties of \( e^{x}  \), all we need is its behavior as \( x \to \pm \infty  \).

\begin{definition}{}{}
    Given \( f : [a,\infty ]  \to \R \), we say that \( \lim_{ x \to \infty  }  f(x) = L  \) if, for all \( \epsilon > 0  \), there exists \( M > a  \) such that whenever \( x \geq M  \) it follows that \( |  f(x) - L  | < \epsilon  \).
\end{definition}





