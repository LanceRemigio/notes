


\section{Cardinality}

\subsection{Correspondence}

\begin{definition}
    A function \( f: A \to B \) is \textit{one-to-one} if \( a_1 \neq a_2 \) in \( A \) implies that 
    \( f(a_1) \neq f(a_2) \) in \( B \). The function \( f \) is \textit{onto} if, given any \( b \in B \), there exists an element \( a \in A \) for which \( f(a) = b\).
\end{definition}

An equivalent defintion for a function to be one-to-one is the following:

\begin{definition}
    A function \( f: A \to B \) is \textit{one-to-one} if \( f(a_1) \neq f(a_2) \) implies that \( a_1 = a_2 \).
\end{definition}
    
A function that is both one-to-one and onto is said to be bijective. Meaning that we have a one-to-one correspodence between the sets \( A \) and \( B \). Another way to explain a function being injective is to say that no two elements from \( A \) can map to the same element in \( B \) ( think of the function \( x^2\)). And a function being onto can be explained as every element in \( A \) has to be mapped to an element in \( B \).

From an algebraic perspective, we can denote a function being bijective to mean the same thing as two sets having the same cardinality i.e we can say that 

\begin{definition}
Two sets \( A \) and \( B \) have the same cardinality if there exists \( f: A \to B \) that is both one-to-one and onto. We can denote this symbolically as \( A \sim B\)
\end{definition}

Some examples of bijective maps are
\begin{enumerate}
    \item[(i)] Let the following map \(f: \N \to \mathbf{E} \) be defined as \( f(n) = 2n \). We can see that \( \N \sim \mathbf{E}\). It's true that \( \mathbf{E}\) is indeed a subset of \( \N \), but do not conclude that it is a smaller set than \( \N \) since they have the same cardinality or isomorphic to each other.
    \item[(ii)] We can show this again. This time let us have a map \( f: \N \to \Z \) such that 
        \[ f(n) = \begin{cases}  (n-1)/2 \text{ if } n \text{ is odd.}       \\
                                 -n/2 \text{ if } n \text{ is even.}
                                                \end{cases}\]
We have that \( \N \sim \Z \) indeed.
\end{enumerate}


\subsection{Countable Sets}

\begin{definition}
A set \( A \) is \textit{countable} if \( \N \sim A \). An infinite set that is not countable is called an \textit{uncountable set}.
\end{definition}

\begin{theorem}
Let \( \Q, \R \). Then 
\begin{enumerate}
    \item[(i)] The set \( \Q \) is countable. 
    \item[(ii)] The set \( \R \) is uncountable.
\end{enumerate}


\end{theorem} 


\begin{proof}
\begin{enumerate}
\item Suppose we define \( A_n \) to be split into two sets. When \( n = 1 \), define \( A_n \) to be 
\[ A_1 = \{  0 \}\] and define \( A_n \) when \( n \geq 2 \) as 
\[ A_n =   \Big\{ \pm \frac{p}{q} : \text{ where } p,q \in \N \text{ are in lowest terms with } p + q = n   \Big\}\]
We can observe here that for every \( n \in \N \) we can find every element of \( \Q \) exactly once in the sets we have defined. So we can conclude that our map is onto. Since we designed our sets so that each rational numer appears once and the fact that for \( n =1 \) and \( n \geq 2 \) produces two disjoint sets, we can see that our map is also one-to-one. 
\item We can prove that second statement of theorem by contradiction. Assume for the sake of contradiction that there exists a \textit{one-to-one} and \textit{onto} function where \( f: \N \to \R \). Letting \( x_1 = f(1)\) and \( x_2 = f(2)\) and so on, then we can enumerate each element of \( \R \) i.e 
    \[ \R = \{ x_1, x_2, x_3, ... \}.\]
    Using the Nested Interval Property, we will now produce a real number that is not in this set. Let \( I_n \) be a closed interval which does not contain  \( x_n \) but contains \( x_{n+1}\). Furthermore, \( I_{n+1}\) is contained within \( I_n \). Note that within \( I_n \) there are two sets which are disjoint and \( x_{n+1}\) can be in either one of these sets. Now consider the following intersection \( \cap_{n=1}^{\infty} I_n \). Using our construction that every \(x_n \not\in I_n \), then we can say that
    \begin{align*} \bigcap_{n=1 }^{\infty} I_n = \emptyset. \end{align*}
But this is a contradiction because the nested interval property asserts that this intersection is nonempty meaning that every \( x \in \R \) is contained in the above set. Hence, we cannot emumerate every single element \( x_n  \) of \( \R \). Therefore, \( \R \) is an \textit{uncountable} set.
\end{enumerate}
\end{proof}

This gives us three insights: 

\begin{enumerate}
    \item The smallest type of infinite set is the countable set.
    \item We can create another set by deleting or inserting elements into it. 
    \item Anything smaller than a countable set is either finite or countable. 
\end{enumerate}

We can create \( \R \) by taking the union of \( \Q \) and \( \mathbb{I} \). Since \( \R \) is not countable and \( \Q \) is, this would mean that the set of irrational numbers \( \mathbb{I}\) would be uncountable. This tells us that \( \mathbb{I}\) is a bigger subset of \( \R \) than \( \Q \). 

We can summarize these results in the follow two theorems: 


\begin{theorem}
    If \( A \subseteq B \) and \( B \) is \textit{countable}, then \( A \) is either countable or finite. 
\end{theorem}

\begin{theorem}
\begin{enumerate}
    \item[(i)] If \( A_1, A_2,... A_n\) are each countable sets, then the union of 
        \[ A_1 \cup A_2 \cup ... \cup A_m \] is countable.
    \item[(ii)] If \( A_n \) is a countable set for each \( n \in \N \), the \( \bigcup_{n=1}^{\infty}A_n \) is countable. 
\end{enumerate}
\end{theorem}








