\section{Compact Sets}

\subsubsection{Exercises 3.3.1} Show that if \( K \) is compact and nonempty, then \( \sup K  \) and \( \inf K \) both exists and are contained in \( K \).
\begin{proof}
   Suppose \( K \) is compact. By the equivalence theorem, \( K \) is bounded. Since \( K \) is nonempty and bounded above and below, \( \sup K  \) exists and \( \inf K \) exists by the Axiom of Completeness. We begin by constructing a sequence \( (x_n) \) that is contained in \( K \) such that we want to show that \( (x_n) \to \sup K \). Let \( \epsilon > 0  \). Since \( \sup K \) is the least upper bound, we can have \( \sup K - \epsilon   \) be not an upper bound. But since \( (x_n) \subseteq K \), we must have \( x_n \leq \sup K \). Together we have 
\[ \sup K - \epsilon \leq x_n \leq \sup K. \]
Since \( \epsilon > 0  \) is arbitrary, the sequence \( (x_n) \to \sup K \). Since \( K  \) is closed \( \sup K \) must be contained in \( K \). By a similar reasoning, we can generate \( (x_n) \)
so that \( (x_n) \to \inf K \) which is also contained in \( K \). Hence, \( \sup K  \) and \( \inf K \) contained in \( K \).
\end{proof}

\subsubsection{Exercise 3.3.2} Decide which of the following sets are compact. For those that are not compact, show how Definition 3.3.1 breaks down. In other words, give an example of a sequence contained in the given set that does not possess a subsequence converging to a limit in the set.

\begin{enumerate}
    \item[(a)] \( \N \).
        \begin{proof}[Solution]
        Not compact. We can see that for every \( n \in \N \), the sequence \( a_n = n  \) diverges to infinity and so does its subsequences; that is, if we define the odd and even subsequences \( (a_n) \) where both sequences also converge to infinity.
        \end{proof}
    \item[(b)] \( \Q \cap [0,1] \).
        \begin{proof}[Solution]
        This set is compact because it is bounded and closed.
        \end{proof}
    \item[(c)] The Cantor set.
    \begin{proof}[Solution]
    The cantor set is closed.
    \end{proof}
\item[(d)] \( \{ 1 + 1/2^2 + 1/3^2 + \dots + 1/n^2 : n \in \N \}.  \)
    \begin{proof}[Solution]
    This set is not compact since the limit point of this set (\( L = \{ 0 \}  \)) is not contained in the set.
    \end{proof}
\item[(e)] \( \{ 1, 1/2, 2/3, 3/4, 4/5, \dots \}.  \)
    \begin{proof}[Solution]
        This set is compact since its limit point \( L = \{ 1 \}  \) is contained in the set and the fact that it is bounded by \( 1 \) and \( 0 \). 
    \end{proof}
\end{enumerate}


\subsubsection{Exercise 3.3.3} Prove the converse of Theorem 3.3.4 by showing that if a set \( K \subseteq \R \) is closed and bounded, then it is compact.

\begin{proof}
    Assume that \( K \) is closed and bounded. Our goal is to show that \( K \) is compact. Let \( (a_n) \) be a sequence contained in \( K \). Since \( K  \) is bounded, we know that \( (a_n) \) is also bounded. Hence, there exists a subsequence \( (a_{n_k}) \) such that \( (a_{n_k}) \to a \). But since \( K \) is closed and \( a \) is a limit point of \( K \), we have that \( \lim a_{n_k} = a  \) is also contained in \( K \).
\end{proof}

\subsubsection{Exercise 3.3.4} Assume \( K  \) is compact and \( F \) is closed. Decide if the following sets are definitely compact, definitely closed, both or neither.

\begin{enumerate}
    \item[(a)] \( K \cap F \).
        \begin{proof}[Solution]
            Compact.
        \end{proof}
    \item[(b)] \( \overline{F^c \cup K^c} \).
        \begin{proof}[Solution]
            Closed.
        \end{proof}
    \item[(c)] \( K \setminus F = \{ x \in K : x \notin F \}  \)
        \begin{proof}[Solution]
        Definitely compact.
        \end{proof}
    \item[(d)] \( \overline{K \cap F^c} \).
        \begin{proof}[Solution]
        Definitely compact.
        \end{proof}
\end{enumerate}

\subsubsection{Exercise 3.3.8} Let \( K  \) and \( L  \) be nonempty compact sets, and define 
\[ d = \inf \{ | x - y  | : x \in K ~ \text{and} ~ y \in L \}.  \]
This turns out to a reasonable definition between \( K \) and \( L \). 
\begin{enumerate}
    \item[(a)] If \( K  \) and \( L \) are disjoint, show \( d > 0  \) and that \( d = | x_0 - y_0 |  \) for some \( x_0 \in K \) and \( y_0 \in L \).
        \begin{proof}
        Suppose \( K \) and \( L \) are disjoint, nonempty compact sets. Our goal is to show that \( d > 0  \) and that \( d = | x_0 - y_0  |  \) for some \( x_0 \in K  \) and \( y_0 \in L \). Since \( K \) and \( L \) are nonempty compact sets, we also have that \( K \) and \( L \) are bounded sets. Hence, \( \inf A \) exists where 
        \[ A = \{ | x - y | : x \in K ~ \text{and} ~ y \in L \}  \]
by the Axiom of Completeness. Then by lemma 1.3.8, we have that for some \( x_0 \in K \) and \( y_0 \in L \), consider the distance \( d = | x_0 - y_0  | \in A   \) such that we have
\[ d = | x_0 - y_0  | > \inf A + \epsilon  \]
for all \( \epsilon > 0  \). Since \( K \) and \( L \) are disjoint and \( \epsilon > 0  \) is arbitrary, we have \(d =  |x_0 - y_0  | > 0  \). Hence, we have that \( d > 0  \).
        \end{proof}
        Technically, the proof above is not valid yet since we haven't gone over functions and how compactness imply continuity.
        \begin{proof}
            Suppose \( K \) and \( L \) are non-empty compact sets that are disjoint. Suppose for sake of contradiction that \( d = 0  \). Since \( K \) and \( L \) are both compact, let \( (x_n) \) and \( (y_n) \) contain subsequences \( (x_{n_k}) \) and \( (y_{n_k}) \) that converge to \( x_0  \) and \( y_0 \) respectively. We want to show that \( \lim | x_{n_k} - y_{n_k} | = | x_0 - y_0 |  \). Let \( \epsilon > 0  \). Then for every \( n_k > N  \) for some \( N \in \N \), we have that 
            \begin{align*}
                | | x_{n_k} - y_{n_k}| - | x_0 - y_0 |   | &\leq | (x_{n_k} - y_{n_k}) - (x_0 - y_0) |  \\
                                                           &= | (x_{n_k} - x_0) + (y_0- y_{n_k}) |  \\
                                                           &\leq | x_{n_k} - x_0 | + | y_{n_k} - y_0  | \\   
                                                           &< \frac{ \epsilon  }{ 2 }  + \frac{ \epsilon  }{ 2 }  \\
                                                           &= \epsilon.
            \end{align*}
            Since \( d = | x_0 - y_0 | = 0   \), we have that \( \lim | x_{n_k} - y_{n_k} | = d = 0   \). But this means that \( K \cap L \neq \emptyset \) since \( K \) and \( L \) are closed sets which is a contradiction. Hence, we must have \( d > 0  \).
        \end{proof}
    \item[(b)] Show that it's possible to have \( d = 0  \) if we assume only that the disjoint sets \( K \) and \( L \) are closed.
        \begin{proof}
            Basically the argument above but taking away the fact that \( K \) and \(  L \) are compact sets.
        \end{proof}
\end{enumerate}

\subsubsection{Exercise 3.3.9} Follow these steps to prove the final implication in Theorem 3.3.8. 
Assume \( K \) satisfies (i) and (ii), and let \( \{ O_{\lambda} : \lambda \in \Lambda  \}  \) be an open cover for \( K \). For contradiction, let's assume that no finite subcover exists. Let \( I_0  \) be a closed interval containing \( K \).
\begin{enumerate}
    \item[(a)] Show that there exists a nested sequence of closed intervals \( I_0 \supseteq I_1 \supseteq I_2 \supseteq \dots \) with the property that, for each \( n  \), \( I_n \cap K \) cannot be finitely covered and \( \lim | I_n | = 0  \).
        \begin{proof}
        Let \( I_1  \) be a half of \( I_0 \) whose intersection with \( K \) does not have a finite subcover, so that \( I_1 \cap K \) cannot be finitely covered and \( I_1 \subseteq I_0  \). Then bisect \( I_1  \) into two closed intervals \( A_2 \) and \( B_2 \), and again let \( A_2 = I_2  \) such that \( A_2 \cap K  \) does not have a finite subcover. Otherwise, \( B_2 = I_2  \) such that \( B_2 \cap K \) does not contain a finite subcover. Hence, \( I_2 \subseteq I_1 \subseteq I_0 \). We can repeat this process of bisecting each interval \( I_n  \) and determining which closed set does not have a finite subcovers until we have our desired sequence \( I_n \) such that \( \lim I_n = 0 \).
    \end{proof}
    \item[(b)] Argue that there exists an \( x \in K \) such that \( x \in I_n \) for all \( n \).
        \begin{proof}
        Since \( K \) is compact, \( K \cap I_n \) is also compact for each \( n \in \N \). Hence, we know that \( \bigcap_{ n=1 }^{ \infty  } I_n \cap K \) is non-empty, and there exists \( x \in K \cap I_n \) for all \( n \).
        \end{proof}
    \item[(c)] Because \( x \in K \), there must exists an open set \( O_{\lambda_0} \) from the original collection that contains \( x  \) as an element. Explain how this leads to the desired contradiction.
\begin{proof}
    Let \( x \in K \) and let \( O_{\lambda_0} \) be an open set that contains \( x \). Since \( O_{\lambda_0} \) is an open set, we can construct an \( \epsilon - \)neighborhood such that \( V_{\epsilon_0 }(x) \subseteq O_{\lambda_0} \). Now choose \( n_0  \) such that for every \( n \geq n_0 \), \( | I_{n_0}| < \epsilon_0  \). Hence, \( I_{n_0} \) is contained within a single open set \( O_{\lambda_0} \) which means it has a finite subcover. This contradiction tells us that \( K \) must have had a finite subcover.
\end{proof}
\end{enumerate}

\subsubsection{Exercise 3.3.10} Here is an alternate proof to the one given in Exercise 3.3.9 for the final implication in the Heine-Borel Theorem. Consider the special case where \( K \) is a closed interval. Let \( \{ O_{\lambda} : \lambda \in \Lambda\}  \) be an open cover for \( [a,b] \) and define \( S \) to be the set of all \( x \in [a,b] \) such that \( [a,x] \) has a finite subcover from \( \{ O_{\lambda} : \lambda \in \Lambda \}  \).
\begin{enumerate}
    \item[(a)] Argue that \( S \) is nonempty and bounded, and thus \( s = \sup S \) exists.
        \begin{proof}
            Let \( S \) be defined as the set of all \( x \in [a,b] \) such that \( [a,x] \) has a finite subcover from \( \{ O_{\lambda} : \lambda \in \Lambda \}  \). Since \( [a,b] \) is a closed interval, we can define a sequence of points \( (x_n) \) that converges to \( x \in [a,b] \). Since \( \{ O_{\lambda} : \lambda \in \Lambda\}  \) is an open cover for \( [a,b] \), we know that there must exists a finite cover for \( [a,x] \). Hence, \( S \) is nonempty and \( S \) is bounded since \( a \leq x \leq b \). By the Axiom of Completeness, \( s = \sup S \) exists.
        \end{proof}
    \item[(b)] Now show \( s = b  \), which implies \( [a,b] \) has a finite subcover. 
        \begin{proof}
            Since \( [a,b] \) is a closed and bounded interval, it follows that \( [a,b] \) is a compact set. By exercise 3.3.1, \( [a,b] \) must contain its supremum. Hence, \( \sup [a,b] \in S \) and hence, \( [a,b] \) must have a finite subcover.
        \end{proof}
    \item[(c)] Finally, prove the theorem for an arbitrary closed and bounded set \( K \).
        \begin{proof}
            Let \( K \) be a closed and bounded set. From (a) and (b), \( K \) must have finite subcover from \( \{ O_{\lambda} : \lambda \in \Lambda \}  \).
        \end{proof}
\end{enumerate}

Other solutions (not mine).

\begin{enumerate}
    \item[(b)] Now show \( s = b  \), which implies \( [a,b] \) has a finite subcover.
        \begin{proof}
            Suppose for contradiction that \( s < b  \), letting \( s \in O_{\lambda_0} \) implies \( [a,s] \) is finitely coverable since we can take the finite cover of an \( x \in O_{\lambda_0} \) with \( x < s  \). This causes a contradiction however since there exists points \( y > s  \) with \( y \in O_{\lambda_0} \) meaning \( [a,y] \) is also finitely coverable. Therefore, the only option is \( s = b  \), since any \( s < b  \) fails.
    \end{proof}
\item[(c)] We must also consider the case where \( y \) does not exists; that is, there exists a "gap". Let \(  y = [s,b] \cap K  \) and suppose \( y \neq s  \). Since \(  y \in [s,b] \cap K \) we know 
    \[ [a,y] \cap K = ([a,s] \cap K ) \cup ([s,y] \cap K) = [a,s] \cap K \cup {y}. \]
    Therefore if \( \{ O_{\lambda_1}, \dots, O_{\lambda_{n}} \}  \) covered \( [a,s] \) then letting \( y \in O_{\lambda_{n+1}} \) would give the finite cover \( \{ O_{\lambda_1}, \dots, O_{\lambda_{n+1}}\}  \) contradicting the assumption that \( s < b  \), therefore \( s = b  \) is the only option, and so \( K \) can be finitely covered.
\end{enumerate}




