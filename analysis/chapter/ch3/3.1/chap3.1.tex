
\section{Discussion: The Cantor Set}

The following construction demonstrates that \( \R  \) is an uncountable set. Let \( C_0  \) be the closed interval \( [0,1] \), and define \( C_1 \) to be the set that results when we remove an open set in the middle third; that is, 
\[ C_1 = C_0 \setminus \Big( \frac{ 1 }{ 3 } , \frac{ 2 }{ 3 }  \Big) = \Big[ 0, \frac{ 1 }{ 3 } \Big] \cup \Big[\frac{ 2 }{ 3 } , 1 \Big]. \]
We can construct the next iteration \( C_2 \) in a similar way above of each of the two sets unioned above. Hence, we have 
\[ C_2 = \Big(  \Big[ 0, \frac{ 1  }{ 9 } \Big] \cup \Big[ \frac{ 2 }{ 9 } , \frac{ 1 }{ 3 } \Big]   \Big) \cup \Big( \Big[ \frac{ 2 }{ 3 } , \frac{ 7 }{ 9 } \Big] \cup \Big[ \frac{ 8 }{ 9 } , 1\Big]\Big)\]
or
\[ C_n = [0,1] \setminus \Big[ \Big(  \frac{ 1 }{ 3 } , \frac{ 2 }{ 3 } \Big) \cup \Big( \frac{ 1 }{ 9 } , \frac{ 2 }{ 9 }  \Big) \cup \Big( \frac{ 7 }{ 8 } , \frac{ 8 }{ 9 }  \Big) \cup \dots   \Big]  \]
If we continue this process inductively, then for each \(  n \in \N  \), we get sets \( C_n \) consisting of \( 2^n \) closed intervals with each having a length of \( 1/3^n \). The Cantor set \( C \) is just the intersection of an infinite number of \( C_n \); that is, 
\[ C = \bigcup_{ n=0 }^{ \infty  } C_n. \]

