\section{Power Series}

\subsubsection{Exercise 6.5.1} Consider the function \( g  \) defined by the power series 
\[  g(x) = x - \frac{ x^2  }{ 2  }  + \frac{ x^3  }{  3  }  - \frac{ x^4  }{ 4  }  + \frac{ x^5  }{  5  }  - \dotsb . \]
\begin{enumerate}
    \item[(a)] Is \( g  \) defined on \( (-1,1)?  \) Is it continuous on this set? Is \( g  \) defined on \( (-1,1]  \)? Is it continuous on this set? What happens on \( [-1,1]  \)? Can the power series for \( g(x) \) possibly converge for any other points \( | x  | > 1  \)? Explain.
        \begin{proof}[Solution]
        
        \end{proof}
    \item[(b)] For what values of \( x \) is \( g'(x)  \) defined? Find a formula for \( g' \).
        \begin{proof}[Solution]
        
        \end{proof}
\end{enumerate}

\subsubsection{Exercise 6.5.3} Use the Weierstrass M-test to prove Theorem 6.5.2.
\begin{proof}
    Suppose a power series \( \sum_{ n=0  }^{  \infty   } a_n x^n  \) converges absolutely at a point \( x_0  \). Then the series \(  \sum_{ n=0  }^{  \infty  } | a_n x_0^n |  \) converges. Let \( x \in [-c , c ] \) where \( c = | x_0  |  \). We proceed via the Weierstrass M-test to show that \( \sum_{ n=0  }^{ \infty  } a_n x^n  \) converges uniformly. We observe that 
    \[  | a_n x^n  |  \leq a_n c^n = a_n | x_0  |^n = a_n | x_0^n | .  \]
    This tells us that 
    \[  \sum_{ n=0 }^{  \infty   } | a_n x^n | \leq \sum_{ n=0  }^{ \infty  } a_n | x_0^n  |.   \]
    Since the right side of the above inequality converges, we know that \( \sum_{ n=0  }^{ \infty  } a_n x^n  \) must converge uniformly on any \( x \in [-c ,c ]  \).
\end{proof}



\subsubsection{Exercise 6.5.3 (Term-by-term Antidifferentiation).} Assume \( f(x) = \sum_{ n=0  }^{  \infty  } a_n x^n  \) converges on \( (-R,R ) \). 
\begin{enumerate}
    \item[(a)] Show 
        \[  F(x) = \sum_{ n=0  }^{ \infty  } \frac{ a_n  }{  n + 1 } x^{n+1} \]
        is defined on \( (-R,R ) \), find a power series representation for \( g  \).
        \begin{proof}
            First we show that \( F(x)  \) converges uniformly on \( (-R,R ) \). Since \( \sum_{ n=0  }^{ \infty  } a_n x^n  \) on \( (-R ,R ) \), we know that the sequence of partial sums must be bounded by some \( M > 0  \). Let \( \epsilon > 0   \). Choose \( N \in \N  \) such that for any \(  n > m \geq N  \), we have that  
            \begin{align*}
                \Big| \sum_{ k = m +1  }^{  n } \frac{ a_k  }{  k +1  } x^{k+1} \Big| &= \Big| \sum_{ k= m+1  }^{ n } (a_k x^k) \Big( \frac{ x }{ k+1  }  \Big) \Big|  \\
                                                                                      &\leq M \Big( \frac{ x }{ m+2  }  \Big) \\
                                                                                      &< M \Big( \frac{ R  }{ m+2  }  \Big) \\
                                                                                      &< \epsilon
            \end{align*}
            by Abel's Lemma.
            Since the power series \( \sum_{ n=0  }^{ \infty  } \frac{ a_n  }{ n + 1  }  x^{n+1}  \) converges on \( (-R,R ) \), we know that \( F(x)  \) must be continuous on \( (-R ,R ) \) and differentiable by Theorem 6.5.6. Hence, we can differentiate according to the formula in Theorem 6.5.6 to write that \[  F'(x) = \sum_{ n=0  }^{ \infty  } a_n x^n = f(x). \]
        \end{proof}
    \item[(b)] Antiderivatives are not unique. If \( g  \) is an arbitrary function satisfying \( g'(x) = f(x)  \) on \( (-R,R ) \), find a power series representation for \( g  \).
        \begin{proof}[Solution]
        Suppose \( g  \) is an arbitrary function satisfying \( g'(x) = f(x)  \) on \( (-R ,R ) \); that is, 
        \[  g'(x) = \sum_{ n=0  }^{ \infty  } a_n x^n. \]
        Since \( f(x) = F'(x)  \) and the fact that Antiderivatives are not unique, we can write 
        \[  g(x) = c +  \sum_{ n=0  }^{ \infty  } \frac{ a_n  }{  n+1  } x^{n+1}\]
        where \( c  \) is some constant.
        \end{proof}
\end{enumerate} 


\subsubsection{Exercise 6.5.5} 
\begin{enumerate}
    \item[(a)] If \( s  \) satisfies \( 0 < s < 1  \), show \( ns^{n-1} \) is bounded for all \( n \geq 1  \).
        \begin{proof}
            Let \( x  \) satisfy \( 0 < s < 1  \). Let us proceed by inducting on \( n  \) where \( P(n) \) is statement that for every \( n \geq 1  \), we have that \( ns^{n-1}  \) is bounded. Let our base case be \( n = 1  \). Then using the fact that \( 0 < s < 1  \), we have that \(  0 < ns^{n-1} < 1 \). Now suppose \( P(n)  \) holds for every \( n \leq k -1  \). We want to show that \( P(n) \) holds for the \( kth \) case. Observe that 
            \begin{align*}
                (k-1)s^{k-2} &= \frac{ ks^k - s^k }{ s^2 }    \\
            \end{align*}
            which means that for any \( 0 < s < 1  \), we have 
            \[  ks^k \leq s^2(k -1) + s^k < (k-1) + 1 = k  .  \]
            Hence, we have that \( ns^{n-1}  \) is bounded for all \( n \geq 1  \).
            \[   \]
        \end{proof}
    \item[(b)] Given an arbitrary \( x \in (-R ,R ) \), pick \( t  \) to satisfy \( | x  |  < t < R  \). Use this start to construct a proof for Theorem 6.5.6.
        \begin{proof}
        Let \( x \in (-R ,R ) \) be arbitrary. Pick \( t \in (-R ,R ) \) such that \( | x  |  < t < R  \). By using this bound on \( t  \), we can write the following
        \[  | na_n x^{n-1}  | = | n x^{n-1} \cdot a_n  | \leq | n t^{n-1}  \cdot a_n |.\]
        From part (a), we know that \( 0 < s < 1  \) implies that \( n s^{n-1} \) is bounded. Hence, there exists \( M > 0  \) such that
        \[  n t^{n-1} = n s^{n-1} \Big( \frac{ t  }{ s  }  \Big)^{n-1} \leq M \Big( \frac{ t }{ s }  \Big)^{n-1}.  \]
        Using the Weierstrass M-test, we can see that the bound above forms the following series
        \[  \sum_{ n=1  }^{ \infty  } Ma_n \Big( \frac{ t }{ s  }  \Big)^{n-1} \tag{1}.\]
        Since \( \sum_{ n=0 }^{ \infty  } a_n x^n  \) converges on (-R, R ), we know that (1) must converge and thus 
        \[  \sum_{ n=0  }^{ \infty  } | na_n x^{n-1}  |  \] must also converge.
        \end{proof}
\end{enumerate}

\subsubsection{Exercise 6.5.6} Previous work on geometric series (Example 2.7.5) justifies the formula 
\[  \frac{ 1 }{ 1 -x  } = 1 + x + x^2 + x^3 + x^4 + \dotsb, \text{~ for all } |  x  |  < 1. \]
Use the results about power series in this section to find values for \( \sum_{ n= 1  }^{ \infty  } n/ 2^n  \) and \( \sum_{ n=1  }^{ \infty  } n^2 / 2^n \). The discussion in Section 6.1 may be helpful.
\begin{proof}[Solution]

\end{proof}


\subsubsection{Exercise 6.5.7} Let \( \sum a_n x^n \) be a power series with \( a_n \neq 0  \) and assume 
\[  L = \lim_{ n \to \infty  } \Big| \frac{ a_{n+1} }{ a_{n} }  \Big|  \] exists.
\begin{enumerate}
    \item[(a)] Show that if \( L \neq 0  \), then the series converges for all \( x \in (- 1/ L, 1 / L ) \). (The advice in Exercise 2.7.9 may be helpful).
        \begin{proof}
        Since 
        \[  L = \lim_{ n \to \infty  } \Big| \frac{ a_{n+1} }{ a_{n} }  \Big|  \]
        where \( L \neq 0  \), we know that the sequence \( (a_{n+1}/ a_{n}) \) must be bounded by some \( M > 0  \); that is, \( | \frac{ a_{n+1} }{ a_n }   | \leq M  \). This implies that 
        \[  | a_{n+1}  |  \leq M | a_n  |. \]
        \end{proof}
        Now choose \( N \in \N  \) such that for any \(  n \geq N  \), we have that 
        \[  | a_N x^{N} | \leq  M | a_n | |x|^{n}. \tag{1}\]
        Since \( L \neq 0  \) and \( | x  |  < 1 / L  \), we must have that 
        \[  M | a_n | |x|^n < M a_n  \Big( \frac{ 1 }{ L  }  \Big)^{n} \tag{2}.  \]
        Since the right side of \( (2) \) forms the geometric series
        \[  \sum M a_n \Big( \frac{ 1 }{ L }  \Big)^n,\] we know that 
        \[  \sum a_n x^n  \] must converge absolutely by the Weierstrass M-test and hence it must converge uniformly for every \( x \in  (-1/L , 1 /L ) \).
    \item[(b)] Show that if \( L = 0  \), then the series converges for all \( x \in \R  \).
        \begin{proof}
        Let \( L = 0  \) and let \( x \in \R  \) be arbitrary. Since \( L < 1  \), we can  \( r' \in (L , 1 ) \) such that 
        \[  | a_n | \leq | a_N | (r')^n.\]
        Using the Weiertrass M-test, we know that \( r'x < 1  \) must imply that 
        \[  | a_n x^n  | \leq | a_N  | (r' x )^n \leq | a_N  | .\]
        Since \( L < 1  \), we know that the series 
        \[  \sum_{ N=1 }^{ \infty  } | a_N |   \] converges and hence the power series 
        \[  \sum_{ n=1  }^{ \infty  } a_n x^n \] must also converge for any \( x \in \R  \).
        
        \end{proof}
    \item[(c)] Show that \( (a)  \) and \( (b)  \) continue to hold if \( L  \) is replaced by the limit 
        \[  L' = \lim_{ n \to \infty  }  s_n \text{~ where ~} s_n = \sup \Big\{  \Big| \frac{ a_{k+1} }{ a_k }  \Big| : k \geq n  \Big\}. \]
        (General properties of the \textit{limit superior} are discussed in Exercise 2.4.7.)
        \begin{proof}
            Let \( L' \neq 0  \). Since \( s_n \to L'  \), we know that \( s_n  \) must be bounded. Hence, there exists some \( M > 0  \) such that \( | s_n | \leq M  \) for all \( n \in \N  \). Let \( x \in (-1/L', 1/L) \). Using the Weierstrass M-test, we can bound \( \frac{ a_{n+1} }{ a_n }  x^n  \) by its supremum (i.e the sequence \( s_n \)) such that 
            \begin{align*}
            \Big| \frac{ a_{n} }{ a_{N} } \cdot x^n  \Big| &\leq | s_n | | x |^n \\
                                                             &\leq M \Big( \frac{ 1 }{ L' }  \Big)^n. \\
            \end{align*}
            Since the last inequality forms the following geometric series
            \[  \sum M \Big( \frac{ 1 }{ L' }  \Big)^n. \]
            Hence, the power series 
            \[  \sum a_n x^n  \] must converge on \( (-1/L', 1 / L' ) \).
        To show that the power series converges for all \( x \in \R  \), we can employ the same process in part (b) since \( \lim | a_{n+1} / a_{n} | = \lim s_n   \).
        \end{proof}
\end{enumerate}


\subsubsection{Exercise 6.5.8} 
\begin{enumerate}
    \item[(a)] Show that power series representation are unique. If we have 
        \[  \sum_{ n=0 }^{ \infty  } a_n x^n = \sum_{ n=0 }^{ \infty  } b_n x^n \]
        for all \( x  \) in an interval \( (-R ,R ) \), prove that \( a_n = b_n  \) for all \( n \in \N  \).
        \begin{proof}
       Since 
       \[  \sum_{ n=0  }^{ \infty  } a_n x^n = \sum_{ n=0 }^{ \infty  } b_n x^n  \tag{1} \]
       for all \( x \in (-R ,R ) \), we know that the series above must also be differentiable and follow the formula below
       \[  \sum_{ n=1 }^{ \infty  } na_n x^{n-1} = \sum_{ n=1 }^{ \infty  } nb_n x^{n-1} \tag{2} \] by Theorem 6.5.6. If \( n = 0  \), we see that (1) implies 
       \[  a_0 = b_0. \] Now suppose \( n = 1  \), then (2) implies that 
       \[  a_1 = b_1. \] An inductive argument can be used to show that for every \( n \geq 0  \) with \( n  \) corresponding to the nth derivative of \( f  \), we can show that \( a_n = b_n  \) for all \( n \geq 0  \).
        \end{proof}
    \item[(b)] Let \( f(x) = \sum_{ n=0  }^{ \infty  } a_n x^n  \) converge on \( (-R ,R ) \), and assume \( f'(x) = f(x)  \) for all \( x \in (-R ,R ) \) and \( f(0) = 1  \). Deduce the values of \( a_n \).
        \begin{proof}[Solution]
        Assume \( f'(x) = f(x)  \). Since 
        \[  f(x) = \sum_{ n=0  }^{ \infty  } a_n x^n  \] converges for all \( x \in (-R ,R ) \), we know that \( f  \) must also be differentiable and satisfy the following formula 
        \[  f'(x) = \sum_{ n=1  }^{ \infty  } na_n x^{n-1}. \]  Furthermore, we know that \( f  \) is \textit{infinitely differentiable}. Since \( f'(x) = f(x)  \) we can change indices of the power series representation of \( f'(x)  \) so that we may use the proposition from part (a) to have 
        \[  \sum_{ n=0  }^{ \infty  } a_n x^n = \sum_{ n=0 }^{ \infty  } (n+1) a_{n+1} x^n \]
       implies \( a_n = b_n  \) for all \(  n \geq 0  \) where 
       \[  b_n = (n+1)a_{n+1}. \]
       Note that \( f(0) = 1  \) which implies that \( a_0 = a_1  \) since \( a_n = (n +1) a_{n+1} \) for all \( n \geq 0  \). Since each \( a_n  \) is defined recursively, we know that for \( n =2  \) that 
       \[  a_2 = \frac{ a_{1} }{ 2  } = \frac{ 1 }{ 1 \cdot 2 }  . \]
       For \( n = 3  \), we apply the same reasoning to get 
       \[  a_{3} = \frac{ a_{2} }{ 3  } = \frac{ 1 }{ 1 \cdot 2 \cdot 3  }.  \] Likewise, \( n =4  \) gives us 
       \[  a_4 = \frac{ a_{3} }{ 4  } = \frac{ 1 }{  1 \cdot 2 \cdot 3 \cdot 4  }.\] Continuing the pattern for \( n \geq 0  \), we arrive at the form of \( a_n  \) which is just 
       \[  a_n = \frac{ f(0) }{ n! }. \]
        \end{proof}
\end{enumerate}

\subsubsection{Exercise 6.5.10}  Let \( g(x) = \sum_{ n=0  }^{ \infty  } b_n x^n  \) converge on \( (-R ,R ) \), and assume \( (x_n) \to 0  \) with \( x_n \neq 0  \). If \( g(x_n) = 0  \) for all \( n \in \N  \), show that \( g(x)  \) must be identically zero on all of \( (-R ,R ) \).
\begin{proof}
Since the series 
\[  g(x) = \sum_{ n=0 }^{ \infty  } b_n x^n  \] converges on \( (-R ,R ) \); that is, uniformly for any compact set \( K \subseteq (-R ,R ) \) and the fact that \( g_n(x) = b_n x^n  \) is continuous for all \( n \geq 0  \), we know by the Continuity theorem that \( g(x)  \) must be a continuous function. Since \( (x_n) \to 0 \) with \( x_n \neq 0  \) and the fact that \( g(x_n) = 0  \) for all \( n \in \N  \), we know that \( g(x_n) \to 0  \). But this is equivalent to saying that 
\[  \lim_{ x \to 0  } g(x) = 0 = g(0).\]
But we need to show that \( g(x) = 0  \) for all \( x \in (-R ,R ) \). Since \( g(x)  \) converges on \( (-R ,R ) \), we know that \( g(x) \) must be differentiable and must follow 
\[  g'(x) = \sum_{ n=1  }^{ \infty  } n b_n x^{n-1}. \]
Since \( g(x_n) = 0  \) for all \( n \in \N  \), it must also follow that the differentiability of \( g  \) implies that \( g'(x_n) = g(x_n) = 0  \). Let \( i \geq 0  \) be the \( i \)th derivative of \( g \), (we know this holds since \( g \) is infinitely differentiable). Hence, part (a) implies that for every \( i \geq 0  \), we must have 
\[ g^{i} (x_{n})  = g^{i+1} (x_{n}) = 0  \] From part (a) of exercise 6.5.8, we know that the for all \( i \geq 0   \) and the fact that \( g(x_n) = 0  \) for all \( n \in \N  \), \( b_i = b_{i+1} = 0  \)   
\[  b_i = b_{i+1} \]  for all \( i \geq 0  \). Hence, for every \( b_n  \) must be zero showing that for any \( x \in (-R ,R ) \) that \( g(x) = 0  \).
\end{proof}

\subsubsection{Exercise 6.5.11} A series \( \sum_{ n=0  }^{ \infty  } a_n  \) is said to be \textit{Abel-summable}  to \( L  \) if the power series 
\[  f(x) = \sum_{ n=0  }^{ \infty  } a_n x^n  \] converges for all \( x \in [0,1) \) and \( L = \lim_{ x \to 1^{-} } f(x) \).
\begin{enumerate}
    \item[(a)] Show that any series that converges to a limit \( L  \) is also Abel-summable to \( L  \).
        \begin{proof}
            Let \( \epsilon > 0  \). To show that \( f(x) = \sum_{ n=0 }^{ \infty  } a_n x^n   \) converges to \( L  \), we will use the Cauchy Criterion and Abel's lemma. Note that since \( x \in [0,1) \), \( x^k   \) forms a monotonically decreasing sequence that contains terms that are greater than or equal to zero. Suppose there exists an \( N \in \N  \) such that for any \( n > n \geq N  \), then by Abel's Lemma 
        \begin{align*}
            \Big| \sum_{ k=m+1 }^{ n } a_k x^k  \Big|  &\leq \epsilon x^k  \\
                                                       &< \epsilon.
        \end{align*}
        Hence, we must have 
        \[  f(x) = \sum_{ n=0 }^{ \infty  } a_n x^n  \] converge to \( L  \) uniformly for all \( x \in [0,1) \). Since each term of \( \sum_{ n=0 }^{ \infty  } a_n x^n  \) is continuous and \( f(x)  \) converges uniformly for all \( [0, 1] \), we know that \( f(x)  \) is continuous on this interval. This means that the functional limit \( \lim_{ x \to 1^{-} } f(x) = L  \) exists and furthermore \( f(1) = L  \).
        \end{proof}
    \item[(b)] Show that \( \sum_{ n=0 }^{ \infty  } (-1)^n  \) is Abel-summable and find the sum.
        \begin{proof}[Solution]
        
        \end{proof}
\end{enumerate}


