\section{Uniform Convergence of a Sequence of Functions}


Just like our studies demonstrated in Chapter 2, we will first study the behaviors and properties of converging \textit{sequences} of functions. The results that we have gathered about sequences and series so fat will be immediately applicable to our study of sequences of functions.

\subsection{Pointwise Convergence} 

\begin{definition}{Pointwise Convergence}{}
For each \( n \in \N  \), let \( f_n \) be a function defined on a set \( A \subseteq \R  \). The sequence \( (f_n)  \) of functions \textit{converges pointwise }  on \( A  \) to a function \( f  \) if, for all \( x \in A  \), the sequence of real numbers \( f_n(x)  \) converges to \( f(x)  \). 

In this case, the following notations are all equivalent to each other
\begin{enumerate}
    \item[(i)] \( f_n \to f  \) 
    \item[(ii)] \( \lim f_n = f  \)
    \item[(iii)] \( \lim_{ n \to \infty  } f_n(x) = f(x)  \).
\end{enumerate}
\end{definition}

(iii) of the definition above is especially useful if there are any confusions the may arise as to whether or not \( x  \) or \( n  \) is the limiting variable.

\begin{example}{}{}
\begin{enumerate}
    \item[(i)] Consider the sequence of functions \( f_n  \) defined by 
        \[  f_n(x) = \frac{ x^2 + nx  }{ n } \]
        on all of \( \R  \). We can compute the limit of \( f_n  \)
        \[  \lim_{ n \to \infty  } f_n(x) = \lim_{ n \to \infty  } \frac{ x^2 + nx  }{ n  } = \lim_{ n \to \infty  } \frac{ x^2  }{ n } + x  = x. \] Thus, we have that \( (f_n) \) converges \textit{pointwise} to \( f(x) = x  \) on \( \R  \).
    \item[(ii)] Let \( g_n(x) = x^n  \) on the set \( [0,1]  \) where we consider the situation as \( n \to \infty  \). If \( 0 \leq x < 1  \), then we know that \( x^n \to 0  \). On the other hand, suppose \( x = 1  \), then we have that \( x^n \to 1  \). It follows that \( g_n \to g  \) converges pointwise on \( [0,1] \), where 
        \[  g(x) = 
        \begin{cases}
            0 &\text{for } 0 \leq x < 1 \\
            1 &\text{for } x = 1.
        \end{cases} \]
        We have a problem when considering continuity at \( x = 1  \).
    \item[(iii)] Consider \( h_n(x) = x^{1+ \frac{ 1 }{ 2n - 1  } } \) on the set \( [-1,1]  \). For a fixed \( x \in [-1,1]  \), we have 
        \[  \lim_{ n \to \infty  } h_n(x) = x \lim_{ n \to \infty  } x^{\frac{ 1 }{ 2n-1 } }  = | x  |.\]
        Note that this function is not differentiable at \( x = 0  \).
\end{enumerate}
\end{example}

\subsection{Continuity of the Limit Function}

We will begin this section by failing to prove that the pointwise limit of continuous functions is continuous. We will then find the holes of the subsequent argument so that we may understand why we need a stronger footing on the meaning of convergence for a sequence of functions.

Let \( (f_n)  \) be a sequence of continuous functions on a set \( A \subseteq \R  \) and let us assume that \( (f_n)  \) converges to a pointwise limit \( f  \). We will try to argue that the limit \( f  \) is continuous. Let us fix \( c \in A  \), and let \( \epsilon > 0  \). Our objective is to find  \( \delta > 0 \) such that whenever \( | x - c  | < \delta  \), we have 
\[  | f(x) - f(c) | < \epsilon. \]
We may use the Triangle Inequality to write
\begin{align*}
    | f(x) - f(c)  | &= | f(x) - f_n(x) + f_n(x) - f_n(c) + f_n(c) - f(c)   |  \\
                     &\leq | f(x) - f_n(x)  | + | f_n(x) - f_n(c)  | + | f_n(c) - f(c)  |.
\end{align*}
Our impression is to make each term of the right hand side of this inequality small by using the fact that \( f_n \to f  \) and the continuity of \( f_n  \).  Since \( c \in A  \) is fixed, let us choose \( N \in \N  \) such that 
\[  | f_N(c) - f(c)  | < \frac{ \epsilon  }{ 3 }. \]
Since \( N  \) is chosen, the continuity of our particular choice \( f_N  \) implies that there exists a \(  \delta > 0  \) such that 
\[  | f_N(x) - f_N(c)  | < \frac{ \epsilon  }{ 3 } \]
for all \( x  \) whenever \( | x - c  | < \delta  \). But here lies the problem of using the continuity of \( f_n  \); that is, we also need the following to hold:
\[  | f_N(x) - f(x)  | < \frac{ \epsilon  }{ 3 }  \] for all \( x  \) satisfying \( | x - c  | < \delta  \). A few problems with this argument include 
\begin{enumerate}
    \item[(i)] Our choice of \( x  \) depends on \( \delta  \) which also depends on our choice of \( N  \). This means for every choice of \( x  \) along \( (c - \delta, c + \delta ) \), we will get a different \( N  \). We want our choice of \( \delta  \) to be uniform for any \( x  \).
    \item[(ii)] The choice of \( x  \) is not fixed the way \( c  \) is on the interval \( (c - \delta, c + \delta ) \). This means that our choice \( x  \) has to work along the interval.
\end{enumerate}
This problem is apparent in our second example at the beginning of this section where the inequality
\[  | g_n(1/2 ) - g(1/2)  | < \frac{ 1 }{ 3 } \]
for \( n \geq 2  \) whereas 
\[  |g_n(9/10) - g(9/10)   | < \frac{ 1 }{ 3 }  \]
is true only after \( n \geq 11 \).

\subsection{Uniform Convergence} 

To solve our the problems of pointwise convergence of functions, we introduce a stronger notion for convergence of functions.

\begin{definition}{Uniform Convergence}{}
    Let \( (f_n)  \) be a sequence of functions defined on a set \( A \subseteq \R  \). Then, \( (f_n)  \) \textit{converges uniformly} on \( A  \) to a limit function \( f  \) defined on \( A  \) if, for every \( \epsilon  > 0 \), there exists an \( N \in \N  \) such that 
    \[  | f_n(x) - f(x)  | < \epsilon   \]
    whenever \( n \geq N  \) and \( x \in A  \).
    \end{definition}

Let us restate the definition of Pointwise convergence so that we are able to distinguish the key differences between the two.

\begin{definition}{Pointwise Convergence}
    Let \( (f_n)  \) be a sequence of functions defined on a set \( A \subseteq \R  \). Then, \( (f_n)  \) \textit{converges pointwise on} A to a limit \( f  \) defined on \( A  \) if, for every \( \epsilon > 0   \) and \( x \in A  \), there exists an \( N \in \N  \) (may depend on x ) such that 
    \[  | f_n(x) - f(x)  | < \epsilon  \] whenever \( n \geq N  \).
    \end{definition}

Key Differences:
\begin{enumerate}
    \item[(i)] In uniform convergence, notice that we only need 
        \[ | f_n(x) - f(x)  | < \epsilon  \] to hold for all \( \epsilon > 0  \); that is, our choice of \( x    \) will not affect our choice of \( N  \). Another way to state this is \( N \neq N(\epsilon, x ) \)
    \item[(ii)] In pointwise convergence, not only do we need convergence to hold for all \( \epsilon > 0   \), we also need it to hold for all \( x  \).
\end{enumerate}

\begin{example}{}{}
\begin{enumerate}
    \item[(i)] Let 
        \[  g_n(x) = \frac{ 1 }{ n(1 + x^2 ) }.  \]
        For any fixed \( x \in \R  \), it is apparent that \( \lim_{ n \to \infty  } g_n(x) = 0  \) so that \( g(x) = 0   \) is the pointwise limit of the sequence \( (g_n)  \) on \( \R  \). We want to know if \( (g_n)  \) uniformly convergent. Since \( 1 / (1 + x^2 )  \leq 1  \) for all \( x \in \R  \) implies that 
        \[  | g_n(x) - g(x)  | = \Big| \frac{ 1 }{ n(1+x^2 )  } - 0  \Big| \leq \frac{ 1 }{ n }.  \]
Hence, any given \( \epsilon > 0  \), we can choose \( N > 1 / \epsilon  \) (which does not depend on \( x  \)), we have that
\begin{center}
    \( n \geq N  \) implies \( | g_n(x) - g(x)  | < \epsilon  \)
\end{center}
for all \( x \in \R  \). Hence, \( g_n \to 0 \) uniformly on \( \R  \).
\item[(ii)] What about our first example from the very beginning of this section? Does it converge uniformly as well?  Let \( f_n(x) = (x^2 + n x ) / n  \). Since \( (f_n) \to f  \) pointwise where \( f(x) =x  \). It turns our that \( f_n  \) is not uniformly convergent. To see why this is the case, we write that
    \[  | f_n(x) - f(x)  | = \Big| \frac{ x^2 + nx  }{ n  } - x  \Big| = \frac{ x^2  }{ n }. \]
    For \( | f_n(x) - f(x)  | < \epsilon  \) to hold, we would need to create a choice of \( N  \) such that 
    \[  N  > \frac{ x^2  }{ \epsilon  }. \]
    While we certainly have convergence for every \( x \in \R  \), we still have our choice of \( N  \) not uniform. Although not uniformly convergent on all of \( \R  \), we do end up having uniform convergence when we consider \( f_n  \) over a closed interval \( [-b,b] \). Hence, we have that 
    \[  \frac{ x^2  }{ n }  \leq \frac{ b^2  }{ n  }. \]
    Given any \( \epsilon > 0  \), we can choose \( N > b^2 / \epsilon  \) that is not dependent on any \( x \in [-b,b]  \).
\end{enumerate}
\end{example}

Graphically speaking, the uniform convergence of \( f_n  \) to a limit \( f  \) on a set \( A  \) can be visualized by constructing an \( \epsilon -  \)neighborhood around the limit \( f  \) for which all of \( f_n  \) is completely contained within the neighborhood for all \(  n \geq N  \) for some point \( N \in \N  \).

\subsection{Cauchy Criterion}

Recall that the Cauchy Criterion states an equivalence between convergent sequences and Cauchy sequences without stating the limit of the sequence. The usefulness of such a theorem creates an opportunity for an analogous  characterization of uniformly convergent sequences of functions.

\begin{theorem}{Cauchy Criterion for Uniform Convergence}{}
    A sequence of functions \( (f_n)  \) defined on a set \( A \subseteq \R  \) converges uniformly on \( A  \) if and only if for every \( \epsilon > 0  \), there exists an \( N \in \N  \) such that 
    \[  | f_n(x) - f_m(x)  | < \epsilon  \]
    whenever \( m,n \geq N  \) and \( x \in A  \).
    \end{theorem}
\begin{proof}
Exercise 6.2.5.
\end{proof}

\subsection{Continuity Revisited}

Let us now prove that the limit function of a sequence of continuous functions is continuous.

\begin{theorem}{Continuous Limit Theorem}{}
    Let \( (f_n)  \) be a sequence of functions defined on \( A \subseteq \R  \) that converges uniformly on \( A  \) to a function \( f  \), If each \( f_n   \) is continuous at \( c \in A  \), then \( f  \) is continuous at \( c  \).
    \end{theorem}

\begin{proof}
Fix \( c \in A  \) and let \( \epsilon > 0  \). Since \( (f_n) \to f   \) is uniformly convergent on \( \R  \), we can choose an \( N \in \N   \) (that does not depend on \( x  \)) such that 
\[  | f_N(x) - f(x)  | < \frac{ \epsilon  }{ 3  }  \]
for all \( x \in A  \). Since \( f_N  \) is continuous, there exists \( \delta > 0  \) for which 
\[  | f_N(x) - f_N(c)  | < \frac{ \epsilon  }{ 3 }  \]
is true whenever \( | x - c  | < \delta  \). Just like our argument at the beginning of this section, we have that
\begin{align*}
    | f(x) - f(c)  | &= | f(x) - f_N(x) + f_N(x) - f_N(c) + f_N(c) - f(c)  |  \\
                     &\leq | f(x) - f_N(x)  | + | f_N(x) - f_N(c)  | + | f_N(c) - f(c)  | \\
                     &< \frac{ \epsilon  }{ 3  } + \frac{ \epsilon  }{ 3 } + \frac{ \epsilon  }{ 3  } \\
                     &= \epsilon.
\end{align*}
Hence, \( f  \) is continuous at \( c \in  A  \).
\end{proof}




