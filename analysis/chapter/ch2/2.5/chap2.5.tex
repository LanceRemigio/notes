\section{Subsequences and Bolzano-Weierstrass}

In the last section, we observed that the convergence of partial sums of a particular series can be determined by the behavior of a subsequence of the partial sums. 

\begin{definition}[Subsequences]
Let \( (a_n) \subseteq \R \), and let \( n_1 < n_2 < n_3 < n_4 < \dots\) be an increasing sequence of natural numbers. Then the sequence 
\[ (a_{n_1}, a_{n_2}, a_{n_3}, a_{n_4}, a_{n_5} \dots)\]
is called a \textit{subsequence} of \( (a_n)\) and is denoted by \( (a_{n_k})\), where \( k \in \N \) indexes the subsequence.
\end{definition}%

A few remarks about subsequences: 

\begin{enumerate}
    \item[(a)] The order of the subsequence is the same as in the original sequence.
        If we have the sequence  
        \[ (a_n) = \Big( 1, \frac{1}{2}, \frac{1}{3}, \frac{1}{4}, ...\Big)\]
        then the subsequences 
        \[ \Big( \frac{1}{2}, \frac{1}{4}, \frac{1}{6}, \frac{1}{8},... \Big)\]
        and 
        \[ \Big( \frac{1}{10}, \frac{1}{100}, \frac{1}{1000}, \frac{1}{10000}, ...\Big)\]
        are permitted.
        
    \item[(b)] Repetitions and swapping are not allowed.
        Like 
        \[ \Big( \frac{1}{10}, \frac{1}{5}, \frac{1}{100}, \frac{1}{50}, \frac{1}{1000}, \frac{1}{500},...\Big)\] 
        and 
        \[ \Big( 1,1, \frac{1}{3}, \frac{1}{5}, \frac{1}{5}, ...\Big)\]
        
\end{enumerate}


Since subsequences have the same ordering as the original sequence, one can conjecture about them converging to the same limit. 

\begin{theorem}{}{}
Subsequences of a convergent sequence converge to the same limit as the original sequence.
\end{theorem}

\begin{proof}
    Let \( (a_n) \to a \) and let \( (a_{n_k})\) be a subsequence for \( (a_n)\). We want to show \( (a_{n_k})\) converges to \( a \) as well. Since \( (a_n) \to a\), there exists an \( N \) such that for any \( n \geq N \), we have \( |a_n - a| < \epsilon \). 

    We claim that \( n_{k} \geq k \) for any \( k \in \N\). Let us proceed by inducting on \( k \). Let the base case be \( k = 1 \). Since \(n_k\) is an \textit{increasing} sequence of natural numbers, we see that \( n_1 \geq 1\). Now let us assume \(n_{k-1} \geq k - 1\). Since \( (a_{n_k})\) in \textit{increasing}, we have \( a_k \geq a_{k-1} \geq k - 1\) which implies that \( n_k \geq k \). 

    Since any choice of \( n \geq N \), we can say that \( n_k \geq k \geq N \). Hence, we have 
    \[ |a_{n_k} - a| < \epsilon \]
which is what we desired.
\end{proof}

Let \(0 < b < 1\). Because 
\[ b > b^2 > b^3 > b^4 > ... > 0,\]
the sequence \((b^n)\) is \textit{decreasing} and \textit{bounded} below. The Monotone Convergence Theorem allows us to conclude that \((b^n)\) converges to some \(\ell\) satisfying \( 0 \leq \ell < b\). To compute \(\ell\), notice that \((b^{2n})\) is a subsequence, so \(b^{2n} \to \ell\) by Theorem 2.5.2. But \( b^{2n} = b^n \cdot b^n\), so by the Algebraic Limit Theorem, \(b^{2n} \to \ell \cdot \ell = \ell^2 \). Because limits are unique (Theorem 2.2.7), \( \ell^2 = \ell\), and thus \( \ell = 0\). 











Suppose we have an oscillating sequence of numbers 
\[ \Big( 1, -\frac{1}{2}, \frac{1}{3}, - \frac{1}{4}, \frac{1}{5}, -\frac{1}{5}, \frac{1}{5}, -\frac{1}{5},... \Big)\]
Note that this sequence does not converge to any proposed limit yet if we take a subsequence of it, weget a sequence that converges! Observe, that the subsequence 
\[ \Big(\frac{1}{5}, \frac{1}{5}, \frac{1}{5}, \frac{1}{5},... \Big)\]
and
\[ \Big(-\frac{1}{5}, -\frac{1}{5}, -\frac{1}{5}, -\frac{1}{5},... \Big)\]
converge to \( 1 / 5 \) and \(- 1 / 5\) respectively. Since we have two subsequences that converge to two different limits, we immediately conclude that the original sequence diverges.

This leads us to our next theorem that states that
% add definitions at the end of each chapter or put references to them
\begin{theorem}[Bolzano-Weierstrass]
        Every bounded sequence contains a convergent subsequence.
\end{theorem}%

\begin{proof}
    Let \( (a_n)\) be a \textit{bounded} sequence. Then there exists \( M > 0 \) such that \( a_n \in [-M,M ]\). Suppose we divide this interval in half for \( k \) times: that is, let the length of the intervals be defined by the sequence \( M (1/2)^{k-1}\). We claim that a subsequence \( (a_{n_k})\) lies in either one of these intervals: that is, let \( n_{k} > n_{k-1}\) for all \( k \in \N \) such that \( a_{n_k} \in I_k\). 

    Let us induct on \( k\). Then let our base case be \( k = 1 \). Since we have an increasing sequence of natural numbers \( n_k\), we have that \( n_2 > n_1 \) which means that \( a_{n_2} \in I_2\) as well as \( a_{n_1} \in I_1\). Now let us assume that this holds for all \( k \leq \ell - 1 \). We want to show that this holds for \( k < \ell \). By the monotonicity of \( n_k \), we have that \( n_{\ell} > n_{\ell - 1} > n_k > n_1\) which implies that \( a_{n_\ell} \in I_\ell\) for all \( \ell \in \N \). Furthermore, the sets 
    \[ I_1 \subseteq I_2 \subseteq I_3 ...\]
form a nested sequence of closed intervals.

    By the \textit{Nested Interval Property}, we can conclude that there exists an \( x \in I_k\) for all \( k \in \N \)  such that \( \bigcup_{k=1}^{ \infty} I_k \neq \emptyset\). Let \( \epsilon > 0  \). Since \( a_{n_k}, x \in I_k \) for all \( k \in \N \) and \( M (1/2)^{k-1} \to 0\) by the Algebraic Limit Theorem, we can choose an \( N \in \N \) such that for any \( k \geq n_k \geq N \), we have
    \[ |a_{n_k} - x | < \epsilon.\] 
    Hence, \( (a_{n_k}) \to x \).
\end{proof}%




