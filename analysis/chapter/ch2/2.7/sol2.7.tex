\section{Properties of Infinite Series}

\subsubsection{Exercise 2.7.1} Proving the \textit{Alternating Series Test} amounts to showing that the sequence of partial sums 
\[ s_n = a_1 - a_2 + a_3 - ... \pm a_n\] converges. (The opening example in Section 2.1 includes a typical illustration of \((s_n)\). Different characterizations of completeness lead to different proofs. 
\begin{enumerate}
    \item[(a)] Prove the \textit{Alternating Series Test} by showing that \( (s_n)\) is a \textit{Cauchy Sequence}. 
        \begin{proof}
            Let \( (a_n)\) be a \textit{decreasing sequence} and suppose \( (a_n) \to 0 \). We want to show that the \textit{Alternating series} \( \sum_{n=1}^{\infty} (-1)^{n+1} a_n\) meets the \textit{Cauchy Criterion}. 

            We first need to show that for every \( n > m\), we have the property
            \[ 0 \leq |a_{m+1} - a_{m+2} + a_{m+3} - ... \pm a_n| \leq |a_{m+1}|\]
            Hence, we proceed by induction on \( k\). Note that 

            \[ \sum_{k=m+1}^{n} (-1)^{k+1}a_k = a_{m+1} - a_{m+2} + a_{m+3} - ... \pm a_n .\]
            Let our base case be \( P(1)\). Then \( a_{m+1} \geq 0\). For \( P(2)\), we have \( a_{m+1} \geq a_{m+2}\) for all \( m \) since \( (a_n)\) is a \textit{decreasing sequence}. Suppose this holds for all \( m \leq k-1   \). We want to show that this holds for \( P(k)\). Since \((a_n)\) is \textit{decreasing}, we have that \( a_{k-1} \geq a_{k}\). Hence, \( a_{k-1} - a_{k} \geq 0 \). Since \( P(k-1)\) holds where 
            \[ 0 \leq a_{m+1} - a_{m+2} + a_{m+3} - ... \pm a_{k-1} \leq a_{m+1}.\]
            But this means that every term leading up to \( a_k \) is bounded by \( a_{m+1}\). Hence, 
            \[ 0 \leq a_{m+1} - a_{m+2} + a_{m+3} - ... \pm a_k \leq a_{m+1}.\]
        
        Let \( \epsilon > 0 \). All is left to show is that 
        \[ \Big|\sum_{k=1}^{n}(-1)^{k+1} a_k \Big| < \epsilon.\]
        Hence, for some \( N \in \N \), let \( n > m \geq N \) and \( (a_n) \to 0 \) such that 
        \begin{align*}
            \Big| \sum_{k=1}^{\infty} (-1)^{k+1} a_k \Big|&\leq |a_{m+1}| \\
                                        &< \epsilon.            
        \end{align*}
        Hence, the series \( \sum_{k=1}^{\infty}(-1)^{k+1} a_k \) meets the \textit{Cauchy Criterion}.
        \end{proof}
    \item[(b)] Supply another proof for this result using the Nested Interval Property. 
        \begin{proof}
            Suppose \( (a_n)\) is \textit{decreasing} sequence and \( (a_n) \to 0 \). Our goal is to show the series \( \sum_{n=1}^{\infty} (-1)^{n+1}a_n\) converges. Since \( (a_n)\) is \textit{decreasing}, we can use the \textit{Nested Interval Property} to construct closed intervals \( I_n = [s_n, s_{n+1}] \) such that the length of these intervals is \( |s_{n} - s_{n+1}| \leq a_n\). The \textit{Nested Interval Property} gurantees the following property that 
            \[ I_1 \subseteq I_2 \subseteq I_3 \subseteq ...  \]
            where \( \bigcap_{n=1}^{\infty} I_n \neq \emptyset\). Hence, \( S \in \R \) can be our candidate limit since \( S \in I_n\) for all \( n \). Let \( \epsilon > 0 \). Since \( (a_n) \to 0 \), there exists \( N \in \N \) such that \( n \geq N \)
            \[ |s_n - S | \leq a_n < \epsilon.\]
        \end{proof}
        Hence, \( (s_n) \to S \). 
    \item[(c)] Consider the subsequences \( (s_{2n})\) and \( (s_{2n+1})\), and show how the \textit{Monotone Convergence Theorem} leads to a third proof for the \textit{Alternating Series Test}. 
        \begin{proof}
        Define the subsequence of partial sums \( (s_{2n})\) as 
        \[ \sum_{k=1}^{n} (-1)^{2k} a_{2k}.\]
        Since \( (a_n)\) is a \textit{decreasing sequence}, we have that \( a_n \geq a_{n+1}\) for all \( n \in \N \). Observe that 
        \begin{align*}
            s_1 &= a_2 \geq 0  \\
            s_2  &= a_2 + a_4  \geq s_1 \\
            s_3 &= a_2 + a_4 + a_6 \geq s_2 \\
                &\vdots \\ 
            s_n &= a_2 + a_4 + a_6 + ... + a_{2n}.
        \end{align*}
        We can see that \( s_{2n}\) is an \textit{increasing sequence}. Also, \(|s_{2n}| < M \) since \( (a_n)\) is a \textit{bounded sequence}. Hence, we can conclude that the subsequence of partial sums \( (s_{2n})\) is converges to some \( S \in \R \). 

        We can show that \( (s_{2n+1})\) converges to \( S \) as well. Since \( s_{2n+1} = s_{2n} + a_{2n+1} \), we can use the \textit{Algebraic Limit Theorem} to say that 
        \begin{align*}
            \lim(s_{2n+1})&= \lim(s_{2n} + a_{2n+1}) \\
                          &= \lim(s_{2n}) + \lim (a_{2n+1}) \\
                          &= S + 0 \\
                          &= S.
        \end{align*}
        Since \( (s_{2n}) \to S \) and \( (s_{2n+1}) \to S \), we have \( (s_n) \to S \) as well. 
        \end{proof}
\end{enumerate}

\subsubsection{Exercise 2.7.4}

\begin{enumerate}
    \item[(a)] Provide the details for the proof of the Comparison Test (Theorem 2.7.4) using the \textit{Cauchy Criterion} for Series.
        \begin{proof}
        Suppose \( (a_k)\) and \( (b_k)\) are sequences such that \( 0 \leq a_n \leq b_n\) for all \( n \in \N \). Assume \( \sum_{n=1}^{\infty} b_n \) converges. Our goal is to show that \( \sum_{k=1}^{\infty} a_k \) converges. Define the sequence of partial sums for \( \sum_{n=1}^{\infty}a_n\) as 
        \[ t_n = \sum_{k=1}^{n}a_k.\]
        Let \(\epsilon > 0 \). Since \( a_k \leq b_k \) and \( \sum_{n=1}^{\infty}b_n\) converges, there exists \( N \in \N \) such that for all \( n > m \geq N \), we have  
        \begin{align*}
             |t_n - t_m |&= \Big|\sum_{k=m+1}^{n} a_k\Big| \\
                         &\leq \Big|\sum_{k=m+1}^{n} b_k \Big| \\
                         &< \epsilon.
        \end{align*}
        Hence, the series \( \sum_{n=1}^{\infty} a_n \) converges. Note that part (ii) is just the contrapositive of part (i). Hence, it is also true.
        \end{proof}
    \item[(b)] Give another proof for the \textit{Comparison Test}, this time using the \textit{Monotone Convergence Theorem}.
        \begin{proof}
        Suppose the series \( \sum_{n=1}^{\infty} b_n \) converges. Our goal is to use the \textit{Monotone Convergence Theorem} to show that \( \sum_{n=1}^{\infty} a_n \) converges i.e our goal is to show that the sequence of partial sums \( t_n = \sum_{k=1}^{n} a_n \) is \textit{bounded} and \textit{monotone}. 

        Since the sequence of partial sums of \( \sum_{n=1}^{\infty} b_n \) are \textit{bounded} and \( 0 \leq a_n \leq b_n \) for all \( n \in \N \), it follows that we have \( |t_n| \leq M  \) as well. 

        Now we want to show that \( (t_n )\) is a \textit{decreasing sequence}. Since \( \sum_{n=1}^{\infty} b_n \) is convergent, we know that \( b_n \to  0 \). Since \( a_n \geq 0 \) and \( (b_n) \to 0 \), the terms \( (t_n)\) must also be \textit{decreasing}. Hence, \( t_{n+1} \leq t_n \) for all \( n \in \N \).
        
        Since \( (t_n)\) is both \textit{decreasing} and \textit{bounded}, it follows that \( \sum_{n=1}^{\infty} a_n \) is a convergent
        \end{proof}
\end{enumerate}

\subsubsection{Exercise 2.7.4} Give an example of each or explain why the request is impossible referencing the proper theorem(s).
\begin{enumerate}
    \item[(a)] Two series \( \sum x_n \) and \( \sum y_n \) that both diverge but where \(\sum x_ny_n \) converges.
        \begin{proof}[Solution]
        Take \(\sum x_n = (-1)^n\) and \( \sum y_n = 1/n\). These two series diverge but \(\sum x_n y_n = (-1)^n / n \) converges.
        \end{proof}
    \item[(b)] A convergent series \( \sum x_n \) and a bounded sequence \( (y_n)\) such that \( \sum x_n y_n \) diverges.
        \begin{proof}[Solution]
        Take the convergent series \( \sum 1 / n^2\) and the bounded sequence \( y_n = \sin(n)\). We have \( \sum x_n y_n = \sum \sin(n)/n^2\) is divergent by the comparison test.
        \end{proof}
    \item[(c)] Two sequences \( (x_n)\) and \((y_n)\) where \(\sum x_n \) and \( \sum (x_n + y_n)\) both converges but \( \sum y_n \) diverges.
        \begin{proof}[Solution]
        This is impossible. By the Algebraic Series Theorem, we cannot have \( \sum (x_n + y_n)\) converge without \( \sum y_n \) converging as well. 
        \end{proof}
    \item[(d)] A sequence \( (x_n)\) satisfying \( 0 \leq x_n \leq 1/n\) where \( \sum (-1)^n x_n\) diverges.
        \begin{proof}[Solution]
        By the comparison test, \( \sum (-1)^n x_n \) diverges.
        \end{proof}
\end{enumerate}

\subsubsection{Exercise 2.7.5} Now that we have proved the basic facts about geometric series, supply a proof for Corollary 2.4.7.
\begin{tcolorbox}
\begin{cor}
The series \( \sum_{n=1}^{\infty} 1/n^p \) converges if and only if \( p > 1 \).
\end{cor}
\end{tcolorbox}
\begin{proof}
    We start with the backwards direction. Suppose \( p > 1 \). Our goal is to show that \( \sum_{n=1}^{\infty} 1/n^p\) converges. Notice that \( b_n = 1/n^p\) where \( b_n \geq 0 \) and \( b_n \)
\textit{decreasing}. By the \textit{Cauchy Condensation Test}, we can prove that 
\[ \sum_{n=0}^{\infty} 2^n b_{2^n} = \sum_{n=0}^{\infty} 2^n \Big( \frac{1}{2^p}\Big)^n.\]
converges. Since \( p > 1 \), we have that 
\[ \sum_{n=0}^{\infty} 2^n \Big( \frac{1}{2^n}\Big)^p = \sum_{n=0}^{\infty} 2^n \Big( \frac{1}{2^n}\Big)^p = \sum_{n=0}^{\infty} 2^n \Big( \frac{1}{2^n}\Big)^{p-1} = \sum_{n=0}^{\infty} \Big(\frac{1}{2^p} \Big)^n.\]
Since \( |r| = |1/2^p| < 1 \), we know that \( \sum_{n=0}^{\infty}2^n b_{2^n}\) is a \textit{Geometric Series}. By the \textit{Cauchy Condensation Test}, we can say that \( \sum_{n=1}^{\infty} b_n \) converges. 

For the forwards direction, since  \(\sum_{n=0}^{\infty} 2^n b_{2^n}\) converges, the only reasonable choice of \( p \) is when \( p > 1 \) or else it is \textit{Harmonic Series} which diverges.
\end{proof}

\subsubsection{Exercise 2.7.6}
Let's say that a series \textit{subverges} if the sequence of partial sums contains a subsequence that converges. Consider this (invented) definition for a  moment, and then decide which of the following statements are valid propositions about \textit{subvergent} series: 
\begin{enumerate}
    \item[(a)] If \( (a_n)\) is \textit{bounded}, then \( \sum a_n \) \textit{subverges}.
        \begin{proof}[Solution]
        This is a valid proposition since the sequence of partial sums for \( \sum_{n=1}^{\infty}a_n\) are bounded which implies that the sequence of partial sums contains a subsequence partial sums that is convergent. Hence, we can say that \( \sum a_n \) is a \textit{subvergent} series. 
        \end{proof}
    \item[(b)] All convergent series are \textit{subvergent}. 
        \begin{proof}[Solution]
        This is valid since the sequence of partial sums for a convergent series converges and hence all of the possible subsequence of partial sums for the series converges to the same limit. 
        \end{proof}
    \item[(c)] If \( \sum |a_n| \) \textit{subverges}, then \(\sum a_n \) \textit{subverges} as well. 
        \begin{proof}[Solution]
        This is not valid. 
        \end{proof}
    \item[(d)] If \( \sum a_n \) \textit{subverges}, then \( (a_n)\) has a convergent subsequence. 
        \begin{proof}[Solution]
        This is not valid.
        \end{proof}
\end{enumerate}

\subsubsection{Exercise 2.7.7}
\begin{enumerate}
    \item[(a)] Show that if \( a_n > 0 \) and \( \lim (na_n) = l  \) with \( l \neq 0 \), then the series \( \sum a_n \) diverges.
        \begin{proof}
        Suppose for sake of contradiction that \( \sum a_n  \) converges. Hence, \( (a_n) \to 0 \). This means that \( \lim (na_n) = 0 \) but this contradicts our assumption that \( \lim (na_n) = l \neq 0 \). Hence, the series \( \sum a_n \) must diverge. 
        \end{proof}
        Another why is to use the limit assumption directly. 
        \begin{proof}
        Suppose \( a_n > 0 \) and \( \lim (na_n) = l \). We want to show that \( \sum a_n  \) diverges. Since \( \lim (na_n) = l \neq 0 \), let \( \epsilon = 1  \) such that there exists \( N \in \N \) such that whenever \( n \geq N \) for all \( n \), we have 
        \[ |na_n - l | < 1 \iff a_n < \frac{1+ l }{n }. \]
        This implies that 
        \[ \sum_{n=1}^{\infty } a_n < \sum_{n=1}^{\infty} \frac{1+l}{n}.\]
        Note that \( \sum \frac{1+l }{n} \) is not a \textit{p-series} since \( n^p\) where \(p=1\). Hence, the series \( \sum \frac{1+l}{n}\) diverges. Hence, we have that \( \sum a_n \) is also a divergent series by the comparison test. 
        \end{proof}
    \item[(b)] Assume \( a_n > 0 \) and \( \lim (n^2 a_n )\) exists. Show that \( \sum a_n \) converges. 
        \begin{proof}
            Suppose \( a_n > 0 \) and \( \lim (n^2 a_n )\) exists. Suppose \( \lim (n^2 a_n  ) = L \) for some \( L \in \R \). Let \( \epsilon = 1 \), there exists \(  N \in \N \) such that whenever \( n \geq N \), we have 
            \[ |n^2a_n - L | < \epsilon.\]
        Hence, we have
        \[ n^2 a_n - L < 1 \iff a_n < \frac{1+ l }{n^2 } \tag{1}\]
        Our goal is to show via \textit{comparison test} that the series \( \sum a_n \) converges.
        From (1), we have 
        \[\sum_{n=1}^{\infty} a_n < \sum_{n=1}^{\infty} \frac{1 + l }{n^2 }.  \]
        Observe that the series \( \sum \frac{1+l}{n^2} \) is a \textit{p-series} test which converges. Hence, the series \( \sum a_n \) converges by the \textit{Comparison test}.
        \end{proof}
\end{enumerate}

\subsubsection{Exercise 2.7.8}

Consider each of the following propositions. Provide short proofs for those that are true and counterexamples for any that are not.

\begin{enumerate}
    \item[(a)] If \( \sum a_n \) \textit{converges absolutely}, then \( \sum a_n^2\) also \textit{converges absolutely}.
        \begin{proof}
        Since \( \sum a_n \) converges absolutely, then we have the series \( \sum |a_n| \) converges. In order for \( \sum a_n^2 \) to converge absolutely, we need to show that \( \sum | a_n^2 |\) converges. Furthermore, \( (a_n) \) is a \textit{bounded} sequence. Hence, there exists \( M > 0 \) such that \( |a_n| \leq M \). Since there exists \( N \in \N \), for any \( n \geq N \), we can write
        \begin{align*}
            \sum |a_n^2|&= \sum | a_n \cdot a_n |  \\
                        &= \sum |a_n | \cdot |a_n | \\
                        &\leq \sum M \cdot |a_n | \\
                        &= M \sum |a_n| \\
        \end{align*}
        We know by the \textit{Algebraic Limit Theorem} for series that \( M \sum |a_n|\) converges. Hence, the series \( \sum a_n^2 \) converges absolutely by the \textit{Comparison Test}.
        \end{proof}
    \item[(b)] If \( \sum a_n \) converges and \( (b_n)\) converges, then \( \sum a_n b_n \) converges. 
        \begin{proof}
        Since \( (b_n)\) converges, we have that \( (b_n)\) is also \textit{bounded}. Hence, there exists \( M > 0 \) such that for all \( n \) we have \( b_n \leq M \). Hence, we have 
        \[ \sum a_n b_n \leq M \sum a_n. \]
        By the \textit{Algebraic Limit Theorem} for series, we have that \( M \sum a_n \) converges. Since \( a_n b_n \leq Ma_n \), we have that the series \( \sum a_n b_n \) also converges by the \textit{Comparison test}.
        \end{proof}
    \item[(c)] If \( \sum a_n \) \textit{converges conditionally}, then \(\sum n^2 a_n \) diverges. \begin{proof}[Solution]
   This is false. Consider the series \( \sum \frac{(-1)^n}{n^2}\) which \textit{converges conditionally} but note that \( \sum n^2 \frac{(-1)^n}{n^2} = \sum (-1)^n\) diverges. 
\end{proof}
\end{enumerate}

\subsubsection{Exercise 2.7.9 (Ratio Test).}  
Given a series \( \sum_{ n=1}^{\infty} a_n \) with \( a_n \neq 0 \), the \textit{Ratio Test} states that if \( (a_n)\) satisfies 
\[ \lim \Big| \frac{a_{n+1}}{a_n}\Big| = r < 1, \]
then the series converges absolutely.

\begin{enumerate}
    \item[(a)] Let \( r' \) satisfy \( r < r' < 1 \). Explain why there exists an \( N \) such that \( n \geq N \) implies \( | a_{n+1}| \leq |a_n|r'\).
        \begin{proof}
            There exists \( N \in \N \) such that \( n \geq N \) because \( \lim  | \frac{a_{n+1}}{a_n}| = r \). This means that \( | \frac{a_{n+1}}{a_n}|\) is \textit{bounded}. Hence, we have that \( | \frac{a_{n+1}}{a_n}| \leq r' \) which means that \( |a_{n+1}| \leq r' |a_n|\).
        \end{proof}
    \item[(b)] Why does \( |a_{N}| \sum (r')^n \) converge?
        \begin{proof}
        The series \( |a_N | \sum (r')^n \) converges because \( | r' | < 1 \) which means that \( |a_N| \sum (r')^n \) is a \textit{geometric series} which converges.  
        \end{proof}
    \item[(c)] Now, show that \( \sum |a_n|\) converges, and conclude that \( \sum a_n \) converges. 
        \begin{proof}
        Consider the series \( \sum |a_n|\) and the fact that
        \[ \sum |a_n| \leq |a_N| \sum (r')^n \]
        for all \( n \geq N \). Since the right hand series is \textit{geometric} which converges, we can conclude that \(\sum | a_n|\) also converges by the comparison test. Hence, the series \( \sum a_n \) converges \textit{absolutely} and thus the series \( \sum a_n \) converges.  



        \end{proof}
\end{enumerate}

\subsubsection{Exercise 2.7.10} Review Exercise 2.4.10 about infinite products and then answer the following questions: 
\begin{enumerate}
    \item[(a)] Does \( \frac{ 2 }{ 1 }  \cdot \frac{ 3 }{ 2 }  \cdot \frac{ 5 }{ 4 } \cdot \frac{ 9 }{ 8 }  \cdot \frac{ 17 }{ 16 }  \dotsb \) converge? 
        \begin{proof}[Solution]
        No, this does not converge. Look at part (a) of Exercise 2.4.10.
        \end{proof}
    \item[(b)] The infinite product \( \frac{ 1 }{ 2 }  \cdot \frac{ 3 }{ 4 }  \cdot \frac{ 5 }{ 6 } \cdot \frac{ 7 }{ 8  }  \cdot \frac{ 9 }{ 10 }  \) certainly converges. Why? Does it converge to zero? 
        \begin{proof}[Solution]
        Suppose we define 
        \[  \gamma_{n} = \prod_{k=1}^{n} \frac{ 2k -1  }{ 2k  }  = \frac{ 1 }{ 2 }  \cdot \frac{ 3 }{ 4 }  \cdot \frac{ 5 }{ 6 } \cdot \frac{ 7 }{ 8 }  \cdot \frac{ 9 }{ 10 } \dotsb \frac{ 2n-1 }{ 2n }. \]
        Observe that \( \gamma_{n}  \) is a decreasing sequence as well as bounded by \( 1/2  \). Using MCT, we know that the sequence of partial products above is a convergent sequence.
        \end{proof}
    \item[(c)] In 1655, John Wallis famously derived the formula 
        \[  \Big( \frac{ 2 \cdot 2  }{  1 \cdot 3  } \Big) \Big( \frac{ 4 \cdot 4  }{ 3 \cdot 5  }  \Big) \Big( \frac{ 6 \cdot 6  }{  5 \cdot 7  }  \Big) \Big( \frac{ 8 \cdot 8  }{ 7 \cdot 9  }  \Big) \dotsb = \frac{ \pi  }{ 2  }.  \]
        Show that the left side of this identity at least converges to something. (A complete proof of this result is taken up in Section 8.3).
        \begin{proof}
        Define the sequence of partial products \( (\gamma_{n}) \) as 
        \[  \gamma_{n} = \prod_{k=1}^{n} \frac{ 4 k^{2}  }{ (2k-1)(2k+1)  } = \Big( \frac{ 2 \cdot 2  }{  1 \cdot 3  } \Big) \Big( \frac{ 4 \cdot 4  }{ 3 \cdot 5  }  \Big) \Big( \frac{ 6 \cdot 6  }{  5 \cdot 7  }  \Big) \Big( \frac{ 8 \cdot 8  }{ 7 \cdot 9  }  \Big) \dotsb \frac{ 4 n^{2}  }{  (2n-1)(2n+1) }.     \]
     Observe that \( \gamma_n   \) is bounded by \( \gamma_{1} = 4 / 3  \). Hence, we have \( 0 \leq \gamma_{n} \leq 4/3 \) We can induct on \( n \in \N  \) to show that \( \gamma_{n}  \) is decreasing. Hence, we must have \( \gamma_{n}  \) convergent by MCT.
        \end{proof}
\end{enumerate}



\subsubsection{Exercise 2.7.12 (Summation-by-parts)} Let \( (x_n)  \) and \( (y_n)  \) be sequences, let \( s_n = \sum_{ k=1  }^{  n  } x_k \) and set \( x_0 = 0  \). Use the observation that \( x_j = s_j - s_{j-1}  \) to verify the formula
\[  \sum_{ j=m  }^{  n  } x_j y_j = s_n y_{n+1 } - s_{m-1}y_m + \sum_{ j=m  }^{  n  } s_j (y_j - y_{j+1} ). \]
\begin{proof}
    Using the observation that \( x_j = s_j - s_{j-1}  \) and the fact that the two series 
    \begin{align*}
        \sum_{ j=m  }^{ n } s_j y_{j+1}, \tag{1}\\ 
        \sum_{ j=m }^{ n } s_{j-1} y_j \tag{2}
    \end{align*} that cancel every term except \( n \)th term (\( s_n y_{n+1} \)) of (1) and the \( m \)th term (\( s_{m-1} y_m \)) of (2) , we can write 
    \begin{align*}
        \sum_{ j=m  }^{ n  } x_j y_j &= \sum_{ j=m }^{ n } (s_j - s_{j-1}) y_j \\
                                     &= \sum_{ j=m }^{ n } (s_j y_j - s_{j-1}y_j) \\
                                     &= \sum_{ j=m }^{ n } (s_j y_j - s_j y_{j+1} + s_j y_{j+1} - s_{j-1} y_{j} ) \\
                                     &= \sum_{ j=m }^{ n } (s_j y_{j+1} - s_{j-1}y_j) + \sum_{ j=m  }^{  n  } s_j (y_j - y_{j+1} ) \\
                                     &= \sum_{ j=m  }^{  n } s_j y_{j+1} - \sum_{ j=m  }^{  n  } s_{j-1} y_j + \sum_{ j=m  }^{ n  } s_j (y_j - y_{j+1} ) \\ \tag{1}
                                     &= s_n y_{n+1} - s_{m-1}y_{m} + \sum_{ j=m  }^{ n  } s_j (y_j - y_{j+1} ). \\
    \end{align*}
\end{proof}

\subsubsection{Exercise 2.7.13 (Abel's Test).} Abel's test for convergence states that if the series \( \sum_{ k=1  }^{  \infty  } x_k  \) converges, and if \( (y_k)  \) is a sequence satisfying \[  y_1 \geq y_2 \geq y_3 \geq \dots \geq  0,  \] 
then the series \( \sum_{ k=1 }^{ \infty  } x_k y_k  \) converges.

\begin{enumerate}
    \item[(a)] Use Exercise 2.7.12 to show that 
        \[  \sum_{ k=1  }^{ n  } x_k y_k = s_n y_{n+1} + \sum_{ k=1  }^{ n } s_k (y_k - y_{k+1} ), \] where \( s_n = \sum_{ k=1  }^{ n } x_k  \).
        \begin{proof}
        Using Exercise 2.7.12, let \( n = 1  \) where \( s_0 = 0  \). Then we have that 
        \begin{align*}
            \sum_{ k=1 }^{ n } x_k y_k &= s_n y_{n+1} - s_0 y_1 + \sum_{ k=1  }^{ n  } s_k (y_k - y_{k+1} ) \\
                                       &= s_n y_{n+1} + \sum_{ k=1  }^{ n  } s_k (y_k - y_{k+1}).\\
        \end{align*}
        \end{proof}
    \item[(b)] Use the Comparison Test to argue that \( \sum_{ k=1  }^{ \infty  } s_k (y_k - y_{k+1}) \) converges absolutely, and show how this leads directly to a proof of Abel's test.
        \begin{proof}
            Since \( (y_n)  \) is a bounded and monotonically decreasing sequence where \(  y_k \geq y_{k+1} \geq 0  \), we know that \( (y_n)  \) must be a convergent sequence by the Monotone Convergence Theorem. Furthermore, for all \(  k \in \N  \), we have \( y_k - y_{k+1} \geq 0  \). Since \( s_n  \) converges, we know that there exists some \( M > 0  \) such that \( | s_n  | \leq M  \). Hence, we also have
            \[  0 \leq | s_k | | y_k - y_{k+1} | \leq M (y_k - y_{k+1} ).\]
            The left right side of the inequality above forms the following telescoping series 
            \[  \sum_{ k=1  }^{ n  } M (y_k - y_{k+1 } ) = M (y_1 - y_{n+1}). \]
            Since \( (y_n)  \) converges, the limit of the above equation must also be convergent. Since the sequence of partial sums of 
            \[  \sum_{ n=1  }^{ \infty  } s_k (y_k - y_{k+1} ) \] converges absolutely, we can now use it in our argument to prove that the series  
            \[  \sum_{ n=1  }^{ \infty  } x_n y_n \tag{1}  \]
            satisfies the Cauchy Criterion. Let \( \epsilon > 0   \). Choose \( N \in \N  \) such that for any \(  n > m \geq N  \), we have that 
            \begin{align*}
                \Big| \sum_{ k= m + 1  }^{ n } x_k y_k  \Big| &= \Big| s_n y_{n+1}  - s_m y_{m+1} + \sum_{ k= m + 1  }^{ n  } s_k (y_k - y_{k+1} ) \Big|  \\
                                                              &\leq | s_n y_{n+1} - s_m y_{m+1} |  + \Big| \sum_{ k = m+1  }^{ n  } s_k (y_k - y_{k+1}) \Big| \\
                                                              &\leq | y_{n+1} | | s_n - s_m |  + | s_m  |  | y_{n+1} - y_{m+1}  |  \\
                                                              &+ \Big| \sum_{ k= m+1 }^{ n } s_k (y_k - y_{k+1}) \Big| \\
                                                              &\leq y_1 | s_n - s_m |  + M  | y_{n+1} - y_{m+1}  |  \\
                                                              &+ \Big| \sum_{ k= m+1 }^{ n } s_k (y_k - y_{k+1}) \Big| \\
                                                              &< \frac{ \epsilon  }{  3  }  + \frac{ \epsilon  }{  3  }  + \frac{ \epsilon  }{  3  } = \epsilon.
            \end{align*}
            Since (1) satisfies the Cauchy Criterion, we must conclude that (1) converges.
        
        \end{proof}
\end{enumerate}





