\section{Properties of Infinite Series}

We have learned the convergence of the series \( \sum_{k=1}^{\infty} a_k \) is defined in terms of the sequence \( (s_n)\) where 
\[ \sum_{k=1}^{\infty} a_k = A \text{ means that } \lim s_n = A.\]
We called \( (s_n)\) the \textit{sequence of partial sums} of the series \( \sum_{k=1}^{\infty} a_k\). Just like the \textit{Algebraic Limit Theorem} for sequences, we can also do the same thing for series. 

\begin{theorem}{Properties of Infinite Series}{}
If \( \sum_{k=1}^{ \infty} a_k = A \) and \( \sum_{k=1}^{ \infty} b_k = B \), then 
\begin{enumerate}
    \item[(i)] \( \sum_{k=1}^{ \infty } ca_k = cA \) for all \( c \in \R \),
    \item[(ii)] \( \sum_{k=1}^{ \infty } (a_k + b_k) = A + B\)
\end{enumerate}
\end{theorem}

\begin{proof}
Suppose \( \sum_{k=1}^{\infty} a_k = A \) and let \( c \in \R \). Define the sequence of partial sums of \( \sum_{k=1}^{ \infty} ca_k \) as 
\[ t_k = cs_n =  ca_1 + ca_2 + ca_3 + ... + ca_n.\]
By the \textit{Algebraic Limit Theorem}, we know that \( \lim cs_n = cA \). Hence, 
\[ \sum_{k=1}^{\infty} ca_k = cA.\]
To prove the addition rule, suppose \( \sum_{k=1}^{ \infty} b_k = B \). We want to show that 
\[ \sum_{k=1}^{\infty} (a_k + b_k) = A + B.\]
Define the sequence of partial sums for the two series as the following:
\begin{align*}
    t_k &= a_1 + a_2 + ... + a_n, \\
    u_k &= b_1 + b_2 + ... + b_n
\end{align*}
Since \( \sum_{k=1}^{\infty} a_k = A \) and \( \sum_{k=1}^{\infty} b_k = B \), their sequence of partial sums also converges to the same value. Hence, let \( \lim t_k = A \) and \( \lim u_k = B \). By the \textit{Algebraic Limit Theorem}, the sum of these two limits also converges i.e  
\[ \lim ( t_k + u_k ) = \lim t_k + \lim u_k = A + B.\] 
Hence, 
\[ \sum_{k=1}^{\infty} (a_k + b_k) = A + B \]
\end{proof}

We can summarize this theorem by keeping in mind that we can perform distribution over infinite addition and that we can add two infinite series together. 

\begin{theorem}{Cauchy Criterion for Series}{}
     The series \( \sum_{k=1}^{ \infty} a_k \) converges if and only if, given \( \epsilon > 0\), there exists \( N \in \N \) such that whenever \( n > m \geq N \) it follows that 
     \[ |a_{m+1} + a_{m+2} + ... + a_n| < \epsilon.\]
    \end{theorem}

\begin{proof}
    Let \(\epsilon > 0 \). We want to show that there exists \( N \in \N \) such that whenever \( n > m \geq N \) it follows that 
    \[|a_{m+1} + a_{m+2} + ... + a_n| < \epsilon.\]
    Suppose \( \sum_{k=1}^{\infty} a_k \) converges. This is true if and only if the sequence of partial sums \( (t_k)\) converges. This is true if and only if \( (s_k)\) is \textit{Cauchy} by the \textit{Cauchy Criterion}. Hence, there exists \( N \in \N \) such that whenever \( n > m \geq N \) 
    \[ |s_n - s_m | < \epsilon.\]
Note that 
\begin{align*}
    |s_n - s_m|&= | \sum_{k=m+1}^{\infty} a_k - \sum_{k=m}^{m} a_k|  \\
               &= |\sum_{k=m+1}^{n}a_k|\\
               &= |a_{m+1} + ... + a_n| < \epsilon
\end{align*}
\end{proof}

This gives us the opportunity to prove some basic facts about series.

\begin{theorem}{}{}
If the series \( \sum_{k=1}^{\infty} a_k \) converges, then \( (a_k) \to 0\).
\end{theorem}

\begin{proof}
From the last theorem, we note that for every \( \epsilon > 0 \) such that whenever \(n  \geq m \geq N \), we have  
\[ | s_n - s_m| = \Big| \sum_{k=m+1}^{ \infty} a_k - 0 \Big| < \epsilon\]
implies that \( (a_n) \to 0 \).
\end{proof}

Keep in mind that the converse of this statement is not true! Just because \( (a_k)\) tends to \( 0 \) does not immediately imply that the series converges! 

\begin{theorem}{Comparison Test}{}
Assume \((a_k)\) and \((b_k)\) are sequences satisfying \( 0 \leq a_k \leq b_k \) for all \( k \in \N \). Then we have 

\begin{enumerate}
    \item[(i)] If \( \sum_{k=1}^{\infty}b_k\) converges, then \( \sum_{k=1}^{\infty} a_k\) converges.
    \item[(ii)] If \( \sum_{k=1}^{\infty} a_k\) diverges, then \( \sum_{k=1}^{\infty} b_k\) diverges.
\end{enumerate}
\end{theorem}

\begin{proof}
Let us show part (i). Suppose \( \sum_{k=1}^{\infty} b_k \) converges. We want to show that \( \sum_{k=1}^{\infty} a_k\) converges. Let \( \epsilon > 0 \). There exists \( N \in \N \) such that for every \( n > m \geq N \) and the fact that \( a_k \leq b_k\) for all \(k \in \N \) 
\begin{align*}
    \Big|\sum_{k=m+1}^{n} a_k \Big| &\leq \Big|\sum_{k=m+1}^{n}b_k\Big| \\ 
                                    &< \epsilon.
\end{align*}
Hence, \(a_k\) converges as well. 

Note that part (ii) is just the contrapositive of part (i) which is also true. 
\end{proof}

Note that the convergence of sequences and series are relatively immutable when it comes to changes in some finite number of initial terms: that is, the behavior of sequences and series can be found past some choice of \( N \in \N \). In order for the above test to be of any use to us, it is important to have a few examples under our belt i.e any \( p > 1 \) implies that 
\begin{center}
    \( \sum_{n=1}^{\infty} 1/ n^p\) converges if and only if \( p > 1\).    
\end{center}

\begin{example}{}{}
A series is called \textit{geometric} if it is of the form 
\[ \sum_{k=0}^{\infty} ar^k = a + ar + ar^2 + ar^3 + ... ~ .\]
If \( r = 1 \) and \( a \neq  0\), the series diverges. We can use the following algebraic identity, for \( r \neq 1 \), to write the following: 
\[ (1-r)(1 + r + r^2 ... + r^{m-1}) = 1 - r^m\]
which allows us to rewrite the partial sum \( (s_m)\) of the above series to say that 
\[ s_m = a+ ar + ar^2 + ar^3 + ... + a r^{m-1} = \frac{a(1-r^m)}{1-r}\]
where \( s_m = at_{m}\) where 
\[ t_m = 1 + r + r^2 + ... + r^{m-1}\]
is a convergent sequence.
Using the \textit{Algebraic Limit Theorem}, therefore, allows us to say that 
\[ \sum_{k=0}^{\infty} ar^k = \frac{a}{1-r}\]
if and only if \( |r| < 1\).
\end{example}

The next theorem is a modification of the \textit{Comparison Test} to handle series that contain negative terms.

\begin{theorem}{Absolute Convergence Test}{}
    If the series \( \sum_{n=1}^{\infty} |a_n| \) converges, then \( \sum_{n=1}^{\infty} a_n\) converges as well.
\end{theorem}

\begin{proof}
    Suppose \( \sum_{n=1}^{\infty}|a_n|\) converges. We want to show that \( \sum_{n=1}^{\infty} a_n\) converges as well. Let \(\epsilon > 0 \). By the \textit{Cauchy Criterion} for series, there exists \( N \in \N \) such that whenever \( n > m \geq N \), we have 
\begin{align*}
    \Big|\sum_{k=m+1}^{n} a_k \Big|&\leq \sum_{k=m+1}^{n} |a_k| \\
                     &< \epsilon.
\end{align*}
Hence, \( \sum_{n=1}^{\infty} a_n \) converges.
\end{proof}

Note that the converse of the above statement is false as taking the absolute value of the alternating harmonic series 
\[ 1 - \frac{1}{2} + \frac{1}{3} - \frac{1}{4} + \frac{1}{5} - \frac{1}{6} + ... ~  \]
produces the regular harmonic series which \textit{diverges}.

\begin{theorem}{Alternating Series Test}{}
    Let \((a_n)\) be a sequence satisfying, 
    \begin{enumerate}
        \item[(i)] \( a_1 \geq a_2 \geq a_3 ... \geq a_n \geq a_{n+1} \geq ... \) and
        \item[(ii)] \((a_n) \to 0\).
    \end{enumerate}
    Then, the alternating series \( \sum_{n=1}^{\infty} (-1)^{n+1} a_n\) converges.
\end{theorem}
% correct this proof
\begin{proof}
See exercise 2.7.1 for proof
\end{proof}

\begin{definition}{Absolute Convergence and Conditional Convergence}{}
If \( \sum_{n=1}^{\infty} |a_n| \) converges, then we say that the original series \( \sum_{n=1}^{\infty} a_n \) \textit{converges absolutely}. If, on the other hand, the series \( \sum_{n=1}^{\infty} a_n \) converges but the series of absolute values \( \sum_{n=1}^{\infty} |a_n|\) does not converges, then we say that the original series \(\sum_{n=1}^{\infty} a_n \) \textit{converges conditionally}.
\end{definition}

We can chart a few examples of some \textit{conditionally convergent } series and \textit{absolutely convergent} series.

\begin{itemize}
    \item \(\sum_{n=1}^{\infty} \frac{(-1)^{n+1}}{n} \implies \) \textit{conditionally convergent}
    \item \(\sum_{n=1}^{\infty} \frac{(-1)^{n+1}}{n^2}, \sum_{n=1}^{\infty} \frac{1}{2^n},  \) and \( \sum_{n=1}^{\infty} \frac{(-1)^{n+1}}{2^n} \implies \) \textit{converges absolutely} 
\end{itemize}

This tells us that any convergent series with positive terms must converge absolutely. 

\subsection{Rearrangements}

We can obtain a rearrangement of an infinite series by permuting terms in the sum in some other order. In order for a sum to be a valid rearrangement, all the terms must appear and there should be no repeats.

\begin{definition}{}{}
    Let \( \sum_{k=1}^{\infty} a_k\) be a series. A series \( \sum_{k=1}^{\infty} b_k\) is called a \textit{rearrangement} of \(\sum_{k=1}^{\infty} a_k \) if there exists a \textit{bijective} function \(f: \N \to \N \) such that \( b_{f(k)} = a_k \) for all \( k \in \N \).
\end{definition}

We can now explain the weird behavior for why the \textit{harmonic series} converges to a different limit when rearranging the terms; that is, it is because the \textit{harmonic series} is a \textit{conditionally convergent} series which leads us to the next theorem. 

\begin{theorem}{Rearrangement of Series}{}
If a series converges absolutely, then any rearrangement of this series converges to the same limit.
\end{theorem}

\begin{proof}
Assume \(\sum_{k=1}^{\infty} a_k \) \textit{converges absolutely} to \(A\), and let \( \sum_{k=1}^{\infty} b_k \) be a rearrangement of \( \sum_{k=1}^{\infty} a_k\). 
Let us define the sequence of partial sums of \( \sum_{k=1}^{\infty} a_k\) as 
\[ s_n = \sum_{k=1}^{n}a_k\] and the sequence of partial sums for the rearranged series \( \sum_{n=1}^{\infty}b_n\) as 
\[ t_m = \sum_{k=1}^{m} b_k.\] Since \( \sum_{n=1}^{\infty}a_n\) \textit{converges absolutely}, let \(\epsilon > 0 \) such that there exists \( N_1 \in \N \) such that whenever \( n \geq N \), we have 
\[ |s_n - A | < \frac{\epsilon}{2}\]
as well some \( N_2 \in \N \) such that whenever \( n > m \geq N_2\), we have 
\[ \sum_{k=m+1}^{n} |a_k| < \frac{\epsilon}{2}.\]
All that is left to do is to set a point in the sequence of the rearranged series where our ultimate goal is to have \( |t_m - A | < \epsilon.\) Hence, define 
\[ M = \max \{ f(k): 1 \leq k \leq N \}.\]
Let \( m \geq M \) such that, when using the \textit{triangle inequality}, we get 
\begin{align*}
    |t_m - A | &= |t_m - s_N + s_N - A |  \\
               &\leq |t_m - s_N | + |s_N - A | \\
               &< \frac{\epsilon}{2} + \frac{\epsilon}{2} \\
               &= \epsilon.
\end{align*}
Hence, we have that \( \sum_{n=1}^{\infty}b_n \) converges to \(A\).
\end{proof}

