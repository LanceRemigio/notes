\section{The Cauchy Criterion}


\begin{definition}[Cauchy Sequence]
A sequence \((a_n)\) is called a \textit{Cauchy Sequence} if, for every \( \epsilon > 0 \), there exists an \( N \in \N \) such that whenever \( m,n \geq N \) it follows that 
\[ | a_n - a_m | < \epsilon.\]
\end{definition}

In the regular convergence definition, we are given any \( \epsilon > 0 \) where there is a point in the sequence \(N \in \N \) such that past this point, all of our terms fall within an \(\epsilon\) range around some limit point. In the Cauchy Criterion defintion, we begin with the same conditions but this time, all the terms of the sequence are all tightly packed together wthin the \(\epsilon > 0 \) range we were given. It turns out, that these two definitions are equivalent: that is, \textit{Cauchy sequences} are convergent sequences and convergent sequences are \textit{Cauchy sequences}. 

\begin{theorem}
Every convergent sequence is a Cauchy sequence. 
\end{theorem}

\begin{proof}
Assume \((x_n)\) converges to \(x\). To show that \((x_n)\) is \textit{Cauchy}, there must existsa point \( N \in \N \) after which we can conclude that 
\[ |x_n - x_m| < \epsilon. \]
Let \( \epsilon > 0 \). Since \( (x_n) \to x \), we can choose \( N \in \N \) such that for any \( n,m \geq N \), we have 
\begin{align*}
    |x_n - x|&< \frac{\epsilon}{2}, \tag{1} \\
    |x_m - x|&< \frac{\epsilon}{2}. \tag{2}
\end{align*}
Consider \( |x_n - x_m| \). Then (1) and (2) imply that 
\begin{align*}
    |x_n - x_m|&= |x_n - x + x - x_m| \\
               &< | x_n - x | + | x - x_n | \tag{\text{Triangle Inequality}}\\  
               &< \frac{\epsilon}{2} + \frac{\epsilon}{2} \\
               &= \epsilon.
\end{align*}
Hence, \((x_n)\) is a \textit{Cauchy Sequence}.
\end{proof}

We can prove the other direction, by using either the \textit{Bolzano Weierstrass Theorem} or the \textit{Monotone Convergence Theorem}. This is a little bit more difficult since we need to have a proposed limit for the sequence to converge to.  
\begin{lemma}{}{}
Cauchy sequences are bounded.
\end{lemma}

\begin{proof}
    Given \( \epsilon = 1\), there exists an \( N \in \N \) such that \( |x_m - x_n | < 1 \) for amm \( m,n \geq N \). Thus, we must have \( |x_n| < |x_{N}| + 1 \) for all \( n \geq N \) (just substituted \(m = N \) here). Hence, define 
    \[ M = \max \{ |x_1|, |x_1|, |x_1|,..., |x_{N-1}|, |x_{N}| + 1 \}.\]
   Therefore, \( |x_n| < M \) for all \( n \in \N \) Hence, the \textit{Cauchy sequence} \((x_n)\) is \textit{bounded}.
\end{proof}

\begin{theorem}[Cauchy Criterion in \( \R  \)]
A sequence converges if and only if it is a Cauchy sequence.
\end{theorem}

\begin{proof}
    (\(\Rightarrow\)) This direction is just Theorem 2.6.2 which we have proved above. 

    (\( \Leftarrow\)) Suppose \((x_n)\) is a \textit{Cauchy sequence}. Let \( \epsilon > 0 \). Since \( (x_n)\) is a \textit{bounded} sequence, there exists a subsequence \((x_{n_k})\) such that \((x_{n_k}) \to x \) by the \textit{Bolzano Weierstrass Theorem}. Let \( \epsilon > 0 \). Then for some \( N \in \N \), every \( n_k \geq N \) has the property
    \[ |x_{n_k} - x| < \epsilon.\]
Our goal now is to show that \( (x_n) \to x\). Hence, consider \( |x_n - x| \). Then for every \(n, n_{k} \geq N \), we have 
\begin{align*}
    |x_n - x | &= |x_n - x_{n_k} + x_{n_k} - x | \\
     &< |x_n - x_{n_k}| + |x_{n_k} - x| \\
     &< \frac{\epsilon}{2} + \frac{\epsilon}{2} \\
     &= \epsilon.
\end{align*}
Hence, \((x_n) \to x \).  
\end{proof}

\subsection{Completeness Revisited}

We can summarize all of our results thus far in the following way 
\[ \text{AOC} 
\begin{cases}
    \text{NIP} \implies \text{BW} \implies \text{CC} \\ 
    \text{MCT} 
\end{cases}   \]
where AOC is our defining axiom to base all our reults on and giving us the notion that an ordered field contains no holes. We could also take the MCT to be our defining axiom and gives us the notion of least upper bounds by proving NIP. In addition, we could also take NIP to be our starting point but we need to have an extra hypothesis; that is, the Archimedean Property to prove all our results above (This is unavoidable).

It could be possible to assume the Arcimedean property holds, suppose one of the results we have proven is true, and derive the others yet this is sort of limited since \( \Q \) contains a set that is not complete. 

Below is the least of implications we can prove based on which theorem we would like to select asour defining axiom. Hence, we have
\[ \text{NIP} + \text{Archimedean Property} \implies \text{AOC} \] and 
\[ \text{BW} \implies \text{MCT} \implies \text{Archimedean Property}\]

