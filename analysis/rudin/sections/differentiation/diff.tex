\documentclass[11pt,a4paper]{book}
\usepackage[utf8]{inputenc}
\usepackage[T1]{fontenc}
\usepackage{textcomp}
\usepackage{hyperref}
% \usepackage{fourier}
% \usepackage[dutch]{babel}
\usepackage{url}
% \usepackage{hyperref}
% \hypersetup{
%     colorlinks,
%     linkcolor={black},
%     citecolor={black},
%     urlcolor={blue!80!black}
% }
\usepackage{graphicx}
\usepackage{float}
\usepackage{booktabs}
\usepackage{enumitem}
% \usepackage{parskip}
\usepackage{emptypage}
\usepackage{subcaption}
\usepackage{multicol}
\usepackage[usenames,dvipsnames]{xcolor}

% \usepackage{cmbright}


\usepackage[margin=1in]{geometry}
\usepackage{amsmath, amsfonts, mathtools, amsthm, amssymb}
\usepackage{mathrsfs}
\usepackage{cancel}
\usepackage{bm}
\newcommand\N{\ensuremath{\mathbb{N}}}
\newcommand\R{\ensuremath{\mathbb{R}}}
\newcommand\Z{\ensuremath{\mathbb{Z}}}
\renewcommand\O{\ensuremath{\emptyset}}
\newcommand\Q{\ensuremath{\mathbb{Q}}}
\newcommand\C{\ensuremath{\mathbb{C}}}
\DeclareMathOperator{\sgn}{sgn}
\usepackage{systeme}
\let\svlim\lim\def\lim{\svlim\limits}
\let\implies\Rightarrow
\let\impliedby\Leftarrow
\let\iff\Leftrightarrow
\let\epsilon\varepsilon
\usepackage{stmaryrd} % for \lightning
\newcommand\contra{\scalebox{1.1}{$\lightning$}}
% \let\phi\varphi
\renewcommand\qedsymbol{$\blacksquare$}




% correct
\definecolor{correct}{HTML}{009900}
\newcommand\correct[2]{\ensuremath{\:}{\color{red}{#1}}\ensuremath{\to }{\color{correct}{#2}}\ensuremath{\:}}
\newcommand\green[1]{{\color{correct}{#1}}}



% horizontal rule
\newcommand\hr{
    \noindent\rule[0.5ex]{\linewidth}{0.5pt}
}


% hide parts
\newcommand\hide[1]{}



% si unitx
\usepackage{siunitx}
\sisetup{locale = FR}
% \renewcommand\vec[1]{\mathbf{#1}}
\newcommand\mat[1]{\mathbf{#1}}


% tikz
\usepackage{tikz}
\usepackage{tikz-cd}
\usetikzlibrary{intersections, angles, quotes, calc, positioning}
\usetikzlibrary{arrows.meta}
\usepackage{pgfplots}
\pgfplotsset{compat=1.13}


\tikzset{
    force/.style={thick, {Circle[length=2pt]}-stealth, shorten <=-1pt}
}

% theorems
\makeatother
\usepackage{thmtools}
\usepackage[framemethod=TikZ]{mdframed}
\mdfsetup{skipabove=1em,skipbelow=0em}


\theoremstyle{definition}

\declaretheoremstyle[
    headfont=\bfseries\sffamily\color{ForestGreen!70!black}, bodyfont=\normalfont,
    mdframed={
        linewidth=2pt,
        rightline=false, topline=false, bottomline=false,
        linecolor=ForestGreen, backgroundcolor=ForestGreen!5,
    }
]{thmgreenbox}

\declaretheoremstyle[
    headfont=\bfseries\sffamily\color{NavyBlue!70!black}, bodyfont=\normalfont,
    mdframed={
        linewidth=2pt,
        rightline=false, topline=false, bottomline=false,
        linecolor=NavyBlue, backgroundcolor=NavyBlue!5,
    }
]{thmbluebox}

\declaretheoremstyle[
    headfont=\bfseries\sffamily\color{NavyBlue!70!black}, bodyfont=\normalfont,
    mdframed={
        linewidth=2pt,
        rightline=false, topline=false, bottomline=false,
        linecolor=NavyBlue
    }
]{thmblueline}

\declaretheoremstyle[
    headfont=\bfseries\sffamily\color{RawSienna!70!black}, bodyfont=\normalfont,
    mdframed={
        linewidth=2pt,
        rightline=false, topline=false, bottomline=false,
        linecolor=RawSienna, backgroundcolor=RawSienna!5,
    }
]{thmredbox}

\declaretheoremstyle[
    headfont=\bfseries\sffamily\color{RawSienna!70!black}, bodyfont=\normalfont,
    numbered=no,
    mdframed={
        linewidth=2pt,
        rightline=false, topline=false, bottomline=false,
        linecolor=RawSienna, backgroundcolor=RawSienna!1,
    },
    qed=\qedsymbol
]{thmproofbox}

\declaretheoremstyle[
    headfont=\bfseries\sffamily\color{NavyBlue!70!black}, bodyfont=\normalfont,
    numbered=no,
    mdframed={
        linewidth=2pt,
        rightline=false, topline=false, bottomline=false,
        linecolor=NavyBlue, backgroundcolor=NavyBlue!1,
    },
]{thmexplanationbox}

\declaretheorem[style=thmgreenbox, numberwithin = section, name=Definition]{definition}
\declaretheorem[style=thmbluebox, name=Example]{eg}
\declaretheorem[style=thmredbox, numberwithin = section, name=Proposition]{prop}
\declaretheorem[style=thmredbox, numberwithin = section, name=Theorem]{theorem}
\declaretheorem[style=thmredbox, numberwithin = section,  name=Lemma]{lemma}
\declaretheorem[style=thmredbox, numberwithin = section,  numbered=no, name=Corollary]{corollary}


\declaretheorem[style=thmproofbox, name=Proof]{replacementproof}
\renewenvironment{proof}[1][\proofname]{\vspace{-10pt}\begin{replacementproof}}{\end{replacementproof}}


\declaretheorem[style=thmexplanationbox, name=Proof]{tmpexplanation}
\newenvironment{explanation}[1][]{\vspace{-10pt}\begin{tmpexplanation}}{\end{tmpexplanation}}


\declaretheorem[style=thmblueline, numbered=no, name=Remark]{remark}
\declaretheorem[style=thmblueline, numbered=no, name=Note]{note}

\newtheorem*{uovt}{UOVT}
\newtheorem*{notation}{Notation}
\newtheorem*{previouslyseen}{As previously seen}
\newtheorem*{problem}{Problem}
\newtheorem*{observe}{Observe}
\newtheorem*{property}{Property}
\newtheorem*{intuition}{Intuition}


\usepackage{etoolbox}
\AtEndEnvironment{vb}{\null\hfill$\diamond$}%
\AtEndEnvironment{intermezzo}{\null\hfill$\diamond$}%
% \AtEndEnvironment{opmerking}{\null\hfill$\diamond$}%

% http://tex.stackexchange.com/questions/22119/how-can-i-change-the-spacing-before-theorems-with-amsthm
\makeatletter
% \def\thm@space@setup{%
%   \thm@preskip=\parskip \thm@postskip=0pt
% }
\newcommand{\oefening}[1]{%
    \def\@oefening{#1}%
    \subsection*{Oefening #1}
}

\newcommand{\suboefening}[1]{%
    \subsubsection*{Oefening \@oefening.#1}
}

\newcommand{\exercise}[1]{%
    \def\@exercise{#1}%
    \subsection*{Exercise #1}
}

\newcommand{\subexercise}[1]{%
    \subsubsection*{Exercise \@exercise.#1}
}


\usepackage{xifthen}

\def\testdateparts#1{\dateparts#1\relax}
\def\dateparts#1 #2 #3 #4 #5\relax{
    \marginpar{\small\textsf{\mbox{#1 #2 #3 #5}}}
}

\def\@lesson{}%
\newcommand{\lesson}[3]{
    \ifthenelse{\isempty{#3}}{%
        \def\@lesson{Lecture #1}%
    }{%
        \def\@lesson{Lecture #1: #3}%
    }%
    \subsection*{\@lesson}
    \testdateparts{#2}
}

% \renewcommand\date[1]{\marginpar{#1}}


% fancy headers
\usepackage{fancyhdr}
\pagestyle{fancy}

\fancyhead[LE,RO]{Lance Remigio}
\fancyhead[RO,LE]{\@lesson}
\fancyhead[RE,LO]{}
\fancyfoot[LE,RO]{\thepage}
\fancyfoot[C]{\leftmark}

\makeatother




% notes
\usepackage{todonotes}
\usepackage{tcolorbox}

\tcbuselibrary{breakable}
\newenvironment{verbetering}{\begin{tcolorbox}[
    arc=0mm,
    colback=white,
    colframe=green!60!black,
    title=Opmerking,
    fonttitle=\sffamily,
    breakable
]}{\end{tcolorbox}}

\newenvironment{noot}[1]{\begin{tcolorbox}[
    arc=0mm,
    colback=white,
    colframe=white!60!black,
    title=#1,
    fonttitle=\sffamily,
    breakable
]}{\end{tcolorbox}}




% figure support
\usepackage{import}
\usepackage{xifthen}
\pdfminorversion=7
\usepackage{pdfpages}
\usepackage{transparent}
\newcommand{\incfig}[1]{%
    \def\svgwidth{\columnwidth}
    \import{./figures/}{#1.pdf_tex}
}

% %http://tex.stackexchange.com/questions/76273/multiple-pdfs-with-page-group-included-in-a-single-page-warning
\pdfsuppresswarningpagegroup=1



\usepackage{standalone}
\usepackage{import}

\begin{document}

\section{The Derivative of a Real Function}

\begin{definition}[5.1]
    Let \( f  \) be defined (and real-valued) on \( [a,b] \). For any \( x \in [a,b] \), form the quotient 
    \[  \phi(t) = \frac{ f(t) - f(x)  }{  t - x  } \ \ (a < t < b, \  t \neq x ),   \]
    and define
    \[  f'(x) = \lim_{ t \to x }  \phi(t). \]
\end{definition}

\begin{theorem}[5.2]
    Let \( f  \) be defined on \( [a,b] \). If \( f  \) differentiable at a point \( x \in [a,b] \), then \( f  \) is continuous at \( x  \).   
\end{theorem}

\begin{theorem}[5.3]
    Suppose \( f  \) and \( g  \) are defined on \( [a,b] \) and are defined on \( [a,b] \) and are differentiable at a point \( x \in [a,b] \). Then \( f + g, fg, \) and \( f/g \) are differentiable at \( x  \), and
    \begin{enumerate}
        \item[(a)] \( (f+g)'(x) = f'(x) + g'(x)  \);
        \item[(b)] \( (fg)'(x) = f'(x) g(x) + f(x) g'(x) \);
        \item[(c)] \( \Big(  \frac{ f  }{ g }  \Big)'(x) = \frac{ g(x) f'(x) - f'(x) f(x)  }{  g^{2}(x) }  \) with \( g(x) \neq 0  \).
    \end{enumerate}
\end{theorem}

\begin{theorem}[5.5]
    Suppose \( f  \) is continuous on \( [a,b] \), \( f'(x) \) exists at some point \( x \in [a,b] \), \( g  \) is defined on an interval \( I  \) which contains the range of \( f  \), and \( g  \) is differentiable at the point \( f(x) \). If 
    \[  h(t) = g(f(t)) \ \ (a \leq t \leq b), \]
    then \( h  \) is differentiable at \( x  \), and
    \[  h'(x) = g'(f(x)) f'(x). \]
\end{theorem}

\section{Mean Value Theorems}

\begin{definition}[5.7]
    Let \( f  \) be a real function defined on a metric space \( X  \). We say that \( f  \) has a \textit{local maximum} at a point \( p \in X  \) if there exists \( \delta > 0  \) such that \( f(q) \leq f(p) \) for all \( q \in X  \) with \( d(p,q) < \delta \). Local minima are defined likewise. 
\end{definition}

\begin{theorem}[5.8]
    Let \( f  \) be defined on \( [a,b] \); if \( f \) has a local maximum at a point \( x \in (a,b) \) and if \( f'(x) \) exists, then \( f'(x) = 0 \).
\end{theorem}

\begin{theorem}[5.9]
    If \(  f  \) and \( g \) are continuous real functions on \( [a,b] \) which are differentiable on \( (a,b) \), then there is a point \( x \in (a,b) \) at which 
    \[  [f(b) - f(a)]g'(x) = [g(b) - g(a)]f'(x). \]
    Note that differentiability is not required at the end points.

\end{theorem}

\begin{theorem}[5.10]
    If \( f  \) is a real continuous function on \( [a,b] \) which is differentiable on \( (a,b) \), then there is a point \( x \in (a,b) \) at which 
    \[  f(b) - f(a) = (b-a) f'(x). \]
\end{theorem}

\begin{theorem}[5.11]
    Suppose \( f  \) is differentiable on \( (a,b) \). 
    \begin{enumerate}
        \item[(a)] If \( f'(x) \geq 0  \) for all \( x \in (a,b) \), then \( f  \) is monotonically increasing.
        \item[(b)] If \( f'(x) = 0  \) for all \( x \in (a,b) \), then \( f  \) is constant.
        \item[(c)] If \( f'(x) \leq 0  \) for all \( x \in (a,b) \), then \( f  \) is monotonically decreasing.
    \end{enumerate}
\end{theorem}

\section{The Continuity of Derivatives}

\begin{theorem}[5.12]
    Suppose \( f  \) is a real differentiable function on \( [a,b] \) and suppose \( f'(a) < \lambda < f'(b) \). Then there is a point \( x \in (a,b) \) such that \( f'(x) = \lambda  \). 
\end{theorem}

\begin{corollary}[5.12]
    If \( f  \) is differentiable on \( [a,b] \), then \( f' \) cannot have any simple discontinuities on \( [a,b] \). 
\end{corollary}

\section{L'Hopital's Rule}

\begin{theorem}[5.13]
    Suppose \( f  \) and \( g  \) are real and differentiable on \( (a,b) \), and \( g'(x) \neq 0  \) for all \( x \in (a,b) \), where \( - \infty  \leq a < b \leq + \infty  \). Suppose
    \[ \frac{ f'(x) }{ g'(x) }  \to A \ \text{as} \ x \to a. \]
    If
    \begin{enumerate}
        \item[(a)] \( f'(x) \to 0  \) and \( g(x) \to 0  \) as \( x \to a  \), or if
        \item[(b)] \( g(x) \to + \infty  \) as \( x \to a  \),
    \end{enumerate}
    then 
    \[  \frac{ f(x) }{ g(x) }  \to A \ \text{as} \ x \to a. \]
\end{theorem}

This result also holds true if \( x \to b  \), or if \( g(x) \to - \infty  \) in part (b).

\end{document}
