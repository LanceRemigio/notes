\documentclass[11pt,a4paper]{book}
\usepackage[utf8]{inputenc}
\usepackage[T1]{fontenc}
\usepackage{textcomp}
\usepackage{hyperref}
% \usepackage{fourier}
% \usepackage[dutch]{babel}
\usepackage{url}
% \usepackage{hyperref}
% \hypersetup{
%     colorlinks,
%     linkcolor={black},
%     citecolor={black},
%     urlcolor={blue!80!black}
% }
\usepackage{graphicx}
\usepackage{float}
\usepackage{booktabs}
\usepackage{enumitem}
% \usepackage{parskip}
\usepackage{emptypage}
\usepackage{subcaption}
\usepackage{multicol}
\usepackage[usenames,dvipsnames]{xcolor}

% \usepackage{cmbright}


\usepackage[margin=1in]{geometry}
\usepackage{amsmath, amsfonts, mathtools, amsthm, amssymb}
\usepackage{mathrsfs}
\usepackage{cancel}
\usepackage{bm}
\newcommand\N{\ensuremath{\mathbb{N}}}
\newcommand\R{\ensuremath{\mathbb{R}}}
\newcommand\Z{\ensuremath{\mathbb{Z}}}
\renewcommand\O{\ensuremath{\emptyset}}
\newcommand\Q{\ensuremath{\mathbb{Q}}}
\newcommand\C{\ensuremath{\mathbb{C}}}
\DeclareMathOperator{\sgn}{sgn}
\usepackage{systeme}
\let\svlim\lim\def\lim{\svlim\limits}
\let\implies\Rightarrow
\let\impliedby\Leftarrow
\let\iff\Leftrightarrow
\let\epsilon\varepsilon
\usepackage{stmaryrd} % for \lightning
\newcommand\contra{\scalebox{1.1}{$\lightning$}}
% \let\phi\varphi
\renewcommand\qedsymbol{$\blacksquare$}




% correct
\definecolor{correct}{HTML}{009900}
\newcommand\correct[2]{\ensuremath{\:}{\color{red}{#1}}\ensuremath{\to }{\color{correct}{#2}}\ensuremath{\:}}
\newcommand\green[1]{{\color{correct}{#1}}}



% horizontal rule
\newcommand\hr{
    \noindent\rule[0.5ex]{\linewidth}{0.5pt}
}


% hide parts
\newcommand\hide[1]{}



% si unitx
\usepackage{siunitx}
\sisetup{locale = FR}
% \renewcommand\vec[1]{\mathbf{#1}}
\newcommand\mat[1]{\mathbf{#1}}


% tikz
\usepackage{tikz}
\usepackage{tikz-cd}
\usetikzlibrary{intersections, angles, quotes, calc, positioning}
\usetikzlibrary{arrows.meta}
\usepackage{pgfplots}
\pgfplotsset{compat=1.13}


\tikzset{
    force/.style={thick, {Circle[length=2pt]}-stealth, shorten <=-1pt}
}

% theorems
\makeatother
\usepackage{thmtools}
\usepackage[framemethod=TikZ]{mdframed}
\mdfsetup{skipabove=1em,skipbelow=0em}


\theoremstyle{definition}

\declaretheoremstyle[
    headfont=\bfseries\sffamily\color{ForestGreen!70!black}, bodyfont=\normalfont,
    mdframed={
        linewidth=2pt,
        rightline=false, topline=false, bottomline=false,
        linecolor=ForestGreen, backgroundcolor=ForestGreen!5,
    }
]{thmgreenbox}

\declaretheoremstyle[
    headfont=\bfseries\sffamily\color{NavyBlue!70!black}, bodyfont=\normalfont,
    mdframed={
        linewidth=2pt,
        rightline=false, topline=false, bottomline=false,
        linecolor=NavyBlue, backgroundcolor=NavyBlue!5,
    }
]{thmbluebox}

\declaretheoremstyle[
    headfont=\bfseries\sffamily\color{NavyBlue!70!black}, bodyfont=\normalfont,
    mdframed={
        linewidth=2pt,
        rightline=false, topline=false, bottomline=false,
        linecolor=NavyBlue
    }
]{thmblueline}

\declaretheoremstyle[
    headfont=\bfseries\sffamily\color{RawSienna!70!black}, bodyfont=\normalfont,
    mdframed={
        linewidth=2pt,
        rightline=false, topline=false, bottomline=false,
        linecolor=RawSienna, backgroundcolor=RawSienna!5,
    }
]{thmredbox}

\declaretheoremstyle[
    headfont=\bfseries\sffamily\color{RawSienna!70!black}, bodyfont=\normalfont,
    numbered=no,
    mdframed={
        linewidth=2pt,
        rightline=false, topline=false, bottomline=false,
        linecolor=RawSienna, backgroundcolor=RawSienna!1,
    },
    qed=\qedsymbol
]{thmproofbox}

\declaretheoremstyle[
    headfont=\bfseries\sffamily\color{NavyBlue!70!black}, bodyfont=\normalfont,
    numbered=no,
    mdframed={
        linewidth=2pt,
        rightline=false, topline=false, bottomline=false,
        linecolor=NavyBlue, backgroundcolor=NavyBlue!1,
    },
]{thmexplanationbox}

\declaretheorem[style=thmgreenbox, numberwithin = section, name=Definition]{definition}
\declaretheorem[style=thmbluebox, name=Example]{eg}
\declaretheorem[style=thmredbox, numberwithin = section, name=Proposition]{prop}
\declaretheorem[style=thmredbox, numberwithin = section, name=Theorem]{theorem}
\declaretheorem[style=thmredbox, numberwithin = section,  name=Lemma]{lemma}
\declaretheorem[style=thmredbox, numberwithin = section,  numbered=no, name=Corollary]{corollary}


\declaretheorem[style=thmproofbox, name=Proof]{replacementproof}
\renewenvironment{proof}[1][\proofname]{\vspace{-10pt}\begin{replacementproof}}{\end{replacementproof}}


\declaretheorem[style=thmexplanationbox, name=Proof]{tmpexplanation}
\newenvironment{explanation}[1][]{\vspace{-10pt}\begin{tmpexplanation}}{\end{tmpexplanation}}


\declaretheorem[style=thmblueline, numbered=no, name=Remark]{remark}
\declaretheorem[style=thmblueline, numbered=no, name=Note]{note}

\newtheorem*{uovt}{UOVT}
\newtheorem*{notation}{Notation}
\newtheorem*{previouslyseen}{As previously seen}
\newtheorem*{problem}{Problem}
\newtheorem*{observe}{Observe}
\newtheorem*{property}{Property}
\newtheorem*{intuition}{Intuition}


\usepackage{etoolbox}
\AtEndEnvironment{vb}{\null\hfill$\diamond$}%
\AtEndEnvironment{intermezzo}{\null\hfill$\diamond$}%
% \AtEndEnvironment{opmerking}{\null\hfill$\diamond$}%

% http://tex.stackexchange.com/questions/22119/how-can-i-change-the-spacing-before-theorems-with-amsthm
\makeatletter
% \def\thm@space@setup{%
%   \thm@preskip=\parskip \thm@postskip=0pt
% }
\newcommand{\oefening}[1]{%
    \def\@oefening{#1}%
    \subsection*{Oefening #1}
}

\newcommand{\suboefening}[1]{%
    \subsubsection*{Oefening \@oefening.#1}
}

\newcommand{\exercise}[1]{%
    \def\@exercise{#1}%
    \subsection*{Exercise #1}
}

\newcommand{\subexercise}[1]{%
    \subsubsection*{Exercise \@exercise.#1}
}


\usepackage{xifthen}

\def\testdateparts#1{\dateparts#1\relax}
\def\dateparts#1 #2 #3 #4 #5\relax{
    \marginpar{\small\textsf{\mbox{#1 #2 #3 #5}}}
}

\def\@lesson{}%
\newcommand{\lesson}[3]{
    \ifthenelse{\isempty{#3}}{%
        \def\@lesson{Lecture #1}%
    }{%
        \def\@lesson{Lecture #1: #3}%
    }%
    \subsection*{\@lesson}
    \testdateparts{#2}
}

% \renewcommand\date[1]{\marginpar{#1}}


% fancy headers
\usepackage{fancyhdr}
\pagestyle{fancy}

\fancyhead[LE,RO]{Lance Remigio}
\fancyhead[RO,LE]{\@lesson}
\fancyhead[RE,LO]{}
\fancyfoot[LE,RO]{\thepage}
\fancyfoot[C]{\leftmark}

\makeatother




% notes
\usepackage{todonotes}
\usepackage{tcolorbox}

\tcbuselibrary{breakable}
\newenvironment{verbetering}{\begin{tcolorbox}[
    arc=0mm,
    colback=white,
    colframe=green!60!black,
    title=Opmerking,
    fonttitle=\sffamily,
    breakable
]}{\end{tcolorbox}}

\newenvironment{noot}[1]{\begin{tcolorbox}[
    arc=0mm,
    colback=white,
    colframe=white!60!black,
    title=#1,
    fonttitle=\sffamily,
    breakable
]}{\end{tcolorbox}}




% figure support
\usepackage{import}
\usepackage{xifthen}
\pdfminorversion=7
\usepackage{pdfpages}
\usepackage{transparent}
\newcommand{\incfig}[1]{%
    \def\svgwidth{\columnwidth}
    \import{./figures/}{#1.pdf_tex}
}

% %http://tex.stackexchange.com/questions/76273/multiple-pdfs-with-page-group-included-in-a-single-page-warning
\pdfsuppresswarningpagegroup=1



\usepackage{standalone}
\usepackage{import}

\begin{document}

\section{Convergent Sequences}

\begin{definition}[3.1]
   A sequence \( ({P}_{n}) \) in a metric space \( X  \) is said to \textbf{converge} if there is a point \( p \in X  \) with the following property: For every \( \epsilon > 0  \) there is an integer \( N  \) such that \( n \geq N  \) implies that \( d({p}_{n}, p) < \epsilon  \) where \( d  \) denotes the distance in \( X  \). 
\end{definition}

\begin{itemize}
    \item Another way we say the statement above is to say that \( ({p}_{n}) \to p  \), or that \( p  \) is the limit of \( ({p}_{n}) \), or that
\[  \lim_{ n \to \infty  } {p}_{n} = p. \]
    \item If \( ({p}_{n}) \) does not converge, then it is said to \textbf{diverge}.
    \item We can be more specific about the convergence of a sequence by specifying the convergence of the sequence with respect to the metric space.
\end{itemize}

\begin{theorem}[3.2]
   Let \( ({p}_{n}) \) be a sequence in a metric space \( X  \). 
   \begin{enumerate}
       \item[(a)] \( ({p}_{n}) \) converges to \( p \in X  \) if and only if every neighborhood of \( p  \) contains \( {p}_{n} \) for all but finitely many \( n \).
       \item[(b)] If \( p \in X  \), \( p' \in X  \), and if \( ({p}_{n}) \) converges to \( p  \) and to \( p' \), then \( p' = p \).  
        \item[(c)] If \( ({p}_{n}) \) converges, then \( ({p}_{n}) \) is bounded.
        \item[(d)] If \( E \subset X  \) and if \( p  \) is a limit point of \( E  \), then there is a sequence \( ({p}_{n}) \) in \( E  \) such that \( p = \lim_{ n \to \infty  } {p}_{n}  \).
   \end{enumerate}
\end{theorem}

\begin{theorem}[3.3]
   Suppose \( ({s}_{n}), ({t}_{n}) \) are complex sequences and \( \lim_{ n \to \infty  }  {s}_{n} = s  \), \( \lim_{ n \to \infty  }  {t}_{n} = t  \). Then  
   \begin{enumerate}
       \item[(a)] \( \lim_{ n \to \infty  }  ({s}_{n} + {t}_{n}) = s + t  \);
        \item[(b)] \( \lim_{ n \to \infty  }  c {s}_{n} = cs  \), \( \lim_{ n \to \infty  }  (c + {s}_{n}) = c + s  \) for any number \( c  \);
        \item[(c)] \( \lim_{ n \to \infty  }  {s}_{n} {t}_{n} = s t    \);
        \item[(d)] \( \lim_{ n \to \infty  }  (1 / s_{n}) = 1/s \), provided \( {s}_{n} \neq 0 (n = 1,2,3,\dots)  \) and \( s \neq 0  \).
   \end{enumerate}
\end{theorem}

\begin{theorem}[3.4]
    \begin{enumerate}
        \item[(a)] Suppose \( {x}_{n} \in \R^{k} \  (n = 1,2,3 \dots ) \) and 
            \[  {x}_{n} = ({\alpha}_{1,n}, \dots , {\alpha}_{k,n}). \] 
            Then \( ({x}_{n}) \to x  \) with \( x =  ({\alpha}_{1}, {\alpha}_{2}, \dots , {\alpha}_{k}) \) if and only if
            \[  \lim_{ n \to \infty  }  {\alpha}_{j, n } = {\alpha}_{j} \ \ (1 \leq j \leq k ). \]
        \item[(b)] Suppose \( ({x}_{n}), ({y}_{n}) \) are sequences in \( \R^{k } \), \( ({\beta}_{n}) \) is a sequence of real numbers, and \( {x}_{n} \to x  \), \( {y}_{n} \to y  \), and \( {\beta}_{n} \to \beta  \). Then
            \[  \lim_{ n \to \infty  }  ({x}_{n} + {y}_{n}) = x + y, \ \ \lim_{ n \to \infty  }  {x}_{n} \cdot {y}_{n} = x \cdot y , \ \ \lim_{ n \to \infty  }  {\beta}_{n} {x}_{n} = \beta x. \]
    \end{enumerate}
\end{theorem}

\section{Subsequences}

\begin{definition}[3.5]
   Given a sequence \( ({p}_{n}) \), consider a sequences \( ({n}_{k}) \) of positive integers such that \( {n}_{1} < {n}_{2} < \dots  \). Then the sequence \( ({p}_{{n}_{i}}) \) is called a \textbf{subsequence} of \( ({p}_{n}) \). If \( ({p}_{{n}_{i}}) \) converges, its limit is called the \textbf{subsequential limit} of \( ({p}_{n}) \). 
\end{definition}

Recall that if \( ({p}_{n}) \) converges to \( p  \) iff every subsequence of \( ({p}_{n}) \) converges to \( p  \).

\begin{theorem}[3.6]
    \begin{enumerate}
        \item[(a)] If \( ({p}_{n}) \) is a sequence in a compact metric space \( X  \), then some subsequence of \( ({p}_{n}) \) converges to a point of \( x  \).
        \item[(b)] Every bounded sequence in \( \R^{k} \) contains a convergent subsequence.
    \end{enumerate}
\end{theorem}

\begin{theorem}[3.7]
    The subsequential limits of a sequence \( ({p}_{n}) \) in a metric space form a closed subset of \( X  \).
\end{theorem}


\section{Cauchy Sequences}

\begin{definition}[3.8]
    A sequence \( ({p}_{n}) \) in a metric space \( X  \) is said to be a \textbf{Cauchy sequence} if for every \( \epsilon > 0  \), there is an integer \( N  \) such that \( d({p}_{n}, {p}_{m}) < \epsilon  \) if \( n \geq N  \) and \( m \geq N  \).
\end{definition}

\begin{definition}[3.9]
    Let \( E  \) be a nonempty subset of a metric space \( X  \) and let \( S  \) be the set of all real numbers of the form \( d(p,q) \) with \( p \in E  \) and \( q \in E  \). The supremum of \( S  \) is called the \textit{diameter} of \( E  \).
\end{definition}

\begin{theorem}[3.10]
   \begin{enumerate}
       \item[(a)] If \( \overline{E} \) is the closure of a set \( E  \) in a metric space \( X  \), then
           \[  \text{diam} \overline{E} = \diam E.  \]
        \item[(b)] If \( {K}_{n} \) is a sequence of compact sets in \( X  \) such that \( {K}_{n} \supset {K}_{n+1} \ (n = 1,2,3,\dots) \) and if  
            \[  \lim_{ n \to \infty  }  \diam {K}_{n} = 0, \]
            then \( \cap_{n=1}^{\infty }{K}_{n}  \) consists of exactly one point.
   \end{enumerate} 
\end{theorem}

\begin{theorem}[3.11]\label{Theorem 3.11}
    \begin{enumerate}
        \item[(a)] In any metric space \( X  \), every convergent sequence is a Cauchy sequence.
        \item[(b)] If \( X  \) is a compact metric space and if \( ({p}_{n}) \) is a Cauchy sequence in \( X  \), then \( ({p}_{n}) \) converges to some point of \( X  \).
        \item[(c)] In \( \R^{k } \), every Cauchy sequence converges.
    \end{enumerate}
\end{theorem}

\begin{definition}[3.12]
   A metric space in which every Cauchy sequence converges is said to be \textit{complete}. \end{definition}

\begin{itemize}
    \item In {\hyperref[Theorem 3.11]{Theorem 3.11}}, we see that \textit{all compact metric spaces and all euclidean spaces are complete}. 
    \item Referring to the same theorem again, we also add that \textit{every closed subset \( E  \) of a complete metric space is complete}.
    \item Convergent sequences in \( \R^{k} \) are bounded, but the converse need not be true.
\end{itemize}

\begin{definition}[3.13]
    A sequence \( ({s}_{n}) \) of real numbers is said to be 
    \begin{enumerate}
        \item[(a)] \textit{monotonically increasing} if \( {s}_{n} \leq {s}_{n+1} \ (n \in \N) \);
        \item[(b)] \textit{monotonically decreasing} if \( {s}_{n} \geq {s}_{n+1} \ (n \in \N) \).
    \end{enumerate}
\end{definition}

The following is the monotone convergence theorem.

\begin{theorem}[3.14]
    Suppose \( ({s}_{n}) \) is monotonic. Then \( ({s}_{n}) \) converges if and only if it is bounded.
\end{theorem}

\section{Upper and Lower Limits}

\begin{definition}[3.15]
    Let \( ({S}_{n}) \) be a sequence of real numbers with the following property: For every real \( M  \) there is an integer \( N \) such that \( n \geq N  \) implies \( {s}_{n} \geq M  \). We then write
    \[  {s}_{n} \to + \infty.  \]
    Similarly, if for every real \( M  \) there exists an integer \( N  \) such that \( n \geq N  \) implies \( {s}_{n} \leq M   \), we write
    \[  {s}_{n} \to -\infty.  \]
\end{definition}

\begin{definition}[3.16]\label{3.16}
    Let \( ({s}_{n}) \) be a sequence of real numbers. Let \( E  \) be the set of numbers \( x  \) (in the extended real number system) such that \( {s}_{{n}_{k}} \to x  \) for some subsequence \( ({s}_{{n}_{k}}) \). This set \( E  \) contains all subsequential limits as in {\hyperref[3.5]{Definition 3.5}} and possibly the numbers \( + \infty, - \infty   \). Set
    \begin{align*}
        s^{*} &= \sup E, \\
        {s}_{*} &= \inf E.
    \end{align*}
We denote the numbers \( s^{*}, {s}_{*} \) as the \textit{upper} and \textit{lower} limits of \( ({s}_{n}) \). We use the notation 
\[  \lim_{ n \to \infty  } \sup {s}_{n} = s^{*}, \ \ \lim_{ n \to \infty  } \inf {s}_{n} = {s}_{*}. \]
\end{definition}

\begin{theorem}[3.17]
    Let \( ({s}_{n}) \) be a sequence of real numbers. Let \( E  \) and \( s^{*} \) have the same meaning as in {\hyperref[3.16]{Definition 3.16}}. Then \( s^{*}  \) has the following two properties:
    \begin{enumerate}
        \item[(a)] \( s^{*} \in E  \).
        \item[(b)] If \( x > s^{*} \), there is an integer \( N  \) such that \( n \geq N  \) implies \( {s}_{n} < x  \).
    \end{enumerate}
\end{theorem}

\begin{theorem}[3.19]\label{3.19}
   If \( {s}_{n} \leq {t}_{n} \) for \( n \geq N  \), where \( N  \) is fixed, then 
   \begin{align*}
       \lim_{ n \to \infty  } \inf {s}_{n} &\leq \lim_{ n \to \infty  }  \inf {t}_{n}, \\
       \lim_{ n \to \infty  }  \sup {s}_{n} &\leq \lim_{ n \to \infty  } \sup {t}_{n}.
   \end{align*}
\end{theorem}

\section{Some Special Sequences}

\begin{theorem}[3.2]\label{3.2}
   \begin{enumerate}
       \item[(a)] If \( p > 0  \), then \( \lim_{ n \to \infty  }  1 / n^{p} = 0  \).
       \item[(b)] If \( p > 0  \), then \( \lim_{ n \to \infty  } \sqrt[n]{p} = 1   \).
       \item[(c)] \( \lim_{ n \to \infty  }  \sqrt[n]{ n } = 1. \)
        \item[(d)] If \( p > 0  \) and \( \alpha  \) is real, then \( \lim_{ n \to \infty  }  n^{\alpha} / (1+p)^{n} = 0 \).
        \item[(e)] If \( | x  |  < 1  \), then \( \lim_{ n \to \infty  }  x^{n} = 0  \).
   \end{enumerate} 
\end{theorem}

\end{document}
