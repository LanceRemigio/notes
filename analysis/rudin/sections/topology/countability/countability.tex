\documentclass[11pt,a4paper]{book}
\usepackage[utf8]{inputenc}
\usepackage[T1]{fontenc}
\usepackage{textcomp}
\usepackage{hyperref}
% \usepackage{fourier}
% \usepackage[dutch]{babel}
\usepackage{url}
% \usepackage{hyperref}
% \hypersetup{
%     colorlinks,
%     linkcolor={black},
%     citecolor={black},
%     urlcolor={blue!80!black}
% }
\usepackage{graphicx}
\usepackage{float}
\usepackage{booktabs}
\usepackage{enumitem}
% \usepackage{parskip}
\usepackage{emptypage}
\usepackage{subcaption}
\usepackage{multicol}
\usepackage[usenames,dvipsnames]{xcolor}

% \usepackage{cmbright}


\usepackage[margin=1in]{geometry}
\usepackage{amsmath, amsfonts, mathtools, amsthm, amssymb}
\usepackage{mathrsfs}
\usepackage{cancel}
\usepackage{bm}
\newcommand\N{\ensuremath{\mathbb{N}}}
\newcommand\R{\ensuremath{\mathbb{R}}}
\newcommand\Z{\ensuremath{\mathbb{Z}}}
\renewcommand\O{\ensuremath{\emptyset}}
\newcommand\Q{\ensuremath{\mathbb{Q}}}
\newcommand\C{\ensuremath{\mathbb{C}}}
\DeclareMathOperator{\sgn}{sgn}
\usepackage{systeme}
\let\svlim\lim\def\lim{\svlim\limits}
\let\implies\Rightarrow
\let\impliedby\Leftarrow
\let\iff\Leftrightarrow
\let\epsilon\varepsilon
\usepackage{stmaryrd} % for \lightning
\newcommand\contra{\scalebox{1.1}{$\lightning$}}
% \let\phi\varphi
\renewcommand\qedsymbol{$\blacksquare$}




% correct
\definecolor{correct}{HTML}{009900}
\newcommand\correct[2]{\ensuremath{\:}{\color{red}{#1}}\ensuremath{\to }{\color{correct}{#2}}\ensuremath{\:}}
\newcommand\green[1]{{\color{correct}{#1}}}



% horizontal rule
\newcommand\hr{
    \noindent\rule[0.5ex]{\linewidth}{0.5pt}
}


% hide parts
\newcommand\hide[1]{}



% si unitx
\usepackage{siunitx}
\sisetup{locale = FR}
% \renewcommand\vec[1]{\mathbf{#1}}
\newcommand\mat[1]{\mathbf{#1}}


% tikz
\usepackage{tikz}
\usepackage{tikz-cd}
\usetikzlibrary{intersections, angles, quotes, calc, positioning}
\usetikzlibrary{arrows.meta}
\usepackage{pgfplots}
\pgfplotsset{compat=1.13}


\tikzset{
    force/.style={thick, {Circle[length=2pt]}-stealth, shorten <=-1pt}
}

% theorems
\makeatother
\usepackage{thmtools}
\usepackage[framemethod=TikZ]{mdframed}
\mdfsetup{skipabove=1em,skipbelow=0em}


\theoremstyle{definition}

\declaretheoremstyle[
    headfont=\bfseries\sffamily\color{ForestGreen!70!black}, bodyfont=\normalfont,
    mdframed={
        linewidth=2pt,
        rightline=false, topline=false, bottomline=false,
        linecolor=ForestGreen, backgroundcolor=ForestGreen!5,
    }
]{thmgreenbox}

\declaretheoremstyle[
    headfont=\bfseries\sffamily\color{NavyBlue!70!black}, bodyfont=\normalfont,
    mdframed={
        linewidth=2pt,
        rightline=false, topline=false, bottomline=false,
        linecolor=NavyBlue, backgroundcolor=NavyBlue!5,
    }
]{thmbluebox}

\declaretheoremstyle[
    headfont=\bfseries\sffamily\color{NavyBlue!70!black}, bodyfont=\normalfont,
    mdframed={
        linewidth=2pt,
        rightline=false, topline=false, bottomline=false,
        linecolor=NavyBlue
    }
]{thmblueline}

\declaretheoremstyle[
    headfont=\bfseries\sffamily\color{RawSienna!70!black}, bodyfont=\normalfont,
    mdframed={
        linewidth=2pt,
        rightline=false, topline=false, bottomline=false,
        linecolor=RawSienna, backgroundcolor=RawSienna!5,
    }
]{thmredbox}

\declaretheoremstyle[
    headfont=\bfseries\sffamily\color{RawSienna!70!black}, bodyfont=\normalfont,
    numbered=no,
    mdframed={
        linewidth=2pt,
        rightline=false, topline=false, bottomline=false,
        linecolor=RawSienna, backgroundcolor=RawSienna!1,
    },
    qed=\qedsymbol
]{thmproofbox}

\declaretheoremstyle[
    headfont=\bfseries\sffamily\color{NavyBlue!70!black}, bodyfont=\normalfont,
    numbered=no,
    mdframed={
        linewidth=2pt,
        rightline=false, topline=false, bottomline=false,
        linecolor=NavyBlue, backgroundcolor=NavyBlue!1,
    },
]{thmexplanationbox}

\declaretheorem[style=thmgreenbox, numberwithin = section, name=Definition]{definition}
\declaretheorem[style=thmbluebox, name=Example]{eg}
\declaretheorem[style=thmredbox, numberwithin = section, name=Proposition]{prop}
\declaretheorem[style=thmredbox, numberwithin = section, name=Theorem]{theorem}
\declaretheorem[style=thmredbox, numberwithin = section,  name=Lemma]{lemma}
\declaretheorem[style=thmredbox, numberwithin = section,  numbered=no, name=Corollary]{corollary}


\declaretheorem[style=thmproofbox, name=Proof]{replacementproof}
\renewenvironment{proof}[1][\proofname]{\vspace{-10pt}\begin{replacementproof}}{\end{replacementproof}}


\declaretheorem[style=thmexplanationbox, name=Proof]{tmpexplanation}
\newenvironment{explanation}[1][]{\vspace{-10pt}\begin{tmpexplanation}}{\end{tmpexplanation}}


\declaretheorem[style=thmblueline, numbered=no, name=Remark]{remark}
\declaretheorem[style=thmblueline, numbered=no, name=Note]{note}

\newtheorem*{uovt}{UOVT}
\newtheorem*{notation}{Notation}
\newtheorem*{previouslyseen}{As previously seen}
\newtheorem*{problem}{Problem}
\newtheorem*{observe}{Observe}
\newtheorem*{property}{Property}
\newtheorem*{intuition}{Intuition}


\usepackage{etoolbox}
\AtEndEnvironment{vb}{\null\hfill$\diamond$}%
\AtEndEnvironment{intermezzo}{\null\hfill$\diamond$}%
% \AtEndEnvironment{opmerking}{\null\hfill$\diamond$}%

% http://tex.stackexchange.com/questions/22119/how-can-i-change-the-spacing-before-theorems-with-amsthm
\makeatletter
% \def\thm@space@setup{%
%   \thm@preskip=\parskip \thm@postskip=0pt
% }
\newcommand{\oefening}[1]{%
    \def\@oefening{#1}%
    \subsection*{Oefening #1}
}

\newcommand{\suboefening}[1]{%
    \subsubsection*{Oefening \@oefening.#1}
}

\newcommand{\exercise}[1]{%
    \def\@exercise{#1}%
    \subsection*{Exercise #1}
}

\newcommand{\subexercise}[1]{%
    \subsubsection*{Exercise \@exercise.#1}
}


\usepackage{xifthen}

\def\testdateparts#1{\dateparts#1\relax}
\def\dateparts#1 #2 #3 #4 #5\relax{
    \marginpar{\small\textsf{\mbox{#1 #2 #3 #5}}}
}

\def\@lesson{}%
\newcommand{\lesson}[3]{
    \ifthenelse{\isempty{#3}}{%
        \def\@lesson{Lecture #1}%
    }{%
        \def\@lesson{Lecture #1: #3}%
    }%
    \subsection*{\@lesson}
    \testdateparts{#2}
}

% \renewcommand\date[1]{\marginpar{#1}}


% fancy headers
\usepackage{fancyhdr}
\pagestyle{fancy}

\fancyhead[LE,RO]{Lance Remigio}
\fancyhead[RO,LE]{\@lesson}
\fancyhead[RE,LO]{}
\fancyfoot[LE,RO]{\thepage}
\fancyfoot[C]{\leftmark}

\makeatother




% notes
\usepackage{todonotes}
\usepackage{tcolorbox}

\tcbuselibrary{breakable}
\newenvironment{verbetering}{\begin{tcolorbox}[
    arc=0mm,
    colback=white,
    colframe=green!60!black,
    title=Opmerking,
    fonttitle=\sffamily,
    breakable
]}{\end{tcolorbox}}

\newenvironment{noot}[1]{\begin{tcolorbox}[
    arc=0mm,
    colback=white,
    colframe=white!60!black,
    title=#1,
    fonttitle=\sffamily,
    breakable
]}{\end{tcolorbox}}




% figure support
\usepackage{import}
\usepackage{xifthen}
\pdfminorversion=7
\usepackage{pdfpages}
\usepackage{transparent}
\newcommand{\incfig}[1]{%
    \def\svgwidth{\columnwidth}
    \import{./figures/}{#1.pdf_tex}
}

% %http://tex.stackexchange.com/questions/76273/multiple-pdfs-with-page-group-included-in-a-single-page-warning
\pdfsuppresswarningpagegroup=1



\usepackage{standalone}
\usepackage{import}

\begin{document}

\subsection{Function Concepts}

\begin{definition}[Functions, Domains, Values, and Range]
    Consider two sets \( A  \) and \( B  \), whose elements may be any objects whatsoever, and suppose that with each element of \( x  \) of \( A  \) there is a associated, in some manner, an element of \( B  \), which we denote by \( f(x) \).
   \begin{itemize}
       \item The function \( f  \) is said to be a \textit{function} from \( A  \) into \( B  \);
        \item The set \( A  \) is called the \textit{domain} of \( f  \); 
        \item The elements \( f(x) \) are called the \textit{values} of \( f  \);
        \item The set of values of \( f  \) is called the \textit{range} of \( f  \)
   \end{itemize} 
\end{definition}

\begin{definition}[Images, Onto]
   Let \( A  \) and \( B  \) be two sets and let \( f  \) be a mapping of \( A  \) into \( B  \). If \( E \subset A  \), then \( f(E)  \) is called the set of all elements \( f(x) \), for \( x \in E \). 
   \begin{itemize}
       \item We call \( f(E)  \), the \textit{image} of \( E  \) under \( f  \).  
        \item We call \( f(A) \) the \textit{range} of \( f  \). Clearly, we have \( f(A) \subset B  \).
        \item If \( f(A) = B  \), we say that \( f  \) maps \textit{onto} B.
   \end{itemize}
\end{definition}

\begin{definition}[Inverse Images, One-to-one]
   \begin{itemize}
       \item If \( E \subset B  \), \( f^{-1}(E) \) denotes the set of all \( x \in A  \) such that \( f(x) \in E  \). We call \( f^{-1}(E) \) the \textit{inverse image} of \( E  \) under \( f  \).
       \item If \( y \in B  \), \( f^{-1}(y) \) is the set of all \( x \in A  \) such that \( f(x) = y \). If, for each \( y \in B  \), \( f^{-1}(y) \) consists of at most one element of \( A  \), then \( f  \) is said to be a 1-1 (\textit{one-to-one} ) mapping of \( A  \) into \( B  \).
        \item Another way to state this is to say that for any \( {x}_{1}, {x}_{2} \in A  \), if \( f({x}_{1}) = f({x}_{2}) \) implies \( {x}_{1} = {x}_{2} \) is called a 1-1 function. 
        \item Alternatively, if \( {x}_{1} \neq {x}_{2}  \) implies \( f({x}_{1}) \neq f({x}_{2})\) is also called a 1-1 function (this is just contrapositive of the last statement).
   \end{itemize} 
\end{definition}

\begin{definition}[Correspondence]
    If there exists a 1-1 mapping of \( A  \) \textit{onto} B, we say that \( A  \) and \( B  \) can be put in 1-1 \textit{correspondence}, or that \( A  \) and \( B  \) have the same \textit{cardinality}, or, that \( A  \) and \( B  \) are \textit{equivalent}. For this, we write \( A \sim B   \).
\end{definition}

This relation contains the following properties:

\begin{itemize}
    \item \textbf{Reflexive}: \( A \sim A  \).
    \item \textbf{Symmetric}: If \( A \sim B  \), then \( B \sim A  \).
    \item \textbf{Transitive:} If \( A \sim B  \) and \( B \sim C  \), then \( A \sim C  \).
\end{itemize}

\begin{definition}[Finite, Infinite, Countable, Uncountable, At most countable]
    For any positive integer \( n  \), let \( {\N}_{n}  \) be the set whose elelemts are the integers \( 1,2, \dots , n  \); let \( \N  \) be the set consisting of all positive integers. For any set \( A \), we say:
    \begin{enumerate}
        \item[(a)] \( A  \) is \textit{finite} if \( A \sim {\N}_{n} \) for some \( n  \) (the empty set is also considered to be finite).
        \item[(b)] \( A  \) is \textit{infinite} if \( A  \) is not finite. 
        \item[(c)] \( A  \) is \textit{countable}  if \( A \sim \N  \).
        \item[(d)] \( A  \) is \textit{uncountable} if \( A  \) is neither finite n or countable.
        \item[(e)] \( A  \) is \textit{at most countable} if \( A  \) is finite or countable.
    \end{enumerate}
\end{definition}

\begin{prop}\label{Infinite subsets of countable sets}
   Every infinite subset of a countable set \( A  \) is countable. 
\end{prop}
\begin{proof}
Suppose \( E \subset A   \), and \( E  \) is infinite. Arrange the elements of \( x  \) of \( A  \) as a sequence \( ({x}_{n}) \) of distinct elements. Construct a sequence \( ({n}_{k }) \) as follows:
Let \( {n}_{1} \) be the smallest positive integer such that \( {x}_{{n}_{1}} \in E  \). Having chosen \( {n}_{1}, \dots {n}_{k-1} \) with \( (k = 2,3,4, \dots) \), let \( {n}_{k } \) be the smallest integer greater than \( {n}_{k-1} \) such that \( {x}_{{n}_{k }} \in E  \), let \( {n}_{k } \) be the smallest integer greater than \( {n}_{k-1} \) such that \( {x}_{{n}_{k }} \in E  \). 
    Putting \( f(k) = {x}_{{n}_{k }}  \) with \( (k = 1,2,3,\dots) \), we obtain a 1-1 correspondence between \( E  \) and \( \N  \). Thus, by definition, we see that \( E  \) is an infinite subset of \( A  \) that is countable.
\end{proof}

\begin{definition}
    Let \( A  \) and \( \Omega \) be sets, and suppose that with each element \( \alpha  \) of \( A  \) there is associated a subset of \( \Omega  \) which we denote by \( {E}_{\alpha} \).
\end{definition}

\begin{itemize}
    \item We can have sets whose elements are also sets. 
    \item To make this easier to understand, we usually denote these kinds of sets as a collection of sets.
\end{itemize}

\begin{definition}[Union]
    The \textit{union} of the sets \( {E}_{\alpha} \) is defined to be the set \( S  \) such that \( x \in S  \) if \( x \in {E}_{\alpha} \) for at least one \( \alpha \in A  \). We use the notation   
    \[  S = \bigcup_{ \alpha \in A  }^{  }  {E}_{\alpha}. \]
\end{definition}

If our collection of sets \( A  \) is finite, then we can use the notation
\[  S = \bigcup_{ m=  1  }^{ n  }  {E}_{m} \]
for \( {E}_{1}, {E}_{2}, \dots, {E}_{n} \in A  \)

On the other hand, when the collection of sets \( A  \) contains a countable number of elements, then we can use the notation
\[  S = \bigcup_{ m = 1  }^{ \infty  }  {E}_{m}. \]
The \( \infty  \) on the top of the union symbol should not be confused with \( + \infty   \) and \( - \infty  \).

\begin{definition}[Intersection]
    The \textit{intersection} of the sets \( {E}_{\alpha} \) is defined to be the set \( P  \) such that \( x \in P  \) if \( x \in {E}_{\alpha} \) for every \( \alpha \in A  \), we have
    \[  P = \bigcap_{ \alpha \in A  }^{  }  {E}_{\alpha}. \]
\end{definition}

Like the union, \( A  \) can either have a finite collection of sets or a countable collection of sets. Thus, we have

\[  P = \bigcap_{ m=1  }^{ n } {E}_{m} \  \text{and} \ P = \bigcap_{  m = 1  }^{ \infty  } {E}_{m},  \]
respectively.

\begin{definition}[Nonempty Intersections and Disjoint Sets]
    If we have \( A \cap B \neq \emptyset  \), then we say that \( A  \) and \( B  \) \textit{intersect}. Otherwise, we say that they are \textit{disjoint}. 
\end{definition}

Here are some list of algebraic properties of sets:

\begin{itemize}
    \item \textbf{Commutativity}: \( A \cup B = B \cup A   \) and \( A \cap B = B \cap A  \).
    \item \textbf{Associativity}: \( (A \cup B) \cup C = A \cup (B \cup C)  \) and \(  (A \cap B ) \cap C = A \cap (B \cap C) \).
    \item \textbf{Distributivity:} \( A \cap (B \cup C) = (A \cap B) \cup (A \cap C) \). 
    \item \( A \subset A \cup B  \).
    \item \( A \cap B \subset A  \).
    \item \( A \cup \emptyset \) and \( A \cap \emptyset = \emptyset \).
    \item If \( A \subset B  \), then
        \[  A \cup B = B, \ \  A \cap B  = A   \]
\end{itemize}

\begin{theorem}[ ]
    Let \( \{ {E}_{n} \}   \) with \( n \in \N \) be a countable collection of countable sets, and put
    \[  S = \bigcup_{ n = 1  }^{ \infty  }  {E}_{n}, \]
    Then \( S  \) is countable.
\end{theorem}
\begin{proof}
Let every set \( {E}_{n} \) be arranged in a sequence \( ({x}_{{n}_{k }}) \) with \( k = 1,2,3, \dots, \). We can consider an infinite array such that, in each row, we have all the elements of each \( {E}_{n} \). If we take the diagonal entries, starting from left to right, we can rearrange these entries into a sequence  
\[  {x}_{11}; {x}_{21}, {x}_{12}; {x}_{31}, {x}_{22}, {x}_{13}; {x}_{41}, {x}_{32}, {x}_{23}, {x}_{14}; \dots \ ,  \]
starting from \( n = 2  \). Notice how the sum of each index in the sequence adds up to the index of the sequence above. Thus, there exists a subset of \( T  \) of the set of all positive integers such that \( S \sim T  \), which shows that \( S  \) is at most countable, using our result about {\hyperref[Infinite subsets of countable sets]{infinite subsets of countable sets}}. Since each \( {E}_{1} \subset S  \), and \( {E}_{1}  \) is infinite, \( S  \) is infinite, and thus \( S  \) is countable.  
\end{proof}

\begin{corollary}
   Suppose \( A  \) is at most countable, and, for every \( \alpha  \in A  \), \( {B}_{\alpha} \) is at most countable. Then
   \[  T = \bigcup_{ \alpha \in A  }^{  }  {B}_{\alpha} \]
   is at most countable, for \( T  \) is equivalent to a subset of
   \[  \bigcup_{ n=1  }^{ \infty   } {B}_{n}, \]
   where \( {B}_{n} \in A  \).
\end{corollary}

\begin{theorem}[ ]
    Let \( A  \) be a countable set, and let \( {B}_{n} \) be the set of all \( n- \)tuples \( ({a}_{1}, \dots , {a}_{n}) \), where \( {a}_{k } \in A (k = 1, \dots, n) \), and the elements \( {a}_{1}, \dots, {a}_{n} \) need not be distinct. Then \( {B}_{m}  \) is countable.
\end{theorem}

\end{document}
