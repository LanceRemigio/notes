\documentclass[11pt,a4paper]{book}
\usepackage[utf8]{inputenc}
\usepackage[T1]{fontenc}
% \usepackage{fourier}
\usepackage{textcomp}
\usepackage{hyperref}
\usepackage[english]{babel}
\usepackage{url}
% \usepackage{hyperref}
% \hypersetup{
%     colorlinks,
%     linkcolor={black},
%     citecolor={black},
%     urlcolor={blue!80!black}
% }
\usepackage{graphicx} \usepackage{float}
\usepackage{booktabs}
\usepackage{enumitem}
% \usepackage{parskip}
% \usepackage{parskip}
\usepackage{emptypage}
\usepackage{subcaption}
\usepackage{multicol}
\usepackage[usenames,dvipsnames]{xcolor}
\usepackage{ocgx}
% \usepackage{cmbright}


\usepackage[margin=1in]{geometry}
\usepackage{amsmath, amsfonts, mathtools, amsthm, amssymb}
\usepackage{thmtools}
\usepackage{mathrsfs}
\usepackage{cancel}
\usepackage{bm}
\newcommand\N{\ensuremath{\mathbb{N}}}
\newcommand\R{\ensuremath{\mathbb{R}}}
\newcommand\Z{\ensuremath{\mathbb{Z}}}
\renewcommand\O{\ensuremath{\emptyset}}
\newcommand\Q{\ensuremath{\mathbb{Q}}}
\newcommand\C{\ensuremath{\mathbb{C}}}
\newcommand\F{\ensuremath{\mathbb{F}}}
\DeclareMathOperator{\sgn}{sgn}
\DeclareMathOperator{\diam}{diam}
\DeclareMathOperator{\LO}{LO}
\DeclareMathOperator{\UP}{UP}
\DeclareMathOperator{\card}{card}
\DeclareMathOperator{\Arg}{Arg}
\DeclareMathOperator{\Dom}{Dom}
\DeclareMathOperator{\Log}{Log}
\DeclareMathOperator{\dist}{dist}
% \DeclareMathOperator{\span}{span}
\usepackage{systeme}
\let\svlim\lim\def\lim{\svlim\limits}
\renewcommand\implies\Longrightarrow
\let\impliedby\Longleftarrow
\let\iff\Longleftrightarrow
\let\epsilon\varepsilon
\usepackage{stmaryrd} % for \lightning
\newcommand\contra{\scalebox{1.1}{$\lightning$}}
% \let\phi\varphi
\renewcommand\qedsymbol{$\blacksquare$}

% correct
\definecolor{correct}{HTML}{009900}
\newcommand\correct[2]{\ensuremath{\:}{\color{red}{#1}}\ensuremath{\to }{\color{correct}{#2}}\ensuremath{\:}}
\newcommand\green[1]{{\color{correct}{#1}}}

% horizontal rule
\newcommand\hr{
    \noindent\rule[0.5ex]{\linewidth}{0.5pt}
}

% hide parts
\newcommand\hide[1]{}

% si unitx
\usepackage{siunitx}
\sisetup{locale = FR}
% \renewcommand\vec[1]{\mathbf{#1}}
\newcommand\mat[1]{\mathbf{#1}}

% tikz
\usepackage{tikz}
\usepackage{tikz-cd}
\usetikzlibrary{intersections, angles, quotes, calc, positioning}
\usetikzlibrary{arrows.meta}
\usepackage{pgfplots}
\pgfplotsset{compat=1.13}

\tikzset{
    force/.style={thick, {Circle[length=2pt]}-stealth, shorten <=-1pt}
}

% theorems
\makeatother
\usepackage{thmtools}
\usepackage[framemethod=TikZ]{mdframed}
\mdfsetup{skipabove=1em,skipbelow=1em}

\theoremstyle{definition}

\declaretheoremstyle[
    headfont=\bfseries\sffamily\color{ForestGreen!70!black}, bodyfont=\normalfont,
    mdframed={
        linewidth=1pt,
        rightline=false, topline=false, bottomline=false,
        linecolor=ForestGreen, backgroundcolor=ForestGreen!5,
    }
]{thmgreenbox}

\declaretheoremstyle[
    headfont=\bfseries\sffamily\color{NavyBlue!70!black}, bodyfont=\normalfont,
    mdframed={
        linewidth=1pt,
        rightline=false, topline=false, bottomline=false,
        linecolor=NavyBlue, backgroundcolor=NavyBlue!5,
    }
]{thmbluebox}

\declaretheoremstyle[
    headfont=\bfseries\sffamily\color{NavyBlue!70!black}, bodyfont=\normalfont,
    mdframed={
        linewidth=1pt,
        rightline=false, topline=false, bottomline=false,
        linecolor=NavyBlue
    }
]{thmblueline}

\declaretheoremstyle[
    headfont=\bfseries\sffamily, bodyfont=\normalfont,
    numbered = no,
    mdframed={
        rightline=true, topline=true, bottomline=true,
    }
]{thmbox}

\declaretheoremstyle[
    headfont=\bfseries\sffamily, bodyfont=\normalfont,
    numbered=no,
    % mdframed={
    %     rightline=true, topline=false, bottomline=true,
    % },
    qed=\qedsymbol
]{thmproofbox}

\declaretheoremstyle[
    headfont=\bfseries\sffamily\color{NavyBlue!70!black}, bodyfont=\normalfont,
    numbered=no,
    mdframed={
        rightline=false, topline=false, bottomline=false,
        linecolor=NavyBlue, backgroundcolor=NavyBlue!1,
    },
]{thmexplanationbox}

\declaretheorem[
    style=thmbox, 
    % numberwithin = section,
    numbered = no,
    name=Definition
    ]{definition}

\declaretheorem[
    style=thmbox, 
    name=Example,
    ]{eg}

\declaretheorem[
    style=thmbox, 
    % numberwithin = section,
    name=Proposition]{prop}

\declaretheorem[
    style = thmbox,
    numbered=yes,
    name =Problem
    ]{problem}

\declaretheorem[style=thmbox, name=Theorem]{theorem}
\declaretheorem[style=thmbox, name=Lemma]{lemma}
\declaretheorem[style=thmbox, name=Corollary]{corollary}

\declaretheorem[style=thmproofbox, name=Proof]{replacementproof}

\declaretheorem[style=thmproofbox, 
                name = Solution
                ]{replacementsolution}

\renewenvironment{proof}[1][\proofname]{\vspace{-1pt}\begin{replacementproof}}{\end{replacementproof}}

\newenvironment{solution}
    {
        \vspace{-1pt}\begin{replacementsolution}
    }
    { 
            \end{replacementsolution}
    }

\declaretheorem[style=thmexplanationbox, name=Proof]{tmpexplanation}
\newenvironment{explanation}[1][]{\vspace{-10pt}\begin{tmpexplanation}}{\end{tmpexplanation}}

\declaretheorem[style=thmbox, numbered=no, name=Remark]{remark}
\declaretheorem[style=thmbox, numbered=no, name=Note]{note}

\newtheorem*{uovt}{UOVT}
\newtheorem*{notation}{Notation}
\newtheorem*{previouslyseen}{As previously seen}
% \newtheorem*{problem}{Problem}
\newtheorem*{observe}{Observe}
\newtheorem*{property}{Property}
\newtheorem*{intuition}{Intuition}

\usepackage{etoolbox}
\AtEndEnvironment{vb}{\null\hfill$\diamond$}%
\AtEndEnvironment{intermezzo}{\null\hfill$\diamond$}%
% \AtEndEnvironment{opmerking}{\null\hfill$\diamond$}%

% http://tex.stackexchange.com/questions/22119/how-can-i-change-the-spacing-before-theorems-with-amsthm
\makeatletter
% \def\thm@space@setup{%
%   \thm@preskip=\parskip \thm@postskip=0pt
% }
\newcommand{\oefening}[1]{%
    \def\@oefening{#1}%
    \subsection*{Oefening #1}
}

\newcommand{\suboefening}[1]{%
    \subsubsection*{Oefening \@oefening.#1}
}

\newcommand{\exercise}[1]{%
    \def\@exercise{#1}%
    \subsection*{Exercise #1}
}

\newcommand{\subexercise}[1]{%
    \subsubsection*{Exercise \@exercise.#1}
}


\usepackage{xifthen}

\def\testdateparts#1{\dateparts#1\relax}
\def\dateparts#1 #2 #3 #4 #5\relax{
    \marginpar{\small\textsf{\mbox{#1 #2 #3 #5}}}
}

\def\@lesson{}%
\newcommand{\lesson}[3]{
    \ifthenelse{\isempty{#3}}{%
        \def\@lesson{Lecture #1}%
    }{%
        \def\@lesson{Lecture #1: #3}%
    }%
    \subsection*{\@lesson}
    \testdateparts{#2}
}

% \renewcommand\date[1]{\marginpar{#1}}


% fancy headers
\usepackage{fancyhdr}
\pagestyle{fancy}

\makeatother

% notes
\usepackage{todonotes}
\usepackage{tcolorbox}

\tcbuselibrary{breakable}
\newenvironment{verbetering}{\begin{tcolorbox}[
    arc=0mm,
    colback=white,
    colframe=green!60!black,
    title=Opmerking,
    fonttitle=\sffamily,
    breakable
]}{\end{tcolorbox}}

\newenvironment{noot}[1]{\begin{tcolorbox}[
    arc=0mm,
    colback=white,
    colframe=white!60!black,
    title=#1,
    fonttitle=\sffamily,
    breakable
]}{\end{tcolorbox}}

% figure support
\usepackage{import}
\usepackage{xifthen}
\pdfminorversion=7
\usepackage{pdfpages}
\usepackage{transparent}
\newcommand{\incfig}[1]{%
    \def\svgwidth{\columnwidth}
    \import{./figures/}{#1.pdf_tex}
}

% %http://tex.stackexchange.com/questions/76273/multiple-pdfs-with-page-group-included-in-a-single-page-warning
\pdfsuppresswarningpagegroup=1


\usepackage{standalone}
\usepackage{import}

\begin{document}

\begin{definition}[Metric Space]
   A set \( X  \), whose elements we shall call \textit{points}, is said to be a \textit{metric space} if with any two points \( p  \) and \( q  \) of \( X  \) there is associated a real number \( d(p,q) \) called the \textit{distance} from \( p  \) to \( q  \), such that 
   \begin{enumerate}
       \item[(a)] \( d(p,q) > 0  \) if \( p \neq q  \); \( d(p,p) = 0  \);
        \item[(b)] \( d(p,q) = d(q,p) \);
        \item[(c)] \( d(p,q) \leq d(p,r) + d(r,q) \) for any \( r \in X  \).
   \end{enumerate}
\end{definition}

\begin{itemize}
    \item Suppose a subset \( Y  \) of \( X  \) is a metric space, with the same distance function.
    \item \( Y  \) must also be a metric space with the same distance function as \( X  \) (metric).
\end{itemize}

\begin{definition}[ ]
    \begin{itemize}
        \item By the \textit{segment} \( (a,b) \) we mean the set of all \( x \in \R  \) such that \( a < x < b \).
        \item We call an \textit{interval} \( [a,b] \) to mean the set of all \( x \in \R  \) such that \( a \leq x \leq b  \) for \( a,b \in \R  \) with \( a < b  \).
        \item We call a \textit{half-open interval} either \( [a,b) \) or \( (a,b] \) to mean \( a \leq x <  b  \) and \( a < x \leq b  \), respectively.
        \item We call a \textit{k-cell} to mean that if \( {a}_{i} < {b}_{i} \), for \( i = 1, \dots, k  \), the set of all points \( x = ({x}_{1}, {x}_{2}, \dots, {x}_{k }) \in \R^{k } \) whose coordinates satisfy \( {a}_{i} \leq {x}_{i} \leq {b}_{i} (1 \leq i \leq k ) \).  
        \item If \( x \in \R^{k}  \) and \( r > 0  \), the \textit{open} (or \textit{closed}) \textit{ball} \( B \) with center at \( x  \) and radius \( r  \) is defined to be the set
            \[ B(x,r) = \{ y \in \R^{k}: | y - x  | < r  \}  \]
            or alternatively, \( | y - x  | \leq  r  \).
        \item We call a set \( E \subset \R^{k} \) \textit{convex} if
            \[  \lambda x + (1 - \lambda) y \in E  \]
            whenever \( x,y \in E  \), and \(  0 < \lambda < 1  \).
    \end{itemize}
\end{definition}

\begin{definition}[ ]
    Let \( X  \) be a metric space. All points and sets mentioned below are understood to be elements and subsets of \( X  \).
    \begin{enumerate}
        \item[(a)] A \textit{neighborhood} of \( p  \) is a set \( {N}_{r}(p) \) consisting of all \( q  \) such that \( d(p,q) < r  \) for some \( r > 0  \). The number \( r  \) is called the \textit{radius} of \( {N}_{r}(p) \). 
        \item[(b)] A point \( p  \) is a \textit{limit point} of the set \( E  \) if \textit{every} neighborhood of \( p  \) contains a point \( q \neq p  \) such that \( q \in E  \).
        \item[(c)] If \( p \in E  \) and \( p  \) is not a limit point of \( E  \), then \( p  \) is called an \textit{isolated point} of \( E  \).
        \item[(d)] \( E  \) is closed if every limit point of \( E  \) is a point of \( E  \).
        \item[(e)] A point \( p  \) is an \textit{interior} point of \( E  \) if there is a neighborhood \( N  \) of \( p  \) such that \( N \subset E  \).
        \item[(f)] \( E  \) is \textit{open} if every point of \( E  \) is an interior point of \( E  \). 
        \item[(g)] The \textit{complement} of \( E  \) (denoted by \( E^{c} \)) is the set of all points \( p \in X  \) such that \( p \neq E  \).
        \item[(h)] \( E  \) is \textit{perfect} if \( E  \) is closed and if every point of \( E  \) is a limit point of \( E  \).
        \item[(i)] \( E  \) is \textit{bounded} if there exists a \( M \in \R  \) and \( q \in X  \) such that \( d(p,q) < M  \) for all \( p \in E  \).
        \item[(j)] \( E  \) is \textit{dense} in \(  X  \) if every point of \( X  \) is a limit point of \( E  \), or a point of \( E  \) (or both).  
    \end{enumerate}
\end{definition}

\begin{remark}
    In \( \R^{1} \), neighborhoods are segments and in \( \R^{2} \) neighborhoods are interiors of circles.
\end{remark}

\begin{theorem}[Neighborhoods are Open]
    Every neighborhood is an open set.
\end{theorem}

\begin{proof}
Let \( p \in X  \). Consider the neighborhood \( {N}_{r}(p)  \) for some \( r > 0  \). Let \( y \in X  \). Similarly, we can construct a neighborhood \( {N}_{h}(y) \) for some \( h > 0  \). Observe that the distance between \( p \) and \( y \)  is   
\[  d(p,y) = r - h. \]
Our goal is to show that \( {N}_{h}(y) \subset {N}_{r}(p)  \) in order for \( {N}_{r}(p) \) to be open. Let \( x \in {N}_{h}(y) \). Using the triangle inequality, we can see that 
\begin{align*}
    d(p,x) &\leq d(p,y) + d(y,x) \\
           &< (r - h) + h \\
           &= r.
\end{align*}
This tells us that \( x \in {N}_{r}(p) \), proving that \( {N}_{r}(p) \) is an open set.
\end{proof}

\begin{theorem}[ ]
    If \( p  \) is a limit point of a set \( E  \), then every neighborhood of \( p  \) contains infinitely many points of \( E  \).
\end{theorem}
\begin{proof}
Suppose for sake of contradiction that there exists a neighborhood \( N \) of \( p  \) which contains only a finite number of points of \( E  \). Let \( {q}_{1}, {q}_{2}, \dots, {q}_{n} \) be the points of \( N \cap E  \) such that \( {q}_{m} \neq p  \) for all \( m  \). Observe that
\[  r = \min_{1 \leq m \leq n} d(p, {q}_{m}) > 0 \]
since each \( d(p, {q}_{m}) > 0 \). Since each \( d(p, {q}_{m}) < {\delta}_{m }  \) and not \( d(p, {q}_{m}) = {\delta}_{m} \), we have that none of the \( {q}_{m} \in {N}_{r}(p) \) where \( {q}_{m} \neq p  \). So, \( p  \) must not be a limit point of \( E  \) which is a contradiction. Thus, every neighborhood of \( p \) must contain infinitely many points of \( E  \).
\end{proof}

\begin{corollary}
   A finite point set has no limit points. 
\end{corollary}

\begin{eg}[Examples of Closed, Open, Perfect, Bounded Sets]
   \begin{enumerate}
       \item[(a)] The set of all \( z \in \C  \) such that \( | z  |  < 1  \). \textbf{Open and Bounded}
        \item[(b)] The set of all \( z \in \C  \) such that \( | z  |  \leq 1  \). \textbf{Closed, Perfect, Bounded}  
        \item[(c)] A nonempty finite set. \textbf{(Closed, Bounded)}
        \item[(d)] The set of all integers. \textbf{(Closed)}
        \item[(e)] The set 
            \[ E =  \Big\{ \frac{ 1 }{ n } : n \in \N  \Big\}.  \]
            Note that no point of \( E  \) is a limit point of \( E  \); that is, there are no limit points contained in \( E  \). \textbf{Bounded}
        \item[(f)] The set of all complex numbers (that is, \( \R^{2} \)).
        \item[(g)] The segment \( (a,b) \). \textbf{(Bounded)}
   \end{enumerate}  
\end{eg}

Note that \( (g) \) is not open in \( \R^{1} \) but open in \( \R^{2} \).

\begin{theorem}[ ]
    Let \( \{ {E}_{\alpha} \}  \) be a (either finite or infinite) collection of sets \( {E}_{\alpha} \). Then
    \[  \Big(  \bigcup_{ \alpha } {E}_{\alpha}  \Big)^{c} = \bigcap_{ \alpha }^{  } ({E}_{\alpha}^{c}). \]
\end{theorem}

\begin{theorem}[ ]
    A set \( E  \) is open if and only if its complement is closed.
\end{theorem}
\begin{proof}

\end{proof}


\end{document}
