\documentclass[11pt,a4paper]{book}
\usepackage[utf8]{inputenc}
\usepackage[T1]{fontenc}
% \usepackage{fourier}
\usepackage{textcomp}
\usepackage{hyperref}
\usepackage[english]{babel}
\usepackage{url}
% \usepackage{hyperref}
% \hypersetup{
%     colorlinks,
%     linkcolor={black},
%     citecolor={black},
%     urlcolor={blue!80!black}
% }
\usepackage{graphicx} \usepackage{float}
\usepackage{booktabs}
\usepackage{enumitem}
% \usepackage{parskip}
% \usepackage{parskip}
\usepackage{emptypage}
\usepackage{subcaption}
\usepackage{multicol}
\usepackage[usenames,dvipsnames]{xcolor}
\usepackage{ocgx}
% \usepackage{cmbright}


\usepackage[margin=1in]{geometry}
\usepackage{amsmath, amsfonts, mathtools, amsthm, amssymb}
\usepackage{thmtools}
\usepackage{mathrsfs}
\usepackage{cancel}
\usepackage{bm}
\newcommand\N{\ensuremath{\mathbb{N}}}
\newcommand\R{\ensuremath{\mathbb{R}}}
\newcommand\Z{\ensuremath{\mathbb{Z}}}
\renewcommand\O{\ensuremath{\emptyset}}
\newcommand\Q{\ensuremath{\mathbb{Q}}}
\newcommand\C{\ensuremath{\mathbb{C}}}
\newcommand\F{\ensuremath{\mathbb{F}}}
\DeclareMathOperator{\sgn}{sgn}
\DeclareMathOperator{\diam}{diam}
\DeclareMathOperator{\LO}{LO}
\DeclareMathOperator{\UP}{UP}
\DeclareMathOperator{\card}{card}
\DeclareMathOperator{\Arg}{Arg}
\DeclareMathOperator{\Dom}{Dom}
\DeclareMathOperator{\Log}{Log}
\DeclareMathOperator{\dist}{dist}
% \DeclareMathOperator{\span}{span}
\usepackage{systeme}
\let\svlim\lim\def\lim{\svlim\limits}
\renewcommand\implies\Longrightarrow
\let\impliedby\Longleftarrow
\let\iff\Longleftrightarrow
\let\epsilon\varepsilon
\usepackage{stmaryrd} % for \lightning
\newcommand\contra{\scalebox{1.1}{$\lightning$}}
% \let\phi\varphi
\renewcommand\qedsymbol{$\blacksquare$}

% correct
\definecolor{correct}{HTML}{009900}
\newcommand\correct[2]{\ensuremath{\:}{\color{red}{#1}}\ensuremath{\to }{\color{correct}{#2}}\ensuremath{\:}}
\newcommand\green[1]{{\color{correct}{#1}}}

% horizontal rule
\newcommand\hr{
    \noindent\rule[0.5ex]{\linewidth}{0.5pt}
}

% hide parts
\newcommand\hide[1]{}

% si unitx
\usepackage{siunitx}
\sisetup{locale = FR}
% \renewcommand\vec[1]{\mathbf{#1}}
\newcommand\mat[1]{\mathbf{#1}}

% tikz
\usepackage{tikz}
\usepackage{tikz-cd}
\usetikzlibrary{intersections, angles, quotes, calc, positioning}
\usetikzlibrary{arrows.meta}
\usepackage{pgfplots}
\pgfplotsset{compat=1.13}

\tikzset{
    force/.style={thick, {Circle[length=2pt]}-stealth, shorten <=-1pt}
}

% theorems
\makeatother
\usepackage{thmtools}
\usepackage[framemethod=TikZ]{mdframed}
\mdfsetup{skipabove=1em,skipbelow=1em}

\theoremstyle{definition}

\declaretheoremstyle[
    headfont=\bfseries\sffamily\color{ForestGreen!70!black}, bodyfont=\normalfont,
    mdframed={
        linewidth=1pt,
        rightline=false, topline=false, bottomline=false,
        linecolor=ForestGreen, backgroundcolor=ForestGreen!5,
    }
]{thmgreenbox}

\declaretheoremstyle[
    headfont=\bfseries\sffamily\color{NavyBlue!70!black}, bodyfont=\normalfont,
    mdframed={
        linewidth=1pt,
        rightline=false, topline=false, bottomline=false,
        linecolor=NavyBlue, backgroundcolor=NavyBlue!5,
    }
]{thmbluebox}

\declaretheoremstyle[
    headfont=\bfseries\sffamily\color{NavyBlue!70!black}, bodyfont=\normalfont,
    mdframed={
        linewidth=1pt,
        rightline=false, topline=false, bottomline=false,
        linecolor=NavyBlue
    }
]{thmblueline}

\declaretheoremstyle[
    headfont=\bfseries\sffamily, bodyfont=\normalfont,
    numbered = no,
    mdframed={
        rightline=true, topline=true, bottomline=true,
    }
]{thmbox}

\declaretheoremstyle[
    headfont=\bfseries\sffamily, bodyfont=\normalfont,
    numbered=no,
    % mdframed={
    %     rightline=true, topline=false, bottomline=true,
    % },
    qed=\qedsymbol
]{thmproofbox}

\declaretheoremstyle[
    headfont=\bfseries\sffamily\color{NavyBlue!70!black}, bodyfont=\normalfont,
    numbered=no,
    mdframed={
        rightline=false, topline=false, bottomline=false,
        linecolor=NavyBlue, backgroundcolor=NavyBlue!1,
    },
]{thmexplanationbox}

\declaretheorem[
    style=thmbox, 
    % numberwithin = section,
    numbered = no,
    name=Definition
    ]{definition}

\declaretheorem[
    style=thmbox, 
    name=Example,
    ]{eg}

\declaretheorem[
    style=thmbox, 
    % numberwithin = section,
    name=Proposition]{prop}

\declaretheorem[
    style = thmbox,
    numbered=yes,
    name =Problem
    ]{problem}

\declaretheorem[style=thmbox, name=Theorem]{theorem}
\declaretheorem[style=thmbox, name=Lemma]{lemma}
\declaretheorem[style=thmbox, name=Corollary]{corollary}

\declaretheorem[style=thmproofbox, name=Proof]{replacementproof}

\declaretheorem[style=thmproofbox, 
                name = Solution
                ]{replacementsolution}

\renewenvironment{proof}[1][\proofname]{\vspace{-1pt}\begin{replacementproof}}{\end{replacementproof}}

\newenvironment{solution}
    {
        \vspace{-1pt}\begin{replacementsolution}
    }
    { 
            \end{replacementsolution}
    }

\declaretheorem[style=thmexplanationbox, name=Proof]{tmpexplanation}
\newenvironment{explanation}[1][]{\vspace{-10pt}\begin{tmpexplanation}}{\end{tmpexplanation}}

\declaretheorem[style=thmbox, numbered=no, name=Remark]{remark}
\declaretheorem[style=thmbox, numbered=no, name=Note]{note}

\newtheorem*{uovt}{UOVT}
\newtheorem*{notation}{Notation}
\newtheorem*{previouslyseen}{As previously seen}
% \newtheorem*{problem}{Problem}
\newtheorem*{observe}{Observe}
\newtheorem*{property}{Property}
\newtheorem*{intuition}{Intuition}

\usepackage{etoolbox}
\AtEndEnvironment{vb}{\null\hfill$\diamond$}%
\AtEndEnvironment{intermezzo}{\null\hfill$\diamond$}%
% \AtEndEnvironment{opmerking}{\null\hfill$\diamond$}%

% http://tex.stackexchange.com/questions/22119/how-can-i-change-the-spacing-before-theorems-with-amsthm
\makeatletter
% \def\thm@space@setup{%
%   \thm@preskip=\parskip \thm@postskip=0pt
% }
\newcommand{\oefening}[1]{%
    \def\@oefening{#1}%
    \subsection*{Oefening #1}
}

\newcommand{\suboefening}[1]{%
    \subsubsection*{Oefening \@oefening.#1}
}

\newcommand{\exercise}[1]{%
    \def\@exercise{#1}%
    \subsection*{Exercise #1}
}

\newcommand{\subexercise}[1]{%
    \subsubsection*{Exercise \@exercise.#1}
}


\usepackage{xifthen}

\def\testdateparts#1{\dateparts#1\relax}
\def\dateparts#1 #2 #3 #4 #5\relax{
    \marginpar{\small\textsf{\mbox{#1 #2 #3 #5}}}
}

\def\@lesson{}%
\newcommand{\lesson}[3]{
    \ifthenelse{\isempty{#3}}{%
        \def\@lesson{Lecture #1}%
    }{%
        \def\@lesson{Lecture #1: #3}%
    }%
    \subsection*{\@lesson}
    \testdateparts{#2}
}

% \renewcommand\date[1]{\marginpar{#1}}


% fancy headers
\usepackage{fancyhdr}
\pagestyle{fancy}

\makeatother

% notes
\usepackage{todonotes}
\usepackage{tcolorbox}

\tcbuselibrary{breakable}
\newenvironment{verbetering}{\begin{tcolorbox}[
    arc=0mm,
    colback=white,
    colframe=green!60!black,
    title=Opmerking,
    fonttitle=\sffamily,
    breakable
]}{\end{tcolorbox}}

\newenvironment{noot}[1]{\begin{tcolorbox}[
    arc=0mm,
    colback=white,
    colframe=white!60!black,
    title=#1,
    fonttitle=\sffamily,
    breakable
]}{\end{tcolorbox}}

% figure support
\usepackage{import}
\usepackage{xifthen}
\pdfminorversion=7
\usepackage{pdfpages}
\usepackage{transparent}
\newcommand{\incfig}[1]{%
    \def\svgwidth{\columnwidth}
    \import{./figures/}{#1.pdf_tex}
}

% %http://tex.stackexchange.com/questions/76273/multiple-pdfs-with-page-group-included-in-a-single-page-warning
\pdfsuppresswarningpagegroup=1


\usepackage{standalone}
\usepackage{import}

\begin{document}

\section{Outer Measure on \( \R \)}

\subsection{Motivation and Definition of Outer Measure}

\begin{itemize}
    \item This section focuses on developing the notion of assigning a "size" to subintervals when it comes to more complicated unions of open intervals.  
    \item The hope is to create an integration theory that will cover a broader class of functions than covered by Riemann integration.
    \item We start off by giving a definition of the length of an open interval. 
\end{itemize}

\begin{definition}[Length of Open Interval; \( \ell(I) \)]
    The \textbf{length} \(\ell(I) \) of an open interval \( I  \) is defined bt  
    \[ \ell(I) = 
    \begin{cases}
        b - a &\text{if } I = (a,b) \text{ for some } a,b \in \R \text{ with } a < b,   \\
        0 &\text{if } I = \emptyset, \\
        \infty &\text{if } I = (- \infty ,a) \text{ or } I = (a, \infty ) \text{ for some } a \in \R, \\
        \infty &\text{if } I = (-\infty, \infty  ).
    \end{cases} \]
\end{definition}

Suppose we have a subset \( A \subset \R  \). Then the size of \( A  \) should be the sum of the lengths of a sequence of open intervals whose union contains \( A  \). The infimum of all possible sums gives the definition of the size of \( A  \), denoted \( | A |  \) which we call the \textbf{outer measure} of \( A  \). 

\begin{definition}[Outer Measure; \( |A| \)]
    The \textbf{outer measure} \( | A |  \) of a set \( A \subset \R  \) is defined by
    \[  | A |  = \inf \Big\{ \sum_{ k=1  }^{ \infty  } \ell({I}_{k}) : {I}_{1}, {I}_{2}, \dots \text{ are open intervals such that } A \subset \bigcup_{ k=1 }^{ \infty  } {I}_{k}  \Big\}.  \]
\end{definition}

\begin{itemize}
    \item Notice that the definition of outer measure involves an infinite sum.
    \item Recall that the infinite sum converges if the sequence of partial sums converges.
    \item If one of the elements in the sequence is \( \infty  \) (that is, \( {t}_{k} = \infty  \)) for some \( k  \), then the infinite sum is \( \infty  \).
    \item Otherwise, the infinite sum \( \sum_{ k=1  }^{ \infty  {t}_{k} } \) is defined to be the limit of the increasing sequence of partial sums where 
        \[  \sum_{ k=1  }^{ \infty } {t}_{k} = \lim_{ n \to \infty  }  \sum_{ k=1  }^{ n  } {t}_{k}. \]
\end{itemize}

\subsection{Properties of Outer Measure}

\begin{prop}[Countable sets have outer measure 0]
    Every countable subset of \( \R  \) has outer measure \( 0  \).
\end{prop}
\begin{proof}
Suppose \( A = \{ {a}_{1}, {a}_{2}, \dots  \}  \) is a countable subset of \( \R  \). Let \( \epsilon > 0  \). For \( k \in \Z^{+} \), let 
\[  {I}_{k} = \Big(  {a}_{k} - \frac{ \epsilon }{ 2^{k} } , {a}_{k} + \frac{ \epsilon }{ 2^{k} }  \Big). \]
Then \( {I}_{1}, {I}_{2}, \dots  \) is a sequence of open intervals whose union contains \( A  \). Since 
\[  \sum_{ k=1  }^{ \infty  } \ell({I}_{k}) = 2 \epsilon, \]
we have \( | A  |  \leq 2 \epsilon. \) Since \( \epsilon \) is an arbitrary positive number, we must have \( | A  |  = 0  \).
\end{proof}

\begin{itemize}
    \item Note that \( \Q  \) is a countable set, and thus contains a measure of \( 0  \).
\end{itemize}

\begin{prop}[Outer Measure Preserves Order]\label{Outer Measure Preserves Order}
   Suppose \( A  \) and \( B  \) are subsets of \( \R  \) with \( A \subset B  \). Then \( | A  |  \leq | B  |  \). 
\end{prop}
\begin{proof}
Suppose  \( {I}_{1}, {I}_{2}, \dots  \) is a sequence of open intervals such that
\[  B \subseteq \bigcup_{ k=1  }^{ \infty  }{I}_{k}. \] 
Since \( A \subseteq  B  \), we can see that \( A  \) is also contained within   
\[  \bigcup_{ k=1  }^{ \infty   }  {I}_{k}. \]
Thus, we have
\[  | A  |  \leq \sum_{ k=1  }^{ \infty  } \ell({I}_{k}). \]
If we take the infimum over all the sequences of open intervals whose union contains \( B  \), we see that \( | A  |  \leq | B  |  \). 
\end{proof}

\begin{definition}[Translation; \( t + A  \)]
   If \( t \in \R  \) and \( A \subseteq  \R  \), then the translation \( t + A  \) is defined by 
   \[  t + A = \{ t + a : a \in A  \}. \]
\end{definition}

\begin{itemize}
    \item If \( t > 0  \), then translating \( A  \) by adding \( t  \) moves \(  A \) by \( | t |   \) units to the right.
    \item If \( t < 0  \), then translating \(A  \) by adding \( t  \) moves \( A  \) to the left by \( | t |   \) units.
    \item In a similar manner to how functions do not change their shape when translated either horizontally or vertically, intervals do not change their shape/length when moved either left or right on the number line. 
    \item By standard convention, we take \( t + (-\infty ) = - \infty  \) and \( t + \infty  = \infty  \).
\end{itemize} 

\begin{prop}[Outer Measure is Translation Invariant]
   Suppose \( t \in \R  \) and \( A \subset \R  \). Then \( | t + A  |  = | A  |  \). 
\end{prop}

\begin{proof}
Let \( t \in \R  \) and \( A \subseteq  \R  \). Suppose \( {I}_{1}, {I}_{2}, \dots  \) is a sequence of open intervals such that 
\[  A \subseteq  \bigcup_{ k=1  }^{ \infty  }  {I}_{k}. \]
Then \( t + {I}_{1}, t + {I}_{2}, \dots  \) is a sequence of open intervals such that
\[  t + A \subseteq  \bigcup_{ k=1  }^{ \infty  }  t + {I}_{k}. \]
Since translation by \( t  \) of intervals \( {I}_{k} \) of \( A  \) does not change, we have that
\[  \ell(t + {I}_{k}) = \ell({I}_{k}). \]
So, we see that
\[  | t + A  |  \leq \sum_{ k=1  }^{ \infty   } \ell(t + {I}_{k }) = \sum_{ k=1  }^{ \infty   } \ell({I}_{k}). \]
But this tells us that the union of the sequence of intervals \( {I}_{k} \) of \( A  \) contains \( t + A  \). So, taking the infimum over all such sequences, we have \(  | t +A  | \leq | A  |   \).

Now, observe that \( A = -t + (t + A) \) for \( t \in \R  \). Then we see that
\[  | A  |  = | -t + (t +A )  | \leq| t + A  |.  \]
Thus, \( | A  |  = | t + A  |  \).
\end{proof}

\begin{prop}[Countable Subadditivity of Outer Measure]
    Suppose \( {A}_{1}, {A}_{2}, \dots, \dots \) is a sequence of subsets of \( \R  \). Then
    \[ \Big| \bigcup_{ k=1  }^{ \infty  } {A}_{k} \Big| \leq \sum_{ k=1  }^{ \infty  } |  {A}_{k} |.  \]
\end{prop}
\begin{proof}
If \( | {a}_{k} | = \infty   \) for some \( k \in \Z^{+} \), then the inequality above holds. Thus, assume \( | {A}_{k} |  < \infty  \) for all \( k \in \Z^{+} \). 

Let \( \epsilon > 0  \). Using a lemma found in introductory real analysis textbooks, we can write that for each \( k \in \Z^{+} \), let \( {I}_{1,k}, {I}_{2,k}, \dots  \) be a sequence of open intervals whose union contains \( {A}_{k } \) such that
\[  \sum_{ j=1  }^{ \infty  } \ell({I}_{j,k}) \leq \frac{ \epsilon }{ 2^{k} }  +  | {A}_{k} |. \]
Taking the infinite sum of both sides of this inequality, we get
\[  \sum_{ k=1  }^{ \infty   } \sum_{ j=1  }^{ \infty  } \ell({I}_{j,k}) \leq \epsilon + \sum_{ k=1  }^{ \infty  } | {A}_{k} |. \]
Note that  
\[  \sum_{ k=1  }^{ \infty  } \frac{ \epsilon }{ 2^{k} } = \epsilon \]
by using the geometric sum formula. Consider the doubly indexed collection of open intervals \( \{ {I}_{j,k }: j,k \in \Z^{+} \}   \) into a sequence of open intervals such that 
\[  \bigcup_{ k=1  }^{ \infty  } \ell({I}_{j,k})  \subseteq \bigcup  \{ {I}_{j,k }: j,k \in \Z^{+} \}.     \]
Suppose in step \( k  \), starting with \( k = 2  \), we adjoin the \( k - 1  \) intervals whose indices add up to \( k  \); that is,
\[  \underbrace{{I}_{1,1}}_{2}, \underbrace{{I}_{1,2}, {I}_{2,1}}_{3}, \underbrace{{I}_{1,3}, {I}_{2,2}}_{4}, \underbrace{{I}_{1,5}, {1}_{2,4}, {I}_{3,3}, {I}_{4,2}, {I}_{5,1}}_{6} \dots.   \]
Thus, we must have 
\[ \Big| \bigcup_{ k=1  }^{ \infty  } {A}_{k} \Big| \leq  \sum_{ k=1  }^{ \infty  } \sum_{ j=1  }^{ \infty  } \ell({I}_{j,k }) \leq  \epsilon + \sum_{ k=1  }^{ \infty  } | {A}_{k} |.   \]
Since \( \epsilon > 0  \) is arbitrary, we must have that 
\[  \Big| \bigcup_{ k=1  }^{ \infty  }  {A}_{k } \Big|  \leq \sum_{ k=1  }^{ \infty  } | {A}_{k } |. \]
\end{proof}

\subsection{Outer Measure of Closed Bounded Interval}
\label{proof of prop 1.1.6}
The next property we will prove is that the length of each closed interval \( [a,b]  \) with \( a < b  \) and \( a,b \in \R  \) is equal to \( [a,b] = b - a  \). If \( \epsilon > 0 \), we see that \( (a- \epsilon, b + \epsilon), \emptyset, \emptyset, \dots  \) is a sequence of open intervals whose union contains \( [a,b] \). Then \( | [a,b] | \leq b - a + 2 \epsilon \). Since this inequaltiy holds for all \( \epsilon > 0 \), we must have that 
\[  | [a,b] |  \leq b - a. \]

\begin{definition}[Open Cover]
    Suppose \( A \subseteq \R  \).
    \begin{itemize}
        \item A collection \( \mathcal{C} \) of open subsets of \( \R  \) is called an \textbf{open cover} of \( A  \) if \( A  \) is contained in the union of all the sets in \( \C  \).
        \item An open cover \( \mathcal{C}  \) of \( A  \) is said to have a \textbf{finite subcover} if \( A  \) is contained in the union of some finite list of sets in \( \mathcal{C}  \)
    \end{itemize}
\end{definition}
\begin{proof}

\end{proof}



\begin{prop}[Heine-Borel Theorem]\label{Heine-Borel Theorem}
   Every open cover of a closed bounded subset of \( \R  \) has a finite subcover. 
\end{prop}
\begin{proof}
    Suppose \( F  \) is a closed bounded subset of \( \R  \) and \( \mathcal{C} \) is an open cover of \( F  \) (Note that \( \mathcal{C}  \) is the collection of open subsets in \( \R  \)). 

    First, suppose \( F = [a,b] \) for some \( a, b \in \R  \) with \( a < b  \). Thus, \( \mathcal{C} \) is an open cover of \( [a,b] \). Let
    \[  D = \{ d \in [a,b] : [a,d] \text{ has a finite subcover from } \mathcal{C}  \}.  \]
    Let's show that this set is nonempty first. Note that \( a \in D  \) since \( a \in G  \) for some \( G \in \mathcal{C} \). Thus, \( D \neq \emptyset  \) and so we can say that a supremum exists for \( D  \) (clearly, \( D \) is bounded above). Let
    \[  s = \sup_{}D. \]
    So, \( s \in [a,b] \). Hence, there exists an open set \( G \in \mathcal{C} \) such that \( s \in G  \). Let \( \delta > 0  \) be such that \( (s- \delta, s + \delta) \subset G  \). Since \( s = \sup_{}D  \), there exists \( d \in (s- \delta, s] \) and \( n \in \Z^{+} \) and \( {G}_{1}, {G}_{2}, \dots, {G}_{n} \in \mathcal{C} \) such that 
    \[  [a,d] \subset \bigcup_{ k=1  }^{ n }  {G}_{k }. \]
    Now, let \( d' \in [s, s+ \delta) \) be arbitrary. Then
    \[  [a,d'] \subset G \cup \bigcup_{ k=1 }^{ n }  {G}_{k}. \]
    Since \( [a,d'] \) contains a finite subcover from \( \mathcal{C} \), we get that \( d' \in D  \) for all \( d' \in [s, s+\delta) \cap [a,b] \). Furthermore, this implies that \(  b = s  \). So, with \( d' = b  \), \( [a,b] \) must contain a finite subcover from \( \mathcal{C} \), completing the proof in the case that \( F = [a,b] \). 

    Now, suppose \( F  \) is an arbitrary closed bounded subset of \( \R  \) and that \( \mathcal{C}  \) is an open cover of \( F  \). Let \( a,b \in \R  \) be such that \( F \subseteq [a,b].  \) Now \( \mathcal{C} \cup \{ \R \setminus F  \}  \) is an open cover of \( \R  \) and hence it is an open cover of \( [a,b] \). By our first case, there exists \( {G}_{1}, \dots, {G}_{n} \in \mathcal{C} \) such that 
    \[  [a,b] \subset \Big( \bigcup_{ k=1  }^{ n }  {G}_{k} \Big) \cup (\R \setminus F).  \]
    Thus, 
    \[  F \subset \bigcup_{ k=1  }^{ n }  {G}_{k}, \]
    completing the proof.
\end{proof}

\begin{prop}[Outer Measure of a Closed Interval]
    Suppose \( a,b \in \R  \), with \( a < b  \). Then \( | [a,b] | = b - a  \). 
\end{prop}
\begin{proof}
The proof of \( | [a,b] | \leq b - a  \) can be found {\hyperref[proof of prop 1.1.6]{here}}.

Now, we want to show that \( | [a,b] | \geq b - a  \). Suppose \( {I}_{1}, {I}_{2}, \dots  \) is a sequence of open intervals such that \( [a,b] \subset \bigcup_{ k=1  }^{ \infty  }  {I}_{k }  \). Using the {\hyperref[Heine-Borel Theorem]{Heine-Borel Theorem}}, there exists \( n \in \Z^{+} \) such that  
\[  [a,b] \subset \bigcup_{ k=1  }^{ n  }  {I}_{k }. \]
We will now show, by induction on \( n  \), that the inclusion above implies that
\[  \sum_{ k=1  }^{ n } \ell({I}_{k }) \geq b - a. \tag{1} \]
Proving this will show that 
\[  \sum_{ k=1  }^{ \infty  } \ell({I}_{k }) \geq \sum_{ k=1  }^{ n } \ell({I}_{k }) \geq b - a, \tag{2}  \]
completing the proof that \( | [a,b] |  \geq b - a  \). If \( n = 1  \), we have that (1) implies (2). Now, suppose \(  n > 1  \). Then (1) implies (2) for all choices of \( a,b \in \R  \) with \( a  < b  \). Suppose \( {I}_{1}, \dots, {I}_{n}, {I}_{n+1} \) are open intervals such that 
\[  [a,b] \subset \bigcup_{ k=1  }^{ n+1 }  {I}_{k }.  \]
Thus, \( b  \) must be in at least one of these intervals. By relabeling, we can assume that \( b \in {I}_{n+1} \). Suppose \( {I}_{n+1} = (c,d) \). If \(  c \leq a  \), then \( \ell({I}_{n+1}) \geq b - a  \) and there is nothing further to prove; thus, assume that \( a < c < b < d  \). Thus, we can assume that   
\[  [a,c] \subset \bigcup_{ k=1  }^{ n }  {I}_{k }. \]
Using our induction hypothesis, we have
\[  \sum_{ k=1  }^{ n } \ell({I}_{k }) \geq c - a. \]
Since \( d > b  \) and \( c \in [a,b] \), we can see that \( \ell( {I}_{n+1}) = d - c  \).  Thus, we have that 
\begin{align*}
    \sum_{ k=1  }^{ n + 1 } \ell({I}_{k })  &= \ell({I}_{n+1}) +  \sum_{ k=1  }^{ n  } \ell({I}_{k})  \\
                                            &\geq  (c - a) + (d - c) \\
                                            &= d - a  \\
                                            &\geq b - a,
\end{align*}
completing the proof.
\end{proof}

\begin{prop}[Nontrivial Intervals Are Uncountable]
   Every interval in \( \R  \) that contains at least two distinct elements is uncountable. 
\end{prop}
\begin{proof}

\end{proof}

\subsection{Outer Measure is Not Additive}

Let us state a result that will be used in the upcoming proof of this result.

The result states that a nonempty intersection of two equivalence classes implies that the two classes are equal to each other. 

\begin{prop}
    If \( a,b \in [-1,1] \) and \(  \tilde{a} \cap \tilde{b} \neq \emptyset  \) where  
    \[  \tilde{a} = \{ c \in [-1,1] : a - c \in \Q  \},  \]
    (\( \tilde{b}  \) is defined similarly ) then \( \tilde{a} = \tilde{b}  \).
\end{prop}
\begin{proof}
Let \( d \in \tilde{a} \cap \tilde{b} \). Then \( d \in \tilde{a} \) and \( d \in \tilde{b} \). By definition of \( \tilde{a}  \) and \( \tilde{b} \), we have \( a - d \in \Q  \) and \(  b - d \in \Q  \), respectively. Since \( \Q  \) is a field, we can use subtraction to get 
\[  a - d - (b-d) = a - b. \]
This implies that \( a - b \in \Q  \) since \( a - d \in \Q  \) and \( b - d \in \Q  \). Thus, we have 
\[  a - c = (a-b) + (b-c)  \] is a rational number if and only \(  b - c \in \Q  \). But, \( \tilde{a} \cap \tilde{b} \neq \emptyset \). So, we must have 
\[  a - c = b - c \implies \tilde{a} = \tilde{b}. \]
\end{proof}


\begin{prop}[Nonadditivity of Outer Measure]\label{Nonadditivity of Outer Measure}
   There exists disjoint subsets of \( A  \) and \( B  \) of \( \R  \) such that 
   \[  | A \cup B  | \neq | A  |  + | B  |. \]
\end{prop}
\begin{proof}
    Continuing from the proof before this result, we can see that \( a \in \tilde{a} \) for each \( a \in [-1,1] \). Thus, we have 
    \[  [-1,1] = \bigcup_{ a \in [-1,1] }^{  }  \tilde{a}. \]
\end{proof}

\subsection{Exercise 2A}

\begin{enumerate}
    \item Prove that if \( A  \) and \( B  \) are subsets of \( \R  \) and \( | B  |  = 0  \), then \( | A \cup B  | = | A  |  \).
        \begin{proof}
        Let \( A  \) and \( B  \) be subsets of \( \R  \) and \( | B  |  = 0  \). Hence, by finite subadditivity, we must have 
        \[  | A \cup B  |  \leq | A  |  + | B  | = | A  |. \]
        Since \( A \subseteq A \cup B  \), we can use result 2.5 to conclude that
        \[  | A |  \leq |  A \cup B |.  \]
        Thus, we can conclude that \( | A  \cup B  |  = | A  | \).
        \end{proof}
    \item Suppose \( A \subset \R  \) and \( t \in \R \). Let \( tA = \{ ta  : a \in A  \} .  \) Prove that \( | tA  |  = | t  |  | A  |  \).
        \begin{proof}
        Let \( A \subset \R  \) and \( t \in \R  \). We proceed by showing the following two inequalities:  
        \begin{center}
            \( | t A  |  \leq | t  |  | A  |   \) and \(  | t A  |  \geq | t  |  | A  |  \). 
        \end{center}
        Let \( {I}_{1}, {I}_{2}, \dots  \) be a sequence of open intervals such that 
        \[  A \subseteq \bigcup_{ k=1  }^{ \infty  }  {I}_{k }. \]
        Thus, we have 
        \[  | A  |  \leq \sum_{ k=1  }^{ \infty  } \ell({I}_{k }) \tag{1} \]
        Now, let \( t {I}_{1}, t {I}_{2}, \dots  \) be a sequence of open interval such that
        \[   tA   \subseteq \bigcup_{ k=1  }^{ \infty   }  t {I}_{k }. \]
        Thus, we have
        \[  | t A  |  \leq \sum_{ k=1  }^{ \infty  } \ell(t {I}_{k }). \tag{2} \]
        Noting that each the length of each subinterval of \( tA  \) is just the length of each subinterval of \( A  \) but scaled by \( | t  |  \) where \( t > 0  \), we must have that \( \ell(t {I}_{k }) = | t  | \ell ({I}_{k })  \). To show the first inequality, we can multiply (1) by \( | t |  \) to get
        \[  | t  |  | A  |  \leq | t  |  \sum_{ k=1  }^{ \infty  } | t  |  \ell({I}_{k }) = \sum_{ k=1  }^{ \infty  } \ell(t {I}_{k }). \]
        Then taking the infimum over all sequences of open intervals containing \( t A  \), we have that \( | t  |  | A  |  \leq | t A  |  \). 

        Now, let \( \epsilon > 0  \). Then we have
        \[  \sum_{ k=1  }^{ \infty   } \ell(t {I}_{k }) \leq | t A  |  + \epsilon.  \]
        By (1), we get that
        \[  \sum_{ k=1  }^{ \infty  } \ell(t {I}_{k }) = | t  |  \sum_{ k=1  }^{ \infty  } \ell({I}_{k })  \geq | t  |  | A  |.  \]
        Since \( \epsilon > 0  \) is arbitrary, we must have \( | t  |  | A  |  \leq | t A  |  \). Thus, we conclude that \( | t A  |  = | t  |  | A  |  \).


        \end{proof}
    \item Prove that if \( A, B \subset \R  \) and \( | A  | < \infty    \), then \( | B \setminus A  |  \geq | B  |  - | A  |  \).
        \begin{proof}
            Let \( A, B \subset \R  \) and \(  | A  |  < \infty  \). Let \( {\mathcal{I}}_{1}, {\mathcal{I}}_{2}, \dots  \) be a sequence of open intervals such that
            \[ B \subseteq \bigcup_{ k=1  }^{ \infty  }  {\mathcal{I}}_{k }.  \]
            Thus, we see that 
            \[  | B  |  \leq \sum_{ k=1  }^{ \infty  } \ell({\mathcal{I}}_{k }). \]
            Similarly, let \( {I}_{1}, {I}_{2}, \dots  \) and \( {\mathcal{J}}_{1}, {\mathcal{J}}_{2}, \dots  \) be sequences of open intervals for \( B \setminus A  \) and \(  A  \), respectively. Then
            we have 
            \[  B \setminus A \subseteq \bigcup_{ k=1  }^{ \infty  } {I}_{k} \ \text{and} \ A \subseteq  \bigcup_{ k=1  }^{ \infty  } {\mathcal{J}}_{k }.   \]
            Observe that

            \[  \sum_{ k=1  }^{ \infty  } \ell({I}_{k }) + \sum_{ k=1  }^{ \infty  } \ell({\mathcal{J}}_{k }) \geq \sum_{ k=1  }^{ \infty  } \ell({\mathcal{I}}_{k }) \geq | B  |. \]
            Now, taking the infimum over all sequences \( {I}_{k }  \) and \( {\mathcal{J}}_{k } \) for \( k \in \N  \), we must have 
            \[  | B  |  \leq | B \setminus A  |  + | A  |  \]
            which can be rewritten to get our result that
            \[  | B \setminus A | \geq | B  |  - | A  |.   \]
        \end{proof}
\end{enumerate}




\end{document}
